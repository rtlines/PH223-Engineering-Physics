
\documentclass{article}
%%%%%%%%%%%%%%%%%%%%%%%%%%%%%%%%%%%%%%%%%%%%%%%%%%%%%%%%%%%%%%%%%%%%%%%%%%%%%%%%%%%%%%%%%%%%%%%%%%%%%%%%%%%%%%%%%%%%%%%%%%%%%%%%%%%%%%%%%%%%%%%%%%%%%%%%%%%%%%%%%%%%%%%%%%%%%%%%%%%%%%%%%%%%%%%%%%%%%%%%%%%%%%%%%%%%%%%%%%%%%%%%%%%%%%%%%%%%%%%%%%%%%%%%%%%%
%TCIDATA{OutputFilter=LATEX.DLL}
%TCIDATA{Version=5.50.0.2960}
%TCIDATA{<META NAME="SaveForMode" CONTENT="1">}
%TCIDATA{BibliographyScheme=Manual}
%TCIDATA{Created=Thursday, October 02, 2025 11:03:35}
%TCIDATA{LastRevised=Thursday, October 02, 2025 14:36:31}
%TCIDATA{<META NAME="GraphicsSave" CONTENT="32">}
%TCIDATA{<META NAME="DocumentShell" CONTENT="Scientific Notebook\Blank Document">}
%TCIDATA{CSTFile=Math with theorems suppressed.cst}
%TCIDATA{PageSetup=72,72,72,72,0}
%TCIDATA{AllPages=
%F=36,\PARA{038<p type="texpara" tag="Body Text" >\hfill \thepage}
%}


\newtheorem{theorem}{Theorem}
\newtheorem{acknowledgement}[theorem]{Acknowledgement}
\newtheorem{algorithm}[theorem]{Algorithm}
\newtheorem{axiom}[theorem]{Axiom}
\newtheorem{case}[theorem]{Case}
\newtheorem{claim}[theorem]{Claim}
\newtheorem{conclusion}[theorem]{Conclusion}
\newtheorem{condition}[theorem]{Condition}
\newtheorem{conjecture}[theorem]{Conjecture}
\newtheorem{corollary}[theorem]{Corollary}
\newtheorem{criterion}[theorem]{Criterion}
\newtheorem{definition}[theorem]{Definition}
\newtheorem{example}[theorem]{Example}
\newtheorem{exercise}[theorem]{Exercise}
\newtheorem{lemma}[theorem]{Lemma}
\newtheorem{notation}[theorem]{Notation}
\newtheorem{problem}[theorem]{Problem}
\newtheorem{proposition}[theorem]{Proposition}
\newtheorem{remark}[theorem]{Remark}
\newtheorem{solution}[theorem]{Solution}
\newtheorem{summary}[theorem]{Summary}
\newenvironment{proof}[1][Proof]{\noindent\textbf{#1.} }{\ \rule{0.5em}{0.5em}}
\input{tcilatex}

\begin{document}


\subsubsection{Strategy for Gauss' law problems}

Let's review what we have done before we go on to our last example. For each
Gauss' law problem, we

\begin{enumerate}
\item draw the charge distribution \FRAME{dtbpF}{0.5038in}{0.5145in}{0pt}{}{%
}{Figure}{\special{language "Scientific Word";type
"GRAPHIC";maintain-aspect-ratio TRUE;display "USEDEF";valid_file "T";width
0.5038in;height 0.5145in;depth 0pt;original-width 1.0475in;original-height
1.0706in;cropleft "0";croptop "1";cropright "1";cropbottom "0";tempfilename
'T3IUKV06.wmf';tempfile-properties "XPR";}}

\item Draw the field lines using symmetry- Don's skip this step, it is
essential \FRAME{dtbpF}{0.8994in}{0.9003in}{0pt}{}{}{Figure}{\special%
{language "Scientific Word";type "GRAPHIC";maintain-aspect-ratio
TRUE;display "USEDEF";valid_file "T";width 0.8994in;height 0.9003in;depth
0pt;original-width 1.6551in;original-height 1.656in;cropleft "0";croptop
"1";cropright "1";cropbottom "0";tempfilename
'T3IUKV07.wmf';tempfile-properties "XPR";}}

\item Choose (make up, invent) a closed surface that makes $\overrightarrow{%
\mathbf{E}}\cdot d\overrightarrow{\mathbf{A}}$ either just $EdA$ or $0.$ And
hopfeully on our Gaussian survace $E$ will be constant\FRAME{dtbpF}{1.0972in%
}{1.0981in}{0pt}{}{}{Figure}{\special{language "Scientific Word";type
"GRAPHIC";maintain-aspect-ratio TRUE;display "USEDEF";valid_file "T";width
1.0972in;height 1.0981in;depth 0pt;original-width 1.938in;original-height
1.9398in;cropleft "0";croptop "1";cropright "1";cropbottom "0";tempfilename
'T3IUKV08.wmf';tempfile-properties "XPR";}}

\item Find $Q_{in}.$

\item Solve $\doint EdA=\frac{Q_{inside}}{\epsilon _{o}}$ for the non, zero
parts. The integral should be trivial now due to our use of symmetry.
Usually, if we picked our surface well, $\doint EdA=EA$

\item Solve for $E$%
\begin{eqnarray*}
EA &=&\frac{Q_{inside}}{\epsilon _{o}} \\
E &=&\frac{Q_{inside}}{\epsilon _{o}A}
\end{eqnarray*}
\end{enumerate}

\subsubsection{Second Example}

{\small Find the field at a point inside at ponint }$P${\small \ inside a
uniformly charged spherical insulator with total charge }$Q_{total}${\small %
\ and radius }$R.${\small \ Let's say that }$P${\small \ is a distance }$r$%
{\small \ from the center of the sphere.}

\begin{enumerate}
\item {\small draw the charge distribution \FRAME{dtbpF}{0.659in}{0.651in}{%
0pt}{}{}{Figure}{\special{language "Scientific Word";type
"GRAPHIC";maintain-aspect-ratio TRUE;display "USEDEF";valid_file "T";width
0.659in;height 0.651in;depth 0pt;original-width 0.8214in;original-height
0.8116in;cropleft "0";croptop "1";cropright "1";cropbottom "0";tempfilename
'T3IUKV09.wmf';tempfile-properties "XPR";}}}

\item {\small Draw the field lines using symmetry- Don's skip this step, it
is essential \FRAME{dtbpF}{1.4538in}{1.4538in}{0pt}{}{}{Figure}{\special%
{language "Scientific Word";type "GRAPHIC";maintain-aspect-ratio
TRUE;display "USEDEF";valid_file "T";width 1.4538in;height 1.4538in;depth
0pt;original-width 1.452in;original-height 1.452in;cropleft "0";croptop
"1";cropright "1";cropbottom "0";tempfilename
'T3IUKV0A.wmf';tempfile-properties "XPR";}}}

\item {\small Choose (make up, invent) a closed surface that that includes
point }$P${\small \ and makes }$\overrightarrow{\mathbf{E}}\cdot d%
\overrightarrow{\mathbf{A}}${\small \ either just }$\pm EdA${\small \ or }$0.
${\small \ And hopfeully on our Gaussian survace }$E${\small \ will be
constant\FRAME{dtbpF}{1.4227in}{1.4227in}{0pt}{}{}{Figure}{\special{language
"Scientific Word";type "GRAPHIC";maintain-aspect-ratio TRUE;display
"USEDEF";valid_file "T";width 1.4227in;height 1.4227in;depth
0pt;original-width 1.4209in;original-height 1.4209in;cropleft "0";croptop
"1";cropright "1";cropbottom "0";tempfilename
'T3IUKV0B.wmf';tempfile-properties "XPR";}}}

\item {\small Find }$Q_{in}.${\small \ This time it is harder. Since the
charge is uniformly distsributed }%
\[
\rho =\frac{Q_{total}}{%
%TCIMACRO{\TeXButton{V}{\ooalign{\hfil$V$\hfil\cr\kern0.1em--\hfil\cr}}}%
%BeginExpansion
\ooalign{\hfil$V$\hfil\cr\kern0.1em--\hfil\cr}%
%EndExpansion
_{total}}=\frac{Q_{in}}{%
%TCIMACRO{\TeXButton{V}{\ooalign{\hfil$V$\hfil\cr\kern0.1em--\hfil\cr}}}%
%BeginExpansion
\ooalign{\hfil$V$\hfil\cr\kern0.1em--\hfil\cr}%
%EndExpansion
_{in}}
\]%
{\small so }%
\[
Q_{in}=\frac{Q_{total}}{%
%TCIMACRO{\TeXButton{V}{\ooalign{\hfil$V$\hfil\cr\kern0.1em--\hfil\cr}}}%
%BeginExpansion
\ooalign{\hfil$V$\hfil\cr\kern0.1em--\hfil\cr}%
%EndExpansion
_{total}}%
%TCIMACRO{\TeXButton{V}{\ooalign{\hfil$V$\hfil\cr\kern0.1em--\hfil\cr}}}%
%BeginExpansion
\ooalign{\hfil$V$\hfil\cr\kern0.1em--\hfil\cr}%
%EndExpansion
_{in}
\]%
{\small We know how to do this for spheres }%
\begin{eqnarray*}
%TCIMACRO{\TeXButton{V}{\ooalign{\hfil$V$\hfil\cr\kern0.1em--\hfil\cr}}}%
%BeginExpansion
\ooalign{\hfil$V$\hfil\cr\kern0.1em--\hfil\cr}%
%EndExpansion
_{total} &=&\frac{4}{3}\pi R^{3} \\
%TCIMACRO{\TeXButton{V}{\ooalign{\hfil$V$\hfil\cr\kern0.1em--\hfil\cr}}}%
%BeginExpansion
\ooalign{\hfil$V$\hfil\cr\kern0.1em--\hfil\cr}%
%EndExpansion
_{in} &=&\frac{4}{3}\pi r^{3}
\end{eqnarray*}%
\[
Q_{in}=\frac{Q_{total}}{\frac{4}{3}\pi R^{3}}\frac{4}{3}\pi r^{3}=\frac{%
Q_{total}}{R^{3}}r^{3}
\]

\item {\small Solve }$\doint EdA=\frac{Q_{inside}}{\epsilon _{o}}${\small \
for the non, zero parts. The integral should be trivial now due to our use
of symmetry. Usually, if we picked our surface well, }$\doint EdA=EA=E4\pi
r^{2}$

\item {\small Solve for }$E$%
\begin{eqnarray*}
EA &=&\frac{Q_{inside}}{\epsilon _{o}} \\
E &=&\frac{\frac{Q_{total}}{R^{3}}r^{3}}{\epsilon _{o}4\pi r^{2}}=\frac{1}{%
4\pi \varepsilon _{o}}\frac{Q_{total}r}{R^{3}}
\end{eqnarray*}
\end{enumerate}

\end{document}

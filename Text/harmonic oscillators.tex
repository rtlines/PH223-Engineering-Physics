
\documentclass{article}
%%%%%%%%%%%%%%%%%%%%%%%%%%%%%%%%%%%%%%%%%%%%%%%%%%%%%%%%%%%%%%%%%%%%%%%%%%%%%%%%%%%%%%%%%%%%%%%%%%%%%%%%%%%%%%%%%%%%%%%%%%%%%%%%%%%%%%%%%%%%%%%%%%%%%%%%%%%%%%%%%%%%%%%%%%%%%%%%%%%%%%%%%%%%%%%%%%%%%%%%%%%%%%%%%%%%%%%%%%%%%%%%%%%%%%%%%%%%%%%%%%%%%%%%%%%%
%TCIDATA{OutputFilter=LATEX.DLL}
%TCIDATA{Version=5.50.0.2960}
%TCIDATA{<META NAME="SaveForMode" CONTENT="1">}
%TCIDATA{BibliographyScheme=Manual}
%TCIDATA{Created=Tuesday, October 28, 2025 16:46:48}
%TCIDATA{LastRevised=Tuesday, October 28, 2025 16:46:57}
%TCIDATA{<META NAME="GraphicsSave" CONTENT="32">}
%TCIDATA{<META NAME="DocumentShell" CONTENT="Scientific Notebook\Blank Document">}
%TCIDATA{CSTFile=Math with theorems suppressed.cst}
%TCIDATA{PageSetup=72,72,72,72,0}
%TCIDATA{AllPages=
%F=36,\PARA{038<p type="texpara" tag="Body Text" >\hfill \thepage}
%}


\newtheorem{theorem}{Theorem}
\newtheorem{acknowledgement}[theorem]{Acknowledgement}
\newtheorem{algorithm}[theorem]{Algorithm}
\newtheorem{axiom}[theorem]{Axiom}
\newtheorem{case}[theorem]{Case}
\newtheorem{claim}[theorem]{Claim}
\newtheorem{conclusion}[theorem]{Conclusion}
\newtheorem{condition}[theorem]{Condition}
\newtheorem{conjecture}[theorem]{Conjecture}
\newtheorem{corollary}[theorem]{Corollary}
\newtheorem{criterion}[theorem]{Criterion}
\newtheorem{definition}[theorem]{Definition}
\newtheorem{example}[theorem]{Example}
\newtheorem{exercise}[theorem]{Exercise}
\newtheorem{lemma}[theorem]{Lemma}
\newtheorem{notation}[theorem]{Notation}
\newtheorem{problem}[theorem]{Problem}
\newtheorem{proposition}[theorem]{Proposition}
\newtheorem{remark}[theorem]{Remark}
\newtheorem{solution}[theorem]{Solution}
\newtheorem{summary}[theorem]{Summary}
\newenvironment{proof}[1][Proof]{\noindent\textbf{#1.} }{\ \rule{0.5em}{0.5em}}
\input{tcilatex}

\begin{document}


It may pay off to recall some details of oscillators.

\subsubsection{Energy of the Simple Harmonic Oscillator}

\FRAME{dtbpF}{1.753in}{0.7878in}{0pt}{}{}{Figure}{\special{language
"Scientific Word";type "GRAPHIC";maintain-aspect-ratio TRUE;display
"USEDEF";valid_file "T";width 1.753in;height 0.7878in;depth
0pt;original-width 2.6757in;original-height 1.1882in;cropleft "0";croptop
"1";cropright "1";cropbottom "0";tempfilename
'T4V5Y91K.wmf';tempfile-properties "XPR";}}

Suppose we have a mass-spring system like something from Principles of
Physics I (PH121). If I pull back on the mass and let go, the mass-spring
system will oscillate. The mass has kinetic energy because the mass is
moving 
\begin{equation}
K=\frac{1}{2}mv^{2}
\end{equation}%
but it goes back and forth. We can for our Simple Harmonic Oscillator we
know that the position of the mass as a function of time is given by 
\[
x\left( t\right) =x_{\max }\cos \left( \omega t+\phi \right) 
\]%
and the speed as a function of time is 
\[
v\left( t\right) =-\omega x_{\max }\sin \left( \omega t+\phi \right) 
\]%
then the kinetic energy as a function of time is%
\begin{eqnarray*}
K &=&\frac{1}{2}m\left( -\omega x_{\max }\sin \left( \omega t+\phi \right)
\right) ^{2} \\
&=&\frac{1}{2}m\omega ^{2}x_{\max }^{2}\sin ^{2}\left( \omega t+\phi \right) 
\\
&=&\frac{1}{2}m\frac{k}{m}x_{\max }^{2}\sin ^{2}\left( \omega t+\phi \right) 
\\
&=&\frac{1}{2}kx_{\max }^{2}\sin ^{2}\left( \omega t+\phi \right) 
\end{eqnarray*}%
The spring has potential energy given by 
\begin{equation}
U=\frac{1}{2}kx^{2}
\end{equation}%
For our mechanical oscillator the potential as a function of time is%
\[
U=\frac{1}{2}kx_{\max }^{2}\cos ^{2}\left( \omega t+\phi \right) 
\]%
The total energy is given by%
\begin{eqnarray*}
E &=&K+U \\
&=&\frac{1}{2}kx_{\max }^{2}\sin ^{2}\left( \omega t+\phi \right) +\frac{1}{2%
}kx_{\max }^{2}\cos ^{2}\left( \omega t+\phi \right)  \\
&=&\frac{1}{2}kx_{\max }^{2}\left( \sin ^{2}\left( \omega t+\phi \right)
+\cos ^{2}\left( \omega t+\phi \right) \right)  \\
&=&\frac{1}{2}kx_{\max }^{2}
\end{eqnarray*}

We can see that the total energy won't change, and the energy switches back
and forth from kinetic to potential as the mass moves back and forth. If we
plot the kinetic and potential energy at points along the mass' path we get
something like this.

\FRAME{dtbpF}{4.7539in}{3.5699in}{0pt}{}{}{Figure}{\special{language
"Scientific Word";type "GRAPHIC";maintain-aspect-ratio TRUE;display
"USEDEF";valid_file "T";width 4.7539in;height 3.5699in;depth
0pt;original-width 10.0258in;original-height 7.5247in;cropleft "0";croptop
"1";cropright "1";cropbottom "0";tempfilename
'T4V5Y91L.wmf';tempfile-properties "XPR";}}

\end{document}

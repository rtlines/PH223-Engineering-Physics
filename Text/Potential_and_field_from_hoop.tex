
\documentclass{article}
%%%%%%%%%%%%%%%%%%%%%%%%%%%%%%%%%%%%%%%%%%%%%%%%%%%%%%%%%%%%%%%%%%%%%%%%%%%%%%%%%%%%%%%%%%%%%%%%%%%%%%%%%%%%%%%%%%%%%%%%%%%%%%%%%%%%%%%%%%%%%%%%%%%%%%%%%%%%%%%%%%%%%%%%%%%%%%%%%%%%%%%%%%%%%%%%%%%%%%%%%%%%%%%%%%%%%%%%%%%%%%%%%%%%%%%%%%%%%%%%%%%%%%%%%%%%
%TCIDATA{OutputFilter=LATEX.DLL}
%TCIDATA{Version=5.50.0.2960}
%TCIDATA{<META NAME="SaveForMode" CONTENT="1">}
%TCIDATA{BibliographyScheme=Manual}
%TCIDATA{Created=Monday, June 17, 2024 10:41:09}
%TCIDATA{LastRevised=Monday, June 17, 2024 11:11:39}
%TCIDATA{<META NAME="GraphicsSave" CONTENT="32">}
%TCIDATA{<META NAME="DocumentShell" CONTENT="Scientific Notebook\Blank Document">}
%TCIDATA{CSTFile=Math with theorems suppressed.cst}
%TCIDATA{PageSetup=72,72,72,72,0}
%TCIDATA{AllPages=
%F=36,\PARA{038<p type="texpara" tag="Body Text" >\hfill \thepage}
%}


\newtheorem{theorem}{Theorem}
\newtheorem{acknowledgement}[theorem]{Acknowledgement}
\newtheorem{algorithm}[theorem]{Algorithm}
\newtheorem{axiom}[theorem]{Axiom}
\newtheorem{case}[theorem]{Case}
\newtheorem{claim}[theorem]{Claim}
\newtheorem{conclusion}[theorem]{Conclusion}
\newtheorem{condition}[theorem]{Condition}
\newtheorem{conjecture}[theorem]{Conjecture}
\newtheorem{corollary}[theorem]{Corollary}
\newtheorem{criterion}[theorem]{Criterion}
\newtheorem{definition}[theorem]{Definition}
\newtheorem{example}[theorem]{Example}
\newtheorem{exercise}[theorem]{Exercise}
\newtheorem{lemma}[theorem]{Lemma}
\newtheorem{notation}[theorem]{Notation}
\newtheorem{problem}[theorem]{Problem}
\newtheorem{proposition}[theorem]{Proposition}
\newtheorem{remark}[theorem]{Remark}
\newtheorem{solution}[theorem]{Solution}
\newtheorem{summary}[theorem]{Summary}
\newenvironment{proof}[1][Proof]{\noindent\textbf{#1.} }{\ \rule{0.5em}{0.5em}}
\input{tcilatex}

\begin{document}


\subsection{Ring of charge}

\FRAME{dtbpF}{5.1084in}{2.8089in}{0in}{}{}{Figure}{\special{language
"Scientific Word";type "GRAPHIC";maintain-aspect-ratio TRUE;display
"USEDEF";valid_file "T";width 5.1084in;height 2.8089in;depth
0in;original-width 5.0531in;original-height 2.7665in;cropleft "0";croptop
"1";cropright "1";cropbottom "0";tempfilename
'SF8IFF00.wmf';tempfile-properties "XPR";}}

We found the field from a uniformly charged hoop

\[
\overrightarrow{\mathbf{E}}=\frac{1}{4\pi \epsilon _{o}}\frac{zQ}{\left(
R^{2}+z^{2}\right) ^{\frac{3}{2}}}\mathbf{\hat{k}}
\]

\subsection{Potential}

Let's find the potential from the hoop%
\[
v=\frac{1}{4\pi \epsilon _{o}}\int \frac{dq}{r}
\]%
Find $dq$%
\[
dq=\lambda dy
\]%
but now we know that for the hoop 
\[
dq=\lambda ds
\]%
where $s$ is the arc length. Recall that 
\begin{eqnarray*}
s &=&R\phi  \\
ds &=&Rd\phi 
\end{eqnarray*}%
where $R$ is the radius of the ring and $\phi $ is an angle measured from
the $x$-axis. So our $dq$ expression becomes 
\[
dq=\lambda Rd\phi 
\]%
For the whole ring 
\begin{eqnarray*}
Q &=&\lambda R2\pi  \\
&=&2\pi R\lambda 
\end{eqnarray*}%
We also need to use geometry to find $r,$ the distance from our $dq$
segments to our point were we want to know the field.%
\[
r=\sqrt{y_{i}^{2}+z^{2}}
\]%
but since this is a ring, our $y_{i}=R$ for all $i.$ So%
\[
r=\sqrt{R^{2}+z^{2}}
\]%
Then we can set up our integral.%
\[
V=\frac{1}{4\pi \epsilon _{o}}\int \frac{dq}{r}
\]%
Putting in all the parts we have found yields%
\begin{eqnarray*}
V &=&\frac{1}{4\pi \epsilon _{o}}\int \frac{dq}{\sqrt{R^{2}+z^{2}}} \\
&&\frac{1}{4\pi \epsilon _{o}}\int \frac{\lambda Rd\phi }{\sqrt{R^{2}+z^{2}}}
\\
V &=&\frac{\lambda R}{4\pi \epsilon _{o}\sqrt{R^{2}+z^{2}}}\int_{0}^{2\pi
}d\phi 
\end{eqnarray*}%
\[
V=\frac{2\pi \lambda R}{4\pi \epsilon _{o}\sqrt{R^{2}+z^{2}}}
\]%
\[
V=\frac{2\pi \frac{Q}{2\pi R}R}{4\pi \epsilon _{o}\sqrt{R^{2}+z^{2}}}
\]%
\[
V=\frac{Q}{4\pi \epsilon _{o}\sqrt{R^{2}+z^{2}}}
\]

Now find the field by going backwards%
\begin{eqnarray*}
E_{z} &=&-\frac{dV}{dz}=-\frac{d}{dz}\left( \frac{Q}{4\pi \epsilon _{o}\sqrt{%
R^{2}+z^{2}}}\right)  \\
&=&-\frac{d}{dz}\frac{Q}{4\pi \epsilon _{o}}\left( \frac{1}{\sqrt{R^{2}+z^{2}%
}}\right)  \\
&=&-\frac{d}{dz}\frac{Q}{4\pi \epsilon _{o}}\left( \frac{1}{\left(
R^{2}+z^{2}\right) ^{\frac{1}{2}}}\right) 
\end{eqnarray*}

\[
E_{z}=-\frac{d}{dz}\frac{Q}{4\pi \epsilon _{o}}\left( \left(
R^{2}+z^{2}\right) ^{-\frac{1}{2}}\right) 
\]%
\[
E_{z}=-\frac{Q}{4\pi \epsilon _{o}}\left( -\frac{1}{2}\left(
R^{2}+z^{2}\right) ^{-\frac{3}{2}}2z\right) 
\]%
\[
E_{z}=\frac{Q}{4\pi \epsilon _{o}}\left( \left( R^{2}+z^{2}\right) ^{-\frac{3%
}{2}}z\right) 
\]%
\[
E_{z}=\frac{1}{4\pi \epsilon _{o}}\left( \frac{zQ}{\left( R^{2}+z^{2}\right)
^{\frac{3}{2}}}\right) 
\]%
which we can copare to our previous answer%
\[
\overrightarrow{\mathbf{E}}=\frac{1}{4\pi \epsilon _{o}}\frac{zQ}{\left(
R^{2}+z^{2}\right) ^{\frac{3}{2}}}\mathbf{\hat{k}}
\]

\bigskip 

But not done, have to show $E_{x.}$ $E_{y}=0$%
\[
E_{x}=-\frac{dV}{dx}=-\frac{d}{dx}\left( \frac{Q}{4\pi \epsilon _{o}\sqrt{%
R^{2}+z^{2}}}\right) =0
\]%
\[
E_{y}=-\frac{dV}{dy}=-\frac{d}{dy}\left( \frac{Q}{4\pi \epsilon _{o}\sqrt{%
R^{2}+z^{2}}}\right) =0
\]

\end{document}

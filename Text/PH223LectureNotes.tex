
\documentclass{sebase}
%%%%%%%%%%%%%%%%%%%%%%%%%%%%%%%%%%%%%%%%%%%%%%%%%%%%%%%%%%%%%%%%%%%%%%%%%%%%%%%%%%%%%%%%%%%%%%%%%%%%%%%%%%%%%%%%%%%%%%%%%%%%%%%%%%%%%%%%%%%%%%%%%%%%%%%%%%%%%%%%%%%%%%%%%%%%%%%%%%%%%%%%%%%%%%%%%%%%%%%%%%%%%%%%%%%%%%%%%%%%%%%%%%%%%%%%%%%%%%%%%%%%%%%%%%%%
\usepackage{amsmath}
\usepackage{amssymb}
\usepackage{LECTURENOTES}

\setcounter{MaxMatrixCols}{10}
%TCIDATA{OutputFilter=LATEX.DLL}
%TCIDATA{Version=5.50.0.2960}
%TCIDATA{<META NAME="SaveForMode" CONTENT="1">}
%TCIDATA{BibliographyScheme=Manual}
%TCIDATA{Created=Thursday, June 02, 2011 09:40:55}
%TCIDATA{LastRevised=Thursday, January 11, 2024 16:48:55}
%TCIDATA{<META NAME="GraphicsSave" CONTENT="32">}
%TCIDATA{<META NAME="DocumentShell" CONTENT="Style Editor\TLLectureNotes">}
%TCIDATA{Language=American English}
%TCIDATA{CSTFile=AG.cst}

\input{tcilatex}
\begin{document}


\qquad 
%TCIMACRO{\TeXButton{reversemarginpar}{\reversemarginpar}}%
%BeginExpansion
\reversemarginpar%
%EndExpansion
\bigskip 
%TCIMACRO{%
%\TeXButton{*}{\pagenumbering{roman}
%\bgroup
%\parindent=0pt
%\thispagestyle{empty}
%\null}}%
%BeginExpansion
\pagenumbering{roman}
\bgroup
\parindent=0pt
\thispagestyle{empty}
\null%
%EndExpansion
\vspace{1in}\vspace{0.5in}

{\Huge PH223: Physics for Chemists and Mechanical Engineers }

{\Huge \vspace{0.1in}}

{\Huge \vspace{0.1in}}

\vspace{0.5in}

\vspace{1in}

{\LARGE R. Todd Lines}

{\LARGE Brigham Young University-Idaho}

%TCIMACRO{%
%\TeXButton{*}{\vfill
%}}%
%BeginExpansion
\vfill
%
%EndExpansion
\newpage

%TCIMACRO{\TeXButton{Bottom of the Page}{\mbox{}\vfill}}%
%BeginExpansion
\mbox{}\vfill%
%EndExpansion

Created in Scientific Workplace 
%TCIMACRO{\TeXButton{TM}{$^{\text{\tiny TM}}$}}%
%BeginExpansion
$^{\text{\tiny TM}}$%
%EndExpansion

Copyright 
%TCIMACRO{\TeXButton{Copyright}{\copyright} }%
%BeginExpansion
\copyright
%EndExpansion
2011 by Author

\newpage

%TCIMACRO{%
%\TeXButton{*}{\thispagestyle{empty}
%\enlargethispage{1in}}}%
%BeginExpansion
\thispagestyle{empty}
\enlargethispage{1in}%
%EndExpansion

\QTP{preface}
Preface

\section{Preface}

This document contains my lecture notes for a new, experimental course. The
goal of the course is to teach the introductory physics of waves, optics,
and electricity and magnetism for mechanical engineering students.

\section{Forward}

\section{Acknowledgments}

\emph{BYU-I} \hfill \emph{R. Todd Lines}.

\newpage

\QTP{contentsection}
Contents

%TCIMACRO{\TeXButton{1.0 Line Spacing}{\setlength{\parskip}{0pt}}}%
%BeginExpansion
\setlength{\parskip}{0pt}%
%EndExpansion

\setlength{\parskip}{0pt}%EndExpansion

%TCIMACRO{\TeXButton{TOC}{\TableOfContents}}%
%BeginExpansion
\TableOfContents%
%EndExpansion

%TCIMACRO{\TeXButton{1.5 Line Spacing}{\setlength{\parskip}{9pt}}}%
%BeginExpansion
\setlength{\parskip}{9pt}%
%EndExpansion

\pagebreak

\pagenumbering{arabic}

\setcounter{page}{1}

\setcounter{chapter}{0}

\chapter{Where We Start}

%TCIMACRO{%
%\TeXButton{Fundamental Concepts}{\hspace{-1.3in}{\LARGE Fundamental Concepts\vspace{0.25in}}}}%
%BeginExpansion
\hspace{-1.3in}{\LARGE Fundamental Concepts\vspace{0.25in}}%
%EndExpansion

\section{What is this class?}

This class is designed to teach the physics of wave motion, electricity and
magnetism, and optics. We have three major goals. One is to teach the
physics that is not covered by Statics, Dynamics, and the Engineering
Electronics Course. This physics can affect the mechanical systems you will
design, build, or test, so knowing this physics is a very good thing. The
second objective is to teach a different method of thinking about how things
work. The third goal is to describe electrical and wave motion enough that
the quantum nature of atoms and molecules make sense as our chemists take
physical chemistry.

In engineering, the design parameters are often the goal. In physics, the
physical relationship is the goal. For design engineers, both views are
useful and important. The design is no good if the underlying principles
preclude it from working!

As an example, I once worked on an optics project with a strong mechanical
component. The system had scanning mechanisms that were fantastic mechanical
devices. It was part of an aircraft and integrated into the aircraft system.
But the optical system required two lasers that were separated in wavelength
by only a few nanometers. The chief engineer knew how to build all the
systems, but did not understand the physics that required the close
wavelength spacing. He judged that the difficulty in building the device at
that wavelength spacing outweighed any benefit, and he changed the specs to
give two wavelengths that were fifty nanometers apart. Fifty nanometers is a
pretty small tolerance. Surely it would be good enough! The resulting
product did not work. For two years he tried to fine tune the scanners, and
servos to make it work. After ten million dollars and two years, he finally
moved the wavelengths closer. The cost of the change was an extra \$100,000
dollars, about 1/100 of the cost of the mistake. The system worked, but
since this was a race to market, the time lost and the reputation lost on
the faulty product destroyed the viability of the business. It is a bad day
when you and your friends lose your jobs because you made a fundamental
physics mistake!

Physics courses stress how we know what we know. They support the discipline
called \emph{system engineering}, which deals with the design of new and
innovative products. As a more positive example, the National Weather
service often releases requests for proposed weather sensing equipment.
Their request might say something like the following:

\begin{quote}
Measure the moisture of the soil globally from an altitude of $800\unit{km}$
with an accuracy of 5\%. The suggested instrument is a passive microwave
radiometer.
\end{quote}

The job of a system engineer is to determine what type of instrument to
build. What is the underlying principle that it will use to do its job? What
signal processing will it need? What mechanical and electrical systems will
support this? This must all be determined before the bearings and
slip-rings, and structures can be designed and built.

The radiometer design that came out of this project is flying today (or one
very like it based on the original design) and is a major part of the
predictive models that tell us what the weather will be in a few days.

Because this type of reasoning is our goal, we will not only do typical
homework problems, but we will also work on our conceptual understanding.

I will also emphasize a problem solving method that I used with my
engineering team in industry. It is a structured approach to finding a
solution that emphasizes understanding as well as providing a numeric answer
for a particular design. When you are part of an innovative design team, you
will have to repeat a calculation over and over again each time some other
part of the sign changes. If you have produced a symbolic solution, a
numerical model, or at least a curve, you are ready for any changes in
specifications. But if you have just \textquotedblleft found the
answer\textquotedblright\ you will have to find that answer again every time
the overall design specs change. This approach is too slow, and, at least in
my team, would have you finding a new job because our design efforts were
always done against exacting schedules and budgets. By thinking in a
structured method, with an eye toward symbolic answers or relationships
rather than end numbers, you will learn to be a more valuable engineer. The
process we will use is the same approach I\ used to teach my new engineers
in the defense industry. It has been proven useful over and over for decades.

This same problem solving process is useful in chemistry, particularly as
you study physical chemistry.

So let's get started. To understand waves, we need to get the waves moving.
You studied Oscillation in Dynamics or PH121. Oscillating systems are often
the disturbance that starts a wave. We will begin with a review of
oscillation.

\section{Simple Harmonic Motion}

You are, no doubt, an expert in simple harmonic motion (SHM) after your
PH121 or Dynamics class. But this will get us warmed up for the semester. In
class we will use our clickers and go through a few questions. We will
usually use the clicker system to answer a few questions to test your
understanding of the reading material. This allows me to not waste time on
things you already know, and to help me find the ones you don't. Most
lectures will consist of me asking you if you have questions, and then if
you don't, I\ will ask you \textquotedblleft clicker
questions.\textquotedblright\ Where there is reason to believe you don't
understand (with a normal cutoff of 80\% of the class answering correctly
being our definition of \textquotedblleft understanding\textquotedblright ),
I\ will use the material from these written lectures to teach the concepts.
So we won't always go through all the ideas and skills demonstrated in these
written lectures. If you feel you would have liked more explanation on
something but we did not cover that concept in class because most people
were \textquotedblleft getting it,\textquotedblright\ you can come and see
me in my office.

\section{SHM}

Let's consider a mass attached to a spring resting on a frictionless surface%
\footnote{%
Yes, I know there are no actual frictionless surfaces, but we are starting
out at freshman level physics, so we will make the math simple enough that a
freshman could do it by making simplifying assumptions. In this case, that
the surface if frictionless.}. This mass-spring system can oscillate.

%TCIMACRO{%
%\TeXButton{Question 223.1.1}{\marginpar {
%\hspace{-0.5in}
%\begin{minipage}[t]{1in}
%\small{Question 223.1.1}
%\end{minipage}
%}}}%
%BeginExpansion
\marginpar {
\hspace{-0.5in}
\begin{minipage}[t]{1in}
\small{Question 223.1.1}
\end{minipage}
}%
%EndExpansion
%TCIMACRO{%
%\TeXButton{Question 223.1.2}{\marginpar {
%\hspace{-0.5in}
%\begin{minipage}[t]{1in}
%\small{Question 223.1.2}
%\end{minipage}
%}}}%
%BeginExpansion
\marginpar {
\hspace{-0.5in}
\begin{minipage}[t]{1in}
\small{Question 223.1.2}
\end{minipage}
}%
%EndExpansion
%TCIMACRO{%
%\TeXButton{Question 223.1.3}{\marginpar {
%\hspace{-0.5in}
%\begin{minipage}[t]{1in}
%\small{QQuestion 223.1.3}
%\end{minipage}
%}}}%
%BeginExpansion
\marginpar {
\hspace{-0.5in}
\begin{minipage}[t]{1in}
\small{QQuestion 223.1.3}
\end{minipage}
}%
%EndExpansion
%TCIMACRO{%
%\TeXButton{Question 223.1.4}{\marginpar {
%\hspace{-0.5in}
%\begin{minipage}[t]{1in}
%\small{Question 223.1.4}
%\end{minipage}
%}}}%
%BeginExpansion
\marginpar {
\hspace{-0.5in}
\begin{minipage}[t]{1in}
\small{Question 223.1.4}
\end{minipage}
}%
%EndExpansion
%TCIMACRO{%
%\TeXButton{Question 223.1.5}{\marginpar {
%\hspace{-0.5in}
%\begin{minipage}[t]{1in}
%\small{Question 223.1.5}
%\end{minipage}
%}}}%
%BeginExpansion
\marginpar {
\hspace{-0.5in}
\begin{minipage}[t]{1in}
\small{Question 223.1.5}
\end{minipage}
}%
%EndExpansion
%TCIMACRO{%
%\TeXButton{Question 223.1.6}{\marginpar {
%\hspace{-0.5in}
%\begin{minipage}[t]{1in}
%\small{Question 223.1.6}
%\end{minipage}
%}}}%
%BeginExpansion
\marginpar {
\hspace{-0.5in}
\begin{minipage}[t]{1in}
\small{Question 223.1.6}
\end{minipage}
}%
%EndExpansion
%TCIMACRO{%
%\TeXButton{Question 15.3.8}{\marginpar {
%\hspace{-0.5in}
%\begin{minipage}[t]{1in}
%\small{Question 15.3.8}
%\end{minipage}
%}}}%
%BeginExpansion
\marginpar {
\hspace{-0.5in}
\begin{minipage}[t]{1in}
\small{Question 15.3.8}
\end{minipage}
}%
%EndExpansion
%TCIMACRO{%
%\TeXButton{Question 15.3.9.2}{\marginpar {
%\hspace{-0.5in}
%\begin{minipage}[t]{1in}
%\small{Question 15.3.9.2}
%\end{minipage}
%}}}%
%BeginExpansion
\marginpar {
\hspace{-0.5in}
\begin{minipage}[t]{1in}
\small{Question 15.3.9.2}
\end{minipage}
}%
%EndExpansion
%TCIMACRO{%
%\TeXButton{Question 15.3.9.3}{\marginpar {
%\hspace{-0.5in}
%\begin{minipage}[t]{1in}
%\small{Question 15.3.9.3}
%\end{minipage}
%}}}%
%BeginExpansion
\marginpar {
\hspace{-0.5in}
\begin{minipage}[t]{1in}
\small{Question 15.3.9.3}
\end{minipage}
}%
%EndExpansion
%TCIMACRO{%
%\TeXButton{Question 15.3.9.5}{\marginpar {
%\hspace{-0.5in}
%\begin{minipage}[t]{1in}
%\small{Question 15.3.9.5}
%\end{minipage}
%}}}%
%BeginExpansion
\marginpar {
\hspace{-0.5in}
\begin{minipage}[t]{1in}
\small{Question 15.3.9.5}
\end{minipage}
}%
%EndExpansion
%TCIMACRO{%
%\TeXButton{Question 15.4}{\marginpar {
%\hspace{-0.5in}
%\begin{minipage}[t]{1in}
%\small{Question 15.4}
%\end{minipage}
%}}}%
%BeginExpansion
\marginpar {
\hspace{-0.5in}
\begin{minipage}[t]{1in}
\small{Question 15.4}
\end{minipage}
}%
%EndExpansion

In the position shown the spring is neither pushing nor pulling on the mass.
We will call this position the \emph{equilibrium position} for the mass.%
\FRAME{dtbpF}{3.007in}{1.561in}{0pt}{}{}{Figure}{\special{language
"Scientific Word";type "GRAPHIC";maintain-aspect-ratio TRUE;display
"USEDEF";valid_file "T";width 3.007in;height 1.561in;depth
0pt;original-width 2.9637in;original-height 1.5247in;cropleft "0";croptop
"1";cropright "1";cropbottom "0";tempfilename
'Oscillatory_Motion/Mass_Sping_System.wmf';tempfile-properties "XNPR";}}

\begin{definition}
Equilibrium Position: The position of the mass when the spring is neither
stretched nor compressed.
\end{definition}

\subsection{Hooke's Law}

A law in physics is a mathematical expression of a mental model of how the
universe works. Long ago it was noticed that the pull of a spring grew in
strength as the spring was pulled out of equilibrium. The mathematical
expression of this is 
\begin{equation}
F_{s}=-kx
\end{equation}
The force, $F_{s}$ is directly proportional to the displacement from
equilibrium, $x.$ Since a man named Hooke wrote this down, it is called
Hooke's law.

Hooke's Law is, strictly speaking, not a law that is always obeyed. It is a
good model for most springs as long as we don't stretch them too far. We
will often use the word \textquotedblleft law\textquotedblright\ to mean 
\emph{an equation that gives a basic relationship.} In that sense, Hook's
law is a law. \FRAME{dtbpF}{1.3318in}{2.8997in}{0pt}{}{}{Figure}{\special%
{language "Scientific Word";type "GRAPHIC";maintain-aspect-ratio
TRUE;display "USEDEF";valid_file "T";width 1.3318in;height 2.8997in;depth
0pt;original-width 2.156in;original-height 4.7288in;cropleft "0";croptop
"1";cropright "1";cropbottom "0";tempfilename
'Oscillatory_Motion/Spring_Mass_Restoring_Force.wmf';tempfile-properties
"XNPR";}}

Lets write Hooke's Law using Newton's second Law

\begin{equation*}
\Sigma F_{x}=ma_{x}
\end{equation*}%
If we assume no friction, we have just 
\begin{equation*}
-kx=ma_{x}
\end{equation*}

We can write this as%
\begin{equation}
a_{x}=-\frac{k}{m}x
\end{equation}

This expression says the acceleration is directly proportional to the
position, and opposite the direction of the displacement from equilibrium.
We can see that the spring force tries to oppose the change in displacement.
We call such a force a \emph{restoring force}.

\begin{definition}
Restoring force:\ A force that is always directed toward the equilibrium
position
\end{definition}

This is a good definition of \emph{simple harmonic motion.}

\section{Mathematical Representation of Simple Harmonic Motion}

Recall from your Dynamics or PH121 classes that acceleration is the second
derivative of position 
\begin{equation*}
a=\frac{dv}{dt}=\frac{d^{2}x}{dt^{2}}
\end{equation*}

Hook's Law tells us%
\begin{eqnarray*}
F &=&ma=-kx \\
m\frac{d^{2}x}{dt^{2}} &=&-kx
\end{eqnarray*}%
We have a new kind of equation. If you are taking this freshman physics
class as a... well... freshman, you may not have seen this kind of equation
before. It is called a differential equation. But really the chances are
that you are a sophomore or junior (or even a senior) and have lot of
experience with differential equations. The solution of this equation is a
function or functions that will describe the motion of our mass-spring
system as a function of time. We will need to know this function, so let's
see how we can find it.

Start by defining a quantity $\omega $ as%
\begin{equation}
\omega ^{2}=\frac{k}{m}
\end{equation}%
why define $\omega ^{2}$? Because experience has shown that it is useful to
define $\omega $ this way! But you probably remember $\omega $ as having to
do with rotational speed, and from trigonometry (trig) you may remember
using $\omega $ to mean angular frequency%
\begin{equation*}
\omega =2\pi f
\end{equation*}%
so our definition of $\omega $ may hint that $k/m$ will have something to do
with the frequency of oscillation of the mass-spring system.

We can write our differential equation as%
\begin{equation}
\frac{d^{2}x}{dt^{2}}=-\omega ^{2}x
\end{equation}%
To solve this differential equation we need a function who's second
derivative is the negative of itself. We know a few of these%
\begin{eqnarray}
x\left( t\right) &=&A\cos \left( \omega t+\phi _{o}\right) \\
x\left( t\right) &=&A\sin \left( \omega t+\phi _{o}\right)  \notag
\end{eqnarray}%
where $A,$ $\omega ,$ and $\phi _{o}$ are constants that we must find. Let's
choose the cosine function and explicitly take its derivatives.%
\begin{eqnarray*}
x\left( t\right) &=&A\cos \left( \omega t+\phi _{o}\right) \\
\frac{dx\left( t\right) }{dt} &=&-\omega A\sin \left( \omega t+\phi
_{o}\right) \\
\frac{d^{2}x\left( t\right) }{dt^{2}} &=&-\omega ^{2}A\cos \left( \omega
t+\phi _{o}\right)
\end{eqnarray*}%
Let's substitute these expressions into our differential equation for the
motion%
\begin{eqnarray*}
\frac{d^{2}x}{dt^{2}} &=&-\omega ^{2}x \\
-\omega ^{2}A\cos \left( \omega t+\phi _{o}\right) &=&-\omega ^{2}A\cos
\left( \omega t+\phi _{o}\right)
\end{eqnarray*}%
As long as the constant $\omega ^{2}$ is our $\omega ^{2}=k/m$ we have a
solution (now you know why we defined it as $\omega ^{2}$!). Since from trig
we remember $\omega $ as the angular frequency.%
\begin{equation*}
\omega =2\pi f
\end{equation*}

Thus%
\begin{equation}
\omega =\sqrt{\frac{k}{m}}=2\pi f
\end{equation}%
The frequency of oscillation depends on the mass and the stiffness of the
spring.%
\begin{equation}
f=\frac{1}{2\pi }\sqrt{\frac{k}{m}}
\end{equation}%
Let's see if this is reasonable. Imagine driving along in your student car
(say, a 1972 Gremlin). You go over a bump, and the car oscillates. Your car
is a mass, and your shock absorbers are springs. You have an oscillation.
But suppose you load your car with everyone in your apartment\footnote{%
If you are married, imagin taking two other couples with you in your car.}.
Now as you hit the bump the car oscillates at a different frequency, a lower
frequency. That is what our frequency equation tells us. Note also that if
we changed to a different set of shocks, the $k$ would change, and we would
get a different frequency.

We still don't have a complete solution to our differential equation,
because we don't know $A$ and $\phi _{o}.$ From trigonometry, we recognize $%
\phi _{o}$ as the initial phase angle. We will call it the \emph{phase
constant} in this class. We will have to find this by knowing the initial
conditions of the motion. We will do this in a minute.

$A$ is the amplitude. We can find its value when the motion has reached its
maximum displacement. Let's look at a specific case%
\begin{equation*}
\begin{tabular}{l}
$A=5$ \\ 
$\phi _{o}=0$ \\ 
$\omega =2$%
\end{tabular}%
\end{equation*}

\FRAME{dtbpFX}{2.7812in}{1.8542in}{0pt}{}{}{Plot}{\special{language
"Scientific Word";type "MAPLEPLOT";width 2.7812in;height 1.8542in;depth
0pt;display "USEDEF";plot_snapshots TRUE;mustRecompute FALSE;lastEngine
"MuPAD";xmin "-5";xmax "5";xviewmin "-5";xviewmax "5";yviewmin
"-10";yviewmax "10";viewset"XY";rangeset"X";plottype 4;labeloverrides
3;x-label "t";y-label "x";axesFont "Times New
Roman,12,0000000000,useDefault,normal";numpoints 100;plotstyle
"patch";axesstyle "normal";axestips FALSE;xis \TEXUX{t};var1name
\TEXUX{$t$};function \TEXUX{$5\cos \left( 2t+0\right) $};linecolor
"blue";linestyle 1;pointstyle "point";linethickness 1;lineAttributes
"Solid";var1range "-5,5";num-x-gridlines 100;curveColor
"[flat::RGB:0x000000ff]";curveStyle "Line";VCamFile
'LTUWDL2G.xvz';valid_file "T";tempfilename
'MS06JD00.wmf';tempfile-properties "XPR";}}We can easily see that the
amplitude $A$ corresponds to the maximum displacement $x_{\max }$.

\subsection{Other useful quantities we can identify}

We know from trigonometry that a cosine function has a period $T.$

\FRAME{fhF}{2.9101in}{1.9484in}{0pt}{}{}{Figure}{\special{language
"Scientific Word";type "GRAPHIC";maintain-aspect-ratio TRUE;display
"USEDEF";valid_file "T";width 2.9101in;height 1.9484in;depth
0pt;original-width 4.4616in;original-height 2.9793in;cropleft "0";croptop
"1";cropright "1";cropbottom "0";tempfilename
'LTUWCG0O.wmf';tempfile-properties "XPR";}}The period is related to the
frequency%
\begin{equation}
T=\frac{1}{f}=\frac{2\pi }{\omega }
\end{equation}%
We can write the period and frequency in terms of our mass and spring
constant%
\begin{eqnarray}
T &=&2\pi \sqrt{\frac{m}{k}} \\
f &=&\frac{1}{2\pi }\sqrt{\frac{k}{m}}
\end{eqnarray}

\subsection{Velocity and Acceleration}

Since we know the derivatives of%
\begin{equation}
x\left( t\right) =A\cos \left( \omega t+\phi _{o}\right)
\end{equation}%
we can identify the velocity of the mass and its acceleration

\begin{equation*}
v\left( t\right) =\frac{dx\left( t\right) }{dt}=-\omega A\sin \left( \omega
t+\phi _{o}\right)
\end{equation*}%
Recall that $A=x_{\max }$%
\begin{equation}
v\left( t\right) =\frac{dx\left( t\right) }{dt}=-\omega x_{\max }\sin \left(
\omega t+\phi _{o}\right)
\end{equation}%
We identify 
\begin{equation}
v_{\max }=\omega x_{\max }=x_{\max }\sqrt{\frac{k}{m}}
\end{equation}

Likewise for the acceleration%
\begin{eqnarray}
a\left( t\right) &=&\frac{dv\left( t\right) }{dt} \\
&=&\frac{d}{dt}\left( -\omega x_{\max }\sin \left( \omega t+\phi _{o}\right)
\right) \\
&=&-\omega ^{2}x_{\max }\cos \left( \omega t+\phi _{o}\right)  \notag
\end{eqnarray}

where we can identify%
\begin{equation}
a_{\max }=\omega ^{2}x_{\max }=\frac{k}{m}x_{\max }
\end{equation}

\subsection{Comparison of position, velocity, acceleration}

%TCIMACRO{%
%\TeXButton{Don't do in class}{\marginpar {
%\hspace{-0.5in}
%\begin{minipage}[t]{1in}
%\small{Don't do in class}
%\end{minipage}
%}}}%
%BeginExpansion
\marginpar {
\hspace{-0.5in}
\begin{minipage}[t]{1in}
\small{Don't do in class}
\end{minipage}
}%
%EndExpansion
Let's plot $x\left( t\right) ,$ $v\left( t\right) ,$ and $a\left( t\right) $
for a specific case\FRAME{dtbpFX}{2.7812in}{2.4163in}{0pt}{}{}{Plot}{\special%
{language "Scientific Word";type "MAPLEPLOT";width 2.7812in;height
2.4163in;depth 0pt;display "USEDEF";plot_snapshots TRUE;mustRecompute
FALSE;lastEngine "MuPAD";xmin "-3";xmax "3";xviewmin "-3";xviewmax
"3";yviewmin "-20";yviewmax "20";viewset"XY";rangeset"X";plottype
4;labeloverrides 3;x-label "t";y-label "Mixed units";axesFont "Times New
Roman,12,0000000000,useDefault,normal";numpoints 100;plotstyle
"patch";axesstyle "normal";axestips FALSE;xis \TEXUX{t};var1name
\TEXUX{$t$};function \TEXUX{$5\cos \left( 2t+0\right) $};linecolor
"blue";linestyle 1;pointstyle "point";linethickness 1;lineAttributes
"Solid";var1range "-3,3";num-x-gridlines 100;curveColor
"[flat::RGB:0x000000ff]";curveStyle "Line";function \TEXUX{$-2\ast 5\sin
\left( 5t+0\right) $};linecolor "green";linestyle 1;pointstyle
"point";linethickness 1;lineAttributes "Solid";var1range
"-3,3";num-x-gridlines 100;curveColor "[flat::RGB:0x00008000]";curveStyle
"Line";function \TEXUX{$-\left( 2\right) ^{2}\left( 5\right) \cos \left(
2t+0\right) $};linecolor "red";linestyle 1;pointstyle "point";linethickness
1;lineAttributes "Solid";var1range "-3,3";num-x-gridlines 100;curveColor
"[flat::RGB:0x00ff0000]";curveStyle "Line";VCamFile
'M2LRMZ03.xvz';valid_file "T";tempfilename
'MS06JX01.wmf';tempfile-properties "XPR";}}Red is the displacement, green is
the velocity, and blue is the acceleration. Note that each has a different
maximum amplitude. That is a bit confusing until we recognize that they each
have different units. We have just plotted them on the same graph to make it
easy to compare their phases. Note that they are not in phase!

\FRAME{dtbpF}{4.0655in}{3.8649in}{0pt}{}{}{Figure}{\special{language
"Scientific Word";type "GRAPHIC";maintain-aspect-ratio TRUE;display
"USEDEF";valid_file "T";width 4.0655in;height 3.8649in;depth
0pt;original-width 6.5639in;original-height 6.2396in;cropleft "0";croptop
"1";cropright "1";cropbottom "0";tempfilename
'LTUWCG0Q.wmf';tempfile-properties "XPR";}}The acceleration is $90\unit{%
%TCIMACRO{\U{b0}}%
%BeginExpansion
{{}^\circ}%
%EndExpansion
}$ out of phase from the velocity.

\FRAME{dtbpF}{3.9323in}{2.8461in}{0pt}{}{}{Figure}{\special{language
"Scientific Word";type "GRAPHIC";maintain-aspect-ratio TRUE;display
"USEDEF";valid_file "T";width 3.9323in;height 2.8461in;depth
0pt;original-width 7.043in;original-height 5.0903in;cropleft "0";croptop
"1";cropright "1";cropbottom "0";tempfilename
'LTUWCG0R.wmf';tempfile-properties "XPR";}}

\section{An example of oscillation}

We want to see how to find $A,$ $\omega ,$ and especially $\phi _{o}.$ These
quantities will be important in our study of waves. So let's do a problem.

Let's take as our system a horizontal mass-spring system where the mass is
on a frictionless surface.\FRAME{ftbpF}{2.1906in}{1.2358in}{0pt}{}{}{Figure}{%
\special{language "Scientific Word";type "GRAPHIC";maintain-aspect-ratio
TRUE;display "USEDEF";valid_file "T";width 2.1906in;height 1.2358in;depth
0pt;original-width 2.1508in;original-height 1.2012in;cropleft "0";croptop
"1";cropright "1";cropbottom "0";tempfilename
'LTUWCG0S.wmf';tempfile-properties "XPR";}}

\subsubsection{Initial Conditions}

Now let's find $A$ and $\phi _{o}.$ To do this we need to know how we
started the mass-spring motion. We call the information on how the system
starts it's motion the \emph{initial conditions.}

Suppose we start the motion by pulling the mass to $x=x_{\max }$ and
releasing it at $t=0.$ These our our initial conditions. Let's see if we can
find the phase. Our initial conditions require 
\begin{eqnarray}
x\left( 0\right) &=&x_{\max } \\
v\left( 0\right) &=&0  \notag
\end{eqnarray}

Using our formula for $x\left( t\right) $ and $v\left( t\right) $ we have%
\begin{eqnarray}
x\left( 0\right) &=&x_{\max }=x_{\max }\cos \left( 0+\phi _{o}\right) \\
v\left( 0\right) &=&0=-v_{\max }\sin \left( 0+\phi _{o}\right)  \notag
\end{eqnarray}%
From the first equation we get 
\begin{equation*}
1=\cos \left( \phi _{o}\right)
\end{equation*}%
which is true if 
\begin{equation*}
\phi _{o}=0,2\pi ,4\pi ,\cdots
\end{equation*}%
from the second equation we have 
\begin{equation*}
0=\sin \phi _{o}
\end{equation*}%
which is true for 
\begin{equation*}
\phi _{o}=0,\pi ,2\pi ,\cdots
\end{equation*}%
If we choose $\phi _{o}=0,$ these conditions are met. Of course we could
choose $\phi _{o}=2\pi ,$ or $\phi _{o}=4\pi ,$ but we will follow the rule
to take the smallest value for $\phi _{o}$ that meets the initial conditions.

\subsection{A second example}

Using the same equipment, let's start with 
\begin{eqnarray}
x\left( 0\right) &=&0 \\
v\left( 0\right) &=&+v_{i}  \notag
\end{eqnarray}%
that is, the mass is moving, and we start watching just as it passes the
equilibrium point.%
\begin{eqnarray}
x\left( 0\right) &=&0=x_{\max }\cos \left( 0+\phi _{o}\right) \\
v\left( 0\right) &=&v_{i}=-v_{\max }\sin \left( 0+\phi _{o}\right)  \notag
\end{eqnarray}

from 
\begin{equation*}
0=x_{\max }\cos \left( \phi _{o}\right)
\end{equation*}%
(first equation above) we see that\footnote{%
Really there are more possibilities, but we are taking the smallest value
for $\phi _{o}$ as we discussed above.} 
\begin{equation*}
\phi _{o}=\pm \frac{\pi }{2}
\end{equation*}%
but we don't know the sign. Using our initial velocity condition 
\begin{eqnarray*}
v_{i} &=&-v_{\max }\sin \left( \pm \frac{\pi }{2}\right) \\
v_{i} &=&-\omega x_{\max }\sin \left( \pm \frac{\pi }{2}\right)
\end{eqnarray*}

We defined the initial velocity as positive, and we insist on having
positive amplitudes, so $x_{\max }$ is positive. Thus we need a minus sign
from $\sin \left( \phi _{o}\right) $ to make $v_{i}$ positive. This tells us
to choose 
\begin{equation*}
\phi _{o}=-\frac{\pi }{2}
\end{equation*}%
with a minus sign.

Our solutions are%
\begin{eqnarray*}
x\left( t\right) &=&\frac{v_{i}}{\omega }\cos \left( \omega t-\frac{\pi }{2}%
\right) \\
v\left( t\right) &=&v_{i}\sin \left( \omega t-\frac{\pi }{2}\right)
\end{eqnarray*}

\begin{remark}
Generally to have a complete solution to a differential equation, you must
find all the constants (like $A$ and $\phi _{o}$) based on the initial
conditions.
\end{remark}

\subsection{A third example}

So far we have concentrated on finding $\phi _{o}.$ Let's do a more complete
example where we find $\phi _{o,}$ $A,$ and $\omega .$

A particle moving along the $x$ axis in simple harmonic motion starts from
its equilibrium position, the origin, at $t=0$ and moves to the right. The
amplitude of its motion is $4.00\unit{cm},$ and the frequency is $1.50\unit{%
Hz}.$

a) show that the position of the particle is given by%
\begin{equation*}
x=\left( 4.00\unit{cm}\right) \sin \left( 3.00\pi t\right)
\end{equation*}%
determine

b) the maximum speed and the earliest time $(t>0)$ at which the particle has
this speed,

c) the maximum acceleration and the earliest time $(t>0)$ at which the
particle has this acceleration, and

d) the total distance traveled between $t=0$ and $t=1.00\unit{s}$

\FRAME{dtbpF}{1.8775in}{0.8484in}{0pt}{}{}{Figure}{\special{language
"Scientific Word";type "GRAPHIC";maintain-aspect-ratio TRUE;display
"USEDEF";valid_file "T";width 1.8775in;height 0.8484in;depth
0pt;original-width 1.8403in;original-height 0.8164in;cropleft "0";croptop
"1";cropright "1";cropbottom "0";tempfilename
'LTUWCG0T.wmf';tempfile-properties "XPR";}}

\textbf{Type of problem}

We can recognize this as an oscillation problem. This leads us to a set of
basic equations

\textbf{Basic Equations}

\begin{eqnarray*}
x\left( t\right) &=&A\cos \left( \omega t+\phi _{o}\right) \\
v\left( t\right) &=&-\omega x_{\max }\sin \left( \omega t+\phi _{o}\right) \\
a\left( t\right) &=&-\omega ^{2}A\cos \left( \omega t+\phi _{o}\right)
\end{eqnarray*}

\begin{equation*}
\omega =2\pi f
\end{equation*}

\begin{equation*}
v_{m}=\omega x_{m}
\end{equation*}

\begin{equation*}
a_{m}=\omega ^{2}x_{m}
\end{equation*}

\begin{equation*}
T=\frac{1}{f}
\end{equation*}

We should write down what we know and give a set of variables

\textbf{Variables}

\begin{equation*}
\begin{tabular}{lll}
$t$ & time, initial time =0 & $t_{o}=0$ \\ 
$x$ & Position, Initial position =0 & $x\left( 0\right) =0$ \\ 
$v$ &  &  \\ 
$a$ &  &  \\ 
$x_{m}$ & $x$ amplitude & $x_{m}=4.00\unit{cm}$ \\ 
$v_{m}$ & $v$ amplitude &  \\ 
$a_{m}$ & $a$ amplitude &  \\ 
$\omega $ & angular frequency &  \\ 
$\phi _{o}$ & phase &  \\ 
$f$ & frequency & $f=1.50\unit{Hz}$%
\end{tabular}%
\end{equation*}

Now we are ready to start solving the problem. We do this with algebraic
symbols first

\textbf{Symbolic Solution}

\textbf{Part (a)}

We can start by recognizing that we can find $\omega $ because we know the
frequency. We just use the basic equation. 
\begin{equation*}
\omega =2\pi f
\end{equation*}%
We also know the amplitude $A=x_{\max }$ which is given. Knowing that%
\begin{equation*}
x\left( 0\right) =0=A\cos \left( 0+\phi _{o}\right)
\end{equation*}%
we can guess that 
\begin{equation*}
\phi _{o}=\pm \frac{\pi }{2}
\end{equation*}

Using%
\begin{equation*}
v\left( 0\right) =-\omega x_{\max }\sin \left( 0\pm \frac{\pi }{2}\right)
\end{equation*}%
again and demanding that amplitudes be positive values, and noting that at $%
t=0$ the velocity is positive from the initial conditions:%
\begin{equation*}
\phi _{o}=-\frac{\pi }{2}
\end{equation*}%
We also note from trigonometry that%
\begin{equation*}
x\left( t\right) =x_{\max }\cos \left( 2\pi ft-\frac{\pi }{2}\right)
\end{equation*}
which is a perfectly good answer. However, if we remember our trig, we could
write this using 
\begin{equation*}
\cos \left( \theta -\frac{\pi }{2}\right) =\sin \left( \theta \right)
\end{equation*}

Then we have%
\begin{eqnarray*}
x\left( t\right) &=&x_{\max }\cos \left( 2\pi ft-\frac{\pi }{2}\right) \\
&=&x_{\max }\sin \left( 2\pi ft\right)
\end{eqnarray*}

\textbf{Part (b)}

We have a basic equation for $v_{\max }$ 
\begin{eqnarray*}
v_{m} &=&\omega x_{\max } \\
&=&2\pi fx_{\max }
\end{eqnarray*}%
to find when this happens, take%
\begin{equation*}
v\left( t\right) =v_{\max }=-\omega x_{\max }\sin \left( 2\pi ft-\frac{\pi }{%
2}\right)
\end{equation*}%
and recognize that $\sin \left( \theta \right) =1$ is at a maximum when $%
\theta =\pi /2$ so the entire argument of the sine function must be $\pi /2$
when we are at the maximum displacement, so 
\begin{equation*}
\frac{\pi }{2}=\left( 2\pi ft-\frac{\pi }{2}\right)
\end{equation*}%
or%
\begin{equation*}
\pi =2\pi ft
\end{equation*}%
then the time is%
\begin{equation*}
\frac{1}{2f}=t
\end{equation*}

\textbf{Part (c)}

Like with the velocity we must use a basic formula, this time

\begin{equation*}
a\left( t\right) =-\omega ^{2}A\cos \left( \omega t+\phi _{o}\right)
\end{equation*}%
but recognize that the maximum is achieved when $\cos \left( \omega t+\phi
_{o}\right) =1$ or when $\omega t+\phi _{o}=0$

\begin{eqnarray*}
t &=&\frac{\phi _{o}}{\omega } \\
&=&\frac{-\frac{\pi }{2}}{2\pi f} \\
&=&\frac{-1}{4f}
\end{eqnarray*}%
The formula for $a_{\max }$ is 
\begin{eqnarray*}
a_{\max } &=&-\omega ^{2}x_{\max } \\
&=&-(2\pi f)^{2}x_{m}
\end{eqnarray*}

\textbf{Part (d)}

We know the period is 
\begin{equation*}
T=\frac{1}{f}
\end{equation*}%
We should find the number of periods in $t=1.00\unit{s}$ 
\begin{equation*}
N_{periods}=\frac{t}{T}
\end{equation*}%
and find the distance traveled in one period, and multiply them together. In
one period the distance traveled is 
\begin{equation*}
d=4x_{m}
\end{equation*}

\begin{equation*}
d_{tot}=d\ast \frac{t}{T}=4fx_{m}t
\end{equation*}

\textbf{Numerical Solutions}

We found algebraic answers (or symbolic answers) to the parts of our problem
above. We will always do this first. Then substitute in the numbers to find
numeric answers.

\textbf{Part (a)}

\begin{eqnarray*}
x\left( t\right) &=&x_{\max }\sin \left( 2\pi ft\right) \\
&=&\left( 4.00\unit{cm}\right) \sin \left( 3.00\pi t\right)
\end{eqnarray*}

\textbf{Part (b)}

\begin{eqnarray*}
v_{m} &=&2\pi \left( 1.50\unit{Hz}\right) \left( 4.00\unit{cm}\right) \\
&&0.377\frac{\unit{m}}{\unit{s}}
\end{eqnarray*}

\begin{equation*}
\frac{1}{2f}=t
\end{equation*}%
\begin{eqnarray*}
\frac{1}{2\left( 1.50\unit{Hz}\right) } &=&t \\
&=&\allowbreak 0.333\,\unit{s}
\end{eqnarray*}

\textbf{Part (c)}

\begin{eqnarray*}
t &=&\frac{-1}{4f} \\
&=&-0.166\,67\unit{s}
\end{eqnarray*}

\begin{eqnarray*}
a_{\max } &=&(2\pi f)^{2}x_{m} \\
&=&\left( 2\pi 1.5\unit{Hz}\right) ^{2}(4.00\unit{cm}) \\
&=&\allowbreak 3.\,\allowbreak 553\,1\frac{\unit{m}}{\unit{s}^{2}}
\end{eqnarray*}

\textbf{Part (d)}

\begin{eqnarray*}
d_{tot} &=&4fx_{m}t \\
&=&4\times 4.00\unit{cm}\ast 1.50\unit{Hz}\ast 1.00\unit{s} \\
&=&\allowbreak 0.24\unit{m}
\end{eqnarray*}

We should make sure the units check. We put in units along the way, so we
can be confident that they do. But if you did not work along the way with
units, check them now.

We should also make sure our answers are reasonable. If the amplitude came
out to be a billion miles, you might guess something went wrong. Always look
over your answers to make sure they seem reasonable.

\section{Energy of the Simple Harmonic Oscillator}

%TCIMACRO{%
%\TeXButton{Stop class here}{\marginpar {
%\hspace{-0.5in}
%\begin{minipage}[t]{1in}
%\small{Stop class here}
%\end{minipage}
%}}}%
%BeginExpansion
\marginpar {
\hspace{-0.5in}
\begin{minipage}[t]{1in}
\small{Stop class here}
\end{minipage}
}%
%EndExpansion
If there is motion, there is energy. We can find the energy in a harmonic
oscillator. Let's start with kinetic energy. Recall that 
\begin{equation*}
K=\frac{1}{2}mv^{2}
\end{equation*}%
for our Simple Harmonic Oscillator (SHO) we have%
\begin{eqnarray*}
K &=&\frac{1}{2}m\left( -\omega x_{\max }\sin \left( \omega t+\phi
_{o}\right) \right) ^{2} \\
&=&\frac{1}{2}m\omega ^{2}x_{\max }^{2}\sin ^{2}\left( \omega t+\phi
_{o}\right) \\
&=&\frac{1}{2}m\frac{k}{m}x_{\max }^{2}\sin ^{2}\left( \omega t+\phi
_{o}\right) \\
&=&\frac{1}{2}kx_{\max }^{2}\sin ^{2}\left( \omega t+\phi _{o}\right)
\end{eqnarray*}%
The potential energy due to a spring is given by (from your PH121 class or
Statics/Dynamics)%
\begin{equation}
U=\frac{1}{2}kx^{2}
\end{equation}%
Again for our SHO we have%
\begin{equation}
U=\frac{1}{2}kx_{\max }^{2}\cos ^{2}\left( \omega t+\phi _{o}\right)
\end{equation}%
The total energy is given by%
\begin{eqnarray}
E &=&K+U \\
&=&\frac{1}{2}kx_{\max }^{2}\sin ^{2}\left( \omega t+\phi _{o}\right) +\frac{%
1}{2}kx_{\max }^{2}\cos ^{2}\left( \omega t+\phi _{o}\right)  \notag \\
&=&\frac{1}{2}kx_{\max }^{2}\left( \sin ^{2}\left( \omega t+\phi _{o}\right)
+\cos ^{2}\left( \omega t+\phi _{o}\right) \right)  \notag \\
&=&\frac{1}{2}kx_{\max }^{2}  \notag
\end{eqnarray}

This is an astounding result! The amount of energy at any given time is
equal to the amount of energy we started with. We are not changing how much
energy we have. We call such a value that does not change a \emph{constant
of motion.}

\begin{remark}
The total mechanical energy of a SHO is a constant of motion
\end{remark}

\FRAME{dtbpF}{4.7539in}{3.5699in}{0pt}{}{}{Figure}{\special{language
"Scientific Word";type "GRAPHIC";maintain-aspect-ratio TRUE;display
"USEDEF";valid_file "T";width 4.7539in;height 3.5699in;depth
0pt;original-width 10.0258in;original-height 7.5247in;cropleft "0";croptop
"1";cropright "1";cropbottom "0";tempfilename
'LTUWCG0U.wmf';tempfile-properties "XPR";}}In the figure you can see that
the kinetic and potential energies trade back and forth, but the total
amount of energy does not change. Note that the kinetic and potential energy
are out of phase with each other. If we plot them on the same scale ( for
the case $\phi _{o}=0$) we have\FRAME{dtbpF}{3.2655in}{2.2364in}{0pt}{}{}{%
Figure}{\special{language "Scientific Word";type
"GRAPHIC";maintain-aspect-ratio TRUE;display "USEDEF";valid_file "T";width
3.2655in;height 2.2364in;depth 0pt;original-width 3.2206in;original-height
2.1958in;cropleft "0";croptop "1";cropright "1";cropbottom "0";tempfilename
'LTUWCG0V.wmf';tempfile-properties "XPR";}}

\section{Circular Motion and SHM}

That circular motion and SHM are related should not be a surprise once we
found the solutions to the equations of motion were trig functions. Recall
that the trig functions are defined on a unit circle

\FRAME{ftbpF}{1.8991in}{1.9303in}{0pt}{}{}{Figure}{\special{language
"Scientific Word";type "GRAPHIC";maintain-aspect-ratio TRUE;display
"USEDEF";valid_file "T";width 1.8991in;height 1.9303in;depth
0pt;original-width 4.8559in;original-height 4.9355in;cropleft "0";croptop
"1";cropright "1";cropbottom "0";tempfilename
'LTUWCH0W.wmf';tempfile-properties "XPR";}}%
\begin{eqnarray}
\tan \theta &=&\frac{x}{y} \\
\cos \theta &=&\frac{x}{h} \\
\sin \theta &=&\frac{y}{h}
\end{eqnarray}

Let's relate this to our equations of motion.\FRAME{dtbpF}{2.7181in}{2.6195in%
}{0pt}{}{}{Figure}{\special{language "Scientific Word";type
"GRAPHIC";maintain-aspect-ratio TRUE;display "USEDEF";valid_file "T";width
2.7181in;height 2.6195in;depth 0pt;original-width 2.6757in;original-height
2.578in;cropleft "0";croptop "1";cropright "1";cropbottom "0";tempfilename
'LTUWCH0X.wmf';tempfile-properties "XPR";}}Look at the projection $x$ of the
point $P$ on the $x$ axis. Lets follow this projection as $P$ travels around
the circle. We find it ranges from $-x_{\max }$ to $x_{\max }.$ If we watch
closely we find its velocity is zero at the extreme points and is a maximum
in the middle. This projection is given as the $\cos $ of the vector from
the origin to $P.$ This model, indeed fits our SHO solution.

Now lets define a projection of $P$ onto the $y$ axis. Again we have SHM,
but this time the projection is a $\sin $ function. Because 
\begin{equation}
\cos \left( \theta -\frac{\pi }{2}\right) =\sin \left( \theta \right)
\end{equation}%
we can see that this is just a SHO that is $90\unit{%
%TCIMACRO{\U{b0}}%
%BeginExpansion
{{}^\circ}%
%EndExpansion
}$ out of phase.

\begin{remark}
We see that uniform circular motion can be thought of as the combination of
two SHOs, with a phase difference of $90\unit{%
%TCIMACRO{\U{b0}}%
%BeginExpansion
{{}^\circ}%
%EndExpansion
}.$
\end{remark}

The angular velocity is given by 
\begin{equation}
\omega =\frac{v}{r}
\end{equation}%
\FRAME{dtbpF}{1.8023in}{1.8325in}{0pt}{}{}{Figure}{\special{language
"Scientific Word";type "GRAPHIC";maintain-aspect-ratio TRUE;display
"USEDEF";valid_file "T";width 1.8023in;height 1.8325in;depth
0pt;original-width 4.4373in;original-height 4.5143in;cropleft "0";croptop
"1";cropright "1";cropbottom "0";tempfilename
'LTUWCH0Y.wmf';tempfile-properties "XPR";}}A particle traveling on the $x$%
-axis in SHM will travel from $x_{\max }\ $to $-x_{\max }$ and from $%
-x_{\max }\ $to $x_{\max }$ (one complete period, $T$) while the particle
traveling with $P$ makes one complete revolution. Thus, the angular
frequency $\omega $ of the SHO and the angular velocity of the particle at $%
P $ are the same. (Now we know why we used the same symbol). The magnitude
of the velocity is then%
\begin{equation}
v=\omega r=\omega x_{\max }
\end{equation}%
and the projection of this velocity onto the $x$-axis is 
\begin{equation}
v_{x}=-\omega x_{\max }\sin \left( \omega t+\phi _{o}\right)
\end{equation}%
Just what we expected!

The angular acceleration of a particle at $P$ is given by 
\begin{equation}
\frac{v^{2}}{r}=\frac{v^{2}}{x_{\max }}=\frac{\omega ^{2}x_{\max }^{2}}{%
x_{\max }}=\omega ^{2}x_{\max }
\end{equation}%
\FRAME{dtbpF}{1.7608in}{1.7971in}{0pt}{}{}{Figure}{\special{language
"Scientific Word";type "GRAPHIC";maintain-aspect-ratio TRUE;display
"USEDEF";valid_file "T";width 1.7608in;height 1.7971in;depth
0pt;original-width 4.4373in;original-height 4.5316in;cropleft "0";croptop
"1";cropright "1";cropbottom "0";tempfilename
'LTUWCH0Z.wmf';tempfile-properties "XPR";}}The direction of the acceleration
is inward toward the origin. If we project this onto the $x$-axis we have%
\begin{equation}
a_{x}=-\omega ^{2}x_{\max }\cos \left( \omega t+\phi _{o}\right)
\end{equation}

\section{The Pendulum}

\FRAME{dtbpF}{2.4457in}{2.8444in}{0pt}{}{}{Figure}{\special{language
"Scientific Word";type "GRAPHIC";maintain-aspect-ratio TRUE;display
"USEDEF";valid_file "T";width 2.4457in;height 2.8444in;depth
0pt;original-width 2.4042in;original-height 2.802in;cropleft "0";croptop
"1";cropright "1";cropbottom "0";tempfilename
'LTUWCH10.wmf';tempfile-properties "XPR";}}A simple pendulum is a mass on a
string. The mass is called a \textquotedblleft bob.\textquotedblright

A simple pendulum exhibits periodic motion, but not exactly simple harmonic
motion.

The forces on the bob, $m,$ are $\mathbf{F}_{g}$, $\mathbf{T}$ the tension
on the string. The tangential component of $F_{g}$ is always directed toward 
$\theta =0.$ This is a restoring force!

Let's call the path the bob takes $s.$ The path, $s$, is along an arc, then
from Jr. High geometry\footnote{%
From Jr. High, but if you are like me you have forgotten it until now.}, we
can use the arc-length formula to describe $s$%
\begin{equation}
s=L\theta  \tag{10.01a}
\end{equation}%
and we can write an equation for the restoring force that brings the bob
back to its equilibrium position as%
\begin{eqnarray}
F_{t} &=&-mg\sin \theta \\
&=&m\frac{d^{2}s}{dt^{2}}  \notag \\
&=&mL\frac{d^{2}\theta }{dt^{2}}  \notag
\end{eqnarray}%
or%
\begin{equation*}
\frac{d^{2}\theta }{dt^{2}}=-\frac{g}{L}\sin \theta
\end{equation*}%
This is a harder differential equation to solve. But suppose we are building
a grandfather clock with our pendulum, and we won't let the pendulum swing
very far. Then we can take $\theta $ as a very small angle, then 
\begin{equation}
\sin \left( \theta \right) \approx \theta
\end{equation}

In this approximation%
\begin{equation*}
\frac{d^{2}\theta }{dt^{2}}=-\frac{g}{L}\theta
\end{equation*}%
and we have a differential equation we recognize! Compare to 
\begin{equation}
\frac{d^{2}x}{dt^{2}}=-\omega ^{2}x
\end{equation}

if 
\begin{equation}
\omega ^{2}=\frac{g}{L}
\end{equation}%
we have all the same solutions for $s$ that we found for $x.$ Since $\omega $
changed, the frequency and period will now be in terms of $g$ and $L.$%
\begin{equation}
T=\frac{2\pi }{\omega }=2\pi \sqrt{\frac{L}{g}}
\end{equation}

\begin{remark}
the period and frequency for a pendulum with small angular displacements
depend only on $L$ and $g$!
\end{remark}

\section{Damped Oscillations}

%TCIMACRO{%
%\TeXButton{Question 223.1.7}{\marginpar {
%\hspace{-0.5in}
%\begin{minipage}[t]{1in}
%\small{Question 223.1.7}
%\end{minipage}
%}}}%
%BeginExpansion
\marginpar {
\hspace{-0.5in}
\begin{minipage}[t]{1in}
\small{Question 223.1.7}
\end{minipage}
}%
%EndExpansion
%TCIMACRO{%
%\TeXButton{Question 223.1.8}{\marginpar {
%\hspace{-0.5in}
%\begin{minipage}[t]{1in}
%\small{Question 223.1.8}
%\end{minipage}
%}}}%
%BeginExpansion
\marginpar {
\hspace{-0.5in}
\begin{minipage}[t]{1in}
\small{Question 223.1.8}
\end{minipage}
}%
%EndExpansion

Suppose we add in another force%
\begin{equation}
\mathbf{F}_{d}=-b\mathbf{v}
\end{equation}%
This force is proportional to the velocity. This is typical of viscus
fluids. So this is what we would get if we place our mass-spring system (or
pendulum) in air or some other fluid. We call $b$ the damping coefficient.
Now our net force is 
\begin{equation*}
\Sigma F=-kx-bv_{x}=ma
\end{equation*}%
We can write the acceleration and velocity as derivatives of the position 
\begin{equation*}
-kx-b\frac{dx}{dt}=m\frac{d^{2}x}{dt^{2}}
\end{equation*}%
This is another differential equation. It is harder to guess its solution%
\begin{equation}
x\left( t\right) =Ae^{-\frac{b}{2m}t}\cos \left( \omega t+\phi _{o}\right)
\label{dampped solution}
\end{equation}%
but now our angular frequency, $\omega ,$ is more complicated%
\begin{equation}
\omega =\sqrt{\frac{k}{m}-\left( \frac{b}{2m}\right) ^{2}}
\end{equation}

We have three cases:

\begin{case}
\begin{enumerate}
\item the retarding force is small: ($bv_{\max }<kA)$ The system oscillates,
but the amplitude is smaller as as time goes on. We call this ``underdamped''

\item the retarding force is large: ($bv_{\max }>kA)$The system does not
oscillate. we call this ``overdamped.'' We can also say that $\frac{b}{2m}%
>\omega _{o}$ (after we define $\omega _{o}$ below)

\item The system is critically damped (see below)
\end{enumerate}
\end{case}

For the following values,

\begin{equation*}
\begin{tabular}{l}
$A=5\unit{cm}$ \\ 
$b=0.005\frac{\unit{kg}}{\unit{s}}$ \\ 
$k=.5\frac{\unit{N}}{\unit{m}}$ \\ 
$m=.5\unit{kg}$%
\end{tabular}%
\end{equation*}

we have a graph that looks like this\FRAME{dtbpFX}{4.4996in}{3in}{0pt}{}{}{%
Plot}{\special{language "Scientific Word";type "MAPLEPLOT";width
4.4996in;height 3in;depth 0pt;display "USEDEF";plot_snapshots
TRUE;mustRecompute FALSE;lastEngine "MuPAD";xmin "0";xmax "500";xviewmin
"0";xviewmax "500";yviewmin "-0.050771";yviewmax
"0.05";viewset"XY";rangeset"X";plottype 4;axesFont "Times New
Roman,12,0000000000,useDefault,normal";numpoints 100;plotstyle
"patch";axesstyle "normal";axestips FALSE;xis \TEXUX{t};var1name
\TEXUX{$t$};function \TEXUX{$\left( 0.05\right) e^{-\frac{0.005}{2\left(
0.5\right) }t}$};linecolor "gray";linestyle 1;pointstyle
"point";linethickness 1;lineAttributes "Solid";var1range
"0,500";num-x-gridlines 100;curveColor "[flat::RGB:0x00c0c0c0]";curveStyle
"Line";rangeset"X";function \TEXUX{$-\left( 0.05\right)
e^{-\frac{0.005}{2\left( 0.5\right) }t}$};linecolor "yellow";linestyle
1;pointstyle "point";linethickness 1;lineAttributes "Solid";var1range
"0,500";num-x-gridlines 100;curveColor "[flat::RGB:0x00808000]";curveStyle
"Line";function \TEXUX{$\left( 0.05\right) e^{-\frac{0.005}{2\left(
0.5\right) }t}\cos \left( \left( \left( \frac{0.5}{0.5}-\left(
\frac{0.005}{2\left( 0.05\right) }\right) ^{2}\right) ^{\frac{1}{2}}\right)
t\right) $};linecolor "blue";linestyle 1;pointstyle "point";linethickness
1;lineAttributes "Solid";var1range "0,500";num-x-gridlines 100;curveColor
"[flat::RGB:0x000000ff]";curveStyle "Line";VCamFile
'LTUWDL2E.xvz';valid_file "T";tempfilename
'LTUWCH11.wmf';tempfile-properties "XPR";}}The gray lines are%
\begin{equation}
\pm Ae^{-\frac{b}{2m}t}
\end{equation}%
They describe how the amplitude changes. We call this the \emph{envelope} of
the curve.%
\begin{equation*}
\begin{tabular}{l}
$A=5\unit{cm}$ \\ 
$b=0.05\frac{\unit{kg}}{\unit{s}}$ \\ 
$k=.5\frac{\unit{N}}{\unit{m}}$ \\ 
$m=.5\unit{kg}$%
\end{tabular}%
\end{equation*}%
\FRAME{dtbpFX}{4.4996in}{3in}{0pt}{}{}{Plot}{\special{language "Scientific
Word";type "MAPLEPLOT";width 4.4996in;height 3in;depth 0pt;display
"USEDEF";plot_snapshots TRUE;mustRecompute FALSE;lastEngine "MuPAD";xmin
"0";xmax "500";xviewmin "0";xviewmax "500";yviewmin "-0.050771";yviewmax
"0.05";viewset"XY";rangeset"X";plottype 4;axesFont "Times New
Roman,12,0000000000,useDefault,normal";numpoints 100;plotstyle
"patch";axesstyle "normal";axestips FALSE;xis \TEXUX{t};var1name
\TEXUX{$t$};function \TEXUX{$\left( 0.05\right) e^{-\frac{0.05}{2\left(
0.5\right) }t}$};linecolor "gray";linestyle 1;pointstyle
"point";linethickness 1;lineAttributes "Solid";var1range
"0,500";num-x-gridlines 100;curveColor "[flat::RGB:0x00c0c0c0]";curveStyle
"Line";rangeset"X";function \TEXUX{$-\left( 0.05\right)
e^{-\frac{0.05}{2\left( 0.5\right) }t}$};linecolor "yellow";linestyle
1;pointstyle "point";linethickness 1;lineAttributes "Solid";var1range
"0,500";num-x-gridlines 100;curveColor "[flat::RGB:0x00808000]";curveStyle
"Line";function \TEXUX{$\left( 0.05\right) e^{-\frac{0.05}{2\left(
0.5\right) }t}\cos \left( \left( \left( \frac{0.5}{0.5}-\left(
\frac{0.05}{2\left( 0.05\right) }\right) ^{2}\right) ^{\frac{1}{2}}\right)
t\right) $};linecolor "blue";linestyle 1;pointstyle "point";linethickness
1;lineAttributes "Solid";var1range "0,500";num-x-gridlines 100;curveColor
"[flat::RGB:0x000000ff]";curveStyle "Line";VCamFile
'LTUWDL2D.xvz';valid_file "T";tempfilename
'LTUWCH12.wmf';tempfile-properties "XPR";}}

\begin{equation*}
\begin{tabular}{l}
$A=5\unit{cm}$ \\ 
$b=0.5\frac{\unit{kg}}{\unit{s}}$ \\ 
$k=.5\frac{\unit{N}}{\unit{m}}$ \\ 
$m=.5\unit{kg}$%
\end{tabular}%
\end{equation*}%
\FRAME{dtbpFX}{4.4996in}{3in}{0pt}{}{}{Plot}{\special{language "Scientific
Word";type "MAPLEPLOT";width 4.4996in;height 3in;depth 0pt;display
"USEDEF";plot_snapshots TRUE;mustRecompute FALSE;lastEngine "MuPAD";xmin
"0";xmax "50";xviewmin "0";xviewmax "50";yviewmin "-0.050005";yviewmax
"0.05";viewset"XY";rangeset"X";plottype 4;axesFont "Times New
Roman,12,0000000000,useDefault,normal";numpoints 100;plotstyle
"patch";axesstyle "normal";axestips FALSE;xis \TEXUX{t};var1name
\TEXUX{$t$};function \TEXUX{$\left( 0.05\right) e^{-\frac{1}{2\left(
0.5\right) }t}$};linecolor "gray";linestyle 1;pointstyle
"point";linethickness 1;lineAttributes "Solid";var1range
"0,50";num-x-gridlines 100;curveColor "[flat::RGB:0x00c0c0c0]";curveStyle
"Line";rangeset"X";function \TEXUX{$-\left( 0.05\right) e^{-\frac{1}{2\left(
0.5\right) }t}$};linecolor "yellow";linestyle 1;pointstyle
"point";linethickness 1;lineAttributes "Solid";var1range
"0,50";num-x-gridlines 100;curveColor "[flat::RGB:0x00808000]";curveStyle
"Line";VCamFile 'LTUWDL2C.xvz';valid_file "T";tempfilename
'LTUWCH13.wmf';tempfile-properties "XPR";}}What happened?

When the damping force gets bigger, the oscillation eventually stops. Only
the exponential decay is observed. This happens when%
\begin{equation}
\frac{b}{2m}=\sqrt{\frac{k}{m}}
\end{equation}%
then%
\begin{equation}
\omega =\sqrt{\frac{k}{m}-\left( \frac{b}{2m}\right) ^{2}}=0
\end{equation}%
We call this situation we call critically damped. We are just on the edge of
oscillation. We define%
\begin{equation}
\omega _{o}=\sqrt{\frac{k}{m}}
\end{equation}%
as the \emph{natural frequency} of the system. Then the value of $b$ that
gives us critically damped behavior is%
\begin{equation}
b_{c}=2m\omega _{o}
\end{equation}

\begin{remark}
When $\frac{b}{2\pi }\geq \omega _{o}$ the solution in equation (\ref%
{dampped solution}) is not valid! If you are a mechanical engineer you will
find out more about this situation in your advanced mechanics classes.
\end{remark}

\section{Driven Oscillations and Resonance}

%TCIMACRO{%
%\TeXButton{Question 223.1.9}{\marginpar {
%\hspace{-0.5in}
%\begin{minipage}[t]{1in}
%\small{Question 223.1.9}
%\end{minipage}
%}}}%
%BeginExpansion
\marginpar {
\hspace{-0.5in}
\begin{minipage}[t]{1in}
\small{Question 223.1.9}
\end{minipage}
}%
%EndExpansion

%TCIMACRO{%
%\TeXButton{Question 223.1.10}{\marginpar {
%\hspace{-0.5in}
%\begin{minipage}[t]{1in}
%\small{Question 223.1.10}
%\end{minipage}
%}}}%
%BeginExpansion
\marginpar {
\hspace{-0.5in}
\begin{minipage}[t]{1in}
\small{Question 223.1.10}
\end{minipage}
}%
%EndExpansion
%TCIMACRO{%
%\TeXButton{Question 223.1.11}{\marginpar {
%\hspace{-0.5in}
%\begin{minipage}[t]{1in}
%\small{Question 223.1.11}
%\end{minipage}
%}}}%
%BeginExpansion
\marginpar {
\hspace{-0.5in}
\begin{minipage}[t]{1in}
\small{Question 223.1.11}
\end{minipage}
}%
%EndExpansion

We found in the last section that if we added a force like

\begin{equation}
\mathbf{F}_{d}=-b\mathbf{v}
\end{equation}%
our oscillation died out. Suppose we want to keep it going? Let's apply a
periodic force like%
\begin{equation*}
F\left( t\right) =F_{o}\sin \left( \omega _{f}t\right)
\end{equation*}%
where $\omega _{f}$ is the angular frequency of this new driving force and
where $F_{o}$ is a constant.%
\begin{equation*}
\Sigma F=F_{o}\sin \left( \omega _{f}t\right) -kx-bv_{x}=ma
\end{equation*}

When this system starts out, the solutions is very messy. It is so messy
that we will not give it in this class! But after a while, a steady-state is
reached. In this state, the energy added by our driving force $F_{o}\sin
\left( \omega _{f}t\right) $ is equal to the energy lost by the drag force,
and we have 
\begin{equation}
x\left( t\right) =A\cos \left( \omega _{f}t+\phi _{o}\right)
\end{equation}%
our old friend! BUT NOW 
\begin{equation}
A=\frac{\frac{F_{o}}{m}}{\sqrt{\left( \omega _{f}^{2}-\omega _{o}^{2}\right)
^{2}+\left( \frac{b\omega _{f}}{m}\right) ^{2}}}
\end{equation}%
and where%
\begin{equation}
\omega _{o}=\sqrt{\frac{k}{m}}
\end{equation}%
as before. It is more convenient to drop the $f$ subscripts 
\begin{equation}
x\left( t\right) =A\cos \left( \omega t+\phi _{o}\right)
\end{equation}%
\begin{equation}
A=\frac{\frac{F_{o}}{m}}{\sqrt{\left( \omega ^{2}-\omega _{o}^{2}\right)
^{2}+\left( \frac{b\omega }{m}\right) ^{2}}}
\end{equation}%
so now our solution looks more like our original SHM solution (except for
the wild formula for $A$).

Lets look at $A$ for some values of $\omega .$ I will pick some nice numbers
for the other values.%
\begin{equation*}
\begin{tabular}{l}
$F_{o}=2\unit{N}$ \\ 
$b=0.5\frac{\unit{kg}}{\unit{s}}$ \\ 
$k=0.5\frac{\unit{N}}{\unit{m}}$ \\ 
$m=0.5\unit{kg}$%
\end{tabular}%
\end{equation*}%
\FRAME{dtbpFX}{4.4996in}{3in}{0pt}{}{}{Plot}{\special{language "Scientific
Word";type "MAPLEPLOT";width 4.4996in;height 3in;depth 0pt;display
"USEDEF";plot_snapshots TRUE;mustRecompute FALSE;lastEngine "MuPAD";xmin
"0";xmax "3";xviewmin "0";xviewmax "3";yviewmin "0";yviewmax
"250.7930";viewset"XY";rangeset"X";plottype 4;axesFont "Times New
Roman,12,0000000000,useDefault,normal";numpoints 100;plotstyle
"patch";axesstyle "normal";axestips FALSE;xis \TEXUX{v58147};var1name
\TEXUX{$\omega $};function \TEXUX{$\allowbreak
\frac{20.0}{\sqrt{0.0001\omega ^{2}+\left( \omega ^{2}-1.0\right)
^{2}}}$};linecolor "green";linestyle 1;pointstyle "point";linethickness
1;lineAttributes "Solid";var1range "0,3";num-x-gridlines 100;curveColor
"[flat::RGB:0x00008000]";curveStyle "Line";VCamFile
'LTUWDL2B.xvz';valid_file "T";tempfilename
'LTUWCH14.wmf';tempfile-properties "XPR";}}now let's calculate $\omega $ 
\begin{eqnarray*}
\omega _{o} &=&\sqrt{\frac{0.5\frac{\unit{N}}{\unit{m}}}{0.5\unit{kg}}} \\
&=&\allowbreak \frac{1.0}{\unit{s}}
\end{eqnarray*}%
Notice that right at $\omega $ our solution gets very big. This is called 
\emph{resonance}. To see why this happens, think of the velocity%
\begin{equation}
\frac{dx\left( t\right) }{dt}=-\omega A\sin \left( \omega t+\phi _{o}\right)
\end{equation}%
note that our driving force is%
\begin{equation}
F\left( t\right) =F_{o}\sin \left( \omega t\right)
\end{equation}%
The rate at which work is done (power) is 
\begin{equation}
\mathcal{P}=\frac{\mathbf{F}\cdot \Delta \mathbf{x}}{\Delta t}=\mathbf{%
F\cdot v}
\end{equation}%
if $F$ and $v$ are in phase, the power will be at a maximum!

We can plot $A$ for several values of $b$\FRAME{dtbpFUX}{4.4996in}{3in}{0pt}{%
\Qcb{Green: b=0.005kg/s; Blue: b=0.05kg/s; Red b=0.01 kg/s}}{}{Plot}{\special%
{language "Scientific Word";type "MAPLEPLOT";width 4.4996in;height 3in;depth
0pt;display "USEDEF";plot_snapshots TRUE;mustRecompute FALSE;lastEngine
"MuPAD";xmin "0";xmax "3";xviewmin "0";xviewmax "3";yviewmin "0";yviewmax
"250.7930";viewset"XY";rangeset"X";plottype 4;axesFont "Times New
Roman,12,0000000000,useDefault,normal";numpoints 100;plotstyle
"patch";axesstyle "normal";axestips FALSE;xis \TEXUX{v58147};var1name
\TEXUX{$\omega $};function \TEXUX{$\allowbreak
\frac{20.0}{\sqrt{0.0001\omega ^{2}+\left( \omega ^{2}-1.0\right)
^{2}}}$};linecolor "green";linestyle 1;pointstyle "point";linethickness
1;lineAttributes "Solid";var1range "0,3";num-x-gridlines 100;curveColor
"[flat::RGB:0x00008000]";curveStyle "Line";function
\TEXUX{$\frac{20.0}{\sqrt{1\omega ^{2}+\left( \omega ^{2}-1.0\right)
^{2}}}$};linecolor "blue";linestyle 1;pointstyle "point";linethickness
1;lineAttributes "Solid";var1range "0,3";num-x-gridlines 100;curveColor
"[flat::RGB:0x000000ff]";curveStyle "Line";function
\TEXUX{$\frac{20.0}{\sqrt{0.01\omega ^{2}+\left( \omega ^{2}-1.0\right)
^{2}}}$};linecolor "red";linestyle 1;pointstyle "point";linethickness
1;lineAttributes "Solid";var1range "0,3";num-x-gridlines 100;curveColor
"[flat::RGB:0x00ff0000]";curveStyle "Line";VCamFile
'LTUWDL2A.xvz';valid_file "T";tempfilename
'LTUWCH15.wmf';tempfile-properties "XPR";}}As $b\rightarrow 0$ we see that
our resonance peak gets larger. In real systems $b$ can never be zero, but
sometimes it can get small. As $b\rightarrow $large, the resonance dies down
and our $A$ gets small.

An example of this is well known to mechanical engineers. The next picture
is of the Tacoma Narrows Bridge. As a steady wind blew across the bridge it
formed turbulent wind gusts.\FRAME{dtbpFU}{2.6212in}{1.8351in}{0pt}{\Qcb{%
Tacoma Narrows Bridge (Image in the Public Domain)}}{}{Figure}{\special%
{language "Scientific Word";type "GRAPHIC";maintain-aspect-ratio
TRUE;display "USEDEF";valid_file "T";width 2.6212in;height 1.8351in;depth
0pt;original-width 6.1644in;original-height 4.3068in;cropleft "0";croptop
"1";cropright "1";cropbottom "0";tempfilename
'LTUWCH16.wmf';tempfile-properties "XPR";}}The wind gusts formed a periodic
driving force that allowed a driving harmonic oscillation to form. Since the
bridge was resonant with the gust frequency, the amplitude grew until the
bridge materials broke.

\chapter{What is a Wave?}

%TCIMACRO{%
%\TeXButton{Fundamental Concepts}{\hspace{-1.3in}{\LARGE Fundamental Concepts\vspace{0.25in}}}}%
%BeginExpansion
\hspace{-1.3in}{\LARGE Fundamental Concepts\vspace{0.25in}}%
%EndExpansion

\begin{enumerate}
\item A wave requires a disturbance, and a medium that can transfer energy

\item Waves are categorized as longitudinal or transverse (or a combination
of the two).
\end{enumerate}

\section{What is a Wave?}

Waves are organized motions in a medium.%
%TCIMACRO{%
%\TeXButton{Spring Demo}{\marginpar {
%\hspace{-0.5in}
%\begin{minipage}[t]{1in}
%\small{Spring Demo}
%\end{minipage}
%}}}%
%BeginExpansion
\marginpar {
\hspace{-0.5in}
\begin{minipage}[t]{1in}
\small{Spring Demo}
\end{minipage}
}%
%EndExpansion

\subsection{Criteria for being a wave}

%TCIMACRO{%
%\TeXButton{Question 223.2.1}{\marginpar {
%\hspace{-0.5in}
%\begin{minipage}[t]{1in}
%\small{Question 223.2.1}
%\end{minipage}
%}}}%
%BeginExpansion
\marginpar {
\hspace{-0.5in}
\begin{minipage}[t]{1in}
\small{Question 223.2.1}
\end{minipage}
}%
%EndExpansion
%TCIMACRO{%
%\TeXButton{Question 223.2.2}{\marginpar {
%\hspace{-0.5in}
%\begin{minipage}[t]{1in}
%\small{Question 223.2.2}
%\end{minipage}
%}}}%
%BeginExpansion
\marginpar {
\hspace{-0.5in}
\begin{minipage}[t]{1in}
\small{Question 223.2.2}
\end{minipage}
}%
%EndExpansion

Another way to think about waves is a transfer of energy through space
without transfer of matter.

Waves require:%
%TCIMACRO{%
%\TeXButton{Spring Demo-marked part}{\marginpar {
%\hspace{-0.5in}
%\begin{minipage}[t]{1in}
%\small{Spring Demo-marked part}
%\end{minipage}
%}}}%
%BeginExpansion
\marginpar {
\hspace{-0.5in}
\begin{minipage}[t]{1in}
\small{Spring Demo-marked part}
\end{minipage}
}%
%EndExpansion

\begin{enumerate}
\item some source of disturbance

\item a medium that can be disturbed

\item some physical mechanism by which the elements of the medium can
influence each other
\end{enumerate}

In the limit that the string mass is negligible we represent a
one-dimensional wave mathematically as a function of two variables, position
and time, $y\left( x,t\right) .$ There are two ways to look at waves, we
call them \textquotedblleft snapshot\textquotedblright\ and
\textquotedblleft history\textquotedblright\ (or video) views.

\subsection{Longitudinal vs. transverse}

We divide the various kinds of waves that occur into two basic types:

%TCIMACRO{%
%\TeXButton{Question 223.2.3}{\marginpar {
%\hspace{-0.5in}
%\begin{minipage}[t]{1in}
%\small{Question 223.2.3}
%\end{minipage}
%}}}%
%BeginExpansion
\marginpar {
\hspace{-0.5in}
\begin{minipage}[t]{1in}
\small{Question 223.2.3}
\end{minipage}
}%
%EndExpansion
%TCIMACRO{%
%\TeXButton{Question 223.2.4}{\marginpar {
%\hspace{-0.5in}
%\begin{minipage}[t]{1in}
%\small{Question 223.2.4}
%\end{minipage}
%}}}%
%BeginExpansion
\marginpar {
\hspace{-0.5in}
\begin{minipage}[t]{1in}
\small{Question 223.2.4}
\end{minipage}
}%
%EndExpansion
%TCIMACRO{%
%\TeXButton{Question 223.2.5}{\marginpar {
%\hspace{-0.5in}
%\begin{minipage}[t]{1in}
%\small{Question 223.2.5}
%\end{minipage}
%}}}%
%BeginExpansion
\marginpar {
\hspace{-0.5in}
\begin{minipage}[t]{1in}
\small{Question 223.2.5}
\end{minipage}
}%
%EndExpansion

\begin{definition}
transverse wave: a traveling wave or pulse that causes the elements of the
disturbed medium to move perpendicular to the direction of propagation
\end{definition}

\begin{definition}
Longitudinal wave: a traveling wave or pulse that causes the elements of the
medium to move parallel to the direction of propagation
\end{definition}

\FRAME{dhF}{2.6705in}{1.7218in}{0pt}{}{}{Figure}{\special{language
"Scientific Word";type "GRAPHIC";maintain-aspect-ratio TRUE;display
"USEDEF";valid_file "T";width 2.6705in;height 1.7218in;depth
0pt;original-width 3.781in;original-height 2.4284in;cropleft "0";croptop
"1";cropright "1";cropbottom "0";tempfilename
'LTUWCH17.wmf';tempfile-properties "XPR";}}%
%TCIMACRO{%
%\TeXButton{Long Spring Demo}{\marginpar {
%\hspace{-0.5in}
%\begin{minipage}[t]{1in}
%\small{Long Spring Demo}
%\end{minipage}
%}}}%
%BeginExpansion
\marginpar {
\hspace{-0.5in}
\begin{minipage}[t]{1in}
\small{Long Spring Demo}
\end{minipage}
}%
%EndExpansion

\subsection{Examples of waves:}

\subsubsection{A pulse on a rope:}

\FRAME{dhF}{1.3837in}{1.8273in}{0in}{}{}{Figure}{\special{language
"Scientific Word";type "GRAPHIC";maintain-aspect-ratio TRUE;display
"USEDEF";valid_file "T";width 1.3837in;height 1.8273in;depth
0in;original-width 1.3482in;original-height 1.7902in;cropleft "0";croptop
"1";cropright "1";cropbottom "0";tempfilename
'LXC0KA02.wmf';tempfile-properties "XPR";}}

In the picture above, you see wave that has just one peak traveling to the
right. We call such a wave a \emph{pulse}. Notice how the piece of the rope
marked $P$ moves up and down, but the wave is moving to the right. This
pulse is a transverse wave because the parts of the medium (observe point $P$%
) move perpendicular to the direction the wave is moving.

\subsubsection{An ocean wave:}

Of course, some waves are a combination of these two basic types\footnote{%
You may have noticed that in Physics we tend to define basic types of
things, and then use these basic types to define more complex objects.}.
Water waves, for example, are transverse at the surface of the water, but
are longitudinal throughout the water.

\FRAME{dtbpF}{1.6622in}{0.7835in}{0pt}{}{}{Figure}{\special{language
"Scientific Word";type "GRAPHIC";maintain-aspect-ratio TRUE;display
"USEDEF";valid_file "T";width 1.6622in;height 0.7835in;depth
0pt;original-width 5.5901in;original-height 2.6212in;cropleft "0";croptop
"1";cropright "1";cropbottom "0";tempfilename
'LTUWCH19.wmf';tempfile-properties "XPR";}}

\subsubsection{Earthquake waves:}

Earthquakes produce both transverse and longitudinal waves. The two types of
waves even travel at different speeds! $P$ waves are longitudinal and travel
faster, $S$ waves are transverse and slower.

\section{Wave speed}

%TCIMACRO{%
%\TeXButton{Question 223.2.6}{\marginpar {
%\hspace{-0.5in}
%\begin{minipage}[t]{1in}
%\small{Question 223.2.6}
%\end{minipage}
%}}}%
%BeginExpansion
\marginpar {
\hspace{-0.5in}
\begin{minipage}[t]{1in}
\small{Question 223.2.6}
\end{minipage}
}%
%EndExpansion
%TCIMACRO{%
%\TeXButton{Question 223.2.7}{\marginpar {
%\hspace{-0.5in}
%\begin{minipage}[t]{1in}
%\small{Question 223.2.7}
%\end{minipage}
%}}}%
%BeginExpansion
\marginpar {
\hspace{-0.5in}
\begin{minipage}[t]{1in}
\small{Question 223.2.7}
\end{minipage}
}%
%EndExpansion
%TCIMACRO{%
%\TeXButton{Question 223.2.8}{\marginpar {
%\hspace{-0.5in}
%\begin{minipage}[t]{1in}
%\small{Question 223.2.8}
%\end{minipage}
%}}}%
%BeginExpansion
\marginpar {
\hspace{-0.5in}
\begin{minipage}[t]{1in}
\small{Question 223.2.8}
\end{minipage}
}%
%EndExpansion
We can perform an experiment with a rope or a long spring. Make a wave on
the rope or spring. Then pull the rope or spring tighter and make another
wave. We see that the wave on the tighter spring travels faster.

It is harder to do, but we can also experiment with two different ropes, one
light and one heavy. We would find that the heaver the rope, the slower the
wave. We can express this as 
\begin{equation*}
v=\sqrt{\frac{T_{s}}{\mu }}
\end{equation*}%
where $T_{s}$ is the tension in the rope, and $\mu $ is the linear mass
density%
\begin{equation*}
\mu =\frac{m}{L}
\end{equation*}%
where $m$ is the mass of the rope, and $L$ is the length.

The term $\mu $ might need an analogy to make it seem helpful. So suppose I
have an iron bar that has a mass of $200\unit{kg}$ and is $2\unit{m}$ long.
Further suppose I want to know how much mass there would be in a $20\unit{cm}
$ section cut of the end of the rod. How would I find out?

This is not very hard, We could say that there are $200\unit{kg}$ spread out
over $2\unit{m},$ so each meter of rod has $100\unit{kg}$ of mass, that is,
there is $100\unit{kg}/\unit{m}$ of mass per unit length. Then to find how
much mass there is in a $0.20\unit{m}$ section of the rod I\ take 
\begin{equation*}
m=100\frac{\unit{kg}}{\unit{m}}\times 0.20\unit{m}=20.0\unit{kg}
\end{equation*}
The $100\unit{kg}/\unit{m}$ is $\mu .$ It is how much mass there is in a
unit length segment of something In this example, it is a unit length of
iron bar, but for waves on string, we want the mass per unit length of
string.

If you are buying stock steel bar, you might be able to buy it with a mass
per unit length. If the mass per unit length is higher then the bar is more
massive. The same is true with string. The larger $\mu ,$ the more massive
equal string segments will be.

We should note that in forming this relationship, we have used our standard
introductory physics assumption that the mass of the rope can be neglected.
Let's consider what would happen if this were not true. Say we make a wave
in a heavy cable that is suspended. The mass at the lower end of the cable
pulls down on the upper part of the cable. The tension will actually change
along the length of the cable, and so will the wave speed. Such a situation
can't be represented by a single wave speed. But for our class, we will
assume that any such changes are small enough to be ignored.

\section{Example: Sound waves}

Sound is a wave. The medium is air particles. The transfer of energy is done
by collision. \FRAME{dhF}{1.0369in}{1.1744in}{0pt}{}{}{Figure}{\special%
{language "Scientific Word";type "GRAPHIC";maintain-aspect-ratio
TRUE;display "USEDEF";valid_file "T";width 1.0369in;height 1.1744in;depth
0pt;original-width 2.5581in;original-height 2.9006in;cropleft "0";croptop
"1";cropright "1";cropbottom "0";tempfilename
'LTUWCH1A.wmf';tempfile-properties "XPR";}}The wave will be a longitudinal
wave. Let's see how it forms. We can take a tube with a piston in it. \FRAME{%
dhF}{3.2707in}{0.7809in}{0pt}{}{}{Figure}{\special{language "Scientific
Word";type "GRAPHIC";maintain-aspect-ratio TRUE;display "USEDEF";valid_file
"T";width 3.2707in;height 0.7809in;depth 0pt;original-width
3.2258in;original-height 0.7498in;cropleft "0";croptop "1";cropright
"1";cropbottom "0";tempfilename 'LTUWCH1B.wmf';tempfile-properties "XPR";}}%
As we exert a force on the piston, the air molecules are compressed into a
group. In the next figure, each dot represents a group of air molecules. In
the top picture, the air molecules are not displaced. But when the piston
moves, the air molecules receive energy by collision. They bunch up. We see
this in the second picture. \FRAME{dhF}{2.9352in}{1.5082in}{0pt}{}{}{Figure}{%
\special{language "Scientific Word";type "GRAPHIC";maintain-aspect-ratio
TRUE;display "USEDEF";valid_file "T";width 2.9352in;height 1.5082in;depth
0pt;original-width 2.8911in;original-height 1.471in;cropleft "0";croptop
"1";cropright "1";cropbottom "0";tempfilename
'LTUWCH1C.wmf';tempfile-properties "XPR";}}The graph below the two pictures
shows how much displacement each molecule group experiences.

Suppose we now pull the piston back. This would allow the molecules to
bounce back to the left, but the molecules that they have collided with will
receive some energy and go to the right. This is shown in the next figure.
Color coded dots are displayed above the before and after picture so you can
see where the molecule groups started.\FRAME{dhF}{2.9334in}{1.5402in}{0pt}{}{%
}{Figure}{\special{language "Scientific Word";type
"GRAPHIC";maintain-aspect-ratio TRUE;display "USEDEF";valid_file "T";width
2.9334in;height 1.5402in;depth 0pt;original-width 3.6979in;original-height
1.9285in;cropleft "0";croptop "1";cropright "1";cropbottom "0";tempfilename
'LTUWCH1D.wmf';tempfile-properties "XPR";}}If we pull the piston back
further, the molecules can pass their original positions.\FRAME{dhF}{2.917in%
}{1.567in}{0pt}{}{}{Figure}{\special{language "Scientific Word";type
"GRAPHIC";maintain-aspect-ratio TRUE;display "USEDEF";valid_file "T";width
2.917in;height 1.567in;depth 0pt;original-width 3.2811in;original-height
1.7495in;cropleft "0";croptop "1";cropright "1";cropbottom "0";tempfilename
'LTUWCH1E.wmf';tempfile-properties "XPR";}}Then we can push inward again and
compress the gas.\FRAME{dhF}{2.8228in}{1.5791in}{0pt}{}{}{Figure}{\special%
{language "Scientific Word";type "GRAPHIC";maintain-aspect-ratio
TRUE;display "USEDEF";valid_file "T";width 2.8228in;height 1.5791in;depth
0pt;original-width 3.7671in;original-height 2.0954in;cropleft "0";croptop
"1";cropright "1";cropbottom "0";tempfilename
'LTUWCI1F.wmf';tempfile-properties "XPR";}}This may seem like a senseless
thing to do, but it is really what a speaker does to produce sound. In
particular, a speaker is a harmonic oscillator. The simple harmonic motion
of the speaker is the disturbance that makes the sound wave.\FRAME{dhF}{%
2.6135in}{2.1344in}{0pt}{}{}{Figure}{\special{language "Scientific
Word";type "GRAPHIC";maintain-aspect-ratio TRUE;display "USEDEF";valid_file
"T";width 2.6135in;height 2.1344in;depth 0pt;original-width
2.5719in;original-height 2.0954in;cropleft "0";croptop "1";cropright
"1";cropbottom "0";tempfilename 'LTUWCI1G.wmf';tempfile-properties "XPR";}}

\section{One dimensional waves}

%TCIMACRO{%
%\TeXButton{Question 223.2.9}{\marginpar {
%\hspace{-0.5in}
%\begin{minipage}[t]{1in}
%\small{Question 223.2.9}
%\end{minipage}
%}}}%
%BeginExpansion
\marginpar {
\hspace{-0.5in}
\begin{minipage}[t]{1in}
\small{Question 223.2.9}
\end{minipage}
}%
%EndExpansion

To mathematically describe a wave we will define a function of both time and
position.%
\begin{equation}
y\left( x,t\right)
\end{equation}

let's take a specific example\footnote{%
This is not an important wave function, just one I picked because it makes a
nice graphic example.}%
\begin{equation}
y=\frac{2}{\left( x-3.0t\right) ^{2}+1}
\end{equation}

Let's plot this for $t=0$

\FRAME{dtbpFX}{2.514in}{1.676in}{0pt}{}{}{Plot}{\special{language
"Scientific Word";type "MAPLEPLOT";width 2.514in;height 1.676in;depth
0pt;display "USEDEF";plot_snapshots TRUE;mustRecompute FALSE;lastEngine
"MuPAD";xmin "-50";xmax "50";xviewmin "-50";xviewmax "50";yviewmin
"0";yviewmax "2";viewset"XY";rangeset"X";plottype 4;axesFont "Times New
Roman,12,0000000000,useDefault,normal";numpoints 100;plotstyle
"patch";axesstyle "normal";axestips FALSE;xis \TEXUX{x};var1name
\TEXUX{$x$};function \TEXUX{$\frac{2}{\left( x\right) ^{2}+1}$};linecolor
"blue";linestyle 1;pointstyle "point";linethickness 3;lineAttributes
"Solid";var1range "-50,50";num-x-gridlines 100;curveColor
"[flat::RGB:0x000000ff]";curveStyle "Line";VCamFile
'LTUWDL29.xvz';valid_file "T";tempfilename
'LTUWCI1H.wmf';tempfile-properties "XPR";}}what will this look like for $%
t=10 $?

\FRAME{dtbpFX}{2.7709in}{1.8472in}{0pt}{}{}{Plot}{\special{language
"Scientific Word";type "MAPLEPLOT";width 2.7709in;height 1.8472in;depth
0pt;display "USEDEF";plot_snapshots TRUE;mustRecompute FALSE;lastEngine
"MuPAD";xmin "-50";xmax "50";xviewmin "-50";xviewmax "50";yviewmin
"0";yviewmax "2";viewset"XY";rangeset"X";plottype 4;axesFont "Times New
Roman,12,0000000000,useDefault,normal";numpoints 100;plotstyle
"patch";axesstyle "normal";axestips FALSE;xis \TEXUX{x};var1name
\TEXUX{$x$};function \TEXUX{$\frac{2}{\left( x-30\right) ^{2}+1}$};linecolor
"blue";linestyle 1;pointstyle "point";linethickness 3;lineAttributes
"Solid";var1range "-50,50";num-x-gridlines 100;curveColor
"[flat::RGB:0x000000ff]";curveStyle "Line";VCamFile
'LTUWDL28.xvz';valid_file "T";tempfilename
'LTUWCI1I.wmf';tempfile-properties "XPR";}}The pulse travels along the $x$%
-axis as a function of time. We denote the speed of the pulse as $v,$ then
we can define a function 
\begin{equation}
y\left( x,t\right) =y\left( x-vt,0\right)
\end{equation}%
that describes a pulse as it travels. An element of the medium (rope,
string, etc.) at position $x$ at some time $t,$ will have the displacement
that an element had earlier at $x-vt$ when $t=0.$

We will give $y\left( x-vt,0\right) $ a special name, the \emph{wave function%
}. It represents the $y$ position, the transverse position in our example,
of any element located at a position $x$ at any time $t$.

Notice that wave functions depend on two variables, $x,$ and $t.$ It is hard
to draw a wave so that this dual dependence is clear. Often we draw two
different graphs of the same wave so we can see independently the position
and time dependence. So far we have used one of these graphs. A graph of our
wave at a specific time, $t_{o}.$ This gives $y\left( x,t_{o}\right) $. This
representation of a wave is very like a photograph of the wave taken with a
digital camera. It gives a picture of the entire wave, but only for one
time, the time at which the photograph was taken. Of course we could take a
series of photographs, but still each would be a picture of the wave at just
one time.\FRAME{dtbpF}{3.1213in}{2.1172in}{0pt}{}{}{Figure}{\special%
{language "Scientific Word";type "GRAPHIC";maintain-aspect-ratio
TRUE;display "USEDEF";valid_file "T";width 3.1213in;height 2.1172in;depth
0pt;original-width 3.814in;original-height 2.5776in;cropleft "0";croptop
"1";cropright "1";cropbottom "0";tempfilename
'MTAANL05.wmf';tempfile-properties "XPR";}}

%TCIMACRO{%
%\TeXButton{Question 223.2.10}{\marginpar {
%\hspace{-0.5in}
%\begin{minipage}[t]{1in}
%\small{Question 223.2.10}
%\end{minipage}
%}}}%
%BeginExpansion
\marginpar {
\hspace{-0.5in}
\begin{minipage}[t]{1in}
\small{Question 223.2.10}
\end{minipage}
}%
%EndExpansion
The second representation is to observe the wave at just one point in the
medium, but for many times. This is very like taking a video camera and
using it to record the displacement of just one part of the medium for many
times. You could envision marking just one part of a rope, and then using
the video recorder to make a movie of the motion of that single part of the
rope. We could then go frame by frame through the video, and plot the
displacement of our marked part of the rope as a function of time. Such a
graph is sometimes called a history graph of the wave.\FRAME{dtbpF}{3.5812in%
}{2.6775in}{0pt}{}{}{Figure}{\special{language "Scientific Word";type
"GRAPHIC";maintain-aspect-ratio TRUE;display "USEDEF";valid_file "T";width
3.5812in;height 2.6775in;depth 0pt;original-width 3.9704in;original-height
2.9603in;cropleft "0";croptop "1";cropright "1";cropbottom "0";tempfilename
'S0UJ0O01.wmf';tempfile-properties "XPR";}}

\chapter{Waves in One and More Dimensions}

We studied waves in general last lecture. This time we will look at a
specific wave, the sinusoidal wave. You might think this is terribly
restrictive, but we will find that using sinusoidal waves, we can represent
most any wave through an elegant mathematical trick, and the idea of
superposition (that we will explain later).

%TCIMACRO{%
%\TeXButton{Fundamental Concepts}{\hspace{-1.3in}{\LARGE Fundamental Concepts\vspace{0.25in}}}}%
%BeginExpansion
\hspace{-1.3in}{\LARGE Fundamental Concepts\vspace{0.25in}}%
%EndExpansion

\begin{enumerate}
\item The mathematical form of a sinusoidal wave is $y\left( x,t\right)
=y_{\max }\cos \left( kx-\omega t+\phi _{o}\right) $

\item There are names for parts of a sinusoidal wave. We need to recognize
the following terms: crest, trough, wavelength period frequency angular
frequency, phase constant, wave number.

\item Spatial frequency is \textquotedblleft how often\textquotedblright\
something happens along some length.

\item The phase of a sinusoidal wave is given by $\phi =kx-\omega t+\phi
_{o} $

\item Spherical waves have the form $y=A\left( r\right) \sin \left(
kx-\omega t+\phi _{o}\right) $

\item Sufficiently far from the source of a wave, we can treat spherical
waves like plane waves.
\end{enumerate}

\section{Sinusoidal Waves}

A sinusoidal graph should be familiar from our PH121 or Dynamics experience.
We can use what we know from oscillation to understand the equation for a
sinusoidal wave. Remember that for simple harmonic oscillators we used the
function%
\begin{equation}
y\left( t\right) =A\cos \left( \omega t+\phi _{o}\right) \qquad SHM
\end{equation}%
but this only gave us a vertical displacement at one $x$-position. Now our
sinusoidal function must also be a function of position along the wave.%
\begin{equation}
y\left( x,t\right) =A\cos \left( kx-\omega t+\phi _{o}\right) \qquad waves
\end{equation}%
But before we study the nature of this function, lets see what we can learn
from the graph of a sinusoidal wave. We will need both of our two views, the
camera snapshot and the video (history) of a point. Look at figure \ref{sine
wave}. \FRAME{fhF}{2.4215in}{1.9865in}{0pt}{}{\Qlb{sine wave}}{Figure}{%
\special{language "Scientific Word";type "GRAPHIC";maintain-aspect-ratio
TRUE;display "USEDEF";valid_file "T";width 2.4215in;height 1.9865in;depth
0pt;original-width 3.736in;original-height 3.0571in;cropleft "0";croptop
"1";cropright "1";cropbottom "0";tempfilename
'S0WDKW01.wmf';tempfile-properties "XPR";}}This is two camera snap shots
superimposed. The red curve shows the wave ($y$ position for each value of $%
x)$ at $t=0.$ At some some later time $t,$ the wave pattern has moved to the
right as shown by the blue curve. The shift is by an amount $x=vt.$ This
reminds us that we can write a wave function in the form%
\begin{equation*}
y\left( x-vt,0\right)
\end{equation*}

\subsection{Parts of a wave}

%TCIMACRO{%
%\TeXButton{Question 123.3.1}{\marginpar {
%\hspace{-0.5in}
%\begin{minipage}[t]{1in}
%\small{Question 123.3.1}
%\end{minipage}
%}}}%
%BeginExpansion
\marginpar {
\hspace{-0.5in}
\begin{minipage}[t]{1in}
\small{Question 123.3.1}
\end{minipage}
}%
%EndExpansion
%TCIMACRO{%
%\TeXButton{Question 123.3.2}{\marginpar {
%\hspace{-0.5in}
%\begin{minipage}[t]{1in}
%\small{Question 123.3.2}
%\end{minipage}
%}}}%
%BeginExpansion
\marginpar {
\hspace{-0.5in}
\begin{minipage}[t]{1in}
\small{Question 123.3.2}
\end{minipage}
}%
%EndExpansion

The peaks in the wave are called crests. For a sine wave we have a series of
crests. We define the wavelength as the distance between any two nearest
identical points (e.g. crests) on the wave.

Notice that this is very similar to the definition of the period, $T,$ when
we graphed SHM on a $y$ vs. $t$ set of axis. In fact, this similarity is
even more apparent if we plot a sinusoidal wave using our two wave pictures.
In the next figure, the snap-shot comes first. We can see that there will be
crests. The distance between the crests is given the name \emph{wavelength}.
We give it the symbol $\lambda .$ This is not the entire length of the whole
wave. But is is a characteristic length of part of the wave that is easy to
identify. The next figure shows all this using our snapshot and history
graphs for a sine wave. \FRAME{dtbpF}{2.4319in}{4.3344in}{0pt}{}{\Qlb{x,y,t
graphs}}{Figure}{\special{language "Scientific Word";type
"GRAPHIC";maintain-aspect-ratio TRUE;display "USEDEF";valid_file "T";width
2.4319in;height 4.3344in;depth 0pt;original-width 4.5143in;original-height
8.0851in;cropleft "0";croptop "1";cropright "1";cropbottom "0";tempfilename
'LTUWCI1L.wmf';tempfile-properties "XPR";}}Note that there are crests in the
history graph view as well. That is because one marked part of the medium is
being displaced as a function of time (think of our marked piece of the rope
going up and down, or think of floating in the ocean at one point, you
travel up and down as the waves go by). But now the horizontal axis is time.
There will be a characteristic time between crests. That time is called the 
\emph{period}. Like the wavelength is not the length of the whole wave, the
period is not the time the whole wave exists. It is just the time it takes
the part of the medium we are watching to go through one complete cycle.
Notice that this video picture is exactly the same as a plot of the motion
of a simple harmonic oscillator! For a sinusoidal wave, each part of the
medium experiences simple harmonic motion.

We remember frequency from simple harmonic motion. But now we have a wave,
and the wave is moving. We can extend our view of frequency by defining it
as follows:

\begin{definition}
The frequency of a periodic wave is the number of crests (or any other point
of the wave) that pass a given point in a unit time interval.
\end{definition}

\FRAME{fhFX}{2.6109in}{1.7417in}{0pt}{}{\Qlb{frequency}}{Plot}{\special%
{language "Scientific Word";type "MAPLEPLOT";width 2.6109in;height
1.7417in;depth 0pt;display "USEDEF";plot_snapshots TRUE;mustRecompute
FALSE;lastEngine "MuPAD";xmin "0";xmax "10";xviewmin "0";xviewmax
"10";yviewmin "-1.000187";yviewmax
"1.000187";viewset"XY";rangeset"X";plottype 4;axesFont "Times New
Roman,12,0000000000,useDefault,normal";numpoints 100;plotstyle
"patch";axesstyle "normal";axestips FALSE;xis \TEXUX{v58130};var1name
\TEXUX{$\theta $};function \TEXUX{$\sin \left( \theta \right) $};linecolor
"blue";linestyle 1;pointstyle "point";linethickness 1;lineAttributes
"Solid";var1range "0,10";num-x-gridlines 100;curveColor
"[flat::RGB:0x000000ff]";curveStyle "Line";function \TEXUX{$\sin \left(
2\theta \right) $};linecolor "red";linestyle 1;pointstyle
"point";linethickness 1;lineAttributes "Solid";var1range
"0,10";num-x-gridlines 100;curveColor "[flat::RGB:0x00ff0000]";curveStyle
"Line";VCamFile 'LTUWDL27.xvz';valid_file "T";tempfilename
'LTUWCI1M.wmf';tempfile-properties "XPR";}}In figure \ref{frequency}, the
blue curve has twice the frequency as the red curve. Notice how it has two
crests for every red crest. The maximum displacement of the wave is called
the \emph{amplitude }just as it was for simple harmonic oscillators.

\subsubsection{Wavenumber and wave speed}

%TCIMACRO{%
%\TeXButton{Question 123.3.3}{\marginpar {
%\hspace{-0.5in}
%\begin{minipage}[t]{1in}
%\small{Question 123.3.3}
%\end{minipage}
%}}}%
%BeginExpansion
\marginpar {
\hspace{-0.5in}
\begin{minipage}[t]{1in}
\small{Question 123.3.3}
\end{minipage}
}%
%EndExpansion
%TCIMACRO{%
%\TeXButton{Question 123.3.4}{\marginpar {
%\hspace{-0.5in}
%\begin{minipage}[t]{1in}
%\small{Question 123.3.4}
%\end{minipage}
%}}}%
%BeginExpansion
\marginpar {
\hspace{-0.5in}
\begin{minipage}[t]{1in}
\small{Question 123.3.4}
\end{minipage}
}%
%EndExpansion
%TCIMACRO{%
%\TeXButton{Question 123.3.5}{\marginpar {
%\hspace{-0.5in}
%\begin{minipage}[t]{1in}
%\small{Question 123.3.5}
%\end{minipage}
%}}}%
%BeginExpansion
\marginpar {
\hspace{-0.5in}
\begin{minipage}[t]{1in}
\small{Question 123.3.5}
\end{minipage}
}%
%EndExpansion
Consider again a sinusoidal wave. 
\begin{equation}
y\left( x,t\right) =A\cos \left( kx-\omega t+\phi _{o}\right)
\end{equation}%
We have drawn the wave in the snapshot picture mode\FRAME{dtbpFX}{2.6109in}{%
1.7417in}{0pt}{}{}{Plot}{\special{language "Scientific Word";type
"MAPLEPLOT";width 2.6109in;height 1.7417in;depth 0pt;display
"USEDEF";plot_snapshots TRUE;mustRecompute FALSE;lastEngine "MuPAD";xmin
"0";xmax "10";xviewmin "0";xviewmax "10";yviewmin "-1.000187";yviewmax
"1.000187";viewset"XY";rangeset"X";plottype 4;axesFont "Times New
Roman,12,0000000000,useDefault,normal";numpoints 100;plotstyle
"patch";axesstyle "normal";axestips FALSE;xis \TEXUX{v58130};var1name
\TEXUX{$\theta $};function \TEXUX{$\sin \left( \theta \right) $};linecolor
"blue";linestyle 1;pointstyle "point";linethickness 1;lineAttributes
"Solid";var1range "0,10";num-x-gridlines 100;curveColor
"[flat::RGB:0x000000ff]";curveStyle "Line";VCamFile
'LTUWDL26.xvz';valid_file "T";tempfilename
'MLRJPN00.wmf';tempfile-properties "XPR";}}To make this graph, we set $t=0$
and plot the resulting function%
\begin{equation}
y\left( x,0\right) =A\sin \left( kx+0\right)
\end{equation}%
$A$ is the amplitude. I want to investigate the meaning of the constant $k.$
Lets find $k$ like we did for SHM when we found $\omega $. Consider the
point $x=0$. At this point 
\begin{equation}
y\left( 0,0\right) =A\sin \left( k\left( 0\right) \right) =0
\end{equation}%
The next time $y=0$ is when $x=\frac{\lambda }{2}$ \FRAME{dtbpF}{2.8781in}{%
2.2373in}{0pt}{}{}{Figure}{\special{language "Scientific Word";type
"GRAPHIC";maintain-aspect-ratio TRUE;display "USEDEF";valid_file "T";width
2.8781in;height 2.2373in;depth 0pt;original-width 4.4858in;original-height
3.4809in;cropleft "0";croptop "1";cropright "1";cropbottom "0";tempfilename
'LTUWCI1O.wmf';tempfile-properties "XPR";}}then%
\begin{equation}
y\left( \frac{\lambda }{2},0\right) =A\sin \left( k\frac{\lambda }{2}\right)
=0
\end{equation}%
From our trigonometry experience, we know that this is true when 
\begin{equation}
k\frac{\lambda }{2}=\pi
\end{equation}%
solving for $k$ gives%
\begin{equation}
k=\frac{2\pi }{\lambda }
\end{equation}%
Then we now have a feeling for what $k$ means. It is $2\pi $ over the
spacing between the crests. The $2\pi $ must have units of radians attached.
Then%
\begin{equation}
y\left( x,0\right) =A\sin \left( \frac{2\pi }{\lambda }x+0\right)
\end{equation}

We have a special name for the quantity $k.$ It is called the \emph{wave
number}$.$ 
\begin{equation}
k\equiv \frac{2\pi }{\lambda }
\end{equation}

Both the name and the symbol are somewhat unfortunate. Neither gives much
insight into the meaning of this quantity. But from what we have done, we
can understand it better. For a harmonic oscillator, we know that 
\begin{equation*}
y\left( t\right) =A\sin \left( \omega t\right)
\end{equation*}%
where 
\begin{equation*}
\omega =2\pi f=\frac{2\pi }{T}
\end{equation*}%
%TCIMACRO{%
%\TeXButton{Question 123.3.6}{\marginpar {
%\hspace{-0.5in}
%\begin{minipage}[t]{1in}
%\small{Question 123.3.6}
%\end{minipage}
%}}}%
%BeginExpansion
\marginpar {
\hspace{-0.5in}
\begin{minipage}[t]{1in}
\small{Question 123.3.6}
\end{minipage}
}%
%EndExpansion
$T$ is how far, in time, the crests are apart, and the inverse of this, $%
\frac{1}{T}$ is the frequency. The frequency tells us how often we encounter
a crest as we march along in time. So $\frac{1}{T}$ must be how many crests
we have in a unit amount of time.

Now think of the relationship between the snapshot and the video
representation for a sinusoidal wave. We have a new quantity 
\begin{equation*}
k=\frac{2\pi }{\lambda }
\end{equation*}%
where $\lambda $ is how far, in distance, the crests are apart. This implies
that $\frac{1}{\lambda }$ plays the same role in the snap shot graph that $f$
plays in the video graph. It must tell us how many crests we have, but this
time it is how many crests in a unit of distance. We found above that $k$
told us something about how often the zeros (well, every other zero) will
occur. But the crests must occur at the same rate. So $k$ tells us how often
we encounter a crest in our snapshot graph.

The frequency is how often we encounter a crest in the video graph, $\frac{1%
}{\lambda }$ is how often we encounter a crest in the snap shot graph. Thus $%
\frac{1}{\lambda }$ is playing the same role for a snap shot graph as
frequency plays for a history graph. We could call $\frac{1}{\lambda }$a 
\emph{spatial frequency. }It is how often we encounter a crest as we march
along in position, or how many crests we have in a unit amount of distance.
And $\lambda $ could be called a \emph{spatial period}. Both $1/T$ and $%
1/\lambda $ answer the question \textquotedblleft how often something
happens in a unit of something\textquotedblright\ but one asks the question
in time and the other in position along the wave.

My mental image for this is the set of groves on the edge of a highway.
There is a distance between them, like a wavelength, and how often I
encounter one as I move a distance along the road is the spatial frequency.
If you are a farmer, you may think of plowed fields, with a distance between
furrows as a wavelength, and how closely spaced the furrows are as a measure
of spatial frequency. We use this concept in optics to test how well an
optical system resolves details in a photograph. The next figure is a test
image. A good camera will resolve all spatial frequencies equally well.
Notice the test image has sets of bars with different spatial frequencies.
By forming an image of this pattern, you can see which spacial frequencies
are faithfully represented by the optical system.\FRAME{dtbpFU}{2.6974in}{%
2.1819in}{0pt}{\Qcb{Resolution test target based on the USAF\ 1951
Resolution Test Pattern (not drawn to exact specifications).}}{}{Figure}{%
\special{language "Scientific Word";type "GRAPHIC";maintain-aspect-ratio
TRUE;display "USEDEF";valid_file "T";width 2.6974in;height 2.1819in;depth
0pt;original-width 2.655in;original-height 2.143in;cropleft "0";croptop
"1";cropright "1";cropbottom "0";tempfilename
'MFF3L701.wmf';tempfile-properties "XPR";}}In class you will see that our
projector does not represent all spatial frequencies equally well! You can
also see this now in the copy you are reading. If you are reading on-line or
an electronic copy, your screen resolution will limit the representation of
some spatial frequencies. Look for the smallest set of three bars where you
can still tell for sure that there are three bars. A printed version that
has been printed on a laser printer will usually allow you to see even
smaller sets of three bars clearly.

Let's place $k$ in the full equation for the sine wave for any time, $t$.%
\begin{equation}
y\left( t\right) =A\cos \left( kx-\omega t+\phi _{o}\right)
\end{equation}%
We would like this to look like our wave function equation 
\begin{equation*}
y\left( x,t\right) =y\left( x-vt,0\right)
\end{equation*}%
With a little algebra we can do this%
\begin{eqnarray*}
y\left( t\right) &=&A\cos \left( kx-\omega t+\phi _{o}\right) \\
&=&A\cos \left( \frac{2\pi }{\lambda }x-\frac{2\pi }{T}t+\phi _{o}\right) \\
&=&A\cos \left( \frac{2\pi }{\lambda }\left( x-\frac{\lambda }{T}t\right)
+\phi _{o}\right)
\end{eqnarray*}%
This is in the form of a wave function so long as%
\begin{equation}
v=\frac{\lambda }{T}
\end{equation}%
then 
\begin{equation}
y\left( x,t\right) =A\sin \left( \frac{2\pi }{\lambda }\left( x-vt\right)
+\phi _{o}\right)
\end{equation}%
We can see that the wave travels one wavelength in one period. The simple
relationship 
\begin{equation}
v=\frac{\lambda }{T}
\end{equation}%
is of tremendous importance.

\subsubsection{Wave speed forms}

We found%
\begin{equation}
v=\frac{\lambda }{T}
\end{equation}%
but it is easy to see that 
\begin{equation}
v=\frac{2\pi \lambda }{2\pi T}=\frac{\omega }{k}
\end{equation}%
and 
\begin{equation}
v=\lambda f
\end{equation}%
This last formula is, perhaps, the most common form encountered in our study
of light.

\subsubsection{Phase}

You may be wondering about the phase constant we learned about in our study
of SHM. We have ignored it up to now. But of course we can shift our sine
just like we did for our plots of position vs. time for oscillation. Only
now with a wave we have two graphs, a history and snapshot graphs, so we
could shift along the $x$ in a snapshot graph or along the $t$ axes in a
history graph. So the sine wave has the form.%
\begin{equation}
y\left( x,t\right) =A\sin \left( kx-\omega t+\phi _{o}\right)
\end{equation}%
were $\phi _{o}$ will need to be determined by initial conditions just like
in SHM problems and those initial conditions will include initial positions
as well as initial times.

Let's consider that we have two views of a wave, the snapshot and history
view. Each of these looks like sinusoids for a sinusoidal wave. Let's
consider a specific wave, 
\begin{equation*}
y\left( x,t\right) =5\sin \left( 3\pi x-\frac{\pi }{5}t+\frac{\pi }{2}\right)
\end{equation*}%
And let's look at a snapshot graph at $t=0$ 
\begin{equation*}
y\left( x,0\right) =5\sin \left( 3\pi x-\frac{\pi }{5}\left( 0\right) +\frac{%
\pi }{2}\right)
\end{equation*}%
\FRAME{dtbpFX}{4.4996in}{0.7706in}{0pt}{}{}{Plot}{\special{language
"Scientific Word";type "MAPLEPLOT";width 4.4996in;height 0.7706in;depth
0pt;display "USEDEF";plot_snapshots TRUE;mustRecompute FALSE;lastEngine
"MuPAD";xmin "-3";xmax "3";xviewmin "-3";xviewmax "3";yviewmin
"-5.000998";yviewmax "5";viewset"XY";rangeset"X";plottype 4;axesFont "Times
New Roman,12,0000000000,useDefault,normal";numpoints 100;plotstyle
"patch";axesstyle "normal";axestips FALSE;xis \TEXUX{x};var1name
\TEXUX{$x$};function \TEXUX{$5\sin \left( 3\pi x-0+\frac{\pi }{2}\right)
$};linecolor "blue";linestyle 1;pointstyle "point";linethickness
3;lineAttributes "Solid";var1range "-3,3";num-x-gridlines 100;curveColor
"[flat::RGB:0x000000ff]";curveStyle "Line";VCamFile
'LXC9P00U.xvz';valid_file "T";tempfilename
'LXC9O803.wmf';tempfile-properties "XPR";}}and another at $t=2\unit{s}$%
\begin{equation*}
y\left( x,2\unit{s}\right) =5\sin \left( 3\pi x-\frac{\pi }{5}\left( 2\unit{s%
}\right) +\frac{\pi }{2}\right)
\end{equation*}%
\FRAME{dtbpFX}{4.4996in}{0.7706in}{0pt}{}{}{Plot}{\special{language
"Scientific Word";type "MAPLEPLOT";width 4.4996in;height 0.7706in;depth
0pt;display "USEDEF";plot_snapshots TRUE;mustRecompute FALSE;lastEngine
"MuPAD";xmin "-3";xmax "3";xviewmin "-3";xviewmax "3";yviewmin
"-5.000998";yviewmax "5";viewset"XY";rangeset"X";plottype 4;axesFont "Times
New Roman,12,0000000000,useDefault,normal";numpoints 100;plotstyle
"patch";axesstyle "normal";axestips FALSE;xis \TEXUX{x};var1name
\TEXUX{$x$};function \TEXUX{$5\sin \left( 3\pi x-\frac{\pi }{5}\left(
2\right) +\frac{\pi }{2}\right) $};linecolor "blue";linestyle 1;pointstyle
"point";linethickness 3;lineAttributes "Solid";var1range
"-3,3";num-x-gridlines 100;curveColor "[flat::RGB:0x000000ff]";curveStyle
"Line";VCamFile 'LXC9TY0Y.xvz';valid_file "T";tempfilename
'LXC9Q804.wmf';tempfile-properties "XPR";}}Comparing the two, we could view
the latter as having a different phase constant%
\begin{equation*}
\phi _{total}=\omega \Delta t+\phi _{o}=-\frac{\pi }{5}\left( 2\unit{s}%
\right) +\frac{\pi }{2}
\end{equation*}%
that is, within the snapshot view, the time dependent part of the argument
of the sine acts like an additional phase constant.

Likewise, in the history view, we can plot our wave at $x=0$%
\begin{equation*}
y\left( 0,t\right) =5\sin \left( 3\pi \left( 0\right) -\frac{\pi }{5}t+\frac{%
\pi }{2}\right)
\end{equation*}%
\FRAME{dtbpFX}{4.4996in}{0.7706in}{0pt}{}{}{Plot}{\special{language
"Scientific Word";type "MAPLEPLOT";width 4.4996in;height 0.7706in;depth
0pt;display "USEDEF";plot_snapshots TRUE;mustRecompute FALSE;lastEngine
"MuPAD";xmin "-8";xmax "8";xviewmin "-8";xviewmax "8";yviewmin
"-5.000998";yviewmax "5";viewset"XY";rangeset"X";plottype 4;labeloverrides
1;x-label "t";axesFont "Times New
Roman,12,0000000000,useDefault,normal";numpoints 100;plotstyle
"patch";axesstyle "normal";axestips FALSE;xis \TEXUX{t};var1name
\TEXUX{$t$};function \TEXUX{$5\sin \left( 3\pi \left( 0\right) -\frac{\pi
}{5}t+\frac{\pi }{2}\right) $};linecolor "blue";linestyle 1;pointstyle
"point";linethickness 3;lineAttributes "Solid";var1range
"-8,8";num-x-gridlines 100;curveColor "[flat::RGB:0x000000ff]";curveStyle
"Line";VCamFile 'LXC9ZC15.xvz';valid_file "T";tempfilename
'LXC9ZC07.wmf';tempfile-properties "XPR";}}and at $x=1.5\unit{m}$%
\begin{equation*}
y\left( 1.5\unit{m},t\right) =5\sin \left( 3\pi \left( 1.5\unit{m}\right) -%
\frac{\pi }{5}t+\frac{\pi }{2}\right)
\end{equation*}%
\FRAME{dtbpFX}{4.4996in}{0.7706in}{0pt}{}{}{Plot}{\special{language
"Scientific Word";type "MAPLEPLOT";width 4.4996in;height 0.7706in;depth
0pt;display "USEDEF";plot_snapshots TRUE;mustRecompute FALSE;lastEngine
"MuPAD";xmin "-8";xmax "8";xviewmin "-8";xviewmax "8";yviewmin
"-5.000998";yviewmax "5";viewset"XY";rangeset"X";plottype 4;labeloverrides
1;x-label "t";axesFont "Times New
Roman,12,0000000000,useDefault,normal";numpoints 100;plotstyle
"patch";axesstyle "normal";axestips FALSE;xis \TEXUX{t};var1name
\TEXUX{$t$};function \TEXUX{$5\sin \left( 3\pi \left( 1.5\right) -\frac{\pi
}{5}t+\frac{\pi }{2}\right) $};linecolor "blue";linestyle 1;pointstyle
"point";linethickness 3;lineAttributes "Solid";var1range
"-8,8";num-x-gridlines 100;curveColor "[flat::RGB:0x000000ff]";curveStyle
"Line";VCamFile 'LXC9ZK16.xvz';valid_file "T";tempfilename
'LXC9YF06.wmf';tempfile-properties "XPR";}}Within the history view, the $kx$
part of the argument acts like a phase constant.%
\begin{equation*}
\phi _{total}=k\Delta x+\phi _{o}=3\pi \left( 1.5\unit{m}\right) +\frac{\pi 
}{2}
\end{equation*}

Of course neither $kx$ nor $\omega t$ are constant, But within individual
views of the wave we have set them as constant to form our snapshot and
history representations. We can see that any part of the argument of the
sine, $kx-\omega t+\phi _{o}$ could contribute to a phase shift, depending
on the view we are taking.

Because of this, it is customary to call the entire argument of the sine
function, $\phi =kx-\omega t+\phi _{o}$ the \emph{phase of the wave}. Where $%
\phi _{o}$ is the phase constant, $\phi $ is the phase. Of course then, $%
\phi $ must be a function of $x$ and $t,$ so we have a different value for $%
\phi \left( x,t\right) $ for every point on the wave for every time.

\subsection{Sinusoidal waves on strings}

Take a jump rope, and shake one end up and down while your partner keeps his
or her end stationary. You can make a sine wave in the rope. You can do a
better job by attaching a wave generator to the end.

Really, as long at the wave forms are identical and periodic, the
relationships 
\begin{equation}
f=\frac{1}{T}
\end{equation}%
and%
\begin{equation}
v=f\lambda
\end{equation}%
will hold. But we will make our device vibrate with simple harmonic motion.

Let's call an element of the rope $\Delta x.$ Here the \textquotedblleft $%
\Delta $\textquotedblright\ is being used to mean \textquotedblleft a small
amount of.\textquotedblright\ We are taking a small amount of the rope and
calling it's length $\Delta x.$

Each element of the rope $\left( \Delta x_{i}\right) $ will also oscillate
with SHM (think of a driven SHO). Note that the elements of the rope
oscillate in the $y$ direction, but the wave travels in $x.$ This is a
transverse wave.

Let's describe the motion of an element of the string at point $P.$

At $t=0,$%
\begin{equation}
y=A\sin \left( kx-\omega t\right)
\end{equation}%
(where I\ have chosen $\phi _{o}=0$ for this example). The element does not
move in the $x$ direction. So we define the \emph{transverse} \emph{speed}, $%
v_{y},$ and the \emph{transverse acceleration, }$a_{y},$ as the velocity and
acceleration of the element of rope in the $y$ direction. These are not the
velocity and acceleration of the wave, just the velocity and acceleration of
the element $\Delta x$ at a point $P.$

Because we are doing this at one specific $x$ location we need partial
derivatives to find the velocity

\begin{equation}
v_{y}=\left. \frac{dy}{dt}\right] _{x=\text{constant}}=\frac{\partial y}{%
\partial t}
\end{equation}%
That is, we take the derivative of $y$ with respect to $t,$ but we pretend
that $x$ is not a variable because we just want one $x$ position. Then%
\begin{equation}
v_{y}=\frac{\partial y}{\partial t}=-\omega A\cos \left( kx-\omega t\right)
\end{equation}%
and%
\begin{equation}
a_{y}=\frac{\partial v_{y}}{\partial t}=-\omega ^{2}A\sin \left( kx-\omega
t\right)
\end{equation}

These solutions should look very familiar! We expect them to be the same as
a harmonic oscillator except that we now have to specify which
oscillator--which part of the rope--we are looking at. That is what the $kx$
part is doing.

\section{The speed of Waves on Strings}

%TCIMACRO{%
%\TeXButton{Only use if there are questions on this.}{\marginpar {
%\hspace{-0.5in}
%\begin{minipage}[t]{1in}
%\small{Only use if there are questions on this.}
%\end{minipage}
%}}}%
%BeginExpansion
\marginpar {
\hspace{-0.5in}
\begin{minipage}[t]{1in}
\small{Only use if there are questions on this.}
\end{minipage}
}%
%EndExpansion

Let's work a problem together. Let's find an expression for the speed of the
wave as it travels along a string.\FRAME{fhF}{3.2422in}{2.4362in}{0pt}{}{}{%
Figure}{\special{language "Scientific Word";type
"GRAPHIC";maintain-aspect-ratio TRUE;display "USEDEF";valid_file "T";width
3.2422in;height 2.4362in;depth 0pt;original-width 5.0004in;original-height
3.7498in;cropleft "0";croptop "1";cropright "1";cropbottom "0";tempfilename
'LTUWCI1Q.wmf';tempfile-properties "XPR";}}

We will use Newton's second law 
\begin{equation*}
\Sigma \overrightarrow{F}=\overrightarrow{F}_{net}=m\overrightarrow{a}
\end{equation*}%
to do this, so we need a sum of the forces. What are the forces acting on an
element of string?

\begin{itemize}
\item Tension on the right hand side (RHS) of the element from the rest of
the string on the right, $T_{r}$

\item Tension on the left hand side (LHS) of the element from the rest of
the string on the left, $T_{l}$

\item The force due to gravity on our element of string, $F_{g}$
\end{itemize}

Lets assume that the element of string, $\Delta s,$ at the crest is
approximately an arc of a circle with radius $R$.

There is a force pulling left on the left end of the element that is tangent
to the arc, there is a force pulling right at the right end of the element
which is also tangent to the arc. The horizontal components of the forces
cancel $\left( T\cos \theta \right) .$ The vertical component, $\left( T\sin
\left( \theta \right) \right) $ is directed toward the center of the arc.
Then these forces must be a mass times an acceleration and because they are
center seeking we can call these accelerations centripetal accelerations%
\begin{equation}
a=\frac{v^{2}}{R}
\end{equation}%
If the rope is not moving in the $x$ direction, then 
\begin{equation*}
\Sigma F_{x}=0=-T_{l}\cos \theta +T_{r}\cos \theta
\end{equation*}%
\begin{equation*}
T_{l}=T_{r}
\end{equation*}%
Then, the radial force $F_{r}$ will have matching components from each side
of the element that together are $2T\sin \left( \theta \right) .$ Since the
element is small,%
\begin{equation}
\Sigma F_{r}=2T\sin \left( \theta \right) \approx 2T\theta
\end{equation}

The element has a mass $m.$ 
\begin{equation}
m=\mu \Delta s
\end{equation}%
where $\mu $ is the mas per unit length. Using the arc length formula%
\begin{equation}
\Delta s=R\left( 2\theta \right)
\end{equation}%
so%
\begin{equation}
m=\mu \Delta s=2\mu R\theta
\end{equation}%
and finally we use the formula for the radial acceleration%
\begin{equation}
F_{r}=ma=\left( 2\mu R\theta \right) \frac{v^{2}}{R}
\end{equation}%
Combining these two expressions for $F_{r}$%
\begin{equation}
2T\theta =\left( 2\mu R\theta \right) \frac{v^{2}}{R}
\end{equation}%
\begin{equation}
T=\left( \mu R\right) \frac{v^{2}}{R}
\end{equation}%
\begin{equation}
\frac{T}{\mu }=v^{2}
\end{equation}%
and we find that 
\begin{equation}
v=\sqrt{\frac{T}{\mu }}
\end{equation}

Note that we made many assumptions along the way. Despite this, the
approximation is quite good.

\section{Waves in two and three dimensions}

So far we have written expressions for waves, but our experience tells us
that waves don't usually come as one dimensional phenomena. In the next
figure, we see the disturbance (a drop) creating a water wave.\FRAME{dtbpFU}{%
2.3713in}{1.5394in}{0pt}{\Qcb{Picture of a water drop (Jon Paul Johnson,
used by permision)}}{}{Figure}{\special{language "Scientific Word";type
"GRAPHIC";maintain-aspect-ratio TRUE;display "USEDEF";valid_file "T";width
2.3713in;height 1.5394in;depth 0pt;original-width 2.3315in;original-height
1.5031in;cropleft "0";croptop "1";cropright "1";cropbottom "0";tempfilename
'MG4N9200.wmf';tempfile-properties "XPR";}} The wave is clearly not one
dimensional. It appears nearly circular. In fact, it is closer to
hemispherical, and this limit is only true because the disturbance is at the
air-water boundary. Most waves in a uniform medium will be roughly
spherical. \FRAME{dtbpF}{2.1006in}{1.9649in}{0pt}{}{}{Figure}{\special%
{language "Scientific Word";type "GRAPHIC";maintain-aspect-ratio
TRUE;display "USEDEF";valid_file "T";width 2.1006in;height 1.9649in;depth
0pt;original-width 4.2078in;original-height 3.9355in;cropleft "0";croptop
"1";cropright "1";cropbottom "0";tempfilename
'Sound_Waves/spherical_wave_1.wmf';tempfile-properties "XNPR";}}As such a
wave travels away from the source, the energy traveling gets more spread
out. This causes the amplitude to decrease. Think of a sound wave, it gets
quieter the farther you are from the source. We change our equation to
account for this by making the amplitude a function of the distance, $r,$
from the source%
\begin{equation}
y=A\left( r\right) \sin \left( \overrightarrow{\mathbf{k}}\cdot 
\overrightarrow{\mathbf{r}}-\omega t+\phi _{o}\right)
\end{equation}%
Of course, if we look at a very large wave, but we only look at part of the
wave, we see that our part looks flatter as the wave expands. \FRAME{dhF}{%
2.4881in}{1.6189in}{0pt}{}{}{Figure}{\special{language "Scientific
Word";type "GRAPHIC";maintain-aspect-ratio TRUE;display "USEDEF";valid_file
"T";width 2.4881in;height 1.6189in;depth 0pt;original-width
2.4474in;original-height 1.5826in;cropleft "0";croptop "1";cropright
"1";cropbottom "0";tempfilename 'LTUWCI1S.wmf';tempfile-properties "XPR";}}%
Very far from the source, our wave is flat enough that we can ignore the
curvature across it's wave fronts. We call such a wave a \emph{plane wave}.
There are no true plane waves in nature, but this idealization makes our
mathematical solutions simpler and many waves come close to this
approximation. We will usually stick with the plane wave approximation in
this class.

%TCIMACRO{%
%\TeXButton{Basic Equations}{\hspace{-1.3in}{\LARGE Basic Equations\vspace{0.25in}}}}%
%BeginExpansion
\hspace{-1.3in}{\LARGE Basic Equations\vspace{0.25in}}%
%EndExpansion

Our geneal wave eqauation is 
\begin{equation}
y\left( x,t\right) =A\cos \left( kx-\omega t+\phi _{o}\right)
\end{equation}%
or%
\begin{equation}
y\left( x,t\right) =A\sin \left( kx-\omega t+\phi _{o}\right)
\end{equation}

In general we found that the wave speed could be written as

\begin{equation}
v=\frac{\omega }{k}
\end{equation}
\begin{equation}
v=\lambda f
\end{equation}%
For waves on strings we also found%
\begin{equation}
v=\sqrt{\frac{T}{\mu }}
\end{equation}

We found that all of 
\begin{equation}
\phi =kx-\omega t+\phi _{o}
\end{equation}
could be called the total phase of the wave, or to be breif, we could call
it just the phase of the wave.

The \emph{transverse} postion, speed, and acceleration of a part of the
medium are given by

\begin{equation}
y=A\sin \left( kx-\omega t\right)
\end{equation}%
\begin{equation}
v_{y}=-\omega A\cos \left( kx-\omega t\right)
\end{equation}%
\begin{equation}
a_{y}=-\omega ^{2}A\sin \left( kx-\omega t\right)
\end{equation}

We found that waves in three dimensions have a more complicated amplitude%
\begin{equation}
y=A\left( r\right) \sin \left( \overrightarrow{\mathbf{k}}\cdot 
\overrightarrow{\mathbf{r}}-\omega t+\phi _{o}\right)
\end{equation}

\chapter{Light, Sound, Power}

Reading Assignment 20.5,20.6

%TCIMACRO{%
%\TeXButton{Fundamental Concepts}{\hspace{-1.3in}{\LARGE Fundamental Concepts\vspace{0.25in}}}}%
%BeginExpansion
\hspace{-1.3in}{\LARGE Fundamental Concepts\vspace{0.25in}}%
%EndExpansion

\begin{itemize}
\item Sound waves are formed when a disturbance causes a chain-reaction of
collisions in the molecules of the air or other substance.

\item Power is an amount of energy expended in an amount of time

\item Intensity is an amount of power spread over an area

\item The human auditory system is not a linear , but rather a logarithmic
detector with perceived sound level given by$\beta =10\log _{10}\left( \frac{%
I}{I_{o}}\right) $
\end{itemize}

\section{Waves in matter-Sound}

We have said that sound is a longitudinal wave with a medium of air. Really
any solid, liquid, or gas will work as a medium for sound. For our study, we
will take sound to be a longitudinal wave and treat liquids and gasses.
Solids have additional forces involved due to the tight bonding of the
atoms, and therefore are more complicated. Technically in a solid sound can
be a transverse wave as well a longitudinal wave, but we usually call
transverse waves of this nature \emph{shear waves.}%
%TCIMACRO{%
%\TeXButton{Question 223.4.1}{\marginpar {
%\hspace{-0.5in}
%\begin{minipage}[t]{1in}
%\small{Question 223.4.1}
%\end{minipage}
%}}}%
%BeginExpansion
\marginpar {
\hspace{-0.5in}
\begin{minipage}[t]{1in}
\small{Question 223.4.1}
\end{minipage}
}%
%EndExpansion

\subsection{Periodic Sound Waves}

Let's go back to making sounds. Suppose we push our piston as we did before.%
\FRAME{dhF}{2.9352in}{1.5082in}{0pt}{}{}{Figure}{\special{language
"Scientific Word";type "GRAPHIC";maintain-aspect-ratio TRUE;display
"USEDEF";valid_file "T";width 2.9352in;height 1.5082in;depth
0pt;original-width 2.8911in;original-height 1.471in;cropleft "0";croptop
"1";cropright "1";cropbottom "0";tempfilename
'LXJIJI07.wmf';tempfile-properties "XPR";}} When we push in the piston, it
creates a region of higher pressure next to it.

When we pull back the piston the fluid expands to fill the void. \FRAME{dhF}{%
2.9334in}{1.5402in}{0pt}{}{}{Figure}{\special{language "Scientific
Word";type "GRAPHIC";maintain-aspect-ratio TRUE;display "USEDEF";valid_file
"T";width 2.9334in;height 1.5402in;depth 0pt;original-width
3.6979in;original-height 1.9285in;cropleft "0";croptop "1";cropright
"1";cropbottom "0";tempfilename 'LXJIJI08.wmf';tempfile-properties "XPR";}}%
We create a rarefaction next to the piston.

Suppose we drive the piston sinusoidally. Can we describe the motion of the
particles and of the wave?

\begin{definition}
Compression: A local region of higher pressure in a fluid
\end{definition}

\begin{definition}
Rarefaction: A local region of lower pressure in a fluid
\end{definition}

We can identify the distance between two compressions as $\lambda .$

We define $s\left( x,t\right) $ like we defined a wave function, $y\left(
x,t\right) )$ as the displacement of a particle of fluid relative to its
equilibrium position.%
\begin{equation}
s\left( x,t\right) =s_{\max }\cos \left( kx-\omega t\right)
\end{equation}%
but what is $s_{\max }?$

We remember that $s_{\max }$ is the maximum displacement of a particle of
fluid from its equilibrium position. We plotted this using a bar graph to
show displacement from the equilibrium position for our molecules. As we
push the piston in and out we will get something like this. \FRAME{dhF}{%
3.1211in}{4.9035in}{0pt}{}{}{Figure}{\special{language "Scientific
Word";type "GRAPHIC";maintain-aspect-ratio TRUE;display "USEDEF";valid_file
"T";width 3.1211in;height 4.9035in;depth 0pt;original-width
6.3676in;original-height 10.0336in;cropleft "0";croptop "1";cropright
"1";cropbottom "0";tempfilename 'LTUWCI1T.wmf';tempfile-properties "XPR";}}%
We found before that we get something that looks like a sine wave, but
remember what the bars represent. They represent the displacement from
original position. \FRAME{dhF}{2.4613in}{1.849in}{0pt}{}{}{Figure}{\special%
{language "Scientific Word";type "GRAPHIC";maintain-aspect-ratio
TRUE;display "USEDEF";valid_file "T";width 2.4613in;height 1.849in;depth
0pt;original-width 5.0004in;original-height 3.7498in;cropleft "0";croptop
"1";cropright "1";cropbottom "0";tempfilename
'LTUWCI1U.wmf';tempfile-properties "XPR";}}We don't usually draw bar graphs
to represent sound waves, we usually just draw the sine wave.\FRAME{dtbpF}{%
2.073in}{1.8706in}{0pt}{}{}{Figure}{\special{language "Scientific Word";type
"GRAPHIC";maintain-aspect-ratio TRUE;display "USEDEF";valid_file "T";width
2.073in;height 1.8706in;depth 0pt;original-width 4.1919in;original-height
3.7803in;cropleft "0";croptop "1";cropright "1";cropbottom "0";tempfilename
'Sound_Waves/Pulses_1.wmf';tempfile-properties "XNPR";}}

When the air molecules bunch up to form a compression, the pressure will be
higher. When the air molecules spread out to form a rarefaction, the
pressure will be lower than normal. The variation of the gas pressure $%
\Delta P$ measured from its equilibrium is also periodic

\FRAME{dhF}{2.6083in}{2.8072in}{0pt}{}{}{Figure}{\special{language
"Scientific Word";type "GRAPHIC";maintain-aspect-ratio TRUE;display
"USEDEF";valid_file "T";width 2.6083in;height 2.8072in;depth
0pt;original-width 6.02in;original-height 6.4809in;cropleft "0";croptop
"1";cropright "1";cropbottom "0";tempfilename
'LTUWCI1V.wmf';tempfile-properties "XPR";}}which is why we often refer to a
sound wave as a pressure wave. Think of when the wave gets to your ear. The
wave consists of a group of particles all headed for your ear drum. When
they hit, they exert a force. Pressure is a force spread over an area, 
\begin{equation*}
P=\frac{F}{A}
\end{equation*}%
so in a sense, we hear changes in air pressure!

\section{Speed of Sound Waves}

%TCIMACRO{%
%\TeXButton{Question 223.4.2}{\marginpar {
%\hspace{-0.5in}
%\begin{minipage}[t]{1in}
%\small{Question 223.4.2}
%\end{minipage}
%}}}%
%BeginExpansion
\marginpar {
\hspace{-0.5in}
\begin{minipage}[t]{1in}
\small{Question 223.4.2}
\end{minipage}
}%
%EndExpansion

The speed of sound in air is around $340\unit{m}/\unit{s}.$ The speed
changes when we change media, and even when we are in the same media but the
temperature changes. For sound in air, a good approximation near standard
pressure and temperature is%
\begin{equation}
v=v_{o}\sqrt{1+\frac{T_{c}}{T_{o}}}
\end{equation}%
where $v_{o}=331\frac{\unit{m}}{\unit{s}}$ and $T_{o}=273\unit{K}$ ($0\unit{%
%TCIMACRO{\U{2103}}%
%BeginExpansion
{}^{\circ}{\rm C}%
%EndExpansion
}).$\footnote{$v=v_{o}\sqrt{1+\frac{T_{C}}{T_{o}}}=v_{o}\sqrt{\frac{T_{o}}{%
T_{o}}+\frac{T_{C}}{T_{o}}}=v_{o}\sqrt{\frac{T_{o}+T_{C}}{T_{o}}}=v_{o}\sqrt{%
\frac{T_{K}}{T_{o}}}$}

Why temperature? The density and pressure of air change with temperature.
The air molecules gain kinetic energy and tend to move farther apart from
each other when they are warm. This changes the time it takes to transfer
energy.

\subsection{Boundaries}

%TCIMACRO{%
%\TeXButton{Question 223.4.3}{\marginpar {
%\hspace{-0.5in}
%\begin{minipage}[t]{1in}
%\small{Question 223.4.3}
%\end{minipage}
%}}}%
%BeginExpansion
\marginpar {
\hspace{-0.5in}
\begin{minipage}[t]{1in}
\small{Question 223.4.3}
\end{minipage}
}%
%EndExpansion
Suppose two pulses travel in the same medium, say, on a rope, and they
approach a different rope with a different linear mass density. If the new
rope is heavier, we expect the wave speed to slow down. So as one pulse
reaches the boundary, it will go slower. This allows the second pulse to
catch-up before it, too, slows down at the boundary.\FRAME{dhF}{2.2727in}{%
1.2644in}{0pt}{}{}{Figure}{\special{language "Scientific Word";type
"GRAPHIC";maintain-aspect-ratio TRUE;display "USEDEF";valid_file "T";width
2.2727in;height 1.2644in;depth 0pt;original-width 3.4618in;original-height
1.9147in;cropleft "0";croptop "1";cropright "1";cropbottom "0";tempfilename
'MACFPS00.wmf';tempfile-properties "XPR";}}

Now suppose a sinusoidal wave approaches the boundary. We can envision the
crests like pulses, and we expect the first crest to slow down when it
reaches the boundary, letting the other crests catch up. Once the wave
passes the boundary, the crests will be closer together. The wavelength
changes as we move to the slower medium.

But does the frequency change? We know that 
\begin{equation*}
v=\lambda f
\end{equation*}%
so%
\begin{equation*}
f=\frac{v}{\lambda }
\end{equation*}%
both the speed and the wavelength have changed, but did they change
proportionately so $f$ is constant? This must be so. Think that the change
in wavelength is due to the relative speed of the wave in the two media. If $%
\Delta v$ is small, the change in $\lambda $ will be small because the
crests are not delayed too long. If $\Delta v$ is large, the crests are
delayed by a large amount and so the change in $\lambda $ is large. We won't
derive the fact that $f$ is constant, but we can see that it is very
believable that it is true.

This is true for all waves, even light. When a wave crosses a boundary from
a fast to a slow or a slow to a fast medium, $\lambda $ will change and $f$
will remain constant.%
%TCIMACRO{%
%\TeXButton{Question 223.4.4}{\marginpar {
%\hspace{-0.5in}
%\begin{minipage}[t]{1in}
%\small{Question 223.4.4}
%\end{minipage}
%}}}%
%BeginExpansion
\marginpar {
\hspace{-0.5in}
\begin{minipage}[t]{1in}
\small{Question 223.4.4}
\end{minipage}
}%
%EndExpansion

Let's find an expression for the new wavelength. The frequency of the light
must be the same.%
\begin{equation*}
f_{i}=f_{f}
\end{equation*}%
and we know that in general 
\begin{equation*}
f=\frac{v}{\lambda }
\end{equation*}%
so%
\begin{equation*}
\frac{v_{i}}{\lambda _{i}}=\frac{v_{f}}{\lambda _{f}}
\end{equation*}%
so%
\begin{equation}
\lambda _{f}=\frac{v_{f}}{v_{i}}\lambda _{i}  \label{WavelengthChange}
\end{equation}%
and we can see that if $v_{f}$ is slower than $v_{i}$ our wavelength does
get shorter.

\section{Waves in fields-Light}

Sound is a wave in matter, but what is light? It will really take the rest
of the course (and then some) to answer this question. But we know that
light can travel through a vacuum. Therefore, light can't be a wave in some
type of matter. We will find later that there exists something called an
electromagnetic field created by charged particles. It turns out that light
seems to be a wave in this electromagnetic field. It will take us a while to
fully understand this concept, but don't worry. Physicists knew that light
was a wave for almost $80$ years before the electric field was shown to be
the medium. We can do a lot just knowing light behaves like a wave.

If light is a wave, the light we see is just one small part of a whole class
of waves that are possible in this electromagnetic field medium. Radio
waves, and microwaves, and $x$-rays are all just different types of
electromagnetic waves. The next figure shows where all of these
electromagnetic waves fit ordered by wavelength (and frequency).\FRAME{dtbpFU%
}{3.8795in}{0.3035in}{0pt}{\Qcb{Electromagnetic Spectrum (Public Domain
image courtesy NASA)}}{}{Figure}{\special{language "Scientific Word";type
"GRAPHIC";maintain-aspect-ratio TRUE;display "USEDEF";valid_file "T";width
3.8795in;height 0.3035in;depth 0pt;original-width 4.0309in;original-height
0.2897in;cropleft "0";croptop "1";cropright "1";cropbottom "0";tempfilename
'MFFY0E00.wmf';tempfile-properties "XPR";}}

\subsection{Speed of Electromagnetic waves}

There is something very unique about this electromagnetic field medium. The
waves in this medium travel at a constant speed-no matter what frame of
reference we are in. This fact leads to the formation of the Special Theory
of Relativity and the famous equation 
\begin{equation*}
E=mc^{2}
\end{equation*}%
where $c$ is this speed of light%
\begin{equation*}
c=299792458\frac{\unit{m}}{\unit{s}}
\end{equation*}

Light does slow down when it enters a material medium, like glass, or even
air. The actual speed that light travels does not change. What happens is
that light is absorbed by the electrons in the atoms of the material
substance. The electron temporarily takes up all the energy from a bit of
the light wave--but only temporarily. It eventually has to give up the
energy and the light wave reforms. But it has lost some time in the process,
so it's average speed is less. How much less depends on how long the
electrons in the atoms can hang-on to the light. Each substance is different.

We can devise a way to express how much slower light will appear to go in a
substance using the ratio%
\begin{equation*}
\frac{c}{v}
\end{equation*}%
the ratio of the speed of light, $c,$ to the average speed in the substance, 
$v.$ This ratio is so useful that we give it a name, the \emph{index of
refraction}.%
\begin{equation*}
n=\frac{c}{v}
\end{equation*}

\section{Power and Intensity}

We know that energy is being transferred by the wave, whether it is a light
or sound wave. We should wonder, how fast is energy transferring? This can
mean the difference between sunlight on a warm summer day and being burned
by a laser beam. We will start by considering the rate of energy transfer, 
\emph{power}.

\subsubsection{Power}

%TCIMACRO{%
%\TeXButton{Question 222.4.5}{\marginpar {
%\hspace{-0.5in}
%\begin{minipage}[t]{1in}
%\small{Question 222.4.5}
%\end{minipage}
%}}}%
%BeginExpansion
\marginpar {
\hspace{-0.5in}
\begin{minipage}[t]{1in}
\small{Question 222.4.5}
\end{minipage}
}%
%EndExpansion
%TCIMACRO{%
%\TeXButton{Question 223.4.6}{\marginpar {
%\hspace{-0.5in}
%\begin{minipage}[t]{1in}
%\small{Question 223.4.6}
%\end{minipage}
%}}}%
%BeginExpansion
\marginpar {
\hspace{-0.5in}
\begin{minipage}[t]{1in}
\small{Question 223.4.6}
\end{minipage}
}%
%EndExpansion
The concept of power should be familiar to us from PH121 or Statics and
Dynamics. We can find the power by%
\begin{equation*}
\mathcal{P}=\frac{\Delta E}{\Delta t}
\end{equation*}%
where $\Delta E$ is the power transferred and $\Delta t$ is the time it
takes to make the transfer.

\subsection{Intensity}

%TCIMACRO{%
%\TeXButton{Tuning Fork Demo}{\marginpar {
%\hspace{-0.5in}
%\begin{minipage}[t]{1in}
%\small{Tuning Fork Demo}
%\end{minipage}
%}}}%
%BeginExpansion
\marginpar {
\hspace{-0.5in}
\begin{minipage}[t]{1in}
\small{Tuning Fork Demo}
\end{minipage}
}%
%EndExpansion

%TCIMACRO{%
%\TeXButton{Question 223.4.7}{\marginpar {
%\hspace{-0.5in}
%\begin{minipage}[t]{1in}
%\small{Question 223.4.7}
%\end{minipage}
%}}}%
%BeginExpansion
\marginpar {
\hspace{-0.5in}
\begin{minipage}[t]{1in}
\small{Question 223.4.7}
\end{minipage}
}%
%EndExpansion
%TCIMACRO{%
%\TeXButton{Question 223.4.8 }{\marginpar {
%\hspace{-0.5in}
%\begin{minipage}[t]{1in}
%\small{QQuestion 223.4.8 }
%\end{minipage}
%}}}%
%BeginExpansion
\marginpar {
\hspace{-0.5in}
\begin{minipage}[t]{1in}
\small{QQuestion 223.4.8 }
\end{minipage}
}%
%EndExpansion
%TCIMACRO{%
%\TeXButton{Question 223.4.9}{\marginpar {
%\hspace{-0.5in}
%\begin{minipage}[t]{1in}
%\small{Question 223.4.9}
%\end{minipage}
%}}}%
%BeginExpansion
\marginpar {
\hspace{-0.5in}
\begin{minipage}[t]{1in}
\small{Question 223.4.9}
\end{minipage}
}%
%EndExpansion
We now define something new.%
\begin{equation}
\mathcal{I}\equiv \frac{\mathcal{P}}{A}
\end{equation}%
that is, the power divided by the area. But what does it mean?

Consider a point source.\FRAME{dtbpF}{2.1006in}{1.9649in}{0pt}{}{}{Figure}{%
\special{language "Scientific Word";type "GRAPHIC";maintain-aspect-ratio
TRUE;display "USEDEF";valid_file "T";width 2.1006in;height 1.9649in;depth
0pt;original-width 4.2078in;original-height 3.9355in;cropleft "0";croptop
"1";cropright "1";cropbottom "0";tempfilename
'Sound_Waves/spherical_wave_1.wmf';tempfile-properties "XNPR";}}it sends out
waves in all directions. The wave crests will define a sphere around the
points source (the figure shows a cross section but remember it is a wave
from a point source, so we are really drawing concentric spheres like
balloons inside of balloons.). Then form our point source%
\begin{equation}
\mathcal{I}=\frac{\mathcal{P}}{4\pi r^{2}}
\end{equation}%
As the wave travels, its power per unit area decreases with the square of
the distance (think gravity) because the area is getting larger.\FRAME{dhF}{%
2.655in}{1.8273in}{0pt}{}{}{Figure}{\special{language "Scientific Word";type
"GRAPHIC";maintain-aspect-ratio TRUE;display "USEDEF";valid_file "T";width
2.655in;height 1.8273in;depth 0pt;original-width 2.6135in;original-height
1.7902in;cropleft "0";croptop "1";cropright "1";cropbottom "0";tempfilename
'LTUWCJ1Y.wmf';tempfile-properties "XPR";}}This quantity that tells us how
spread our our power has become is called the \emph{intensity} of the wave.

Suppose we cup our hand to our ear. We can now hear fainter sounds. But what
are we doing that makes the difference?

We are increasing the area of our ear. Our ears work by transferring the
energy of the sound wave to a electro-chemical-mechanical device that
creates a nerve signal. The more energy, the stronger the signal. If we are
a distance $r$ away from the source of the sound then the intensity is 
\begin{equation*}
\mathcal{I}=\frac{\mathcal{P}_{source}}{A_{wave}}
\end{equation*}%
But we are collecting the sound wave with another area, the area of our
hand. The power received is 
\begin{eqnarray*}
\mathcal{P}_{received} &=&\mathcal{I}A_{hand} \\
&=&\frac{A_{hand}}{A_{wave}}\mathcal{P}_{source}
\end{eqnarray*}%
and we can see that, indeed, the larger the hand, the more power, and
therefore more energy we collect. This is the idea behind a dish antenna for
communications and the idea behind the acoustic dish microphones we see at
sporting events. In next figure, we can see that it would take an
increasingly larger dish to maintain the same power gathering capability as
we get farther from the source.\FRAME{dhF}{3.0312in}{2.2131in}{0pt}{}{}{%
Figure}{\special{language "Scientific Word";type
"GRAPHIC";maintain-aspect-ratio TRUE;display "USEDEF";valid_file "T";width
3.0312in;height 2.2131in;depth 0pt;original-width 4.9355in;original-height
3.595in;cropleft "0";croptop "1";cropright "1";cropbottom "0";tempfilename
'LTUWCJ1Z.wmf';tempfile-properties "XPR";}}

\subsection{Sound Levels in Decibels}

Our Design Engineer made an interesting choice in building us. We need to
hear very faint sounds, and very loud sounds too. In order to make us able
to hear the soft sounds without causing extreme discomfort when we hear the
loud, He gave up linearity. That is, we don't hear twice the sound intensity
as twice as loud.%
%TCIMACRO{%
%\TeXButton{Question 223.4.10}{\marginpar {
%\hspace{-0.5in}
%\begin{minipage}[t]{1in}
%\small{Question 223.4.10}
%\end{minipage}
%}}}%
%BeginExpansion
\marginpar {
\hspace{-0.5in}
\begin{minipage}[t]{1in}
\small{Question 223.4.10}
\end{minipage}
}%
%EndExpansion
%TCIMACRO{%
%\TeXButton{Question 223.4.11}{\marginpar {
%\hspace{-0.5in}
%\begin{minipage}[t]{1in}
%\small{Question 223.4.11}
%\end{minipage}
%}} }%
%BeginExpansion
\marginpar {
\hspace{-0.5in}
\begin{minipage}[t]{1in}
\small{Question 223.4.11}
\end{minipage}
}
%EndExpansion
The mathematical expression that matches our perception of loudness to the
intensity is 
\begin{equation}
\beta =10\log _{10}\left( \frac{I}{I_{o}}\right)
\end{equation}

where the quantity $I_{o}$ is a reference intensity.%
%TCIMACRO{%
%\TeXButton{Question 223.4.12}{\marginpar {
%\hspace{-0.5in}
%\begin{minipage}[t]{1in}
%\small{Question 223.4.12}
%\end{minipage}
%}} }%
%BeginExpansion
\marginpar {
\hspace{-0.5in}
\begin{minipage}[t]{1in}
\small{Question 223.4.12}
\end{minipage}
}
%EndExpansion
We are comparing the intensity of our sound with some reference intensity, $%
I_{o},$ to see how much louder our sound seems to be.

%TCIMACRO{%
%\TeXButton{Sound Meter Demo}{\marginpar {
%\hspace{-0.5in}
%\begin{minipage}[t]{1in}
%\small{Sound Meter Demo}
%\end{minipage}
%}}}%
%BeginExpansion
\marginpar {
\hspace{-0.5in}
\begin{minipage}[t]{1in}
\small{Sound Meter Demo}
\end{minipage}
}%
%EndExpansion
We call $\beta $ the \emph{sound level}. $I_{o}$ we choose to be the \emph{%
threshold of hearing}, the intensity that is just barely audible. Measured
this way, we say that intensity is in units of decibels (dB). The decibel,
is an engineer's friend (and useful for physicists too!) because it can
describe a non-linear response in a linear way that is easy to match to our
human experience.

Suppose we double the intensity by a factor of $2.$%
\begin{eqnarray*}
\beta &=&10\log _{10}\left( \frac{2I_{o}}{I_{o}}\right) \\
&=&10\log _{10}2 \\
&=&\allowbreak 3.\,\allowbreak 010\,3dB
\end{eqnarray*}%
The sound intensity level is not twice as large, but only $3dB$ larger. It
is a tiny increase. This is what we hear. 
%TCIMACRO{%
%\TeXButton{Question 223.4.13}{\marginpar {
%\hspace{-0.5in}
%\begin{minipage}[t]{1in}
%\small{Question 223.4.13}
%\end{minipage}
%}} }%
%BeginExpansion
\marginpar {
\hspace{-0.5in}
\begin{minipage}[t]{1in}
\small{Question 223.4.13}
\end{minipage}
}
%EndExpansion
A good rule to remember is that $3dB$ corresponds to a doubling of the
intensity.

The tables that follow give some common sounds in units of dB and $\unit{W}/%
\unit{m}^{2}.$ Just for reference, I have measured a Guns n Roses concert at
120 dB outside the stadium.%
%TCIMACRO{%
%\TeXButton{Question 223.4.14}{\marginpar {
%\hspace{-0.5in}
%\begin{minipage}[t]{1in}
%\small{Question 223.4.14}
%\end{minipage}
%}}}%
%BeginExpansion
\marginpar {
\hspace{-0.5in}
\begin{minipage}[t]{1in}
\small{Question 223.4.14}
\end{minipage}
}%
%EndExpansion
%TCIMACRO{%
%\TeXButton{Question 223.4.15 GR}{\marginpar {
%\hspace{-0.5in}
%\begin{minipage}[t]{1in}
%\small{Question 223.4.15 GR}
%\end{minipage}
%}}}%
%BeginExpansion
\marginpar {
\hspace{-0.5in}
\begin{minipage}[t]{1in}
\small{Question 223.4.15 GR}
\end{minipage}
}%
%EndExpansion
%TCIMACRO{%
%\TeXButton{Question 223.4.16}{\marginpar {
%\hspace{-0.5in}
%\begin{minipage}[t]{1in}
%\small{Question 223.4.16}
%\end{minipage}
%}}}%
%BeginExpansion
\marginpar {
\hspace{-0.5in}
\begin{minipage}[t]{1in}
\small{Question 223.4.16}
\end{minipage}
}%
%EndExpansion

\begin{equation*}
\begin{tabular}{|c|c|}
\hline
Sound & Sound Level (dB) \\ \hline
Jet Airplane at 30m & 140 \\ \hline
Rock Concert & 120 \\ \hline
Siren at 30m & 100 \\ \hline
Car interior when Traveling $60\unit{mi}/\unit{h}$ & 90 \\ \hline
Street Traffic & 70 \\ \hline
Talk at $30\unit{cm}$ & 65 \\ \hline
Whisper & 20 \\ \hline
Rustle of Leaves & 10 \\ \hline
Quietest thing we can hear ($I_{o}$) & 0 \\ \hline
\end{tabular}%
\end{equation*}

\subsubsection{Loudness and frequency}

%TCIMACRO{%
%\TeXButton{Frequency Range Demo}{\marginpar {
%\hspace{-0.5in}
%\begin{minipage}[t]{1in}
%\small{Frequency Range Demo}
%\end{minipage}
%}}}%
%BeginExpansion
\marginpar {
\hspace{-0.5in}
\begin{minipage}[t]{1in}
\small{Frequency Range Demo}
\end{minipage}
}%
%EndExpansion
\FRAME{dhFU}{2.8126in}{2.8499in}{0pt}{\Qcb{Robinson-Dadson equal loudness
curves (Image in the Public Domain courtesy Lindosland)}}{}{Figure}{\special%
{language "Scientific Word";type "GRAPHIC";maintain-aspect-ratio
TRUE;display "USEDEF";valid_file "T";width 2.8126in;height 2.8499in;depth
0pt;original-width 2.8374in;original-height 2.8756in;cropleft "0";croptop
"1";cropright "1";cropbottom "0";tempfilename
'N4LQTD00.wmf';tempfile-properties "XPR";}}

Our ears are truly amazing in their range and ability. But, sounds with the
same intensity at different frequencies do not appear to us to have the same
loudness. The frequency response graph above show how this relationship
works for test subjects. We don't hear high or low frequencies as well. We
have a peak response around $4000\unit{Hz}.$

\chapter{Doppler Effect and Superposition}

%TCIMACRO{%
%\TeXButton{Fundamental Concepts}{\hspace{-1.3in}{\LARGE Fundamental Concepts\vspace{0.25in}}}}%
%BeginExpansion
\hspace{-1.3in}{\LARGE Fundamental Concepts\vspace{0.25in}}%
%EndExpansion

\begin{itemize}
\item The frequency of a wave depends on the relative motion between the
source and detector.

\item Two waves in the same medium add up point for point at every location
in the medium. This process is called superposition.
\end{itemize}

\section{Doppler Effect}

We have learned what happens when a sound wave is generated. But so far, we
have assumed that sound emitter was staying still. But we know of many sound
emitters that move. What happens if the emitter of the sound is moving?
Worse, Back in PH121 or Dynamics we considered the relative motion between
two reference frames. What happens to the sound emitter is stationary, but
we, the listener, are moving?

%TCIMACRO{%
%\TeXButton{Question 223.5.1}{\marginpar {
%\hspace{-0.5in}
%\begin{minipage}[t]{1in}
%\small{Question 223.5.1}
%\end{minipage}
%}}}%
%BeginExpansion
\marginpar {
\hspace{-0.5in}
\begin{minipage}[t]{1in}
\small{Question 223.5.1}
\end{minipage}
}%
%EndExpansion
%TCIMACRO{%
%\TeXButton{Doppler Ball Demo}{\marginpar {
%\hspace{-0.5in}
%\begin{minipage}[t]{1in}
%\small{Doppler Ball Demo}
%\end{minipage}
%}}}%
%BeginExpansion
\marginpar {
\hspace{-0.5in}
\begin{minipage}[t]{1in}
\small{Doppler Ball Demo}
\end{minipage}
}%
%EndExpansion
Let's start by considering an inertial reference frame (remember this from
Dynamics/PH121?)

Suppose we pick two reference frames, one traveling with a velocity $v_{x}$
with respect to the other. Let's also place them far away from any other
object.

\FRAME{dtbpF}{3.3667in}{1.2306in}{0pt}{}{}{Figure}{\special{language
"Scientific Word";type "GRAPHIC";maintain-aspect-ratio TRUE;display
"USEDEF";valid_file "T";width 3.3667in;height 1.2306in;depth
0pt;original-width 5.6181in;original-height 2.0374in;cropleft "0";croptop
"1";cropright "1";cropbottom "0";tempfilename
'Sound_Waves/relative_motion0.wmf';tempfile-properties "XNPR";}}

Person $A$ sees himself as stationary and sees person $B$ traveling with
velocity $v_{x}.$ Person $B$ sees himself as stationary, and person $A$
traveling with velocity $-v_{x}.$ We can't tell which view point is correct.
In fact, both are equally valid. So it seems that whether the emitter moves,
or the detector moves, either way if the motion matters, it should matter
the same for both cases. From this brief review, it seems that is the \emph{%
relative} speed $v_{x}$ that we must consider when thinking about our sound
waves.

Now suppose we have a wave generator (a point source) creating spherical
waves. Let the point source be at rest, say, in frame $A$. 
%TCIMACRO{%
%\TeXButton{BYU Demo}{\marginpar {
%\hspace{-0.5in}
%\begin{minipage}[t]{1in}
%\small{BYU Demo}
%\end{minipage}
%}}}%
%BeginExpansion
\marginpar {
\hspace{-0.5in}
\begin{minipage}[t]{1in}
\small{BYU Demo}
\end{minipage}
}%
%EndExpansion
\FRAME{dtbpF}{2.7276in}{1.5618in}{0pt}{}{}{Figure}{\special{language
"Scientific Word";type "GRAPHIC";maintain-aspect-ratio TRUE;display
"USEDEF";valid_file "T";width 2.7276in;height 1.5618in;depth
0pt;original-width 8.6667in;original-height 4.9449in;cropleft "0";croptop
"1";cropright "1";cropbottom "0";tempfilename
'Sound_Waves/Doppler_1.wmf';tempfile-properties "XNPR";}}%
%TCIMACRO{%
%\TeXButton{Question 223.5.2}{\marginpar {
%\hspace{-0.5in}
%\begin{minipage}[t]{1in}
%\small{Question 223.5.2}
%\end{minipage}
%}}}%
%BeginExpansion
\marginpar {
\hspace{-0.5in}
\begin{minipage}[t]{1in}
\small{Question 223.5.2}
\end{minipage}
}%
%EndExpansion
Let's also assume a detector. If the detector is stationary with respect to
the emitter, the detector sees a frequency of the wave, $f$ just as it is
created by the emitter. But lets have the detector be in frame $B$ \FRAME{%
dtbpF}{3.257in}{1.303in}{0in}{}{}{Figure}{\special{language "Scientific
Word";type "GRAPHIC";maintain-aspect-ratio TRUE;display "USEDEF";valid_file
"T";width 3.257in;height 1.303in;depth 0in;original-width
3.2907in;original-height 1.2994in;cropleft "0";croptop "1";cropright
"1";cropbottom "0";tempfilename 'MTE6ST00.wmf';tempfile-properties "XPR";}}%
so that it moves relative to the emitter and take our point of view from the
detector frame.%
%TCIMACRO{%
%\TeXButton{Move George}{\marginpar {
%\hspace{-0.5in}
%\begin{minipage}[t]{1in}
%\small{Move George}
%\end{minipage}
%}} }%
%BeginExpansion
\marginpar {
\hspace{-0.5in}
\begin{minipage}[t]{1in}
\small{Move George}
\end{minipage}
}
%EndExpansion
A top view might look like this

\FRAME{dtbpF}{1.4503in}{1.3967in}{0pt}{}{}{Figure}{\special{language
"Scientific Word";type "GRAPHIC";maintain-aspect-ratio TRUE;display
"USEDEF";valid_file "T";width 1.4503in;height 1.3967in;depth
0pt;original-width 5.1374in;original-height 4.9449in;cropleft "0";croptop
"1";cropright "1";cropbottom "0";tempfilename
'Sound_Waves/Dopler_2.wmf';tempfile-properties "XNPR";}}Remember, that the
frequency is the number of crests that pass by a given point in a unit time.
Does the moving detector see the same number of crests per unit time as when
it was stationary?

No, the frequency appears to be higher! That is because every time a wave
crest hits the detector, the detector moves toward the next wave crest.

How about if we let the detector move the other way?\FRAME{dtbpF}{3.6641in}{%
1.4653in}{0in}{}{}{Figure}{\special{language "Scientific Word";type
"GRAPHIC";maintain-aspect-ratio TRUE;display "USEDEF";valid_file "T";width
3.6641in;height 1.4653in;depth 0in;original-width 3.7067in;original-height
1.4644in;cropleft "0";croptop "1";cropright "1";cropbottom "0";tempfilename
'MTE6V801.wmf';tempfile-properties "XPR";}}

The top view might look like this\FRAME{dtbpF}{1.5506in}{1.4918in}{0pt}{}{}{%
Figure}{\special{language "Scientific Word";type
"GRAPHIC";maintain-aspect-ratio TRUE;display "USEDEF";valid_file "T";width
1.5506in;height 1.4918in;depth 0pt;original-width 5.1374in;original-height
4.9449in;cropleft "0";croptop "1";cropright "1";cropbottom "0";tempfilename
'Sound_Waves/Dopler_3.wmf';tempfile-properties "XNPR";}}Again the frequency
seen by the detector is different, but this time lower. Each time a wave
crest hits the detector, the detector moves away from the next wave crest,
giving more time for the wave to catch up to the detector (if the period
between wave crests goes up, then the frequency must go down because $f=%
\frac{1}{T}$).

%TCIMACRO{%
%\TeXButton{Question 223.5.3}{\marginpar {
%\hspace{-0.5in}
%\begin{minipage}[t]{1in}
%\small{Question 223.5.3}
%\end{minipage}
%}}}%
%BeginExpansion
\marginpar {
\hspace{-0.5in}
\begin{minipage}[t]{1in}
\small{Question 223.5.3}
\end{minipage}
}%
%EndExpansion
We can quantify this change. Take our usual variables $f_{A},$ $\lambda _{A}$
for the stationary emitter, $f_{B},$ $\lambda _{B}$ for the moving detector,
and and the velocity of sound $v_{sound}$. When the detector moves toward
the source, it sees the wave velocity as 
\begin{equation*}
v_{B}=v_{sound}+v_{x}
\end{equation*}%
Since the detector is riding along with frame $B$ which is moving with speed 
$v_{x}$ we could write 
\begin{equation*}
v_{d}=v_{x}
\end{equation*}%
\begin{equation}
v_{B}=v_{sound}+v_{d}
\end{equation}%
where we are using $v_{d}$ as the detector speed. In effect, the relative
speed adds to the speed of sound making the wave crests come faster from the
point of view of the detector. The wavelength will not be changed $\left(
\lambda _{A}=\lambda _{B}\right) $, since the distance between wave crests
does not change, so 
\begin{equation*}
v=\lambda f
\end{equation*}%
tells us the frequency must change. The new frequency $f_{B}$ is given by%
\begin{equation*}
f_{B}=\frac{v_{B}}{\lambda }=\frac{v_{sound}+v_{d}}{\lambda }
\end{equation*}%
We can eliminate $\lambda $ from this expression for the change in $f$ by
using $v_{A}=\lambda f_{A}$ again, this time solving for $\lambda $%
\begin{equation}
f_{B}=\frac{v_{sound}+v_{d}}{v_{sound}}f_{A}\text{ \qquad observer moving
toward the source}
\end{equation}

%TCIMACRO{%
%\TeXButton{Change the Demo}{\marginpar {
%\hspace{-0.5in}
%\begin{minipage}[t]{1in}
%\small{Change the Demo}
%\end{minipage}
%}}}%
%BeginExpansion
\marginpar {
\hspace{-0.5in}
\begin{minipage}[t]{1in}
\small{Change the Demo}
\end{minipage}
}%
%EndExpansion
Now if the detector is going the other way $v_{d}$ is negative.%
\begin{equation*}
v_{B}=v_{sound}-v_{d}
\end{equation*}%
We expect that as the wave crest approaches the detector, the detector moves
away from it. It takes longer for the crest to reach the detector. The
frequency will be smaller.%
\begin{equation}
f_{B}=\frac{v_{sound}-v_{d}}{v_{sound}}f_{A}\text{ \qquad observer moving
away from the source}
\end{equation}

From our thinking about the motion of two inertial reference frames, we
expect a similar situation if the detector is stationary and the source
moves. \FRAME{dtbpF}{3.4805in}{1.7021in}{0in}{}{}{Figure}{\special{language
"Scientific Word";type "GRAPHIC";maintain-aspect-ratio TRUE;display
"USEDEF";valid_file "T";width 3.4805in;height 1.7021in;depth
0in;original-width 3.5178in;original-height 1.7057in;cropleft "0";croptop
"1";cropright "1";cropbottom "0";tempfilename
'MZCZJP00.wmf';tempfile-properties "XPR";}}Since the emitter is now riding
along with frame $A$ at the relative speed, $v_{x}$ we could write 
\begin{equation*}
v_{e}=v_{x}
\end{equation*}%
where $v_{e}$ is the speed of the emitter. In this case the detector will
see a different wavelength. The top down view might look like this \FRAME{%
dtbpF}{1.5307in}{1.4754in}{0pt}{}{}{Figure}{\special{language "Scientific
Word";type "GRAPHIC";maintain-aspect-ratio TRUE;display "USEDEF";valid_file
"T";width 1.5307in;height 1.4754in;depth 0pt;original-width
1.4944in;original-height 1.4399in;cropleft "0";croptop "1";cropright
"1";cropbottom "0";tempfilename 'MAHZY400.wmf';tempfile-properties "XPR";}}%
In fact, if we measure the distance between the crests we must account for
the fact that the source moved by an amount%
\begin{equation*}
\Delta x=v_{e}T=\frac{v_{e}}{f_{A}}
\end{equation*}%
during one period. Then the wavelength is seen to be shorter by this amount. 
\begin{equation*}
\lambda _{B}=\lambda _{A}-\frac{v_{e}}{f_{A}}
\end{equation*}

Using the basic equation 
\begin{equation*}
\lambda =\frac{v}{f}
\end{equation*}%
once more, we can write the frequency as 
\begin{equation*}
f_{B}=\frac{v_{sound}}{\lambda _{B}}=\frac{v_{sound}}{\lambda _{A}-\frac{%
v_{e}}{f_{A}}}
\end{equation*}%
Using the basic equation 
\begin{equation*}
\lambda =\frac{v}{f}
\end{equation*}%
we can write this as%
\begin{equation*}
f_{B}=\frac{v_{sound}}{\frac{v_{sound}}{f_{A}}-\frac{v_{e}}{f_{A}}}
\end{equation*}%
or%
\begin{equation}
f_{B}=\frac{v_{sound}}{v_{sound}-v_{e}}f_{A}\qquad \text{Source moving
toward observer}
\end{equation}%
When the source is moving away from the detector,\FRAME{dtbpF}{3.8477in}{%
1.9638in}{0pt}{}{}{Figure}{\special{language "Scientific Word";type
"GRAPHIC";maintain-aspect-ratio TRUE;display "USEDEF";valid_file "T";width
3.8477in;height 1.9638in;depth 0pt;original-width 3.8921in;original-height
1.9726in;cropleft "0";croptop "1";cropright "1";cropbottom "0";tempfilename
'MZCZLD03.wmf';tempfile-properties "XPR";}}the top down view might look like
this \FRAME{dtbpF}{1.8507in}{1.7841in}{0pt}{}{}{Figure}{\special{language
"Scientific Word";type "GRAPHIC";maintain-aspect-ratio TRUE;display
"USEDEF";valid_file "T";width 1.8507in;height 1.7841in;depth
0pt;original-width 1.8135in;original-height 1.7469in;cropleft "0";croptop
"1";cropright "1";cropbottom "0";tempfilename
'MAI01101.wmf';tempfile-properties "XPR";}}we expect the wavelength to be
larger. This gives 
\begin{equation}
f_{B}=\frac{v_{sound}}{v_{sound}+v_{e}}f_{A}\qquad \text{Source moving away
from observer}
\end{equation}%
We can combine these formulae to make one expression 
\begin{equation}
f_{B}=\frac{v_{sound}\pm v_{d}}{v_{sound}\mp v_{e}}f_{A}
\end{equation}%
where we use the top sign for the speed when the mover (detector or emitter)
is going toward the non-mover.

We can see that by moving the emitter or the detector, we get a frequency
change. This fact is named after the scientist who studied it. It is called
the \emph{Doppler effect }and the change in frequency is called the \emph{%
Doppler shift}.

\subsection{Doppler effect in light}

Light is also a wave, and so we would expect a Doppler shift in light.
Indeed we do see a Doppler shift when we look at moving glowing objects.
Here is an optical spectrum of the Sun on the top and a spectrum of a
similar star moving away from us in the middle. The final spectrum is for a
star moving toward us.\FRAME{dtbpFU}{1.5753in}{1.3784in}{0pt}{\Qcb{Top:
Normal 'dark' spectral line positions at rest. Middle: Source moving away
from observer. Bottom: Source moving towards observer. (Public domain image
courtesy NASA: http://www.jwst.nasa.gov/education/7Page45.pdf)}}{}{Figure}{%
\special{language "Scientific Word";type "GRAPHIC";maintain-aspect-ratio
TRUE;display "USEDEF";valid_file "T";width 1.5753in;height 1.3784in;depth
0pt;original-width 1.5385in;original-height 1.343in;cropleft "0";croptop
"1";cropright "1";cropbottom "0";tempfilename
'MFFZ7B03.wmf';tempfile-properties "XPR";}}Note that the wavelength of the
lines is shifted toward the red part of the spectrum when the glowing object
moves away from us. This is equivalent to lowering of the frequency of a
truck engine noise as it goes away from us. The larger wavelengths indicate
a lower frequency of light because 
\begin{equation*}
f=\frac{c}{\lambda }
\end{equation*}%
This gives us a way to determine if distant stars and galaxies are moving
toward or away from us. We look for the chemical signature pattern of lines,
then see whether they are shifted to the red (moving away from us) or blue
(moving toward us) compared to the position in their spectrum of the Sun.
This photo is of some of the most distant galaxies that are moving very fast
away from us. Their redshift is very large.\FRAME{dhFU}{2.3753in}{2.6663in}{%
0pt}{\Qcb{High Readshift Galaxy Cluster shown here in false color from the
Spitzer Space Telescope. (Public domain image courtacy NASA/JPL-Caltech/S.A.
Stanford (UC Davis/LLNL)}}{}{Figure}{\special{language "Scientific
Word";type "GRAPHIC";maintain-aspect-ratio TRUE;display "USEDEF";valid_file
"T";width 2.3753in;height 2.6663in;depth 0pt;original-width
2.3359in;original-height 2.623in;cropleft "0";croptop "1";cropright
"1";cropbottom "0";tempfilename 'LXOYRR06.wmf';tempfile-properties "XPR";}}

Deriving the Doppler equation for light is more tricky because the speed of
light is constant in all reference frames. We really tackle this in our
PH279 class. So I\ will just quote the result here.

\begin{eqnarray}
\lambda _{-} &=&\lambda _{o}\sqrt{\frac{1+\frac{v}{c}}{1-\frac{v}{c}}}\text{ 
}\qquad \text{receding source} \\
\lambda _{-} &=&\lambda _{o}\sqrt{\frac{1-\frac{v}{c}}{1+\frac{v}{c}}}\text{ 
}\qquad \text{Approaching source}
\end{eqnarray}

\section{Superposition Principle}

%TCIMACRO{%
%\TeXButton{Wave Machine Demo}{\marginpar {
%\hspace{-0.5in}
%\begin{minipage}[t]{1in}
%\small{Wave Machine Demo}
%\end{minipage}
%}}}%
%BeginExpansion
\marginpar {
\hspace{-0.5in}
\begin{minipage}[t]{1in}
\small{Wave Machine Demo}
\end{minipage}
}%
%EndExpansion
%TCIMACRO{%
%\TeXButton{Question 223.5.5}{\marginpar {
%\hspace{-0.5in}
%\begin{minipage}[t]{1in}
%\small{Question 223.5.5}
%\end{minipage}
%}}}%
%BeginExpansion
\marginpar {
\hspace{-0.5in}
\begin{minipage}[t]{1in}
\small{Question 223.5.5}
\end{minipage}
}%
%EndExpansion

What happens if we have more than one wave propagating in a medium? You
probably experienced this as a child. Your parents made you take a bath. You
discovered that you could make waves with your arm. But chances are you have
two arms, and that you discovered you could make two waves, one with each
arm. And when the two waves met in the middle, the water left the bath tub!
What happened was that the two wave crests met in the same place and the
medium (water) piled up there. We call the combination of two waves in the
same medium \emph{superposition.} The word literally means putting one wave
on top of another. When we superimpose two waves, their wave functions
simply add.

\begin{definition}
Superposition: If two or more traveling waves are moving through a medium,
the resultant wave formed at any point is the algebraic sum of the values of
the individual wave forms.
\end{definition}

So if we have 
\begin{equation}
y_{1}\left( x,t\right)
\end{equation}%
and 
\begin{equation}
y_{2}\left( x,t\right)
\end{equation}%
both propagating on a string, then we would see a resultant wave%
\begin{equation}
y_{r}\left( x,t\right) =y_{1}\left( x,t\right) +y_{2}\left( x,t\right)
\end{equation}

This is a fantastically simple way for the universe to act!

Let's look at an example. let's add the top wave (red) to the middle wave
(green). We get the bottom wave (purple)\FRAME{dhF}{2.77in}{3.3788in}{0pt}{}{%
}{Figure}{\special{language "Scientific Word";type
"GRAPHIC";maintain-aspect-ratio TRUE;display "USEDEF";valid_file "T";width
2.77in;height 3.3788in;depth 0pt;original-width 7.2705in;original-height
8.8816in;cropleft "0";croptop "1";cropright "1";cropbottom "0";tempfilename
'LTUWCJ20.wmf';tempfile-properties "XPR";}}Of course we are adding these in
the snapshot view. So this is all done for just one instant of time.

Let's see how to do this. \FRAME{dhF}{2.7121in}{2.1715in}{0pt}{}{}{Figure}{%
\special{language "Scientific Word";type "GRAPHIC";maintain-aspect-ratio
TRUE;display "USEDEF";valid_file "T";width 2.7121in;height 2.1715in;depth
0pt;original-width 6.576in;original-height 5.2598in;cropleft "0";croptop
"1";cropright "1";cropbottom "0";tempfilename
'LTUWCJ21.wmf';tempfile-properties "XPR";}}As an example, start at $x=-2.$
In the figure, I drew a red bar to show the $y$ value at $x=-2$ for the red
curve. Likewise, I have a green bar sowing the value of $y$ at $x=-2$ for
the green wave. Note that this is negative. On the bottom graph, the bars
have been repeated, and we can see that the red bar minus the green bar
brings us to the value for the resulting wave at the point $x=-2.$ We need
to do this at every point along all the waves for this instant of time.

This is tedious by hand, so we won't generally do this calculation by hand.
But a computer can do it easily.

Note that this is really only true for \emph{linear} systems. Let's take the
example of a Slinky%
%TCIMACRO{\TeXButton{\texttrademark}{\texttrademark}}%
%BeginExpansion
\texttrademark%
%EndExpansion
. If we form two waves in the Slinky, they behave according to the
superposition principle most of the time. But suppose we make the amplitude
of the individual waves large. They may travel individually OK, but when the
amplitudes add we may overstretch the Slinky. Then it would never return to
it's original shape. The wave form would have to change. Such a wave is not
linear. There is a good rule of thumb for when waves are linear.

\begin{Note}
A wave is generally linear when its amplitude is much smaller than its
wavelength.
\end{Note}

\subsection{Consequences of superposition}

%TCIMACRO{%
%\TeXButton{Question 223.5.6}{\marginpar {
%\hspace{-0.5in}
%\begin{minipage}[t]{1in}
%\small{Question 223.5.6}
%\end{minipage}
%}}}%
%BeginExpansion
\marginpar {
\hspace{-0.5in}
\begin{minipage}[t]{1in}
\small{Question 223.5.6}
\end{minipage}
}%
%EndExpansion
%TCIMACRO{%
%\TeXButton{Question 223.5.7}{\marginpar {
%\hspace{-0.5in}
%\begin{minipage}[t]{1in}
%\small{Question 223.5.7}
%\end{minipage}
%}}}%
%BeginExpansion
\marginpar {
\hspace{-0.5in}
\begin{minipage}[t]{1in}
\small{Question 223.5.7}
\end{minipage}
}%
%EndExpansion
%TCIMACRO{%
%\TeXButton{Question 223.5.8}{\marginpar {
%\hspace{-0.5in}
%\begin{minipage}[t]{1in}
%\small{Question 223.5.8}
%\end{minipage}
%}}}%
%BeginExpansion
\marginpar {
\hspace{-0.5in}
\begin{minipage}[t]{1in}
\small{Question 223.5.8}
\end{minipage}
}%
%EndExpansion

Linear waves traveling in media can pass through each other without being
destroyed or altered!

\FRAME{dhFU}{1.7633in}{2.0817in}{0pt}{\Qcb{Constructive Interference (Public
Domain image by Inductiveload,
http://commons.wikimedia.org/wiki/File:Constructive\_interference.svg)}}{}{%
Figure}{\special{language "Scientific Word";type
"GRAPHIC";maintain-aspect-ratio TRUE;display "USEDEF";valid_file "T";width
1.7633in;height 2.0817in;depth 0pt;original-width 1.7678in;original-height
2.0924in;cropleft "0";croptop "1";cropright "1";cropbottom "0";tempfilename
'LX1R0V02.wmf';tempfile-properties "XPR";}}

Our wave on the string makes the string segments move in the $y$-direction.
Both waves do this. So putting the two waves together just makes the string
segments move more! There is a special name for what we observe

\begin{definition}
\emph{interference}: The combination of separate waves in the same region of
space to produce a resultant wave.
\end{definition}

We also have a special name for when the amplitude of the resultant wave
gets larger.

\begin{definition}
\emph{Constructive Interference}: interference between waves when the
displacements caused by the two waves are in the same direction
\end{definition}

What happens if one of the pulses is inverted?\FRAME{dhFU}{1.7828in}{2.2033in%
}{0pt}{\Qcb{Destructive Interference (Public Domain image by Inductiveload,
http://commons.wikimedia.org/wiki/File:Destructive\_interference1.svg)}}{}{%
Figure}{\special{language "Scientific Word";type
"GRAPHIC";maintain-aspect-ratio TRUE;display "USEDEF";valid_file "T";width
1.7828in;height 2.2033in;depth 0pt;original-width 1.5833in;original-height
1.9638in;cropleft "0";croptop "1";cropright "1";cropbottom "0";tempfilename
'LX1R4P03.wmf';tempfile-properties "XPR";}}

When the two pulses meet, they \textquotedblleft cancel each other
out.\textquotedblright\ But do they go away? No! the energy is still there,
the string segment motions have just summed vectorially to zero, the energy
carried by each wave is still there in the stretched string. Because we
momentarily seem to destroy the wave pulses, we call this type of
interference \textquotedblleft destructive interference.\textquotedblright

\begin{definition}
\emph{Destructive Interference}:\ Interference between waves when the
displacements caused by the two waves are opposite in direction
\end{definition}

\section{Superposition and Doppler: Shock waves}

%TCIMACRO{%
%\TeXButton{Question 223.5.4}{\marginpar {
%\hspace{-0.5in}
%\begin{minipage}[t]{1in}
%\small{Question 223.5.4}
%\end{minipage}
%}}}%
%BeginExpansion
\marginpar {
\hspace{-0.5in}
\begin{minipage}[t]{1in}
\small{Question 223.5.4}
\end{minipage}
}%
%EndExpansion
What happens when the speed of the source is greater than the wave speed?

Remember that the wave speed depends only on the medium. Let's call the
crests of a wave the \emph{wave front}. In the picture below, a point source
is generating a wave and the red lines are the wave fronts.

When $v_{s}=v_{sound}$ the waves superimpose. They begin to pile up. If we
allow $v_{s}>v_{sound}$ then the wave fronts are no longer generated within
each other. \FRAME{dtbpF}{1.9917in}{1.9804in}{0pt}{}{}{Figure}{\special%
{language "Scientific Word";type "GRAPHIC";maintain-aspect-ratio
TRUE;display "USEDEF";valid_file "T";width 1.9917in;height 1.9804in;depth
0pt;original-width 5.5179in;original-height 5.486in;cropleft "0";croptop
"1";cropright "1";cropbottom "0";tempfilename
'Sound_Waves/Dopler_4.wmf';tempfile-properties "XNPR";}}The leading edge of
the wave fronts superimpose\ to form a cone shape. The half angle of this
cone is called the \emph{Mach angle} 
\begin{equation}
\sin \theta =\frac{vt}{v_{s}t}=\frac{v}{v_{s}}
\end{equation}%
This ration $v/v_{s}$ is called the Mach number and the conical wave front
is called a shock wave. We see them often in water

\FRAME{dhFU}{1.9558in}{1.3402in}{0pt}{\Qcb{Boat wakes as a Doppler cone.
Image courtesy US\ Navy. Image is in the Public Domain.}}{}{Figure}{\special%
{language "Scientific Word";type "GRAPHIC";maintain-aspect-ratio
TRUE;display "USEDEF";valid_file "T";width 1.9558in;height 1.3402in;depth
0pt;original-width 1.9182in;original-height 1.305in;cropleft "0";croptop
"1";cropright "1";cropbottom "0";tempfilename
'MM0U9K00.wmf';tempfile-properties "XPR";}}

and hear them when jet aircraft go supersonic. 
%TCIMACRO{%
%\TeXButton{Dopler Movie}{\marginpar {
%\hspace{-0.5in}
%\begin{minipage}[t]{1in}
%\small{Dopler Movie}
%\end{minipage}
%}} }%
%BeginExpansion
\marginpar {
\hspace{-0.5in}
\begin{minipage}[t]{1in}
\small{Dopler Movie}
\end{minipage}
}
%EndExpansion
In the next figure we can see a picture of a T-38 breaking the sound
barrier. You can see the Mach cones, but notice that there are several!
Remember that a disturbance creates a wave. There are disturbances created
by the nose of the plane, the rudder, and the wings, and perhaps the cockpit
in this Schlieren photograph.\FRAME{dhFU}{1.593in}{1.2842in}{0pt}{\Qcb{Dr.
Leonard Weinstein's Schlieren photograph of a T-38 Talon at Mach 1.1,
altitude 13,700 feet, taken at NASA Langley Research Center, Wallops in
1993. Image Courtacy NASA, image is in the Public Domain.}}{}{Figure}{%
\special{language "Scientific Word";type "GRAPHIC";maintain-aspect-ratio
TRUE;display "USEDEF";valid_file "T";width 1.593in;height 1.2842in;depth
0pt;original-width 1.5575in;original-height 1.2488in;cropleft "0";croptop
"1";cropright "1";cropbottom "0";tempfilename
'MM0U9K01.wmf';tempfile-properties "XPR";}}

\section{Importance of superposition}

The combination of waves is important for both scientists and engineers. In
engineering this is the hart of vibrometry. \FRAME{dhFU}{3.5639in}{2.3851in}{%
0pt}{\Qcb{Marshall and Cal Poly technicians wired the NanoSail-D spacecraft
to accelerometers, instruments which measure vibration response during
simulated launch conditions. Image couracy NASA, image in the Public Domain.}%
}{}{Figure}{\special{language "Scientific Word";type
"GRAPHIC";maintain-aspect-ratio TRUE;display "USEDEF";valid_file "T";width
3.5639in;height 2.3851in;depth 0pt;original-width 3.5172in;original-height
2.3454in;cropleft "0";croptop "1";cropright "1";cropbottom "0";tempfilename
'LXOZU509.wmf';tempfile-properties "XPR";}}Mechanical systems have moving
parts. These moving parts can be the disturbance that creates a wave. If
more than one wave crest arrives at a location in the device, the amplitude
at that location could become large. The oscillation of this part of the
device could rattle apart welds or bolts, destroying the device. Later, as
we study spectroscopy, we will see how to diagnose such a problem and hint
at how to correct it.

\chapter{Standing Waves}

A special case of superposition is that of two waves of the same frequency
traveling opposite directions. Mixing two such waves can give rise to
resonant patterns. These resonant patterns are the basis of music, and are
of concern in building structures, among other things.

%TCIMACRO{%
%\TeXButton{Fundamental Concepts}{\hspace{-1.3in}{\LARGE Fundamental Concepts\vspace{0.25in}}}}%
%BeginExpansion
\hspace{-1.3in}{\LARGE Fundamental Concepts\vspace{0.25in}}%
%EndExpansion

\begin{itemize}
\item When a wave meets a boundary, it will reflect

\item Reflected waves will invert if the boundary is fixed or more like a
fixed boundary.

\item Reflected waves will not invert if the boundary is free or more like a
free boundary.

\item Two waves of equal frequency but traveling opposite directions can
cause resonant patterns called standing waves.

\item Only certain frequencies will produce standing waves. The boundary
conditions determine which frequencies will work.

\item The series of frequencies that produce standing waves is called the
harmonic series.
\end{itemize}

\section{Mathematical Description of Superposition}

We know what superposition is, but we don't really want to add values for
millions of points in a medium to find out what a combination of waves will
look like. At the very least, we want to make a computer do that (and
programs like OpenFoam do something very akin to this!). But where we can,
we would like to combine wave functions algebraically. Let's see how this
can work.

Let's define two wave functions%
\begin{equation*}
y_{1}=A\sin \left( kx-\omega t\right)
\end{equation*}%
and 
\begin{equation*}
y_{2}=A\sin \left( kx-\omega t+\phi _{o}\right)
\end{equation*}%
These are two waves with the same frequency and wave number traveling the
same direction in the medium, but they start at different times. The graph
of $y_{2}$ is shifted by an amount $\phi _{o}.$

I will pick some values for the constants

\begin{equation*}
\begin{tabular}{l}
$\lambda =2$ \\ 
$k=\frac{2\pi }{\lambda }$ \\ 
$\omega =1$ \\ 
$\phi _{o}=\frac{\pi }{6}$ \\ 
$t=0$ \\ 
$A=1$%
\end{tabular}%
\end{equation*}%
then for $y_{1}$ we have

\begin{eqnarray*}
y_{1} &=&\left( 1\right) \sin \left( \frac{2\pi }{\lambda }x-\left( 1\right)
t\right) \\
&=&\sin \left( \frac{2\pi }{2}x-\left( 1\right) t\right) \\
&=&\sin \left( \pi x-t\right)
\end{eqnarray*}%
here is a plot of the wave function, $y_{1}$ \FRAME{dtbpF}{2.9689in}{1.2816in%
}{0pt}{}{}{Plot}{\special{language "Scientific Word";type "MAPLEPLOT";width
2.9689in;height 1.2816in;depth 0pt;display "USEDEF";plot_snapshots
TRUE;mustRecompute FALSE;lastEngine "MuPAD";xmin "0";xmax
"5.001000";xviewmin "0";xviewmax "5.001000";yviewmin "-2";yviewmax
"2";viewset"XY";rangeset"X";plottype 4;axesFont "Times New
Roman,12,0000000000,useDefault,normal";numpoints 100;plotstyle
"patch";axesstyle "normal";axestips FALSE;xis \TEXUX{x};var1name
\TEXUX{$x$};function \TEXUX{$\allowbreak \sin \pi x$};linecolor
"green";linestyle 1;pointstyle "point";linethickness 3;lineAttributes
"Solid";var1range "0,5.001000";num-x-gridlines 100;curveColor
"[flat::RGB:0x00006000]";curveStyle "Line";VCamFile
'MAJUKH02.xvz';valid_file "T";tempfilename
'MAJUKH00.wmf';tempfile-properties "XPR";}}

Now let's consider $y_{2.}$ Using the values we chose, $y_{2}$ can be
written as%
\begin{eqnarray*}
y_{2} &=&A\sin \left( kx-\omega t+\phi _{o}\right) \\
&=&\sin \left( \pi x-t+\frac{\pi }{6}\right)
\end{eqnarray*}%
which looks like this\FRAME{dtbpF}{3.0519in}{1.2427in}{0pt}{}{}{Plot}{%
\special{language "Scientific Word";type "MAPLEPLOT";width 3.0519in;height
1.2427in;depth 0pt;display "USEDEF";plot_snapshots TRUE;mustRecompute
FALSE;lastEngine "MuPAD";xmin "0";xmax "5.001000";xviewmin "0";xviewmax
"5.001000";yviewmin "-2";yviewmax "2";viewset"XY";rangeset"X";plottype
4;axesFont "Times New Roman,12,0000000000,useDefault,normal";numpoints
100;plotstyle "patch";axesstyle "normal";axestips FALSE;xis
\TEXUX{x};var1name \TEXUX{$x$};function \TEXUX{$\sin \left( \frac{1}{6}\pi
+\pi x\right) $};linecolor "red";linestyle 1;pointstyle
"point";linethickness 3;lineAttributes "Solid";var1range
"0,5.001000";num-x-gridlines 100;curveColor
"[flat::RGB:0x00ff0000]";curveStyle "Line";VCamFile
'MAJULJ03.xvz';valid_file "T";tempfilename
'MAJULJ01.wmf';tempfile-properties "XPR";}}What does it look like if we add
these waves using superposition? Symbolically we have%
\begin{equation}
y_{r}=A\sin \left( kx-\omega t\right) +A\sin \left( kx-\omega t+\phi
_{o}\right)  \label{Superposition basic}
\end{equation}

and putting in the numbers gives%
\begin{equation*}
y_{r}=\sin \left( \pi x-t\right) +\sin \left( \pi x-t+\frac{\pi }{6}\right)
\end{equation*}%
which is shown in the next graph.\FRAME{dtbpF}{3.0727in}{1.3292in}{0pt}{}{}{%
Plot}{\special{language "Scientific Word";type "MAPLEPLOT";width
3.0727in;height 1.3292in;depth 0pt;display "USEDEF";plot_snapshots
TRUE;mustRecompute FALSE;lastEngine "MuPAD";xmin "0";xmax
"5.001000";xviewmin "0";xviewmax "5.001000";yviewmin "-2";yviewmax
"2";viewset"XY";rangeset"X";plottype 4;axesFont "Times New
Roman,12,0000000000,useDefault,normal";numpoints 100;plotstyle
"patch";axesstyle "normal";axestips FALSE;xis \TEXUX{x};var1name
\TEXUX{$x$};function \TEXUX{$\sin \left( \frac{1}{6}\pi +\pi x\right) +\sin
\pi x$};linecolor "green";linestyle 1;pointstyle "point";linethickness
3;lineAttributes "Solid";var1range "0,5.001000";num-x-gridlines
100;curveColor "[flat::RGB:0x00008000]";curveStyle "Line";VCamFile
'MAJUMB05.xvz';valid_file "T";tempfilename
'MAJUMB02.wmf';tempfile-properties "XPR";}}Notice that the wave form is
taller (larger amplitude). Also notice it is shifted along the $x$ axis.

We can find out how much by rewriting $y_{r}.$ We want to rewrite equation (%
\ref{Superposition basic}) so it is easier to interpret. To do this we need
to remember a trig identity%
\begin{equation*}
\sin a+\sin b=2\cos \left( \frac{a-b}{2}\right) \sin \left( \frac{a+b}{2}%
\right)
\end{equation*}%
Then, for our case, let $a=kx-\omega t$ and $b=kx-\omega t+\phi _{o}$. This
lets us rewrite our resultant wave.%
\begin{eqnarray*}
y_{r} &=&A\sin \left( kx-\omega t\right) +A\sin \left( kx-\omega t+\phi
_{o}\right) \\
&=&2A\cos \left( \frac{\left( kx-\omega t\right) -\left( kx-\omega t+\phi
_{o}\right) }{2}\right) \sin \left( \frac{\left( kx-\omega t\right) +\left(
kx-\omega t+\phi _{o}\right) }{2}\right) \\
&=&2A\cos \left( \frac{-\phi _{o}}{2}\right) \sin \left( \frac{2kx-2\omega
t+\phi _{o}}{2}\right) \\
&=&2A\cos \left( \frac{-\phi _{o}}{2}\right) \sin \left( kx-\omega t+\frac{%
\phi _{o}}{2}\right) \\
&=&2A\cos \left( \frac{\phi _{o}}{2}\right) \sin \left( kx-\omega t+\frac{%
\phi _{o}}{2}\right)
\end{eqnarray*}

where we used the fact that $\cos \left( -\theta \right) =\cos \left( \theta
\right) .$

To interpret this new form of our resultant wave equation, let's look at the
parts of this expression. First take

\begin{equation}
\sin \left( kx-\omega t+\frac{\phi _{o}}{2}\right)
\end{equation}%
This part is a traveling wave with the same $k$ and $\omega $ as our
original waves, but it has a phase constant of $\phi _{o}/2.$ So our
combined wave is shifted by $\phi _{o}/2$ or half the phase shift of $y_{2}.$

Now let's look at other factor

\begin{equation}
2A\cos \left( \frac{\phi _{o}}{2}\right)
\end{equation}%
This part has no time dependence. We recognize from our basic equation%
\begin{equation*}
y\left( x,t\right) =A\sin \left( kx-\omega t+\phi _{o}\right)
\end{equation*}%
that the amplitude, $A$ is a constant--not dependent on $x$ or $t,$ that
multiplies the sine function. But now we have a more complex term that is
not dependent on $x$ or $t$ that multiplies the sine function. The whole
term must be the new amplitude! It has a maximum value when $\phi _{o}=0$

\FRAME{dtbpFX}{2.5313in}{1.062in}{0pt}{}{}{Plot}{\special{language
"Scientific Word";type "MAPLEPLOT";width 2.5313in;height 1.062in;depth
0pt;display "USEDEF";plot_snapshots TRUE;mustRecompute FALSE;lastEngine
"MuPAD";xmin "-5";xmax "5";xviewmin "-5.0010000010002";xviewmax
"5.0010000010002";yviewmin "-1.60264739640757";yviewmax "2";plottype
4;axesFont "Times New Roman,12,0000000000,useDefault,normal";numpoints
100;plotstyle "patch";axesstyle "normal";axestips FALSE;xis
\TEXUX{x};var1name \TEXUX{$x$};function \TEXUX{$2\left( 1\right) \cos \left(
\frac{x}{2}\right) $};linecolor "blue";linestyle 1;pointstyle
"point";linethickness 1;lineAttributes "Solid";var1range
"-5,5";num-x-gridlines 100;curveColor "[flat::RGB:0x000000ff]";curveStyle
"Line";VCamFile 'LTUWDL22.xvz';valid_file "T";tempfilename
'LTUWCJ25.wmf';tempfile-properties "XPR";}}When $\phi _{o}=\pi ,$ then 
\begin{equation*}
2A\cos \left( \frac{\pi }{2}\right) =0
\end{equation*}%
so when $\phi _{o}=0$ we have a new maximum amplitude of twice the original
amplitude, $2A,$ and when $\phi _{o}=\pi $ we have no amplitude. Here is our
wave for several choices of $\phi _{o}.$

\FRAME{dtbpFX}{3.1254in}{1.177in}{0pt}{}{}{Plot}{\special{language
"Scientific Word";type "MAPLEPLOT";width 3.1254in;height 1.177in;depth
0pt;display "USEDEF";plot_snapshots TRUE;mustRecompute FALSE;lastEngine
"MuPAD";xmin "0";xmax "5.001000";xviewmin "0";xviewmax "5.001000";yviewmin
"-2";yviewmax "2";viewset"XY";rangeset"X";plottype 4;axesFont "Times New
Roman,12,0000000000,useDefault,normal";numpoints 100;plotstyle
"patch";axesstyle "normal";axestips FALSE;xis \TEXUX{x};var1name
\TEXUX{$x$};function \TEXUX{$\allowbreak 2\sin \left( \frac{1}{2}\left(
0\right) +\pi x\right) \cos \frac{1}{2}\left( 0\right) $};linecolor
"black";linestyle 1;pointstyle "point";linethickness 2;lineAttributes
"Solid";var1range "0,5.001000";num-x-gridlines 100;curveColor
"[flat::RGB:0000000000]";curveStyle "Line";function \TEXUX{$\allowbreak
2\sin \left( \frac{1}{2}\frac{\pi }{9}+\pi x\right) \cos
\frac{1}{2}\frac{\pi }{9}$};linecolor "blue";linestyle 1;pointstyle
"point";linethickness 2;lineAttributes "Solid";var1range
"0,5.001000";num-x-gridlines 100;curveColor
"[flat::RGB:0x00000080]";curveStyle "Line";function \TEXUX{$\allowbreak
2\sin \left( \frac{1}{2}\frac{2\pi }{9}+\pi x\right) \cos
\frac{1}{2}\frac{2\pi }{9}$};linecolor "blue";linestyle 1;pointstyle
"point";linethickness 2;lineAttributes "Solid";var1range
"0,5.001000";num-x-gridlines 100;curveColor
"[flat::RGB:0x000000ff]";curveStyle "Line";function \TEXUX{$\allowbreak
2\sin \left( \frac{1}{2}\frac{3\pi }{9}+\pi x\right) \cos
\frac{1}{2}\frac{3\pi }{9}$};linecolor "green";linestyle 1;pointstyle
"point";linethickness 2;lineAttributes "Solid";var1range
"0,5.001000";num-x-gridlines 100;curveColor
"[flat::RGB:0x00008000]";curveStyle "Line";function \TEXUX{$2\sin \left(
\frac{1}{2}\frac{4\pi }{9}+\pi x\right) \cos \frac{1}{2}\frac{4\pi
}{9}$};linecolor "cyan";linestyle 1;pointstyle "point";linethickness
2;lineAttributes "Solid";var1range "0,5.001000";num-x-gridlines
100;curveColor "[flat::RGB:0x00008080]";curveStyle "Line";function
\TEXUX{$\allowbreak 2\sin \left( \frac{1}{2}\frac{5\pi }{9}+\pi x\right)
\cos \frac{1}{2}\frac{5\pi }{9}$};linecolor "cyan";linestyle 1;pointstyle
"point";linethickness 2;lineAttributes "Solid";var1range
"0,5.001000";num-x-gridlines 100;curveColor
"[flat::RGB:0x000080c0]";curveStyle "Line";function \TEXUX{$2\sin \left(
\frac{1}{2}\frac{6\pi }{9}+\pi x\right) \cos \frac{1}{2}\frac{6\pi
}{9}$};linecolor "green";linestyle 1;pointstyle "point";linethickness
2;lineAttributes "Solid";var1range "0,5.001000";num-x-gridlines
100;curveColor "[flat::RGB:0x0000ff00]";curveStyle "Line";function
\TEXUX{$2\sin \left( \frac{1}{2}\frac{7\pi }{9}+\pi x\right) \cos
\frac{1}{2}\frac{7\pi }{9}$};linecolor "black";linestyle 1;pointstyle
"point";linethickness 2;lineAttributes "Solid";var1range
"0,5.001000";num-x-gridlines 100;curveColor
"[flat::RGB:0x00400000]";curveStyle "Line";function \TEXUX{$2\sin \left(
\frac{1}{2}\frac{8\pi }{9}+\pi x\right) \cos \frac{1}{2}\frac{8\pi
}{9}$};linecolor "black";linestyle 1;pointstyle "point";linethickness
2;lineAttributes "Solid";var1range "0,5.001000";num-x-gridlines
100;curveColor "[flat::RGB:0x00404040]";curveStyle "Line";function
\TEXUX{$2\sin \left( \frac{1}{2}\frac{9\pi }{9}+\pi x\right) \cos
\frac{1}{2}\frac{9\pi }{9}$};linecolor "gray";linestyle 1;pointstyle
"point";linethickness 2;lineAttributes "Solid";var1range
"0,5.001000";num-x-gridlines 100;curveColor
"[flat::RGB:0x00cd99ff]";curveStyle "Line";VCamFile
'LXP18O0A.xvz';valid_file "T";tempfilename
'LXP14R0A.wmf';tempfile-properties "XPR";}}We can see that in our case the
fact that the two waves added to produce a larger amplitude was just luck.
We could have gotten anything from twice the single wave amplitude to no
amplitude at all.

\section{Reflection and Transmission}

In our examples so far, we have not explained how we got two waves into a
medium. One way is to simply reflect one wave back on top of itself.

In class we made pulses on a long spring with one end of the spring fixed
(held by a class member). What happens when the pulse reaches the end of the
rope?%
%TCIMACRO{%
%\TeXButton{Spring Pulse Inversion Demo}{\marginpar {
%\hspace{-0.5in}
%\begin{minipage}[t]{1in}
%\small{Spring Pulse Inversion Demo}
%\end{minipage}
%}}}%
%BeginExpansion
\marginpar {
\hspace{-0.5in}
\begin{minipage}[t]{1in}
\small{Spring Pulse Inversion Demo}
\end{minipage}
}%
%EndExpansion

\subsection{Case I: Fixed rope end.}

%TCIMACRO{%
%\TeXButton{Question 223.6.1}{\marginpar {
%\hspace{-0.5in}
%\begin{minipage}[t]{1in}
%\small{Question 223.6.1}
%\end{minipage}
%}}}%
%BeginExpansion
\marginpar {
\hspace{-0.5in}
\begin{minipage}[t]{1in}
\small{Question 223.6.1}
\end{minipage}
}%
%EndExpansion
\FRAME{dhF}{3.6063in}{4.0456in}{0pt}{}{}{Figure}{\special{language
"Scientific Word";type "GRAPHIC";maintain-aspect-ratio TRUE;display
"USEDEF";valid_file "T";width 3.6063in;height 4.0456in;depth
0pt;original-width 3.5587in;original-height 3.9963in;cropleft "0";croptop
"1";cropright "1";cropbottom "0";tempfilename
'M2ZQ7S05.wmf';tempfile-properties "XPR";}}

There is a big change in the medium at the end of the rope--the rope ends.
This change in medium causes a reflection.

In the fixed end case, the pulse is inverted. We should consider why this
inversion happens.

\FRAME{dtbpF}{4.0101in}{2.3315in}{0in}{}{}{Figure}{\special{language
"Scientific Word";type "GRAPHIC";maintain-aspect-ratio TRUE;display
"USEDEF";valid_file "T";width 4.0101in;height 2.3315in;depth
0in;original-width 3.9617in;original-height 2.2917in;cropleft "0";croptop
"1";cropright "1";cropbottom "0";tempfilename
'MJ9ASQ05.wmf';tempfile-properties "XPR";}}

The end of the rope pushes up on the support (person, or nail or whatever).
By Newton's third law the support must push back in an equal, but opposite
direction, on the rope. This force sends the pieces of rope near it
downward. We could think of the squashed nail atoms as having been given an
amount of spring potential energy. They will transfer this energy back to
the rope by pushing the end of the rope down. This downward motion becomes a
new pulse that is an inversion of the original pulse traveling the opposite
direction.

%TCIMACRO{%
%\TeXButton{Popper demo}{\marginpar {
%\hspace{-0.5in}
%\begin{minipage}[t]{1in}
%\small{Popper demo}
%\end{minipage}
%}}}%
%BeginExpansion
\marginpar {
\hspace{-0.5in}
\begin{minipage}[t]{1in}
\small{Popper demo}
\end{minipage}
}%
%EndExpansion
Does this seem reasonable? Remember studying normal forces? Consider a book
on a table. The book has a force due to gravity. The table exerts a force
equal to $m_{book}g$ on the book, or else the book smashes through the table.%
\FRAME{dtbpF}{3.7516in}{1.6553in}{0pt}{}{}{Figure}{\special{language
"Scientific Word";type "GRAPHIC";maintain-aspect-ratio TRUE;display
"USEDEF";valid_file "T";width 3.7516in;height 1.6553in;depth
0pt;original-width 3.7031in;original-height 1.6189in;cropleft "0";croptop
"1";cropright "1";cropbottom "0";tempfilename
'MJ9M7H07.wmf';tempfile-properties "XPR";}}

The normal force exerted by the atoms of the table keep the book up. It is
this same type of force that keeps the rope on the nail. The atoms of the
nail must push down on the end of the rope. They exert a force and this
causes the inversion.

\subsection{Case II: Loose rope end.}

\FRAME{dhF}{4.0395in}{4.2419in}{0pt}{}{}{Figure}{\special{language
"Scientific Word";type "GRAPHIC";maintain-aspect-ratio TRUE;display
"USEDEF";valid_file "T";width 4.0395in;height 4.2419in;depth
0pt;original-width 3.9902in;original-height 4.1909in;cropleft "0";croptop
"1";cropright "1";cropbottom "0";tempfilename
'LTUWCK2B.wmf';tempfile-properties "XPR";}}

But what happens if the rope end is not fixed?

The rope end rises, and therefore there is no force exerted. The pulse (or
at least part of the pulse energy) is still reflected, but there is no
inversion!\FRAME{dtbpF}{3.9522in}{2.4293in}{0in}{}{}{Figure}{\special%
{language "Scientific Word";type "GRAPHIC";maintain-aspect-ratio
TRUE;display "USEDEF";valid_file "T";width 3.9522in;height 2.4293in;depth
0in;original-width 3.9029in;original-height 2.3886in;cropleft "0";croptop
"1";cropright "1";cropbottom "0";tempfilename
'MJ9MAJ08.wmf';tempfile-properties "XPR";}}The end of the rope will come
down, but the reason is that the force due to gravity acts on the mass of
the rope end. The energy of the wave was made into potential energy of the
rope end. As the rope end loses potential energy, that energy is put back
into a wave going the opposite direction.

\subsection{Case III: Partially attached rope end}

%TCIMACRO{%
%\TeXButton{Question 223.6.2}{\marginpar {
%\hspace{-0.5in}
%\begin{minipage}[t]{1in}
%\small{Question 223.6.2}
%\end{minipage}
%}}}%
%BeginExpansion
\marginpar {
\hspace{-0.5in}
\begin{minipage}[t]{1in}
\small{Question 223.6.2}
\end{minipage}
}%
%EndExpansion
%TCIMACRO{%
%\TeXButton{Question 223.6.3}{\marginpar {
%\hspace{-0.5in}
%\begin{minipage}[t]{1in}
%\small{Question 223.6.3}
%\end{minipage}
%}}}%
%BeginExpansion
\marginpar {
\hspace{-0.5in}
\begin{minipage}[t]{1in}
\small{Question 223.6.3}
\end{minipage}
}%
%EndExpansion
Now lets tie the rope to another rope that is larger, more dense, than the
rope we have been using, what will happen when we make waves in this
combined rope?

The light end of the rope exerts a force on the heavy beginning of the new
rope\FRAME{dtbpF}{4.446in}{2.495in}{0in}{}{}{Figure}{\special{language
"Scientific Word";type "GRAPHIC";maintain-aspect-ratio TRUE;display
"USEDEF";valid_file "T";width 4.446in;height 2.495in;depth
0in;original-width 4.395in;original-height 2.4535in;cropleft "0";croptop
"1";cropright "1";cropbottom "0";tempfilename
'MJ9MBC09.wmf';tempfile-properties "XPR";}}

In this case consider momentum\FRAME{dtbpF}{3.0104in}{1.0196in}{0in}{}{}{%
Figure}{\special{language "Scientific Word";type
"GRAPHIC";maintain-aspect-ratio TRUE;display "USEDEF";valid_file "T";width
3.0104in;height 1.0196in;depth 0in;original-width 2.9672in;original-height
0.9868in;cropleft "0";croptop "1";cropright "1";cropbottom "0";tempfilename
'MJ9MC80A.wmf';tempfile-properties "XPR";}}The heavier rope resists being
moved because of its larger mass. This resistance to motion is a little like
our fixed end case. It is harder to transfer energy to the heavy rope, and
the heavy rope resists the pull of the light rope. This resistance pushes
back on the end of the light rope. This is a downward push. So once again we
will have a reflected pulse in the light rope that is inverted. \FRAME{dtbpF%
}{1.7971in}{1.4105in}{0pt}{}{}{Figure}{\special{language "Scientific
Word";type "GRAPHIC";maintain-aspect-ratio TRUE;display "USEDEF";valid_file
"T";width 1.7971in;height 1.4105in;depth 0pt;original-width
1.7599in;original-height 1.375in;cropleft "0";croptop "1";cropright
"1";cropbottom "0";tempfilename 'M2ZQ3U04.wmf';tempfile-properties "XPR";}}%
We could also make a pulse in the heavy rope. What would happen then when
the pulse reached the interface? You might be able to guess that the light
rope won't have much effect on the end of the heavy rope. The light rope
will cause a reflection, but it's weak downward push is not enough to cause
the reflected pulse to invert. \FRAME{dtbpF}{1.8239in}{1.2505in}{0pt}{}{}{%
Figure}{\special{language "Scientific Word";type
"GRAPHIC";maintain-aspect-ratio TRUE;display "USEDEF";valid_file "T";width
1.8239in;height 1.2505in;depth 0pt;original-width 2.373in;original-height
1.6189in;cropleft "0";croptop "1";cropright "1";cropbottom "0";tempfilename
'M2ZPJ601.wmf';tempfile-properties "XPR";}}Going from a heavy rope to a
light rope makes an interface that is more like a free end.

Notice that in both cases there is a \emph{transmitted} pulse. The
transmitted pulse is what is left of the energy from the original pulse that
has not been reflected. So we would not expect it to be inverted, and,
indeed, it never is. We have split the amount of energy traveling along the
rope

\section{Mathematical description of standing waves}

Now that we have a way to make two waves to superimpose, we can study the
special case of a reflected wave. We will find that this special case can
produce interesting patterns of constructive and destructive interference. 
%TCIMACRO{%
%\TeXButton{Standing Wave Demo}{\marginpar {
%\hspace{-0.5in}
%\begin{minipage}[t]{1in}
%\small{Standing Wave Demo}
%\end{minipage}
%}}}%
%BeginExpansion
\marginpar {
\hspace{-0.5in}
\begin{minipage}[t]{1in}
\small{Standing Wave Demo}
\end{minipage}
}%
%EndExpansion

%TCIMACRO{%
%\TeXButton{Question 223.6.4}{\marginpar {
%\hspace{-0.5in}
%\begin{minipage}[t]{1in}
%\small{Question 223.6.4}
%\end{minipage}
%}}}%
%BeginExpansion
\marginpar {
\hspace{-0.5in}
\begin{minipage}[t]{1in}
\small{Question 223.6.4}
\end{minipage}
}%
%EndExpansion
The patterns of constructive and destructive interference are the result of
the superposition of two traveling waves with the same frequency going in
opposite directions. Let's start with two standing waves with the same phase
constant for simplicity.%
\begin{eqnarray*}
y_{1} &=&A\sin \left( kx-\omega t\right) \\
y_{2} &=&A\sin \left( kx+\omega t\right)
\end{eqnarray*}

The sum is 
\begin{equation*}
y=y_{1}+y_{2}=A\sin \left( kx-\omega t\right) +A\sin \left( kx+\omega
t\right)
\end{equation*}%
To gain insight into what these two waves produce, we use another of our
favorite trig identities%
\begin{equation*}
\sin \left( a\pm b\right) =\sin \left( a\right) \cos \left( b\right) \pm
\cos \left( a\right) \sin \left( b\right)
\end{equation*}%
to get%
\begin{eqnarray*}
y &=&A\sin \left( kx-\omega t\right) +A\sin \left( kx+\omega t\right) \\
&=&A\sin \left( kx\right) \cos \left( \omega t\right) -A\cos \left(
kx\right) \sin \left( \omega t\right) +A\sin \left( kx\right) \cos \left(
\omega t\right) +A\cos \left( kx\right) \sin \left( \omega t\right) \\
&=&2A\sin \left( kx\right) \cos \left( \omega t\right) \\
&=&\left( 2A\sin \left( kx\right) \right) \cos \left( \omega t\right)
\end{eqnarray*}%
This looks like the harmonic oscillator equation%
\begin{equation*}
y=A\cos \left( \omega t+\phi _{o}\right)
\end{equation*}%
with $\phi _{o}=0.$ The factor $2A\sin \left( kx\right) $ has no time
dependence, so it could be considered the amplitude of the harmonic
oscillator. But this is a very odd amplitude. It depends on position. That
is, we can view the rope as a set of harmonic oscillators who's amplitudes
are different for each value of $x.$

\FRAME{dhF}{2.6273in}{2.3851in}{0pt}{}{}{Figure}{\special{language
"Scientific Word";type "GRAPHIC";maintain-aspect-ratio TRUE;display
"USEDEF";valid_file "T";width 2.6273in;height 2.3851in;depth
0pt;original-width 2.5858in;original-height 2.3454in;cropleft "0";croptop
"1";cropright "1";cropbottom "0";tempfilename
'LXP1BX0B.wmf';tempfile-properties "XPR";}}But this is just what we see in
our standing wave! 
%TCIMACRO{%
%\TeXButton{Question 223.6.5}{\marginpar {
%\hspace{-0.5in}
%\begin{minipage}[t]{1in}
%\small{Question 223.6.5}
%\end{minipage}
%}}}%
%BeginExpansion
\marginpar {
\hspace{-0.5in}
\begin{minipage}[t]{1in}
\small{Question 223.6.5}
\end{minipage}
}%
%EndExpansion
%TCIMACRO{%
%\TeXButton{Question 223.6.6}{\marginpar {
%\hspace{-0.5in}
%\begin{minipage}[t]{1in}
%\small{Question 223.6.6}
%\end{minipage}
%}}}%
%BeginExpansion
\marginpar {
\hspace{-0.5in}
\begin{minipage}[t]{1in}
\small{Question 223.6.6}
\end{minipage}
}%
%EndExpansion
We can identify spots along the $x$ axis where the amplitude is always zero!
we will call these spots \emph{nodes.} These happen when $\sin \left(
kx\right) =0$ or when 
\begin{equation*}
kx=n\pi
\end{equation*}

By using 
\begin{equation*}
k=\frac{2\pi }{\lambda }
\end{equation*}%
we have%
\begin{eqnarray*}
\frac{2\pi }{\lambda }x &=&n\pi \\
\frac{2}{\lambda }x &=&n \\
x &=&n\frac{\lambda }{2}
\end{eqnarray*}

We can also find the places along $x$ where the amplitude will be largest.
this occurs when $\sin \left( kx\right) =1$ or when%
\begin{equation*}
kx=n\frac{\pi }{2}
\end{equation*}%
or%
\begin{eqnarray*}
\frac{2\pi }{\lambda }x &=&n\frac{\pi }{2} \\
x &=&n\frac{\lambda }{4}
\end{eqnarray*}%
these are called \emph{antinodes}.

This combination of two waves does not look like it goes anywhere. It seems
to \textquotedblleft stand\textquotedblright\ in place. We call it a \emph{%
standing wave}. We can also create standing waves with sound or even light
waves! But let's look at standing waves in some detail first.

\section{Standing Waves in a String Fixed at Both Ends}

\FRAME{dhF}{3.2846in}{2.4699in}{0pt}{}{}{Figure}{\special{language
"Scientific Word";type "GRAPHIC";maintain-aspect-ratio TRUE;display
"USEDEF";valid_file "T";width 3.2846in;height 2.4699in;depth
0pt;original-width 3.2396in;original-height 2.4284in;cropleft "0";croptop
"1";cropright "1";cropbottom "0";tempfilename
'LTUWCK2F.wmf';tempfile-properties "XPR";}}%
%TCIMACRO{%
%\TeXButton{Question 223.6.7}{\marginpar {
%\hspace{-0.5in}
%\begin{minipage}[t]{1in}
%\small{Question 223.6.7}
%\end{minipage}
%}}}%
%BeginExpansion
\marginpar {
\hspace{-0.5in}
\begin{minipage}[t]{1in}
\small{Question 223.6.7}
\end{minipage}
}%
%EndExpansion
If we attach a string to something on both ends, we find something
interesting in the standing wave pattern. Not all imaginable standing waves
can be realized. Some frequencies are preferred, and some never show up.
These non-preferred frequencies will make waves, but not standing waves. We
say that the standing wave pattern is \emph{quantized}, meaning that only
certain frequency values will make a standing wave pattern. The patterns
that are allowed are called \emph{normal modes}. We will see this any time a
wave confined by boundary conditions (light in a resonant cavity, radio
waves in a wave guide, electrons in an atom, etc.). In the last figure we
saw some standing waves on a ukulele string. But we can draw the standing
waves without the instrument.\FRAME{dhF}{2.0271in}{2.0237in}{0pt}{}{}{Figure%
}{\special{language "Scientific Word";type "GRAPHIC";maintain-aspect-ratio
TRUE;display "USEDEF";valid_file "T";width 2.0271in;height 2.0237in;depth
0pt;original-width 1.9882in;original-height 1.9847in;cropleft "0";croptop
"1";cropright "1";cropbottom "0";tempfilename
'LXP1CR0C.wmf';tempfile-properties "XPR";}}

The figure shows three normal modes for a string. Of course there are many
more.

We find which modes are allowed by first imposing the boundary condition
that each end must be a node. We start with 
\begin{equation*}
y=2A\sin \left( kx\right) \cos \left( \omega t\right)
\end{equation*}%
and recognize that we have one condition met because $y=0$ when $x=0.$ We
need $y=0$ when $x=L.$ That happens when%
\begin{equation*}
kL=n\pi
\end{equation*}%
I will write this as%
\begin{equation*}
k_{n}L=n\pi
\end{equation*}%
to indicate there are many values of $k$ that could make a standing wave
pattern. Solving this for $\lambda _{n}$ gives%
\begin{eqnarray*}
\frac{2\pi }{\lambda _{n}}L &=&n\pi \\
\frac{2L}{n} &=&\lambda _{n}
\end{eqnarray*}%
Let's see how this works, the first mode will have 
\begin{equation*}
\lambda _{1}=2L
\end{equation*}%
where $L$ is the length of the string. Looking at the figure, we can see
that this is true. The first normal mode has a length that is half the first
mode wavelength.

The second mode has three nodes (one on each end and one in the middle).
This gives%
\begin{equation*}
\lambda _{2}=L
\end{equation*}%
We can keep going, the third mode will have five nodes%
\begin{equation*}
\lambda _{3}=\frac{2L}{3}
\end{equation*}%
and so forth to give%
\begin{equation*}
\lambda _{n}=\frac{2L}{n}
\end{equation*}

We use our old friend%
\begin{equation*}
v=f\lambda
\end{equation*}%
to find the frequencies of the modes%
\begin{equation*}
f=\frac{v}{\lambda }
\end{equation*}

Thus%
\begin{equation*}
f_{1}=\frac{v}{\lambda _{1}}=\frac{v}{2L}
\end{equation*}%
or, in general%
\begin{eqnarray*}
f_{n} &=&\frac{v}{\lambda _{n}}=n\frac{v}{2L} \\
&=&\frac{n}{2L}v \\
&=&\frac{n}{2L}\sqrt{\frac{T}{\mu }}
\end{eqnarray*}%
The lowest frequency that works has a special name, the \emph{fundamental
frequency}. The higher frequencies are integer multiples of the fundamental.
When this happens we say that the frequencies form a \emph{harmonic series, }%
and the modes are called \emph{harmonics}.%
%TCIMACRO{%
%\TeXButton{Question 223.6.7}{\marginpar {
%\hspace{-0.5in}
%\begin{minipage}[t]{1in}
%\small{Question 223.6.7}
%\end{minipage}
%}}}%
%BeginExpansion
\marginpar {
\hspace{-0.5in}
\begin{minipage}[t]{1in}
\small{Question 223.6.7}
\end{minipage}
}%
%EndExpansion
%TCIMACRO{%
%\TeXButton{Question 223.6.9}{\marginpar {
%\hspace{-0.5in}
%\begin{minipage}[t]{1in}
%\small{Question 223.6.9}
%\end{minipage}
%}}}%
%BeginExpansion
\marginpar {
\hspace{-0.5in}
\begin{minipage}[t]{1in}
\small{Question 223.6.9}
\end{minipage}
}%
%EndExpansion

\subsection{Starting the waves}

So, suppose we do not have a vibrating blade to make the wave patterns for a
string that is fixed at both ends. Can we just pluck it to make it vibrate
on a natural frequency?

Yes! only the normal modes will be excited by the pluck, any other
frequencies will die out quickly (we won't show this mathematically in this
class). So the only allowed frequencies (the ones that will result from a
pluck) are the natural frequencies or harmonics. The frequency of waves on
the string is \emph{quantized}! That is, only some values are allowed. This
idea is the basis behind Quantum mechanics (which views light and even
matter as waves).

\subsection{Musical Strings}

So how do we get different notes on a guitar or Piano?%
\begin{equation}
f_{n}=\frac{n}{2L}\sqrt{\frac{T}{\mu }}
\end{equation}

A guitar uses tension to change the frequency or pitch (tuning) and length
of string (your fingers pressing on the strings) to change notes.

A Piano uses both tension and length of string (and mass per unit length as
well!). What do you expect and organ will do?

\chapter{Light and Sound Standing waves}

Reading Assignment 21.4, 21.5

%TCIMACRO{%
%\TeXButton{Fundamental Concepts}{\hspace{-1.3in}{\LARGE Fundamental Concepts\vspace{0.25in}}}}%
%BeginExpansion
\hspace{-1.3in}{\LARGE Fundamental Concepts\vspace{0.25in}}%
%EndExpansion

\begin{itemize}
\item The harmonic series expressed by a system experiencing standing waves
depends on the \emph{boundary conditions.}

\item The harmonic series for open pipes is different that the harmonic
series for a pipe closed on one end.

\item Energy persists in the waves that have the harmonic series frequencies
because of resonance.
\end{itemize}

\section{Sound Standing waves (music)}

Suppose we send a sound wave down a pipe. When the air molecules strike the
molecules next to them they end up being reflected back. This happens as the
wave goes down the pipe until the wave reaches the end of the pipe. Remember
that where the molecules bunch up, the pressure is higher.

%TCIMACRO{%
%\TeXButton{Question 223.7.1}{\marginpar {
%\hspace{-0.5in}
%\begin{minipage}[t]{1in}
%\small{Question 223.7.1}
%\end{minipage}
%}}}%
%BeginExpansion
\marginpar {
\hspace{-0.5in}
\begin{minipage}[t]{1in}
\small{Question 223.7.1}
\end{minipage}
}%
%EndExpansion
When we reach the end of the pipe the molecules can't bounce off the walls
of the pipe anymore. They travel out into the surrounding air. It is harder
to get the room pressure to change because molecules can come from any
direction to fill up a vacancy. This is an effective medium change, and
there will be some energy reflected back from this pipe-to-room interface.
The reflected wave can make a standing wave. This is the basis of wind
instruments. Let's repeat the analysis we did last time and find the
possible frequencies that can make a standing wave, but this time for a
sound wave in a pipe.

Take a pipe as shown in the next figures.\FRAME{dhF}{3.7403in}{2.2892in}{0pt%
}{}{}{Figure}{\special{language "Scientific Word";type
"GRAPHIC";maintain-aspect-ratio TRUE;display "USEDEF";valid_file "T";width
3.7403in;height 2.2892in;depth 0pt;original-width 5.5054in;original-height
3.3589in;cropleft "0";croptop "1";cropright "1";cropbottom "0";tempfilename
'MALPPH00.wmf';tempfile-properties "XPR";}}

If we have a pipe open at both ends, we can see that air molecules are free
to move in and out of the ends of the pipe. If the air molecules can move,
the ends must not be nodes. This is different than the string case we
studied last time! We expect that there must be a node somewhere. We can
reasonably guess that there will be a node in the middle of the pipe due to
symmetry. Of course, the pressure on both ends must be atmospheric pressure.
So, remembering that pressure and displacement are $90\unit{%
%TCIMACRO{\U{b0}}%
%BeginExpansion
{{}^\circ}%
%EndExpansion
}$ out of phase for sound waves, we can guess that there are pressure nodes
on both ends.

For the first harmonic we can draw a displacement node in the middle and we
see that 
\begin{equation*}
\lambda _{1}=2L
\end{equation*}%
It takes two lengths of pipe to have the same length as the wavelength that
is in our one single pipe. Of course, our wave is hanging out of our pipe.
But if we set two additional pipes of length $L$ along side our pipe, these
two pipes would be the same length as the wavelength. The frequency would
be. 
\begin{equation}
f_{1}=\frac{v}{2L}
\end{equation}%
The next mode fits a whole wavelength%
\begin{eqnarray*}
\lambda _{2} &=&L \\
f_{2} &=&\frac{v}{L}
\end{eqnarray*}%
\FRAME{dhF}{3.0597in}{2.7786in}{0pt}{}{}{Figure}{\special{language
"Scientific Word";type "GRAPHIC";maintain-aspect-ratio TRUE;display
"USEDEF";valid_file "T";width 3.0597in;height 2.7786in;depth
0pt;original-width 4.4348in;original-height 4.0248in;cropleft "0";croptop
"1";cropright "1";cropbottom "0";tempfilename
'LTUWCK2H.wmf';tempfile-properties "XPR";}}but the next mode fits a
wavelength and a half%
\begin{eqnarray*}
\lambda _{3} &=&\frac{2}{3}L \\
f_{3} &=&\frac{3v}{2L}
\end{eqnarray*}%
If we keep going 
\begin{eqnarray}
\lambda _{n} &=&\frac{2}{n}L \\
f_{n} &=&n\frac{v}{2L}\qquad n=1,2,3,4\ldots
\end{eqnarray}%
This is the same mathematical form that we achieved for a standing wave on a
string!%
%TCIMACRO{%
%\TeXButton{Boom Whacker and Length}{\marginpar {
%\hspace{-0.5in}
%\begin{minipage}[t]{1in}
%\small{Boom Whacker and Length}
%\end{minipage}
%}}}%
%BeginExpansion
\marginpar {
\hspace{-0.5in}
\begin{minipage}[t]{1in}
\small{Boom Whacker and Length}
\end{minipage}
}%
%EndExpansion

\subsubsection{Pipes closed on one end}

%TCIMACRO{%
%\TeXButton{Question 223.7.2}{\marginpar {
%\hspace{-0.5in}
%\begin{minipage}[t]{1in}
%\small{Question 223.7.2}
%\end{minipage}
%}}}%
%BeginExpansion
\marginpar {
\hspace{-0.5in}
\begin{minipage}[t]{1in}
\small{Question 223.7.2}
\end{minipage}
}%
%EndExpansion
%TCIMACRO{%
%\TeXButton{Boom Whacker Demo}{\marginpar {
%\hspace{-0.5in}
%\begin{minipage}[t]{1in}
%\small{Boom Whacker Demo}
%\end{minipage}
%}}}%
%BeginExpansion
\marginpar {
\hspace{-0.5in}
\begin{minipage}[t]{1in}
\small{Boom Whacker Demo}
\end{minipage}
}%
%EndExpansion
But what happens if we put a cap on one end of the pipe? The air molecules
cannot move longitudinally once they hit the end. This must be a
displacement node. So then it must also be a pressure anti-node.

The open end is a pressure node because it stays at atmospheric pressure.
This is just the same as the open ends in the open pipe case we did before.
The simplest possible standing wave is shown below. \FRAME{dhF}{4.3474in}{%
2.8876in}{0pt}{}{}{Figure}{\special{language "Scientific Word";type
"GRAPHIC";maintain-aspect-ratio TRUE;display "USEDEF";valid_file "T";width
4.3474in;height 2.8876in;depth 0pt;original-width 4.2964in;original-height
2.8444in;cropleft "0";croptop "1";cropright "1";cropbottom "0";tempfilename
'LTUWCK2I.wmf';tempfile-properties "XPR";}}In the next figure we draw the
first few harmonics for this case.\FRAME{dhF}{4.1796in}{2.0358in}{0pt}{}{}{%
Figure}{\special{language "Scientific Word";type
"GRAPHIC";maintain-aspect-ratio TRUE;display "USEDEF";valid_file "T";width
4.1796in;height 2.0358in;depth 0pt;original-width 6.3252in;original-height
3.0666in;cropleft "0";croptop "1";cropright "1";cropbottom "0";tempfilename
'LTUWCK2J.wmf';tempfile-properties "XPR";}}

The first harmonic for the closed pipe are found by using 
\begin{equation*}
v=\lambda f
\end{equation*}%
\begin{equation*}
f=\frac{v}{\lambda }
\end{equation*}%
just as we did for the string and open pipe cases. We know the speed of
sound, so we have $v.$ Knowing that the first harmonic has a node at one end
and an anti node at the other end gives us the wavelength. If the pipe is $L$
in length, then $L$ must be 
\begin{equation*}
L=\frac{1}{4}\lambda _{1}
\end{equation*}%
or%
\begin{equation*}
\lambda _{1}=4L
\end{equation*}%
We see it now takes four lengths of of pipe to be the same size as the wave
that is in our single pipe! Then the frequency is given by%
\begin{equation*}
f_{1}=\frac{v}{\lambda _{1}}=\frac{v}{4L}
\end{equation*}

The next configuration that will have a node on one end and an antinode on
the other will have 
\begin{equation*}
L=\frac{3}{4}\lambda _{2}
\end{equation*}%
which gives%
\begin{equation*}
\lambda _{2}=\frac{4}{3L}
\end{equation*}%
and%
\begin{equation*}
f_{2}=\frac{v}{\lambda _{2}}=\frac{3v}{4L}
\end{equation*}

If we continued, we would find%
\begin{equation}
\lambda _{n}=\frac{4}{n}L
\end{equation}%
and 
\begin{equation}
f_{n}=n\frac{v}{4L}\qquad n=1,3,5\ldots
\end{equation}

This is different from the string and open pipe cases. Note that only odd
values of $n$ make a standing wave. Changing the end condition changed which
frequencies would make standing waves.

\subsection{Example: organ pipe}

\FRAME{dhF}{1.6907in}{1.4088in}{0in}{}{}{Figure}{\special{language
"Scientific Word";type "GRAPHIC";maintain-aspect-ratio TRUE;display
"USEDEF";valid_file "T";width 1.6907in;height 1.4088in;depth
0in;original-width 1.6544in;original-height 1.3742in;cropleft "0";croptop
"1";cropright "1";cropbottom "0";tempfilename
'LTUWCK2K.wmf';tempfile-properties "XPR";}}

%TCIMACRO{%
%\TeXButton{Organ Pipe Demo}{\marginpar {
%\hspace{-0.5in}
%\begin{minipage}[t]{1in}
%\small{Organ Pipe Demo}
%\end{minipage}
%}}}%
%BeginExpansion
\marginpar {
\hspace{-0.5in}
\begin{minipage}[t]{1in}
\small{Organ Pipe Demo}
\end{minipage}
}%
%EndExpansion
The organ pipe shown is closed at one end so we expect%
\begin{equation}
f_{n}=n\frac{v}{4L}\qquad n=1,3,5\ldots
\end{equation}
Measuring the pipe, and assuming about $20\unit{%
%TCIMACRO{\U{2103}}%
%BeginExpansion
{}^{\circ}{\rm C}%
%EndExpansion
}$ for the room temperature we have 
\begin{equation}
\begin{tabular}{l}
$L=0.41\unit{m}$ \\ 
$R=0.06\unit{m}$ \\ 
$v=343\frac{\unit{m}}{\unit{s}}$%
\end{tabular}%
\end{equation}%
There is a detail we have ignored in our analysis, the width of the pipe
matters a little. I will include a fudge factor to account for this. With
the fudge factor, the wavelength is 
\begin{eqnarray*}
\lambda _{1} &=&4\left( L+0.6R\right) \\
&=&178.\,\allowbreak 4\unit{cm}
\end{eqnarray*}%
then our fundamental frequency is 
\begin{eqnarray}
f_{1} &=&\frac{v}{\lambda _{1}} \\
&=&192.\,\allowbreak 26
\end{eqnarray}

We can identify this note, and compare to a standard, like a tuning fork or
a piano to verify our prediction.\FRAME{dhF}{4.6345in}{1.8844in}{0pt}{}{}{%
Figure}{\special{language "Scientific Word";type
"GRAPHIC";maintain-aspect-ratio TRUE;display "USEDEF";valid_file "T";width
4.6345in;height 1.8844in;depth 0pt;original-width 6.3529in;original-height
2.5676in;cropleft "0";croptop "1";cropright "1";cropbottom "0";tempfilename
'LTUWCK2L.wmf';tempfile-properties "XPR";}}

\section{Lasers and standing waves}

Light is a wave. Can we make a standing wave with light? The answer is yes,
and surprisingly we do it all the time. A laser creates a standing wave as
part of it's amplification system. Here is a laser just getting started. 
\FRAME{dhF}{2.6135in}{1.6604in}{0pt}{}{}{Figure}{\special{language
"Scientific Word";type "GRAPHIC";maintain-aspect-ratio TRUE;display
"USEDEF";valid_file "T";width 2.6135in;height 1.6604in;depth
0pt;original-width 2.5719in;original-height 1.6233in;cropleft "0";croptop
"1";cropright "1";cropbottom "0";tempfilename
'LTUWCK2M.wmf';tempfile-properties "XPR";}}A flash of light from a flash
bulb send light out in all directions. But some of the light goes to the
right toward a mirror. This is the light that will eventually become our
laser beam. This light is reflected off of the mirror. The wave inverts and
travels back along the center of the laser. Because it is inverted, it can
cause destructive interference in places along the laser cavity. But this
only works if we have just the right frequency of light. The light has to
fit an integer number of half wavelengths between the two mirrors for the
standing wave to form.\FRAME{dhF}{3.1868in}{1.6466in}{0pt}{}{}{Figure}{%
\special{language "Scientific Word";type "GRAPHIC";maintain-aspect-ratio
TRUE;display "USEDEF";valid_file "T";width 3.1868in;height 1.6466in;depth
0pt;original-width 3.1427in;original-height 1.6103in;cropleft "0";croptop
"1";cropright "1";cropbottom "0";tempfilename
'LTUWCL2N.wmf';tempfile-properties "XPR";}}

So only certain frequencies will work. That is why lasers usually only have
one color, different frequencies of light give us different colors. So a red
laser has a frequency of about $4.\,\allowbreak 762\times 10^{14}\unit{Hz}.$
We would expect another frequency to work that is twice this fundamental
frequency. 
\begin{eqnarray*}
f_{2} &=&2f_{1} \\
&=&2\times 762\times 10^{14}\unit{Hz} \\
&=&\allowbreak 152\,4\times 10^{14}\unit{Hz}
\end{eqnarray*}%
But this frequency is outside the visible range, so we can't see it and
chances are it won't go through the glass mirrors. So lasers usually only
produce one frequency of light. But gas lasers can be built with special
mirrors that allow many harmonics to be produced at once (e.g. CO$_{2}$
lasers).

The laser has an additional complication, and that is that it amplifies the
light with a laser medium. That medium gives a new photon for every photon
that passes through it, doubling the amount of light each time the wave
passes through this \emph{gain medium}. How that works is a subject for
PH279. But for us, we can see that we can make standing waves in light.

%TCIMACRO{%
%\TeXButton{Question 223.7.3}{\marginpar {
%\hspace{-0.5in}
%\begin{minipage}[t]{1in}
%\small{Question 223.7.3}
%\end{minipage}
%}}}%
%BeginExpansion
\marginpar {
\hspace{-0.5in}
\begin{minipage}[t]{1in}
\small{Question 223.7.3}
\end{minipage}
}%
%EndExpansion
%TCIMACRO{%
%\TeXButton{Question 223.7.4}{\marginpar {
%\hspace{-0.5in}
%\begin{minipage}[t]{1in}
%\small{Question 223.7.4}
%\end{minipage}
%}}}%
%BeginExpansion
\marginpar {
\hspace{-0.5in}
\begin{minipage}[t]{1in}
\small{Question 223.7.4}
\end{minipage}
}%
%EndExpansion

\section{Standing Waves in Rods and Membranes}

%TCIMACRO{%
%\TeXButton{singing Rod Demo}{\marginpar {
%\hspace{-0.5in}
%\begin{minipage}[t]{1in}
%\small{singing Rod Demo}
%\end{minipage}
%}}}%
%BeginExpansion
\marginpar {
\hspace{-0.5in}
\begin{minipage}[t]{1in}
\small{singing Rod Demo}
\end{minipage}
}%
%EndExpansion
We have hinted all chapter that the analysis techniques we were building
apply to structures. We need more math and computational tools to analyze
complex structures like bridges and buildings, but we can tackle a simple
structure like a rod that is clamped. The atoms in the rod can vibrate
longitudinally \FRAME{ftbpF}{4.1554in}{1.1926in}{0pt}{}{}{Figure}{\special%
{language "Scientific Word";type "GRAPHIC";maintain-aspect-ratio
TRUE;display "USEDEF";valid_file "T";width 4.1554in;height 1.1926in;depth
0pt;original-width 10.2704in;original-height 2.9199in;cropleft "0";croptop
"1";cropright "1";cropbottom "0";tempfilename
'Superposition_and_Standing_Waves/clamped_rod.wmf';tempfile-properties
"XNPR";}}Since we have motion possible on both ends and not in the middle,
we surmise that this system will have similar solutions as did the open
ended pipe.

\begin{equation*}
f_{n}=n\frac{v}{2L}
\end{equation*}%
The fundamental looks like

\FRAME{dtbpFX}{2.3443in}{1.5638in}{0pt}{}{}{Plot}{\special{language
"Scientific Word";type "MAPLEPLOT";width 2.3443in;height 1.5638in;depth
0pt;display "USEDEF";plot_snapshots TRUE;mustRecompute FALSE;lastEngine
"MuPAD";xmin "-1.5708";xmax "1.5708";xviewmin "-1.5708";xviewmax
"1.5708";yviewmin "-1.2";yviewmax "1.2";viewset"XY";rangeset"X";plottype
4;plottickdisable TRUE;axesFont "Times New
Roman,12,0000000000,useDefault,normal";numpoints 100;plotstyle
"patch";axesstyle "normal";axestips FALSE;xis \TEXUX{x};var1name
\TEXUX{$x$};function \TEXUX{$\sin \left( x\right) $};linecolor
"blue";linestyle 1;pointstyle "point";linethickness 2;lineAttributes
"Solid";var1range "-1.5708,1.5708";num-x-gridlines 100;curveColor
"[flat::RGB:0x000000ff]";curveStyle "Line";function \TEXUX{$-\sin \left(
x\right) $};linecolor "blue";linestyle 1;pointstyle "point";linethickness
2;lineAttributes "Solid";var1range "-1.5708,1.5708";num-x-gridlines
100;curveColor "[flat::RGB:0x000000ff]";curveStyle "Line";VCamFile
'LTUWDL1Z.xvz';valid_file "T";tempfilename
'LTUWCL2O.wmf';tempfile-properties "XPR";}}

But suppose we move the clamp. The clamp forces a node where it is placed.
If we place the clap at $L/4$

\FRAME{ftbpF}{3.7887in}{1.0879in}{0in}{}{}{Figure}{\special{language
"Scientific Word";type "GRAPHIC";maintain-aspect-ratio TRUE;display
"USEDEF";valid_file "T";width 3.7887in;height 1.0879in;depth
0in;original-width 10.2704in;original-height 2.9199in;cropleft "0";croptop
"1";cropright "1";cropbottom "0";tempfilename
'Superposition_and_Standing_Waves/clamped_rod_again.wmf';tempfile-properties
"XNPR";}}

\FRAME{dtbpFX}{2.3443in}{1.5638in}{0pt}{}{}{Plot}{\special{language
"Scientific Word";type "MAPLEPLOT";width 2.3443in;height 1.5638in;depth
0pt;display "USEDEF";plot_snapshots TRUE;mustRecompute FALSE;lastEngine
"MuPAD";xmin "-3.1459";xmax "3.1459";xviewmin "-3.1459";xviewmax
"3.1459";yviewmin "-1.2";yviewmax "1.2";viewset"XY";rangeset"X";plottype
4;plottickdisable TRUE;axesFont "Times New
Roman,12,0000000000,useDefault,normal";numpoints 100;plotstyle
"patch";axesstyle "normal";axestips FALSE;xis \TEXUX{x};var1name
\TEXUX{$x$};function \TEXUX{$\cos \left( x\right) $};linecolor
"blue";linestyle 1;pointstyle "point";linethickness 2;lineAttributes
"Solid";var1range "-3.1459,3.1459";num-x-gridlines 100;curveColor
"[flat::RGB:0x000000ff]";curveStyle "Line";function \TEXUX{$-\cos \left(
x\right) $};linecolor "blue";linestyle 1;pointstyle "point";linethickness
2;lineAttributes "Solid";var1range "-3.1459,3.1459";num-x-gridlines
100;curveColor "[flat::RGB:0x000000ff]";curveStyle "Line";VCamFile
'LTUWDL1Y.xvz';valid_file "T";tempfilename
'LTUWCL2P.wmf';tempfile-properties "XPR";}}

We can perform a similar analysis for a drum head, but it is much more
complicated. The modes are not points, but lines or curves, and the
frequencies of oscillation are not integer multiples of each other. See for
example http://physics.usask.ca/\symbol{126}%
hirose/ep225/animation/drum/anim-drum.htm.

Of course structures can also waggle on the ends. the ends can rotate
counter to each other, etc. These are more complex modes than the
longitudinal modes we have considered.

%TCIMACRO{%
%\TeXButton{Question 223.7.5}{\marginpar {
%\hspace{-0.5in}
%\begin{minipage}[t]{1in}
%\small{Question 223.7.5}
%\end{minipage}
%}}}%
%BeginExpansion
\marginpar {
\hspace{-0.5in}
\begin{minipage}[t]{1in}
\small{Question 223.7.5}
\end{minipage}
}%
%EndExpansion

\chapter{Single Frequency Interference, Multiple Dimensions}

So far we have had only waves mixed in a one-dimensional medium and we have
only allowed for reflections to mix the waves. But surely we can make waves
with different sources and mix them. 
%TCIMACRO{%
%\TeXButton{Two speaker demo}{\marginpar {
%\hspace{-0.5in}
%\begin{minipage}[t]{1in}
%\small{Two speaker demo}
%\end{minipage}
%}} }%
%BeginExpansion
\marginpar {
\hspace{-0.5in}
\begin{minipage}[t]{1in}
\small{Two speaker demo}
\end{minipage}
}
%EndExpansion
Consider setting up two speakers playing the same frequency. We expect that
we will still get regions of constructive and destructive interference.
Where these regions will be really depends on the total phase difference
between the two waves. 
\begin{eqnarray*}
\Delta \phi &=&\left( kx_{2}-\omega t+\phi _{2}\right) -\left( kx_{1}-\omega
t+\phi _{1}\right) \\
&=&k\left( x_{2}-x_{1}\right) +\left( \phi _{2}-\phi _{1}\right) \\
&=&\frac{2\pi }{\lambda }\left( \Delta x\right) +\Delta \phi _{o}
\end{eqnarray*}%
where we can see that there are at least two sources of phase difference
here. One can be from the two waves traveling different paths and then
combining $\left( \Delta x\right) $ and the other is from them starting with
a different phase to begin with $\Delta \phi .$

If we have two waves 
\begin{eqnarray*}
y_{1} &=&A\sin \left( kx_{1}-\omega t+\phi _{1}\right) \\
y_{2} &=&A\sin \left( kx_{2}-\omega t+\phi _{2}\right)
\end{eqnarray*}%
and we look at a particular part of the medium, that part will oscillate
with an amplitude that depends on the relative starting points of the two
waves, $\Delta \phi _{o}$ and on how the relative distances the waves have
traveled to get to our particular location in the medium, $\Delta x.$

%TCIMACRO{%
%\TeXButton{Fundamental Concepts}{\hspace{-1.3in}{\LARGE Fundamental Concepts\vspace{0.25in}}}}%
%BeginExpansion
\hspace{-1.3in}{\LARGE Fundamental Concepts\vspace{0.25in}}%
%EndExpansion

\begin{itemize}
\item In two dimensional problems, the total phase difference is given by $%
\Delta \phi =\left( 2\pi \frac{\Delta r}{\lambda }+\Delta \phi _{o}\right) $

\item In the total phase difference $\Delta \phi =\frac{2\pi }{\lambda }%
\left( \Delta x\right) +\Delta \phi _{o},$ the first term is due to path
differences, the second to initial phase differences (whether the two mixed
waves start together).
\end{itemize}

\section{Mathematical treatment of single frequency interference}

%TCIMACRO{%
%\TeXButton{Question 223.8.1}{\marginpar {
%\hspace{-0.5in}
%\begin{minipage}[t]{1in}
%\small{Question 223.8.1}
%\end{minipage}
%}}}%
%BeginExpansion
\marginpar {
\hspace{-0.5in}
\begin{minipage}[t]{1in}
\small{Question 223.8.1}
\end{minipage}
}%
%EndExpansion
%TCIMACRO{%
%\TeXButton{Question 223.8.2}{\marginpar {
%\hspace{-0.5in}
%\begin{minipage}[t]{1in}
%\small{Question 223.8.2}
%\end{minipage}
%}}}%
%BeginExpansion
\marginpar {
\hspace{-0.5in}
\begin{minipage}[t]{1in}
\small{Question 223.8.2}
\end{minipage}
}%
%EndExpansion
It is time to put our treatment of interference on a more general
mathematical footing.

We start with two waves in the same medium%
\begin{eqnarray*}
y_{1} &=&y_{\max }\sin \left( kx_{1}-\omega t+\phi _{1}\right) \\
y_{2} &=&y_{\max }\sin \left( kx_{2}-\omega t+\phi _{2}\right)
\end{eqnarray*}%
Each wave has its own phase constant. Each wave starts from a different
position (one at $x_{1}$ and the other at $x_{2}$), The superposition yields.%
\begin{equation*}
y_{r}=y_{\max }\sin \left( kx_{1}-\omega t+\phi _{1}\right) +y_{\max }\sin
\left( kx_{2}-\omega t+\phi _{2}\right)
\end{equation*}

which is graphed in the next figure.\FRAME{dtbpFX}{3.0727in}{1.3292in}{0pt}{%
}{}{Plot}{\special{language "Scientific Word";type "MAPLEPLOT";width
3.0727in;height 1.3292in;depth 0pt;display "USEDEF";plot_snapshots
TRUE;mustRecompute FALSE;lastEngine "MuPAD";xmin "0";xmax
"5.001000";xviewmin "0";xviewmax "5.001000";yviewmin "-2";yviewmax
"2";viewset"XY";rangeset"X";plottype 4;axesFont "Times New
Roman,12,0000000000,useDefault,normal";numpoints 100;plotstyle
"patch";axesstyle "normal";axestips FALSE;xis \TEXUX{x};var1name
\TEXUX{$x$};function \TEXUX{$\sin \left( \frac{1}{6}\pi +\pi x\right) +\sin
\pi x$};linecolor "green";linestyle 1;pointstyle "point";linethickness
2;lineAttributes "Solid";var1range "0,5.001000";num-x-gridlines
100;curveColor "[flat::RGB:0x00008000]";curveStyle "Line";VCamFile
'LTUWDL1X.xvz';valid_file "T";tempfilename
'LTUWCL2Q.wmf';tempfile-properties "XPR";}}Notice that the wave form is
taller (larger amplitude). Noticed it is shifted along the $x$ axis. This
graph is not surprising to us now, because we have done a case like this
before. We can find the shift in general rewriting $y_{r}.$ We need a trig
identity%
\begin{equation*}
\sin a+\sin b=2\cos \left( \frac{a-b}{2}\right) \sin \left( \frac{a+b}{2}%
\right)
\end{equation*}%
then let $a=kx-\omega t$ and $b=kx-\omega t+\phi $%
\begin{eqnarray*}
y_{r} &=&y_{\max }\sin \left( kx_{2}-\omega t+\phi _{2}\right) +y_{\max
}\sin \left( kx_{1}-\omega t+\phi _{1}\right) \\
&=&2y_{\max }\cos \left( \frac{\left( kx_{2}-\omega t+\phi _{1}\right)
-\left( kx_{1}-\omega t+\phi _{2}\right) }{2}\right) \sin \left( \frac{%
\left( kx_{2}-\omega t+\phi _{2}\right) +\left( kx_{1}-\omega t+\phi
_{1}\right) }{2}\right) \\
&=&2y_{\max }\cos \left( \frac{kx_{2}-kx_{1}}{2}+\frac{\phi _{2}-\phi _{1}}{2%
}\right) \sin \left( \frac{kx_{2}+kx_{1}-2\omega t+\phi _{2}+\phi _{1}}{2}%
\right) \\
&=&2y_{\max }\cos \left( k\frac{x_{2}-x_{1}}{2}+\frac{\phi _{2}-\phi _{1}}{2}%
\right) \sin \left( k\frac{x_{2}+x_{1}}{2}-\omega t+\frac{\phi _{2}+\phi _{1}%
}{2}\right) \\
&=&2y_{\max }\cos \left( k\frac{\Delta x}{2}+\frac{\Delta \phi _{o}}{2}%
\right) \sin \left( k\frac{x_{2}+x_{1}}{2}-\omega t+\frac{\phi _{2}+\phi _{1}%
}{2}\right) \\
&=&2y_{\max }\cos \left( \frac{1}{2}\left( \frac{2\pi }{\lambda }\Delta
x+\Delta \phi _{o}\right) \right) \sin \left( k\frac{x_{2}+x_{1}}{2}-\omega
t+\frac{\phi _{2}+\phi _{1}}{2}\right) \\
&=&2y_{\max }\cos \left( \frac{1}{2}\left( \Delta \phi \right) \right) \sin
\left( k\frac{x_{2}+x_{1}}{2}-\omega t+\frac{\phi _{2}+\phi _{1}}{2}\right)
\end{eqnarray*}%
where the last line is just a rearrangement to match the form we got last
time we did this problem with just one phase constant. Clearly, we see the
amplitude depends on what we call the phase difference And we can see our
two sources of phase difference. One can be from the two waves traveling
different paths and then combining $\left( \Delta x\right) $ and the other
is from the two waves starting with a different phase to begin with, $\Delta
\phi _{o}.$ If the total phase difference between the two waves is a
multiple of $2\pi ,$ then the two waves will experience constructive
interference%
\begin{equation*}
\Delta \phi =m2\pi \qquad m=0,\pm 1,\pm 2,\pm 3,\cdots
\end{equation*}%
Let's see that this works. Our amplitude is 
\begin{equation*}
A=2y_{\max }\cos \left( \frac{1}{2}\left( \Delta \phi \right) \right)
\end{equation*}%
and if we look at a cosine function we see that $\cos \left( \theta \right) $
is either $1$ or $-1$ at $\theta =n\pi .$\FRAME{dtbpFX}{4.4996in}{0.7913in}{%
0pt}{}{}{Plot}{\special{language "Scientific Word";type "MAPLEPLOT";width
4.4996in;height 0.7913in;depth 0pt;display "USEDEF";plot_snapshots
TRUE;mustRecompute FALSE;lastEngine "MuPAD";xmin "-5";xmax "5";xviewmin
"-5.0010000010002";xviewmax "5.0010000010002";yviewmin
"-1.00019994985428";yviewmax "1";plottype 4;labeloverrides 1;x-label "x
(units of pi)";axesFont "Times New
Roman,12,0000000000,useDefault,normal";numpoints 100;plotstyle
"patch";axesstyle "normal";axestips FALSE;xis \TEXUX{x};var1name
\TEXUX{$x$};function \TEXUX{$\cos \left( \pi x\right) $};linecolor
"black";linestyle 1;pointstyle "point";linethickness 1;lineAttributes
"Solid";var1range "-5,5";num-x-gridlines 100;curveColor
"[flat::RGB:0000000000]";curveStyle "Line";VCamFile
'S1PJ1S0L.xvz';valid_file "T";tempfilename
'S1PJ1S04.wmf';tempfile-properties "XPR";}}So if $\Delta \phi =m2\pi $ then
the amplitude is 
\begin{eqnarray*}
A &=&2y_{\max }\left( \frac{1}{2}\left( m2\pi \right) \right) \\
&=&2y_{\max }\cos \left( m\pi \right)
\end{eqnarray*}%
We don't really care if the amplitude function is big positively or
negatively. So we get constructive interference for either $1$ or $-1$.
Then, this our case for constructive interference.%
\begin{equation*}
\Delta \phi =\left( \frac{2\pi }{\lambda }\Delta x+\Delta \phi _{o}\right)
=n2\pi \qquad n=0,\pm 1,\pm 2,\pm 3,\cdots
\end{equation*}

How about for destructive interference? We start again with our amplitude
function%
\begin{equation*}
A=2y_{\max }\cos \left( \frac{1}{2}\left( \Delta \phi \right) \right)
\end{equation*}%
but now we want when the cosine part is zero. 
\begin{equation*}
\cos \left( \theta \right) =0
\end{equation*}

Looking at our cosine graph again, that happens for cosine when $\theta =%
\frac{\pi }{2}$, $\frac{3\pi }{2},$ $\frac{5\pi }{2},$ $\cdots .$ We could
write this as $\theta =\left( m+\frac{1}{2}\right) \pi $ for $m=0,1,2,\cdots
.$ But remember that in our amplitude function, we already have the $1/2$ in
the function, so we want $\Delta \phi $ to have just the odd integer
multiple of $\pi $. We could write this as 
\begin{equation*}
\Delta \phi =\left( 2m+1\right) \pi \qquad m=0,\pm 1,\pm 2,\pm 3,\cdots 
\text{ }
\end{equation*}%
So our condition for destructive interference is 
\begin{equation*}
\Delta \phi =\left( \frac{2\pi }{\lambda }\Delta x+\Delta \phi _{o}\right)
=\left( 2m+1\right) \pi \qquad m=0,\pm 1,\pm 2,\pm 3,\cdots \text{ }
\end{equation*}
We have developed a useful matched set of equations that will tell us if we
mix two waves when we will have constructive and destructive interference:

\begin{eqnarray*}
\Delta \phi &=&\left( \frac{2\pi }{\lambda }\Delta r+\Delta \phi _{o}\right)
=m2\pi \qquad m=0,\pm 1,\pm 2,\pm 3,\cdots \text{\qquad Constructive} \\
\Delta \phi &=&\left( \frac{2\pi }{\lambda }\Delta r+\Delta \phi _{o}\right)
=\left( 2m+1\right) \pi \qquad m=0,\pm 1,\pm 2,\pm 3,\cdots \text{\qquad\
Destructive}
\end{eqnarray*}

Let's take an example to see how this can be used.

\subsection{Example of two wave interference: Stealth Fighter}

\FRAME{dhF}{4.6034in}{2.2053in}{0pt}{}{}{Figure}{\special{language
"Scientific Word";type "GRAPHIC";maintain-aspect-ratio TRUE;display
"USEDEF";valid_file "T";width 4.6034in;height 2.2053in;depth
0pt;original-width 5.8254in;original-height 2.7752in;cropleft "0";croptop
"1";cropright "1";cropbottom "0";tempfilename
'LTUWCL2R.wmf';tempfile-properties "XPR";}}

The stealth fighter is coated with an anti-reflective polymer. This is part
of it's mechanism for making the plane invisible to radar. Suppose we have a
radar system with a wavelength of $3.00\unit{cm}.$ Further suppose that the
index of refraction of the anti-reflective polymer is $n=1.50$, and that the
aircraft index of refraction is very large, how thick would you make the
coating?

We want destructive interference, so let's start with our destructive
interference condition 
\begin{equation*}
\Delta \phi =\left( \frac{2\pi }{\lambda }\Delta r+\Delta \phi _{o}\right)
=\left( 2m+1\right) \pi \qquad m=0,\pm 1,\pm 2,\pm 3,\cdots \text{\qquad
Destructive}
\end{equation*}

The radar waves all hit the plane in phase. From the figure, we see that the
radar wave will reflect off of the coating. Because the index of refraction
of the coating is large, this is like a fixed end of a rope. There will be
an inversion.

But some of the wave will penetrate the polymer. This will reflect off of
the plane body. The plane body has a very large index of refraction, so once
again the wave will experience an inversion. The outgoing waves would then
be in phase and create constructive interference because 
\begin{equation*}
\Delta \phi _{o}=\pi =\pi =0
\end{equation*}%
at this point. Thus 
\begin{equation*}
\Delta \phi =\left( \frac{2\pi }{\lambda }\Delta r\right) =\left(
2m+1\right) \pi \qquad m=0,\pm 1,\pm 2,\pm 3,\cdots \text{\qquad Destructive}
\end{equation*}%
But we have to remember the path difference! The part of the wave that
entered the polymer travels farther. If that path difference, $\Delta r,$ is
just right so that 
\begin{equation*}
\frac{2\pi }{\lambda }\Delta r=\pi
\end{equation*}%
This is the $m=0$ case in our destructive interference case%
\begin{equation*}
\Delta \phi =\left( \frac{2\pi }{\lambda }\Delta r\right) =\left( 2\left(
0\right) +1\right) \pi =\pi
\end{equation*}%
then the amplitude function would be 
\begin{eqnarray*}
A &=&2E_{\max }\cos \left( \frac{2\pi }{\lambda }\Delta r\right) \\
&=&2E_{\max }\cos \left( \frac{1}{2}\left( \pi \right) \right) \\
&=&0
\end{eqnarray*}%
and we have destructive interference. Note that these are electromagnetic
waves, so instead of $y_{\max }$ we have used $E_{\max }$ as the individual
wave amplitude. But the important thing is that the plane cannot be seen by
the radar! Of course, this works for $m=1$ and $m=2,$ etc. as well. Any odd
multiple of $\pi $ will work. 
\begin{equation*}
\frac{2\pi }{\lambda }\Delta r=\left( 2m+1\right) \pi
\end{equation*}%
where $m=0,$ $1,$ $2,\cdots $ so that we are guaranteed an odd multiple of $%
\pi .$ This is our condition for destructive interference.

But we are interested in the thickness. We realize that $\Delta r$ is about
twice the thickness, since the wave travels though the coating and back. So
let's let $\Delta r\approx 2t$ 
\begin{equation*}
\frac{2\pi }{\lambda }2t\approx \left( 2m+1\right) \pi
\end{equation*}%
\begin{equation*}
2t\approx \left( 2m+1\right) \frac{\lambda }{2}
\end{equation*}%
\begin{equation*}
2t\approx \left( m+\frac{1}{2}\right) \lambda
\end{equation*}%
\begin{equation*}
t\approx \left( m+\frac{1}{2}\right) \frac{\lambda }{2}
\end{equation*}

But there is a further complication. We should write our thickness equation
as 
\begin{equation*}
t\approx \left( m+\frac{1}{2}\right) \frac{\lambda _{in}}{2}
\end{equation*}%
because $\Delta r$ has to provide an odd integer times the wavelength \emph{%
inside the coating} for the phase to be right. After all, the wave is
traveling inside the coating. We know that the wavelength will change as we
enter the slower material.

To see this, consider two waves traveling to the right. One passes through a
slower medium. We expect the wavelength to shorten. We can see that,
depending on the thickness $t,$ the wave may be in phase or out of phase. In
the next picture, the thickness is just right so that we have destructive
interference. \FRAME{dtbpF}{3.7585in}{1.4226in}{0pt}{}{}{Figure}{\special%
{language "Scientific Word";type "GRAPHIC";maintain-aspect-ratio
TRUE;display "USEDEF";valid_file "T";width 3.7585in;height 1.4226in;depth
0pt;original-width 3.2206in;original-height 1.2012in;cropleft "0";croptop
"1";cropright "1";cropbottom "0";tempfilename
'S1D0GK01.wmf';tempfile-properties "XPR";}} We have such a wavelength shift
in the coating. But we don't know the wavelength inside the coating. All we
know is the radar wavelength, $\lambda _{out}$. We can fix it by writing the
wavelength inside in terms of the wavelength outside. Earlier in our studies
we found that the new wavelength will be given by equation (\ref%
{WavelengthChange})%
\begin{equation*}
\lambda _{f}=\frac{v_{f}}{v_{i}}\lambda _{i}
\end{equation*}%
Let's rewrite this for our case%
\begin{equation*}
\lambda _{in}=\frac{v_{in}}{v_{out}}\lambda _{out}
\end{equation*}%
We can express this in terms of the index of refraction%
\begin{equation*}
n=\frac{c}{v}
\end{equation*}%
by multiplying the left hand side by $c/c$ then 
\begin{equation*}
\lambda _{in}=\frac{cv_{in}}{cv_{out}}\lambda _{out}
\end{equation*}%
or%
\begin{eqnarray*}
\lambda _{in} &=&\frac{\frac{c}{v_{out}}}{\frac{c}{v_{in}}}\lambda _{out} \\
&=&\frac{n_{out}}{n_{in}}\lambda _{out}
\end{eqnarray*}%
in the case of our aircraft coating the outside medium is air so $%
n_{out}\approx 1$ 
\begin{equation*}
\lambda _{in}=\frac{1}{n_{in}}\lambda _{out}
\end{equation*}%
This is this wavelength we need to match as the radar signal enters the
medium.

Using this expression for $\lambda _{in}$ in 
\begin{equation*}
t\approx \left( m+\frac{1}{2}\right) \frac{\lambda _{in}}{2}\qquad
m=0,1,2,\cdots
\end{equation*}%
will give us the condition for destructive interference. Let's rewrite our $%
\lambda _{in}$ equation for our case of a coating and air 
\begin{equation*}
\lambda _{in}=\lambda _{coating}=\frac{1}{n_{in}}\lambda _{out}=\frac{1}{%
n_{coating}}\lambda _{air}
\end{equation*}%
thus 
\begin{equation*}
t\approx \left( m+\frac{1}{2}\right) \frac{1}{2}\left( \frac{\lambda _{air}}{%
n_{coating}}\right) \qquad m=0,1,2,\cdots
\end{equation*}%
is our condition for being stealthy.

Let's assume we want the thinnest coating possible, so we set $m=0.$ Then 
\begin{equation*}
t\approx \left( \frac{1}{4}\right) \left( \frac{\lambda _{air}}{n_{coating}}%
\right)
\end{equation*}%
and our thickness would be 
\begin{equation*}
t\approx \left( \frac{1}{4}\right) \left( \frac{3.00\unit{cm}}{1.50}\right)
=0.5\unit{cm}
\end{equation*}%
This seems doable for an aircraft coating!

Of course we could also make a plane that would be more visible to radar by
choosing the constructive interference case. Suppose we are building a
search and rescue plane. We want to enhance it's ability to be seen by radar
in fog. We start with the condition for constructive interference 
\begin{equation*}
\Delta \phi =\left( \frac{2\pi }{\lambda }\Delta r+\Delta \phi _{o}\right)
=m2\pi \qquad m=0,\pm 1,\pm 2,\pm 3,\cdots \text{\qquad Constructive}
\end{equation*}%
It will still be true that $\Delta \phi _{o}=0.$ 
\begin{equation*}
\Delta \phi =\left( \frac{2\pi }{\lambda }\Delta r\right) =m2\pi
\end{equation*}%
and it is still true that $\Delta r\approx 2t.$ 
\begin{equation*}
\left( \frac{2\pi }{\lambda }2t\right) \approx m2\pi
\end{equation*}%
then 
\begin{equation*}
\left( \frac{1}{\lambda }2t\right) \approx m
\end{equation*}%
\begin{equation*}
t\approx \frac{1}{2}m\lambda
\end{equation*}%
and we still have to adjust for the coating index of refraction 
\begin{equation*}
t\approx \frac{m}{2}\left( \frac{\lambda _{air}}{n_{coating}}\right)
\end{equation*}%
And once again we have several choices for $m$ 
\begin{equation*}
t\approx \frac{m}{2}\left( \frac{\lambda _{air}}{n_{coating}}\right) \qquad
m=0,1,2,\cdots
\end{equation*}%
But now the coating will provide constructive interference, making it easier
to track on radar from the command center. For the thinnest possibility, set 
$m=1$ because the $m=0$ case doesn't give us any thickness.%
\begin{eqnarray*}
t &\approx &m\frac{1}{2}\left( \frac{\lambda _{air}}{n_{coating}}\right) \\
&=&\frac{1}{2}\left( \frac{3.00\unit{cm}}{1.50}\right) \\
&=&\allowbreak 1\unit{cm}
\end{eqnarray*}

Note that we reasoned out these equations for the boundary conditions that
we have in our problem. If the boundary conditions change, so do the
equations.

\subsection{Example of two wave interference: soap bubble}

Take a soap bubble for example.\FRAME{dtbpFU}{3.5293in}{3.1408in}{0pt}{\Qcb{%
Interference from a soap bubble. (Bubble image in the Public Domain,
courtesy Marcin Der\k{e}gowski)}}{}{Figure}{\special{language "Scientific
Word";type "GRAPHIC";maintain-aspect-ratio TRUE;display "USEDEF";valid_file
"T";width 3.5293in;height 3.1408in;depth 0pt;original-width
3.5674in;original-height 3.1718in;cropleft "0";croptop "1";cropright
"1";cropbottom "0";tempfilename 'NNZQU603.wmf';tempfile-properties "XPR";}}%
Now we have a phase shift on the first reflection, but not one on the
reflection from the inside surface of the bubble because the bubble is full
of air. The index of refraction of air is less than that for the bubble
material. So as we leave the bubble material it is more like having a free
end of a rope. As the waves leave the surface, they are half a wavelength
out of phase due to $\Delta \phi _{o}$ because of the single inversion from
the bubble outer surface. We would have destructive interference due to just
this, but we also have to account for the bubble thickness. If this
thickness is a multiple of a wavelength, then we are still have half a
wavelength out of phase and we have destructive interference.

Here are our basic equations%
\begin{eqnarray*}
\Delta \phi &=&\left( \frac{2\pi }{\lambda }\Delta r+\Delta \phi _{o}\right)
=m2\pi \qquad m=0,\pm 1,\pm 2,\pm 3,\cdots \text{ Constructive} \\
\Delta \phi &=&\left( \frac{2\pi }{\lambda }\Delta r+\Delta \phi _{o}\right)
=\left( 2m+1\right) \pi \qquad m=0,\pm 1,\pm 2,\pm 3,\cdots \text{
Destructive}
\end{eqnarray*}%
Suppose we want want constructive interference to get our colors, so we take
the first

\begin{equation*}
\Delta \phi =\left( \frac{2\pi }{\lambda }\Delta r+\Delta \phi _{o}\right)
=m2\pi
\end{equation*}%
and this time we have 
\begin{eqnarray*}
\Delta \phi _{o} &=&\phi _{transmitted}-\phi _{reflected} \\
&=&0-\pi \\
&=&-\pi
\end{eqnarray*}%
It is still true that $\Delta r\approx 2t$ so from our constructive
interference equation%
\begin{equation*}
\frac{2\pi }{\lambda }2t-\pi =m2\pi
\end{equation*}%
\begin{equation*}
\frac{2}{\lambda }t-\frac{1}{2}=m
\end{equation*}%
\begin{equation*}
\frac{2}{\lambda }t=m+\frac{1}{2}
\end{equation*}%
\begin{equation*}
t=\frac{\lambda }{2}\left( m+\frac{1}{2}\right)
\end{equation*}%
We again have the problem that this wavelength must be the wavelength inside
the bubble material $\lambda =\lambda _{in}.$ But we see the outside
wavelength $\lambda _{out}$. We can reuse our conversion from outside to
inside wavelength from our last problem because we are once again in air and 
$n_{air}\approx 1$.%
\begin{equation*}
\lambda _{in}=\frac{1}{n_{in}}\lambda _{out}
\end{equation*}%
then%
\begin{equation*}
t=\frac{\lambda _{out}}{2n_{in}}\left( m+\frac{1}{2}\right) \qquad
m=0,1,2,\cdots
\end{equation*}%
Or writing this with $n_{in}=n_{\text{bubble }}$to make it clear that the
inside material is the bubble solution,%
\begin{equation*}
t=\left( m+\frac{1}{2}\right) \frac{1}{2}\left( \frac{\lambda _{air}}{n_{%
\text{bubble}}}\right) \qquad m=0,1,2,\cdots
\end{equation*}%
but this was the equation for destructive interference for the plane! We can
see that memorizing the thickness equations won't work. We need to start
with our conditions on $\Delta \phi $ for constructive and destructive
interference to be safe!

How about the dark parts of the bubble with no color (the parts we can see
through). These would be destructive interference%
\begin{equation*}
\Delta \phi =\left( \frac{2\pi }{\lambda }\Delta x+\Delta \phi _{o}\right)
=\left( 2m+1\right) \pi
\end{equation*}%
We can fill in the pieces to obtain%
\begin{eqnarray*}
\left( \frac{2\pi }{\lambda }2t-\pi \right) &=&\left( 2m+1\right) \pi \\
\frac{2}{\lambda }2t-1 &=&\left( 2m+1\right) \\
\frac{2}{\lambda }2t &=&\left( 2m+1\right) +1 \\
\frac{4}{\lambda }t &=&\left( 2m+1\right) +1 \\
t &=&\frac{\lambda }{4}\left( 2m+2\right) \\
t &=&\frac{\lambda }{2}\left( m+1\right) \\
t &=&\frac{m+1}{2}\left( \frac{\lambda _{out}}{n_{\text{bubble}}}\right)
\end{eqnarray*}%
This is our condition for destructive interference for the bubble. We don't
have to, but we could write $m+1=p$ where $p$ is an integer that starts at $%
1 $ instead of zero. 
\begin{equation*}
t=\frac{p}{2}\left( \frac{\lambda _{out}}{n_{\text{bubble}}}\right) \qquad
p=1,2,\cdots
\end{equation*}%
But this is very like the condition for constructive interference for the
plane.

Hopefully, it is apparent that we have to start with our basic equations 
\begin{eqnarray*}
\Delta \phi &=&\left( \frac{2\pi }{\lambda }\Delta x+\Delta \phi _{o}\right)
=m2\pi \qquad m=0,\pm 1,\pm 2,\pm 3,\cdots \text{ Constructive} \\
\Delta \phi &=&\left( \frac{2\pi }{\lambda }\Delta x+\Delta \phi _{o}\right)
=\left( 2m+1\right) \pi \qquad m=0,\pm 1,\pm 2,\pm 3,\cdots \text{
Destructive}
\end{eqnarray*}%
each time we attempt an interference problem because the outcome depends on
both $\Delta x$ and $\Delta \phi _{o}.$ We have to construct the equation
each time for the interference condition we want (constructive or
destructive) finding $\Delta x$ and $\Delta \phi _{o}$ for the boundary
conditions we have.

\section{Single frequency interference in more than one dimension}

%TCIMACRO{%
%\TeXButton{Two speaker demo}{\marginpar {
%\hspace{-0.5in}
%\begin{minipage}[t]{1in}
%\small{Two speaker demo}
%\end{minipage}
%}}}%
%BeginExpansion
\marginpar {
\hspace{-0.5in}
\begin{minipage}[t]{1in}
\small{Two speaker demo}
\end{minipage}
}%
%EndExpansion
Suppose I put two speakers facing each other $3\unit{m}$ apart. And suppose
I\ want four nodes in the middle so we can easily find them. Then I will
need 
\begin{equation*}
2\lambda =3\unit{m}
\end{equation*}%
or $\lambda =\frac{3}{2}\unit{m}.$ If we are at about $20\unit{%
%TCIMACRO{\U{2103}}%
%BeginExpansion
{}^{\circ}{\rm C}%
%EndExpansion
}$ then $v=343\unit{m}/\unit{s}$ and 
\begin{eqnarray*}
f &=&\frac{343\unit{m}/\unit{s}}{\frac{3}{2}\unit{m}.} \\
&=&228.\,\allowbreak 67\unit{Hz}
\end{eqnarray*}%
If $\Delta \phi _{o}=0$ the nodes should be spaced symmetrically between the
two speakers. 
%TCIMACRO{%
%\TeXButton{Are the nodes symmetrically placed?}{\marginpar {
%\hspace{-0.5in}
%\begin{minipage}[t]{1in}
%\small{Are the nodes symmetrically placed?}
%\end{minipage}
%}} }%
%BeginExpansion
\marginpar {
\hspace{-0.5in}
\begin{minipage}[t]{1in}
\small{Are the nodes symmetrically placed?}
\end{minipage}
}
%EndExpansion
The rest of the phase comes from the difference in starting positions.

%TCIMACRO{%
%\TeXButton{Question 223.8.3}{\marginpar {
%\hspace{-0.5in}
%\begin{minipage}[t]{1in}
%\small{Question 223.8.3}
%\end{minipage}
%}}}%
%BeginExpansion
\marginpar {
\hspace{-0.5in}
\begin{minipage}[t]{1in}
\small{Question 223.8.3}
\end{minipage}
}%
%EndExpansion
%TCIMACRO{%
%\TeXButton{Question 223.8.4}{\marginpar {
%\hspace{-0.5in}
%\begin{minipage}[t]{1in}
%\small{Question 223.8.4}
%\end{minipage}
%}}}%
%BeginExpansion
\marginpar {
\hspace{-0.5in}
\begin{minipage}[t]{1in}
\small{Question 223.8.4}
\end{minipage}
}%
%EndExpansion
But what happens if our waves don't travel along the same line? Suppose you
are at a dance, and there are two speakers. Further suppose that you are
testing the system with a constant tone (either that, or you have somewhat
boring music with constant tones). Suppose the two speakers make waves in
phase. If you are equal distance from the two speakers, you would expect
constructive interference because both $\Delta \phi _{o}=0$ and $\Delta x=0$
for this case.\FRAME{dhF}{1.4218in}{1.9527in}{0pt}{}{}{Figure}{\special%
{language "Scientific Word";type "GRAPHIC";maintain-aspect-ratio
TRUE;display "USEDEF";valid_file "T";width 1.4218in;height 1.9527in;depth
0pt;original-width 3.0441in;original-height 4.1909in;cropleft "0";croptop
"1";cropright "1";cropbottom "0";tempfilename
'LTUWCL2S.wmf';tempfile-properties "XPR";}}But there are more places where
we expect constructive interference, because we know the sound wave is
really spherical. Any time the path difference, $\Delta x=n\lambda ,$ then 
\begin{equation*}
\Delta \phi =\frac{2\pi }{\lambda }\left( n\lambda \right) =n2\pi
\end{equation*}%
and we will have constructive interference The next figure shows an example
where the path difference is one wavelength.

\FRAME{dhF}{1.5489in}{2.1318in}{0pt}{}{}{Figure}{\special{language
"Scientific Word";type "GRAPHIC";maintain-aspect-ratio TRUE;display
"USEDEF";valid_file "T";width 1.5489in;height 2.1318in;depth
0pt;original-width 3.8233in;original-height 5.2736in;cropleft "0";croptop
"1";cropright "1";cropbottom "0";tempfilename
'LTUWCL2T.wmf';tempfile-properties "XPR";}}But any of these spots will
experience constructive interference. Note the loud spots are where there
are two crests or two troughs together.\FRAME{dhF}{1.6181in}{2.1231in}{0pt}{%
}{}{Figure}{\special{language "Scientific Word";type
"GRAPHIC";maintain-aspect-ratio TRUE;display "USEDEF";valid_file "T";width
1.6181in;height 2.1231in;depth 0pt;original-width 3.7256in;original-height
4.8983in;cropleft "0";croptop "1";cropright "1";cropbottom "0";tempfilename
'LTUWCM2U.wmf';tempfile-properties "XPR";}}We also expect to see destructive
interference. This should occur where path differences are multiples of $%
\Delta x=\lambda /2$ so that 
\begin{equation*}
\Delta \phi =\frac{2\pi }{\lambda }\left( n\frac{\lambda }{2}\right) =n\pi
\end{equation*}
The situation of being just half a wavelength off is shown next\FRAME{dhF}{%
1.6933in}{2.0401in}{0pt}{}{}{Figure}{\special{language "Scientific
Word";type "GRAPHIC";maintain-aspect-ratio TRUE;display "USEDEF";valid_file
"T";width 1.6933in;height 2.0401in;depth 0pt;original-width
1.9882in;original-height 2.4007in;cropleft "0";croptop "1";cropright
"1";cropbottom "0";tempfilename 'LTUWCM2V.wmf';tempfile-properties "XPR";}}%
but there are many places were we could be a multiple of a wavelength plus
and extra half a wavelength off. Each of these will produce destructive
interference.\FRAME{dhF}{1.7141in}{1.9969in}{0pt}{}{}{Figure}{\special%
{language "Scientific Word";type "GRAPHIC";maintain-aspect-ratio
TRUE;display "USEDEF";valid_file "T";width 1.7141in;height 1.9969in;depth
0pt;original-width 2.5581in;original-height 2.9836in;cropleft "0";croptop
"1";cropright "1";cropbottom "0";tempfilename
'LTUWCM2W.wmf';tempfile-properties "XPR";}}

Recall that when you moved from one dimension to two dimensions in PH 121 or
Dynamics problems, you changed from the variables $x$ and $y$ to the
variable $r$ where 
\begin{equation*}
r=\sqrt{x^{2}+y^{2}}
\end{equation*}%
Thus our phase becomes 
\begin{equation*}
\Delta \phi =\left( 2\pi \frac{\Delta r}{\lambda }+\Delta \phi _{o}\right)
\end{equation*}

In our dance example, suppose we have speakers that are $4\unit{m}$ apart
and we are standing $3\unit{m}$ directly in front of one of the speakers.
Further suppose that we play an $A$ just above middle $C$ which has a
frequency of $440\unit{Hz}.$ The speed of sound is $343\unit{m}/\unit{s}$.
Our speakers are connected to the same stereo with equal length wires. What
is the phase difference at this spot?

\FRAME{dhF}{3.1592in}{3.2932in}{0pt}{}{}{Figure}{\special{language
"Scientific Word";type "GRAPHIC";maintain-aspect-ratio TRUE;display
"USEDEF";valid_file "T";width 3.1592in;height 3.2932in;depth
0pt;original-width 3.1142in;original-height 3.2474in;cropleft "0";croptop
"1";cropright "1";cropbottom "0";tempfilename
'LTUWCM2X.wmf';tempfile-properties "XPR";}}From the geometry we can tell
that the path from the second speaker must be $5\unit{m}.$ So 
\begin{eqnarray*}
\Delta r &=&5\unit{m}-3\unit{m} \\
&=&2\unit{m}
\end{eqnarray*}%
We can tell that the wavelength is%
\begin{eqnarray*}
\lambda &=&\frac{v}{f} \\
&=&\frac{343\unit{m}/\unit{s}}{440\unit{Hz}} \\
&=&\allowbreak 0.779\,55\unit{m}:
\end{eqnarray*}%
Since the speakers are connected to the same stereo with equal length wires, 
$\Delta \phi _{o}=0.$ Then 
\begin{eqnarray*}
\Delta \phi &=&\frac{2\pi }{\lambda }\Delta r+\Delta \phi _{o} \\
&=&\frac{2\pi }{\allowbreak 0.779\,55\unit{m}}\left( 2\unit{m}\right) +0 \\
&=&\allowbreak 5.\,\allowbreak 131\,2\pi \\
&=&2\pi +\allowbreak 3.\,\allowbreak 131\,2\pi
\end{eqnarray*}%
$\allowbreak \allowbreak $We should ask, is this constructive or destructive
interference? Well, it is neither purely constructive interference nor total
destructive interference. Our amplitude would be 
\begin{equation*}
2A\cos \left( \frac{1}{2}\left( \frac{2\pi }{\lambda }\Delta r+\Delta \phi
_{o}\right) \right)
\end{equation*}%
so in this case we get%
\begin{equation*}
2A\cos \left( \frac{1}{2}\left( 2\pi +\allowbreak 3.\,\allowbreak 131\,2\pi
\right) \right) =-0.409\,27A
\end{equation*}%
which is smaller (in magnitude) than $A,$ so it is partial destructive
interference. It would be quieter at this spot than if we had just one
speaker operating.

You might guess that this sort of analysis plays a large part in design of
concert halls. It also is important in mechanical designs.

But you should have seen a deficit in what we have learned so far. Up to
this point, we have only mixed waves that have the same frequency. Can we
mix waves that have different frequencies? That will be the subject of our
next lecture.

\chapter{Multiple Frequency Interference}

%TCIMACRO{%
%\TeXButton{Fundamental Concepts}{\hspace{-1.3in}{\LARGE Fundamental Concepts\vspace{0.25in}}}}%
%BeginExpansion
\hspace{-1.3in}{\LARGE Fundamental Concepts\vspace{0.25in}}%
%EndExpansion

\begin{itemize}
\item Mixing waves of different frequencies produces a time-varying
amplitude called beating.

\item Complex waves can be treated as a superposition of simple sinusoidal
waves.

\item Limited signals are multi-frequency
\end{itemize}

\section{Beats}

%TCIMACRO{%
%\TeXButton{Question 223.9.1}{\marginpar {
%\hspace{-0.5in}
%\begin{minipage}[t]{1in}
%\small{Question 223.9.1}
%\end{minipage}
%}}}%
%BeginExpansion
\marginpar {
\hspace{-0.5in}
\begin{minipage}[t]{1in}
\small{Question 223.9.1}
\end{minipage}
}%
%EndExpansion
%TCIMACRO{%
%\TeXButton{Beat Demo}{\marginpar {
%\hspace{-0.5in}
%\begin{minipage}[t]{1in}
%\small{Beat Demo}
%\end{minipage}
%}}}%
%BeginExpansion
\marginpar {
\hspace{-0.5in}
\begin{minipage}[t]{1in}
\small{Beat Demo}
\end{minipage}
}%
%EndExpansion
Up till now we only superposed waves that had the same frequency. But what
happens if we take waves with different frequencies?%
\begin{equation*}
y_{1}=y_{\max }\sin \left( kx-\omega _{1}t\right)
\end{equation*}%
\begin{equation*}
y_{2}=y_{\max }\sin \left( kx-\omega _{2}t\right)
\end{equation*}%
We can plot both waves on the same graph, in this case a history graph.

\FRAME{dtbpFX}{4.9023in}{2.2902in}{0pt}{}{}{Plot}{\special{language
"Scientific Word";type "MAPLEPLOT";width 4.9023in;height 2.2902in;depth
0pt;display "USEDEF";plot_snapshots TRUE;mustRecompute FALSE;lastEngine
"MuPAD";xmin "-15";xmax "15";xviewmin "-10";xviewmax "10";yviewmin
"-2.5";yviewmax "2.5";viewset"XY";rangeset"X";plottype 4;labeloverrides
3;x-label "t";y-label "y(t)";axesFont "Times New
Roman,12,0000000000,useDefault,normal";numpoints 100;plotstyle
"patch";axesstyle "normal";axestips FALSE;xis \TEXUX{x};var1name
\TEXUX{$x$};function \TEXUX{$\cos \left( 10x\right) $};linecolor
"maroon";linestyle 1;pointstyle "point";linethickness 1;lineAttributes
"Solid";var1range "-15,15";num-x-gridlines 900;curveColor
"[flat::RGB:0x00800000]";curveStyle "Line";rangeset"X";function \TEXUX{$\cos
\left( 11.x\right) $};linecolor "green";linestyle 1;pointstyle
"point";linethickness 1;lineAttributes "Solid";var1range
"-10,10";num-x-gridlines 900;curveColor "[flat::RGB:0x00008000]";curveStyle
"Line";rangeset"X";VCamFile 'MTQWM00H.xvz';valid_file "T";tempfilename
'MTQWLF00.wmf';tempfile-properties "XPR";}}Notice that there are places
where the waves are in phase, and places where they are not. The
superposition looks like this\FRAME{dtbpFX}{5.0372in}{2.3532in}{0pt}{}{}{Plot%
}{\special{language "Scientific Word";type "MAPLEPLOT";width 5.0372in;height
2.3532in;depth 0pt;display "USEDEF";plot_snapshots TRUE;mustRecompute
FALSE;lastEngine "MuPAD";xmin "-10";xmax "10";xviewmin "-10";xviewmax
"10";yviewmin "-5";yviewmax "5";viewset"XY";rangeset"X";plottype
4;labeloverrides 3;x-label "t";y-label "y(t)";axesFont "Times New
Roman,12,0000000000,useDefault,normal";numpoints 100;plotstyle
"patch";axesstyle "normal";axestips FALSE;xis \TEXUX{x};var1name
\TEXUX{$x$};function \TEXUX{$\cos \left( 10.0x\right) +\cos \left(
11.x\right) $};linecolor "green";linestyle 1;pointstyle
"point";linethickness 1;lineAttributes "Solid";var1range
"-10,10";num-x-gridlines 900;curveColor "[flat::RGB:0x00008000]";curveStyle
"Line";rangeset"X";VCamFile 'MTQWN00I.xvz';valid_file "T";tempfilename
'MTQWN001.wmf';tempfile-properties "XPR";}}where there is constructive
interference, the resulting wave amplitude is large, where there is
destructive interference, the resulting amplitude is zero. We get a
traveling wave who's amplitude varies. We can find the amplitude function
algebraically.

We can write these as%
\begin{equation*}
y_{1}=y_{\max }\sin \left( kx-2\pi f_{1}t\right)
\end{equation*}%
\begin{equation*}
y_{2}=y_{\max }\sin \left( kx-2\pi f_{2}t\right)
\end{equation*}

The sum is just%
\begin{equation*}
y=y_{\max }\sin \left( kx-2\pi f_{1}t\right) +y_{\max }\sin \left( kx-2\pi
f_{2}t\right)
\end{equation*}

We use another trig identity%
\begin{equation*}
\sin \left( a\right) +\sin \left( b\right) =2\cos \left( \frac{a-b}{2}%
\right) \sin \left( \frac{a+b}{2}\right)
\end{equation*}%
which allows us to write tis as%
\begin{eqnarray*}
y &=&2y_{\max }\cos \left( \frac{kx-2\pi f_{2}t-\left( kx-2\pi f_{1}t\right) 
}{2}\right) \sin \left( \frac{kx-2\pi f_{2}t+kx-2\pi f_{1}t}{2}\right) \\
&=&2y_{\max }\cos \left( 2\pi \frac{f_{1}-f_{2}}{2}t\right) \sin \left(
-2\pi \frac{f_{1}+f_{2}}{2}t\right) \\
&=&\left[ 2y_{\max }\cos \left( 2\pi \frac{f_{1}-f_{2}}{2}t\right) \right]
\sin \left( kx-2\pi \frac{f_{1}+f_{2}}{2}t\right)
\end{eqnarray*}

We see that we have a part that has a frequency that is the average of $%
f_{1} $ and $f_{2}.$ This is the frequency we hear. But we have another
complicated amplitude term, and this time it is a function of time (just to
be confusing). The amplitude has its own frequency that is half the
difference of $f_{1}$ and $f_{2}.$ 
\begin{equation*}
A_{\text{resultant}}=2y_{\max }\cos \left( 2\pi \frac{f_{1}-f_{2}}{2}t\right)
\end{equation*}%
So the sound amplitude will vary in time for a given spatial location.

The situation is odder still. We have a cosine function, but it is really an
envelope for the higher frequency motion of the air particles. \FRAME{dtbpFX%
}{5.0375in}{2.3531in}{0pt}{}{}{Plot}{\special{language "Scientific
Word";type "MAPLEPLOT";width 5.0375in;height 2.3531in;depth 0pt;display
"USEDEF";plot_snapshots TRUE;mustRecompute FALSE;lastEngine "MuPAD";xmin
"-10";xmax "10";xviewmin "-10";xviewmax "10";yviewmin "-5";yviewmax
"5";viewset"XY";rangeset"X";plottype 4;labeloverrides 1;x-label "t";axesFont
"Times New Roman,12,0000000000,useDefault,normal";numpoints 100;plotstyle
"patch";axesstyle "normal";axestips FALSE;xis \TEXUX{x};var1name
\TEXUX{$x$};function \TEXUX{$\cos \left( 10.0x\right) +\cos \left(
11.x\right) $};linecolor "green";linestyle 1;pointstyle
"point";linethickness 1;lineAttributes "Solid";var1range
"-10,10";num-x-gridlines 900;curveColor "[flat::RGB:0x00008000]";curveStyle
"Line";rangeset"X";function \TEXUX{$2\cos (\frac{10-11}{2}x)$};linecolor
"blue";linestyle 1;pointstyle "point";linethickness 3;lineAttributes
"Solid";var1range "-10,10";num-x-gridlines 100;curveColor
"[flat::RGB:0x000000ff]";curveStyle "Line";VCamFile
'LTUWDL1U.xvz';valid_file "T";tempfilename
'LTUWCM30.wmf';tempfile-properties "XPR";}}Our ear drum does not care which
way the envelope function goes. We can see that the green (thin line) wave
will push and pull air molecules, and therefore our ear drums, with maximum
loudness at twice this frequency. So we will hear two maxima for every
envelope period!

This frequency with which we hear the sound get loud at a given location as
the wave goes by is called the \emph{beat frequency}. The red envelope
(solid heavy line in the last figure) has a frequency of%
\begin{equation*}
f_{A}=\frac{f_{1}-f_{2}}{2}
\end{equation*}
So our beat frequency is 
\begin{equation*}
f_{beat}=\left\vert f_{1}-f_{2}\right\vert
\end{equation*}%
%TCIMACRO{%
%\TeXButton{Question 223.9.2}{\marginpar {
%\hspace{-0.5in}
%\begin{minipage}[t]{1in}
%\small{Question 223.9.2}
%\end{minipage}
%}}}%
%BeginExpansion
\marginpar {
\hspace{-0.5in}
\begin{minipage}[t]{1in}
\small{Question 223.9.2}
\end{minipage}
}%
%EndExpansion

\section{Non Sinusoidal Waves}

You have probably wondered if all waves are sinusoidal. Can the universe
really be described by such simple mathematics? The answer is both no, and
yes. There are non-sinusoidal waves, in fact, most waves are not sinusoidal.
But it turns out that we can use a very clever mathematical trick to make
any shape wave out of a superposition of many sinusoidal waves. So our
mathematics for sinusoidal waves turns out to be quite general.

\subsection{Music and Non-sinusoidal waves}

Let's take the example of music.

From our example of standing waves on strings, we know that a string can
support a series of standing waves with discrete frequencies--the harmonic
series. We have also discussed that usually we excite the waves with a pluck
or some discrete event, not with an oscillator. Only the harmonic series of
frequencies will resonate, creating standing waves. Other frequencies waves
die out quickly. But there is no reason to suppose that we get energy in
only one standing wave at a time. Most sounds are a combination of harmonics.

The fundamental mode tends to give us the pitch we hear, but what are the
other standing waves for?

To understand, lets take an analogy. Making cookies and cakes.

Here is the beginning of a recipe for cookies.

\FRAME{dtbpF}{3.6011in}{2.0807in}{0in}{}{}{Figure}{\special{language
"Scientific Word";type "GRAPHIC";maintain-aspect-ratio TRUE;display
"USEDEF";valid_file "T";width 3.6011in;height 2.0807in;depth
0in;original-width 3.5535in;original-height 2.0418in;cropleft "0";croptop
"1";cropright "1";cropbottom "0";tempfilename
'S1ET2A03.wmf';tempfile-properties "XPR";}}The recipe is a list of
ingredients, and a symbolic instruction to mix and bake. The product is
chocolate chip cookies. Of course we need more information. We need to know
now much of each ingredient to use.\FRAME{dhF}{3.0052in}{2.3575in}{0pt}{}{}{%
Figure}{\special{language "Scientific Word";type
"GRAPHIC";maintain-aspect-ratio TRUE;display "USEDEF";valid_file "T";width
3.0052in;height 2.3575in;depth 0pt;original-width 2.962in;original-height
2.3177in;cropleft "0";croptop "1";cropright "1";cropbottom "0";tempfilename
'LTUWCM32.wmf';tempfile-properties "XPR";}}This graph gives us the amount of
each ingredient by mass.

Now suppose we want chocolate cake.\FRAME{dtbpF}{3.7265in}{2.3186in}{0in}{}{%
}{Figure}{\special{language "Scientific Word";type
"GRAPHIC";maintain-aspect-ratio TRUE;display "USEDEF";valid_file "T";width
3.7265in;height 2.3186in;depth 0in;original-width 3.6789in;original-height
2.2788in;cropleft "0";croptop "1";cropright "1";cropbottom "0";tempfilename
'S1ET3304.wmf';tempfile-properties "XPR";}}

The predominant taste in each of these foods is chocolate. But chocolate
cake and chocolate chip cookies don't taste exactly the same. We can easily
see that the differences in the other ingredients make the difference
between the \textquotedblleft cookie\textquotedblright\ taste and the
\textquotedblleft cake\textquotedblright\ taste that goes along with the
\textquotedblleft chocolate\textquotedblright\ taste that predominates.

The sound waves produced by musical instruments work in a similar way. Here
is a recipe for an \textquotedblleft A\textquotedblright\ note from a
clarinet.\FRAME{dhF}{3.7879in}{2.7207in}{0in}{}{}{Figure}{\special{language
"Scientific Word";type "GRAPHIC";maintain-aspect-ratio TRUE;display
"USEDEF";valid_file "T";width 3.7879in;height 2.7207in;depth
0in;original-width 3.7395in;original-height 2.6783in;cropleft "0";croptop
"1";cropright "1";cropbottom "0";tempfilename
'LTUWCM34.wmf';tempfile-properties "XPR";}}and here is one for a trumpet
playing the same \textquotedblleft A\textquotedblright\ note.\FRAME{dhF}{%
3.2569in}{2.3021in}{0in}{}{}{Figure}{\special{language "Scientific
Word";type "GRAPHIC";maintain-aspect-ratio TRUE;display "USEDEF";valid_file
"T";width 3.2569in;height 2.3021in;depth 0in;original-width
3.2119in;original-height 2.2623in;cropleft "0";croptop "1";cropright
"1";cropbottom "0";tempfilename 'LTUWCN35.wmf';tempfile-properties "XPR";}}

A trumpet sounds different than a clarinet, and now we see why. There are
more harmonics involved with the trumpet sound than the clarinet sound.
These extra standing waves make up the \textquotedblleft
brassiness\textquotedblright\ of the trumpet sound. As with our baking
example, we need to know how much of each standing wave we have. Each will
have a different amplitude. For our trumpet, we might get amplitudes as
shown.

\FRAME{dhF}{3.0052in}{2.3575in}{0pt}{}{}{Figure}{\special{language
"Scientific Word";type "GRAPHIC";maintain-aspect-ratio TRUE;display
"USEDEF";valid_file "T";width 3.0052in;height 2.3575in;depth
0pt;original-width 2.962in;original-height 2.3177in;cropleft "0";croptop
"1";cropright "1";cropbottom "0";tempfilename
'LTUWCN36.wmf';tempfile-properties "XPR";}}Note that the second harmonic has
a larger amplitude, but we still hear the musical not as \textquotedblleft
A\textquotedblright\ at $440\unit{Hz}.$ A Flugelhorn horn would still sound
brassy, but would have a different mix of harmonics.

The clarinet graph would look quite different, perhaps something like this%
\FRAME{dhF}{2.6697in}{2.0928in}{0pt}{}{}{Figure}{\special{language
"Scientific Word";type "GRAPHIC";maintain-aspect-ratio TRUE;display
"USEDEF";valid_file "T";width 2.6697in;height 2.0928in;depth
0pt;original-width 2.6273in;original-height 2.0539in;cropleft "0";croptop
"1";cropright "1";cropbottom "0";tempfilename
'LY1YVQ02.wmf';tempfile-properties "XPR";}}because it does not have as many
\textquotedblleft ingredients\textquotedblright\ as the trumpet.

All of this should remind you of our analysis of open and closed pipes.
Remember when we closed a pipe, we lost all the even multiples the
fundamental frequency. A similar thing is happening with our instruments.
The rich sound of the brass instrument includes more harmonics and this is
achieved by the shape of the instrument (the flared bell is a big part of
making these extra harmonics and providing the rich trumpet sound).

%TCIMACRO{%
%\TeXButton{Spectrometer Demo}{\marginpar {
%\hspace{-0.5in}
%\begin{minipage}[t]{1in}
%\small{Spectrometer Demo}
%\end{minipage}
%}}}%
%BeginExpansion
\marginpar {
\hspace{-0.5in}
\begin{minipage}[t]{1in}
\small{Spectrometer Demo}
\end{minipage}
}%
%EndExpansion
We have a tool that you can download to your PC to detect the mix of
harmonics of musical instruments, or mechanical systems. In music, the
different harmonics are called \emph{partials} because they make up part of
the sound. A graph that shows which harmonics are involved is called a \emph{%
spectrum}. The next figure is the spectrum of a six holed bamboo flute. Note
that there are several harmonics involved.

\FRAME{dhF}{3.2145in}{2.3445in}{0pt}{}{}{Figure}{\special{language
"Scientific Word";type "GRAPHIC";maintain-aspect-ratio TRUE;display
"USEDEF";valid_file "T";width 3.2145in;height 2.3445in;depth
0pt;original-width 3.1695in;original-height 2.3039in;cropleft "0";croptop
"1";cropright "1";cropbottom "0";tempfilename
'LTUWCN37.wmf';tempfile-properties "XPR";}}Note that our software display
has two parts. One is the instantaneous spectrum, and one is the spectrum
time history.\FRAME{dhF}{4.6267in}{1.9389in}{0pt}{}{}{Figure}{\special%
{language "Scientific Word";type "GRAPHIC";maintain-aspect-ratio
TRUE;display "USEDEF";valid_file "T";width 4.6267in;height 1.9389in;depth
0pt;original-width 4.574in;original-height 1.9009in;cropleft "0";croptop
"1";cropright "1";cropbottom "0";tempfilename
'LTUWCN38.wmf';tempfile-properties "XPR";}}By observing the time history, we
can see changes in the spectrum. We can also see that we don't have pure
harmonics. The graph shows some response off the specific harmonic
frequencies. This six holed flute is very \textquotedblleft
breathy\textquotedblright\ giving a lot of wind noise along with the notes,
and we see this in the spectrum. In the next picture, I played a scale on
the flute.\FRAME{dhF}{2.7674in}{1.8421in}{0pt}{}{}{Figure}{\special{language
"Scientific Word";type "GRAPHIC";maintain-aspect-ratio TRUE;display
"USEDEF";valid_file "T";width 2.7674in;height 1.8421in;depth
0pt;original-width 2.725in;original-height 1.8049in;cropleft "0";croptop
"1";cropright "1";cropbottom "0";tempfilename
'LTUWCN39.wmf';tempfile-properties "XPR";}}The instantaneous spectrum is not
active in this figure (since it can't show more than one note at a time on
the instantaneous graph) but in the time history we see that as the
fundamental frequency changes by shorting the length of the flute
(uncovering holes), we see that every partial also goes up in frequency. The
flute still has the characteristic spectrum of a flute, but shifted to new
set of frequencies. We can use this fact to identify things by their
vibration spectrum. In fact, that is how you recognize voices and
instruments within your auditory system!

The technique of taking apart a wave into its components is very powerful.
With light waves, the spectrum is an indication of the chemical composition
of the emitter. For example, the spectrum of the sun looks something like
this\FRAME{dhFU}{4.6267in}{1.465in}{0pt}{\Qcb{Solar coronal spectrum taken
during a solar eclipse. The successive curved lines are each different
wavelengths, and the dark lines are wavelengths that are absorbed. The
pattern of absorbed wavelengths allows a chemical analysis of the corona.
(Image in the Public Domain, orignally published in Bailey, Solon, L, \emph{%
Popular Science Monthly}, Vol 60, Nov. 1919, pp 244)}}{}{Figure}{\special%
{language "Scientific Word";type "GRAPHIC";maintain-aspect-ratio
TRUE;display "USEDEF";valid_file "T";width 4.6267in;height 1.465in;depth
0pt;original-width 4.574in;original-height 1.4295in;cropleft "0";croptop
"1";cropright "1";cropbottom "0";tempfilename
'LY20PR04.wmf';tempfile-properties "XPR";}}The lines in this graph show the
amplitude of each harmonic component of the light. Darker lines have larger
amplitudes. The harmonics come from the excitation of electrons in their
orbitals. Each orbital is a different energy state, and when the electrons
jump from orbital to orbital, they produce specific wave frequencies. By
observing the mix of dark lines in pervious figure, and comparing to
laboratory measurements from each element (see next figure) we can find the
composition of the source. This figure shows the emission spectrum for
Calcium. Because it is an emission spectrum the lines are bright instead of
dark. We can even see the color of each line!\FRAME{dhFU}{4.7694in}{0.9772in%
}{0pt}{\Qcb{Emission spectrum of Calcium (Image in the Public Domain,
courtesy NASA)}}{}{Figure}{\special{language "Scientific Word";type
"GRAPHIC";maintain-aspect-ratio TRUE;display "USEDEF";valid_file "T";width
4.7694in;height 0.9772in;depth 0pt;original-width 6.2699in;original-height
1.2626in;cropleft "0";croptop "1";cropright "1";cropbottom "0";tempfilename
'LTUWCO3B.wmf';tempfile-properties "XPR";}}

\subsection{Vibrometry}

Just like each atom has a specific spectrum, and each instrument, each
engine, machine, or anything that vibrates has a spectrum. We can use this
to monitor the health of machinery, or even to identify a piece of
equipment. Laser or acoustic vibrometers are available commercially.\FRAME{%
dhFU}{3.7572in}{1.64in}{0pt}{\Qcb{Laser Vibrometer Schematic (Public Domain
Image from Laderaranch:
http://commons.wikimedia.org/wiki/File:LDV\_Schematic.png)}}{}{Figure}{%
\special{language "Scientific Word";type "GRAPHIC";maintain-aspect-ratio
TRUE;display "USEDEF";valid_file "T";width 3.7572in;height 1.64in;depth
0pt;original-width 9.7293in;original-height 4.2309in;cropleft "0";croptop
"1";cropright "1";cropbottom "0";tempfilename
'LX32VJ00.bmp';tempfile-properties "XPR";}}They provide a way to monitor
equipment in places where it would be dangerous or even impossible to send a
person. The equipment also does not need to be shut down, a great benefit
for factories that are never shut down, or for a satellite system that
cannot be reached by anyone.

\subsection{Fourier Series: Mathematics of Non-Sinusoidal Waves}

We should take a quick look at the mathematics of non-sinusoidal waves.

Let' start with a superposition of many sinusoidal waves. The math looks
like this%
\begin{equation*}
y\left( t\right) =\dsum\limits_{n}\left( A_{n}\sin \left( 2\pi f_{n}t\right)
+B_{n}\cos \left( 2\pi f_{n}t\right) \right)
\end{equation*}%
where $A_{n}$ and $B_{n}$ are a series of coefficients and $f_{n}$ are the
harmonic series of frequencies. The coefficients are amplitudes for the many
individual waves making up the complicated wave.

\subsection{Example: Fourier representation of a square wave.}

For example, we could represent a function $f\left( x\right) $ with the
following series%
\begin{eqnarray}
f\left( x\right) &=&C_{o}+C_{1}\cos \left( \frac{2\pi }{\lambda }%
x+\varepsilon _{1}\right) \\
&&+C_{2}\cos \left( \frac{2\pi }{\frac{\lambda }{2}}x+\varepsilon _{2}\right)
\\
&&+C_{3}\cos \left( \frac{2\pi }{\frac{\lambda }{3}}x+\varepsilon _{3}\right)
\\
&&+\ldots \\
&&+C_{n}\cos \left( \frac{2\pi }{\frac{\lambda }{n}}x+\varepsilon _{n}\right)
\\
&&+\ldots
\end{eqnarray}

where we will let $\varepsilon _{i}=\omega _{i}t+\phi _{i}$

The $C^{\prime }s$ are just coefficients that tell us the amplitude of the
individual cosine waves. The more terms in the series we take, the better
the approximation we will have, with the series exactly matching $f\left(
x\right) $ when the number of terms, $N\rightarrow \infty .$

Usually we rewrite the terms of the series as 
\begin{equation}
C_{m}\cos \left( mkx+\varepsilon _{m}\right) =A_{m}\cos \left( mkx\right)
+B_{m}\sin \left( mkx\right)
\end{equation}%
where $k$ is the wavenumber%
\begin{equation}
k=\frac{2\pi }{\lambda }
\end{equation}%
and $\lambda $ is the wavelength of the complicated but still periodic
function $f\left( x\right) .$ Then we identify 
\begin{eqnarray}
A_{m} &=&C_{m}\cos \left( \varepsilon _{m}\right) \\
B_{m} &=&-C_{m}\sin \left( \varepsilon _{m}\right)
\end{eqnarray}%
then%
\begin{equation}
f\left( x\right) =\frac{A_{o}}{2}+\dsum\limits_{m-1}^{\infty }A_{m}\cos
\left( mkx\right) +\dsum\limits_{m-1}^{\infty }B_{m}\sin \left( mkx\right)
\label{Fourier Series}
\end{equation}%
where we separated out the $A_{o}/2$ term because it makes things nicer
later.

\subsubsection{Fourier Analysis}

The process of finding the coefficients of the series is called \emph{%
Fourier analysis}. We start by integrating equation (\ref{Fourier Series})%
\begin{equation}
\int_{0}^{\lambda }f\left( x\right) dx=\int_{0}^{\lambda }\frac{A_{o}}{2}%
dx+\int_{0}^{\lambda }\dsum\limits_{m-1}^{\infty }A_{m}\cos \left(
mkx\right) dx+\int_{0}^{\lambda }\dsum\limits_{m-1}^{\infty }B_{m}\sin
\left( mkx\right) dx
\end{equation}%
We can see immediately that all the sine and cosine terms integrate to zero
(we integrated over a wavelength) so%
\begin{equation}
\int_{0}^{\lambda }f\left( x\right) dx=\int_{0}^{\lambda }\frac{A_{o}}{2}dx=%
\frac{A_{o}}{2}\lambda
\end{equation}%
We solve this for $A_{o}$%
\begin{equation}
A_{o}=\frac{2}{\lambda }\int_{0}^{\lambda }f\left( x\right) dx
\end{equation}

To find the rest of the coefficients we need to remind ourselves of the
orthogonality of sinusoidal functions%
\begin{eqnarray}
\int_{0}^{\lambda }\sin \left( akx\right) \cos \left( bkx\right) dx &=&0 \\
\int_{0}^{\lambda }\cos \left( akx\right) \cos \left( bkx\right) dx &=&\frac{%
\lambda }{2}\delta _{ab} \\
\int_{0}^{\lambda }\sin \left( akx\right) \sin \left( bkx\right) dx &=&\frac{%
\lambda }{2}\delta _{ab}
\end{eqnarray}%
where $\delta _{ab}$ is the Kronecker delta.

To find the coefficients, then, we multiply both sides of equation (\ref%
{Fourier Series}) by $\cos \left( lkx\right) $ where $l$ is a positive
integer. Then we integrate over one wavelength.%
\begin{eqnarray}
\int_{0}^{\lambda }f\left( x\right) \cos \left( lkx\right) dx
&=&\int_{0}^{\lambda }\frac{A_{o}}{2}\cos \left( lkx\right) dx \\
&&+\int_{0}^{\lambda }\dsum\limits_{m-1}^{\infty }A_{m}\cos \left(
mkx\right) \cos \left( lkx\right) dx \\
&&+\int_{0}^{\lambda }\dsum\limits_{m-1}^{\infty }B_{m}\sin \left(
mkx\right) \cos \left( lkx\right) dx
\end{eqnarray}%
which gives 
\begin{equation}
\int_{0}^{\lambda }f\left( x\right) \cos \left( mkx\right)
dx=\int_{0}^{\lambda }A_{m}\cos \left( mkx\right) \cos \left( mkx\right) dx
\end{equation}%
that is, only the term with two cosine functions where $l=m$ will be non
zero. So%
\begin{equation}
\int_{0}^{\lambda }f\left( x\right) \cos \left( mkx\right) dx=\frac{\lambda 
}{2}A_{m}
\end{equation}%
solving for $A_{m}$ we have%
\begin{equation}
A_{m}=\frac{2}{\lambda }\int_{0}^{\lambda }f\left( x\right) \cos \left(
mkx\right) dx
\end{equation}

We can perform the same steps to find $B_{m}$ only we use $\sin \left(
lkx\right) $ as the multiplier. Then we find%
\begin{equation}
B_{m}=\frac{2}{\lambda }\int_{0}^{\lambda }f\left( x\right) \sin \left(
mkx\right) dx
\end{equation}

\subsubsection{Square wave}

Let's find the series for a square wave using our Fourier analysis technique.

Let's take 
\begin{equation}
\lambda =2
\end{equation}%
\begin{equation}
f(x)=\left\{ 
\begin{array}{ccl}
1 & \text{if} & 0<x<\frac{\lambda }{2} \\ 
-1 & \text{if} & \frac{\lambda }{2}<x<\lambda%
\end{array}%
\right.
\end{equation}%
\FRAME{dtbpFX}{2.3644in}{1.5774in}{0pt}{}{}{Plot}{\special{language
"Scientific Word";type "MAPLEPLOT";width 2.3644in;height 1.5774in;depth
0pt;display "USEDEF";plot_snapshots TRUE;mustRecompute FALSE;lastEngine
"MuPAD";xmin "-5.0020002";xmax "5.0020002";xviewmin "-5.0020002";xviewmax
"5.0020002";yviewmin "-2.00080009";yviewmax
"2.00090009";viewset"XY";rangeset"X";plottype 4;axesFont "Times New
Roman,12,0000000000,useDefault,normal";numpoints 100;plotstyle
"patch";axesstyle "normal";axestips FALSE;xis \TEXUX{x};var1name
\TEXUX{$x$};function \TEXUX{$\left\{
\MATRIX{3,2}{c}\VR{,,c,,,}{,,c,,,}{,,l,,,}{,,,,,}\HR{,,}\CELL{1}\CELL{%
\text{if}}\CELL{0<x<\frac{\left( 2\right)
}{2}}\CELL{-1}\CELL{\text{if}}\CELL{\frac{\left( 2\right) }{2}<x<\left(
2\right) }\right. $};linecolor "green";linestyle 1;pointstyle
"point";linethickness 3;lineAttributes "Solid";var1range
"-5.0020002,5.0020002";num-x-gridlines 100;curveColor
"[flat::RGB:0x00008000]";curveStyle "Line";function \TEXUX{$\left\{
\MATRIX{3,2}{c}\VR{,,c,,,}{,,c,,,}{,,l,,,}{,,,,,}\HR{,,}\CELL{1}\CELL{%
\text{if}}\CELL{\left( 2\right) <x<\frac{3\left( 2\right)
}{2}}\CELL{-1}\CELL{\text{if}}\CELL{\frac{3\left( 2\right) }{2}<x<2\left(
2\right) }\right. $};linecolor "green";linestyle 1;pointstyle
"point";linethickness 3;lineAttributes "Solid";var1range
"-5.0020002,5.0020002";num-x-gridlines 100;curveColor
"[flat::RGB:0x00008000]";curveStyle "Line";function \TEXUX{$\left\{
\MATRIX{3,2}{c}\VR{,,c,,,}{,,c,,,}{,,l,,,}{,,,,,}\HR{,,}\CELL{1}\CELL{%
\text{if}}\CELL{2\left( 2\right) <x<\frac{5\left( 2\right)
}{2}}\CELL{-1}\CELL{\text{if}}\CELL{\frac{5\left( 2\right) }{2}<x<3\left(
2\right) }\right. $};linecolor "green";linestyle 1;pointstyle
"point";linethickness 3;lineAttributes "Solid";var1range
"-5.0020002,5.0020002";num-x-gridlines 100;curveColor
"[flat::RGB:0x00008000]";curveStyle "Line";function \TEXUX{$\left\{
\MATRIX{3,2}{c}\VR{,,c,,,}{,,c,,,}{,,l,,,}{,,,,,}\HR{,,}\CELL{-1}\CELL{%
\text{if}}\CELL{0>x>-1\frac{\left( 2\right)
}{2}}\CELL{1}\CELL{\text{if}}\CELL{-\frac{\left( 2\right) }{2}>x>-\left(
2\right) }\right. $};linecolor "green";linestyle 1;pointstyle
"point";linethickness 3;lineAttributes "Solid";var1range
"-5.0020002,5.0020002";num-x-gridlines 100;curveColor
"[flat::RGB:0x00008000]";curveStyle "Line";function \TEXUX{$\left\{
\MATRIX{3,2}{c}\VR{,,c,,,}{,,c,,,}{,,l,,,}{,,,,,}\HR{,,}\CELL{-1}\CELL{%
\text{if}}\CELL{-\left( 2\right) >x>-\frac{3\left( 2\right)
}{2}}\CELL{1}\CELL{\text{if}}\CELL{-\frac{3\left( 2\right) }{2}>x>-2\left(
2\right) }\right. $};linecolor "green";linestyle 1;pointstyle
"point";linethickness 3;lineAttributes "Solid";var1range
"-5.0020002,5.0020002";num-x-gridlines 100;curveColor
"[flat::RGB:0x00008000]";curveStyle "Line";function \TEXUX{$\left\{
\MATRIX{3,2}{c}\VR{,,c,,,}{,,c,,,}{,,l,,,}{,,,,,}\HR{,,}\CELL{-1}\CELL{%
\text{if}}\CELL{-2\left( 2\right) >x>-\frac{5\left( 2\right)
}{2}}\CELL{1}\CELL{\text{if}}\CELL{-\frac{5\left( 2\right) }{2}>x>-3\left(
2\right) }\right. $};linecolor "green";linestyle 1;pointstyle
"point";linethickness 3;lineAttributes "Solid";var1range
"-5.0020002,5.0020002";num-x-gridlines 100;curveColor
"[flat::RGB:0x00008000]";curveStyle "Line";VCamFile
'LTUY0L09.xvz';valid_file "T";tempfilename
'LTUY0L00.wmf';tempfile-properties "XPR";}}

since $f\left( x\right) is$ odd, $A_{m}=0$ for all $m$. We have%
\begin{equation}
B_{m}=\frac{2}{\lambda }\int_{0}^{\frac{\lambda }{2}}\left( 1\right) \sin
\left( mkx\right) dx+\frac{2}{\lambda }\int_{\frac{\lambda }{2}}^{\lambda
}\left( -1\right) \sin \left( mkx\right) dx
\end{equation}%
so%
\begin{equation}
B_{m}=\frac{1}{m\pi }\left( -\cos \left( mkx\right) \mathstrut \right\vert
_{0}^{\frac{\lambda }{2}}+\frac{1}{m\pi }\left( \cos \left( mkx\right)
\mathstrut \right\vert _{\frac{\lambda }{2}}^{\lambda }
\end{equation}%
Which is%
\begin{equation}
B_{m}=\frac{1}{m\pi }\left( 1\cos \left( m\frac{2\pi }{\lambda }x\right)
\mathstrut \right\vert _{0}^{\frac{\lambda }{2}}+\frac{1}{m\pi }\left( \cos
\left( m\frac{2\pi }{\lambda }x\right) \right\vert _{\frac{\lambda }{2}%
}^{\lambda }
\end{equation}%
so%
\begin{eqnarray}
B_{m} &=&\frac{1}{m\pi }\left( \left( -\cos \left( m\frac{2\pi }{\lambda }%
\frac{\lambda }{2}\right) \right) +\cos \left( m\frac{2\pi }{\lambda }\left(
0\right) \right) \right) \\
&&+\frac{1}{m\pi }\left( \left( \cos \left( m\frac{2\pi }{\lambda }\lambda
\right) -\cos \left( m\frac{2\pi }{\lambda }\frac{\lambda }{2}\right)
\right) \right)
\end{eqnarray}%
which is%
\begin{equation}
B_{m}=\frac{2}{m\pi }\left( 1-\cos \left( m\pi \right) \right)
\end{equation}

Our series is then just

\begin{equation}
f\left( x\right) =\dsum\limits_{m-1}^{\infty }\frac{2}{m\pi }\left( 1-\cos
\left( m\pi \right) \right) \sin \left( mkx\right)
\end{equation}%
and we can write a few terms%
\begin{equation}
\begin{tabular}{|l|l|}
\hline
$Term$ &  \\ \hline
1 & $\frac{4}{\pi }\sin \left( kx\right) $ \\ \hline
2 & $0$ \\ \hline
3 & $\frac{4}{3\pi }\sin \left( 3kx\right) $ \\ \hline
4 & $0$ \\ \hline
5 & $\frac{4}{5\pi }\sin \left( 5kx\right) $ \\ \hline
\end{tabular}%
\end{equation}%
then the partial sum up to $m=5$ looks like\FRAME{dtbpFX}{2.3644in}{1.5774in%
}{0pt}{}{}{Plot}{\special{language "Scientific Word";type "MAPLEPLOT";width
2.3644in;height 1.5774in;depth 0pt;display "USEDEF";plot_snapshots
TRUE;mustRecompute FALSE;lastEngine "MuPAD";xmin "-5.1";xmax "5";xviewmin
"-5.1";xviewmax "5";yviewmin "-2";yviewmax
"2.0001";viewset"XY";rangeset"X";plottype 4;axesFont "Times New
Roman,12,0000000000,useDefault,normal";numpoints 100;plotstyle
"patch";axesstyle "normal";axestips FALSE;xis \TEXUX{x};var1name
\TEXUX{$x$};function \TEXUX{$\left\{
\MATRIX{3,2}{c}\VR{,,c,,,}{,,c,,,}{,,l,,,}{,,,,,}\HR{,,}\CELL{1}\CELL{%
\text{if}}\CELL{0<x<\frac{\left( 2\right)
}{2}}\CELL{-1}\CELL{\text{if}}\CELL{\frac{\left( 2\right) }{2}<x<\left(
2\right) }\right. $};linecolor "green";linestyle 1;pointstyle
"point";linethickness 3;lineAttributes "Solid";var1range
"-5.1,5";num-x-gridlines 100;curveColor "[flat::RGB:0x00008000]";curveStyle
"Line";function \TEXUX{$\left\{
\MATRIX{3,2}{c}\VR{,,c,,,}{,,c,,,}{,,l,,,}{,,,,,}\HR{,,}\CELL{1}\CELL{%
\text{if}}\CELL{\left( 2\right) <x<\frac{3\left( 2\right)
}{2}}\CELL{-1}\CELL{\text{if}}\CELL{\frac{3\left( 2\right) }{2}<x<2\left(
2\right) }\right. $};linecolor "green";linestyle 1;pointstyle
"point";linethickness 3;lineAttributes "Solid";var1range
"-5.1,5";num-x-gridlines 100;curveColor "[flat::RGB:0x00008000]";curveStyle
"Line";function \TEXUX{$\left\{
\MATRIX{3,2}{c}\VR{,,c,,,}{,,c,,,}{,,l,,,}{,,,,,}\HR{,,}\CELL{1}\CELL{%
\text{if}}\CELL{2\left( 2\right) <x<\frac{5\left( 2\right)
}{2}}\CELL{-1}\CELL{\text{if}}\CELL{\frac{5\left( 2\right) }{2}<x<3\left(
2\right) }\right. $};linecolor "green";linestyle 1;pointstyle
"point";linethickness 3;lineAttributes "Solid";var1range
"-5.1,5";num-x-gridlines 100;curveColor "[flat::RGB:0x00008000]";curveStyle
"Line";function \TEXUX{$\left\{
\MATRIX{3,2}{c}\VR{,,c,,,}{,,c,,,}{,,l,,,}{,,,,,}\HR{,,}\CELL{-1}\CELL{%
\text{if}}\CELL{0>x>-1\frac{\left( 2\right)
}{2}}\CELL{1}\CELL{\text{if}}\CELL{-\frac{\left( 2\right) }{2}>x>-\left(
2\right) }\right. $};linecolor "green";linestyle 1;pointstyle
"point";linethickness 3;lineAttributes "Solid";var1range
"-5.1,5";num-x-gridlines 100;curveColor "[flat::RGB:0x00008000]";curveStyle
"Line";function \TEXUX{$\left\{
\MATRIX{3,2}{c}\VR{,,c,,,}{,,c,,,}{,,l,,,}{,,,,,}\HR{,,}\CELL{-1}\CELL{%
\text{if}}\CELL{-\left( 2\right) >x>-\frac{3\left( 2\right)
}{2}}\CELL{1}\CELL{\text{if}}\CELL{-\frac{3\left( 2\right) }{2}>x>-2\left(
2\right) }\right. $};linecolor "green";linestyle 1;pointstyle
"point";linethickness 3;lineAttributes "Solid";var1range
"-5.1,5";num-x-gridlines 100;curveColor "[flat::RGB:0x00008000]";curveStyle
"Line";function \TEXUX{$\left\{
\MATRIX{3,2}{c}\VR{,,c,,,}{,,c,,,}{,,l,,,}{,,,,,}\HR{,,}\CELL{-1}\CELL{%
\text{if}}\CELL{-2\left( 2\right) >x>-\frac{5\left( 2\right)
}{2}}\CELL{1}\CELL{\text{if}}\CELL{-\frac{5\left( 2\right) }{2}>x>-3\left(
2\right) }\right. $};linecolor "green";linestyle 1;pointstyle
"point";linethickness 3;lineAttributes "Solid";var1range
"-5.1,5";num-x-gridlines 100;curveColor "[flat::RGB:0x00008000]";curveStyle
"Line";function \TEXUX{$0.424\,41\sin 9.\,\allowbreak 424\,8x+\allowbreak
1.\,\allowbreak 273\,2\sin 3.\,\allowbreak 141\,6x+\allowbreak 0.254\,65\sin
15.\,\allowbreak 708x\allowbreak $};linecolor "blue";linestyle 1;pointstyle
"point";linethickness 1;lineAttributes "Solid";var1range
"-5.1,5";num-x-gridlines 100;curveColor "[flat::RGB:0x000000ff]";curveStyle
"Line";VCamFile 'LTUXYJ07.xvz';valid_file "T";tempfilename
'LTUY0L01.wmf';tempfile-properties "XPR";}}%
\begin{equation}
f\left( x\right) =\frac{4}{\pi }\sin \left( kx\right) +\frac{4}{3\pi }\sin
\left( 3kx\right) +\frac{4}{5\pi }\sin \left( 5kx\right)
\end{equation}%
With twenty terms we would have \FRAME{dtbpFX}{5.1975in}{1.1026in}{0pt}{}{}{%
Plot}{\special{language "Scientific Word";type "MAPLEPLOT";width
5.1975in;height 1.1026in;depth 0pt;display "USEDEF";plot_snapshots
TRUE;mustRecompute FALSE;lastEngine "MuPAD";xmin "-2.5";xmax "2.5";xviewmin
"-2.51";xviewmax "2.5";yviewmin "-2";yviewmax
"2.0001";viewset"XY";rangeset"X";plottype 4;axesFont "Times New
Roman,12,0000000000,useDefault,normal";numpoints 100;plotstyle
"patch";axesstyle "normal";axestips FALSE;xis \TEXUX{x};var1name
\TEXUX{$x$};function \TEXUX{$\left\{
\MATRIX{3,2}{c}\VR{,,c,,,}{,,c,,,}{,,l,,,}{,,,,,}\HR{,,}\CELL{1}\CELL{%
\text{if}}\CELL{0<x<\frac{\left( 2\right)
}{2}}\CELL{-1}\CELL{\text{if}}\CELL{\frac{\left( 2\right) }{2}<x<\left(
2\right) }\right. $};linecolor "green";linestyle 1;pointstyle
"point";linethickness 3;lineAttributes "Solid";var1range
"-2.5,2.5";num-x-gridlines 100;curveColor
"[flat::RGB:0x00008000]";curveStyle "Line";rangeset"X";function
\TEXUX{$\left\{
\MATRIX{3,2}{c}\VR{,,c,,,}{,,c,,,}{,,l,,,}{,,,,,}\HR{,,}\CELL{1}\CELL{%
\text{if}}\CELL{\left( 2\right) <x<\frac{3\left( 2\right)
}{2}}\CELL{-1}\CELL{\text{if}}\CELL{\frac{3\left( 2\right) }{2}<x<2\left(
2\right) }\right. $};linecolor "green";linestyle 1;pointstyle
"point";linethickness 3;lineAttributes "Solid";var1range
"-2.51,2.5";num-x-gridlines 100;curveColor
"[flat::RGB:0x00008000]";curveStyle "Line";function \TEXUX{$\left\{
\MATRIX{3,2}{c}\VR{,,c,,,}{,,c,,,}{,,l,,,}{,,,,,}\HR{,,}\CELL{1}\CELL{%
\text{if}}\CELL{2\left( 2\right) <x<\frac{5\left( 2\right)
}{2}}\CELL{-1}\CELL{\text{if}}\CELL{\frac{5\left( 2\right) }{2}<x<3\left(
2\right) }\right. $};linecolor "green";linestyle 1;pointstyle
"point";linethickness 3;lineAttributes "Solid";var1range
"-2.51,2.5";num-x-gridlines 100;curveColor
"[flat::RGB:0x00008000]";curveStyle "Line";function \TEXUX{$\left\{
\MATRIX{3,2}{c}\VR{,,c,,,}{,,c,,,}{,,l,,,}{,,,,,}\HR{,,}\CELL{-1}\CELL{%
\text{if}}\CELL{0>x>-1\frac{\left( 2\right)
}{2}}\CELL{1}\CELL{\text{if}}\CELL{-\frac{\left( 2\right) }{2}>x>-\left(
2\right) }\right. $};linecolor "green";linestyle 1;pointstyle
"point";linethickness 3;lineAttributes "Solid";var1range
"-2.51,2.5";num-x-gridlines 100;curveColor
"[flat::RGB:0x00008000]";curveStyle "Line";function \TEXUX{$\left\{
\MATRIX{3,2}{c}\VR{,,c,,,}{,,c,,,}{,,l,,,}{,,,,,}\HR{,,}\CELL{-1}\CELL{%
\text{if}}\CELL{-\left( 2\right) >x>-\frac{3\left( 2\right)
}{2}}\CELL{1}\CELL{\text{if}}\CELL{-\frac{3\left( 2\right) }{2}>x>-2\left(
2\right) }\right. $};linecolor "green";linestyle 1;pointstyle
"point";linethickness 3;lineAttributes "Solid";var1range
"-2.51,2.5";num-x-gridlines 100;curveColor
"[flat::RGB:0x00008000]";curveStyle "Line";function \TEXUX{$\left\{
\MATRIX{3,2}{c}\VR{,,c,,,}{,,c,,,}{,,l,,,}{,,,,,}\HR{,,}\CELL{-1}\CELL{%
\text{if}}\CELL{-2\left( 2\right) >x>-\frac{5\left( 2\right)
}{2}}\CELL{1}\CELL{\text{if}}\CELL{-\frac{5\left( 2\right) }{2}>x>-3\left(
2\right) }\right. $};linecolor "green";linestyle 1;pointstyle
"point";linethickness 3;lineAttributes "Solid";var1range
"-2.51,2.5";num-x-gridlines 100;curveColor
"[flat::RGB:0x00008000]";curveStyle "Line";function \TEXUX{$\frac{4}{\pi
}\sin \pi x+\frac{4}{3\pi }\sin 3\pi x+\frac{4}{5\pi }\sin 5\pi
x+\frac{4}{7\pi }\sin 7\pi x+\allowbreak \frac{4}{11\pi }\sin 11\pi
x+\frac{4}{13\pi }\sin 13\pi x+\frac{4}{15\pi }\sin 15\pi x+\allowbreak
\frac{4}{17\pi }\sin 17\pi x+\frac{4}{19\pi }\sin 19\pi x+\frac{4}{9\pi
}\sin 95\pi x\allowbreak $};linecolor "blue";linestyle 1;pointstyle
"point";linethickness 1;lineAttributes "Solid";var1range
"-2.5,2.5";num-x-gridlines 411;curveColor
"[flat::RGB:0x000000ff]";curveStyle "Line";rangeset"X";VCamFile
'LTUY0F08.xvz';valid_file "T";tempfilename
'LTUY0L02.wmf';tempfile-properties "XPR";}}

In the limit of infinitely many waves, the match would be perfect. But we
don't usually need an infinite number of terms. we can pick the part of the
spectrum that best represents the phenomena we desire to observe. For
example, oil based compounds all have specific spectral signatures in the
wavelength range between $3-5$ micrometers. If you wish to tell the
difference between gasoline and crude oil, you can restrict your study to
these wavelengths alone.

\section{Frequency Uncertainty for Signals and Particles}

Up to this lecture, when we thought of a wave we have mostly thought of
something like this 
\begin{equation*}
y=y_{\max }\sin \left( kx-\omega t-\phi \right)
\end{equation*}%
which in practice might look like this\FRAME{dtbpF}{4.4996in}{1.2211in}{0pt}{%
}{}{Plot}{\special{language "Scientific Word";type "MAPLEPLOT";width
4.4996in;height 1.2211in;depth 0pt;display "USEDEF";plot_snapshots
TRUE;mustRecompute FALSE;lastEngine "MuPAD";xmin "-5";xmax "5";xviewmin
"-5.0010000010002";xviewmax "5.0010000010002";yviewmin "-10";yviewmax
"10";plottype 4;axesFont "Times New
Roman,12,0000000000,useDefault,normal";numpoints 100;plotstyle
"patch";axesstyle "normal";axestips FALSE;xis \TEXUX{x};var1name
\TEXUX{$x$};function \TEXUX{$10\sin \left( 5x-0\right) $};linecolor
"blue";linestyle 1;pointstyle "point";linethickness 3;lineAttributes
"Solid";var1range "-5,5";num-x-gridlines 100;curveColor
"[flat::RGB:0x000000ff]";curveStyle "Line";VCamFile
'S1ESU40C.xvz';valid_file "T";tempfilename
'S1ESU401.wmf';tempfile-properties "XPR";}}And we have noticed that there is
no start or stop to this kind of wave. Our figure starts at $x=-5$ and ends
at $x=5,$ but the equation does not! There is a value of $y$ for every $x$
from $-\infty $ to +$\infty .$ But many signals are not such waves. They may
be very limited in size.\FRAME{dtbpF}{2.0833in}{0.8172in}{0pt}{}{}{Figure}{%
\special{language "Scientific Word";type "GRAPHIC";maintain-aspect-ratio
TRUE;display "USEDEF";valid_file "T";width 2.0833in;height 0.8172in;depth
0pt;original-width 2.0942in;original-height 0.8036in;cropleft "0";croptop
"1";cropright "1";cropbottom "0";tempfilename
'../../../../PH279/Repository/IntroductoryPhysicsIV/Textbook/FigureLimetedWave2.wmf';tempfile-properties "XNPR";}%
}We should investigate what happens when you have a limited wave. I did this
investigation using Python. Suppose we have a sine wave with $f=200\unit{Hz}%
, $ but I\ limit this wave's existence by making it start at $t_{i}=0$ and
then make it end at $t_{f}=10\unit{s}.$ I could do this in practice by
turning on a radio transmission or even an acoustic speaker, and then
turning the device off ten seconds later. Our screen resolution is terrible
for plotting such a function, but in the figure below you can see that our
signal only exists from $t=0$ to $t=10\unit{s}.$ \FRAME{dtbpF}{5.0202in}{%
0.6685in}{0pt}{}{}{Plot}{\special{language "Scientific Word";type
"MAPLEPLOT";width 5.0202in;height 0.6685in;depth 0pt;display
"USEDEF";plot_snapshots TRUE;mustRecompute FALSE;lastEngine "MuPAD";xmin
"-5.001000";xmax "20";xviewmin "-2.001000";xviewmax "12";yviewmin
"-1.1";yviewmax "1.1";viewset"XY";rangeset"X";plottype 4;axesFont "Times New
Roman,12,0000000000,useDefault,normal";numpoints 100;plotstyle
"patch";axesstyle "normal";axestips FALSE;xis \TEXUX{t};var1name
\TEXUX{$t$};function \TEXUX{$\left\{
\MATRIX{3,3}{c}\VR{,,l,,,}{,,l,,,}{,,l,,,}{,,,,,}\HR{,,,}\CELL{0}\CELL{%
\text{if}}\CELL{t<0}\CELL{1\sin (400\pi t)}\CELL{\text{if}}\CELL{0\leq t\leq
10}\CELL{0}\CELL{\text{if}}\CELL{t>10}\right. $};linecolor "blue";linestyle
1;pointstyle "point";linethickness 3;lineAttributes "Solid";var1range
"-5.001000,20";num-x-gridlines 901;curveColor
"[flat::RGB:0x000000ff]";curveStyle "Line";rangeset"X";VCamFile
'../../../../PH279/Repository/IntroductoryPhysicsIV/Textbook/QBQ5CO5X.xvz';valid_file "T";tempfilename 'S1ESU402.wmf';tempfile-properties "XPR";}%
}If I\ zoom in on a part of the graph, we can see that it is really a sin
wave.\FRAME{dtbpF}{5.0202in}{0.6685in}{0pt}{}{}{Plot}{\special{language
"Scientific Word";type "MAPLEPLOT";width 5.0202in;height 0.6685in;depth
0pt;display "USEDEF";plot_snapshots TRUE;mustRecompute FALSE;lastEngine
"MuPAD";xmin "-5.001000";xmax "20";xviewmin "2.00000";xviewmax
"2.5";yviewmin "-1.1";yviewmax "1.1";viewset"XY";rangeset"X";plottype
4;axesFont "Times New Roman,12,0000000000,useDefault,normal";numpoints
100;plotstyle "patch";axesstyle "normal";axestips FALSE;xis
\TEXUX{t};var1name \TEXUX{$t$};function \TEXUX{$\left\{
\MATRIX{3,3}{c}\VR{,,l,,,}{,,l,,,}{,,l,,,}{,,,,,}\HR{,,,}\CELL{0}\CELL{%
\text{if}}\CELL{t<0}\CELL{1\sin (400\pi t)}\CELL{\text{if}}\CELL{0\leq t\leq
10}\CELL{0}\CELL{\text{if}}\CELL{t>10}\right. $};linecolor "blue";linestyle
1;pointstyle "point";linethickness 3;lineAttributes "Solid";var1range
"-5.001000,20";num-x-gridlines 700;curveColor
"[flat::RGB:0x000000ff]";curveStyle "Line";rangeset"X";VCamFile
'../../../../PH279/Repository/IntroductoryPhysicsIV/Textbook/QBQ5CO5W.xvz';valid_file "T";tempfilename 'S1ESU400.wmf';tempfile-properties "XPR";}%
}

Python did equally bad at plotting this. All we see is a blue band. \FRAME{%
dtbpF}{4.3388in}{1.8317in}{0pt}{}{}{Figure}{\special{language "Scientific
Word";type "GRAPHIC";maintain-aspect-ratio TRUE;display "USEDEF";valid_file
"T";width 4.3388in;height 1.8317in;depth 0pt;original-width
4.3932in;original-height 1.8378in;cropleft "0";croptop "1";cropright
"1";cropbottom "0";tempfilename
'../../../../PH279/Repository/IntroductoryPhysicsIV/Textbook/Figure10s200Hz.wmf';tempfile-properties "XNPR";}%
}But in the second graph, notice that we have plotted frequency. Python and
most scientific programing languages have functions to find a digital
version of a Fourier transform to produce a spectrum. It takes in a signal
and finds which frequencies are in a signal. It performs the job of a
spectrometer, so we would call the figure to the right a spectrogram (or
just a spectrum).

We expect only one frequency, $200\unit{Hz},$ and that is mostly what we
get. Since our period for our wave is 
\begin{equation*}
T=\frac{1}{200\unit{Hz}}=0.005\,\unit{s}
\end{equation*}%
and we have $10\unit{s}$ of data, that is four orders of magnitude more
signal than a period. The whole signal seems very long compared to a period.
We expect this to look kind of like an infinite signal. But suppose we take
the same wave, but for less time. We limit the wave more.\FRAME{dtbpF}{%
3.8104in}{1.5549in}{0pt}{}{}{Figure}{\special{language "Scientific
Word";type "GRAPHIC";maintain-aspect-ratio TRUE;display "USEDEF";valid_file
"T";width 3.8104in;height 1.5549in;depth 0pt;original-width
3.8548in;original-height 1.5558in;cropleft "0";croptop "1";cropright
"1";cropbottom "0";tempfilename
'../../../../PH279/Repository/IntroductoryPhysicsIV/Textbook/Figure1s200Hz.wmf';tempfile-properties "XNPR";}%
}So we still get a blue blur for our wave picture, but now the wave only
exists for one instead of ten seconds. If you look closely at the frequency
graph, you will notice that the $200\unit{Hz}$ peak representing our wave is
a bit wider right at the bottom.

We could limit our wave more, say, so it only lasts $t_{f}=0.1\unit{s}.$ We
would get a set of graphs that look like this. \FRAME{dtbpF}{3.7351in}{%
1.4788in}{0pt}{}{}{Figure}{\special{language "Scientific Word";type
"GRAPHIC";maintain-aspect-ratio TRUE;display "USEDEF";valid_file "T";width
3.7351in;height 1.4788in;depth 0pt;original-width 3.7776in;original-height
1.4786in;cropleft "0";croptop "1";cropright "1";cropbottom "0";tempfilename
'../../../../PH279/Repository/IntroductoryPhysicsIV/Textbook/Figure0.1s200Hz.wmf';tempfile-properties "XNPR";}%
}

Notice that not only can Python render the wave now, but more importantly
the $200\unit{Hz}$ frequency peak is noticeable wider. This is profound! It
means that by limiting the wave, we no longer have just one frequency! The
graph tells us we have mostly $200\unit{Hz}$ but we also have some $199\unit{%
Hz}$ and some $201\unit{Hz}$ and some $190\unit{Hz}$ and some $210\unit{Hz},$
etc. The very fact that the wave does not go on so long requires that we
have more than one frequency in the wave. We could say that as $\Delta t$
gets smaller, our $\Delta f$ is getting bigger. Here are two more examples
with smaller $\Delta t$ values. \FRAME{dtbpF}{3.9288in}{1.4866in}{0pt}{}{}{%
Figure}{\special{language "Scientific Word";type
"GRAPHIC";maintain-aspect-ratio TRUE;display "USEDEF";valid_file "T";width
3.9288in;height 1.4866in;depth 0pt;original-width 3.9746in;original-height
1.4866in;cropleft "0";croptop "1";cropright "1";cropbottom "0";tempfilename
'../../../../PH279/Repository/IntroductoryPhysicsIV/Textbook/Figure0.05s200Hz.wmf';tempfile-properties "XNPR";}%
}\FRAME{dtbpF}{4.6285in}{1.8152in}{0pt}{}{}{Figure}{\special{language
"Scientific Word";type "GRAPHIC";maintain-aspect-ratio TRUE;display
"USEDEF";valid_file "T";width 4.6285in;height 1.8152in;depth
0pt;original-width 3.3164in;original-height 1.2817in;cropleft "0";croptop
"1";cropright "1";cropbottom "0";tempfilename
'../../../../PH279/Repository/IntroductoryPhysicsIV/Textbook/Figure0.005s200Hz.wmf';tempfile-properties "XNPR";}%
}

The cost of limiting our waves is that we can't have a single frequency for
the wave. For an engineer, this means that if you only measure a short
segment of the signal, you have an increased uncertainty in the frequency
you will find from that signal. For chemists, it means that when we look at
quantum wave functions we can expect uncertainty in the frequency (or
wavelength) because the quantum particle (like an electron) is limited. It
is important to know what we mean by uncertainty in this case. In our
example above, when we say that $\Delta f$ increased we really mean that we
have more than one frequency. We don't just mean that we don't know the
frequency well. We really are mixing more than one frequency. This idea of
limiting waves creating uncertainty shows up in physical chemistry as
Heisenberg's uncertainty principle.

\chapter{Interference of light waves}

%TCIMACRO{%
%\TeXButton{Fundamental Concepts}{\hspace{-1.3in}{\LARGE Fundamental Concepts\vspace{0.25in}}}}%
%BeginExpansion
\hspace{-1.3in}{\LARGE Fundamental Concepts\vspace{0.25in}}%
%EndExpansion

\begin{itemize}
\item Light is a wave in the electromagnetic field

\item Light is a superposition of many small waves called photons

\item The energy in a photon is proportional to the frequency of the photon

\item If we mix two coherent light sources, we get interference, with an
intensity pattern given by $I=I_{\max }\cos ^{2}\left( \frac{1}{2}\left( 
\frac{2\pi }{\lambda }d\sin \theta \right) \right) $

\item Most detectors cannot follow the fluctuation of light because their
integration time is too long.
\end{itemize}

\section{The Nature of Light}

\subsection{Physical Ideas of the nature of Light}

Before the 19th century (1800's) light was assumed to be a stream of
particles. Newton was the chief proponent of this theory. The theory was
able to explain reflection of light from mirrors and other objects and
therefore explain vision. In 1678 Huygens showed that wave theory could also
explain reflection and vision.

In 1801 Thomas Young demonstrated that light had attributes that were best
explained by wave theory. We will study Young's experiment later today. The
crux of his experiment was to show that light displayed constructive and
destructive interference--clearly a wave phenomena! The theory of the nature
of light took a dramatic shift

In 1805 Joseph Smith was born in Sharon, Vermont.

In September of 1832 Joseph Smith received a revelation that said in part :

\begin{quotation}
For the word of the Lord is truth, and whatsoever is truth is light, and
whatsoever is light is Spirit, even the Spirit of Jesus Christ. And the
Spirit giveth light to every man that cometh into the world; and the Spirit
enlighteneth every man through the world, that hearkeneth to the voice of
the Spirit. (D\&C 84:45-46)
\end{quotation}

In December of 1832 Joseph Smith received another revelation that says in
part:

\begin{quotation}
This Comforter is the promise which I give unto you of eternal life, even
the glory of the celestial kingdom; which glory is that of the church of the
Firstborn, even of God, the holiest of all, through Jesus Christ his
Son---He that ascended up on high, as also he descended below all things, in
that he comprehended all things, that he might be in all and through all
things, the light of truth; which truth shineth. This is the light of
Christ. As also he is in the sun, and the light of the sun, and the power
thereof by which it was made. As also he is in the moon, and is the light of
the moon, and the power thereof by which it was made; as also the light of
the stars, and the power thereof by which they were made; and the earth
also, and the power thereof, even the earth upon which you stand. And the
light which shineth, which giveth you light, is through him who enlighteneth
your eyes, which is the same light that quickeneth your understandings;
which light proceedeth forth from the presence of God to fill the immensity
of space---the light which is in all things, which giveth life to all
things, which is the law by which all things are governed, even the power of
God who sitteth upon his throne, who is in the bosom of eternity, who is in
the midst of all things. (D\&C 88:5-12)
\end{quotation}

Light, even real, physical light, seems to be of interest to Latter Day
Saints.

In 1847 the saints entered the Salt Lake Valley.

In 1873 Maxwell published his findings that light is an electromagnetic wave
(something we will try to show before this course is over!).

Planck's work in quantization theory (1900) was used by Einstein In 1905 to
give an explantation of the photoelectric effect that again made light look
like a particle.

Current theory allows light to exhibit the characteristics of a wave in some
situations and like a particle in others. We will study both before the end
of the semester.

The results of Einstein's work give us the concept of a \emph{photon} or a
quantized unit of radiant energy. Each \textquotedblleft piece of
light\textquotedblright\ or photon has energy 
\begin{equation}
E=hf
\end{equation}%
where $f$ is the frequency of the light and $h$ is a constant 
\begin{equation}
h=6.63\times 10^{-34}\unit{J}\unit{s}
\end{equation}

%TCIMACRO{%
%\TeXButton{Question 223.10.1}{\marginpar {
%\hspace{-0.5in}
%\begin{minipage}[t]{1in}
%\small{Question 223.10.1}
%\end{minipage}
%}}}%
%BeginExpansion
\marginpar {
\hspace{-0.5in}
\begin{minipage}[t]{1in}
\small{Question 223.10.1}
\end{minipage}
}%
%EndExpansion
%TCIMACRO{%
%\TeXButton{Question 223.10.2}{\marginpar {
%\hspace{-0.5in}
%\begin{minipage}[t]{1in}
%\small{Question 223.10.2}
%\end{minipage}
%}}}%
%BeginExpansion
\marginpar {
\hspace{-0.5in}
\begin{minipage}[t]{1in}
\small{Question 223.10.2}
\end{minipage}
}%
%EndExpansion
The nature of light is fascinating and useful both in physical and religious
areas of thought.

\section{Measurements of the Speed of Light}

One of the great foundations of modern physical theory is that the speed of
light is constant in a vacuum. Galileo first tried to measure the speed of
light. He used two towers in town and placed a lantern and an assistant on
each tower. The lanterns had shades. The plan was for one assistant to
remove his shade, and then for the assistant on the other tower to remove
his shade as soon as he saw the light from the first lantern. Back at the
first tower, the first assistant would use a clock to determine the time
difference between when the first lantern was un-shaded, and when they saw
the light from the second tower. The light would have traveled twice the
inter-tower distance. Dividing that distance by the time would give the
speed of light. You can probably guess that this did not work. Light travels
very quickly. The clocks of Galileo's day could not measure such a small
time difference. Ole R\o mer was the first to succeed in measuring the speed
of light.

\subsection{R\o mer's Measurement of the speed of light}

\FRAME{dhFU}{1.1761in}{2.0626in}{0pt}{\Qcb{A diagram illustrating R\o mer's
determination of the speed of light. Point A is the Sun, piont B is Jupiter.
Point C is the immersion of Io into Jupiter's shadow at the start of an
eclipse}}{}{Figure}{\special{language "Scientific Word";type
"GRAPHIC";maintain-aspect-ratio TRUE;display "USEDEF";valid_file "T";width
1.1761in;height 2.0626in;depth 0pt;original-width 1.7252in;original-height
3.0468in;cropleft "0";croptop "1";cropright "1";cropbottom "0";tempfilename
'LX1RAN04.wmf';tempfile-properties "XPR";}}R\o mer performed his measurement
in 1675, $\allowbreak 269$ years before digital devices existed!. He used
the period of revolution of Io, a moon of Jupiter, as Jupiter revolved
around the sun. He first measured the period of Io's rotation about Jupiter,
then he predicted an eclipse of Io three months later. But he found his
calculation was off by $600\unit{s}.$ After careful thought, he realized
that the Earth had moved in its orbit, and that the light had to travel the
extra distance due to the Earth's new position. Given R\o mer's best
estimate for the orbital radius of the earth and his time difference, R\o %
mer arrived at a estimate of $c=2.3\times 10^{8}\frac{\unit{m}}{\unit{s}}.$
Amazing for 1675!

\subsection{Fizeau's Measurement of the speed of light}

\FRAME{dtbpF}{2.5659in}{1.6881in}{0pt}{}{}{Figure}{\special{language
"Scientific Word";type "GRAPHIC";maintain-aspect-ratio TRUE;display
"USEDEF";valid_file "T";width 2.5659in;height 1.6881in;depth
0pt;original-width 2.5253in;original-height 1.6518in;cropleft "0";croptop
"1";cropright "1";cropbottom "0";tempfilename
'MFGL0I00.wmf';tempfile-properties "XPR";}}Hippolyte Fizeau measured the
speed of light in 1849 using a toothed wheel and a mirror and a beam of
light. The light passed through the open space in the wheel's teeth as the
wheel rotated. Then was reflected by the mirror. The speed would be%
\begin{equation*}
v=\frac{\Delta x}{\Delta t}
\end{equation*}%
We just need $\Delta x$ and $\Delta t.$

It is easy to see that 
\begin{equation*}
\Delta x=2d
\end{equation*}%
because the light travels twice the distance to the mirror ($d$) and back.
So the speed is just 
\begin{equation*}
v=\frac{2d}{\Delta t}
\end{equation*}

If the wheel turned just at the right angular speed, then the reflected
light would hit the next tooth and be blocked. Think of angular speed%
\begin{equation*}
\omega =\frac{\Delta \theta }{\Delta t}
\end{equation*}%
so the time difference would be 
\begin{equation*}
\Delta t=\frac{\Delta \theta }{\omega }
\end{equation*}%
we find $\Delta \theta $ by taking the number of teeth on the wheel and
dividing by $2\pi $ by that number.

Then the speed of light must be 
\begin{eqnarray*}
c &=&v=\frac{2d}{\frac{\Delta \theta }{\omega }} \\
&=&\frac{2d\omega }{\Delta \theta } \\
&=&\frac{2d\omega N_{teeth}}{2\pi } \\
&=&\frac{d\omega N_{teeth}}{\pi }
\end{eqnarray*}

then if we have $720$ teeth and $\omega $ is measured to be $d=7500\unit{m}$%
\begin{eqnarray*}
c &=&\frac{\left( 7500\unit{m}\right) \left( 172.\,\allowbreak 79\unit{Hz}%
\right) \left( 720\right) }{\pi } \\
&=&2.\,\allowbreak 97\times 10^{8}\frac{\unit{m}}{\unit{s}}
\end{eqnarray*}%
which is Fizeau's number and it is pretty good!

Modern measurements are performed in very much the same way that Fizeau did
his calculation. The current value is 
\begin{equation}
c=2.9979\times 10^{8}\frac{\unit{m}}{\unit{s}}
\end{equation}

\subsection{Faster than light}

%TCIMACRO{%
%\TeXButton{Pass the photon (ball) demo}{\marginpar {
%\hspace{-0.5in}
%\begin{minipage}[t]{1in}
%\small{Pass the photon (ball) demo}
%\end{minipage}
%}}}%
%BeginExpansion
\marginpar {
\hspace{-0.5in}
\begin{minipage}[t]{1in}
\small{Pass the photon (ball) demo}
\end{minipage}
}%
%EndExpansion
%TCIMACRO{%
%\TeXButton{Question 223.10.3}{\marginpar {
%\hspace{-0.5in}
%\begin{minipage}[t]{1in}
%\small{Question 223.10.3}
%\end{minipage}
%}}}%
%BeginExpansion
\marginpar {
\hspace{-0.5in}
\begin{minipage}[t]{1in}
\small{Question 223.10.3}
\end{minipage}
}%
%EndExpansion
The speed of light in a vacuum is constant, but in matter the speed of light
changes. 
%TCIMACRO{%
%\TeXButton{Question 223.10.4}{\marginpar {
%\hspace{-0.5in}
%\begin{minipage}[t]{1in}
%\small{Question 223.10.4}
%\end{minipage}
%}} }%
%BeginExpansion
\marginpar {
\hspace{-0.5in}
\begin{minipage}[t]{1in}
\small{Question 223.10.4}
\end{minipage}
}
%EndExpansion
We will study this in detail when we look at refraction. But for now, a
dramatic example is Cherenkov radiation. It is an eerie blue glow around the
core of nuclear reactors. It occurs when electrons are accelerated past the
speed of light in the water surrounding the core. The electrons emit light
and the light waves form a Doppler cone or a light-sonic boom! The result is
the blue glow.

\bigskip

\FRAME{dtbpFUw}{2.9464in}{2.2035in}{0pt}{\Qcb{Cherenkov radiation from a
250kW TRIGA reactor. (Image in the Public Domain, courtesy US\ Department of
Energy) }}{}{Figure}{\special{language "Scientific Word";type
"GRAPHIC";maintain-aspect-ratio TRUE;display "USEDEF";valid_file "T";width
2.9464in;height 2.2035in;depth 0pt;original-width 6.2595in;original-height
4.6717in;cropleft "0";croptop "1";cropright "1";cropbottom "0";tempfilename
'LTUWCO3E.wmf';tempfile-properties "XPR";}}

This does bring up a problem in terminology. What does the word
\textquotedblleft medium\textquotedblright\ mean? We have used it to mean
the substance through which a wave travels. This substance must have the
property of transferring energy between it's parts, like the coils of a
spring can transfer energy to each other, or like air molecules can transfer
energy by collision. For light the wave medium is the electromagnetic field.
This field can store and transfer energy (we will see this later in the
course). But many books on physics call materials like glass a
\textquotedblleft medium\textquotedblright\ through which light travels. The
water in our last example is such a medium. Are glass and water wave mediums
for light? The answer is no. Light does not need any matter to form it's
wave. The wave medium is the electromagnetic field. So we will have to keep
this in mind as we allow light to travel through matter. We may call the
matter a \textquotedblleft medium,\textquotedblright\ but it is not the wave
medium.

\section{Interference and Young's Experiment}

Waves do some funny things when they encounter barriers. Think of a water
wave. If we pass the wave through a small opening in a barrier, the wave
can't all get through the small hole, but it can cause a disturbance. We
know that a small disturbance will cause a wave. But this wave will be due
to a very small--almost point--source. So the waves will be spherical
leaving the opening. The smaller the opening the more pronounced the curving
of the wave, because the source (the hole) is more like a point source.%
\FRAME{dhF}{2.1378in}{2.0375in}{0pt}{}{}{Figure}{\special{language
"Scientific Word";type "GRAPHIC";maintain-aspect-ratio TRUE;display
"USEDEF";valid_file "T";width 2.1378in;height 2.0375in;depth
0pt;original-width 2.0989in;original-height 1.9986in;cropleft "0";croptop
"1";cropright "1";cropbottom "0";tempfilename
'LTUWCO3F.wmf';tempfile-properties "XPR";}}Now suppose we have two of these
openings. We expect the two sources to make curved waves and those waves can
interfere. \FRAME{dhF}{2.2139in}{2.5901in}{0pt}{}{}{Figure}{\special%
{language "Scientific Word";type "GRAPHIC";maintain-aspect-ratio
TRUE;display "USEDEF";valid_file "T";width 2.2139in;height 2.5901in;depth
0pt;original-width 2.962in;original-height 3.4696in;cropleft "0";croptop
"1";cropright "1";cropbottom "0";tempfilename
'LTUWCO3G.wmf';tempfile-properties "XPR";}}In the figure, we can already see
that there will be constructive and destructive interference were the waves
from the two holes meet. Thomas young predicted that we should see
constructive and destructive interference in light (he drew figures very
like the ones we have used to explain his idea). \FRAME{dhF}{2.3618in}{%
2.6654in}{0pt}{}{}{Figure}{\special{language "Scientific Word";type
"GRAPHIC";maintain-aspect-ratio TRUE;display "USEDEF";valid_file "T";width
2.3618in;height 2.6654in;depth 0pt;original-width 2.322in;original-height
2.623in;cropleft "0";croptop "1";cropright "1";cropbottom "0";tempfilename
'LTUWCO3H.wmf';tempfile-properties "XPR";}}

Young set up a coherent source of light and placed it in front of this
source a barrier with two very thin slits cut in it to test his idea.. He
set up a screen beyond the barrier and observed the pattern on the screen
formed by the light. This (in part) is what he saw%
%TCIMACRO{%
%\TeXButton{Young's Experiment demo}{\marginpar {
%\hspace{-0.5in}
%\begin{minipage}[t]{1in}
%\small{Young's Experiment demo}
%\end{minipage}
%}}}%
%BeginExpansion
\marginpar {
\hspace{-0.5in}
\begin{minipage}[t]{1in}
\small{Young's Experiment demo}
\end{minipage}
}%
%EndExpansion

\bigskip

\FRAME{dhF}{2.3618in}{0.7671in}{0pt}{}{}{Figure}{\special{language
"Scientific Word";type "GRAPHIC";maintain-aspect-ratio TRUE;display
"USEDEF";valid_file "T";width 2.3618in;height 0.7671in;depth
0pt;original-width 2.322in;original-height 0.7351in;cropleft "0";croptop
"1";cropright "1";cropbottom "0";tempfilename
'LTUWCO3I.wmf';tempfile-properties "XPR";}}%
%TCIMACRO{%
%\TeXButton{Question 223.10.5}{\marginpar {
%\hspace{-0.5in}
%\begin{minipage}[t]{1in}
%\small{Question 223.10.5}
%\end{minipage}
%}}}%
%BeginExpansion
\marginpar {
\hspace{-0.5in}
\begin{minipage}[t]{1in}
\small{Question 223.10.5}
\end{minipage}
}%
%EndExpansion
We see bright spots (constructive interference) and dark spots (destructive
interference). Only wave phenomena can interfere, so this is fairly good
evidence that light is a wave.

\subsection{Constructive Interference}

We can find the condition for getting a bright or a dark band if we think
about it a bit. For constructive interference, the difference in phase, $%
\Delta \phi ,$ must be a multiple of $2\pi .$ That means the path difference
between the two slit-sources must be an even number of wavelengths. We have
been calling the path difference in the total phase $\Delta x,$ or for
spherical waves $\Delta r,$ but in optics it is customary to call this path
difference $\delta .$ So%
\begin{equation*}
\delta =\Delta r
\end{equation*}%
\FRAME{dhF}{5.3186in}{2.5849in}{0pt}{}{}{Figure}{\special{language
"Scientific Word";type "GRAPHIC";maintain-aspect-ratio TRUE;display
"USEDEF";valid_file "T";width 5.3186in;height 2.5849in;depth
0pt;original-width 3.6564in;original-height 1.7634in;cropleft "0";croptop
"1";cropright "1";cropbottom "0";tempfilename
'LTUWCO3J.wmf';tempfile-properties "XPR";}}and our total phase equation
becomes%
\begin{equation*}
\Delta \phi =k\delta +\Delta \phi _{o}
\end{equation*}%
Then, if the screen is much farther away than $d,$ the slit distance, we can
say that the blue triangle is almost a right triangle, and then $\delta $ is 
\begin{equation*}
\delta =r_{2}-r_{1}\approx d\sin \theta
\end{equation*}%
Our wave repeats every $2\pi $ radians or every wavelength, $\lambda ,$ then
we have constructive interference (a bright spot) when 
\begin{equation*}
\delta =d\sin \theta =m\lambda \quad \left( m=0,\pm 1,\pm 2\ldots \right)
\end{equation*}%
where $m$ is called the \emph{order number}. 
%TCIMACRO{%
%\TeXButton{Question 223.10.6}{\marginpar {
%\hspace{-0.5in}
%\begin{minipage}[t]{1in}
%\small{Question 223.10.6}
%\end{minipage}
%}} }%
%BeginExpansion
\marginpar {
\hspace{-0.5in}
\begin{minipage}[t]{1in}
\small{Question 223.10.6}
\end{minipage}
}
%EndExpansion
That is, if we are off by any number of whole wavelengths then our total
phase due to path difference will be $2\pi $. In our experimental setup, we
have guaranteed that $\Delta \phi _{o}=0$ by starting out with one large
wave that hits two openings. Both openings will have the same $\phi _{o}$
because they experience the same wave hitting them.

If we assume that $\lambda \ll d$ we can find the distance from the axis for
each fringe more easily. This condition guarantees that $\theta $ will be
small. Using the yellow triangle we see%
\begin{equation*}
\tan \theta =\frac{y}{L}
\end{equation*}%
but if $\theta $ is small this is just about the same as 
\begin{equation*}
\sin \theta =\frac{y}{L}
\end{equation*}%
because for small angles $\tan \theta \approx \sin \theta \approx \theta .$
So if theta is small then 
\begin{eqnarray*}
\delta &=&d\sin \theta \\
&=&d\frac{y}{L}
\end{eqnarray*}%
and for a bright spot or \emph{"fringe"} we find 
\begin{equation*}
d\frac{y}{L}=m\lambda
\end{equation*}%
Solving for the position of the bright spots gives%
\begin{equation}
y_{bright}=\frac{\lambda L}{d}m\quad \left( m=0,\pm 1,\pm 2\ldots \right)
\end{equation}%
We can measure up from the central spot and predict where each successive
bright spot will be.

\subsection{Destructive Interference}

We can also find a condition for destructive interference. We know that a
path difference of an odd multiple of a half wavelength will give
distractive interference. so 
\begin{equation*}
\delta =d\sin \theta =\left( m+\frac{1}{2}\right) \lambda \quad \left(
m=0,\pm 1,\pm 2\ldots \right)
\end{equation*}%
will give a dark fringe. The location of the dark fringes will be 
\begin{equation}
y_{dark}=\frac{\lambda L}{d}\left( m+\frac{1}{2}\right) \quad \left( m=0,\pm
1,\pm 2\ldots \right)
\end{equation}

\section{Double Slit Intensity Pattern}

The fringes we have seen are not just points, but are patterns that fade
from a maximum intensity. We can calculate the intensity pattern. We need to
know a little bit about electric fields to do this.

\subsection{Electric field preview/review}

We can represent an electromagnetic wave using just the electric field (the
magnetic field pattern is very similar and can be derived from the electric
field pattern).

We represent the field by an equation like%
\begin{equation*}
y=y_{o}\sin \left( kr-\omega t\right)
\end{equation*}%
but since the medium for light waves is the electric field, let's use the
symbol $E$ instead of $y$ so we can see that we have a change in the field
strength and not a displacement of some material thing.%
\begin{equation}
E=E_{\max }\sin \left( kr-\omega t\right)
\end{equation}%
where the amplitude of the wave is $E_{\max }$ and $\omega $ is the angular
frequency. This is just our traveling wave equation, but with electric field
strength, labeled $E,$ for the amplitude.

Then to find the intensity pattern, we take two waves in the electric field,
one from slit one 
\begin{equation*}
E_{1}=E_{\max }\sin \left( kr_{1}-\omega t+\phi _{o}\right)
\end{equation*}%
and the other from slit two.%
\begin{equation*}
E_{2}=E_{\max }\sin \left( kr_{2}-\omega t+\phi _{o}\right)
\end{equation*}%
This is mathematically just like superposition of sound waves.

\subsection{Superposition of two light waves}

Remember when we superimposed waves before, we mixed the waves%
\begin{eqnarray*}
y_{1} &=&A\sin \left( kr_{1}-\omega t+\phi _{1}\right) \\
y_{2} &=&A\sin \left( kr_{2}-\omega t+\phi _{2}\right)
\end{eqnarray*}

and using

\begin{equation*}
\sin a+\sin b=2\cos \left( \frac{a-b}{2}\right) \sin \left( \frac{a+b}{2}%
\right)
\end{equation*}%
we found the resultant wave%
\begin{equation*}
y_{r}=2A\cos \left( \frac{1}{2}\left( \Delta \phi \right) \right) \sin
\left( k\frac{r_{2}+r_{1}}{2}-\omega t+\frac{\phi _{2}+\phi _{1}}{2}\right)
\end{equation*}%
Our light waves are just two waves. They may be the superposition of many
individual photons, but the combined wave is just a wave.

At the slits, the waves have the same amplitude $E_{\max }$ and the same
phase constant, $\phi _{1}=\phi _{2}=\phi _{o}$, but $E_{2}$ travels farther
than $E_{1},$ so $\Delta \phi $ is due to the path difference. The path
difference would be%
\begin{eqnarray*}
\Delta \phi &=&k\Delta r+\Delta \phi _{o} \\
&=&k\delta +0 \\
&=&\frac{2\pi }{\lambda }d\sin \theta
\end{eqnarray*}

Now superimposing $E_{1}$ and $E_{2}$ at point $P$ on the screen gives%
\begin{eqnarray*}
E_{P} &=&E_{2}+E_{1} \\
&=&E_{\max }\sin \left( kr_{2}-\omega t\right) +E_{o}\sin \left(
kr_{1}-\omega t\right)
\end{eqnarray*}%
and using our prior result, we have

\begin{equation*}
E_{P}=2E_{\max }\cos \left( \frac{1}{2}\Delta \phi \right) \sin \left( k%
\frac{\left( r_{2}+r_{1}\right) }{2}-\omega t+\phi _{o}\right)
\end{equation*}%
or%
\begin{equation*}
E_{P}=2E_{\max }\cos \left( \frac{1}{2}\left( \frac{2\pi }{\lambda }d\sin
\theta \right) \right) \sin \left( k\frac{\left( r_{2}+r_{1}\right) }{2}%
-\omega t+\phi _{o}\right)
\end{equation*}

We have a combined wave at point $P$ that is a traveling wave $\left( \sin
\left( k\frac{\left( r_{2}+r_{1}\right) }{2}-\omega t+\phi _{o}\right)
\right) $ but with amplitude $\left( 2E_{\max }\cos \left( \frac{1}{2}\left( 
\frac{2\pi }{\lambda }d\sin \theta \right) \right) \right) $ that depends on
our total phase $\Delta \phi =\frac{2\pi }{\lambda }d\sin \theta .$

But the situation is more complicated because of how we detect light. Our
eyes, film, and most detectors measure the intensity of the light. We know
that 
\begin{equation*}
I=\frac{\mathcal{P}}{A}
\end{equation*}%
later in the course we will show that the power is proportional to the
square of the electric field displacement. 
\begin{equation}
\mathcal{P}\varpropto E_{P}^{2}
\end{equation}%
For now, let's just assume this is true. Then the intensity must be
proportional to the amplitude of the electric field squared.%
\begin{eqnarray*}
I &\varpropto &E_{P}^{2} \\
&=&4E_{\max }^{2}\cos ^{2}\left( \frac{1}{2}\left( \frac{2\pi }{\lambda }%
d\sin \theta \right) \right) \sin ^{2}\left( \frac{k\left(
r_{2}+r_{1}\right) }{2}-\omega t+\phi _{o}\right)
\end{eqnarray*}

Light detectors collect energy for a set amount of time. So most light
detection will be a value averaged over a set \emph{integration time}. This
means that the detector sums up (or integrates) the amount of power received
over the detector time. Usually the integration time is much longer than a
period, so what is really detected is a time-average of our intensity.%
\begin{eqnarray*}
\int_{\text{many T}}Idt &\varpropto &=\int_{\text{many T}}4E_{\max }^{2}\cos
^{2}\left( \frac{1}{2}\left( \frac{2\pi }{\lambda }d\sin \theta \right)
\right) \sin ^{2}\left( \frac{k\left( r_{2}+r_{1}\right) }{2}-\omega t+\phi
_{o}\right) dt \\
&=&4E_{\max }\cos ^{2}\left( \frac{1}{2}\left( \frac{2\pi }{\lambda }d\sin
\theta \right) \right) \int_{\text{many T}}\sin ^{2}\left( \frac{k\left(
rx_{2}+r_{1}\right) }{2}-\omega t+\phi _{o}\right) dt
\end{eqnarray*}%
but the term%
\begin{equation}
\int_{\text{many T}}\sin ^{2}\left( \frac{k\left( r_{2}+r_{1}\right) }{2}%
-\omega t+\phi _{o}\right) dt=\frac{1}{2}  \label{average_of_sine_squared}
\end{equation}

To convince yourself of this, think that $\sin ^{2}\left( \omega t\right) $
has a maximum value of $1$ and a minimum of $0.$ Looking at the graph\FRAME{%
dhF}{1.9986in}{1.4927in}{0in}{}{}{Figure}{\special{language "Scientific
Word";type "GRAPHIC";maintain-aspect-ratio TRUE;display "USEDEF";valid_file
"T";width 1.9986in;height 1.4927in;depth 0in;original-width
1.9597in;original-height 1.4572in;cropleft "0";croptop "1";cropright
"1";cropbottom "0";tempfilename 'LY47UP0F.wmf';tempfile-properties "XPR";}}%
should be convincing that the average value over a period is $1/2.$ The
average over many periods will still be $1/2.$

So we have 
\begin{equation}
\bar{I}=\int_{\text{many periods}}Idt\varpropto 2E_{\max }\cos ^{2}\left( 
\frac{1}{2}\left( \frac{2\pi }{\lambda }d\sin \theta \right) \right)
\end{equation}%
where $\bar{I}$ is the time average intensity. The important part is that
the time varying part has averaged out.

So, usually in optics, we ignore the fast fluctuating parts of such
calculations because we can't see them and so we write%
\begin{equation*}
I=I_{\max }\cos ^{2}\left( \frac{1}{2}\left( \frac{2\pi }{\lambda }d\sin
\theta \right) \right)
\end{equation*}%
where we have dropped the bar from the $I,$ but it is understood that the
intensity we report is a time average over many periods.

We should remind ourselves, of our intensity pattern 
\begin{equation*}
I=I_{\max }\cos ^{2}\left( \frac{1}{2}\frac{2\pi }{\lambda }d\sin \theta
\right)
\end{equation*}%
is really 
\begin{equation*}
I=I_{\max }\cos ^{2}\left( \frac{\Delta \phi }{2}\right)
\end{equation*}%
Which is just our amplitude squared for the mixing of two waves. All we have
done to find the intensity pattern is to find and expression for the phase
difference $\Delta \phi .$

Our intensity pattern should give the same location for the center of the
bright spots as we got before. Let's check that it works. We used the small
angle approximation before, so let's use it again now. For for small angles 
\begin{eqnarray*}
I &=&I_{\max }\cos ^{2}\left( \frac{\pi d}{\lambda }\theta \right) \\
&=&I_{\max }\cos ^{2}\left( \frac{\pi d}{\lambda }\frac{y}{L}\right)
\end{eqnarray*}%
Then we have constructive interference when 
\begin{equation*}
\frac{\pi d}{\lambda }\frac{y}{L}=m\pi
\end{equation*}%
or%
\begin{equation*}
y=m\frac{L\lambda }{d}
\end{equation*}%
which is what we found before.

The plot of normalized intensity 
\begin{equation*}
\frac{I}{I_{\max }}=\cos ^{2}\left( \frac{\Delta \phi }{2}\right)
\end{equation*}%
verses $\Delta \phi /2$ is given next, \FRAME{dtbpFX}{2.2917in}{1.5281in}{0pt%
}{}{}{Plot}{\special{language "Scientific Word";type "MAPLEPLOT";width
2.2917in;height 1.5281in;depth 0pt;display "USEDEF";plot_snapshots
TRUE;mustRecompute FALSE;lastEngine "MuPAD";xmin "-15";xmax "15";xviewmin
"-15";xviewmax "15";yviewmin "0";yviewmax
"1";viewset"XY";rangeset"X";plottype 4;labeloverrides 3;x-label
"delta_phi/2";y-label "I /I_max";axesFont "Times New
Roman,12,0000000000,useDefault,normal";numpoints 100;plotstyle
"patch";axesstyle "normal";axestips FALSE;xis \TEXUX{x};var1name
\TEXUX{$x$};function \TEXUX{$\cos ^{2}\left( \frac{x}{2}\right) $};linecolor
"blue";linestyle 1;pointstyle "point";linethickness 1;lineAttributes
"Solid";var1range "-15,15";num-x-gridlines 100;curveColor
"[flat::RGB:0x000000ff]";curveStyle "Line";VCamFile
'LY47VE1U.xvz';valid_file "T";tempfilename
'LY47VX0G.wmf';tempfile-properties "XPR";}}but we will find that we are not
quite through with this analysis. Next time we will find that there is
another compounding factor that reduces the intensity as we move away from
the midpoint.\FRAME{dtbpFX}{2.1871in}{1.4581in}{0pt}{}{}{Plot}{\special%
{language "Scientific Word";type "MAPLEPLOT";width 2.1871in;height
1.4581in;depth 0pt;display "USEDEF";plot_snapshots TRUE;mustRecompute
FALSE;lastEngine "MuPAD";xmin "-15";xmax "15";xviewmin "-15";xviewmax
"15";yviewmin "0";yviewmax "1";viewset"XY";rangeset"X";plottype
4;labeloverrides 3;x-label "delta_phi/2";y-label "I /I_max";axesFont "Times
New Roman,12,0000000000,useDefault,normal";numpoints 100;plotstyle
"patch";axesstyle "normal";axestips FALSE;xis \TEXUX{x};var1name
\TEXUX{$x$};function \TEXUX{$\cos ^{2}\left( \frac{x}{2}\right) \frac{4\sin
^{2}\left( \frac{0.2x}{2}\right) }{\left( 0.2x\right) ^{2}}$};linecolor
"blue";linestyle 1;pointstyle "point";linethickness 1;lineAttributes
"Solid";var1range "-15,15";num-x-gridlines 100;curveColor
"[flat::RGB:0x000000ff]";curveStyle "Line";VCamFile
'LY47NR1H.xvz';valid_file "T";tempfilename
'LY47VX0H.wmf';tempfile-properties "XPR";}}Let's pause to remember what this
pattern means. This is the intensity of light due to interference. It is
instructive to match our intensity pattern that Young saw with our graph.%
\FRAME{dhF}{4.4719in}{2.2606in}{0pt}{}{}{Figure}{\special{language
"Scientific Word";type "GRAPHIC";maintain-aspect-ratio TRUE;display
"USEDEF";valid_file "T";width 4.4719in;height 2.2606in;depth
0pt;original-width 4.4209in;original-height 2.2208in;cropleft "0";croptop
"1";cropright "1";cropbottom "0";tempfilename
'LY3VJH04.wmf';tempfile-properties "XPR";}}The high intensity peaks are the
bright fringes and the low intensity troughs are the dark fringes. The
pattern moves smoothly and continuously from bright to dark.

\chapter{Many slits, and single slits}

Last lecture we found the pattern that results from sending light through
two slits. This lecture takes on many slits, and even the pattern that
results from a single slit.

%TCIMACRO{%
%\TeXButton{Fundamental Concepts}{\hspace{-1.3in}{\LARGE Fundamental Concepts\vspace{0.25in}}}}%
%BeginExpansion
\hspace{-1.3in}{\LARGE Fundamental Concepts\vspace{0.25in}}%
%EndExpansion

\begin{itemize}
\item Many slit devices are called diffraction gratings

\item These devices can be use to build spectrometers

\item Single slits also produce an interference pattern
\end{itemize}

\section{Diffraction Gratings}

We have discussed the interference that comes from having two small slits.
But what if we have more slits?\FRAME{dhF}{1.0905in}{1.9251in}{0pt}{}{}{%
Figure}{\special{language "Scientific Word";type
"GRAPHIC";maintain-aspect-ratio TRUE;display "USEDEF";valid_file "T";width
1.0905in;height 1.9251in;depth 0pt;original-width 1.0568in;original-height
1.887in;cropleft "0";croptop "1";cropright "1";cropbottom "0";tempfilename
'LY9MPQ0X.wmf';tempfile-properties "XPR";}}

%TCIMACRO{%
%\TeXButton{Rainbow Glasses}{\marginpar {
%\hspace{-0.5in}
%\begin{minipage}[t]{1in}
%\small{Rainbow Glasses}
%\end{minipage}
%}}}%
%BeginExpansion
\marginpar {
\hspace{-0.5in}
\begin{minipage}[t]{1in}
\small{Rainbow Glasses}
\end{minipage}
}%
%EndExpansion
A diffraction grating is an optical element with many many parallel slits
spaced very close together. Here is a typical diffraction grating created by
etching lines in a piece of glass. The etchings scatter the light, but the
un-etched part allows the light to pass through. The un-etched parts are
essentially a series of slits.\FRAME{dhFU}{1.9986in}{1.625in}{0pt}{\Qcb{%
{\protect\small Surface of a diffraction grating (600 lines/mm). Image taken
with optical transmission microscope. (Image in the public domain courtesy
Scapha)}}}{}{Figure}{\special{language "Scientific Word";type
"GRAPHIC";maintain-aspect-ratio TRUE;display "USEDEF";valid_file "T";width
1.9986in;height 1.625in;depth 0pt;original-width 2.4474in;original-height
1.9847in;cropleft "0";croptop "1";cropright "1";cropbottom "0";tempfilename
'LY9BWG06.wmf';tempfile-properties "XPR";}}A typical grating might have $%
5000 $ slits per unit centimeter. You have probably used a diffraction
grating to see rainbow colors in a beginning science class.

If we use $5000\frac{\text{slits}}{\unit{cm}}$ for an example, we see that
the slit spacing is 
\begin{eqnarray}
d &=&\frac{1}{5000}\unit{cm} \\
&=&\allowbreak 2.0\times 10^{-6}\unit{m}
\end{eqnarray}%
Take a section of diffraction grating as shown below\FRAME{dhF}{1.9951in}{%
3.3788in}{0pt}{}{}{Figure}{\special{language "Scientific Word";type
"GRAPHIC";maintain-aspect-ratio TRUE;display "USEDEF";valid_file "T";width
1.9951in;height 3.3788in;depth 0pt;original-width 5.5893in;original-height
9.5207in;cropleft "0";croptop "1";cropright "1";cropbottom "0";tempfilename
'LTUWCO3N.wmf';tempfile-properties "XPR";}}

At some point, two of the slits will have a path difference that is a whole
wavelength, and we would expect a bright spot. But what about the other
slits? If we have a slit spacing such that each of the succeeding slits has
a path difference that is just an additional wavelength, then each of the
slits will contribute to the constructive interference at our point, and the
point will become a very bright spot.

\FRAME{dhF}{0.9755in}{2.3679in}{0pt}{}{}{Figure}{\special{language
"Scientific Word";type "GRAPHIC";maintain-aspect-ratio TRUE;display
"USEDEF";valid_file "T";width 0.9755in;height 2.3679in;depth
0pt;original-width 1.6821in;original-height 4.1217in;cropleft "0";croptop
"1";cropright "1";cropbottom "0";tempfilename
'LTUWCP3O.wmf';tempfile-properties "XPR";}} The light leaves each slit in
phase with the light from the rest of the slits. At some distance $L$ away
and at some angle $\theta ,$ we will have a path difference%
\begin{equation}
\delta =d\sin \left( \theta _{bright}\right) =m\lambda \quad m=0,\pm 1,\pm
2,\ldots
\end{equation}%
This looks a lot like our condition for constructive interference for two
slits.

This equation tells us that each wavelength, $\lambda ,$ will experience
constructive interference at a slightly different angle $\theta _{bright}.$
Knowing $d$ and $\theta $ allows an accurate calculation of $\lambda .$ This
may seem a silly thing to do, but suppose we add into our system a sample of
a chemical to identify \FRAME{dhF}{3.5958in}{1.7465in}{0pt}{}{}{Figure}{%
\special{language "Scientific Word";type "GRAPHIC";maintain-aspect-ratio
TRUE;display "USEDEF";valid_file "T";width 3.5958in;height 1.7465in;depth
0pt;original-width 3.6357in;original-height 1.7509in;cropleft "0";croptop
"1";cropright "1";cropbottom "0";tempfilename
'LX20L307.wmf';tempfile-properties "XPR";}}

We could then record the intensity of the transmitted light as a function of
angle, which is equivalent to $\lambda .$ We can again generate a spectrum.
This is a traditional way to build a spectrometer and many such devices are
available today.%
%TCIMACRO{%
%\TeXButton{Demo a student spectrometer with a gas tube}{\marginpar {
%\hspace{-0.5in}
%\begin{minipage}[t]{1in}
%\small{Demo a student spectrometer with a gas tube}
%\end{minipage}
%}}}%
%BeginExpansion
\marginpar {
\hspace{-0.5in}
\begin{minipage}[t]{1in}
\small{Demo a student spectrometer with a gas tube}
\end{minipage}
}%
%EndExpansion

\subsection{Resolving power of diffraction gratings}

For two nearly equal wavelengths $\lambda _{1}$ and $\lambda _{2},$ we say
that the diffraction grating can resolve the wavelengths if we can
distinguish the two using the grating. The \emph{resolving power} of the
grating is defined as 
\begin{equation}
R=\frac{\left( \lambda _{1}+\lambda _{2}\right) }{2\left( \lambda
_{1}-\lambda _{2}\right) }=\frac{\bar{\lambda}}{\Delta \lambda }
\end{equation}%
We can show that for the $m$-th order diffraction, the resolving power is%
\begin{equation}
R=Nm
\end{equation}%
where $N$ is the number of slits. So our ability to distinguish wavelengths
increases with the number of slits and with the order (which is related to
how far off-axis we look).

Note that for $m=0$ we have no ability to resolve wavelengths. The central
peak is a mix of all wavelengths and usually looks white for normal
illumination.

That the resolution depends on the number of slits, $N,$ means that we can
improve our spectrometer by using more lines. Here is a representation of
what happens as we increase $N$\FRAME{dhF}{3.103in}{2.6654in}{0in}{}{}{Figure%
}{\special{language "Scientific Word";type "GRAPHIC";maintain-aspect-ratio
TRUE;display "USEDEF";valid_file "T";width 3.103in;height 2.6654in;depth
0in;original-width 3.0588in;original-height 2.623in;cropleft "0";croptop
"1";cropright "1";cropbottom "0";tempfilename
'LY9EIJ0O.wmf';tempfile-properties "XPR";}}we can see that the peaks get
narrower as $N$ increases. These graphs are for a particular $\lambda .$ If
the peaks for a particular $\lambda $ get narrower, then there will be less
overlap with adjacent $\lambda ^{\prime }s$ which means that each wavelength
can more easily be resolved.

Spectrometers are used in many places. One that has some public interest
today is monitoring the atmosphere. Instruments like the one shown below
detect the amount of special gasses in the atmosphere using IR spectrometers.

\FRAME{dhFU}{4.6544in}{3.0554in}{0pt}{\Qcb{{\protect\small AIRS\ sensor,
spectrometer design, and global CO}$_{2}${\protect\small \ data. (Images in
the Public Domain courtesy NASA)}}}{}{Figure}{\special{language "Scientific
Word";type "GRAPHIC";maintain-aspect-ratio TRUE;display "USEDEF";valid_file
"T";width 4.6544in;height 3.0554in;depth 0pt;original-width
4.6017in;original-height 3.0113in;cropleft "0";croptop "1";cropright
"1";cropbottom "0";tempfilename 'LY9FNK0P.wmf';tempfile-properties "XPR";}}%
The instrument shown is the AIRS spectrometer. You can see in the diagram
that it uses a grating spectrometer. The picture of the Earth is a composite
of AIRS data showing the northern and southern bands of CO$_{2}.$

\section{Single Slits}

We have looked at interference from two slits, and for many slits. The two
slits acted like two coherent sources. We might expect that a single slit
will give only a single bright spot. But let's consider a single slit very
closely. To do this, let's return to the work of Huygens.\footnote{%
Huygens method is technically not a correct representation of what happens.
The actual wave leaving the single opening is a superposition of the
original wave, and the wave scattered from the sides of the opening. You can
see this scattering by tearing a small hole in a piece of paper and looking
through the hole at a light source. You will see the bright ring around the
hole where the edges of the paper are scattering the light. But the
mathematical result we will get using Huygens method gives a mathematically
identical result for the resulting wave leaving the slit with much less high
power math. So we will stick with Huygens in this class.} His idea for the
nature of light was simple. He suggested that every point on the wave front
of a light wave was the source (the disturbance) for a new set of small
spherical waves. The next wavefront would be formed by the superposition of
the little \textquotedblleft wavelets.\textquotedblright\ Here is an example
for a plane wave and a spherical wave.\FRAME{dhF}{2.2208in}{1.8697in}{0pt}{}{%
}{Figure}{\special{language "Scientific Word";type
"GRAPHIC";maintain-aspect-ratio TRUE;display "USEDEF";valid_file "T";width
2.2208in;height 1.8697in;depth 0pt;original-width 4.0594in;original-height
3.4143in;cropleft "0";croptop "1";cropright "1";cropbottom "0";tempfilename
'LTUWCP3R.wmf';tempfile-properties "XPR";}}In each case we have drawn spots
on the wave front and drawn spherical waves around those spots. where the
wavefronts of the little wavelets combine, we have new wave front of our
wave. This is sort of what happens in bulk matter. Remember that light is
absorbed and re-emitted by the atoms of the material. This is why light
slows down in a medium. Because of the time it is absorbed, it effectively
goes slower. But the light is not necessarily re-emitted in the same
direction. Sometimes it is, but sometimes it is not. This creates a small,
spherical wave (called a wavelet) that is emitted by that atom. So Huygens
idea is not too bad.

We can use this idea for a single slit and look at what happens as the light
goes through. Here is such a slit.

\FRAME{fhF}{2.0686in}{2.693in}{0pt}{}{\Qlb{SingleSlitGeometry}}{Figure}{%
\special{language "Scientific Word";type "GRAPHIC";maintain-aspect-ratio
TRUE;display "USEDEF";valid_file "T";width 2.0686in;height 2.693in;depth
0pt;original-width 2.0297in;original-height 2.6507in;cropleft "0";croptop
"1";cropright "1";cropbottom "0";tempfilename
'LY9D3A0A.wmf';tempfile-properties "XPR";}}

In the figure above, we have divided a single slit of width $a$ into two
parts, each of size $a/2.$ According to Huygens' principle, each position of
the slit acts as a source of light rays. So we can treat half a slit as two
coherent sources. These two sources should interfere. So what do we see when
we perform such an experiment?

\FRAME{dhF}{2.3065in}{0.8233in}{0in}{}{}{Figure}{\special{language
"Scientific Word";type "GRAPHIC";maintain-aspect-ratio TRUE;display
"USEDEF";valid_file "T";width 2.3065in;height 0.8233in;depth
0in;original-width 2.2667in;original-height 0.7904in;cropleft "0";croptop
"1";cropright "1";cropbottom "0";tempfilename
'LY9C6O09.wmf';tempfile-properties "XPR";}}

The figure shows a diffraction pattern for a thin slit. There are several
terms that are in common use to describe the pattern

\begin{enumerate}
\item Central Maximum: The broad intense central band.

\item Secondary Maxima:The fainter bright bands to both sides of the central
maxima

\item Minima: The dark bands between the maxima
\end{enumerate}

\section{Narrow Slit Intensity Pattern}

Let's use figure \ref{SingleSlitGeometry} to find the dark minima of the
single slit pattern. First we should notice that figure \ref%
{SingleSlitGeometry} could have another set of rays that contribute to the
bright spot because they will also have a path difference of $\left(
a/2\right) \sin \theta .$ Let's fill these in. They are rays $2$ and $4$ of
the next figure. \FRAME{dhF}{2.1525in}{2.6507in}{0pt}{}{}{Figure}{\special%
{language "Scientific Word";type "GRAPHIC";maintain-aspect-ratio
TRUE;display "USEDEF";valid_file "T";width 2.1525in;height 2.6507in;depth
0pt;original-width 2.1136in;original-height 2.6091in;cropleft "0";croptop
"1";cropright "1";cropbottom "0";tempfilename
'LY9D7I0D.wmf';tempfile-properties "XPR";}}

Before we started with what we are now calling rays $1$ and $3.$ Ray $1$
travels a distance 
\begin{equation}
\delta =\frac{a}{2}\sin \left( \theta \right)
\end{equation}%
farther than ray $3.$ As we just argued, rays $2$ and $4$ also have the same
path difference, and so do rays $3$ and $5.$ If this path difference is $%
\lambda /2$ then we will have destructive interference. The condition for a
minima is then%
\begin{equation}
\frac{a}{2}\sin \left( \theta \right) =\pm \frac{\lambda }{2}
\end{equation}%
or%
\begin{equation}
\sin \left( \theta \right) =\pm \frac{\lambda }{a}
\end{equation}

Now we could also divide the slit into four equal parts. Then we have a path
difference of 
\begin{equation}
\delta =\frac{a}{4}\sin \left( \theta \right)
\end{equation}%
and to have destructive interference we need this path difference to be $%
\lambda /2$%
\begin{equation}
\frac{a}{4}\sin \left( \theta \right) =\pm \frac{\lambda }{2}
\end{equation}%
or%
\begin{equation}
\sin \left( \theta \right) =\pm \frac{2\lambda }{a}
\end{equation}%
We can keep going to find a minima at\FRAME{dhF}{1.9285in}{2.5114in}{0in}{}{%
}{Figure}{\special{language "Scientific Word";type
"GRAPHIC";maintain-aspect-ratio TRUE;display "USEDEF";valid_file "T";width
1.9285in;height 2.5114in;depth 0in;original-width 1.8905in;original-height
2.4708in;cropleft "0";croptop "1";cropright "1";cropbottom "0";tempfilename
'LY9MMW0W.wmf';tempfile-properties "XPR";}}%
\begin{equation}
\sin \left( \theta \right) =\pm \frac{3\lambda }{a}
\end{equation}%
and in general at%
\begin{equation}
\sin \left( \theta \right) =m\frac{\lambda }{a}\quad m=\pm 1,\pm 2,\pm
3\ldots  \label{Single Slit minima}
\end{equation}

%TCIMACRO{%
%\TeXButton{Question 223.11.1}{\marginpar {
%\hspace{-0.5in}
%\begin{minipage}[t]{1in}
%\small{Question 223.11.1}
%\end{minipage}
%}}}%
%BeginExpansion
\marginpar {
\hspace{-0.5in}
\begin{minipage}[t]{1in}
\small{Question 223.11.1}
\end{minipage}
}%
%EndExpansion

\subsection{Intensity of the single-slit pattern}

I will not derive the intensity pattern for the single slit (though it is
not really too hard to do) but I\ will give it here%
\begin{equation}
I=I_{\max }\left( \frac{\sin \left( \frac{\pi }{\lambda }a\sin \theta
\right) }{\frac{\pi }{\lambda }a\sin \theta }\right) ^{2}
\end{equation}

Notice this has the form%
\begin{equation*}
\frac{\sin x}{x}
\end{equation*}%
which has a distinctive shape.\FRAME{dtbpFX}{4.4996in}{1.3958in}{0pt}{}{}{%
Plot}{\special{language "Scientific Word";type "MAPLEPLOT";width
4.4996in;height 1.3958in;depth 0pt;display "USEDEF";plot_snapshots
TRUE;mustRecompute FALSE;lastEngine "MuPAD";xmin "-30";xmax "30";xviewmin
"-30";xviewmax "30";yviewmin "-0.217351";yviewmax
"1.000122";viewset"XY";rangeset"X";plottype 4;labeloverrides 3;x-label
"x";y-label "sinc(x)";axesFont "Times New
Roman,12,0000000000,useDefault,normal";numpoints 100;plotstyle
"patch";axesstyle "normal";axestips FALSE;xis \TEXUX{x};var1name
\TEXUX{$x$};function \TEXUX{$\frac{\sin x}{x}$};linecolor "blue";linestyle
1;pointstyle "point";linethickness 3;lineAttributes "Solid";var1range
"-30,30";num-x-gridlines 100;curveColor "[flat::RGB:0x000000ff]";curveStyle
"Line";VCamFile 'ME4QYW0K.xvz';valid_file "T";tempfilename
'S1LQPU00.wmf';tempfile-properties "XPR";}}this is known as a sinc function
(pronounced like \textquotedblleft sink\textquotedblright ). It has a
central maximum as we would expect. Of course our intensity pattern has a
sinc squared \FRAME{dtbpFX}{4.4996in}{2.2502in}{0pt}{}{}{Plot}{\special%
{language "Scientific Word";type "MAPLEPLOT";width 4.4996in;height
2.2502in;depth 0pt;display "USEDEF";plot_snapshots TRUE;mustRecompute
FALSE;lastEngine "MuPAD";xmin "-15";xmax "15";xviewmin "-15";xviewmax
"15";yviewmin "-1";yviewmax "1.002079";viewset"XY";rangeset"X";plottype
4;labeloverrides 3;x-label "pi*a*sin(theta)/lambda";y-label "I";axesFont
"Times New Roman,12,0000000000,useDefault,normal";numpoints 100;plotstyle
"patch";axesstyle "normal";axestips FALSE;xis \TEXUX{x};var1name
\TEXUX{$x$};function \TEXUX{$\left( \frac{\sin x}{x}\right) ^{2}$};linecolor
"blue";linestyle 1;pointstyle "point";linethickness 3;lineAttributes
"Solid";var1range "-15,15";num-x-gridlines 100;curveColor
"[flat::RGB:0x000000ff]";curveStyle "Line";VCamFile
'ME4QY70J.xvz';valid_file "T";tempfilename
'ME4QY704.wmf';tempfile-properties "XPR";}}You can see the central maximum
and the much weaker minima produced by this function. Indeed, it seems to
match what we saw very well. Putting it all together, our pattern looks like
this.\FRAME{dhF}{3.0329in}{1.535in}{0pt}{}{}{Figure}{\special{language
"Scientific Word";type "GRAPHIC";maintain-aspect-ratio TRUE;display
"USEDEF";valid_file "T";width 3.0329in;height 1.535in;depth
0pt;original-width 2.9888in;original-height 1.4987in;cropleft "0";croptop
"1";cropright "1";cropbottom "0";tempfilename
'LY9GM20T.wmf';tempfile-properties "XPR";}}

This is really an interesting result. You might wonder why, when we found
the two slit interference pattern, there was no evidence of the single slit
fringing that we discovered in this chapter. After all, a double slit system
is made from single slits. Shouldn't there be some effect due to the fact
that the slits are individually single slits? The answer is that we did see
some hint of the single slit pattern. Remember the figure below. \FRAME{dhF}{%
4.4719in}{2.2606in}{0pt}{}{}{Figure}{\special{language "Scientific
Word";type "GRAPHIC";maintain-aspect-ratio TRUE;display "USEDEF";valid_file
"T";width 4.4719in;height 2.2606in;depth 0pt;original-width
4.4209in;original-height 2.2208in;cropleft "0";croptop "1";cropright
"1";cropbottom "0";tempfilename 'LY9DD10E.wmf';tempfile-properties "XPR";}}%
The intensity of the peaks seems to fall off with distance from the center.
We dealt with only the center-most part of the pattern. If we draw the
pattern for larger angles, we see the following.\FRAME{dtbpF}{4.1883in}{%
2.3497in}{0pt}{}{}{Figure}{\special{language "Scientific Word";type
"GRAPHIC";maintain-aspect-ratio TRUE;display "USEDEF";valid_file "T";width
4.1883in;height 2.3497in;depth 0pt;original-width 4.1381in;original-height
2.3099in;cropleft "0";croptop "1";cropright "1";cropbottom "0";tempfilename
'M3I96B02.wmf';tempfile-properties "XPR";}}

It takes a bright laser or dark room to see the secondary groups of fringes
easily, but we can do it. We can also graph the intensity pattern. It is the
combination of the two slit and single slit pattern with the single slit
pattern acting as an envelope.

\begin{equation}
I=I_{\max }\cos ^{2}\left( \frac{\pi d\sin \left( \theta \right) }{\lambda }%
\right) \left( \frac{\sin \left( \frac{\pi a\sin \left( \theta \right) }{%
\lambda }\right) }{\frac{\pi a\sin \left( \theta \right) }{\lambda }}\right)
^{2}
\end{equation}%
Note that one of the double slit maxima is clobbered by a minimum in the
single slit pattern. We can find out the order of the missing maximum.
Recall that 
\begin{equation*}
d\sin \left( \theta \right) =m\lambda
\end{equation*}%
describes the maxima from the double slit. But%
\begin{equation*}
a\sin \left( \theta \right) =\lambda
\end{equation*}%
describes the minimum from the single slit. Dividing these yields%
\begin{eqnarray*}
\frac{d\sin \left( \theta \right) }{a\sin \left( \theta \right) } &=&\frac{%
m\lambda }{\lambda } \\
\frac{d}{a} &=&m
\end{eqnarray*}%
so the 
\begin{equation}
m=\frac{d}{a}
\end{equation}%
double slit maximum will be missing.

\chapter{Apertures and Interferometers}

%TCIMACRO{%
%\TeXButton{Fundamental Concepts}{\hspace{-1.3in}{\LARGE Fundamental Concepts\vspace{0.25in}}}}%
%BeginExpansion
\hspace{-1.3in}{\LARGE Fundamental Concepts\vspace{0.25in}}%
%EndExpansion

\begin{itemize}
\item Round apertures act very much like slits, with some added numerical
factors

\item If $\lambda \ll D$ we see little evidence for the wave nature of
light. This limit is called the geometric optics limit or the ray
approximation

\item Interferometers can measure phenomenally small displacements using the
wave nature of light
\end{itemize}

\section{Circular Apertures}

%TCIMACRO{%
%\TeXButton{Question 223.12.1}{\marginpar {
%\hspace{-0.5in}
%\begin{minipage}[t]{1in}
%\small{Question 223.12.1}
%\end{minipage}
%}}}%
%BeginExpansion
\marginpar {
\hspace{-0.5in}
\begin{minipage}[t]{1in}
\small{Question 223.12.1}
\end{minipage}
}%
%EndExpansion
%TCIMACRO{%
%\TeXButton{Question 223.12.3}{\marginpar {
%\hspace{-0.5in}
%\begin{minipage}[t]{1in}
%\small{Question 223.12.3}
%\end{minipage}
%}}}%
%BeginExpansion
\marginpar {
\hspace{-0.5in}
\begin{minipage}[t]{1in}
\small{Question 223.12.3}
\end{minipage}
}%
%EndExpansion
Our analysis of light going through holes has been somewhat limited by
squarish holes or slits. But most optical systems, including our eyes don't
have rectangular holes. So what happens when the hole is round? The
situation is as shown in the next figure.\FRAME{dhF}{3.5189in}{1.9138in}{0pt%
}{}{}{Figure}{\special{language "Scientific Word";type
"GRAPHIC";maintain-aspect-ratio TRUE;display "USEDEF";valid_file "T";width
3.5189in;height 1.9138in;depth 0pt;original-width 4.1572in;original-height
2.2485in;cropleft "0";croptop "1";cropright "1";cropbottom "0";tempfilename
'LYD3KH00.wmf';tempfile-properties "XPR";}}Before we discuss this situation,
let's think about what we know about the width of a single slit pattern. We
remember that 
\begin{equation*}
\sin \left( \theta \right) =\left( 1\right) \frac{\lambda }{a}
\end{equation*}%
for the first minima, or, because the angles are small, 
\begin{equation*}
\theta \approx \frac{\lambda }{a}
\end{equation*}%
and from the figure we can see that%
\begin{equation*}
\tan \theta =\frac{y}{L}
\end{equation*}%
or 
\begin{equation*}
\theta \approx \frac{y}{L}
\end{equation*}%
so long as $\theta $ is small, then we find the position of the first
minimum to be%
\begin{equation*}
y=\frac{\lambda }{a}L
\end{equation*}%
This is the distance from the center bright spot to the first dark spot. The
width of the bright spot is twice this distance%
\begin{equation*}
w=2\frac{\lambda }{a}L
\end{equation*}%
We expect something like this for our circular aperture. The derivation for
the circular aperture is not really to hard, but it involves Bessel
functions, which are beyond the math requirement for this course. So I will
give you the answer%
\begin{equation*}
\theta =1.22\frac{\lambda }{D}
\end{equation*}%
where $D$ is the diameter of the circular aperture (like $a$ was the width
of the slit) and as before%
\begin{equation*}
\tan \theta =\frac{y}{L}
\end{equation*}%
so%
\begin{equation*}
\theta \approx \frac{y}{L}
\end{equation*}%
which gives us a first minimum location of%
\begin{equation}
y=1.22\frac{\lambda }{D}L
\end{equation}%
and a width of 
\begin{equation}
w=2.44\frac{\lambda }{D}L
\end{equation}%
%TCIMACRO{%
%\TeXButton{Airy Pattern Demo}{\marginpar {
%\hspace{-0.5in}
%\begin{minipage}[t]{1in}
%\small{Airy Pattern Demo}
%\end{minipage}
%}}}%
%BeginExpansion
\marginpar {
\hspace{-0.5in}
\begin{minipage}[t]{1in}
\small{Airy Pattern Demo}
\end{minipage}
}%
%EndExpansion
The picture in most books is a little bit deceptive. The pattern looks a
little like the slit pattern. But the secondary maxima are actually very
small for the circular aperture case. Much smaller than the secondary maxima
in the slit case. Here is a larger version of a cross section of the
intensity pattern.

\bigskip

\FRAME{dtbpFUX}{3.896in}{2.6731in}{0pt}{\Qcb{{}}}{}{Plot}{\special{language
"Scientific Word";type "MAPLEPLOT";width 3.896in;height 2.6731in;depth
0pt;display "USEDEF";plot_snapshots TRUE;mustRecompute FALSE;lastEngine
"MuPAD";xmin "-10";xmax "10";xviewmin "-10";xviewmax "10";yviewmin
"1.000100E-6";yviewmax "1.000100";viewset"XY";rangeset"X";plottype
4;axesFont "Times New Roman,12,0000000000,useDefault,normal";numpoints
100;plotstyle "patch";axesstyle "normal";axestips FALSE;xis
\TEXUX{JQSUB1ESUB};var1name \TEXUX{$J_{1}$};function
\TEXUX{$\frac{1}{\allowbreak 0.25}\left( \text{ }\frac{J_{1}\left( r\right)
}{r}\right) ^{2}$};linecolor "blue";linestyle 1;pointstyle
"point";linethickness 3;lineAttributes "Solid";var1range
"-10,10";num-x-gridlines 100;curveColor "[flat::RGB:0x000000ff]";curveStyle
"Line";VCamFile 'LYD49N08.xvz';valid_file "T";tempfilename
'LYD49N03.wmf';tempfile-properties "XPR";}}Notice how small the secondary
and tertiary maxima are. A three dimensional version of the intensity
pattern from the circular aperture.\FRAME{dtbpFUX}{4.0594in}{2.706in}{0pt}{%
\Qcb{{}}}{}{Plot}{\special{language "Scientific Word";type "MAPLEPLOT";width
4.0594in;height 2.706in;depth 0pt;display "USEDEF";plot_snapshots
TRUE;mustRecompute FALSE;lastEngine "MuPAD";xmin "-0.02";xmax "0.02";ymin
"-0.02";ymax "0.02";xviewmin "-0.02";xviewmax "0.02";yviewmin
"-0.02";yviewmax "0.02";zviewmin "-0.1";zviewmax
"1";viewset"XYZ";rangeset"XYZ";phi 72;theta -119;cameraDistance
"4.72728";cameraOrientation "[0,0,4.0674]";cameraOrientationFixed
TRUE;plottype 5;axesFont "Times New
Roman,12,0000000000,useDefault,normal";num-x-gridlines 25;num-y-gridlines
25;plotstyle "hidden";axesstyle "normal";axestips FALSE;plotshading
"Z";lighting 0;xis \TEXUX{JQSUB1ESUB};yis \TEXUX{y};var1name
\TEXUX{$J_{1}$};var2name \TEXUX{$y$};function \TEXUX{$\frac{1}{\allowbreak
0.25}\left( \text{ }\frac{J_{1}\left( \frac{2\pi \left(
\sqrt{x^{2}+y^{2}}\right) \left( 0.05\right) }{\left( 500\times
10^{-6}\right) \left( 1\right) }\right) }{\frac{2\pi \left(
\sqrt{x^{2}+y^{2}}\right) \left( 0.05\right) }{\left( 500\times
10^{-6}\right) \left( 1\right) }}\right) ^{2}$};linestyle 1;pointstyle
"point";linethickness 1;lineAttributes "Solid";var1range
"-0.02,0.02";var2range "-0.02,0.02";surfaceColor
"[linear:Z:RGB:0000000000:0000000000]";surfaceStyle "Hidden
Line";num-x-gridlines 75;num-y-gridlines 75;surfaceMesh
"Mesh";rangeset"XY";VCamFile 'LYD4E20A.xvz';valid_file "T";tempfilename
'LYH7UI09.wmf';tempfile-properties "XPR";}}With a bright enough laser, they
pattern becomes visible.\FRAME{dhFU}{2.1058in}{2.0557in}{0pt}{\Qcb{{}}}{}{%
Figure}{\special{language "Scientific Word";type
"GRAPHIC";maintain-aspect-ratio TRUE;display "USEDEF";valid_file "T";width
2.1058in;height 2.0557in;depth 0pt;original-width 1.0404in;original-height
1.0136in;cropleft "0";croptop "1";cropright "1";cropbottom "0";tempfilename
'LTUWCQ40.wmf';tempfile-properties "XPR";}}

\section{Interferometers}

%TCIMACRO{%
%\TeXButton{Interferomenter Demo}{\marginpar {
%\hspace{-0.5in}
%\begin{minipage}[t]{1in}
%\small{Interferomenter Demo}
%\end{minipage}
%}}}%
%BeginExpansion
\marginpar {
\hspace{-0.5in}
\begin{minipage}[t]{1in}
\small{Interferomenter Demo}
\end{minipage}
}%
%EndExpansion
Before we leave wave properties of physics and go to the ray approximation,
we should study some devices that use interference.

\subsection{The Michelson Interferometer}

The Michelson interferometer is another device that uses path differences to
create interference fringes. \FRAME{dhF}{3.2007in}{3.1263in}{0pt}{}{}{Figure%
}{\special{language "Scientific Word";type "GRAPHIC";maintain-aspect-ratio
TRUE;display "USEDEF";valid_file "T";width 3.2007in;height 3.1263in;depth
0pt;original-width 3.1566in;original-height 3.0805in;cropleft "0";croptop
"1";cropright "1";cropbottom "0";tempfilename
'LYDHE108.wmf';tempfile-properties "XPR";}}The device is shown in the
figure. A coherent light source is used. The light beam is split into two
beams that are usually at $90\unit{%
%TCIMACRO{\U{b0}}%
%BeginExpansion
{{}^\circ}%
%EndExpansion
}$ apart. The beams are reflected off of two mirrors back along the same
path and are mixed at the telescope. The result (with perfect alignment) is
a target fringe pattern like the first two shown below. \FRAME{dhF}{3.5362in%
}{1.6466in}{0pt}{}{}{Figure}{\special{language "Scientific Word";type
"GRAPHIC";maintain-aspect-ratio TRUE;display "USEDEF";valid_file "T";width
3.5362in;height 1.6466in;depth 0pt;original-width 3.4895in;original-height
1.6103in;cropleft "0";croptop "1";cropright "1";cropbottom "0";tempfilename
'LYDCNT06.wmf';tempfile-properties "XPR";}}If the alignment is off, you get
smaller fringes, but the system can still work. This is shown in the last
image in the previous figure.

In the figure, we have constructive interference in the center, but if we
move one of the mirrors half a wavelength, we would have destructive
interference and would see a dark spot in the center. This device gives us
the ability to measure distances on the order of the wavelength of the
light. When the distance is continuously changed, the pattern seems to grow
from the center (or collapse into the center).

Notice that if the mirror is moved $\frac{\lambda }{2},$ the path distance
changes by $\lambda $ because the light travels the distance to the mirror
and then back from the mirror (it travels the path twice!).

\subsection{Holography}

%TCIMACRO{%
%\TeXButton{Hologram demo-picture of woman}{\marginpar {
%\hspace{-0.5in}
%\begin{minipage}[t]{1in}
%\small{Hologram demo-picture of woman}
%\end{minipage}
%}}}%
%BeginExpansion
\marginpar {
\hspace{-0.5in}
\begin{minipage}[t]{1in}
\small{Hologram demo-picture of woman}
\end{minipage}
}%
%EndExpansion
%TCIMACRO{%
%\TeXButton{Hologram demo-chess pieces}{\marginpar {
%\hspace{-0.5in}
%\begin{minipage}[t]{1in}
%\small{Hologram demo-chess pieces}
%\end{minipage}
%}}}%
%BeginExpansion
\marginpar {
\hspace{-0.5in}
\begin{minipage}[t]{1in}
\small{Hologram demo-chess pieces}
\end{minipage}
}%
%EndExpansion
You may have seen holograms in the past. We have enough understanding of
light to understand how they are generated now.

\FRAME{dtbpF}{2.853in}{1.7772in}{0pt}{}{}{Figure}{\special{language
"Scientific Word";type "GRAPHIC";maintain-aspect-ratio TRUE;display
"USEDEF";valid_file "T";width 2.853in;height 1.7772in;depth
0pt;original-width 8.0834in;original-height 5.0246in;cropleft "0";croptop
"1";cropright "1";cropbottom "0";tempfilename
'LTUWCQ45.wmf';tempfile-properties "XPR";}}

A device for generating a hologram is shown in the figure above. Light from
a laser or other coherent source is expanded and split into two beams. One
travels to a photographic plate, the other is directed to an object. At the
object, light is scattered and the scattered light also reaches the
photographic plate. The combination of the direct and scattered beams
generates a complicated interference pattern.

\bigskip The pattern can be developed (like you develop photographic film).
Once developed, it can be re- illuminated with a direct beam. The emulsion
on the plate creates complicated patterns of light transmission, which
combine to create interference. It is like a very complicated slit pattern
or grating pattern. The result is a three-dimensional image generated by the
interference. The interference pattern generates an image that looks like
the original object.\FRAME{dtbpF}{2.687in}{1.689in}{0pt}{}{}{Figure}{\special%
{language "Scientific Word";type "GRAPHIC";display "USEDEF";valid_file
"T";width 2.687in;height 1.689in;depth 0pt;original-width
8.0834in;original-height 5.0246in;cropleft "0";croptop "1";cropright
"1";cropbottom "0";tempfilename 'LTUWCR46.wmf';tempfile-properties "XPR";}}

\section{Diffraction of X-rays by Crystals}

If we make the wavelength of light very small, then we can deal with very
small diffraction gratings. This concept is used to investigate the
structure of crystals with x-rays. The crystal latus of molecules or atoms
creates the regular pattern we need for a grating. The pattern is three
dimensional, so the patterns are complex.

Let's start with a simple crystal with a square regular latus. $NaCl$ has
such a structure.

\FRAME{dhF}{3.2038in}{1.4387in}{0in}{}{}{Figure}{\special{language
"Scientific Word";type "GRAPHIC";maintain-aspect-ratio TRUE;display
"USEDEF";valid_file "T";width 3.2038in;height 1.4387in;depth
0in;original-width 3.2375in;original-height 1.4378in;cropleft "0";croptop
"1";cropright "1";cropbottom "0";tempfilename
'LX23FV09.wmf';tempfile-properties "XPR";}}If we illuminate the crystal with
x-rays, the x-rays can reflect off the top layer of atoms, or off the second
layer of atoms (or off any other layer, but for now let's just consider two
layers). If the spacing between the layers is $d,$ then the path difference
will be 
\begin{equation}
\delta =2\left( d\sin \left( \theta \right) \right)
\end{equation}%
then for constructive interference%
\begin{equation}
2d\sin \left( \theta \right) =m\lambda \qquad m=1,2,3,\ldots
\end{equation}%
This is known as \emph{Bragg's law.} This relationship can be used to
measure the distance between the crystal planes.

A resulting pattern is given in the following figure.\FRAME{dhFU}{1.6492in}{%
1.6604in}{0pt}{\Qcb{Diffraction image of protein crystal. Hen egg lysozyme,
X-ray souce Bruker I$\protect\mu $S, $\protect\lambda =0.154188\unit{nm}$, $%
45\unit{kV}$, Exposure $10\unit{s}.$}}{}{Figure}{\special{language
"Scientific Word";type "GRAPHIC";maintain-aspect-ratio TRUE;display
"USEDEF";valid_file "T";width 1.6492in;height 1.6604in;depth
0pt;original-width 1.6129in;original-height 1.6233in;cropleft "0";croptop
"1";cropright "1";cropbottom "0";tempfilename
'LTUWCR49.wmf';tempfile-properties "XPR";}}DNA\ makes in interesting
diffraction pattern.\FRAME{dhFU}{1.5806in}{1.8272in}{0pt}{\Qcb{X-ray
diffraction pattern of DNA}}{}{Figure}{\special{language "Scientific
Word";type "GRAPHIC";maintain-aspect-ratio TRUE;display "USEDEF";valid_file
"T";width 1.5806in;height 1.8272in;depth 0pt;original-width
1.5437in;original-height 1.7902in;cropleft "0";croptop "1";cropright
"1";cropbottom "0";tempfilename 'LTUWCR4A.wmf';tempfile-properties "XPR";}}

\section{Transition to the ray model}

%TCIMACRO{%
%\TeXButton{Question 223.12.4}{\marginpar {
%\hspace{-0.5in}
%\begin{minipage}[t]{1in}
%\small{Question 223.12.4}
%\end{minipage}
%}}}%
%BeginExpansion
\marginpar {
\hspace{-0.5in}
\begin{minipage}[t]{1in}
\small{Question 223.12.4}
\end{minipage}
}%
%EndExpansion
In the next figure, two waves of different wavelets go through a single
opening. The wave representing the central maximum is shown in each case,
but not the secondary maxima.\FRAME{dhF}{2.7951in}{2.776in}{0pt}{}{}{Figure}{%
\special{language "Scientific Word";type "GRAPHIC";maintain-aspect-ratio
TRUE;display "USEDEF";valid_file "T";width 2.7951in;height 2.776in;depth
0pt;original-width 2.7527in;original-height 2.7337in;cropleft "0";croptop
"1";cropright "1";cropbottom "0";tempfilename
'S1RCAB00.wmf';tempfile-properties "XPR";}}Notice that the smaller
wavelength has a narrower central maxima as we would expect from%
\begin{equation*}
\sin \left( \theta \right) =1.22\frac{\lambda }{D}
\end{equation*}%
or 
\begin{equation*}
\theta \approx 1.22\frac{\lambda }{D}
\end{equation*}%
we see that the ratio of the wavelength to the hole size determines the
angular extent of the central maxima. The smaller the ratio, the smaller the
central region. We can use this to explain why the wave nature of light was
so hard to find.

The patch of light on a screen that is created by light passing through the
aperture is created by the central maximum. \FRAME{dhF}{3.0191in}{2.2468in}{%
0pt}{}{}{Figure}{\special{language "Scientific Word";type
"GRAPHIC";maintain-aspect-ratio TRUE;display "USEDEF";valid_file "T";width
3.0191in;height 2.2468in;depth 0pt;original-width 2.975in;original-height
2.207in;cropleft "0";croptop "1";cropright "1";cropbottom "0";tempfilename
'S1RCAB01.wmf';tempfile-properties "XPR";}}For the long wavelength (red) the
central maximum is larger than the screen. The short wavelength spot will be
wholly on the screen as shown. The geometric spot is what we would see if
the light traveled straight through the opening. Notice that the short
wavelength spot is closer to the size of the geometric spot. In the limit
that 
\begin{equation*}
\lambda \ll a
\end{equation*}%
or for circular openings 
\begin{equation*}
\lambda \ll D
\end{equation*}%
then 
\begin{equation*}
\theta \approx \frac{\lambda }{a}\approx 0
\end{equation*}%
or%
\begin{equation*}
\theta \approx \frac{\lambda }{D}\approx 0
\end{equation*}%
and the spot size would be very nearly equal to the geometric spot size.

This is the limit we will call the \emph{ray approximation}.

For most of mankind's time on the Earth, it was very hard to build holes
that were comparable to the size of a wavelength of visible light. So it is
no wonder that the waviness of light was missed for so many years.

But this ray limit is very useful for apertures the size of camera lenses.
So starting next lecture we will begin to use this small $\lambda ,$ large
aperture approximation.

\chapter{Ray Model}

%TCIMACRO{%
%\TeXButton{Fundamental Concepts}{\hspace{-1.3in}{\LARGE Fundamental Concepts\vspace{0.25in}}}}%
%BeginExpansion
\hspace{-1.3in}{\LARGE Fundamental Concepts\vspace{0.25in}}%
%EndExpansion

\begin{itemize}
\item When the aperture size is given by $D_{aperture}=\sqrt{2.44\lambda L}$
we are at a critical size bounding the geometric and wave optics regions

\item Coherent light is light that maintains a common phase, direction, and
wavelength.

\item Light reflects from a specular surface with equal angles
\end{itemize}

\section{The Ray Approximation in Geometric Optics}

Last time we said that when the geometric spot size was larger than the spot
due to diffraction, we could ignore diffraction and use the simpler ray
model. This is usually true in our personal experiences. But this may not be
true in experiments or devices we design. We should see where the crossover
point is.

Intuitively, if the aperture and the spot are the same size, that ought to
be some sort of critical point. That is when the aperture size is equal to
the spot size

\begin{equation*}
D_{aperture}=2.44\frac{\lambda }{D_{aperture}}L
\end{equation*}%
This gives 
\begin{equation*}
D_{aperture}=\sqrt{2.44\lambda L}
\end{equation*}

Of course this is for round apertures, but for square apertures we know we
remove the $2.44.$ This gives about a millimeter for visible wavelengths.%
\begin{eqnarray*}
D_{aperture} &=&\sqrt{2.44\left( 500\unit{nm}\right) \left( 1\unit{m}\right) 
} \\
&=&1.\,\allowbreak 104\,5\times 10^{-3}\allowbreak \unit{m}
\end{eqnarray*}

for apertures much larger than a millimeter, we expect interference effects
due to diffraction through the aperture to be much harder to see. We expect
them to be easy to see if the aperture is smaller than a millimeter. But
what about when the aperture is about a millimeter in size? That is a
subject for PH375, and so we will avoid this case in this class. But this is
not too restrictive. Most good optical systems have apertures larger than $1%
\unit{mm}.$ Cell phone cameras may be an exception (but I don't consider
cell phone cameras to be good optical systems). Even our eyes have an
aperture that varies from about $2\unit{mm}$ to about $7\unit{mm},$ so most
common experiences in visible wavelengths will work fine with what we learn.
Note that for microwave or radio wave systems this may really not be true!

How about the other extreme? Suppose $\lambda \gg D.$ This is really beyond
our class (requires partial differential equations), but in the extreme
case, we can use reason to find out what happens. If the opening is much
smaller than the wavelength, then the wave does not see the opening, and no
wave is produced on the other side. This is the case of a microwave oven
door. If the wavelength is much larger than the spacing of the little dots
or lines that span the door, then the waves will not leave the interior of
the microwave oven. Of course as the wavelength becomes closer to $D$ this
is less true, and this case is more challenging to calculate, and we will
save it for a 300 level electrodynamics course.

To summarize%
\begin{equation*}
\begin{tabular}{ll}
$\lambda \ll D$ & Wave nature of light is not visible \\ 
$\lambda \approx D$ & Wave nature of light is apparent \\ 
$\lambda \gg D$ & Little to no penetration of aperture by the wave%
\end{tabular}%
\end{equation*}%
We can see that early researchers might not have spent a lot of time with
sub-millimeter sized holes, so the wave nature of light was not as apparent
to Newton and his contemporaries.

\subsection{The ray model and phase}

%TCIMACRO{%
%\TeXButton{Question 223.13.1}{\marginpar {
%\hspace{-0.5in}
%\begin{minipage}[t]{1in}
%\small{Question 223.13.1}
%\end{minipage}
%}}}%
%BeginExpansion
\marginpar {
\hspace{-0.5in}
\begin{minipage}[t]{1in}
\small{Question 223.13.1}
\end{minipage}
}%
%EndExpansion
There is a further complication that helps to explain why the wave nature of
light was not immediately apparent to early researchers. Let's consider a
light source.\FRAME{dhF}{0.6339in}{1.0957in}{0pt}{}{}{Figure}{\special%
{language "Scientific Word";type "GRAPHIC";maintain-aspect-ratio
TRUE;display "USEDEF";valid_file "T";width 0.6339in;height 1.0957in;depth
0pt;original-width 1.3214in;original-height 2.3039in;cropleft "0";croptop
"1";cropright "1";cropbottom "0";tempfilename
'LTUWCR4B.wmf';tempfile-properties "XPR";}}For a typical light source, the
filament or light emitting diode (LED) is larger than about a millimeter,
which is much larger than the wavelength. So, we should already expect that
diffraction might be hard to see. But the filament is made of hot metal (we
will leave LED workings for another class). The atoms of the hot metal emit
light because of the extra energy they have. The method of producing this
light is that the atom's excited electrons are in upper shells because of
the extra thermal energy provided by the electricity flowing through the
filament. But the electrons eventually fall to their proper shell, and in
doing so they give off the extra energy as light. It is not too hard to
believe that this process of exciting electrons and having them fall back
down is a random process. Each electron that moves starts a wave. The atoms
have different positions, so there will be a path difference $\Delta r$
between each atom's waves. There will also be a time difference $\Delta t$
between when the waves start. We can model this with a $\Delta \phi _{o}.$

%TCIMACRO{%
%\TeXButton{Question 223.13.2}{\marginpar {
%\hspace{-0.5in}
%\begin{minipage}[t]{1in}
%\small{Question 223.13.2}
%\end{minipage}
%}}}%
%BeginExpansion
\marginpar {
\hspace{-0.5in}
\begin{minipage}[t]{1in}
\small{Question 223.13.2}
\end{minipage}
}%
%EndExpansion
It is also true that not all of the electrons fall from the same shell. This
gives us different frequencies, so we expect beating between different waves
from different atoms. It is also true that we have millions of atoms, so we
have millions of waves.

Let's look at just two of these waves

\begin{equation*}
\begin{tabular}{l}
$\lambda =2$ \\ 
$k=\frac{2\pi }{\lambda }$ \\ 
$\omega =1$ \\ 
$\phi _{o}=\frac{\pi }{6}$ \\ 
$t=0$ \\ 
$E_{o}=1\frac{\unit{N}}{\unit{C}}$%
\end{tabular}%
\end{equation*}

\begin{equation*}
E_{1}=E_{\max }\sin \left( kx-\omega t\right)
\end{equation*}

\FRAME{dtbpFX}{2.156in}{0.6875in}{0pt}{}{}{Plot}{\special{language
"Scientific Word";type "MAPLEPLOT";width 2.156in;height 0.6875in;depth
0pt;display "USEDEF";plot_snapshots TRUE;mustRecompute FALSE;lastEngine
"MuPAD";xmin "0";xmax "5.001000";xviewmin "0";xviewmax "5.001000";yviewmin
"-2";yviewmax "2";viewset"XY";rangeset"X";plottype 4;labeloverrides
2;y-label "E";axesFont "Times New
Roman,12,0000000000,useDefault,normal";numpoints 100;plotstyle
"patch";axesstyle "normal";axestips FALSE;xis \TEXUX{x};var1name
\TEXUX{$x$};function \TEXUX{$\allowbreak \sin \pi x$};linecolor
"blue";linestyle 1;pointstyle "point";linethickness 3;lineAttributes
"Solid";var1range "0,5.001000";num-x-gridlines 100;curveColor
"[flat::RGB:0x000000ff]";curveStyle "Line";VCamFile
'MHE8V007.xvz';valid_file "T";tempfilename
'MHE8V005.wmf';tempfile-properties "XPR";}}

\begin{equation*}
E_{2}=E_{\max }\sin \left( kx-\omega t+\phi _{o}\right)
\end{equation*}%
\FRAME{dtbpFX}{2.156in}{0.6875in}{0pt}{}{}{Plot}{\special{language
"Scientific Word";type "MAPLEPLOT";width 2.156in;height 0.6875in;depth
0pt;display "USEDEF";plot_snapshots TRUE;mustRecompute FALSE;lastEngine
"MuPAD";xmin "0";xmax "5.001000";xviewmin "0";xviewmax "5.001000";yviewmin
"-2";yviewmax "2";viewset"XY";rangeset"X";plottype 4;labeloverrides
2;y-label "E";axesFont "Times New
Roman,12,0000000000,useDefault,normal";numpoints 100;plotstyle
"patch";axesstyle "normal";axestips FALSE;xis \TEXUX{x};var1name
\TEXUX{$x$};function \TEXUX{$\sin \left( \frac{1}{6}\pi +\pi x\right)
$};linecolor "maroon";linestyle 1;pointstyle "point";linethickness
3;lineAttributes "Solid";var1range "0,5.001000";num-x-gridlines
100;curveColor "[flat::RGB:0x00800000]";curveStyle "Line";VCamFile
'MHE8V708.xvz';valid_file "T";tempfilename
'MHE8V706.wmf';tempfile-properties "XPR";}}then%
\begin{equation*}
E_{r}=E_{\max }\sin \left( kx-\omega t\right) +E_{o}\sin \left( kx-\omega
t+\phi _{o}\right)
\end{equation*}

We found a nice meaningful way to write the resultant wave.%
\begin{equation*}
E_{r}=2E_{\max }\cos \left( \frac{\phi _{o}}{2}\right) \sin \left( kx-\omega
t+\frac{\phi _{o}}{2}\right)
\end{equation*}%
\FRAME{dtbpFX}{2.354in}{0.5846in}{0pt}{}{}{Plot}{\special{language
"Scientific Word";type "MAPLEPLOT";width 2.354in;height 0.5846in;depth
0pt;display "USEDEF";plot_snapshots TRUE;mustRecompute FALSE;lastEngine
"MuPAD";xmin "0";xmax "5.001000";xviewmin "0";xviewmax "5.001000";yviewmin
"-2";yviewmax "2";viewset"XY";rangeset"X";plottype 4;labeloverrides
2;y-label "E";axesFont "Times New
Roman,12,0000000000,useDefault,normal";numpoints 100;plotstyle
"patch";axesstyle "normal";axestips FALSE;xis \TEXUX{x};var1name
\TEXUX{$x$};function \TEXUX{$\sin \left( \frac{1}{6}\pi +\pi x\right) +\sin
\pi x$};linecolor "green";linestyle 1;pointstyle "point";linethickness
3;lineAttributes "Solid";var1range "0,5.001000";num-x-gridlines
100;curveColor "[flat::RGB:0x00008000]";curveStyle "Line";VCamFile
'MHE8VH09.xvz';valid_file "T";tempfilename
'MHE8VH07.wmf';tempfile-properties "XPR";}}

But suppose we complicate the situation by sending out lots of waves at
random times, each with different amplitudes and wavelengths. If we look at
a single point for a specific time, we might be experiencing interference,
but it would be hard to tell. Lets try this mathematically. I will combine
many waves with random phases, some coming from the right and some coming
from the left.

\begin{eqnarray*}
E_{1} &=&E_{\max }\sin \left( 5x-\omega t+\frac{\pi }{4}\right) +0.5E_{\max
}\sin \left( 0.2x-\omega t-\frac{\pi }{6}\right) \\
&&+3.6E_{\max }\sin \left( .4x-\omega t+\frac{\pi }{10}\right) +4E_{\max
}\sin \left( 20x-\omega t-\frac{\pi }{7}\right) \\
&&+.2E_{\max }\sin \left( 15x-\omega t+1\right) +0.7E_{\max }\sin \left(
.7x-\omega t-.25\right)
\end{eqnarray*}%
Here is what $E_{1}$ would look like.\FRAME{dtbpFX}{2.354in}{1.1943in}{0pt}{%
}{}{Plot}{\special{language "Scientific Word";type "MAPLEPLOT";width
2.354in;height 1.1943in;depth 0pt;display "USEDEF";plot_snapshots
TRUE;mustRecompute FALSE;lastEngine "MuPAD";xmin "0";xmax "5";xviewmin
"0";xviewmax "5";yviewmin "-20";yviewmax
"20";viewset"XY";rangeset"X";plottype 4;labeloverrides 2;y-label
"E";axesFont "Times New Roman,12,0000000000,useDefault,normal";numpoints
100;plotstyle "patch";axesstyle "normal";axestips FALSE;xis
\TEXUX{x};var1name \TEXUX{$x$};function \TEXUX{$0.2\sin \left( 15x+1\right)
+0.7\sin \left( 0.7x-0.25\right) +\allowbreak 0.5\sin \left(
0.2x-\frac{1}{6}\pi \right) +3.\,\allowbreak 6\sin \left( \frac{1}{10}\pi
+0.4x\right) +\allowbreak \sin \left( \frac{1}{4}\pi +5x\right) +4\sin
\left( 20x-\frac{1}{7}\pi \right) $};linecolor "blue";linestyle 1;pointstyle
"point";linethickness 1;lineAttributes "Solid";var1range
"0,5";num-x-gridlines 100;curveColor "[flat::RGB:0x000000ff]";curveStyle
"Line";VCamFile 'MHE8WY0C.xvz';valid_file "T";tempfilename
'MHE8WY08.wmf';tempfile-properties "XPR";}}And now let's make another random
wave, $E_{2}$%
\begin{eqnarray*}
E_{2} &=&E_{\max }\sin \left( 0.2x+\omega t+\pi \right) +2E_{\max }\sin
\left( 5x+\omega t+\frac{\pi }{6}\right) \\
&&+6E_{\max }\sin \left( 0.4x+\omega t+\frac{\pi }{3.5}\right) +0.4E_{\max
}\sin \left( 20x+\omega t-0\right) \\
&&+E_{\max }\sin \left( 15x+\omega t+1\right) +0.7E_{\max }\sin \left(
.7x+\omega t-4\right)
\end{eqnarray*}%
Which looks like this\FRAME{dtbpFX}{2.271in}{1.0343in}{0pt}{}{}{Plot}{%
\special{language "Scientific Word";type "MAPLEPLOT";width 2.271in;height
1.0343in;depth 0pt;display "USEDEF";plot_snapshots TRUE;mustRecompute
FALSE;lastEngine "MuPAD";xmin "0";xmax "5.001000";xviewmin "0";xviewmax
"5.001000";yviewmin "-20";yviewmax "20";viewset"XY";rangeset"X";plottype
4;labeloverrides 2;y-label "E";axesFont "Times New
Roman,12,0000000000,useDefault,normal";numpoints 100;plotstyle
"patch";axesstyle "normal";axestips FALSE;xis \TEXUX{x};var1name
\TEXUX{$x$};function \TEXUX{$0.7\sin \left( 0.7x-4\right) +6\sin \left(
0.285\,71\pi +0.4x\right) +\allowbreak \sin \left( 15x+1\right) -\sin
0.2x+2\sin \left( \frac{1}{6}\pi +5x\right) +0.4\allowbreak \sin
20x$};linecolor "red";linestyle 1;pointstyle "point";linethickness
2;lineAttributes "Solid";var1range "0,5.001000";num-x-gridlines
100;curveColor "[flat::RGB:0x00ff0000]";curveStyle "Line";VCamFile
'MHE8X80D.xvz';valid_file "T";tempfilename
'MHE8X809.wmf';tempfile-properties "XPR";}}

Then $E_{1}+E_{2}$ looks like\FRAME{dtbpFX}{2.3333in}{1.1182in}{0pt}{}{}{Plot%
}{\special{language "Scientific Word";type "MAPLEPLOT";width 2.3333in;height
1.1182in;depth 0pt;display "USEDEF";plot_snapshots TRUE;mustRecompute
FALSE;lastEngine "MuPAD";xmin "0";xmax "5.001000";xviewmin "0";xviewmax
"5.001000";yviewmin "-20";yviewmax "20";viewset"XY";rangeset"X";plottype
4;labeloverrides 2;y-label "E";axesFont "Times New
Roman,12,0000000000,useDefault,normal";numpoints 100;plotstyle
"patch";axesstyle "normal";axestips FALSE;xis \TEXUX{x};var1name
\TEXUX{$x$};function \TEXUX{$\allowbreak 0.7\sin \left( 0.7x-4\right) +6\sin
\left( 0.285\,71\pi +0.4x\right) +\allowbreak 1.\,\allowbreak 2\sin \left(
15x+1\right) +0.7\sin \left( 0.7x-0.25\right) +\allowbreak 0.5\sin \left(
0.2x-\frac{1}{6}\pi \right) +3.\,\allowbreak 6\sin \left( \frac{1}{10}\pi
+0.4x\right) -\allowbreak \sin 0.2x+\sin \left( \frac{1}{4}\pi +5x\right)
+2\sin \left( \frac{1}{6}\pi +5x\right) +4\sin \left( 20x-\frac{1}{7}\pi
\right) +\allowbreak 0.4\sin 20x$};linecolor "green";linestyle 1;pointstyle
"point";linethickness 1;lineAttributes "Solid";var1range
"0,5.001000";num-x-gridlines 100;curveColor
"[flat::RGB:0x00008000]";curveStyle "Line";VCamFile
'MHE8XG0E.xvz';valid_file "T";tempfilename
'MHE8XG0A.wmf';tempfile-properties "XPR";}}

In this example, you could think about the superposition of $E_{1}$ and $%
E_{2}$ and predict the outcome, but if there were millions of waves, each
with it's own wavelength, phase, and amplitude, the situation would be
hopeless. Note that the fluctuations in these waves are much more frequent
than our original waves. With all the added waves, we get a rapid change in
amplitude.

Now if these waves are light waves, our eyes and most detectors are not able
to react fast enough to detect the rapid fluctuations. So if there is
constructive or destructive interference that might be simple enough to
distinguish, the interference pattern will change so fast that we will miss
it due to our detection systems' integration times. To describe this rapidly
fluctuating interference pattern that we can't track with our detectors, we
just say that light bulbs emit \emph{incoherent light}. The ray
approximation assumes incoherent light.

But then light bulbs and hot ovens and most things must emit incoherent
light. Does any thing emit coherent light? Sure, today the easiest source of
coherent light is a laser. That is why I have used lasers in the class
demonstrations so far. Really though, even a laser is not perfectly
coherent. One property of the laser is that it produces light with a long 
\emph{coherence length}, or it produces light that can be treated under most
circumstances as begin monochromatic and having a single phase across the
wave for much of the beam length. Radar and microwave transmitters emit
coherent light (but at frequencies we can't see) and so do radio stations.

In the past, one could carefully create a monochromatic beam with filters.
Then split the beam into two beams and remix the two beams. This would
generate two mostly coherent sources if the distances traveled were not too
large. This is what Young did.

\subsection{Coherency}

%TCIMACRO{%
%\TeXButton{Question 223.13.3}{\marginpar {
%\hspace{-0.5in}
%\begin{minipage}[t]{1in}
%\small{Question 223.13.3}
%\end{minipage}
%}}}%
%BeginExpansion
\marginpar {
\hspace{-0.5in}
\begin{minipage}[t]{1in}
\small{Question 223.13.3}
\end{minipage}
}%
%EndExpansion
To be coherent,

\begin{enumerate}
\item A given part of the wave must maintain a constant phase with respect
to the rest of the wave.

\item The wave must be monochromatic
\end{enumerate}

These are very hard criteria to achieve. Most light, like that from our
light bulb, is not coherent.

\section{Reflection}

%TCIMACRO{%
%\TeXButton{Question 223.13.4}{\marginpar {
%\hspace{-0.5in}
%\begin{minipage}[t]{1in}
%\small{Question 223.13.4}
%\end{minipage}
%}}}%
%BeginExpansion
\marginpar {
\hspace{-0.5in}
\begin{minipage}[t]{1in}
\small{Question 223.13.4}
\end{minipage}
}%
%EndExpansion
In the \emph{Star Wars} movies inter-galactic star ships blast each other
with laser cannons. The laser beams streak across the screen. This is
dramatic, but not realistic. For us to see the light, some of the light must
get to our eyes. The light must either travel directly to our eyes from the
source, or it must bounce off of something.

%TCIMACRO{%
%\TeXButton{Question 223.13.5}{\marginpar {
%\hspace{-0.5in}
%\begin{minipage}[t]{1in}
%\small{Question 223.13.5}
%\end{minipage}
%}}}%
%BeginExpansion
\marginpar {
\hspace{-0.5in}
\begin{minipage}[t]{1in}
\small{Question 223.13.5}
\end{minipage}
}%
%EndExpansion
Using the ray approximation we wish to find what happens when a bundle of
rays reaches a boundary between media. If the media boundary is very smooth,
then the rays are reflected in a uniform way. This is called \emph{specular }%
reflection%
%TCIMACRO{%
%\TeXButton{Specular and Diffuse Reflector Demo}{\marginpar {
%\hspace{-0.5in}
%\begin{minipage}[t]{1in}
%\small{Specular and Diffuse Reflector Demo}
%\end{minipage}
%}}}%
%BeginExpansion
\marginpar {
\hspace{-0.5in}
\begin{minipage}[t]{1in}
\small{Specular and Diffuse Reflector Demo}
\end{minipage}
}%
%EndExpansion
\FRAME{dhF}{2.3895in}{2.1344in}{0pt}{}{}{Figure}{\special{language
"Scientific Word";type "GRAPHIC";maintain-aspect-ratio TRUE;display
"USEDEF";valid_file "T";width 2.3895in;height 2.1344in;depth
0pt;original-width 2.3497in;original-height 2.0954in;cropleft "0";croptop
"1";cropright "1";cropbottom "0";tempfilename
'LTUWCR4I.wmf';tempfile-properties "XPR";}}If it is not smooth, then
something different happens. The rays are reflected, but they are reflected
randomly

\FRAME{dhF}{2.2131in}{1.7193in}{0pt}{}{}{Figure}{\special{language
"Scientific Word";type "GRAPHIC";maintain-aspect-ratio TRUE;display
"USEDEF";valid_file "T";width 2.2131in;height 1.7193in;depth
0pt;original-width 1.7936in;original-height 1.388in;cropleft "0";croptop
"1";cropright "1";cropbottom "0";tempfilename
'LTUWCR4J.wmf';tempfile-properties "XPR";}}This is called \emph{diffuse}
reflection%
%TCIMACRO{%
%\TeXButton{Question 223.13.6}{\marginpar {
%\hspace{-0.5in}
%\begin{minipage}[t]{1in}
%\small{Question 223.13.6}
%\end{minipage}
%}}}%
%BeginExpansion
\marginpar {
\hspace{-0.5in}
\begin{minipage}[t]{1in}
\small{Question 223.13.6}
\end{minipage}
}%
%EndExpansion

This difference can be seen in real life\FRAME{dhF}{3.8441in}{1.1718in}{0in}{%
}{}{Figure}{\special{language "Scientific Word";type
"GRAPHIC";maintain-aspect-ratio TRUE;display "USEDEF";valid_file "T";width
3.8441in;height 1.1718in;depth 0in;original-width 3.7948in;original-height
1.1381in;cropleft "0";croptop "1";cropright "1";cropbottom "0";tempfilename
'LYFL0M04.wmf';tempfile-properties "XPR";}}

We said the surface must be smooth for there to be specular reflection. What
does smooth mean? Generally the size of the rough spots must be much smaller
than a wavelength to be considered smooth. So suppose we have a red laser.
How small do the surface variations have to be for the surface to be
considered smooth? The wavelength of a $HeNe$ laser is 
\begin{equation*}
\lambda _{HeNe}=633\unit{nm}
\end{equation*}%
This is very small. Modern optics for remote sensing are often manufactured
to $1/10$ of a wavelength, which would be $63\unit{nm}.$

How about a microwave beam of light like your cell phone uses?

\begin{eqnarray*}
c &=&\lambda f \\
\lambda &=&\frac{c}{f}=\frac{3\times 10^{8}\frac{\unit{m}}{\unit{s}}}{1\unit{%
GHz}} \\
&=&\allowbreak 0.3\unit{m}
\end{eqnarray*}

We can see that we must be careful in our definition of \textquotedblleft
smooth.\textquotedblright

\subsection{Law of reflection}

%TCIMACRO{%
%\TeXButton{Ball Bounce Demo}{\marginpar {
%\hspace{-0.5in}
%\begin{minipage}[t]{1in}
%\small{Ball Bounce Demo}
%\end{minipage}
%}}}%
%BeginExpansion
\marginpar {
\hspace{-0.5in}
\begin{minipage}[t]{1in}
\small{Ball Bounce Demo}
\end{minipage}
}%
%EndExpansion

Experience shows that if we do have a smooth surface, that light bounces
much like a ball. This is why Newton though light was a particle. Suppose we
take a flat surface and we shine a light on it. We have a ray that
approaches at an angle $\theta _{i}$ measured from the normal. Then the
reflected ray will leave the surface with an angle $\theta _{r}$ measured
from the normal such that 
\begin{equation*}
\theta _{r}=\theta _{i}
\end{equation*}

\FRAME{dhF}{1.8585in}{1.1857in}{0in}{}{}{Figure}{\special{language
"Scientific Word";type "GRAPHIC";maintain-aspect-ratio TRUE;display
"USEDEF";valid_file "T";width 1.8585in;height 1.1857in;depth
0in;original-width 1.8213in;original-height 1.1519in;cropleft "0";croptop
"1";cropright "1";cropbottom "0";tempfilename
'LTUWCS4L.wmf';tempfile-properties "XPR";}}This is called the \emph{law of
reflection}.

%TCIMACRO{%
%\TeXButton{Question 223.13.7}{\marginpar {
%\hspace{-0.5in}
%\begin{minipage}[t]{1in}
%\small{Question 223.13.7}
%\end{minipage}
%}}}%
%BeginExpansion
\marginpar {
\hspace{-0.5in}
\begin{minipage}[t]{1in}
\small{Question 223.13.7}
\end{minipage}
}%
%EndExpansion
%TCIMACRO{%
%\TeXButton{Question 223.13.8}{\marginpar {
%\hspace{-0.5in}
%\begin{minipage}[t]{1in}
%\small{Question 223.13.8}
%\end{minipage}
%}}}%
%BeginExpansion
\marginpar {
\hspace{-0.5in}
\begin{minipage}[t]{1in}
\small{Question 223.13.8}
\end{minipage}
}%
%EndExpansion

\subsection{Retroreflection}

Let's take an example\FRAME{dtbpF}{1.5523in}{1.452in}{0pt}{}{}{Figure}{%
\special{language "Scientific Word";type "GRAPHIC";maintain-aspect-ratio
TRUE;display "USEDEF";valid_file "T";width 1.5523in;height 1.452in;depth
0pt;original-width 3.8579in;original-height 3.608in;cropleft "0";croptop
"1";cropright "1";cropbottom "0";tempfilename
'LTUWCS4M.wmf';tempfile-properties "XPR";}}

Let's take our system to be two mirrors set at a right angle. We have a beam
of light incident at angle $\theta _{1}$. By the law of reflection, it must
leave the mirror at $\theta _{2}=\theta _{1}.$ We can see that $\alpha $
must be $90\unit{%
%TCIMACRO{\U{b0}}%
%BeginExpansion
{{}^\circ}%
%EndExpansion
}-\theta _{2}$ and it is clear that $\theta _{3}=\alpha .$ By the law of
reflection, $\theta _{3}=\theta _{4}.$ Then, since 
\begin{eqnarray*}
90\unit{%
%TCIMACRO{\U{b0}}%
%BeginExpansion
{{}^\circ}%
%EndExpansion
} &=&\theta _{2}+\alpha \\
&=&\theta _{2}+\theta _{3}
\end{eqnarray*}%
and 
\begin{equation*}
90\unit{%
%TCIMACRO{\U{b0}}%
%BeginExpansion
{{}^\circ}%
%EndExpansion
}=\theta _{1}+\theta _{4}
\end{equation*}%
then the total angular change is 
\begin{equation*}
90\unit{%
%TCIMACRO{\U{b0}}%
%BeginExpansion
{{}^\circ}%
%EndExpansion
}+90\unit{%
%TCIMACRO{\U{b0}}%
%BeginExpansion
{{}^\circ}%
%EndExpansion
}=180\unit{%
%TCIMACRO{\U{b0}}%
%BeginExpansion
{{}^\circ}%
%EndExpansion
}
\end{equation*}%
or the outgoing ray is sent back toward the source! If we do this in three
dimensions we have a corner cube.

\FRAME{dhFU}{2.1932in}{2.7908in}{0pt}{\Qcb{Radar retroreflector tower
located in the center of Yucca Flat dry lake bed. Used as a radar target by
maneuvering aircraft during "inert" contact fusing bomb drops at Yucca Flat.
Sandia National Laboratories conducted the tests on the lake bed from 1954
to 1956. (Image in the Public Domain in the United States\_}}{}{Figure}{%
\special{language "Scientific Word";type "GRAPHIC";maintain-aspect-ratio
TRUE;display "USEDEF";valid_file "T";width 2.1932in;height 2.7908in;depth
0pt;original-width 2.1543in;original-height 2.7484in;cropleft "0";croptop
"1";cropright "1";cropbottom "0";tempfilename
'LYF41G00.wmf';tempfile-properties "XPR";}}

The figure above is a radar corner cube set. The one below is an optical
corner cube set on the moon.\FRAME{dhFU}{2.2226in}{2.7631in}{0pt}{\Qcb{%
Apollo Retroreflector (Images in the Public Domain courtesy NASA)}}{}{Figure%
}{\special{language "Scientific Word";type "GRAPHIC";maintain-aspect-ratio
TRUE;display "USEDEF";valid_file "T";width 2.2226in;height 2.7631in;depth
0pt;original-width 2.1819in;original-height 2.7207in;cropleft "0";croptop
"1";cropright "1";cropbottom "0";tempfilename
'LYF4G901.wmf';tempfile-properties "XPR";}}We use this optical corner cube
array to reflect light off of the moon. The time it takes the light to go to
the moon and back can be converted into an Earth-Moon distance for
monitoring how close the moon is to the Earth.%
%TCIMACRO{%
%\TeXButton{Question 223.13.9}{\marginpar {
%\hspace{-0.5in}
%\begin{minipage}[t]{1in}
%\small{Question 223.13.9}
%\end{minipage}
%}}}%
%BeginExpansion
\marginpar {
\hspace{-0.5in}
\begin{minipage}[t]{1in}
\small{Question 223.13.9}
\end{minipage}
}%
%EndExpansion

\section{Reflections, Objects, and seeing}

Armed with the law of reflection, we can start to understand how we see
things. Using the ray concept, we can say that a ray of light must leave the
light source. That ray then reflects from something. Suppose you look at the
person sitting next to you in class. Light from the ceiling lights has
reflected from that person. But is the person a specular or diffuse
reflector?

Once again, we can only give an answer relative to the wavelength of light.
For visible light, your neighbors do not look like mirrors. The are diffuse
reflectors. Light bounces off of them in every direction. Your eye is
designed to take this diverging set of rays and condense it into a picture
of the person that your brain can interpret.\FRAME{dhF}{3.173in}{1.3534in}{%
0pt}{}{}{Figure}{\special{language "Scientific Word";type
"GRAPHIC";maintain-aspect-ratio TRUE;display "USEDEF";valid_file "T";width
3.173in;height 1.3534in;depth 0pt;original-width 3.128in;original-height
1.3188in;cropleft "0";croptop "1";cropright "1";cropbottom "0";tempfilename
'LTUWCS4N.wmf';tempfile-properties "XPR";}}We tend to not draw the rays that
bounce off the diffuse reflector but that don't get to our eyes, because we
don't see them. So a ray diagram is usually much simpler.\FRAME{dhF}{2.2926in%
}{1.9951in}{0pt}{}{}{Figure}{\special{language "Scientific Word";type
"GRAPHIC";maintain-aspect-ratio TRUE;display "USEDEF";valid_file "T";width
2.2926in;height 1.9951in;depth 0pt;original-width 2.2528in;original-height
1.9562in;cropleft "0";croptop "1";cropright "1";cropbottom "0";tempfilename
'LTUWCS4O.wmf';tempfile-properties "XPR";}}This is easy to understand, but
we must keep in mind the wildly fluctuating waviness that is masked by our
macroscopic view.%
%TCIMACRO{%
%\TeXButton{Question 223.13.10}{\marginpar {
%\hspace{-0.5in}
%\begin{minipage}[t]{1in}
%\small{Question 223.13.10}
%\end{minipage}
%}}}%
%BeginExpansion
\marginpar {
\hspace{-0.5in}
\begin{minipage}[t]{1in}
\small{Question 223.13.10}
\end{minipage}
}%
%EndExpansion

We can use the idea of a ray diagram to solve problems. Suppose you hold a
mirror half a meter in front of you and look at your reflection. Where would
the reflection appear to be?\FRAME{dhF}{1.9847in}{1.9527in}{0pt}{}{}{Figure}{%
\special{language "Scientific Word";type "GRAPHIC";maintain-aspect-ratio
TRUE;display "USEDEF";valid_file "T";width 1.9847in;height 1.9527in;depth
0pt;original-width 1.9458in;original-height 1.9147in;cropleft "0";croptop
"1";cropright "1";cropbottom "0";tempfilename
'LTUWCS4P.wmf';tempfile-properties "XPR";}}%
%TCIMACRO{%
%\TeXButton{Question 223.13.11}{\marginpar {
%\hspace{-0.5in}
%\begin{minipage}[t]{1in}
%\small{Question 223.13.11}
%\end{minipage}
%}}}%
%BeginExpansion
\marginpar {
\hspace{-0.5in}
\begin{minipage}[t]{1in}
\small{Question 223.13.11}
\end{minipage}
}%
%EndExpansion
Knowing that rays travel in straight lines and that our mind interprets rays
as going in straight lines, then we can use rays to see where the light
appears to be from. The image is half a meter behind the mirror. Now suppose
we look at an image of that image in a mirror behind us. \FRAME{dhF}{4.3777in%
}{1.6734in}{0pt}{}{}{Figure}{\special{language "Scientific Word";type
"GRAPHIC";maintain-aspect-ratio TRUE;display "USEDEF";valid_file "T";width
4.3777in;height 1.6734in;depth 0pt;original-width 2.5858in;original-height
0.9712in;cropleft "0";croptop "1";cropright "1";cropbottom "0";tempfilename
'LTUWCS4Q.wmf';tempfile-properties "XPR";}}The ray diagram makes it easy to
see that the image will appear to be $2\unit{m}$ behind the big mirror.

\chapter{Refraction and images}

\FRAME{dhF}{3.8303in}{2.8876in}{0pt}{}{}{Figure}{\special{language
"Scientific Word";type "GRAPHIC";maintain-aspect-ratio TRUE;display
"USEDEF";valid_file "T";width 3.8303in;height 2.8876in;depth
0pt;original-width 3.781in;original-height 2.8444in;cropleft "0";croptop
"1";cropright "1";cropbottom "0";tempfilename
'LTUWCS4R.wmf';tempfile-properties "XPR";}}We studied light reflecting from
a surface. We can see the reflection in the image above. But light also is
transmitted through the piece of glass in the figure. Note the change in
direction at the interfaces. This is penetration of a material by light is
called refraction, and will be the subject of this lecture.

%TCIMACRO{%
%\TeXButton{Fundamental Concepts}{\hspace{-1.3in}{\LARGE Fundamental Concepts\vspace{0.25in}}}}%
%BeginExpansion
\hspace{-1.3in}{\LARGE Fundamental Concepts\vspace{0.25in}}%
%EndExpansion

\begin{itemize}
\item Refraction is a change of direction of a light ray as it crosses an
interface

\item The wavelength of the light changes at an interface

\item The angle changes according to Snell's law $n_{i}\sin \theta
_{i}=n_{t}\sin \theta _{t}$

\item When going from a high index to a low index material, the light may
totally reflect, with no transmission

\item Refraction can form images
\end{itemize}

\section{Refraction}

Not all surfaces reflect all the light. Some, like the lenses shown below,
reflect some light at visible wavelengths, but are transparent so most of
the light travels through them. \FRAME{dhF}{3.2846in}{1.6042in}{0pt}{}{}{%
Figure}{\special{language "Scientific Word";type
"GRAPHIC";maintain-aspect-ratio TRUE;display "USEDEF";valid_file "T";width
3.2846in;height 1.6042in;depth 0pt;original-width 3.2396in;original-height
1.5679in;cropleft "0";croptop "1";cropright "1";cropbottom "0";tempfilename
'LTUWCS4S.wmf';tempfile-properties "XPR";}} We need a way to deal with
transparent materials. This is tricky, because different wavelengths of
light penetrate different materials in different ways. As an example, this
is also a lens\FRAME{dhFU}{1.0732in}{1.6284in}{0pt}{\Qcb{IR lens. (Image in
the Public Domain, courtesy US\ Navy)}}{}{Figure}{\special{language
"Scientific Word";type "GRAPHIC";maintain-aspect-ratio TRUE;display
"USEDEF";valid_file "T";width 1.0732in;height 1.6284in;depth
0pt;original-width 5.4587in;original-height 8.3195in;cropleft "0";croptop
"1";cropright "1";cropbottom "0";tempfilename
'LTUWCT4T.bmp';tempfile-properties "XPR";}}but it clearly is not transparent
at visible wavelengths. But it is transparent in the infrared. So what might
be transparent at one wavelength might not be at another.

When light travels into a material, we say it is transmitted. The situation
is shown schematically below.

\FRAME{dhF}{2.348in}{1.9666in}{0in}{}{}{Figure}{\special{language
"Scientific Word";type "GRAPHIC";maintain-aspect-ratio TRUE;display
"USEDEF";valid_file "T";width 2.348in;height 1.9666in;depth
0in;original-width 2.3082in;original-height 1.9285in;cropleft "0";croptop
"1";cropright "1";cropbottom "0";tempfilename
'LTUWCT4U.wmf';tempfile-properties "XPR";}}

In the figure we see a ray incident on an air-glass boundary. Some of the
light is reflected just as we saw before. But some passes into the glass.
Notice that the angle between the normal and the new transmitted ray is 
\emph{not} equal to the incident ray. We say the ray has been bent or \emph{%
refracted} by the change of media. Many experiments were performed to find a
relationship between the incident and the refracted angles. It was found
that 
\begin{equation}
\frac{\sin \left( \theta _{2}\right) }{\sin \left( \theta _{1}\right) }=%
\frac{v_{2}}{v_{1}}=\text{constant}
\end{equation}%
Many optics books write this as 
\begin{equation}
\frac{\sin \left( \theta _{t}\right) }{\sin \left( \theta _{i}\right) }=%
\frac{v_{2}}{v_{1}}=\text{constant}
\end{equation}%
where the subscript $i$ stands for \textquotedblleft
incident\textquotedblright\ and the subscript $t$ stands for
\textquotedblleft transmitted.\textquotedblright\ Note that we are using the
fact that the average speed of light changes in a material. We should
probably recall why this should occur

\subsection{Speed of light in a material}

In a vacuum, light travels as a disturbance in the electromagnetic field
with nothing to encounter. In a material (like glass) the light waves
continually hit atoms. We have not studied antennas, but I\ think many of
you know that an antenna works because the electrons in the metal act like
driven harmonic oscillators. The incoming radio waves drive the electron
motion. Here each atom has electrons, and the atoms act like little
antennas, their electrons moving and absorbing the light. But the atom
cannot keep the extra energy (PH433), so it is readmitted. It travels to the
next atom and the process repeats. Quantum mechanics tells us that there is
a time delay in the re-emission of the light. This causes a secondary wave
to mix with the incoming wave. The combined result is that the propagation
energy in the light wave slows down. Thus the speed of light is slower in a
material.\FRAME{dhF}{4.6267in}{1.8135in}{0pt}{}{}{Figure}{\special{language
"Scientific Word";type "GRAPHIC";maintain-aspect-ratio TRUE;display
"USEDEF";valid_file "T";width 4.6267in;height 1.8135in;depth
0pt;original-width 4.574in;original-height 1.7763in;cropleft "0";croptop
"1";cropright "1";cropbottom "0";tempfilename
'LTUWCT4V.wmf';tempfile-properties "XPR";}}But why does this slowing cause
the light ray to bend? As a mechanical analog, consider a rolling barrel.%
\FRAME{dhF}{2.4604in}{1.8005in}{0pt}{}{}{Figure}{\special{language
"Scientific Word";type "GRAPHIC";maintain-aspect-ratio TRUE;display
"USEDEF";valid_file "T";width 2.4604in;height 1.8005in;depth
0pt;original-width 2.4189in;original-height 1.7634in;cropleft "0";croptop
"1";cropright "1";cropbottom "0";tempfilename
'LTUWCT4W.wmf';tempfile-properties "XPR";}}As the barrel rolls from a flat
low-friction concrete to a higher-friction grass lawn, the friction slows
the barrel. If the barrel hits the lawn parallel to the boundary (so it's
velocity vector is perpendicular to the boundary), then the barrel continues
in the same direction at the slower speed. But if it hits at an angle, the
leading edge is slowed first. \FRAME{dhF}{2.3575in}{1.7236in}{0pt}{}{}{Figure%
}{\special{language "Scientific Word";type "GRAPHIC";maintain-aspect-ratio
TRUE;display "USEDEF";valid_file "T";width 2.3575in;height 1.7236in;depth
0pt;original-width 2.725in;original-height 1.9847in;cropleft "0";croptop
"1";cropright "1";cropbottom "0";tempfilename
'LTUWCT4X.wmf';tempfile-properties "XPR";}}This makes the trailing edge
travel faster than the leading edge, and the barrel turns slightly.\FRAME{dhF%
}{2.4267in}{1.7711in}{0pt}{}{}{Figure}{\special{language "Scientific
Word";type "GRAPHIC";maintain-aspect-ratio TRUE;display "USEDEF";valid_file
"T";width 2.4267in;height 1.7711in;depth 0pt;original-width
3.7948in;original-height 2.7622in;cropleft "0";croptop "1";cropright
"1";cropbottom "0";tempfilename 'LTUWCT4Y.wmf';tempfile-properties "XPR";}}

We expect the same behavior from light.\FRAME{dhF}{2.8236in}{2.0237in}{0pt}{%
}{}{Figure}{\special{language "Scientific Word";type
"GRAPHIC";maintain-aspect-ratio TRUE;display "USEDEF";valid_file "T";width
2.8236in;height 2.0237in;depth 0pt;original-width 2.7804in;original-height
1.9847in;cropleft "0";croptop "1";cropright "1";cropbottom "0";tempfilename
'LTUWCT4Z.wmf';tempfile-properties "XPR";}}We can see that the left hand
side of the wave hits the slower (green) material first and slows down. The
rest of the wave front moves quicker. The result is the turning of the wave.

%TCIMACRO{%
%\TeXButton{Question 223.14.1}{\marginpar {
%\hspace{-0.5in}
%\begin{minipage}[t]{1in}
%\small{Question 223.14.1}
%\end{minipage}
%}}}%
%BeginExpansion
\marginpar {
\hspace{-0.5in}
\begin{minipage}[t]{1in}
\small{Question 223.14.1}
\end{minipage}
}%
%EndExpansion

\subsection{Change of wavelength}

We have found that when a wave enters a material, its speed may change. But
we remember from wave theory%
\begin{equation}
v=\lambda f
\end{equation}

But it is time to review: does $\lambda $ change, or does $f$ change? If you
will recall, we found that the change in speed at the boundary changes the
wavelength. Recall that if we go from a fast material to a slow material,
the forward part of the wave slows and the rest of the wave catches up to
it. \FRAME{dhF}{2.2727in}{1.2644in}{0pt}{}{}{Figure}{\special{language
"Scientific Word";type "GRAPHIC";maintain-aspect-ratio TRUE;display
"USEDEF";valid_file "T";width 2.2727in;height 1.2644in;depth
0pt;original-width 3.4618in;original-height 1.9147in;cropleft "0";croptop
"1";cropright "1";cropbottom "0";tempfilename
'LYGRON00.wmf';tempfile-properties "XPR";}}This will comperes pulses, and
lower the wavelength. Now that we know more about light we can also argue
that $f$ cannot change because 
\begin{equation*}
E=hf
\end{equation*}%
If $f$ changed, then we would either require an input of energy or we would
store energy at the boundary because%
\begin{equation*}
\Delta f=\frac{\Delta E}{h}
\end{equation*}%
This can't be true. If the wavelength changes, there is no such change in
energy.

Since 
\begin{equation*}
v_{1}=\lambda _{1}f
\end{equation*}%
and 
\begin{equation*}
v_{2}=\lambda _{2}f
\end{equation*}%
then the ratio%
\begin{equation*}
\frac{v_{1}}{v_{2}}=\frac{\lambda _{1}}{\lambda _{2}}
\end{equation*}%
and we again have our solution for the wavelength in the material%
\begin{equation*}
\lambda _{2}=\lambda _{1}\frac{v_{2}}{v_{1}}
\end{equation*}%
which agrees with our previous analysis.

\subsection{Index of refraction and Snell's Law}

%TCIMACRO{%
%\TeXButton{Question 223.14.2}{\marginpar {
%\hspace{-0.5in}
%\begin{minipage}[t]{1in}
%\small{Question 223.14.2}
%\end{minipage}
%}}}%
%BeginExpansion
\marginpar {
\hspace{-0.5in}
\begin{minipage}[t]{1in}
\small{Question 223.14.2}
\end{minipage}
}%
%EndExpansion
%TCIMACRO{%
%\TeXButton{Question 223.14.3}{\marginpar {
%\hspace{-0.5in}
%\begin{minipage}[t]{1in}
%\small{Question 223.14.3}
%\end{minipage}
%}}}%
%BeginExpansion
\marginpar {
\hspace{-0.5in}
\begin{minipage}[t]{1in}
\small{Question 223.14.3}
\end{minipage}
}%
%EndExpansion

Because the equation%
\begin{equation*}
\frac{\sin \left( \theta _{2}\right) }{\sin \left( \theta 1\right) }=\frac{%
v_{2}}{v_{1}}=\text{constant}
\end{equation*}%
has a constant ratio of velocities, it is convenient to define a term that
represents that ratio. We already have a concept that can help. The \emph{%
index of refraction} is just such a term. It assumes that one speed is the
speed of light in vacuum, $c.$%
\begin{equation*}
n\equiv \frac{c}{v}
\end{equation*}%
Then for our example 
\begin{equation*}
\frac{\sin \left( \theta _{2}\right) }{\sin \left( \theta 1\right) }=\frac{%
cv_{2}}{cv_{1}}=\frac{n_{1}}{n_{2}}
\end{equation*}%
or%
\begin{equation*}
\frac{\sin \left( \theta _{2}\right) }{\sin \left( \theta 1\right) }=\frac{1%
}{n_{2}}
\end{equation*}

Suppose we don't have a vacuum (or air that is close to a vacuum). We can
write our formula as 
\begin{equation}
n_{1}\sin \left( \theta _{1}\right) =n_{2}\sin \left( \theta _{2}\right)
\end{equation}%
where we have determined 
\begin{equation*}
n_{1}=\frac{c}{v_{1}}
\end{equation*}%
and 
\begin{equation*}
n_{2}=\frac{c}{v_{2}}
\end{equation*}

This is called \emph{Snell's law of refraction} after the scientist who
experimentally determined the relationship.

Again let's consider our wavelength change. Using the index of refraction we
can write our equation relating the ratio of velocities and wavelengths as%
\begin{equation*}
\frac{v_{1}}{v_{2}}=\frac{\lambda _{1}}{\lambda _{2}}=\frac{\frac{2}{n_{1}}}{%
\frac{c}{n_{2}}}=\frac{n_{2}}{n_{1}}
\end{equation*}%
which gives 
\begin{equation*}
\lambda _{1}n_{1}=\lambda _{2}n_{2}
\end{equation*}%
and if we have vacuum and a single material we can find the index of
refraction from 
\begin{equation}
n=\frac{\lambda }{\lambda _{material}}
\end{equation}%
where $\lambda _{material}$ is the wavelength in the material.%
%TCIMACRO{%
%\TeXButton{Question 223.14.4}{\marginpar {
%\hspace{-0.5in}
%\begin{minipage}[t]{1in}
%\small{Question 223.14.4}
%\end{minipage}
%}}}%
%BeginExpansion
\marginpar {
\hspace{-0.5in}
\begin{minipage}[t]{1in}
\small{Question 223.14.4}
\end{minipage}
}%
%EndExpansion
%TCIMACRO{%
%\TeXButton{Question 223.14.5}{\marginpar {
%\hspace{-0.5in}
%\begin{minipage}[t]{1in}
%\small{Question 223.14.5}
%\end{minipage}
%}}}%
%BeginExpansion
\marginpar {
\hspace{-0.5in}
\begin{minipage}[t]{1in}
\small{Question 223.14.5}
\end{minipage}
}%
%EndExpansion

\section{Total Internal Reflection}

%TCIMACRO{%
%\TeXButton{Question 223.14.6}{\marginpar {
%\hspace{-0.5in}
%\begin{minipage}[t]{1in}
%\small{Question 223.14.6}
%\end{minipage}
%}}}%
%BeginExpansion
\marginpar {
\hspace{-0.5in}
\begin{minipage}[t]{1in}
\small{Question 223.14.6}
\end{minipage}
}%
%EndExpansion
Up to now we have assumed that light was coming from a region of low index
of refraction into a region of high index of refraction. We should pause to
look at what can happen if we go the other way.\FRAME{dhF}{3.0744in}{2.5815in%
}{0pt}{}{}{Figure}{\special{language "Scientific Word";type
"GRAPHIC";maintain-aspect-ratio TRUE;display "USEDEF";valid_file "T";width
3.0744in;height 2.5815in;depth 0pt;original-width 3.0303in;original-height
2.5399in;cropleft "0";croptop "1";cropright "1";cropbottom "0";tempfilename
'LTUWCT50.wmf';tempfile-properties "XPR";}}

We start with Snell's law%
\begin{equation*}
n_{1}\sin \theta _{1}=n_{2}\sin \theta _{2}
\end{equation*}%
but this time $n=n_{1}$ and $n_{2}\approx 1$ so 
\begin{equation*}
n\sin \theta _{1}=\sin \theta _{2}
\end{equation*}%
which gives 
\begin{equation}
\theta _{2}=\sin ^{-1}\left( n\sin \theta _{1}\right)
\end{equation}%
If we take $n=1.33$ (water) we can plot this expression as a function of $%
\theta _{1}$

\FRAME{dtbpF}{3.0312in}{2.0211in}{0pt}{}{}{Plot}{\special{language
"Scientific Word";type "MAPLEPLOT";width 3.0312in;height 2.0211in;depth
0pt;display "USEDEF";plot_snapshots TRUE;mustRecompute FALSE;lastEngine
"MuPAD";xmin "0";xmax "1.2";xviewmin "0";xviewmax "1.2";yviewmin
"0";yviewmax "2";viewset"XY";rangeset"X";plottype 4;labeloverrides 3;x-label
"Theta 2";y-label "Theta 1";axesFont "Times New
Roman,12,0000000000,useDefault,normal";numpoints 100;plotstyle
"patch";axesstyle "normal";axestips FALSE;xis \TEXUX{x};var1name
\TEXUX{$x$};function \TEXUX{$\sin ^{-1}\left( 1.33\sin x\right) $};linecolor
"blue";linestyle 1;pointstyle "point";linethickness 1;lineAttributes
"Solid";var1range "0,1.2";num-x-gridlines 100;curveColor
"[flat::RGB:0x000000ff]";curveStyle "Line";function
\TEXUX{$\MATRIX{2,2}{c}\VR{,,c,,,}{,,c,,,}{,,,,,}\HR{,,}\CELL{0.850\,91}%
\CELL{0}\CELL{0.850\,91}\CELL{2}$};linecolor "black";linestyle 2;pointstyle
"point";linethickness 1;lineAttributes "Dash";curveColor
"[flat::RGB:0000000000]";curveStyle "Line";VCamFile
'LYGTEW2P.xvz';valid_file "T";tempfilename
'LYGTCE03.wmf';tempfile-properties "XPR";}}we see that at $\theta
_{1}=\allowbreak 0.850\,91\unit{rad}$ $(48.\,\allowbreak 754\unit{%
%TCIMACRO{\U{b0}}%
%BeginExpansion
{{}^\circ}%
%EndExpansion
})$ the curve becomes infinitely steep. If we use this value in our equation
this gives%
\begin{eqnarray}
\theta _{2} &=&\sin ^{-1}\left( n\sin \left( \allowbreak 0.850\,91\right)
\right)  \notag \\
&=&1.\,\allowbreak 570\,8\unit{rad} \\
&=&90\unit{%
%TCIMACRO{\U{b0}}%
%BeginExpansion
{{}^\circ}%
%EndExpansion
}  \notag
\end{eqnarray}%
The light skims along the edge of the water!%
%TCIMACRO{%
%\TeXButton{Internal Reflection Demo}{\marginpar {
%\hspace{-0.5in}
%\begin{minipage}[t]{1in}
%\small{Internal Reflection Demo}
%\end{minipage}
%}}}%
%BeginExpansion
\marginpar {
\hspace{-0.5in}
\begin{minipage}[t]{1in}
\small{Internal Reflection Demo}
\end{minipage}
}%
%EndExpansion
\FRAME{dhF}{2.3756in}{1.9251in}{0pt}{}{}{Figure}{\special{language
"Scientific Word";type "GRAPHIC";maintain-aspect-ratio TRUE;display
"USEDEF";valid_file "T";width 2.3756in;height 1.9251in;depth
0pt;original-width 2.3359in;original-height 1.887in;cropleft "0";croptop
"1";cropright "1";cropbottom "0";tempfilename
'LTUWCU52.wmf';tempfile-properties "XPR";}}We can find the value of $\theta
_{1}$ that makes this happen without graphing. Set $\theta _{2}=90\unit{%
%TCIMACRO{\U{b0}}%
%BeginExpansion
{{}^\circ}%
%EndExpansion
}$ then

\begin{equation*}
n_{1}\sin \theta _{1}=n_{2}\sin \theta _{2}
\end{equation*}%
becomes%
\begin{equation*}
n_{1}\sin \theta _{1}=\sin \left( 90\unit{%
%TCIMACRO{\U{b0}}%
%BeginExpansion
{{}^\circ}%
%EndExpansion
}\right) =1
\end{equation*}%
\begin{equation*}
\sin \theta _{1}=\frac{1}{n_{1}}
\end{equation*}%
so then $\theta _{1}$ is given by%
\begin{equation}
\theta _{1}=\theta _{c}\equiv \sin ^{-1}\left( \frac{1}{n}\right)
\end{equation}%
We give this value of $\theta _{1}$ a special name. It is the \emph{critical
angle} for internal reflection. But what happens if we go farther than this (%
$\theta _{1}>\theta _{c}$)? We will no longer have a transmitted ray. The
ray will be reflected. This is why when you dive into a pool and look up,
you see a region of the roof of the pool area (or sky) but off to the side
of the pool the surface looks mirrored. It is also why you sometimes see the
sides of a fish tank appear to be mirrored when you look through the front. 
\FRAME{dhF}{1.7616in}{1.3258in}{0pt}{}{}{Figure}{\special{language
"Scientific Word";type "GRAPHIC";maintain-aspect-ratio TRUE;display
"USEDEF";valid_file "T";width 1.7616in;height 1.3258in;depth
0pt;original-width 1.7236in;original-height 1.2912in;cropleft "0";croptop
"1";cropright "1";cropbottom "0";tempfilename
'LTUWCU53.wmf';tempfile-properties "XPR";}}It is also why cut gems (like
diamonds) sparkle. They capture the light with facets that are cut at angles
that create total internal reflection. The light that enters the gem comes
back out the front (We will study how to make the pretty colored sparkles
next time).%
%TCIMACRO{%
%\TeXButton{Question 223.14.7}{\marginpar {
%\hspace{-0.5in}
%\begin{minipage}[t]{1in}
%\small{Question 223.14.7}
%\end{minipage}
%}}}%
%BeginExpansion
\marginpar {
\hspace{-0.5in}
\begin{minipage}[t]{1in}
\small{Question 223.14.7}
\end{minipage}
}%
%EndExpansion
%TCIMACRO{%
%\TeXButton{Question 223.14.8}{\marginpar {
%\hspace{-0.5in}
%\begin{minipage}[t]{1in}
%\small{Question 223.14.8}
%\end{minipage}
%}}}%
%BeginExpansion
\marginpar {
\hspace{-0.5in}
\begin{minipage}[t]{1in}
\small{Question 223.14.8}
\end{minipage}
}%
%EndExpansion
%TCIMACRO{%
%\TeXButton{Question 223.14.9}{\marginpar {
%\hspace{-0.5in}
%\begin{minipage}[t]{1in}
%\small{Question 223.14.9}
%\end{minipage}
%}}}%
%BeginExpansion
\marginpar {
\hspace{-0.5in}
\begin{minipage}[t]{1in}
\small{Question 223.14.9}
\end{minipage}
}%
%EndExpansion

\subsection{Fiber Optics}

Beyond pretty pebbles, this effect is very useful! It is the heart and soul
of fiber optics. \FRAME{dhF}{3.6763in}{2.0652in}{0in}{}{}{Figure}{\special%
{language "Scientific Word";type "GRAPHIC";maintain-aspect-ratio
TRUE;display "USEDEF";valid_file "T";width 3.6763in;height 2.0652in;depth
0in;original-width 3.6288in;original-height 2.0263in;cropleft "0";croptop
"1";cropright "1";cropbottom "0";tempfilename
'LYFJ3S03.wmf';tempfile-properties "XPR";}}

An interior material with a lower index of refraction is inclosed in a
cladding with a higher index. This creates a light pipe that traps the light
in the fiber. \FRAME{dhF}{3.4662in}{1.535in}{0pt}{}{}{Figure}{\special%
{language "Scientific Word";type "GRAPHIC";maintain-aspect-ratio
TRUE;display "USEDEF";valid_file "T";width 3.4662in;height 1.535in;depth
0pt;original-width 3.4203in;original-height 1.4987in;cropleft "0";croptop
"1";cropright "1";cropbottom "0";tempfilename
'LTUWCU55.wmf';tempfile-properties "XPR";}}

%TCIMACRO{%
%\TeXButton{Giant Fiber Demo}{\marginpar {
%\hspace{-0.5in}
%\begin{minipage}[t]{1in}
%\small{Giant Fiber Demo}
%\end{minipage}
%}}}%
%BeginExpansion
\marginpar {
\hspace{-0.5in}
\begin{minipage}[t]{1in}
\small{Giant Fiber Demo}
\end{minipage}
}%
%EndExpansion
Modern fibers don't always have a hard boundary. The fibers have a gradual
change in index of refraction that changes the direction of the light
gradually. This keeps the light in the fiber but tends to direct along the
fiber so the beam is not crisscrossing as it goes.

The cutting edge of fiber design today uses hollow fibers or fibers filled
with different index material.

\section{Images Formed by Refraction}

Let's think about what an image is.%
%TCIMACRO{%
%\TeXButton{Make Images with Lens Demo}{\marginpar {
%\hspace{-0.5in}
%\begin{minipage}[t]{1in}
%\small{Make Images with Lens Demo}
%\end{minipage}
%}} }%
%BeginExpansion
\marginpar {
\hspace{-0.5in}
\begin{minipage}[t]{1in}
\small{Make Images with Lens Demo}
\end{minipage}
}
%EndExpansion
Take a piece of paper and a lens, and hold up the lens is a darkened room
that has some bright object in it. Move the lens or the paper back and
forth, and at just the right distance, a miniature picture of the bright
object will appear. We should think about what the word \textquotedblleft
picture\textquotedblright\ means in this sense.

\FRAME{dhF}{2.6835in}{2.0237in}{0pt}{}{}{Figure}{\special{language
"Scientific Word";type "GRAPHIC";maintain-aspect-ratio TRUE;display
"USEDEF";valid_file "T";width 2.6835in;height 2.0237in;depth
0pt;original-width 2.6411in;original-height 1.9847in;cropleft "0";croptop
"1";cropright "1";cropbottom "0";tempfilename
'LTUWCU59.wmf';tempfile-properties "XPR";}}We have talked about how we see
something. Remember the BYU-I guys from last time.\FRAME{dhF}{3.173in}{%
1.3534in}{0pt}{}{}{Figure}{\special{language "Scientific Word";type
"GRAPHIC";maintain-aspect-ratio TRUE;display "USEDEF";valid_file "T";width
3.173in;height 1.3534in;depth 0pt;original-width 3.128in;original-height
1.3188in;cropleft "0";croptop "1";cropright "1";cropbottom "0";tempfilename
'LTUWCU5A.wmf';tempfile-properties "XPR";}}Our eyes gather rays that are
diverging from the object because light has bounced off of the object. Our
eyes intersect a diverging set of rays that form a definite pattern. That
diverging set of rays forming a pattern is the picture of the object.

So when we say that the lens has formed a miniature picture of our dark room
object, we mean that the lens has somehow formed a diverging set of rays
that form a pattern that looks like the pattern formed by the diverging set
of rays coming from the object, itself. In other words, the object forms a
diverging set of rays, as normal, and our lens forms a duplicate set of rays
in the same pattern, so we see the same thing. The lens' version is smaller,
and upside down, but it is still essentially the same pattern.

As a first step to see how this works, consider our fish tank again. It
would be bad on the fish, but think about looking at a fish in air. The room
light would bounce off of the fish, and we would have a diverging set of
rays from every point on the fish. \FRAME{dhF}{3.8121in}{1.9078in}{0pt}{}{}{%
Figure}{\special{language "Scientific Word";type
"GRAPHIC";maintain-aspect-ratio TRUE;display "USEDEF";valid_file "T";width
3.8121in;height 1.9078in;depth 0pt;original-width 5.4916in;original-height
2.7337in;cropleft "0";croptop "1";cropright "1";cropbottom "0";tempfilename
'LVHU6803.wmf';tempfile-properties "XPR";}}We can see that the picture is
made from every point on the fish being \textquotedblleft
imaged\textquotedblright\ to a point on the retina. We collect the rays
leaving every point on the fish, and bring them to corresponding points on
the retina to make the picture.

It will take us a few lectures to see exactly how this is done by the lens
system in the eye, but as a first step, let's consider the fish tank,
itself. Put the fish back in the tank and look at it.

\FRAME{dhF}{3.5924in}{2.3021in}{0pt}{}{}{Figure}{\special{language
"Scientific Word";type "GRAPHIC";maintain-aspect-ratio TRUE;display
"USEDEF";valid_file "T";width 3.5924in;height 2.3021in;depth
0pt;original-width 3.5449in;original-height 2.2623in;cropleft "0";croptop
"1";cropright "1";cropbottom "0";tempfilename
'LYGTN208.wmf';tempfile-properties "XPR";}}Rays still come from the fish.
But we now know that the change from a slow material to a fast material will
bend the light. These bent rays are collected by our eyes, and the picture
of the fish is formed on the retina just as before. But our eyes interpret
the light as though it went in straight lines with no bends (dotted lines in
the last figure). our mind is designed to believe light travels in straight
lines, so our mind tells us there is a fish, but that the fish head (and
every other part of the fish) is closer than it really is. We call this
apparent fish at the closer location an image of the fish, because this is
where we think the diverging set of rays come from that form the fish
pattern.

The next figure shows the details of the rays leaving a dot on the fish head 
\FRAME{dhF}{3.5639in}{3.3486in}{0in}{}{}{Figure}{\special{language
"Scientific Word";type "GRAPHIC";maintain-aspect-ratio TRUE;display
"USEDEF";valid_file "T";width 3.5639in;height 3.3486in;depth
0in;original-width 3.5172in;original-height 3.3027in;cropleft "0";croptop
"1";cropright "1";cropbottom "0";tempfilename
'LTUWCV5D.wmf';tempfile-properties "XPR";}}The dot on the fish head is our
object for this set of rays. The distance from the fish-head dot and the
edge of the water/air boundary is called the \emph{object distance} and is
given the symbol $s.$

The distance from the image of the fish-head dot to the edge of the
water/air boundary is called the \emph{image distance} and is given the
symbol $s^{\prime }.$ Note that this is not a derivative, it is just a
distance like $s,$ because it appears to be where the rays come from, but it
is a different distance because of the refraction of the rays. So to make it
look different we put a prime mark on it.

%TCIMACRO{%
%\TeXButton{Do this math, you will circle back to this}{\marginpar {
%\hspace{-0.5in}
%\begin{minipage}[t]{1in}
%\small{Do this math, you will circle back to this}
%\end{minipage}
%}}}%
%BeginExpansion
\marginpar {
\hspace{-0.5in}
\begin{minipage}[t]{1in}
\small{Do this math, you will circle back to this}
\end{minipage}
}%
%EndExpansion
We can find where the image is $\left( s^{\prime }\right) $ knowing $s.$ We
can see from the figure that%
\begin{eqnarray*}
\ell &=&s\tan \theta _{1} \\
&=&s^{\prime }\tan \theta _{2}
\end{eqnarray*}%
so%
\begin{equation*}
s\tan \theta _{1}=s^{\prime }\tan \theta _{2}
\end{equation*}%
or%
\begin{equation*}
s\frac{\tan \theta _{1}}{\tan \theta _{2}}=s^{\prime }
\end{equation*}%
from Snell's law, we know that 
\begin{equation*}
\frac{\sin \theta _{1}}{\sin \theta _{2}}=\frac{n_{2}}{n_{1}}
\end{equation*}

Usually we can take the small angle approximation. This would limit our
analysis to rays that are near the central axes. Let's call this central
axis the \emph{optics axis} and the rays that are not too far away from this
axis \emph{paraxial rays.} Then for our small angles we can write 
\begin{equation*}
\tan \theta _{i}\approx \sin \theta _{i}
\end{equation*}%
so%
\begin{equation*}
s\frac{\tan \theta _{1}}{\tan \theta _{2}}\approx s\frac{\sin \theta _{1}}{%
\sin \theta _{2}}=s\frac{n_{2}}{n_{1}}=s^{\prime }
\end{equation*}%
and we have the image distance

\begin{equation*}
s^{\prime }=s\frac{n_{2}}{n_{1}}
\end{equation*}

This is not so useful unless you have some burning need to know where your
fish are in a tank. But we now have the vocabulary to discuss the larger
problem of how a lens works, which we will take up next time.

\chapter{Dispersion and Thin Lenses}

%TCIMACRO{%
%\TeXButton{Fundamental Concepts}{\hspace{-1.3in}{\LARGE Fundamental Concepts\vspace{0.25in}}}}%
%BeginExpansion
\hspace{-1.3in}{\LARGE Fundamental Concepts\vspace{0.25in}}%
%EndExpansion

\begin{itemize}
\item Index of refraction is wavelength dependent

\item We can describe the operation of thin lenses using three easy-to-draw
rays.
\end{itemize}

\section{Dispersion}

\FRAME{dhF}{4.3613in}{2.2745in}{0pt}{}{}{Figure}{\special{language
"Scientific Word";type "GRAPHIC";maintain-aspect-ratio TRUE;display
"USEDEF";valid_file "T";width 4.3613in;height 2.2745in;depth
0pt;original-width 4.3102in;original-height 2.2347in;cropleft "0";croptop
"1";cropright "1";cropbottom "0";tempfilename
'LTUWCV5E.wmf';tempfile-properties "XPR";}}%
%TCIMACRO{%
%\TeXButton{Question 223.15.1}{\marginpar {
%\hspace{-0.5in}
%\begin{minipage}[t]{1in}
%\small{Question 223.15.1}
%\end{minipage}
%}}}%
%BeginExpansion
\marginpar {
\hspace{-0.5in}
\begin{minipage}[t]{1in}
\small{Question 223.15.1}
\end{minipage}
}%
%EndExpansion

Who hasn't played with a prism? We immediately recognize a rainbow. But why
does the prism make a rainbow? The secret lies in the nature of the
refractive index.

Notice that in the figure, the index of refraction depends on wavelength.
This means that as light enters a material, different wavelengths will be
refracted at different angles.

White light is not really a color of light. White light is made up of many
colors--in fact, all the colors of the rainbow!\footnote{%
Ah, but some light sources fool us. As long as there are the right amounts
of red, green, and blue, we can think the light is white. Fluorescent lights
and LED lights do this, and the lack of a full spectrum of light explains
why plants don't grow well under fluorescent lights and some LED lights.}
Thus white light is pulled apart by refraction into a rainbow. This process
is called \emph{dispersion}. The reason is that for different wavelengths of
light it is more likely to for the light waves to be absorbed and re-emitted
than for other wavelengths. This has to do with the spacing of the atoms
relative to the wavelength, and it has to do with the electron structure of
the material. Here is a graph that shows the index of refraction for some
materials as a function of wavelength. \FRAME{fhFU}{4.3535in}{2.6195in}{0pt}{%
\Qcb{{\protect\small Index of refraction as a function of wavelength ( Ohara
optical glass http://www.oharacorp.com/fused-silica-quartz.html data and
Schott optical glass data
http://www.uqgoptics.com/materials\_glasses\_schott.aspx)}}}{\Qlb{Dispersion
of Glass}}{Figure}{\special{language "Scientific Word";type
"GRAPHIC";maintain-aspect-ratio TRUE;display "USEDEF";valid_file "T";width
4.3535in;height 2.6195in;depth 0pt;original-width 5.1436in;original-height
3.084in;cropleft "0";croptop "1";cropright "1";cropbottom "0";tempfilename
'MRWWVA00.wmf';tempfile-properties "XPR";}}

The graph tells us that blue light bends more than red light for these
materials. \FRAME{dhF}{2.4379in}{1.2816in}{0pt}{}{}{Figure}{\special%
{language "Scientific Word";type "GRAPHIC";maintain-aspect-ratio
TRUE;display "USEDEF";valid_file "T";width 2.4379in;height 1.2816in;depth
0pt;original-width 6.5207in;original-height 3.4143in;cropleft "0";croptop
"1";cropright "1";cropbottom "0";tempfilename
'LTUWCV5G.wmf';tempfile-properties "XPR";}}

We call the change in direction measured from the original direction of
travel, $\delta ,$ the \emph{angle of deviation}. The colors we can see are
called the visible spectrum.

%TCIMACRO{%
%\TeXButton{Question 223.15.2}{\marginpar {
%\hspace{-0.5in}
%\begin{minipage}[t]{1in}
%\small{Question 223.15.2}
%\end{minipage}
%}}}%
%BeginExpansion
\marginpar {
\hspace{-0.5in}
\begin{minipage}[t]{1in}
\small{Question 223.15.2}
\end{minipage}
}%
%EndExpansion

Let's look at a natural rainbow. The dispersion is caused buy small droplets
of water. The white sunlight enters the drop and is dispersed. It bounces
off the back of the drop and then leaves the drop, again being dispersed.
Red light leaves the drop at about $42\unit{%
%TCIMACRO{\U{b0}}%
%BeginExpansion
{{}^\circ}%
%EndExpansion
}$ from its input direction, and blue light leaves at about $40\unit{%
%TCIMACRO{\U{b0}}%
%BeginExpansion
{{}^\circ}%
%EndExpansion
}.$\FRAME{dhF}{3.1964in}{2.3039in}{0pt}{}{}{Figure}{\special{language
"Scientific Word";type "GRAPHIC";maintain-aspect-ratio TRUE;display
"USEDEF";valid_file "T";width 3.1964in;height 2.3039in;depth
0pt;original-width 9.1488in;original-height 6.5778in;cropleft "0";croptop
"1";cropright "1";cropbottom "0";tempfilename
'LTUWCW5H.wmf';tempfile-properties "XPR";}}

\subsection{Calculation of n using a prism}

Let's do a problem using the idea of dispersion. Let's find the index of
refraction of a the material. Suppose we make a prism as shown\footnote{%
In this problem, we have carefully arranged the light so it goes
horizontally across the prism. This is not always the case--and it is not
usually the case in the problems in the homework.}. We know the apex angle
of the triangle, $\Phi ,$ and can measure the exit angle $\delta .$ In terms
of these two variables, what is $n?$

\FRAME{dtbpF}{3.8112in}{2.4033in}{0in}{}{}{Figure}{\special{language
"Scientific Word";type "GRAPHIC";maintain-aspect-ratio TRUE;display
"USEDEF";valid_file "T";width 3.8112in;height 2.4033in;depth
0in;original-width 3.7628in;original-height 2.3635in;cropleft "0";croptop
"1";cropright "1";cropbottom "0";tempfilename
'MHHY1J01.wmf';tempfile-properties "XPR";}}

Our strategy should be to use Snell's law%
\begin{equation*}
n_{1}\sin \theta _{1}=n_{2}\sin \theta _{2}
\end{equation*}%
If we can find $\theta _{1}$ and $\theta _{2}$ in terms of $\Phi $ and $%
\delta $ then we can solve for the index of refraction of the material (we
know $n_{1}\approx 1).$ Using the notation indicated in the figure, we
choose $\theta _{1}$ such that the interior ray is horizontal.\footnote{%
WARNING! in the homework problems you can't make the same assumptions!}

\begin{equation*}
\theta _{1}=\theta _{2}+\alpha
\end{equation*}

\begin{equation*}
\delta =180-\beta
\end{equation*}

\begin{equation*}
180=\beta +2\alpha
\end{equation*}

Then%
\begin{equation*}
\delta =\beta +2\alpha -\beta
\end{equation*}%
or

\begin{equation*}
\delta =2\alpha
\end{equation*}%
and%
\begin{equation*}
\alpha =\frac{\delta }{2}
\end{equation*}

\begin{equation*}
90=\alpha +\theta _{2}+\phi
\end{equation*}

and

\begin{equation*}
180=\Phi +2\alpha +2\phi
\end{equation*}%
or%
\begin{equation*}
90=\frac{\Phi }{2}+\alpha +\phi
\end{equation*}%
Then%
\begin{eqnarray*}
\alpha +\theta _{2}+\phi &=&\frac{\Phi }{2}+\alpha +\phi \\
\theta _{2} &=&\frac{\Phi }{2}
\end{eqnarray*}%
We can put these in our equation for $\theta _{1}$%
\begin{eqnarray*}
\theta _{1} &=&\theta _{2}+\alpha \\
&=&\frac{\Phi }{2}+\frac{\delta }{2} \\
&=&\frac{\Phi +\delta }{2}
\end{eqnarray*}%
We now know $\theta _{1}$ and $\theta _{2}.$ Now we can use Snell's Law to
find $n$%
\begin{eqnarray*}
\sin \left( \theta _{1}\right) &=&n\sin \left( \theta _{2}\right) \\
\sin \left( \frac{\Phi +\delta }{2}\right) &=&n\sin \left( \frac{\Phi }{2}%
\right)
\end{eqnarray*}%
then%
\begin{equation}
n=\frac{\sin \left( \frac{\Phi +\delta }{2}\right) }{\sin \left( \frac{\Phi 
}{2}\right) }
\end{equation}%
This gives a value for the index of refraction, but it would be better to
repeat the analysis for several wavelengths. The resulting values for $n$
can be combined into a $n$ vs. $\lambda $ curve like the one shown in figure %
\ref{Dispersion of Glass}.

\subsection{Filters and other color phenomena}

We have assumed without proof, that white light is made up of all the colors
of the rainbow. Diffraction gratings were pretty good hints that this is
true. Now that we know how prisms work, we have additional evidence of this.
But knowing that white light is made a superposition of waves of different
wavelengths, we should ask why a red shirt is red, or why passing light
through a green film makes the light look green as it leaves.

Both of these phenomena are examples of removing wavelengths from white
light.

In the case of the red shirt, the red dye in the cloth absorbs all of the
visible colors except red. The red is reflected, so the shirt looks red. The
filter is much the same. The green pigment in the film causes nearly all
visible colors to be absorbed except green. So only green light is
transmitted. This is why leaves are green. Chlorophyll absorbs red and blue
wavelengths, so the green is reflected or transmitted.

\FRAME{dhFU}{2.4738in}{2.19in}{0pt}{\Qcb{chlorophyll spectrum (Public Domain
image courtesy Kurzon)}}{}{Figure}{\special{language "Scientific Word";type
"GRAPHIC";maintain-aspect-ratio TRUE;display "USEDEF";valid_file "T";width
2.4738in;height 2.19in;depth 0pt;original-width 2.4336in;original-height
2.1508in;cropleft "0";croptop "1";cropright "1";cropbottom "0";tempfilename
'LTUWCX5J.wmf';tempfile-properties "XPR";}}Knowing the nature of white
light, we can start to understand lens systems and their challenges.

\section{Ray Diagrams for Lenses}

Before we do a lot of math to describe how lenses work, lets think about our
early childhood experiences. You may have burned things with a magnifying
glass. Using the idea of a ray diagram, here is what happens.\FRAME{dhF}{%
3.8017in}{2.2329in}{0pt}{}{}{Figure}{\special{language "Scientific
Word";type "GRAPHIC";maintain-aspect-ratio TRUE;display "USEDEF";valid_file
"T";width 3.8017in;height 2.2329in;depth 0pt;original-width
3.7533in;original-height 2.1923in;cropleft "0";croptop "1";cropright
"1";cropbottom "0";tempfilename 'LTUWCX5K.wmf';tempfile-properties "XPR";}}%
The rays from the Sun come from so far away that they are essentially
parallel. We know that these rays come together to a fine point that can
start a fire. The point where these rays converge is important to us. We
will call this the \emph{focal point}.

Knowing that the light will follow the same paths either direction, we would
expect that if we put a light source at the focal distance, the rays should
come out parallel. \FRAME{dhF}{4.1926in}{2.4561in}{0pt}{}{}{Figure}{\special%
{language "Scientific Word";type "GRAPHIC";maintain-aspect-ratio
TRUE;display "USEDEF";valid_file "T";width 4.1926in;height 2.4561in;depth
0pt;original-width 4.1433in;original-height 2.4146in;cropleft "0";croptop
"1";cropright "1";cropbottom "0";tempfilename
'LTUWCX5L.wmf';tempfile-properties "XPR";}}We need one other bit of
information, to understand lenses. We have seen this case before.\FRAME{dhF}{%
1.996in}{1.6933in}{0pt}{}{}{Figure}{\special{language "Scientific Word";type
"GRAPHIC";maintain-aspect-ratio TRUE;display "USEDEF";valid_file "T";width
1.996in;height 1.6933in;depth 0pt;original-width 3.6841in;original-height
3.122in;cropleft "0";croptop "1";cropright "1";cropbottom "0";tempfilename
'LTUWCX5M.wmf';tempfile-properties "XPR";}}A flat block does refract the
light, but when the light leaves the block it is only displaced, it retains
the original direction. We will use these three situations to describe what
happens when light travels through a lens.

We know that for every point on the object, we get millions of reflected
rays that diverge. The lens must collect these rays together to form the
corresponding point on the image.

\FRAME{dhF}{4.1373in}{2.4275in}{0pt}{}{}{Figure}{\special{language
"Scientific Word";type "GRAPHIC";maintain-aspect-ratio TRUE;display
"USEDEF";valid_file "T";width 4.1373in;height 2.4275in;depth
0pt;original-width 4.088in;original-height 2.3869in;cropleft "0";croptop
"1";cropright "1";cropbottom "0";tempfilename
'LTUWCX5N.wmf';tempfile-properties "XPR";}}In the figure, the object is an
upright arrow. We suppose that the arrow either glows, or that light is
reflecting off the arrow. The arrow is a diffuse reflector, so the light
bounces off in all directions. In the figure, you see light bouncing off the
tip. Of course, this happens for every point on the image. Here is another
drawing with light bouncing off the middle of the arrow. \FRAME{dhF}{4.1926in%
}{2.4561in}{0pt}{}{}{Figure}{\special{language "Scientific Word";type
"GRAPHIC";maintain-aspect-ratio TRUE;display "USEDEF";valid_file "T";width
4.1926in;height 2.4561in;depth 0pt;original-width 4.1433in;original-height
2.4146in;cropleft "0";croptop "1";cropright "1";cropbottom "0";tempfilename
'LTUWCX5O.wmf';tempfile-properties "XPR";}}But we usually pick the top of
the object. If we place the bottom of the object on the optic axis, the
bottom of the image will also be on the optic axis. So knowing where the
bottom of the image is, and finding the top of the image gives a pretty good
idea of where the rest of the image must be. So we will draw diagrams for
the top of the object to find the top of the image.

But suppose this is not true? For example, when we use a camera, we do not
align the optical system so the bottom of the subject is in the middle of
the lens, on an axis, before we shoot.\FRAME{dhF}{3.2292in}{1.8421in}{0pt}{}{%
}{Figure}{\special{language "Scientific Word";type
"GRAPHIC";maintain-aspect-ratio TRUE;display "USEDEF";valid_file "T";width
3.2292in;height 1.8421in;depth 0pt;original-width 3.1842in;original-height
1.8049in;cropleft "0";croptop "1";cropright "1";cropbottom "0";tempfilename
'LTUWCX5P.wmf';tempfile-properties "XPR";}}We can, of course, trace some
rays for the bottom of the images well as for the top in this case and find
the location of the bottom of the image. The middle of the image will still
be in between the top and the bottom.

Notice that we said that light bounced off the arrow in all directions, but
we did not draw all the rays going in all directions. Drawing millions of
rays is impractical, and fortunately, not needed. We instinctively only drew
rays that headed toward the lens. Any ray that does not head toward the lens
won't take part in forming the image created by the lens. But could we make
due with even less rays?

It turns out that we can choose three simple rays that leave the top of the
object , and see where these rays converge to form the top of the image.
Let's start with a ray that travels from the top of the object and travels
parallel to the optic axis. We recognize this ray as being like one of the
rays from the Sun. It comes in parallel, so it will leave the lens and
travel through the focal point.\FRAME{dhF}{4.0395in}{1.7158in}{0pt}{}{}{%
Figure}{\special{language "Scientific Word";type
"GRAPHIC";maintain-aspect-ratio TRUE;display "USEDEF";valid_file "T";width
4.0395in;height 1.7158in;depth 0pt;original-width 3.9902in;original-height
1.6786in;cropleft "0";croptop "1";cropright "1";cropbottom "0";tempfilename
'LYMVUI0T.wmf';tempfile-properties "XPR";}}For a second easy ray, lets take
the case that is like our flat block. Near the center of the lens, the sides
are nearly flat. So we expect that the ray will leave in about the same
direction as it was going before it struck the lens.\FRAME{dhF}{4.4451in}{%
2.623in}{0pt}{}{}{Figure}{\special{language "Scientific Word";type
"GRAPHIC";maintain-aspect-ratio TRUE;display "USEDEF";valid_file "T";width
4.4451in;height 2.623in;depth 0pt;original-width 4.3932in;original-height
2.5815in;cropleft "0";croptop "1";cropright "1";cropbottom "0";tempfilename
'LYMVV20U.wmf';tempfile-properties "XPR";}}

Two rays are really enough to determine where the top of the image will be,
but there is a third ray that is easy to draw, so let's draw it to give us
more confidence in our answer. That ray is one that leaves the top of the
object and passes through the focal point on the object side of the lens.
This situation we also recognize. This ray will leave the lens parallel to
the optic axis.

\FRAME{dhF}{4.8222in}{2.8323in}{0pt}{}{}{Figure}{\special{language
"Scientific Word";type "GRAPHIC";maintain-aspect-ratio TRUE;display
"USEDEF";valid_file "T";width 4.8222in;height 2.8323in;depth
0pt;original-width 4.7686in;original-height 2.789in;cropleft "0";croptop
"1";cropright "1";cropbottom "0";tempfilename
'LYMVVM0V.wmf';tempfile-properties "XPR";}}Where all three rays interest, we
will have the top of the image. \FRAME{dhF}{4.7098in}{2.776in}{0pt}{}{}{%
Figure}{\special{language "Scientific Word";type
"GRAPHIC";maintain-aspect-ratio TRUE;display "USEDEF";valid_file "T";width
4.7098in;height 2.776in;depth 0pt;original-width 4.657in;original-height
2.7337in;cropleft "0";croptop "1";cropright "1";cropbottom "0";tempfilename
'LYMVXP0W.wmf';tempfile-properties "XPR";}}Because the rays come together or 
\emph{converge} we call a lens like this a \emph{converging lens.} Notice
that in this case, the image is upside down. That is normal. Also notice
that it is smaller than the object. We say that the image is magnified,
which may seem a little bit strange. But in optics, a magnification of
greater than one means that the image is bigger than the object. This is
like a movie projector that makes a large image of a small film segment. The
magnification can be equal to one, meaning the object and image are the same
size. And finally, the magnification can be less than one. This means that
the image is smaller than the object. This is a convenient definition,
because then we can use the same equation to describe all three situations.%
\begin{equation*}
m\equiv \frac{\text{Image height}}{\text{Object height}}=\frac{h^{\prime }}{h%
}
\end{equation*}%
where $h$ is the object height, and $h^{\prime }$ is the image height. It
turns out that we can also write the magnification of a lens in terms of the
object distance, $s$ and image distance $s^{\prime }$ (see more on this
below).%
\begin{equation*}
m=-\frac{s^{\prime }}{s}
\end{equation*}%
Notice the negative sign. By convention (meaning physicists got together and
voted on this) we say that an upside down image has a negative
magnification. You just have to memorize this, there is no obvious reason
for this except it is mathematically convenient.

%TCIMACRO{%
%\TeXButton{Question 223.15.4}{\marginpar {
%\hspace{-0.5in}
%\begin{minipage}[t]{1in}
%\small{Question 223.15.4}
%\end{minipage}
%}}}%
%BeginExpansion
\marginpar {
\hspace{-0.5in}
\begin{minipage}[t]{1in}
\small{Question 223.15.4}
\end{minipage}
}%
%EndExpansion

\subsection{Thin Lenses}

It is time to introduce another approximation Suppose the lens is very thin.
Then ray number $2$ would travel through the lens with no deviation at all.
This is sometimes a good approximation, and will make the math easier, so
for this class we will often use it. But there are times when it really does
not work, so in practice you have to be careful. PH375 goes beyond the thin
lens formulation.%
%TCIMACRO{%
%\TeXButton{Question 223.15.3}{\marginpar {
%\hspace{-0.5in}
%\begin{minipage}[t]{1in}
%\small{Question 223.15.3}
%\end{minipage}
%}}}%
%BeginExpansion
\marginpar {
\hspace{-0.5in}
\begin{minipage}[t]{1in}
\small{Question 223.15.3}
\end{minipage}
}%
%EndExpansion

We should pause to realize that our new understanding of Snell's law tells
us that we have a problem with our lenses. The index of refraction is
wavelength dependent. This means that different wavelengths will focus in
different positions. Here is our light from the Sun again, but note that I
drew blue light and red light only.

\FRAME{dhF}{2.7121in}{1.8005in}{0pt}{}{}{Figure}{\special{language
"Scientific Word";type "GRAPHIC";maintain-aspect-ratio TRUE;display
"USEDEF";valid_file "T";width 2.7121in;height 1.8005in;depth
0pt;original-width 2.6697in;original-height 1.7634in;cropleft "0";croptop
"1";cropright "1";cropbottom "0";tempfilename
'LYMTCR0P.wmf';tempfile-properties "XPR";}}Having removed all the other
colors, we can see that the blue light focuses nearer to the lens than the
red light. This is because the index of refraction for the blue light is
larger. Each visible wavelength will focus somewhere in between these two
(except for purple, of course). When we make an image, this means that we
get multiple images of our object, one in each color. Usually the images
overlap, so we end up with a colored blur.\FRAME{dhF}{3.2984in}{2.2191in}{0pt%
}{}{}{Figure}{\special{language "Scientific Word";type
"GRAPHIC";maintain-aspect-ratio TRUE;display "USEDEF";valid_file "T";width
3.2984in;height 2.2191in;depth 0pt;original-width 3.2534in;original-height
2.1785in;cropleft "0";croptop "1";cropright "1";cropbottom "0";tempfilename
'LYMTEP0Q.wmf';tempfile-properties "XPR";}}This problem is called Chromatic
Aberration. We can fix this by using a combination of lenses.\FRAME{dhF}{%
0.4886in}{0.9349in}{0pt}{}{}{Figure}{\special{language "Scientific
Word";type "GRAPHIC";maintain-aspect-ratio TRUE;display "USEDEF";valid_file
"T";width 0.4886in;height 0.9349in;depth 0pt;original-width
0.4583in;original-height 0.902in;cropleft "0";croptop "1";cropright
"1";cropbottom "0";tempfilename 'LYMVLW0S.wmf';tempfile-properties "XPR";}}%
where each lens has a different index of refraction. The converging lens is
designed to form the image, while the diverging lens (a term we explain
below) realigns all the colors.

\subsection{Virtual images}

Lets take another case and draw a ray diagram. This time let's place the
object closer than the focal distance. This is the case when we use a lens
as a magnifying glass. The rays will look like this.\FRAME{dhF}{3.6201in}{%
2.3713in}{0pt}{}{}{Figure}{\special{language "Scientific Word";type
"GRAPHIC";maintain-aspect-ratio TRUE;display "USEDEF";valid_file "T";width
3.6201in;height 2.3713in;depth 0pt;original-width 3.5725in;original-height
2.3315in;cropleft "0";croptop "1";cropright "1";cropbottom "0";tempfilename
'LTUWCY5U.wmf';tempfile-properties "XPR";}}Notice that these rays never
converge! We won't get an image that could project on a paper. But we know
that there is an image, we can look through the lens and see it! And that is
the key. The image does not really exist. We look through the lens, and our
mind interprets the diverging rays coming from the lens as though they had
only traveled in straight lines. If we extend these rays backwards along
straight lines, they appear to come from a common point. This is the point
they would have had to have come from if there were no lens. Because our
brain believes light travels in straight lines, we believe we see an image
at this location. But no light really goes there! Because this image is not
really made from light diverging from this position, we recognize this as a 
\emph{virtual image}. The image we formed before that could be projected on
a screen is called a \emph{real image}.

By convention, we say the distance, $s^{\prime },$ from the lens to the
virtual image has a negative value.

\subsection{Diverging Lenses}

So far our lenses have only been the sort that work as magnifying glasses.
We call these \emph{converging lenses.} These lenses are fatter in the
middle and thinner on the edges. Because of this they are sometimes called 
\emph{convex lenses}. By convention, we say the focal distance for this type
of lens is positive. For this reason, they are often called \emph{positive
lenses.}

But what if we make a lens that is thinner in the middle and thicker on the
edges. We can call this sort of lens a \emph{concave lens}, and we will give
in a negative focal length by convention, so we can also call it a \emph{%
negative lens}. But what would this lens do? If we think about our three
rays and Snell's law, ray $1$ won't be bent toward the optic axis for this
type of lens. In fact, if we observe an object through this lens, ray number
1 will appear to come from the focal point. Ray number 2 will still go
through the middle of the lens, and if the lens is thin enough, ray 2 will
pass through undeviated.

\FRAME{dhF}{3.2984in}{1.7582in}{0pt}{}{}{Figure}{\special{language
"Scientific Word";type "GRAPHIC";maintain-aspect-ratio TRUE;display
"USEDEF";valid_file "T";width 3.2984in;height 1.7582in;depth
0pt;original-width 3.2534in;original-height 1.7201in;cropleft "0";croptop
"1";cropright "1";cropbottom "0";tempfilename
'LTUWCZ5V.wmf';tempfile-properties "XPR";}}finally ray three will go as if
it were aiming for the far focal point, but it will hit the lens and leave
parallel to the optic axis. From the figure we see that these three rays
will never converge. We expect they will form a virtual image. If we extend
the rays backward as shown, we see that the extensions all meet at a point.
The rays leaving the lens appear to come from this point. This is the
location of the virtual image.

You might wonder what good such a lens could do, but we will find that this
type of lens is used to correct vision for nearsighted people.

\chapter{Image Formation}

Last lecture we learned how to find an image location graphically, now let's
do it algebraically.

%TCIMACRO{%
%\TeXButton{Fundamental Concepts}{\hspace{-1.3in}{\LARGE Fundamental Concepts\vspace{0.25in}}}}%
%BeginExpansion
\hspace{-1.3in}{\LARGE Fundamental Concepts\vspace{0.25in}}%
%EndExpansion

\begin{itemize}
\item A curved interface between two media can cause light rays to cross

\item The lens-maker's formula is given by $\frac{1}{f}=\left( n-1\right)
\left( \frac{1}{R_{1}}-\frac{1}{R_{2}}\right) $

\item The thin lens formula is given by $\frac{1}{s}+\frac{1}{s^{\prime }}=%
\frac{1}{f}$

\item The sign of quantities that go into the lens-maker's equation and the
thin lens formula are determined by a sign convention.
\end{itemize}

\section{Thin lenses and image equation}

%TCIMACRO{%
%\TeXButton{Question 223.16.1}{\marginpar {
%\hspace{-0.5in}
%\begin{minipage}[t]{1in}
%\small{Question 223.165.1}
%\end{minipage}
%}}}%
%BeginExpansion
\marginpar {
\hspace{-0.5in}
\begin{minipage}[t]{1in}
\small{Question 223.165.1}
\end{minipage}
}%
%EndExpansion
In this lecture, we will work toward understanding the equations that allow
us to solve for the image location given the object location for a thin
lens. Let's start by thinking of a special case for refraction. A circular
or spherically curved surface on a very large piece of glass. We will assume
that the piece of glass is semi-infinite, but all it has to be is very large.

We can call this a semi-infinite bump of glass.\FRAME{fhF}{3.2491in}{1.1865in%
}{0pt}{}{\Qlb{Semi-infinite bump of glass}}{Figure}{\special{language
"Scientific Word";type "GRAPHIC";maintain-aspect-ratio TRUE;display
"USEDEF";valid_file "T";width 3.2491in;height 1.1865in;depth
0pt;original-width 5.2555in;original-height 1.9009in;cropleft "0";croptop
"1";cropright "1";cropbottom "0";tempfilename
'LTUWCZ5W.wmf';tempfile-properties "XPR";}}

Take a point object that either glows, or has rays of light reflecting from
it. The rays leave the object and reach the surface of the glass. The rays
will refract at the surface. Each bends toward the normal, but because of
the curvature of the glass, the rays all converge toward the center. We can
identify this convergence point as the image of the point object. Since our
object is a point, so is our image. Of course we could make an extended
object out of many points, and then we would have many image points as well.

At the surface we can find the refracted angles using Snell's law%
\begin{equation*}
n_{1}\sin \theta _{1}=n_{2}\sin \theta _{2}
\end{equation*}

We will again use the small angle approximation. Then $\theta _{1}$ and $%
\theta _{2}$ are small and none of the rays are very far away from the axis.
This is our paraxial approximation. Snell's law becomes 
\begin{equation*}
n_{1}\theta _{1}=n_{2}\theta _{2}
\end{equation*}%
Let's try to see where the image will be using Snell's law.

\FRAME{dtbpF}{4.2687in}{1.6466in}{0in}{}{}{Figure}{\special{language
"Scientific Word";type "GRAPHIC";maintain-aspect-ratio TRUE;display
"USEDEF";valid_file "T";width 4.2687in;height 1.6466in;depth
0in;original-width 4.2186in;original-height 1.6103in;cropleft "0";croptop
"1";cropright "1";cropbottom "0";tempfilename
'S22VDD00.wmf';tempfile-properties "XPR";}} Using the more detailed figure
above, we observe triangles $SAC$ and $PAC.$ We recall that for triangle $%
SAC $ the top angle labeled $\eta $, plus $\theta _{1}$ must be $180.$ 
\begin{equation*}
180\unit{%
%TCIMACRO{\U{b0}}%
%BeginExpansion
{{}^\circ}%
%EndExpansion
}=\theta _{1}+\eta
\end{equation*}%
or 
\begin{equation*}
\eta =180\unit{%
%TCIMACRO{\U{b0}}%
%BeginExpansion
{{}^\circ}%
%EndExpansion
}-\theta _{1}
\end{equation*}%
We also know that the sum of interior angles must equal $180.$ So for
triangle $SAC$ we know 
\begin{equation*}
180\unit{%
%TCIMACRO{\U{b0}}%
%BeginExpansion
{{}^\circ}%
%EndExpansion
}=\eta +\alpha +\beta
\end{equation*}%
\begin{equation*}
180\unit{%
%TCIMACRO{\U{b0}}%
%BeginExpansion
{{}^\circ}%
%EndExpansion
}=180\unit{%
%TCIMACRO{\U{b0}}%
%BeginExpansion
{{}^\circ}%
%EndExpansion
}-\theta _{1}+\alpha +\beta
\end{equation*}%
then 
\begin{equation*}
\theta _{1}=\alpha +\beta
\end{equation*}%
Likewise, from triangle $PAC$,%
\begin{equation*}
180\unit{%
%TCIMACRO{\U{b0}}%
%BeginExpansion
{{}^\circ}%
%EndExpansion
}=\sigma +\theta _{2}+\gamma
\end{equation*}%
and 
\begin{equation*}
180\unit{%
%TCIMACRO{\U{b0}}%
%BeginExpansion
{{}^\circ}%
%EndExpansion
}=\beta +\sigma
\end{equation*}%
so 
\begin{equation*}
\sigma =180\unit{%
%TCIMACRO{\U{b0}}%
%BeginExpansion
{{}^\circ}%
%EndExpansion
}-\beta
\end{equation*}%
and%
\begin{equation*}
180\unit{%
%TCIMACRO{\U{b0}}%
%BeginExpansion
{{}^\circ}%
%EndExpansion
}=\left( 180\unit{%
%TCIMACRO{\U{b0}}%
%BeginExpansion
{{}^\circ}%
%EndExpansion
}-\beta \right) +\theta _{2}+\gamma
\end{equation*}%
which reduced to 
\begin{equation*}
\beta =\theta _{2}+\gamma
\end{equation*}%
then, 
\begin{equation*}
\theta _{2}=\beta -\gamma
\end{equation*}%
and we can write our paraxial Snell's law as%
\begin{eqnarray*}
n_{1}\theta _{1} &=&n_{2}\theta _{2} \\
n_{1}\left( \alpha +\beta \right) &=&n_{2}\left( \beta -\gamma \right) \\
n_{1}\alpha +n_{1}\beta &=&n_{2}\beta -n_{2}\gamma \\
n_{1}\alpha +n_{2}\gamma &=&n_{2}\beta -n_{1}\beta \\
n_{1}\alpha +n_{2}\gamma &=&\beta \left( n_{2}-n_{1}\right)
\end{eqnarray*}

Looking at the figure. We see that $d$ is a leg of three different right
triangles ($SAV,$ $ACV$, and $PAV$). The ray in the figure is clearly not a
paraxial ray. If we use a paraxial ray, then the point $V$ will approach the
air-glass boundary. When this happens, then $SV=s,$ $VC=R$, and $%
VP=s^{\prime }.$ So we can write

\begin{eqnarray*}
\tan \alpha &\approx &\alpha \approx \frac{d}{s} \\
\tan \beta &\approx &\beta \approx \frac{d}{R} \\
\tan \gamma &\approx &\gamma \approx \frac{d}{s^{\prime }}
\end{eqnarray*}%
so our Snell's law becomes%
\begin{eqnarray*}
n_{1}\alpha +n_{2}\gamma &=&\beta \left( n_{2}-n_{1}\right) \\
n_{1}\frac{d}{s}+n_{2}\frac{d}{s^{\prime }} &=&\frac{d}{R}\left(
n_{2}-n_{1}\right)
\end{eqnarray*}%
We can divide out the common factor, $d.$%
\begin{equation}
\frac{n_{1}}{s}+\frac{n_{2}}{s^{\prime }}=\frac{\left( n_{2}-n_{1}\right) }{R%
}  \label{Bump_Equation}
\end{equation}

We can use this formula to convince ourselves that no matter what the angle
is (providing it is small), the rays will form an image at $P.$ So all the
rays in figure (\ref{Semi-infinite bump of glass}) will converge at $P.$

Real images will be in the glass for our example. This may seem a problem,
but we will fix this with non-infinite lenses soon. And for the case of our
eyes, this is exactly what happens. We have fluid (sort of a jelly) in our
eyes, and the image is formed in the fluid. The curved surface is our cornea
(the spot where your contacts go).

Physicists got together and decided on a mathematical system of signs to
make the math easier and consistent. We have called such a scheme a sign
convention already. We started collecting parts of this system last lecture.
Let's write it all out in a table so we can use it in today's lecture. Here
is the convention for the case of a curved semi-infinte surface.

\begin{equation*}
\begin{tabular}{|l|l|l|}
\hline
\textbf{Quantity} & \textbf{Positive if} & \textbf{Negative if} \\ \hline
{\small Object location }$\left( s\right) $ & 
\begin{tabular}{l}
{\small Object is in front of surface} \\ 
\end{tabular}
& 
\begin{tabular}{l}
{\small Object is in back of surface} \\ 
{\small (virtual object)}%
\end{tabular}%
{\small \ } \\ \hline
{\small Image location }$\left( s^{\prime }\right) $ & 
\begin{tabular}{l}
{\small Image is in back of surface} \\ 
{\small (real image)}%
\end{tabular}
& 
\begin{tabular}{l}
{\small Image is in front of surface } \\ 
{\small (virtual image)}%
\end{tabular}
\\ \hline
{\small Image height }$\left( h^{\prime }\right) $ & 
\begin{tabular}{l}
{\small Image is upright}%
\end{tabular}
& 
\begin{tabular}{l}
{\small Image is inverted}%
\end{tabular}
\\ \hline
{\small Radius }$\left( R\right) $ & 
\begin{tabular}{l}
{\small Center of curvature is in} \\ 
{\small back of surface}%
\end{tabular}%
{\small \ } & 
\begin{tabular}{l}
{\small Center of curvature is in} \\ 
{\small front of surface}%
\end{tabular}%
{\small \ } \\ \hline
\end{tabular}%
\end{equation*}%
where the \textquotedblleft front\textquotedblright\ of the lens is the side
that gets the light from the object. \FRAME{dtbpF}{4.2717in}{1.5451in}{0pt}{%
}{}{Figure}{\special{language "Scientific Word";type
"GRAPHIC";maintain-aspect-ratio TRUE;display "USEDEF";valid_file "T";width
4.2717in;height 1.5451in;depth 0pt;original-width 4.324in;original-height
1.5451in;cropleft "0";croptop "1";cropright "1";cropbottom "0";tempfilename
'N0FWDC00.wmf';tempfile-properties "XPR";}}%
%TCIMACRO{%
%\TeXButton{Question 223.16.2}{\marginpar {
%\hspace{-0.5in}
%\begin{minipage}[t]{1in}
%\small{Question 223.165.2}
%\end{minipage}
%}}}%
%BeginExpansion
\marginpar {
\hspace{-0.5in}
\begin{minipage}[t]{1in}
\small{Question 223.165.2}
\end{minipage}
}%
%EndExpansion
%TCIMACRO{%
%\TeXButton{Question 223.16.3}{\marginpar {
%\hspace{-0.5in}
%\begin{minipage}[t]{1in}
%\small{Question 223.165.3}
%\end{minipage}
%}}}%
%BeginExpansion
\marginpar {
\hspace{-0.5in}
\begin{minipage}[t]{1in}
\small{Question 223.165.3}
\end{minipage}
}%
%EndExpansion
%TCIMACRO{%
%\TeXButton{Question 223.16.4}{\marginpar {
%\hspace{-0.5in}
%\begin{minipage}[t]{1in}
%\small{Question 223.165.4}
%\end{minipage}
%}}}%
%BeginExpansion
\marginpar {
\hspace{-0.5in}
\begin{minipage}[t]{1in}
\small{Question 223.165.4}
\end{minipage}
}%
%EndExpansion

We could go through the entire derivation and switch the indices of
refraction (in effect, go along the same path, but going backwards). It
turns out we get the same equation. The light is bending the other way as it
travels the path, but the equation will be the same. So our equation
describes light entering a pice of glass, or light leaving a piece of glass.

\subsection{Flat Refracting surfaces}

\FRAME{dhF}{3.5085in}{3.2785in}{0in}{}{}{Figure}{\special{language
"Scientific Word";type "GRAPHIC";maintain-aspect-ratio TRUE;display
"USEDEF";valid_file "T";width 3.5085in;height 3.2785in;depth
0in;original-width 3.4618in;original-height 3.2335in;cropleft "0";croptop
"1";cropright "1";cropbottom "0";tempfilename
'LYQ1P600.wmf';tempfile-properties "XPR";}}

Let's return to our fist tank. The fish tank has an interface, but it is
flat. Can we use our equation (\ref{Bump_Equation}) to describe this?

The answer is yes, if we let $R=\infty .$ This makes sense for a flat
surface. If we have an infinitely large sphere, then our small part of that
spherical surface that makes up the fish tank wall will be very flat.

Then%
\begin{equation*}
\frac{n_{1}}{s}+\frac{n_{2}}{s^{\prime }}=\frac{\left( n_{2}-n_{1}\right) }{%
\infty }
\end{equation*}%
or%
\begin{equation*}
\frac{n_{1}}{s}+\frac{n_{2}}{s^{\prime }}=0
\end{equation*}%
we see that 
\begin{equation*}
s^{\prime }=-s\frac{n_{2}}{n_{1}}
\end{equation*}

This is what we got before for this case, except before we just got the
distance, and now we have included the effects of our sign convention. The
negative sign means that the image is in front of the surface. By
\textquotedblleft in front\textquotedblright\ we always mean to follow the
light from the source (fish) to the optical boundary. This boundary is the
water/air boundary of the tank, so the fact that our image is in the water
means that our image is in front of the optical boundary. As we know, this
means the image is virtual.

\section{Thin Lenses}

%TCIMACRO{%
%\TeXButton{Question 223.16.5}{\marginpar {
%\hspace{-0.5in}
%\begin{minipage}[t]{1in}
%\small{Question 223.165.5}
%\end{minipage}
%}}}%
%BeginExpansion
\marginpar {
\hspace{-0.5in}
\begin{minipage}[t]{1in}
\small{Question 223.165.5}
\end{minipage}
}%
%EndExpansion
Lets' find an equation for a lens made from sections of spherical surfaces
once more. But this time, let's let it be more practical and not make the
\textquotedblleft lens\textquotedblright\ semi-infinite. We will need to
deal with two sides of the lens because (usually) both will be curved.

We found that for refraction%
\begin{equation*}
\frac{n_{1}}{s}+\frac{n_{2}}{s^{\prime }}=\frac{\left( n_{2}-n_{1}\right) }{R%
}
\end{equation*}

but we did this for a spherical bump on a semi-infinite piece of glass. For
this problem let's make a few assumptions:

\begin{itemize}
\item We have two spherical surfaces, with $R_{1}$ and $R_{2}$ as the radii
of curvature

\item We have only paraxial rays

\item The image formed by one refractive surface serves as the object for
the second surface

\item The lens is not very thick (the thickness is much smaller than both $%
R_{1}$ and $R_{2})$
\end{itemize}

The answer we will get is quite simple

\begin{equation}
\frac{1}{s}+\frac{1}{s^{\prime }}=\frac{1}{f}  \label{Thin_lens}
\end{equation}%
where%
\begin{equation}
\frac{1}{f}=\left( n-1\right) \left( \frac{1}{R_{1}}-\frac{1}{R_{2}}\right)
\label{Lens_maker}
\end{equation}%
but to appreciate what it means, lets find out where it comes from.

\subsection{Derivation of the lens equation}

Let's find the thin lens formula. Really, we could just assume the formula
and be fine, but we are going through the derivation because it will help
teach us how to deal with multiple surfaces in an optical system. Telescopes
and microscopes all have multiple surfaces. So this will help us understand
how they work.

Consider the optical element in the figure below. Notice that our object is
a dot, so our image will also be a dot. \FRAME{dhF}{4.0542in}{2.4561in}{0pt}{%
}{}{Figure}{\special{language "Scientific Word";type
"GRAPHIC";maintain-aspect-ratio TRUE;display "USEDEF";valid_file "T";width
4.0542in;height 2.4561in;depth 0pt;original-width 4.0041in;original-height
2.4146in;cropleft "0";croptop "1";cropright "1";cropbottom "0";tempfilename
'LTUWD05Z.wmf';tempfile-properties "XPR";}}%
%TCIMACRO{%
%\TeXButton{Question 223.16.6}{\marginpar {
%\hspace{-0.5in}
%\begin{minipage}[t]{1in}
%\small{Question 223.165.6}
%\end{minipage}
%}}}%
%BeginExpansion
\marginpar {
\hspace{-0.5in}
\begin{minipage}[t]{1in}
\small{Question 223.165.6}
\end{minipage}
}%
%EndExpansion
By now you have realized that this is not as boring as it sounds if we
consider any object can be considered as a collection of dots.\FRAME{dhFU}{%
1.7184in}{2.2329in}{0pt}{\Qcb{The Pont Neuf by Hippolyte Petitjean.
Petitjean was a Pointillist, one who painted with dots of paint instead of
continuous application of paint. This illustrates the thought that all
objects can be though of collections of small points that reflect or emit
light. So we can consider an optical system by considering individual points
of light and how the system reacts to those points of light. (Image in the
Public Domain)}}{\Qlb{Point Neuf}}{Figure}{\special{language "Scientific
Word";type "GRAPHIC";maintain-aspect-ratio TRUE;display "USEDEF";valid_file
"T";width 1.7184in;height 2.2329in;depth 0pt;original-width
1.6821in;original-height 2.1923in;cropleft "0";croptop "1";cropright
"1";cropbottom "0";tempfilename 'LYQ21P01.wmf';tempfile-properties "XPR";}}%
So we can consider anything as a collection of dots, and work out our
formulas for a dot (because it is easier to think about just one dot at a
time).

Light enters at a spherical surface on the left hand side. We use a point
object located at $S$ on the principal axis, and trace two rays. The ray
along the principal axis crosses each spherical surface at right angles, and
therefore traves straight through the optic. The second ray hits the first
spherical surface at point $A.$ It is refracted and travels to point $B$. It
is again refracted and travels toward the principal axis, crossing at $P.$
The image location is the intersection of these rays, so we have an image at 
$P.$

Lets study the surfaces separately

Surface 1:

\FRAME{dhF}{3.7879in}{2.5676in}{0pt}{}{}{Figure}{\special{language
"Scientific Word";type "GRAPHIC";maintain-aspect-ratio TRUE;display
"USEDEF";valid_file "T";width 3.7879in;height 2.5676in;depth
0pt;original-width 3.7395in;original-height 2.5261in;cropleft "0";croptop
"1";cropright "1";cropbottom "0";tempfilename
'LTUWD060.wmf';tempfile-properties "XPR";}}

We treat surface $1$ as though surface $2$ did not exist. After all, the
light does not know about surface $2$ as it hits surface $1$.

By considering surface $1$ on its own, we have just our semi-infinite bump
problem, so we know that 
\begin{equation*}
\frac{n_{1}}{s}+\frac{n_{2}}{s^{\prime }}=\frac{\left( n_{2}-n_{1}\right) }{R%
}
\end{equation*}

We can consider $n_{1}=1$ and $n_{2}=n$ for an air-glass interface and
noting that $s_{1}^{\prime }$ is negative by our convention. Then%
\begin{equation}
\frac{1}{s_{1}}-\frac{n}{s_{1}^{\prime }}=\frac{\left( n-1\right) }{R_{1}}
\label{lens1}
\end{equation}

Note that our rays are \emph{not} converging in the glass. We can find the
image formed by this surface $1$ of our lens by tracing the diverging rays
backward as we did for the fish tank or magnifying glass. The image formed
from the first side of the lens is virtual.

Surface 2: Now consider the second surface.

\FRAME{dhF}{4.4166in}{2.1906in}{0pt}{}{}{Figure}{\special{language
"Scientific Word";type "GRAPHIC";maintain-aspect-ratio TRUE;display
"USEDEF";valid_file "T";width 4.4166in;height 2.1906in;depth
0pt;original-width 4.3656in;original-height 2.1508in;cropleft "0";croptop
"1";cropright "1";cropbottom "0";tempfilename
'LYQG3B06.wmf';tempfile-properties "XPR";}}The second surface sees light
diverging as though it came from a semi-infinite piece of glass with the
origin at $P_{1}.$ The virtual image formed by surface $1$ serves as the
object for surface $2$ because the diverging light is just the same pattern
as if there were a light source at $P_{1}$. The distance from $P_{1}$ to
surface $2$ is%
\begin{equation*}
s_{2}=s_{1}^{\prime }+t
\end{equation*}

We again use our refractive equation for a semi-infinite bump%
\begin{equation*}
\frac{n_{1}}{s}+\frac{n_{2}}{s^{\prime }}=\frac{\left( n_{2}-n_{1}\right) }{R%
}
\end{equation*}%
but we identify $n_{1}=n$ and $n_{2}=1$. We have for surface $2$%
\begin{equation}
\frac{n}{s_{2}}+\frac{1}{s_{2}^{\prime }}=\frac{\left( 1-n\right) }{R_{2}}
\end{equation}%
or%
\begin{equation}
\frac{n}{s_{1}^{\prime }+t}+\frac{1}{s_{2}^{\prime }}=\frac{\left(
1-n\right) }{R_{2}}  \label{lens2}
\end{equation}

Now we take our thin lens approximation. Let $t\rightarrow 0.$ Then
equations \ref{lens1} and \ref{lens2} become%
\begin{equation*}
\frac{1}{s_{1}}-\frac{n}{s_{1}^{\prime }}=\frac{\left( n-1\right) }{R_{1}}
\end{equation*}

\begin{equation*}
\frac{n}{s_{1}}+\frac{1}{s_{2}^{\prime }}=\frac{\left( 1-n\right) }{R_{2}}
\end{equation*}

I would like a single equation that gives $s_{2}^{\prime }$ in terms of $%
s_{1}.$ That is the form of the thin lens equation that we are looking for.
Adding these two equations can give me such an equation.%
\begin{equation*}
\frac{1}{s_{1}}-\frac{n}{s_{1}^{\prime }}+\frac{n}{s_{1}^{\prime }}+\frac{1}{%
s_{2}^{\prime }}=\frac{\left( n-1\right) }{R_{1}}+\frac{\left( 1-n\right) }{%
R_{2}}
\end{equation*}%
or%
\begin{equation*}
\frac{1}{s_{1}}+\frac{1}{s_{2}^{\prime }}=\left( n-1\right) \left( \frac{1}{%
R_{1}}-\frac{1}{R_{2}}\right)
\end{equation*}

This equation is very useful. If we again let $s_{1}=\infty $ (put the
object at $\infty $ so the rays enter surface $1$ parallel) we find%
\begin{equation*}
\frac{1}{s_{2}^{\prime }}=\left( n-1\right) \left( \frac{1}{R_{1}}-\frac{1}{%
R_{2}}\right)
\end{equation*}%
The spot where the rays gather if the object is infinitely far away is the
focal point, $f,$ so we can identify $s_{2}^{\prime }=f$ as the focal length
of the optic in this special case.

Then we can identify%
\begin{equation*}
\frac{1}{f}=\left( n-1\right) \left( \frac{1}{R_{1}}-\frac{1}{R_{2}}\right)
\end{equation*}

which is known as the \emph{lens makers' equation}. It gives us a way to
make a lens that will have a particular focal distance. You grind one side
of the lens to have a radius of curvature $R_{1}$ and the other side to have
radius of curvature $R_{2.}$ Then with index $n,$ you will have the focal
length you desire.

We have a relationship between the object distance in front of the lens, and
the final image in back of the lens: 
\begin{eqnarray*}
\frac{1}{s_{1}}+\frac{1}{s_{2}^{\prime }} &=&\left( n-1\right) \left( \frac{1%
}{R_{1}}-\frac{1}{R_{2}}\right) \\
&=&\frac{1}{f}
\end{eqnarray*}%
If we drop the subscripts (which we can do now that we let $t=0$ since the
internal distances for the inside points are not important) then.%
\begin{equation*}
\frac{1}{s}+\frac{1}{s^{\prime }}=\frac{1}{f}
\end{equation*}%
This is called the \emph{thin lens equation}. The resulting approximate
geometry is shown below.

\FRAME{dhF}{4.4581in}{1.3396in}{0in}{}{}{Figure}{\special{language
"Scientific Word";type "GRAPHIC";maintain-aspect-ratio TRUE;display
"USEDEF";valid_file "T";width 4.4581in;height 1.3396in;depth
0in;original-width 4.4071in;original-height 1.305in;cropleft "0";croptop
"1";cropright "1";cropbottom "0";tempfilename
'LTUWD162.wmf';tempfile-properties "XPR";}}

Of course any real object is made of lots of points, but each point is
imaged in a corresponding point on the image\FRAME{dhF}{2.3255in}{1.8792in}{%
0pt}{}{}{Figure}{\special{language "Scientific Word";type
"GRAPHIC";maintain-aspect-ratio TRUE;display "USEDEF";valid_file "T";width
2.3255in;height 1.8792in;depth 0pt;original-width 4.2817in;original-height
3.4558in;cropleft "0";croptop "1";cropright "1";cropbottom "0";tempfilename
'LTUWD263.wmf';tempfile-properties "XPR";}}so, as we claimed earlier, our
simple analysis explains the formation of actual images and not just point
images.%
%TCIMACRO{%
%\TeXButton{Question 123.16.7}{\marginpar {
%\hspace{-0.5in}
%\begin{minipage}[t]{1in}
%\small{Question 123.16.7}
%\end{minipage}
%}}}%
%BeginExpansion
\marginpar {
\hspace{-0.5in}
\begin{minipage}[t]{1in}
\small{Question 123.16.7}
\end{minipage}
}%
%EndExpansion

\subsection{Sign Convention}

We need to add to our sign convention table a second radius, and the focal
length.

\begin{equation*}
\begin{tabular}{|l|l|l|}
\hline
\textbf{Quantity} & \textbf{Positive if} & \textbf{Negative if} \\ \hline
{\small Object location }$\left( s\right) $ & 
\begin{tabular}{l}
{\small Object is in front of surface} \\ 
\end{tabular}
& 
\begin{tabular}{l}
{\small Object is in back of surface} \\ 
{\small (virtual object)}%
\end{tabular}%
{\small \ } \\ \hline
{\small Image location }$\left( s^{\prime }\right) $ & 
\begin{tabular}{l}
{\small Image is in back of surface} \\ 
{\small (real image)}%
\end{tabular}
& 
\begin{tabular}{l}
{\small Image is in front of surface } \\ 
{\small (virtual image)}%
\end{tabular}
\\ \hline
{\small Image height }$\left( h^{\prime }\right) $ & 
\begin{tabular}{l}
{\small Image is upright}%
\end{tabular}
& 
\begin{tabular}{l}
{\small Image is inverted}%
\end{tabular}
\\ \hline
{\small Radius }$\left( R_{1}\text{ and }R_{2}\right) $ & 
\begin{tabular}{l}
{\small Center of curvature is in} \\ 
{\small back of surface}%
\end{tabular}%
{\small \ } & 
\begin{tabular}{l}
{\small Center of curvature is in} \\ 
{\small front of surface}%
\end{tabular}%
{\small \ } \\ \hline
Focal length $\left( f\right) $ & Converging lens & Diverging lens \\ \hline
\end{tabular}%
\end{equation*}%
Again the front surface is the surface that gets the light from the object.%
\FRAME{dtbpF}{3.822in}{1.3402in}{0in}{}{}{Figure}{\special{language
"Scientific Word";type "GRAPHIC";maintain-aspect-ratio TRUE;display
"USEDEF";valid_file "T";width 3.822in;height 1.3402in;depth
0in;original-width 3.8663in;original-height 1.3376in;cropleft "0";croptop
"1";cropright "1";cropbottom "0";tempfilename
'MWHA9G02.wmf';tempfile-properties "XPR";}}

Note that each radius has a sign. If the two radii are the same magnitude,
it looks like 
\begin{equation*}
\frac{1}{f}=\left( n-1\right) \left( \frac{1}{R_{1}}-\frac{1}{R_{2}}\right)
\end{equation*}%
should be undefined (the focal length should be infinite) but usually that
is not true because either $R_{1}$ or $R_{2}$ will be negative.

\subsection{Magnification}

The image is not likely to be the same size as the object. We would like to
have a quantity that tells us how big the image is. The measure of choice is
the ratio of the two heights. 
\begin{equation}
m=\frac{h^{\prime }}{h}
\end{equation}%
where $h$ is the object height and $h^{\prime }$ is the image height. Note
that with our sign convention, if $m>0$ then the image is upright, and if $%
m<0$ the image is inverted (upside down).We call this ration the \emph{%
magnification} of the lens. \FRAME{dtbpF}{3.1583in}{1.6769in}{0pt}{}{}{Figure%
}{\special{language "Scientific Word";type "GRAPHIC";maintain-aspect-ratio
TRUE;display "USEDEF";valid_file "T";width 3.1583in;height 1.6769in;depth
0pt;original-width 3.1133in;original-height 1.6397in;cropleft "0";croptop
"1";cropright "1";cropbottom "0";tempfilename
'MHR7AI01.wmf';tempfile-properties "XPR";}}

We can find an expression for the magnification in terms of $s$ and $%
s^{\prime }.$ By observing the figure, and using the ray that goes right
through the middle of the lens, we can see that 
\begin{equation*}
\tan \theta =\frac{h}{s}
\end{equation*}%
and 
\begin{equation*}
\tan \theta =\frac{h^{\prime }}{s^{\prime }}
\end{equation*}%
thus%
\begin{equation*}
\frac{h}{s}=\frac{h^{\prime }}{s^{\prime }}
\end{equation*}%
then 
\begin{equation*}
\frac{s^{\prime }}{s}=\frac{h^{\prime }}{h}
\end{equation*}%
which we can use to form a new equation for the magnification.%
\begin{equation}
m=-\frac{s^{\prime }}{s}
\end{equation}

\section{Images formed by Mirrors}

%TCIMACRO{%
%\TeXButton{Question 123.16.8}{\marginpar {
%\hspace{-0.5in}
%\begin{minipage}[t]{1in}
%\small{Question 123.16.8}
%\end{minipage}
%}}}%
%BeginExpansion
\marginpar {
\hspace{-0.5in}
\begin{minipage}[t]{1in}
\small{Question 123.16.8}
\end{minipage}
}%
%EndExpansion
%TCIMACRO{%
%\TeXButton{Question 123.16.9}{\marginpar {
%\hspace{-0.5in}
%\begin{minipage}[t]{1in}
%\small{Question 123.16.9}
%\end{minipage}
%}}}%
%BeginExpansion
\marginpar {
\hspace{-0.5in}
\begin{minipage}[t]{1in}
\small{Question 123.16.9}
\end{minipage}
}%
%EndExpansion
All of us have looked in a mirror at some time. We know what to expect. We
see an image of ourselves. To study mirrors we need to establish a sign
convention and some standard notation

\FRAME{dhF}{1.3699in}{1.4364in}{0pt}{}{}{Figure}{\special{language
"Scientific Word";type "GRAPHIC";maintain-aspect-ratio TRUE;display
"USEDEF";valid_file "T";width 1.3699in;height 1.4364in;depth
0pt;original-width 1.3353in;original-height 1.4019in;cropleft "0";croptop
"1";cropright "1";cropbottom "0";tempfilename
'LTUWD264.wmf';tempfile-properties "XPR";}}In the figure above, we have a
person observing an object $O$ in a mirror. The object is located at a
distance $s$ from the mirror. Just like with lenses, we will call this the 
\emph{object distance}. The image appears to be located at a point $I$
beyond the mirror a distance $s^{\prime }.$ This is the \emph{image distance}%
.

It might be good to review how images are formed. Images are located at a
point from which rays of light diverge \emph{or at a point from which rays
of light appear to diverge}. This only makes sense. If you remember how we
see things, our eyes intercept rays of light diverging from an object. So if
we can create a situation that makes rays diverge in the same way the object
did, we will have an image of the object.

Mirrors create what we have called \emph{virtual} images because the image
appears to be created from diverging rays from behind the mirror, but if we
look behind the mirror no rays exist at the image location (if they did
exist, they would not make it through the mirror!).

\subsection{Image from a Flat Mirror}

\FRAME{dhF}{2.655in}{1.5212in}{0in}{}{}{Figure}{\special{language
"Scientific Word";type "GRAPHIC";maintain-aspect-ratio TRUE;display
"USEDEF";valid_file "T";width 2.655in;height 1.5212in;depth
0in;original-width 2.6135in;original-height 1.4849in;cropleft "0";croptop
"1";cropright "1";cropbottom "0";tempfilename
'LTUWD265.wmf';tempfile-properties "XPR";}}

Let's look at a simple image as shown in the figure above. The object (of
course) is an arrow. We could trace all the rays that diverge from this
object and build a very nice representation of the arrow\footnote{%
Ray tracing-based computer graphics actually does this--the way movies like 
\emph{Toy Story} are made.} but that would take time and computation power.
We only really need to use two rays, and remember what the object looked
like.

We pick one ray from the top of the arrow that travels straight to the
mirror. This ray will travel a distance $s$ and bounce back. We pick a
second ray from point $P$ that travels the path $PR.$ This ray bounces off
the mirror at an angle $\theta .$ So it appears that the tip of the arrow is
at position $P^{\prime }$ and the rays from the tip appear to travel the
paths $P^{\prime }P$ and $P^{\prime }R.$ Again, our visual processing center
in our brain interprets the rays as traveling in straight lines.

\section{Mirror reversal}

%TCIMACRO{%
%\TeXButton{Question 223.16.10}{\marginpar {
%\hspace{-0.5in}
%\begin{minipage}[t]{1in}
%\small{Question 223.16.10}
%\end{minipage}
%}}}%
%BeginExpansion
\marginpar {
\hspace{-0.5in}
\begin{minipage}[t]{1in}
\small{Question 223.16.10}
\end{minipage}
}%
%EndExpansion
Look into a mirror. Raise your left hand. Your image raises what appears to
be a right hand. It looks like a mirror switches the left and right sides of
the image. But lie sideways on the ground in front of the mirror and raise a
hand. Your hand does not get inverted (and neither do your feet and head).
What is happening? A flat mirror performs a front-back reversal. What this
means is that, following the light direction, the object is positioned so
that the back is encountered first, then the front, but in the image the
front of the image is encountered first, then the back.\FRAME{dhF}{3.9695in}{%
2.2191in}{0in}{}{}{Figure}{\special{language "Scientific Word";type
"GRAPHIC";maintain-aspect-ratio TRUE;display "USEDEF";valid_file "T";width
3.9695in;height 2.2191in;depth 0in;original-width 3.9202in;original-height
2.1785in;cropleft "0";croptop "1";cropright "1";cropbottom "0";tempfilename
'LYQ41W02.wmf';tempfile-properties "XPR";}}this has the effect of making it
look like the left hand is raised when the object's right hand is raised.

\subsection{Concave Mirrors}

%TCIMACRO{%
%\TeXButton{Question 223.16.11}{\marginpar {
%\hspace{-0.5in}
%\begin{minipage}[t]{1in}
%\small{Question 223.16.11}
%\end{minipage}
%}}}%
%BeginExpansion
\marginpar {
\hspace{-0.5in}
\begin{minipage}[t]{1in}
\small{Question 223.16.11}
\end{minipage}
}%
%EndExpansion
Concave mirrors can form images. I'm sure you know that many telescopes are
made with mirrors. We should see how this works. Let's proceed like we did
for lenses, but looking at what happens when light strikes the surface of
the new material, this time the mirror surface. First, lets look at rays
that come from very distant objects so they enter parallel to the optic
axis. We recall the law of reflection\FRAME{dhF}{1.8585in}{1.1857in}{0pt}{}{%
}{Figure}{\special{language "Scientific Word";type
"GRAPHIC";maintain-aspect-ratio TRUE;display "USEDEF";valid_file "T";width
1.8585in;height 1.1857in;depth 0pt;original-width 1.8213in;original-height
1.1519in;cropleft "0";croptop "1";cropright "1";cropbottom "0";tempfilename
'LTUWD266.wmf';tempfile-properties "XPR";}}

\begin{equation*}
\theta _{i}=\theta _{r}
\end{equation*}

Armed with this, we can see what would happen. Each ray has a different
normal due to the curvature of the mirror. The result is that the rays all
meet at a spot on the axis.

\FRAME{dhF}{3.243in}{2.079in}{0pt}{}{}{Figure}{\special{language "Scientific
Word";type "GRAPHIC";maintain-aspect-ratio TRUE;display "USEDEF";valid_file
"T";width 3.243in;height 2.079in;depth 0pt;original-width
3.1981in;original-height 2.0401in;cropleft "0";croptop "1";cropright
"1";cropbottom "0";tempfilename 'LTUWD267.wmf';tempfile-properties "XPR";}}%
This is a focal point!

\subsection{Paraxial Approximation for Mirrors}

The correct shape of a mirror is more like a parabola, but parabolas are
hard to machine or build. Spherical shapes are relatively easy. So we often
see spherical mirrors just like we often see spherical lenses. This will
work so long as we allow only rays that make small angles with respect to
the principal axis. We can see why this works if we plot a sphere and a
parabola (and a hyperbola). For small deviations from the center, the shape
of the functions all look alike. \FRAME{dhF}{4.3613in}{1.3949in}{0pt}{}{}{%
Figure}{\special{language "Scientific Word";type
"GRAPHIC";maintain-aspect-ratio TRUE;display "USEDEF";valid_file "T";width
4.3613in;height 1.3949in;depth 0pt;original-width 4.3102in;original-height
1.3604in;cropleft "0";croptop "1";cropright "1";cropbottom "0";tempfilename
'LTUWD268.wmf';tempfile-properties "XPR";}}We would expect the reflections
to be similar under these circumstances, so, if we meet the criteria for the
paraxial approximation, our spherical mirrors should work. Note that when
you need the entire mirror, say, in a communications antenna, you must do
better than a spherical approximation to the correct shape for your mirror. 
\FRAME{dhF}{2.8685in}{1.7598in}{0pt}{}{}{Figure}{\special{language
"Scientific Word";type "GRAPHIC";maintain-aspect-ratio TRUE;display
"USEDEF";valid_file "T";width 2.8685in;height 1.7598in;depth
0pt;original-width 2.8942in;original-height 1.7642in;cropleft "0";croptop
"1";cropright "1";cropbottom "0";tempfilename
'LVQW1703.wmf';tempfile-properties "XPR";}}

Like with our flat mirror. we will measure distances from the mirror surface
(from point $V$). We can find the image location by again taking two rays.
One convenient ray is the ray that passes through the center of curvature, $%
C.$ This ray will strike the mirror surface at right angles and bounce back
along the same path. Another convenient ray is the ray from the tip of the
object to point $V.$ This ray will bounce back with angle $\theta $. Where
these two reflected rays cross, we will find the image of the tip of our
object (a tree this time, I\ got tired of imaging arrows). Knowing the shape
of the object and that the bottom is on the axis, we can fill in the rest of
the image.

We can calculate the magnification for this case. We use the gold triangle
to determine that 
\begin{equation*}
\tan \theta =\frac{h}{s}
\end{equation*}%
and the blue triangle to determine that 
\begin{equation*}
\tan \theta =\frac{h^{\prime }}{s^{\prime }}
\end{equation*}%
so we have 
\begin{equation*}
m=\frac{h^{\prime }}{h}=\frac{-s^{\prime }\tan \theta }{s\tan \theta }=-%
\frac{s^{\prime }}{s}
\end{equation*}%
We want to indicate that the image is inverted by making it's sign negative.
We recall that $h^{\prime }$ is negative if the image is inverted. So we
will added a negative sign to make this fit with our sign convention.

\subsection{Mirror Equation}

We can further exploit this geometry to get a relationship between $s,$ $%
s^{\prime },$ and $R.$ Notice that 
\begin{equation*}
\tan \alpha =\frac{h}{s-R}
\end{equation*}%
and that 
\begin{equation*}
\tan \alpha =\frac{-h^{\prime }}{R-s^{\prime }}
\end{equation*}

Then%
\begin{equation*}
\frac{h}{s-R}=\frac{-h^{\prime }}{R-s^{\prime }}
\end{equation*}%
or%
\begin{equation*}
\frac{R-s^{\prime }}{s-R}=-\frac{h^{\prime }}{h}
\end{equation*}

We can use our magnification definition to replace $h^{\prime }/h$%
\begin{equation*}
\frac{R-s^{\prime }}{s-R}=\frac{s^{\prime }}{s}
\end{equation*}%
we perform some algebra%
\begin{eqnarray*}
\left( R-s^{\prime }\right) s &=&s^{\prime }\left( s-R\right) \\
-s^{\prime }s+Rs &=&ss^{\prime }-Rs^{\prime } \\
Rs+Rs^{\prime } &=&ss^{\prime }+s^{\prime }s \\
Rs+Rs^{\prime } &=&2ss^{\prime } \\
\frac{Rs^{\prime }}{Rss^{\prime }}+\frac{Rs}{Rss^{\prime }} &=&\frac{%
2ss^{\prime }}{Rss^{\prime }} \\
\frac{1}{s}+\frac{1}{s^{\prime }} &=&\frac{2}{R}
\end{eqnarray*}%
This is called the \emph{mirror equation}.

\subsection{Focal Point for Mirrors}

Now that we know the mirror equation, let's let $s$ be very large. Then 
\begin{equation*}
\frac{1}{s^{\prime }}\approx \frac{2}{R}
\end{equation*}%
or 
\begin{equation*}
s^{\prime }\approx \frac{R}{2}
\end{equation*}

Using the same logic as with the lens, we can identify this as the \emph{%
focal point}, $F$ and the distance $s^{\prime }$ in this case will be called
the \emph{focal length}, $f.$ We see that 
\begin{equation}
f=\frac{R}{2}
\end{equation}%
so we can write the mirror equation as%
\begin{equation}
\frac{1}{s}+\frac{1}{s^{\prime }}=\frac{1}{f}
\end{equation}

For a mirror, the value of $f$ does not depend on the mirror material (this
is not true for refractive optics). Of course we have a sign convention, but
it is similar to the convention for lenses. Here is the convention for
mirrors.

\bigskip

\begin{equation*}
\begin{tabular}{|l|l|l|}
\hline
\textbf{Quantity} & \textbf{Positive if} & \textbf{Negative if} \\ \hline
{\small Object location }$\left( s\right) $ & 
\begin{tabular}{l}
{\small Object is in front of surface} \\ 
\end{tabular}
& 
\begin{tabular}{l}
{\small Object is in back of surface} \\ 
{\small (virtual object)}%
\end{tabular}%
{\small \ } \\ \hline
{\small Image location }$\left( s^{\prime }\right) $ & 
\begin{tabular}{l}
{\small Image is in front of surface} \\ 
{\small (real image)}%
\end{tabular}
& 
\begin{tabular}{l}
{\small Image is in back of surface } \\ 
{\small (virtual image)}%
\end{tabular}
\\ \hline
{\small Image height }$\left( h^{\prime }\right) $ & 
\begin{tabular}{l}
{\small Image is upright}%
\end{tabular}
& 
\begin{tabular}{l}
{\small Image is inverted}%
\end{tabular}
\\ \hline
{\small Radius }$\left( R_{1}\text{ and }R_{2}\right) $ & 
\begin{tabular}{l}
{\small Center of curvature is in} \\ 
{\small front of surface}%
\end{tabular}%
{\small \ } & 
\begin{tabular}{l}
{\small Center of curvature is in} \\ 
{\small back of surface}%
\end{tabular}%
{\small \ } \\ \hline
Focal length $\left( f\right) $ & Concave mirror & Convex mirror \\ \hline
\end{tabular}%
\end{equation*}%
Where the front is, as usual, the part of the mirror that receives the light
first from the object.\FRAME{dtbpF}{3.3306in}{1.7119in}{0pt}{}{}{Figure}{%
\special{language "Scientific Word";type "GRAPHIC";maintain-aspect-ratio
TRUE;display "USEDEF";valid_file "T";width 3.3306in;height 1.7119in;depth
0pt;original-width 3.3652in;original-height 1.7163in;cropleft "0";croptop
"1";cropright "1";cropbottom "0";tempfilename
'MWHABY03.wmf';tempfile-properties "XPR";}}\bigskip Notice that $s^{\prime }$
is negative for virtual images as always.

\subsection{Ray Diagrams for Mirrors}

We have been drawing diagrams to find where images are formed for lenses, we
should do the same for mirrors. We use a similar set of three rays. These
rays are defined as follows:

Principal rays for a concave mirror:

\begin{enumerate}
\item Ray 1 is drawn from the top of the object such that its reflected ray
must pass through $f$.

\item Ray 2 is drawn from the top of the object through the focal point to
reflect parallel to the principal axis.

\item Ray 3 is drawn from the top of the object through the center of
curvature. This ray will be incident on the mirror surface at a right angle
and will be reflected back on itself.
\end{enumerate}

\FRAME{dhF}{2.8228in}{1.804in}{0pt}{}{}{Figure}{\special{language
"Scientific Word";type "GRAPHIC";maintain-aspect-ratio TRUE;display
"USEDEF";valid_file "T";width 2.8228in;height 1.804in;depth
0pt;original-width 3.7948in;original-height 2.4146in;cropleft "0";croptop
"1";cropright "1";cropbottom "0";tempfilename
'LTUWD36A.wmf';tempfile-properties "XPR";}}

We can do the same for an object closer than a focal length\FRAME{dhF}{%
2.9482in}{1.8498in}{0pt}{}{}{Figure}{\special{language "Scientific
Word";type "GRAPHIC";maintain-aspect-ratio TRUE;display "USEDEF";valid_file
"T";width 2.9482in;height 1.8498in;depth 0pt;original-width
3.7533in;original-height 2.3454in;cropleft "0";croptop "1";cropright
"1";cropbottom "0";tempfilename 'LTUWD36B.wmf';tempfile-properties "XPR";}}

We also may have a mirror that curves, but curves the other way.

Principal rays for a convex mirror:

\begin{enumerate}
\item Ray 1 is drawn from the top of the object such that its reflected ray
appears to have come from $f$.

\item Ray 2 is drawn from the top of the object to reflect parallel to the
principal axis.

\item Ray 3 is drawn from the top of the object so that it appears to have
come from the center of curvature. This ray will be incident on the mirror
surface at a right angle and will be reflected back on itself.
\end{enumerate}

\FRAME{dhF}{2.7397in}{2.1767in}{0pt}{}{}{Figure}{\special{language
"Scientific Word";type "GRAPHIC";maintain-aspect-ratio TRUE;display
"USEDEF";valid_file "T";width 2.7397in;height 2.1767in;depth
0pt;original-width 2.6974in;original-height 2.1369in;cropleft "0";croptop
"1";cropright "1";cropbottom "0";tempfilename
'LTUWD36C.wmf';tempfile-properties "XPR";}}

\subsection{Spherical Aberration}

%TCIMACRO{%
%\TeXButton{Question 223.16.12}{\marginpar {
%\hspace{-0.5in}
%\begin{minipage}[t]{1in}
%\small{Question 223.16.12}
%\end{minipage}
%}}}%
%BeginExpansion
\marginpar {
\hspace{-0.5in}
\begin{minipage}[t]{1in}
\small{Question 223.16.12}
\end{minipage}
}%
%EndExpansion
Spherical shapes are easier to make than parabolas or hyperbole, or other
shapes. So optics manufacturers have been using spherical optics for
centuries.\FRAME{dhF}{3.55in}{2.1767in}{0pt}{}{}{Figure}{\special{language
"Scientific Word";type "GRAPHIC";maintain-aspect-ratio TRUE;display
"USEDEF";valid_file "T";width 3.55in;height 2.1767in;depth
0pt;original-width 3.5034in;original-height 2.1369in;cropleft "0";croptop
"1";cropright "1";cropbottom "0";tempfilename
'LYQ96604.wmf';tempfile-properties "XPR";}}

If we let rays converge from any direction from our spherical mirror we find
we have a problem. The rays do not form a single image. Instead, they
converge on a volume near where the image should be. Rays from larger angles
converge at different distances than rays from small angles. This problem is
known as \emph{spherical aberration}. Most of the time, we will point our
optics so the object is near the principal axis, so we can make the paraxial
approximation that fixes this problem.%
%TCIMACRO{%
%\TeXButton{Question 223.16.5}{\marginpar {
%\hspace{-0.5in}
%\begin{minipage}[t]{1in}
%\small{Question 223.16.5}
%\end{minipage}
%}}}%
%BeginExpansion
\marginpar {
\hspace{-0.5in}
\begin{minipage}[t]{1in}
\small{Question 223.16.5}
\end{minipage}
}%
%EndExpansion

The same problem happens with lenses\FRAME{dhF}{2.348in}{1.7867in}{0in}{}{}{%
Figure}{\special{language "Scientific Word";type
"GRAPHIC";maintain-aspect-ratio TRUE;display "USEDEF";valid_file "T";width
2.348in;height 1.7867in;depth 0in;original-width 2.3082in;original-height
1.7495in;cropleft "0";croptop "1";cropright "1";cropbottom "0";tempfilename
'LYQ8QY03.wmf';tempfile-properties "XPR";}}

This problem is called spherical aberration and it was made famous as the
main problem with the Hubble Telescope.

There are many aberrations that come from making lenses that are easy to
manufacture, but that are not the perfect shape. We won't study these in
this class. If you are curious, we cover these in PH375.

Just a note, we have run into another aberration, chromatic aberration,
before. Mirrors in optical systems don't experience chromatic aberration.
this is because mirrors in optical systems don't include a glass layer in
front of the reflective surface like mirrors in your bathroom do. That glass
is to protect the reflective surface from damage due to water (or
toothpaste, etc.). In an optical system, this glass layer would cause
unwanted reflection and absorption of the light, so it is not included. So
mirrors don't have any refraction associated with them. This means that
there will be no dispersion from a mirror.

\chapter{Optical systems}

\section{Combinations of lenses}

So far we have only used one lens or mirror at a time. But most optical
systems are made from several lenses or mirrors (or a combination of both
lenses and mirrors). We should think about how lenses work together to form
optical systems like telescopes or microscopes or even compound camera
lenses.

%TCIMACRO{%
%\TeXButton{Question 223.17.1}{\marginpar {
%\hspace{-0.5in}
%\begin{minipage}[t]{1in}
%\small{Question 223.17.1}
%\end{minipage}
%}}}%
%BeginExpansion
\marginpar {
\hspace{-0.5in}
\begin{minipage}[t]{1in}
\small{Question 223.17.1}
\end{minipage}
}%
%EndExpansion
%TCIMACRO{%
%\TeXButton{Question 223.17.2}{\marginpar {
%\hspace{-0.5in}
%\begin{minipage}[t]{1in}
%\small{Question 223.17.2}
%\end{minipage}
%}}}%
%BeginExpansion
\marginpar {
\hspace{-0.5in}
\begin{minipage}[t]{1in}
\small{Question 223.17.2}
\end{minipage}
}%
%EndExpansion
%TCIMACRO{%
%\TeXButton{Question 223.17.3}{\marginpar {
%\hspace{-0.5in}
%\begin{minipage}[t]{1in}
%\small{Question 223.17.3}
%\end{minipage}
%}}}%
%BeginExpansion
\marginpar {
\hspace{-0.5in}
\begin{minipage}[t]{1in}
\small{Question 223.17.3}
\end{minipage}
}%
%EndExpansion
To combine lenses, we do the same thing we did for the two surfaces of a
thin lens. We form the image from the first lens as though the second lens
is not there. Then we use the image from the first lens as the object for
the second lens. Suppose we take two lenses of focal lengths $f_{1}$ and $%
f_{2}$ and place them a distance $d$ apart. \FRAME{dhF}{3.1912in}{1.2834in}{%
0pt}{}{}{Figure}{\special{language "Scientific Word";type
"GRAPHIC";maintain-aspect-ratio TRUE;display "USEDEF";valid_file "T";width
3.1912in;height 1.2834in;depth 0pt;original-width 5.9093in;original-height
2.3592in;cropleft "0";croptop "1";cropright "1";cropbottom "0";tempfilename
'LTUWD56F.wmf';tempfile-properties "XPR";}}

Because this system would use a magnified image as the object for lens $2,$
the final magnification is the product of the two lens magnifications

\begin{equation}
M_{\text{combined}}=M_{1}M_{2}
\end{equation}%
Let's see that this must be true%
\begin{equation*}
M_{1}=-\frac{s_{1}^{\prime }}{s_{1}}=\frac{h_{1}^{\prime }}{h}
\end{equation*}%
and 
\begin{equation*}
M_{2}=-\frac{s_{2}^{\prime }}{s_{2}}=\frac{h_{2}^{\prime }}{h_{1}^{\prime }}
\end{equation*}%
then 
\begin{equation*}
M_{1}M_{2}=\frac{h_{1}^{\prime }}{h}\frac{h_{2}^{\prime }}{h_{1}^{\prime }}=%
\frac{h_{2}^{\prime }}{h}
\end{equation*}%
which is what we mean when we give the magnification of the optical system.
We compare the output image size with the original object size.

It is a little more complicated to show where the final image will be. For
the first lens we have%
\begin{equation}
\frac{1}{s_{1}}+\frac{1}{s_{1}^{\prime }}=\frac{1}{f_{1}}
\end{equation}%
where $s_{1}^{\prime }$ is our first lens image distance. We can solve for $%
s_{1}^{\prime }$%
\begin{equation}
s_{1}^{\prime }=\frac{s_{1}f_{1}}{s_{1}-f_{1}}  \label{q}
\end{equation}

We then take as the second object distance 
\begin{equation*}
s_{2}=d-s_{1}^{\prime }
\end{equation*}

we use the lens formula again.%
\begin{equation*}
\frac{1}{s_{2}}+\frac{1}{s_{2}^{\prime }}=\frac{1}{f_{2}}
\end{equation*}%
and again find the image distance%
\begin{equation*}
s_{2}^{\prime }=\frac{s_{2}f_{2}}{s_{2}-f_{2}}
\end{equation*}%
but we can use our value of $s_{2}$ to find%
\begin{eqnarray*}
s_{2}^{\prime } &=&\frac{\left( d-s_{1}^{\prime }\right) f_{2}}{\left(
d-s_{1}^{\prime }\right) -f_{2}} \\
&=&\frac{\left( d-s_{1}^{\prime }\right) f_{2}}{d-s_{1}^{\prime }-f_{2}}
\end{eqnarray*}%
We have and expression relating the image distances, $d$ and $f_{2}.$ But we
would really like to have an expression that relates $s_{1}$ and $%
s_{2}^{\prime }.$ Lets use 
\begin{equation*}
s_{1}^{\prime }=\frac{s_{1}f_{1}}{s_{1}-f_{1}}
\end{equation*}

and substitute it into our expression for $s_{2}^{\prime }$%
\begin{equation*}
s_{2}^{\prime }=\frac{\left( d-\frac{s_{1}f_{1}}{s_{1}-f_{1}}\right) f_{2}}{%
d-\frac{s_{1}f_{1}}{s_{1}-f_{1}}-f_{2}}
\end{equation*}%
This looks messy, but we can do some simplification%
\begin{equation}
s_{2}^{\prime }=\frac{df_{2}-\frac{s_{1}f_{1}f_{2}}{s_{1}-f_{1}}}{d-f_{2}-%
\frac{s_{1}f_{1}}{s_{1}-f_{1}}}  \label{TwoLensesSeparated}
\end{equation}%
Well, it is still a little messy, but we have achieved our goal. We have $%
s_{2}^{\prime }$ in therms of the focal lengths, $d,$ and $s_{1}.$

Suppose we let $d\rightarrow 0.$ Then%
\begin{eqnarray*}
s_{2}^{\prime } &=&\frac{-\frac{s_{1}f_{1}f_{2}}{s_{1}-f_{1}}}{-f_{2}-\frac{%
s_{1}f_{1}}{s_{1}-f_{1}}} \\
&=&\frac{\frac{s_{1}f_{1}f_{2}}{s_{1}-f_{1}}}{\frac{f_{2}\left(
s_{1}-f_{1}\right) }{s_{1}-f_{1}}+\frac{s_{1}f_{1}}{s_{1}-f_{1}}} \\
&=&\frac{s_{1}f_{1}f_{2}}{f_{2}s_{1}-f_{2}f_{1}+s_{1}f_{1}} \\
&=&\frac{s_{1}f_{1}f_{2}}{s_{1}\left( f_{2}+f_{1}\right) -f_{2}f_{1}}
\end{eqnarray*}%
So%
\begin{equation*}
s_{2}^{\prime }=\frac{s_{1}f_{1}f_{2}}{s_{1}\left( f_{2}+f_{1}\right)
-f_{2}f_{1}}
\end{equation*}%
Lets undo the math that brought us $s_{2}^{\prime }$ in the first place 
\begin{eqnarray*}
\frac{1}{s_{2}^{\prime }} &=&\frac{s_{1}\left( f_{2}+f_{1}\right) -f_{2}f_{1}%
}{s_{1}f_{1}f_{2}} \\
&=&\frac{s_{1}\left( f_{2}+f_{1}\right) }{s_{1}f_{1}f_{2}}-\frac{f_{2}f_{1}}{%
s_{1}f_{1}f_{2}} \\
&=&\frac{\left( f_{2}+f_{1}\right) }{f_{1}f_{2}}-\frac{1}{s_{1}}
\end{eqnarray*}%
or%
\begin{equation*}
\frac{1}{s_{2}^{\prime }}+\frac{1}{s_{1}}=\frac{\left( f_{2}+f_{1}\right) }{%
f_{1}f_{2}}
\end{equation*}%
Which looks very like the lens formula with 
\begin{equation*}
\frac{1}{f}=\frac{\left( f_{2}+f_{1}\right) }{f_{1}f_{2}}
\end{equation*}%
If we unwind this expression, we find%
\begin{equation}
\frac{1}{f}=\frac{f_{2}}{f_{1}f_{2}}+\frac{f_{1}}{f_{1}f_{2}}  \notag
\end{equation}%
\begin{equation}
\frac{1}{f}=\frac{1}{f_{1}}+\frac{1}{f_{2}}
\end{equation}

This is how we combine thin lenses. We see that the two lenses are
equivalent to a single lens with focal length $f$ as long as they are close
together.

Of course, we had to place our lenses right next to each other for this to
work. This is not the case for a telescope or microscope. We should look at
such a case. There is no need for more math. We can go back to equation (\ref%
{TwoLensesSeparated}). 
\begin{equation*}
s_{2}^{\prime }=\frac{df_{2}-\frac{s_{1}f_{1}f_{2}}{s_{1}-f_{1}}}{d-f_{2}-%
\frac{s_{1}f_{1}}{s_{1}-f_{1}}}
\end{equation*}%
But let's look at a case using ray diagrams. For this case, let's take two
lenses, and let's have the first lens make a real image. Once again, let's
have that image be the object for the second lens. But this time, let's move
the second lens so that the image from the first lens (object for the second
lens) is closer to the second lens than $f_{2}.$ If that is the case, the
second lens works like a magnifier. The final image is enlarged.

\FRAME{dhF}{2.6074in}{1.6622in}{0pt}{}{}{Figure}{\special{language
"Scientific Word";type "GRAPHIC";maintain-aspect-ratio TRUE;display
"USEDEF";valid_file "T";width 2.6074in;height 1.6622in;depth
0pt;original-width 4.9355in;original-height 3.1367in;cropleft "0";croptop
"1";cropright "1";cropbottom "0";tempfilename
'LTUWD56G.wmf';tempfile-properties "XPR";}}

\section{The Camera}

in 1900 George Eastman introduced the Brownie Camera. This event has changed
society dramatically. The idea behind a camera is very simple.\FRAME{dhF}{%
1.9908in}{1.1329in}{0pt}{}{}{Figure}{\special{language "Scientific
Word";type "GRAPHIC";maintain-aspect-ratio TRUE;display "USEDEF";valid_file
"T";width 1.9908in;height 1.1329in;depth 0pt;original-width
3.9487in;original-height 2.2347in;cropleft "0";croptop "1";cropright
"1";cropbottom "0";tempfilename 'LTUWD56H.wmf';tempfile-properties "XPR";}}

The camera has a lens (often a compound lens like the ones we have just
discussed) and a screen for projecting a real image created by the lens.

Let's take an example camera. Say we wish to take a picture of Aunt Sally.
Aunt Sally is about $1.\,\allowbreak 5\unit{m}$ tall. She is standing about $%
5\unit{m}$ away. Then to fit the image of Aunt Sally on our $35\unit{mm}$
detector, we must have

\begin{equation*}
\begin{tabular}{l}
$h=1.5\unit{m}$ \\ 
$h^{\prime }=0.035\unit{m}$ \\ 
$s=5\unit{m}$ \\ 
$f=0.058\unit{m}$%
\end{tabular}%
\end{equation*}

We wish to find $s^{\prime }$ and $m.$ Let's do $m$ first. 
\begin{eqnarray*}
m &=&\frac{h^{\prime }}{h}=\frac{-0.035\unit{m}}{1.5\unit{m}} \\
&=&-2.\,\allowbreak 333\,3\times 10^{-2}
\end{eqnarray*}%
so our image is small and inverted. The small size we wanted. But now we
know that the image in our cameras is upside down. A digital camera uses
it's built-in computer to turn the image right side up for us on the display
on the back of the camera.

Now let's find $s^{\prime }.$ 
\begin{eqnarray*}
s^{\prime } &=&\frac{fs}{s-f} \\
&=&5.\,\allowbreak 868\,1\times 10^{-2}\unit{m} \\
&=&58.681\unit{mm}
\end{eqnarray*}%
so our detector must be $58.681\unit{mm}$ from the lens.

Now suppose we want to photograph a $1000\unit{m}$ tower from $2\unit{km}$
away. Then%
\begin{eqnarray*}
m &=&-\frac{0.035\unit{m}}{1000\unit{m}} \\
&=&-3.\,\allowbreak 5\times 10^{-5}
\end{eqnarray*}%
and 
\begin{eqnarray*}
s^{\prime } &=&\frac{\left( 0.058\unit{m}\right) \left( 2000\unit{m}\right) 
}{2000\unit{m}-\left( 0.058\unit{m}\right) } \\
&=&5.\,\allowbreak 800\,2\times 10^{-2}\unit{m} \\
&=&58.002\unit{mm}
\end{eqnarray*}%
Notice that the image distance changed, but not by very much. This is why
you need a focus adjustment on the lens of a good camera. Objects far away
require a different $s^{\prime }$ value than objects that are close. Usually
you twist the lens housing to make this adjustment. The lens housing has a
threaded screw system that increases or decreases $s^{\prime }$ as you
twist. Cell phones and consumer cameras often have an motor that makes this
adjustment for you. In some cameras you may see the lens move back and forth
as someone takes a picture.

There are several things that govern whether a picture will be good. When
you buy a quality manual lens, it will be marked in $f\#s$. The
specification of an automatic lens will be given in terms of $f/\#s$. To
help us buy such lenses, we should understand what the terminology means.

%TCIMACRO{%
%\TeXButton{Question 223.17.4}{\marginpar {
%\hspace{-0.5in}
%\begin{minipage}[t]{1in}
%\small{Question 223.17.4}
%\end{minipage}
%}}}%
%BeginExpansion
\marginpar {
\hspace{-0.5in}
\begin{minipage}[t]{1in}
\small{Question 223.17.4}
\end{minipage}
}%
%EndExpansion
Most things we want to take a snapshot of are much farther than $58\unit{mm}$
from the camera. For such objects we can revisit the magnification.%
\begin{equation*}
m=-\frac{s^{\prime }}{s}
\end{equation*}%
but from the thin lens formula%
\begin{equation*}
\frac{1}{s}+\frac{1}{s^{\prime }}=\frac{1}{f}
\end{equation*}%
If $s\gg f$ then we can say that $1/s\approx 0$ and so $s^{\prime }\approx
f. $ 
%TCIMACRO{%
%\TeXButton{Question 223.17.5}{\marginpar {
%\hspace{-0.5in}
%\begin{minipage}[t]{1in}
%\small{Question 223.17.5}
%\end{minipage}
%}}}%
%BeginExpansion
\marginpar {
\hspace{-0.5in}
\begin{minipage}[t]{1in}
\small{Question 223.17.5}
\end{minipage}
}%
%EndExpansion
Then%
\begin{equation*}
m=-\frac{f}{s}
\end{equation*}%
and we see that the size of the image is directly proportional to the focal
distance. If we change the focal distance, we can change the size of the
image. This is how a zoom lens works. A zoom lens is a compound lens, and
the focal length is changed by increasing the distance between the component
lenses. This is what your camera is doing when it zooms in and out when you
push the telephoto button.

Remember we studied intensity%
\begin{equation*}
I=\frac{P}{A}
\end{equation*}

Photographic film and digital focal plane arrays detect the intensity of
light falling on them. We can see that the area of our image depends on our
magnification, which depends on $s^{\prime }$ and for our distant objects it
is proportional to $f.$ The image area is proportional to $s^{\prime
2}\approx f^{2}.$ So we can say that the area is proportional to $f^{2}.$
Then 
\begin{equation*}
I\propto \frac{P}{f^{2}}
\end{equation*}%
%TCIMACRO{%
%\TeXButton{Question 223.17.6}{\marginpar {
%\hspace{-0.5in}
%\begin{minipage}[t]{1in}
%\small{Question 223.17.6}
%\end{minipage}
%}}}%
%BeginExpansion
\marginpar {
\hspace{-0.5in}
\begin{minipage}[t]{1in}
\small{Question 223.17.6}
\end{minipage}
}%
%EndExpansion
The power entering the camera is proportional to the size of the aperture
(hole the light goes through). A bigger aperture lets in more light. A
smaller aperture lets in less light. If the camera has a circular opening,
this area is proportional to the square of the diameter of the opening, $%
D^{2}$ so%
\begin{equation*}
I\propto \frac{D^{2}}{f^{2}}
\end{equation*}%
This ratio is useful because it tells us how much intensity we get in terms
of things we can easily know. Good cameras have changeable aperture sizes,
and good lenses have changeable focal lengths. By using the combination of
these two terms, we can ensure we will get enough light (but not too much)
when we take the picture.

It would be good to give this ratio a special name. But instead, we named
the ratio 
\begin{equation}
f/\#\equiv \frac{f}{D}
\end{equation}%
It is called the $f/\#$ (pronounced f-number) so 
\begin{equation}
I\propto \frac{1}{\left( f/\#\right) ^{2}}
\end{equation}%
So good cameras have adjustable lens systems marked in $f/\#^{\prime }s$.
Typical values are $f/2.8,$ $f/4,$ $f/5.6,$ $f/8,$ $f/11,$ and $f/16.$

This terminology is used for telescope design as well. The Hubble telescope
is an $f/24$ Ritchey-Chretien Cassegrainian system with a $2.4\unit{m}$
diameter aperture. The effective focal length is $57.6\unit{m}$.

\FRAME{dhF}{3.7593in}{2.3298in}{0pt}{}{}{Figure}{\special{language
"Scientific Word";type "GRAPHIC";maintain-aspect-ratio TRUE;display
"USEDEF";valid_file "T";width 3.7593in;height 2.3298in;depth
0pt;original-width 3.7118in;original-height 2.29in;cropleft "0";croptop
"1";cropright "1";cropbottom "0";tempfilename
'LYHAI20A.wmf';tempfile-properties "XPR";}}

It is important to realize that electronic (and biological) sensors don't
react instantly to what we see. The intensity is 
\begin{equation*}
I=\frac{P}{A}=\frac{\Delta E}{\Delta tA}
\end{equation*}%
so there is a time involved. The time it takes to collect enough light to
form an image on the sensor is called the \emph{exposure time}. 
\begin{equation*}
\Delta t=\frac{\Delta E}{IA}
\end{equation*}%
So changing our $f/\#$ changes the needed exposure time by changing the
intensity. How sensitive our camera sensor is also affects the exposure
time. Modern sensors have adjustable sensitivity. The photography world
gives the three letters ISO for the name for this detector sensitivity.
There isn't a standard for exactly what ISO setting gives what exposure.
Different manufacturers use slightly different numbers. But a change in ISO
settings usually are equivalent to one $f/\#$ change in exposure.

This is part of what a good photographer does in taking a picture. The
photographer will adjust the $f/\#$ and the exposure time and ISO to get a
photograph that is not too exposed (too light) or underexposed (to dark).

\chapter{Eyes and magnifiers}

%TCIMACRO{%
%\TeXButton{Fundamental Concepts}{\hspace{-1.3in}{\LARGE Fundamental Concepts\vspace{0.25in}}}}%
%BeginExpansion
\hspace{-1.3in}{\LARGE Fundamental Concepts\vspace{0.25in}}%
%EndExpansion

\begin{itemize}
\item Angular magnification compares the apparent size of an image with and
without an optical system.

\item The power of a lens is measured in Dipopters which are defined to be $%
1/f\left( \unit{m}\right) $

\item Compound magnifiers use an objective lens to form an image, and an
eyepiece to magnify the image.
\end{itemize}

\section{The Eye}

%TCIMACRO{%
%\TeXButton{Question 223.18.1}{\marginpar {
%\hspace{-0.5in}
%\begin{minipage}[t]{1in}
%\small{Question 223.18.1}
%\end{minipage}
%}}}%
%BeginExpansion
\marginpar {
\hspace{-0.5in}
\begin{minipage}[t]{1in}
\small{Question 223.18.1}
\end{minipage}
}%
%EndExpansion
\FRAME{dhF}{3.48in}{1.3673in}{0pt}{}{}{Figure}{\special{language "Scientific
Word";type "GRAPHIC";maintain-aspect-ratio TRUE;display "USEDEF";valid_file
"T";width 3.48in;height 1.3673in;depth 0pt;original-width
3.4342in;original-height 1.3327in;cropleft "0";croptop "1";cropright
"1";cropbottom "0";tempfilename 'LTUWD66J.wmf';tempfile-properties "XPR";}}

The figure above shows the parts of the eye. The eye is like a camera in its
operation, but is much more complex. It is truly a marvel. The parts that
concern us are the cornea, crystalline lens, pupil, and the retina. \FRAME{%
dhF}{2.5754in}{1.8585in}{0pt}{}{}{Figure}{\special{language "Scientific
Word";type "GRAPHIC";maintain-aspect-ratio TRUE;display "USEDEF";valid_file
"T";width 2.5754in;height 1.8585in;depth 0pt;original-width
4.2402in;original-height 3.0528in;cropleft "0";croptop "1";cropright
"1";cropbottom "0";tempfilename 'LTUWD66K.wmf';tempfile-properties "XPR";}}%
%TCIMACRO{%
%\TeXButton{Question 223.18.2}{\marginpar {
%\hspace{-0.5in}
%\begin{minipage}[t]{1in}
%\small{Question 223.18.2}
%\end{minipage}
%}}}%
%BeginExpansion
\marginpar {
\hspace{-0.5in}
\begin{minipage}[t]{1in}
\small{Question 223.18.2}
\end{minipage}
}%
%EndExpansion
The Cornea-lens system refracts the light onto the retina, which detects the
light. The lens is focused by a set of mussels that flatten the lens to
change it's focal length. The focusing process is different from a standard
camera. The camera moves the lens to achieve a different image distance. Our
eye can't change the distance between the lens system and the retina. So our
eye changes the shape of the lens, changing it's focal length. \FRAME{dhFU}{%
3.2387in}{1.446in}{0pt}{\Qcb{The crystaline lens becomes thicker, and
therefore more curved when the cilliary musscle flexes. Austin Flint,
\textquotedblleft The Eye as an Optical Instrument,\textquotedblright\ \emph{%
Popular Science Monthly,} Vol. 45, p203, 1894 (Image in the public domain)}}{%
}{Figure}{\special{language "Scientific Word";type
"GRAPHIC";maintain-aspect-ratio TRUE;display "USEDEF";valid_file "T";width
3.2387in;height 1.446in;depth 0pt;original-width 4.1987in;original-height
1.8602in;cropleft "0";croptop "1";cropright "1";cropbottom "0";tempfilename
'LYS4ON01.wmf';tempfile-properties "XPR";}}

The focusing system is called accommodation. This system becomes less
effective at about 40 because the lens becomes less flexible. The closest
point that can be focused by accommodation is called the near point. It is
about $25\unit{cm}$ on average. There is, of course, no such thing as an
average person. All of us are a little bit different. You young students
probably have a much shorter near point than $25\unit{cm}.$ For those of us
that are a little older, $25\unit{cm}$ or more is more likely.

The farthest point that can be focused is ideally a long way away. It is
called the far point. Both the near and far points degrade with years
leading to bifocals.

The iris changes the area of the pupil (the aperture of the eye). The pupil
is, on average, about $7\unit{mm}$ in diameter.

\subsection{Nearsightedness}

%TCIMACRO{%
%\TeXButton{Question 223.18.3}{\marginpar {
%\hspace{-0.5in}
%\begin{minipage}[t]{1in}
%\small{Question 223.18.3}
%\end{minipage}
%}}}%
%BeginExpansion
\marginpar {
\hspace{-0.5in}
\begin{minipage}[t]{1in}
\small{Question 223.18.3}
\end{minipage}
}%
%EndExpansion
In some people the cornea-lens system focuses in front of the retina. This
is called nearsightedness or myopia.

\FRAME{dhF}{3.5475in}{2.4189in}{0pt}{}{}{Figure}{\special{language
"Scientific Word";type "GRAPHIC";maintain-aspect-ratio TRUE;display
"USEDEF";valid_file "T";width 3.5475in;height 2.4189in;depth
0pt;original-width 4.7547in;original-height 3.2335in;cropleft "0";croptop
"1";cropright "1";cropbottom "0";tempfilename
'LYS5MR03.wmf';tempfile-properties "XPR";}}

\subsection{Farsightedness}

Sometimes the cornea-lens system focuses in back of the retina. This is
called farsightedness or hyperopia. It is corrected with a converging lens

\FRAME{dhF}{3.7593in}{2.693in}{0in}{}{}{Figure}{\special{language
"Scientific Word";type "GRAPHIC";maintain-aspect-ratio TRUE;display
"USEDEF";valid_file "T";width 3.7593in;height 2.693in;depth
0in;original-width 3.7118in;original-height 2.6507in;cropleft "0";croptop
"1";cropright "1";cropbottom "0";tempfilename
'LYS5M702.wmf';tempfile-properties "XPR";}}

\subsection{Diopters}

%TCIMACRO{%
%\TeXButton{Question 223.18.4}{\marginpar {
%\hspace{-0.5in}
%\begin{minipage}[t]{1in}
%\small{Question 223.18.4}
%\end{minipage}
%}}}%
%BeginExpansion
\marginpar {
\hspace{-0.5in}
\begin{minipage}[t]{1in}
\small{Question 223.18.4}
\end{minipage}
}%
%EndExpansion
Eye glasses use a different unit of measure to describe how they bend light.
The unit is the \emph{diopter} and it is equal one over the focal length%
\begin{equation}
\text{diopter}=\frac{1}{f\left( \unit{m}\right) }
\end{equation}

\subsection{Color Perception}

The eye detects different colors. The respecters called cones can detect
red, green, and blue light.\FRAME{dhF}{2.4587in}{1.0326in}{0pt}{}{}{Figure}{%
\special{language "Scientific Word";type "GRAPHIC";maintain-aspect-ratio
TRUE;display "USEDEF";valid_file "T";width 2.4587in;height 1.0326in;depth
0pt;original-width 4.3656in;original-height 1.8178in;cropleft "0";croptop
"1";cropright "1";cropbottom "0";tempfilename
'LTUWD66M.wmf';tempfile-properties "XPR";}}The eye combines the red, green,
and blue response to allow us to perceive many different colors.

Most digital cameras also have red, green, and, blue pixels to provide color
to images. The detectors in digital cameras are often have much narrower
frequency bands than the eye. Likewise, television displays and monitors
have red, green, and blue pixels. By targeting the eye receptors, power need
not be wasted in creating light that is not detected well by the eye. The
difference in band width can cause problems in color mixing. Yellow school
busses (perceived as different amounts of green and red light) may be
reddish or green if the bandwidths are chosen poorly.

The science of human visual perception of imagery is called \emph{image
science}. There are many applications for this field, from forensics to
intelligence gathering.

\section{Optical Systems that Magnify}

\subsection{Simple Magnifier}

You may have noticed that, so far, when we say \textquotedblleft
magnification\textquotedblright\ we are defining it in a way that is
different than every-day usage of the word. We defined magnification to be
how big the image is compared to how big the object is. But in every-day
speech, magnification means how big the image is using a lens or optical
system compared to how big it looks without the lens or optical system. We
will call this kind of magnification the \emph{angular magnification.}

We already encountered the simple magnifier when we studied ray diagrams.
Let's use the simple magnifier to define angular magnification. To
understand angular magnification, we can use what we know about easy rays to
draw. Let's draw the rays that go straight through the lens of the eye. If
we pick a ray from the top of our object that goes through the center of the
lens, that ray won't seem to change direction at all. It will hit the retina
to form the top of the image of the object. We can do the same for the
bottom of the object. Then we can see from the next figure \FRAME{dtbpF}{%
2.367in}{0.646in}{0pt}{}{}{Figure}{\special{language "Scientific Word";type
"GRAPHIC";maintain-aspect-ratio TRUE;display "USEDEF";valid_file "T";width
2.367in;height 0.646in;depth 0pt;original-width 2.3263in;original-height
0.6149in;cropleft "0";croptop "1";cropright "1";cropbottom "0";tempfilename
'M46NSV01.wmf';tempfile-properties "XPR";}}that the angle $\theta _{o}$
subtends both the object and the image of the object. If the angle
increases, so does the size of the image on the retina.

\FRAME{fhF}{4.5431in}{1.4804in}{0pt}{}{\Qlb{ViewNoLens}}{Figure}{\special%
{language "Scientific Word";type "GRAPHIC";maintain-aspect-ratio
TRUE;display "USEDEF";valid_file "T";width 4.5431in;height 1.4804in;depth
0pt;original-width 4.6017in;original-height 1.4804in;cropleft "0";croptop
"1";cropright "1";cropbottom "0";tempfilename
'MURUW500.wmf';tempfile-properties "XPR";}} If we move the object closer, $%
\theta _{o}$ increases, and so does the size of the image. When we get to
about $25\unit{cm},$ we reach the limit of the eye for focusing. If we move
the object any closer, it will appear fuzzy. We called this position, the
closest point where we can place an object and still bring it into focus
with our eye, the \emph{near point.} Thus the maximum value of $\theta _{o}$
will be at the near point for unaided viewing.

\FRAME{fhF}{3.8424in}{2.5173in}{0pt}{}{\Qlb{ViewWithLens}}{Figure}{\special%
{language "Scientific Word";type "GRAPHIC";maintain-aspect-ratio
TRUE;display "USEDEF";valid_file "T";width 3.8424in;height 2.5173in;depth
0pt;original-width 3.8867in;original-height 2.5368in;cropleft "0";croptop
"1";cropright "1";cropbottom "0";tempfilename
'MURUWW02.wmf';tempfile-properties "XPR";}}

But suppose we want to see this object in more detail. We can use a
magnifying glass. If we place the object closer to the magnifying glass than
the focal distance ($s<f$), then (lower part of the figure) we have a
virtual image with magnification 
\begin{equation}
m=\frac{-s^{\prime }}{s}
\end{equation}%
which is larger than one and positive (because $s^{\prime }$ is negative).

But what we really want to know is how much bigger the image looks with the
lens than it did without the lens. We define the angular magnification 
\begin{equation}
M=\frac{\theta }{\theta _{o}}
\end{equation}%
This is the ratio of the image sizes with and without the magnifier lens.

This is really different than the magnification we have studied before. The
magnification we have been using compared the size of the image with the
size of the object. So, the angular magnification compares how big the
object seems to be with and without a lens or lens system. We can think of
this as a comparison between the size of the real image on the retina formed
with just our eye, and the one formed with the magnifier.\FRAME{dtbpF}{%
3.4646in}{3.0947in}{0pt}{}{}{Figure}{\special{language "Scientific
Word";type "GRAPHIC";maintain-aspect-ratio TRUE;display "USEDEF";valid_file
"T";width 3.4646in;height 3.0947in;depth 0pt;original-width
4.6185in;original-height 4.1209in;cropleft "0";croptop "1";cropright
"1";cropbottom "0";tempfilename 'N0JL9P01.wmf';tempfile-properties "XPR";}}

If the virtual image formed is farther than the near point of the eye, ($%
s^{\prime }>\symbol{126}25\unit{cm}$) it will seem smaller than it would be
at the near point because it is farther away. If the virtual image is closer
than the near point, it will be fuzzy because the eye cannot focus closer
than the near point. Thus, the value of $M$ will be maximum when $s^{\prime
} $ is at the near point of the eye. We can find where to place the image so
that we get maximum magnification. Taking just the magnifier,%
\begin{eqnarray*}
\frac{1}{s}+\frac{1}{s^{\prime }} &=&\frac{1}{f} \\
\frac{1}{s}+\frac{1}{-25\unit{cm}} &=&\frac{1}{f}
\end{eqnarray*}%
and%
\begin{equation*}
\frac{1}{s}=\frac{-25\unit{cm}-f}{-f\left( 25\unit{cm}\right) }
\end{equation*}%
or%
\begin{equation}
s=\frac{\left( 25\unit{cm}\right) f}{25\unit{cm}+f}
\end{equation}

Returning to figure (\ref{ViewWithLens}). Note that the person has adjusted
her viewpoint so the ray that passes through the middle of the lens also
passes through the middle of her eye lens (cornea). So the angle $\theta $
in the figure is also the angle that subtends the image on her retina. This
was nice of her because it makes our math easier. Using small angle
approximations, we can write%
\begin{equation*}
\tan \theta _{o}=\frac{h}{25\unit{cm}}\approx \theta _{o}
\end{equation*}%
and 
\begin{equation*}
\tan \theta =\frac{h}{s}\approx \theta
\end{equation*}%
then the maximum angular magnification is%
\begin{eqnarray*}
m_{\max } &=&\frac{\theta }{\theta _{o}}=\frac{\frac{h}{s}}{\frac{h}{25\unit{%
cm}}} \\
&=&\frac{25\unit{cm}}{\frac{25\unit{cm}f}{25\unit{cm}+f}} \\
&=&\frac{25\unit{cm}+f}{f} \\
&=&1+\frac{25\unit{cm}}{f}
\end{eqnarray*}

We can also find the minimum magnification by letting $s$ be at $f.$ This
gives 
\begin{equation*}
\theta =\frac{h}{f}
\end{equation*}%
\begin{eqnarray*}
m_{\min } &=&\frac{\theta }{\theta _{o}}=\frac{\frac{h}{f}}{\frac{h}{25\unit{%
cm}}} \\
&=&\frac{25\unit{cm}}{f}
\end{eqnarray*}%
When you use a magnifying glass, notice that you move the lens back and
forth between these extremes until you can see the level of detail you want.

But the idea of a magnifier is more than just seeing the details small
objects. We use the idea of a simple magnifier in combination with other
lenses to make the magnification happen in telescopes, microscopes, and
other instruments that magnify.

\subsection{The Microscope}

To see things that are very small, we add another lens to our simple
magnifier. We will place this lens near the object. Since this new lens is
near the object, let's give it the name \emph{objective lens} or just \emph{%
objective}. We will keep a simple magnifier and place it near the eye. Since
our simple magnifier is near our eye, let's call it the \emph{eyepiece}.

The objective will have a very short focal length. The eyepiece will have a
longer focal length (a few centimeters).\FRAME{dtbpF}{5.2146in}{3.7502in}{0in%
}{}{}{Figure}{\special{language "Scientific Word";type
"GRAPHIC";maintain-aspect-ratio TRUE;display "USEDEF";valid_file "T";width
5.2146in;height 3.7502in;depth 0in;original-width 5.2855in;original-height
3.7927in;cropleft "0";croptop "1";cropright "1";cropbottom "0";tempfilename
'MKN07K03.wmf';tempfile-properties "XPR";}}

We separate the lenses by a distance $L$ where 
\begin{eqnarray*}
L &>&f_{o} \\
L &>&f_{e}
\end{eqnarray*}

We place the object just outside the focal point of the objective. The image
formed by the objective lens is then real and inverted. We use this image as
the object for the eyepiece. The image formed is upright and virtual, but it
looks upside down because the object for the eyepiece (first image for the
objective) is upside down.

The magnification of the first lens is 
\begin{equation*}
m_{o}=\frac{-s_{1}^{\prime }}{s_{1}}\approx -\frac{L}{f_{1}}
\end{equation*}%
because $s_{1}\approx f_{1}.$ and $s_{1}^{\prime }\approx L$ The
magnification of the eyepiece is just that of a simple magnifier when the
object is placed at the focal point $f_{1}$%
\begin{equation*}
m_{e}=\frac{-s_{2}^{\prime }}{s_{2}}\approx \frac{25\unit{cm}}{f_{2}}
\end{equation*}%
The combined magnification is 
\begin{equation}
m=m_{o}m_{e}=-\frac{L}{f_{1}}\frac{25\unit{cm}}{f_{2}}
\end{equation}%
this is the minimum magnification.

\section{Telescopes}

There are two types of telescopes \emph{refracting} and \emph{reflecting}.
We will study refracting telescopes first.

\subsection{Refracting Telescopes}

Like the microscope, we combine two lenses and call one the objective and
the other the eyepiece. The eyepiece again plays the role of a simple
magnifier, magnifying the image produced by the objective.\FRAME{dtbpF}{%
4.8367in}{2.2724in}{0in}{}{}{Figure}{\special{language "Scientific
Word";type "GRAPHIC";maintain-aspect-ratio TRUE;display "USEDEF";valid_file
"T";width 4.8367in;height 2.2724in;depth 0in;original-width
4.9015in;original-height 2.2875in;cropleft "0";croptop "1";cropright
"1";cropbottom "0";tempfilename 'N0JLPQ02.wmf';tempfile-properties "XPR";}}%
We again form a real, inverted image with the objective. We are now looking
at distant objects, so the image distance $s_{o}^{\prime }\approx f_{o}.$ We
use the image from the objective as the object for the eyepiece. The eye
piece forms an upright virtual image (that looks inverted because the object
for the eyepiece is the image from the objective, and the real image from
the objective is inverted). The largest magnification is when the rays exit
the eyepiece parallel to the principal axis. Then the image from the
eyepiece is formed at infinity (but it is very big, so it is easy to see).
This gives a lens separation of $f_{o}+f_{e}$ which will be roughly the
length of the telescope tube.

The angular magnification will be 
\begin{equation}
M=\frac{\theta }{\theta _{o}}
\end{equation}%
where $\theta _{o}$ is the angle subtended by the object at the objective.
That is the angle we would have with no lenses and just our eye, because it
is the angle subtended by the object without the optical system. The angle $%
\theta $ is subtended by the final image at the viewer's eye using the
optical system. Consider $s_{o}$ is very large. We see from the figure that 
\begin{equation}
\tan \theta _{o}=-\frac{h^{\prime }}{f_{o}}
\end{equation}%
and with $s_{o}$ large we can use small angles. 
\begin{equation}
\theta _{o}=-\frac{h^{\prime }}{f_{o}}
\end{equation}

The angle $\theta $ will be the angle formed by rays bent by the lens of the
eye. This angle will be the same as the angle formed by a ray traveling from
the tip of the first image and traveling parallel to the principal axis.
This ray is bent by the objective to pass through $f_{e.}$ Then 
\begin{equation}
\tan \theta =\frac{h^{\prime }}{f_{e}}\approx \theta
\end{equation}%
so

The magnification is then%
\begin{equation}
m=\frac{\theta }{\theta _{o}}=\frac{\frac{h^{\prime }}{f_{e}}}{-\frac{%
h^{\prime }}{f_{o}}}=-\frac{f_{o}}{f_{e}}
\end{equation}

\subsection{Reflecting Telescopes}

Reflecting telescopes use a series of mirrors to replace the objective lens.
Usually, there is an eyepiece that is refractive (although there need not
be, radio frequency telescopes rarely have refractive pieces).\FRAME{dhF}{%
3.6348in}{2.7354in}{0in}{}{}{Figure}{\special{language "Scientific
Word";type "GRAPHIC";maintain-aspect-ratio TRUE;display "USEDEF";valid_file
"T";width 3.6348in;height 2.7354in;depth 0in;original-width
3.5864in;original-height 2.693in;cropleft "0";croptop "1";cropright
"1";cropbottom "0";tempfilename 'LYSIF208.wmf';tempfile-properties "XPR";}}

There are two reasons to build reflective telescopes. The first is that
reflective optics do not suffer from chromatic aberration. The second is
that large mirrors are much easier to make and mount than refractive optics.
The Keck Observatory in Hawaii has a $10\unit{m}$ reflective system. The
largest refractive system is a $1\unit{m}$ system. The Hubble telescope has
a $2.5\unit{m}$ aperture.

The telescope pictured in the figure is a Newtonian, named after Newton, who
designed this focus mechanism. Many other designs exist. Popular designs for
space applications include the Cassegrain telescope.

The rough design of a reflective telescope can be worked out using
refractive pieces, then the rough details of the reflective optics can be
formed.

\chapter{Resolution and Charge}

%TCIMACRO{%
%\TeXButton{Fundamental Concepts}{\hspace{-1.3in}{\LARGE Fundamental Concepts\vspace{0.25in}}}}%
%BeginExpansion
\hspace{-1.3in}{\LARGE Fundamental Concepts\vspace{0.25in}}%
%EndExpansion

\begin{itemize}
\item Two points can be distinguished when imaged if their angular
separation is a minimum of $\theta _{\min }=1.22\frac{\lambda }{D}$

\item There is a property of matter called \textquotedblleft
charge.\textquotedblright

\item There seem to be two types of charges, called \textquotedblleft
positive\textquotedblright\ and \textquotedblleft negative.\textquotedblright
\end{itemize}

\section{Resolution}

We have emphasized that an extended object can be viewed as a collection of
point objects. Then the image is formed from the collection if images of
those point objects. 
%TCIMACRO{%
%\TeXButton{Question 223.19.1}{\marginpar {
%\hspace{-0.5in}
%\begin{minipage}[t]{1in}
%\small{Question 223.19.1}
%\end{minipage}
%}}}%
%BeginExpansion
\marginpar {
\hspace{-0.5in}
\begin{minipage}[t]{1in}
\small{Question 223.19.1}
\end{minipage}
}%
%EndExpansion
It would be great if optical systems could form images with infinite
precision--that is, the image of a point object would be a point image. The
fact that light acts as a wave prevents this from being true. The quality of
our image depends on how poorly a point object is imaged. If each point
object makes a large circle of light on the screen or detector array, we get
a very confusing image (it will look blurry to us).\footnote{%
In Fourier Optics, the intensity pattern that comes from imaging a single
point is called a \emph{point spread function} because it shows how spread
out the light from a single point will be. In mechanical engineering, we
might call this an impulse response function. It is the same idea applied to
optics.} Let's see why this will happen so we can know how to minimize the
effect.

We already know that if we take light and pass it through a single slit, we
get an intensity pattern that has a central bright region.\FRAME{dhF}{%
2.9914in}{2.2191in}{0pt}{}{}{Figure}{\special{language "Scientific
Word";type "GRAPHIC";maintain-aspect-ratio TRUE;display "USEDEF";valid_file
"T";width 2.9914in;height 2.2191in;depth 0pt;original-width
2.9473in;original-height 2.1785in;cropleft "0";croptop "1";cropright
"1";cropbottom "0";tempfilename 'LYZJJU01.wmf';tempfile-properties "XPR";}}

Remember that normal objects will be made up of many small points of light
(either due to reflection or glowing) and each of these will form such an
intensity pattern on a screen. Here is a bright point source that is not on
the axis, and we see that it too makes a bright spot on the screen (and
smaller bright spots or rings, depending on the shape of the aperture) 
\FRAME{dhF}{2.7674in}{2.0513in}{0pt}{}{}{Figure}{\special{language
"Scientific Word";type "GRAPHIC";maintain-aspect-ratio TRUE;display
"USEDEF";valid_file "T";width 2.7674in;height 2.0513in;depth
0pt;original-width 2.725in;original-height 2.0124in;cropleft "0";croptop
"1";cropright "1";cropbottom "0";tempfilename
'LYZJMI02.wmf';tempfile-properties "XPR";}}So our images will be made up of
many central bright spots, each of which represents a point of light from
the object. These central bright spots may overlap, (and their secondary
maxima certainly will overlap).

Let's take a simple case of two points of light, $S_{1}$ and $S_{2}.$ If we
take a single slit and pass light from two distant point sources through the
slit, we do not get two sharp images of the point sources. Instead, we get
two diffraction patterns.\FRAME{dhF}{3.3537in}{2.6507in}{0pt}{}{}{Figure}{%
\special{language "Scientific Word";type "GRAPHIC";maintain-aspect-ratio
TRUE;display "USEDEF";valid_file "T";width 3.3537in;height 2.6507in;depth
0pt;original-width 3.3088in;original-height 2.6091in;cropleft "0";croptop
"1";cropright "1";cropbottom "0";tempfilename
'LYSJUN09.wmf';tempfile-properties "XPR";}}

If these patterns are formed sufficiently far from each other, it is easy to
tell they were formed from two distinct objects. Each point became a small
blur, but that is really not so bad. We can still tell that the two blurs
came from different sources. If our pixel size is about the same size of the
blur, we won't even notice the blurriness in the digital imagery.\FRAME{dhF}{%
3.6487in}{2.8738in}{0pt}{}{}{Figure}{\special{language "Scientific
Word";type "GRAPHIC";maintain-aspect-ratio TRUE;display "USEDEF";valid_file
"T";width 3.6487in;height 2.8738in;depth 0pt;original-width
3.6011in;original-height 2.8305in;cropleft "0";croptop "1";cropright
"1";cropbottom "0";tempfilename 'LYSK6W0A.wmf';tempfile-properties "XPR";}}
But if the patterns are formed close to each other, it gets hard to tell
whether they were formed from two objects or one bright object. We now have
a problem. Suppose you are trying to look at a star and see if it has a
planet. But all you can see is a blur. You can't tell if there is one source
of light or two.

Long ago an early researcher titled Lord Rayleigh developed a test to
determine if you can distinguish between two diffraction patterns.

\begin{quote}
When the central maximum of one point's image falls on the first minimum of
anther point's image, the images are said to be just resolved.
\end{quote}

This test is known as \emph{Rayleigh's criterion}.

We can find the required separation for a slit. Remember that 
\begin{equation}
\sin \left( \theta \right) =m\frac{\lambda }{a}\quad m=\pm 1,\pm 2,\pm
3\ldots
\end{equation}

gives the minima. We want the first minimum, so%
\begin{equation}
\sin \left( \theta \right) =\frac{\lambda }{a}
\end{equation}%
If we place the second image maximum so it is just at this location, the two
images will be just barely resolvable. In the small angle approximation, $%
\sin \left( \theta \right) \approx \theta $ so%
\begin{equation}
\theta _{\min }=\frac{\lambda }{a}
\end{equation}

Now you may be saying to yourself that you don't often take pictures through
single illuminated slits, so this is nice, but not really very interesting.

Suppose, instead, that we image a circular aperture. Again, we won't go
through all the math (there are Bessel functions involved) but the criterion
becomes%
\begin{equation}
\theta _{\min }=1.22\frac{\lambda }{D}
\end{equation}%
where $D$ is the aperture diameter.

Still, you may say, I don't like pictures taken through small circles any
better than through small slits! Yet, in fact, you do. Most cameras have
circular apertures. The light that passes into your phone camera must pass
though the circular lens. For that matter, the pupil of our eye is a
circular aperture. So most images we see are made using circular apertures.

The Rayleigh criteria tells you, based on your camera aperture size, how a
point source will be imaged on the film or sensor array. If we consider
extended sources (like your favorite car or Aunt Matilda) to be collections
of many point sources, then we have a way to tell what features will be
clearly resolved on the image and what features will not (like you may not
be able to see the lettering on the car to tell what model it is, or you may
not be able to distinguish between the gem stones in Aunt Matilda's necklace
because the image is too blurry to see these features clearly).

%TCIMACRO{%
%\TeXButton{Question 223.19.2}{\marginpar {
%\hspace{-0.5in}
%\begin{minipage}[t]{1in}
%\small{Question 223.19.2}
%\end{minipage}
%}}}%
%BeginExpansion
\marginpar {
\hspace{-0.5in}
\begin{minipage}[t]{1in}
\small{Question 223.19.2}
\end{minipage}
}%
%EndExpansion
\FRAME{dtbpFUX}{4.9061in}{1.8663in}{0pt}{\Qcb{{\protect\small Pattern from
two circular sources.}}}{}{Plot}{\special{language "Scientific Word";type
"MAPLEPLOT";width 4.9061in;height 1.8663in;depth 0pt;display
"USEDEF";plot_snapshots TRUE;mustRecompute FALSE;lastEngine "MuPAD";xmin
"-10";xmax "10";xviewmin "-10";xviewmax "10";yviewmin "1.000100E-6";yviewmax
"1.000100";viewset"XY";rangeset"X";plottype 4;axesFont "Times New
Roman,12,0000000000,useDefault,normal";numpoints 100;plotstyle
"patch";axesstyle "normal";axestips FALSE;xis \TEXUX{JQSUB1ESUB};var1name
\TEXUX{$J_{1}$};function \TEXUX{$\frac{1}{\allowbreak 0.25}\left( \text{
}\frac{J_{1}\left( r+3.83\right) }{r+3.83}\right) ^{2}$};linecolor
"maroon";linestyle 1;pointstyle "point";linethickness 1;lineAttributes
"Solid";var1range "-10,10";num-x-gridlines 100;curveColor
"[flat::RGB:0x00800000]";curveStyle "Line";function
\TEXUX{$\frac{1}{\allowbreak 0.25}\left( \text{ }\frac{J_{1}\left(
r-3.83\right) }{r-3.83}\right) ^{2}$};linecolor "blue";linestyle
1;pointstyle "point";linethickness 1;lineAttributes "Solid";var1range
"-10,10";num-x-gridlines 100;curveColor "[flat::RGB:0x000000ff]";curveStyle
"Line";function \TEXUX{$\frac{1}{\allowbreak 0.25}\left( \text{
}\frac{J_{1}\left( r+3.83\right) }{r+3.83}\right) ^{2}+\frac{1}{\allowbreak
0.25}\left( \text{ }\frac{J_{1}\left( r-3.83\right) }{r-3.83}\right)
^{2}$};linecolor "magenta";linestyle 1;pointstyle "point";linethickness
3;lineAttributes "Solid";var1range "-10,10";num-x-gridlines 100;curveColor
"[flat::RGB:0x00800080]";curveStyle "Line";VCamFile
'LTUWDK1H.xvz';valid_file "T";tempfilename
'LTUWD76S.wmf';tempfile-properties "XPR";}}\FRAME{dtbpFUX}{4.9061in}{1.7936in%
}{0pt}{\Qcb{{\protect\small Rayleigh Criteria: Pattern from two circular
soruces where the sources are close enough that the maximum from one pattern
is placed on the minimum of the other. Lord Rayleigh gave this as the
criteria for just barily being resolved.}}}{}{Plot}{\special{language
"Scientific Word";type "MAPLEPLOT";width 4.9061in;height 1.7936in;depth
0pt;display "USEDEF";plot_snapshots TRUE;mustRecompute FALSE;lastEngine
"MuPAD";xmin "-10";xmax "10";xviewmin "-10";xviewmax "10";yviewmin
"1.000100E-6";yviewmax "1.000100";viewset"XY";rangeset"X";plottype
4;axesFont "Times New Roman,12,0000000000,useDefault,normal";numpoints
100;plotstyle "patch";axesstyle "normal";axestips FALSE;xis
\TEXUX{JQSUB1ESUB};var1name \TEXUX{$J_{1}$};function
\TEXUX{$\frac{1}{\allowbreak 0.25}\left( \text{ }\frac{J_{1}\left(
r+2\right) }{r+2}\right) ^{2}$};linecolor "maroon";linestyle 1;pointstyle
"point";linethickness 1;lineAttributes "Solid";var1range
"-10,10";num-x-gridlines 100;curveColor "[flat::RGB:0x00800000]";curveStyle
"Line";function \TEXUX{$\frac{1}{\allowbreak 0.25}\left( \text{
}\frac{J_{1}\left( r-2\right) }{r-2}\right) ^{2}$};linecolor
"blue";linestyle 1;pointstyle "point";linethickness 1;lineAttributes
"Solid";var1range "-10,10";num-x-gridlines 100;curveColor
"[flat::RGB:0x000000ff]";curveStyle "Line";function
\TEXUX{$\frac{1}{\allowbreak 0.25}\left( \text{ }\frac{J_{1}\left(
r+2\right) }{r+2}\right) ^{2}+\frac{1}{\allowbreak 0.25}\left( \text{
}\frac{J_{1}\left( r-2\right) }{r-2}\right) ^{2}$};linecolor
"magenta";linestyle 1;pointstyle "point";linethickness 3;lineAttributes
"Solid";var1range "-10,10";num-x-gridlines 100;curveColor
"[flat::RGB:0x00800080]";curveStyle "Line";VCamFile
'LTUWDK1G.xvz';valid_file "T";tempfilename
'LTUWD76T.wmf';tempfile-properties "XPR";}}\FRAME{dtbpFUX}{4.9061in}{1.7936in%
}{0pt}{\Qcb{Sparrow Criteria: This is a less concervative resolution
criteria than Rayleigh. When the intenisty pattern is flat on the top, there
must be two sources. This criterial is used in astronomy.}}{}{Plot}{\special%
{language "Scientific Word";type "MAPLEPLOT";width 4.9061in;height
1.7936in;depth 0pt;display "USEDEF";plot_snapshots TRUE;mustRecompute
FALSE;lastEngine "MuPAD";xmin "-10";xmax "10";xviewmin "-10";xviewmax
"10";yviewmin "1.000100E-6";yviewmax
"2.000100";viewset"XY";rangeset"X";plottype 4;axesFont "Times New
Roman,12,0000000000,useDefault,normal";numpoints 100;plotstyle
"patch";axesstyle "normal";axestips FALSE;xis \TEXUX{JQSUB1ESUB};var1name
\TEXUX{$J_{1}$};function \TEXUX{$\frac{1}{\allowbreak 0.25}\left( \text{
}\frac{J_{1}\left( r+1.55\right) }{r+1.55}\right) ^{2}$};linecolor
"maroon";linestyle 1;pointstyle "point";linethickness 1;lineAttributes
"Solid";var1range "-10,10";num-x-gridlines 100;curveColor
"[flat::RGB:0x00800000]";curveStyle "Line";function
\TEXUX{$\frac{1}{\allowbreak 0.25}\left( \text{ }\frac{J_{1}\left(
r-1.55\right) }{r-1.55}\right) ^{2}$};linecolor "blue";linestyle
1;pointstyle "point";linethickness 1;lineAttributes "Solid";var1range
"-10,10";num-x-gridlines 100;curveColor "[flat::RGB:0x000000ff]";curveStyle
"Line";function \TEXUX{$\frac{1}{\allowbreak 0.25}\left( \text{
}\frac{J_{1}\left( r+1.55\right) }{r+1.55}\right) ^{2}+\frac{1}{\allowbreak
0.25}\left( \text{ }\frac{J_{1}\left( r-1.55\right) }{r-1.55}\right)
^{2}$};linecolor "magenta";linestyle 1;pointstyle "point";linethickness
3;lineAttributes "Solid";var1range "-10,10";num-x-gridlines 100;curveColor
"[flat::RGB:0x00800080]";curveStyle "Line";VCamFile
'LTUWDK1F.xvz';valid_file "T";tempfilename
'LTUWD76U.wmf';tempfile-properties "XPR";}}\FRAME{dtbpFUX}{4.9061in}{1.7936in%
}{0pt}{\Qcb{{\protect\small Two circular sorces unresolved.}}}{}{Plot}{%
\special{language "Scientific Word";type "MAPLEPLOT";width 4.9061in;height
1.7936in;depth 0pt;display "USEDEF";plot_snapshots TRUE;mustRecompute
FALSE;lastEngine "MuPAD";xmin "-10";xmax "10";xviewmin "-10";xviewmax
"10";yviewmin "1.000100E-6";yviewmax
"2.000100";viewset"XY";rangeset"X";plottype 4;axesFont "Times New
Roman,12,0000000000,useDefault,normal";numpoints 100;plotstyle
"patch";axesstyle "normal";axestips FALSE;xis \TEXUX{JQSUB1ESUB};var1name
\TEXUX{$J_{1}$};function \TEXUX{$\frac{1}{\allowbreak 0.25}\left( \text{
}\frac{J_{1}\left( r+1\right) }{r+1}\right) ^{2}$};linecolor
"maroon";linestyle 1;pointstyle "point";linethickness 1;lineAttributes
"Solid";var1range "-10,10";num-x-gridlines 100;curveColor
"[flat::RGB:0x00800000]";curveStyle "Line";function
\TEXUX{$\frac{1}{\allowbreak 0.25}\left( \text{ }\frac{J_{1}\left(
r-1\right) }{r-1}\right) ^{2}$};linecolor "blue";linestyle 1;pointstyle
"point";linethickness 1;lineAttributes "Solid";var1range
"-10,10";num-x-gridlines 100;curveColor "[flat::RGB:0x000000ff]";curveStyle
"Line";function \TEXUX{$\frac{1}{\allowbreak 0.25}\left( \text{
}\frac{J_{1}\left( r+1\right) }{r+1}\right) ^{2}+\frac{1}{\allowbreak
0.25}\left( \text{ }\frac{J_{1}\left( r-1\right) }{r-1}\right)
^{2}$};linecolor "magenta";linestyle 1;pointstyle "point";linethickness
3;lineAttributes "Solid";var1range "-10,10";num-x-gridlines 100;curveColor
"[flat::RGB:0x00800080]";curveStyle "Line";VCamFile
'LTUWDK1E.xvz';valid_file "T";tempfilename
'LTUWD76V.wmf';tempfile-properties "XPR";}}\FRAME{dtbpF}{3.5613in}{1.0793in}{%
0pt}{}{}{Figure}{\special{language "Scientific Word";type
"GRAPHIC";maintain-aspect-ratio TRUE;display "USEDEF";valid_file "T";width
3.5613in;height 1.0793in;depth 0pt;original-width 3.5146in;original-height
1.0456in;cropleft "0";croptop "1";cropright "1";cropbottom "0";tempfilename
'M4854902.wmf';tempfile-properties "XPR";}}

From our equation we can see that we have better resolution if $D$ is
bigger. This is why professional photographers use large lenses and not
their cell phones. The cell phone cameras have apertures that are a few
millimeters. Typical professional cameras have $67\unit{mm}$ apertures. We
can see that for a cell phone the angle for minimal resolution is about 
\begin{equation*}
\theta _{\min }=1.22\frac{500\unit{nm}}{3\unit{mm}}=\allowbreak
2.\,\allowbreak 033\,3\times 10^{-4}\unit{rad}
\end{equation*}

For the professional lens, the minimum resolution is about 
\begin{equation*}
\theta _{\min }=1.22\frac{500\unit{nm}}{67\unit{mm}}=\allowbreak
9.\,\allowbreak 104\,5\times 10^{-6}\unit{rad}
\end{equation*}
That is a whopping factor of $22$ better resolution. If you need to find a
small crack in a structure, or if you want to print a wall sized portrait of
your Aunt Miltilda, the extra resolution might be necessary for your
application.

\section{Charge model}

\subsection{Light from charges}

So far we have claimed that light is a wave in an electromagnetic field. But
we have not proved it to be so. We will find that it will take the rest of
the semester to do so!

But let's think about this conceptually and see if we can motivate our study.

We know that there is an electromagnetic spectrum, and that visible light is
just a small part of that spectrum. Radio waves are also part of the
spectrum of light.

%TCIMACRO{%
%\TeXButton{Question 223.19.3}{\marginpar {
%\hspace{-0.5in}
%\begin{minipage}[t]{1in}
%\small{Question 223.19.3}
%\end{minipage}
%}}}%
%BeginExpansion
\marginpar {
\hspace{-0.5in}
\begin{minipage}[t]{1in}
\small{Question 223.19.3}
\end{minipage}
}%
%EndExpansion
%TCIMACRO{%
%\TeXButton{PHET Radio Wave Applet}{\marginpar {
%\hspace{-0.5in}
%\begin{minipage}[t]{1in}
%\small{PHET Radio Wave Applet}
%\end{minipage}
%}}}%
%BeginExpansion
\marginpar {
\hspace{-0.5in}
\begin{minipage}[t]{1in}
\small{PHET Radio Wave Applet}
\end{minipage}
}%
%EndExpansion
How are radio waves produced? We know electricity is involved.

The answer is that charged particles, like the electrons flowing through the
antenna of our radio station, create an electromagnetic field. That field is
drug along when the electrons move in the antenna. If we make the electrons
oscillate, we can make waves in the field. This is much like having a 3rd
grade class all hold the edges of a parachute and having the 3rd graders
jump up and down. Waves are made in the parachute.

But what is charge? How do we know there are such things as charged
particles?

That is the subject we will take up next. Then we will study the motion and
actions of these charged particles. Finally we will show that the fields
made by charged particles can act as a medium for waves, and that there is
good evidence that those waves exist.

\subsection{Evidence of Charge}

Let's start with something we all know. Let's rub a balloon in someone's
hair. If we do this we will find that the balloon sticks to the wall. Why?

%TCIMACRO{%
%\TeXButton{Baloon and 2 by 4 demo}{\marginpar {
%\hspace{-0.5in}
%\begin{minipage}[t]{1in}
%\small{Baloon and 2 by 4 demo}
%\end{minipage}
%}}}%
%BeginExpansion
\marginpar {
\hspace{-0.5in}
\begin{minipage}[t]{1in}
\small{Baloon and 2 by 4 demo}
\end{minipage}
}%
%EndExpansion
%TCIMACRO{%
%\TeXButton{Balloon on wall demo}{\marginpar {
%\hspace{-0.5in}
%\begin{minipage}[t]{1in}
%\small{Balloon on wall demo}
%\end{minipage}
%}}}%
%BeginExpansion
\marginpar {
\hspace{-0.5in}
\begin{minipage}[t]{1in}
\small{Balloon on wall demo}
\end{minipage}
}%
%EndExpansion
%TCIMACRO{%
%\TeXButton{Comb and bits of paper demo}{\marginpar {
%\hspace{-0.5in}
%\begin{minipage}[t]{1in}
%\small{Comb and bits of paper demo}
%\end{minipage}
%}}}%
%BeginExpansion
\marginpar {
\hspace{-0.5in}
\begin{minipage}[t]{1in}
\small{Comb and bits of paper demo}
\end{minipage}
}%
%EndExpansion
We say the balloon and comb have become \emph{charged}. What does this mean?
We will have to investigate this more as we learn more about how matter is
structured, but for now let's assume charge is some property that provides
this phenomena we have observed with the balloon (i.e. it sticks to the
wall). Now lets try rubbing other things. We could rub two rubber or plastic
rods.

%TCIMACRO{%
%\TeXButton{Glass and Rubber Rod Demo}{\marginpar {
%\hspace{-0.5in}
%\begin{minipage}[t]{1in}
%\small{Glass and Rubber Rod Demo}
%\end{minipage}
%}}}%
%BeginExpansion
\marginpar {
\hspace{-0.5in}
\begin{minipage}[t]{1in}
\small{Glass and Rubber Rod Demo}
\end{minipage}
}%
%EndExpansion

\FRAME{dhFU}{3.2309in}{1.3171in}{0pt}{\Qcb{Two charged rubber rods are
placed close together. The rods repel each other.}}{\Qlb{rubber_rods}}{Figure%
}{\special{language "Scientific Word";type "GRAPHIC";maintain-aspect-ratio
TRUE;display "USEDEF";valid_file "T";width 3.2309in;height 1.3171in;depth
0pt;original-width 9.6081in;original-height 3.896in;cropleft "0";croptop
"1";cropright "1";cropbottom "0";tempfilename
'LTUWD76W.wmf';tempfile-properties "XPR";}}and we could also rub two glass
rods

\FRAME{dhF}{3.4255in}{1.3673in}{0pt}{}{}{Figure}{\special{language
"Scientific Word";type "GRAPHIC";maintain-aspect-ratio TRUE;display
"USEDEF";valid_file "T";width 3.4255in;height 1.3673in;depth
0pt;original-width 9.6081in;original-height 3.8121in;cropleft "0";croptop
"1";cropright "1";cropbottom "0";tempfilename
'LTUWD76X.wmf';tempfile-properties "XPR";}}Notice that in each case we have
created a force between the two rods. The rods now repel each other.

Now let's try a glass and a rubber rod\FRAME{dhF}{3.6227in}{1.4633in}{0pt}{}{%
}{Figure}{\special{language "Scientific Word";type
"GRAPHIC";maintain-aspect-ratio TRUE;display "USEDEF";valid_file "T";width
3.6227in;height 1.4633in;depth 0pt;original-width 3.9202in;original-height
1.5679in;cropleft "0";croptop "1";cropright "1";cropbottom "0";tempfilename
'LTUWD76Y.wmf';tempfile-properties "XPR";}}Now the two different rods
attract each other.

Notice that in our demo, rods that are the same repel and rods that are
different attract. We make the intellectual leap that the different rods
have different charges. So we are really saying:

\begin{enumerate}
\item There are two types of charge.

\item Charges that are the same repel one another and charges that are
different

attract one another.

\item Friction seems to produce charge, but you have to rub the right
materials

together.
\end{enumerate}

We will call the rubber or plastic rod charges \emph{negative} and the glass
rod charges \emph{positive} but the choice is arbitrary. Ben Franklin is
credited with making the choice of names. He really did not know much about
charge, so he just picked two names (we will see that in some ways his
choice was somewhat unfortunate, but hay, he was an early researcher who
helped us understand much about charge , so we will give him a break!).

\subsection{Types of Charge}

We now have reason to believe that there are at least two types of charge,
one for rubber and one for glass. But are there more?

Let's start by introducing a new object, only this time we won't rub it with
anything.%
%TCIMACRO{%
%\TeXButton{No-rubbing demo}{\marginpar {
%\hspace{-0.5in}
%\begin{minipage}[t]{1in}
%\small{No-rubbing demo}
%\end{minipage}
%}}}%
%BeginExpansion
\marginpar {
\hspace{-0.5in}
\begin{minipage}[t]{1in}
\small{No-rubbing demo}
\end{minipage}
}%
%EndExpansion

Now this is strange. The new item is attracted to both rods! What is going
on? Have we discovered a new type of charge, one that attracts the other two
types we have found?

Maybe, but maybe the explanation of this phenomena is a little different. To
understand this, let's consider how charge moves around.%
%TCIMACRO{%
%\TeXButton{Question 223.19.4}{\marginpar {
%\hspace{-0.5in}
%\begin{minipage}[t]{1in}
%\small{Question 223.19.4}
%\end{minipage}
%}}}%
%BeginExpansion
\marginpar {
\hspace{-0.5in}
\begin{minipage}[t]{1in}
\small{Question 223.19.4}
\end{minipage}
}%
%EndExpansion
%TCIMACRO{%
%\TeXButton{Question 223.19.5}{\marginpar {
%\hspace{-0.5in}
%\begin{minipage}[t]{1in}
%\small{Question 223.19.5}
%\end{minipage}
%}}}%
%BeginExpansion
\marginpar {
\hspace{-0.5in}
\begin{minipage}[t]{1in}
\small{Question 223.19.5}
\end{minipage}
}%
%EndExpansion

\subsection{Movement of Charge}

One of the strange things about charge is that it is \emph{quantized}. We
learned this word in when we found that only certain standing waves could be
formed between boundaries. We are using this word in a similar way now. It
means that charge has a smallest unit, and that it only comes in whole
number multiples of that unit. Charge comes in a basic amount that can't be
divided into smaller amounts. So like our standing wave frequencies, only
certain amounts are possible As far as we know, the smallest amount of
charge possible is the electron charge.\footnote{%
I am not counting quarks here, which have a charge of $\frac{1}{3}$ or $%
\frac{2}{3}$ of the basic electron charge. But still, $\frac{1}{3}$ of the
basic electron charge seems to be a real fundamental unit.} This charge we
will call negative. We say that the electron is the principle charge carrier
for negative charge. This fundamental unit of charge was found to be about%
\begin{equation}
e=1.60219\times 10^{-19}\unit{C}
\end{equation}%
where the $\unit{C}$ stands for \emph{Coulomb, }the $SI$ unit of charge.

Any larger charge must be a multiple of this fundamental charge%
\begin{equation}
Q=n\times e
\end{equation}

The proton is the principle charge carrier for positive charge. From
chemistry, you know protons are located in the nucleus of an atom, along
with the neutron. In the Bohr model of the atom, the nucleus is surrounded
by a cloud of electrons. The proton has the same amount charge as the
electron ($e$), but is opposite in sign.

In a gram of mater, there are many, many, units of charge. There are about $%
\allowbreak 5.\,\allowbreak 012\,5\times 10^{22}$ carbon atoms in one gram
of carbon. Each carbon atom has twelve protons and about twelve electrons.
That is a lot of charge! But notice that the net charge is zero (or very
close to it!). It is common for most mater to have zero net charge.

As far as we know, charge is always conserved. We can create charge, but
only in plus or minus pairs, so the net charge does not change. We can
destroy charge, but we end up destroying both a positive and a negative
charge at the same time. The net charge in the universe does not seem to
change much. So when something becomes charged, we expect to find that the
charge has come from another object.

Lets go back to our rubber rod and glass rod demo. We rubbed the rod that
was in our hand, but where did the charge come from? We believe that we are
moving charge carriers (usually electrons) from one object to another,
stripping them from their atoms. This happens when we use friction (rubbing)
to charge the rods.

But what about our object that we did not rub, or our paper (we did not rub
the bits of paper). We believe that charge can move, that is why scientists
looked for and found charge carriers. Even in an atom, if I\ bring a charged
object near the atom then the negative charge carriers (electrons) will
experience a force directed away from the charged object, and the positively
charged nucleus will experience a force pulling toward the charge object 
\FRAME{dhF}{1.695in}{0.9556in}{0pt}{}{}{Figure}{\special{language
"Scientific Word";type "GRAPHIC";maintain-aspect-ratio TRUE;display
"USEDEF";valid_file "T";width 1.695in;height 0.9556in;depth
0pt;original-width 9.7395in;original-height 5.476in;cropleft "0";croptop
"1";cropright "1";cropbottom "0";tempfilename
'LTUWD86Z.wmf';tempfile-properties "XPR";}}Notice that the electrons and the
nucleus will \emph{attract} each other, so the atom won't split apart. But
it will become positively charged on one side because there are more
positive charge carriers on that side. It will become more negatively
charged on the other side, because there are more negative charge carriers
on that side. We could draw the atom like this (figure \ref{Polarized_Atom})%
\FRAME{fhFU}{0.8761in}{0.4557in}{0pt}{\Qcb{Polarized Atom}}{\Qlb{%
Polarized_Atom}}{Figure}{\special{language "Scientific Word";type
"GRAPHIC";maintain-aspect-ratio TRUE;display "USEDEF";valid_file "T";width
0.8761in;height 0.4557in;depth 0pt;original-width 3.4428in;original-height
1.7798in;cropleft "0";croptop "1";cropright "1";cropbottom "0";tempfilename
'LTUWD870.wmf';tempfile-properties "XPR";}}The force due to charge depends
on how far away the charges are from each other. The attractive force
between the positively charged side of the atom and the negative rod will
have a stronger force than the negatively charged side of the atom and
negatively charged rod will experience because the negative side if farther
away. We will say that the atom has become \emph{polarized}. \FRAME{dhF}{%
2.9464in}{0.5621in}{0pt}{}{}{Figure}{\special{language "Scientific
Word";type "GRAPHIC";maintain-aspect-ratio TRUE;display "USEDEF";valid_file
"T";width 2.9464in;height 0.5621in;depth 0pt;original-width
9.4688in;original-height 1.7798in;cropleft "0";croptop "1";cropright
"1";cropbottom "0";tempfilename 'LTUWD871.wmf';tempfile-properties "XPR";}}%
The positive side will experience an attractive force. The negative side
will experience a repelling force. The net force due to the charge will be
an attractive force. The atom will be accelerated toward the rod! We have
seen something like this before. Remember an object in a fluid experiences a
downward pressure force on the top, and an upward pressure force on the
bottom. The pressure force is larger on the bottom, so there is an upward
Buoyant force. The case with our polarized atom is very similar. We have a
net electrical attractive force.\FRAME{dtbpF}{2.7873in}{0.5509in}{0pt}{}{}{%
Figure}{\special{language "Scientific Word";type
"GRAPHIC";maintain-aspect-ratio TRUE;display "USEDEF";valid_file "T";width
2.7873in;height 0.5509in;depth 0pt;original-width 9.1073in;original-height
1.7798in;cropleft "0";croptop "1";cropright "1";cropbottom "0";tempfilename
'LTUWD872.wmf';tempfile-properties "XPR";}}Now suppose we have lots of atoms
(like our uncharged object or our bits of paper). Will they be attracted to
the rod? Yes!

How about if we use a glass rod?

\FRAME{dhF}{2.4924in}{0.5068in}{0pt}{}{}{Figure}{\special{language
"Scientific Word";type "GRAPHIC";maintain-aspect-ratio TRUE;display
"USEDEF";valid_file "T";width 2.4924in;height 0.5068in;depth
0pt;original-width 8.8574in;original-height 1.7798in;cropleft "0";croptop
"1";cropright "1";cropbottom "0";tempfilename
'LTUWD873.wmf';tempfile-properties "XPR";}}Everything is the same, only we
switch the signs. The glass rod is positively charged. It will attract the
electrons, and repel the nucleus. The atom becomes charged. The net force is
attractive (positive rod and closer negative side of the atom)

We sometimes call the separation of charge in an insulator \emph{polarization%
}.

\subsection{Flow of Charge}

Let's start by introducing a new object, a salt shaker (my salt shaker is
glass with a metal top). We will rub the salt shaker and see if it gets
charged by placing it next to our charged rods.%
%TCIMACRO{%
%\TeXButton{Salt Shaker Demo}{\marginpar {
%\hspace{-0.5in}
%\begin{minipage}[t]{1in}
%\small{Salt Shaker Demo}
%\end{minipage}
%}}}%
%BeginExpansion
\marginpar {
\hspace{-0.5in}
\begin{minipage}[t]{1in}
\small{Salt Shaker Demo}
\end{minipage}
}%
%EndExpansion

Now this is strange. We rubbed the object, but it was attracted to both rods
as if there were no charge. We know glass can be charged. What is the
problem?

%TCIMACRO{%
%\TeXButton{Question 223.19.6}{\marginpar {
%\hspace{-0.5in}
%\begin{minipage}[t]{1in}
%\small{Question 223.19.6}
%\end{minipage}
%}}}%
%BeginExpansion
\marginpar {
\hspace{-0.5in}
\begin{minipage}[t]{1in}
\small{Question 223.19.6}
\end{minipage}
}%
%EndExpansion
It turns out that some materials allow charge carriers to flow through them.
Our experience with the lighting in our house might suggest that metals will
do this. Let's try some other metal objects and see what we find.%
%TCIMACRO{%
%\TeXButton{Metal Demo}{\marginpar {
%\hspace{-0.5in}
%\begin{minipage}[t]{1in}
%\small{Metal Demo}
%\end{minipage}
%}}}%
%BeginExpansion
\marginpar {
\hspace{-0.5in}
\begin{minipage}[t]{1in}
\small{Metal Demo}
\end{minipage}
}%
%EndExpansion

It seems that the atoms are not maintaining a charge separation in these
metal atoms! Some materials allow charge carriers to move through them.
Usually these materials are metals, but most materials will allow some
charge to go through them-even you-which is what is happening in this case.
I\ charge the rod, but the charge leaves through my body. Other materials
resist the flow of charge. Materials that allow charge to flow are called 
\emph{conductors}. Materials that resist the flow of charge are called \emph{%
insulators}.

\subsection{Charging by Induction}

Knowing that charge carriers can flow though a material, we can think of a
way to charge a conductor. Lets suspend a conducting rod.

\FRAME{dhF}{2.3471in}{0.5933in}{0pt}{}{}{Figure}{\special{language
"Scientific Word";type "GRAPHIC";maintain-aspect-ratio TRUE;display
"USEDEF";valid_file "T";width 2.3471in;height 0.5933in;depth
0pt;original-width 4.4304in;original-height 1.0974in;cropleft "0";croptop
"1";cropright "1";cropbottom "0";tempfilename
'LTUWD874.wmf';tempfile-properties "XPR";}}It is not initially "charged"
meaning that it has the same number of positive charges and negative
charges, and they are evenly mixed together. I will bring a charged rod next
to it.

\FRAME{dhF}{3.7109in}{0.4765in}{0pt}{}{}{Figure}{\special{language
"Scientific Word";type "GRAPHIC";maintain-aspect-ratio TRUE;display
"USEDEF";valid_file "T";width 3.7109in;height 0.4765in;depth
0pt;original-width 9.436in;original-height 1.1805in;cropleft "0";croptop
"1";cropright "1";cropbottom "0";tempfilename
'LTUWD875.wmf';tempfile-properties "XPR";}}but let's attach a wire to the
other end of the rod to allow the charge to flow away from our conducting
rod. We will connect the rod to the ground (in this case, to a water pipe)
because the ground seems to be able to accept large amounts of charge
carriers. So the charge carriers will flow to the ground.

\FRAME{fhF}{3.7567in}{0.5604in}{0pt}{}{}{Figure}{\special{language
"Scientific Word";type "GRAPHIC";maintain-aspect-ratio TRUE;display
"USEDEF";valid_file "T";width 3.7567in;height 0.5604in;depth
0pt;original-width 9.9358in;original-height 1.4512in;cropleft "0";croptop
"1";cropright "1";cropbottom "0";tempfilename
'LTUWD876.wmf';tempfile-properties "XPR";}}(The strange little triangular
striped thing is the electronics sign for a connection to the ground)

Now let's disconnect the wire from the rod. Is there a net charge on the
conducting rod?\FRAME{dphF}{2.1024in}{0.4514in}{0pt}{}{}{Figure}{\special%
{language "Scientific Word";type "GRAPHIC";maintain-aspect-ratio
TRUE;display "USEDEF";valid_file "T";width 2.1024in;height 0.4514in;depth
0pt;original-width 4.4304in;original-height 0.9305in;cropleft "0";croptop
"1";cropright "1";cropbottom "0";tempfilename
'LTUWD877.wmf';tempfile-properties "XPR";}}The answer is yes, because we now
have more positive charges in the conducting rod than we have negative
charges, so the net charge is positive.

\subsection{Charging by Conduction}

Suppose instead, I perform the same experiment, but I\ touch the rods. Now
charge carriers can flow. Starting with and uncharged conductor,

\FRAME{dhF}{2.0781in}{0.5241in}{0pt}{}{}{Figure}{\special{language
"Scientific Word";type "GRAPHIC";maintain-aspect-ratio TRUE;display
"USEDEF";valid_file "T";width 2.0781in;height 0.5241in;depth
0pt;original-width 4.4304in;original-height 1.0974in;cropleft "0";croptop
"1";cropright "1";cropbottom "0";tempfilename
'LTUWD878.wmf';tempfile-properties "XPR";}}I again bring in a charged rod.
Again the charges separate in our conducting rod.

\FRAME{dhF}{4.0318in}{0.5172in}{0pt}{}{}{Figure}{\special{language
"Scientific Word";type "GRAPHIC";maintain-aspect-ratio TRUE;display
"USEDEF";valid_file "T";width 4.0318in;height 0.5172in;depth
0pt;original-width 9.436in;original-height 1.1805in;cropleft "0";croptop
"1";cropright "1";cropbottom "0";tempfilename
'LTUWD879.wmf';tempfile-properties "XPR";}}Then we touch the two rods. The
excess charge on our charged rod flows to the conductor. Since in our
drawing, the excess charge is negative, then some of the positive charge on
the conductor is neutralized.

\FRAME{dhF}{3.8674in}{0.4895in}{0pt}{}{}{Figure}{\special{language
"Scientific Word";type "GRAPHIC";maintain-aspect-ratio TRUE;display
"USEDEF";valid_file "T";width 3.8674in;height 0.4895in;depth
0pt;original-width 8.8522in;original-height 1.0974in;cropleft "0";croptop
"1";cropright "1";cropbottom "0";tempfilename
'LTUWD97A.wmf';tempfile-properties "XPR";}}%
%TCIMACRO{%
%\TeXButton{Scotch Tape Assignment}{\marginpar {
%\hspace{-0.5in}
%\begin{minipage}[t]{1in}
%\small{Take home lab assignment (using Scotch Brand Tape)}
%\end{minipage}
%}}}%
%BeginExpansion
\marginpar {
\hspace{-0.5in}
\begin{minipage}[t]{1in}
\small{Take home lab assignment (using Scotch Brand Tape)}
\end{minipage}
}%
%EndExpansion
When we separate the rods, our conducting rod will have an excess of
negative charge.\FRAME{dhF}{1.9649in}{0.4583in}{0pt}{}{}{Figure}{\special%
{language "Scientific Word";type "GRAPHIC";maintain-aspect-ratio
TRUE;display "USEDEF";valid_file "T";width 1.9649in;height 0.4583in;depth
0pt;original-width 4.4304in;original-height 1.0136in;cropleft "0";croptop
"1";cropright "1";cropbottom "0";tempfilename
'LTUWD97B.wmf';tempfile-properties "XPR";}}

Notice that there is something different in our study of this new force. In
the past, it was easy to tell which object was creating the environment and
which was the mover. The Earth, being so much larger than normal objects,
was the environmental object creating the gravitational acceleration that
balls and cars and people move it. Then the balls and cars and people were
the movers. Generally the thing causing the force, the environmental object,
was much bigger than the mover. That is not true in our charge experiments
so far. The rods are about the same size. So which is the environmental
object and which is the mover? We will have to pick one to be our
environmental object, and the other to be our mover. Sometimes the context
of the problem helps. If the problem you are solving asks for the motion or
the force on the rod on the right side of the diagram, then it is the mover
and the rod on the left is the environmental object. If one charge is much
larger than the other, we might be justified in calling this large charge
the environmental object and a smaller charge near the big charge would be
the mover.

%TCIMACRO{%
%\TeXButton{Basic Equations}{\hspace{-1.3in}{\LARGE Basic Equations\vspace{0.25in}}}}%
%BeginExpansion
\hspace{-1.3in}{\LARGE Basic Equations\vspace{0.25in}}%
%EndExpansion

The minimum angle between two objects that can be resolved (according to the
Rayleigh criteria) is 
\begin{equation*}
\theta _{\min }=1.22\frac{\lambda }{D}
\end{equation*}

\chapter{Electric charge}

%TCIMACRO{%
%\TeXButton{Fundamental Concepts}{\hspace{-1.3in}{\LARGE Fundamental Concepts\vspace{0.25in}}}}%
%BeginExpansion
\hspace{-1.3in}{\LARGE Fundamental Concepts\vspace{0.25in}}%
%EndExpansion

\begin{itemize}
\item We have a model for how charge acts. The model tells us there are two
types of charge, and that charges of similar type repel and charges of
different type attract.

\item We call the types of charge \textquotedblleft
positive\textquotedblright\ and \textquotedblleft negative\textquotedblright

\item In metals, the valence electrons are free to move around. We call
materials where the charges move \textquotedblleft
conductors.\textquotedblright

\item Materials where the valence electrons cannot move are called
\textquotedblleft insulators.\textquotedblright

\item In insulators, the atoms can \textquotedblleft
polarize.\textquotedblright
\end{itemize}

\section{Charge}

%TCIMACRO{%
%\TeXButton{Question 223.20.1}{\marginpar {
%\hspace{-0.5in}
%\begin{minipage}[t]{1in}
%\small{Question 223.20.1}
%\end{minipage}
%}}}%
%BeginExpansion
\marginpar {
\hspace{-0.5in}
\begin{minipage}[t]{1in}
\small{Question 223.20.1}
\end{minipage}
}%
%EndExpansion
%TCIMACRO{%
%\TeXButton{Question 223.20.2}{\marginpar {
%\hspace{-0.5in}
%\begin{minipage}[t]{1in}
%\small{Question 223.20.2}
%\end{minipage}
%}}}%
%BeginExpansion
\marginpar {
\hspace{-0.5in}
\begin{minipage}[t]{1in}
\small{Question 223.20.2}
\end{minipage}
}%
%EndExpansion
%TCIMACRO{%
%\TeXButton{Question 223.20.3}{\marginpar {
%\hspace{-0.5in}
%\begin{minipage}[t]{1in}
%\small{Question 223.20.3}
%\end{minipage}
%}}}%
%BeginExpansion
\marginpar {
\hspace{-0.5in}
\begin{minipage}[t]{1in}
\small{Question 223.20.3}
\end{minipage}
}%
%EndExpansion
%TCIMACRO{%
%\TeXButton{Question 223.20.4}{\marginpar {
%\hspace{-0.5in}
%\begin{minipage}[t]{1in}
%\small{Question 223.20.4}
%\end{minipage}
%}}}%
%BeginExpansion
\marginpar {
\hspace{-0.5in}
\begin{minipage}[t]{1in}
\small{Question 223.20.4}
\end{minipage}
}%
%EndExpansion
Let's summarize what we tried to learn last time:

\begin{equation*}
\begin{tabular}{|c|}
\hline
{\large Model for Charge} \\ \hline
{\small Frictional forces can add or remove charge from an object} \\ \hline
{\small There are two, and only two kinds of charge} \\ \hline
{\small Two objects with the same kind of charge repel each other} \\ \hline
{\small To objects with different kinds of charge attract each other} \\ 
\hline
{\small The force between two charged objects is long ranged} \\ \hline
{\small The force between two charged objects decreases with distance} \\ 
\hline
{\small Uncharged objects have an equal mix of both kinds of charge} \\ 
\hline
\begin{tabular}{l}
{\small There are two types of materials, conductors (in which charges can
move)} \\ 
{\small and insulators (in which charges are fixed in place)}%
\end{tabular}
\\ \hline
{\small Charge can be transferred from one object to another by contact
between the two objects} \\ \hline
\end{tabular}%
\end{equation*}

A serious shortcoming of this model is that it does not tell us what charge
is. This is a shortcoming we will have to live with. We don't know what
charge is any more than we can say exactly what mass or energy are. Charge
is fundamental, as far as we can tell. We can't find a way to change charge
into something else to change something else into charge. For fundamental
particles (like protons and electrons) either a particle has charge, or it
does not.

\subsection{Conservation of charge}

In some ways, this is really great! We have a new quantity that does not
ever change. We can say that charge is conserved in the universe. Like
energy, we can move charge around, but we don't create or destroy it. When
we rubbed the plastic rods with rabbit fur or wool, we were removing charge
that was already there in the atoms of the fur. If you take PH279 you might
find that there are some caveats to this rule. We can make positron and
electron pairs from high energy gamma rays. But when we do this we must
always make a pair; one positive, and one negative. So the net charge
remains unaffected.

\section{Insulators and Conductors}

%TCIMACRO{%
%\TeXButton{Question 223.20.5}{\marginpar {
%\hspace{-0.5in}
%\begin{minipage}[t]{1in}
%\small{Question 223.20.5}
%\end{minipage}
%}}}%
%BeginExpansion
\marginpar {
\hspace{-0.5in}
\begin{minipage}[t]{1in}
\small{Question 223.20.5}
\end{minipage}
}%
%EndExpansion
Let's return to charges and atoms. We have an intuitive feeling for what is
a conductor and what is an insulator, but let's see why conductors act the
way they do.

\subsection{Potential Diagrams for Molecules}

Back in high school or in a collage chemistry class you learned that
electrons move around an atom. \FRAME{dhF}{2.3376in}{1.6907in}{0pt}{}{}{%
Figure}{\special{language "Scientific Word";type
"GRAPHIC";maintain-aspect-ratio TRUE;display "USEDEF";valid_file "T";width
2.3376in;height 1.6907in;depth 0pt;original-width 3.9617in;original-height
2.8591in;cropleft "0";croptop "1";cropright "1";cropbottom "0";tempfilename
'LTUWD97C.wmf';tempfile-properties "XPR";}}In the figure there are two
energy states represented. You may even remember the names of these energy
states. The orange-yellow lines show one \textquotedblleft orbital
distance\textquotedblright\ for the electrons near the nucleus. The red line
shows another electron at a larger orbital distance. The inner orbital is a $%
1s$ state and the outer orbital is a $2s$ state. If these were satellites
orbiting the earth, you would recognize that the two orbits have different
amounts of potential energy. This is also true for electrons in orbitals. If
we plot the potential energy for each state we get something that looks like
this\FRAME{dhF}{2.066in}{1.5912in}{0pt}{}{}{Figure}{\special{language
"Scientific Word";type "GRAPHIC";maintain-aspect-ratio TRUE;display
"USEDEF";valid_file "T";width 2.066in;height 1.5912in;depth
0pt;original-width 7.0768in;original-height 5.4405in;cropleft "0";croptop
"1";cropright "1";cropbottom "0";tempfilename
'LTUWDA7D.wmf';tempfile-properties "XPR";}}You can think of this as
potential energy \textquotedblleft shelves\textquotedblright\ where we can
put electrons. If you were a advanced high school student, you learned that
on the first two shelves you can only fit two electrons each. The higher
shelves can take six, and so forth. But that won't concern us in this class.

\subsection{Building a solid}

So far I have really only talked about single atoms. What happens when we
bind atoms together? Let's take two identical atoms. When they are far
apart, they act as independent systems. But when they get closer, they start
acting like one quantum mechanical system. What does that mean for the
electrons in the atoms?

Electrons are funny things. They won't occupy the exactly the same energy
state. I can only have two electrons in a $1s$ state, but as I\ bring two
atoms near each other I\ will have four! How does the compound solve this
problem? The energy \textquotedblleft shelves\textquotedblright\ split into
more shelves. As the atoms get closer, we see something like this\FRAME{dtbpF%
}{2.7216in}{2.1119in}{0pt}{}{}{Figure}{\special{language "Scientific
Word";type "GRAPHIC";maintain-aspect-ratio TRUE;display "USEDEF";valid_file
"T";width 2.7216in;height 2.1119in;depth 0pt;original-width
5.2693in;original-height 4.0836in;cropleft "0";croptop "1";cropright
"1";cropbottom "0";tempfilename 'LTUWDA7E.wmf';tempfile-properties "XPR";}}%
At some distance, $r,$ the states split. So each electron is now in a
different state. Suppose we bring $5$ atoms together.\FRAME{dtbpF}{3.1488in}{%
2.444in}{0pt}{}{}{Figure}{\special{language "Scientific Word";type
"GRAPHIC";maintain-aspect-ratio TRUE;display "USEDEF";valid_file "T";width
3.1488in;height 2.444in;depth 0pt;original-width 5.2693in;original-height
4.0836in;cropleft "0";croptop "1";cropright "1";cropbottom "0";tempfilename
'LTUWDA7F.wmf';tempfile-properties "XPR";}}I get additional splitting of
states. Now I have five different $1s$ states, enough for $5$ atoms worth of 
$1s$ electrons. But solids have more than five atoms. Let's bring many atoms
together.\FRAME{dtbpF}{3.0338in}{2.3549in}{0pt}{}{}{Figure}{\special%
{language "Scientific Word";type "GRAPHIC";maintain-aspect-ratio
TRUE;display "USEDEF";valid_file "T";width 3.0338in;height 2.3549in;depth
0pt;original-width 5.2693in;original-height 4.0836in;cropleft "0";croptop
"1";cropright "1";cropbottom "0";tempfilename
'LTUWDA7G.wmf';tempfile-properties "XPR";}}Now there are so many states that
we just have a blue blur in between the original two split states. We have
created a nearly continuous set of states in two bands. Each electron has a
different energy, but those energy differences might be tiny fractions of a
Joule. The former two states have almost become continuous bands of allowed
energy states.

The atoms won't allow themselves to be too close. They will reach an
equilibrium distance, $r_{o}$ where they will want to stay. \FRAME{dtbpF}{%
2.9092in}{2.3791in}{0pt}{}{}{Figure}{\special{language "Scientific
Word";type "GRAPHIC";maintain-aspect-ratio TRUE;display "USEDEF";valid_file
"T";width 2.9092in;height 2.3791in;depth 0pt;original-width
5.2693in;original-height 4.3059in;cropleft "0";croptop "1";cropright
"1";cropbottom "0";tempfilename 'LTUWDA7H.wmf';tempfile-properties "XPR";}}%
Since this is where the atoms usually are. We will not draw the whole
diagram anymore. We will instead just draw bands at $r_{o}.$(along the
dotted line). Here is an example.

\FRAME{dtbpF}{2.5503in}{2.7138in}{0pt}{}{}{Figure}{\special{language
"Scientific Word";type "GRAPHIC";maintain-aspect-ratio TRUE;display
"USEDEF";valid_file "T";width 2.5503in;height 2.7138in;depth
0pt;original-width 4.5403in;original-height 4.8317in;cropleft "0";croptop
"1";cropright "1";cropbottom "0";tempfilename
'LTUWDA7I.wmf';tempfile-properties "XPR";}}

This means we have \emph{bands} of energies that are allowed, that electrons
can use, and \emph{gaps} of energy where no electron can exist.

\section{Conduction in solids}

Notice that in our last picture, the $3s$ and $3p$ bands have grown so much
that they overlap. The situation with solids is complicated. Notice also
that the lower states are blue. We will let blue mean that they are filled
with electrons taking up every available energy state. The upper states are
only partially filled. Yellow will mean the energy states are empty. We will
call the highest completely filled band the \emph{valance band} and the next
higher empty band the $\emph{conduction}$ band.

We have three different conditions possible.

\subsection{Metals}

In a metal, the highest occupied band is only partially filled\FRAME{dtbpF}{%
2.3091in}{0.7178in}{0pt}{}{}{Figure}{\special{language "Scientific
Word";type "GRAPHIC";maintain-aspect-ratio TRUE;display "USEDEF";valid_file
"T";width 2.3091in;height 0.7178in;depth 0pt;original-width
3.5639in;original-height 1.0888in;cropleft "0";croptop "1";cropright
"1";cropbottom "0";tempfilename 'LTUWDA7J.wmf';tempfile-properties "XPR";}}%
the electrons in this band require only very little energy to jump to the
next state up since they are in the same band and the allowed energies are
very closely spaced. Remember that movement requires energy. So if I connect
a battery to provide energy, the electrons must be allowed to gain the extra
energy, kinetic energy in this case, or they will not move. But in the case
of a metal, there are easily accessible energy states, and the electrons
flow through the metal.

We can say that the outer electrons are shared by all the atoms of the
entire metal, so the electrons are easy to move for metals.

\subsection{Insulators}

A second condition is to have a full valance band and an empty conduction
band. The bands are separated by an energy gap of energy $E_{g}.$

\FRAME{dtbpF}{1.6535in}{1.7244in}{0pt}{}{}{Figure}{\special{language
"Scientific Word";type "GRAPHIC";maintain-aspect-ratio TRUE;display
"USEDEF";valid_file "T";width 1.6535in;height 1.7244in;depth
0pt;original-width 3.6997in;original-height 3.8597in;cropleft "0";croptop
"1";cropright "1";cropbottom "0";tempfilename
'LTUWDA7K.wmf';tempfile-properties "XPR";}}In this case, it would take a
whopping big battery to make the electrons move. The battery would have to
supply all of the gap energy plus a little more to get the electron to move.
You might envision this as if there were an electrical \textquotedblleft
glue\textquotedblright\ that keeps the electrons in place. Before they can
move, you have to free them from the \textquotedblleft
glue.\textquotedblright\ It takes an amount of energy, $E_{g},$ to free the
electrons before they are able to accept kinetic energy. If we do connect a
very large battery, say, $33000\unit{V},$ then we can get electrons to jump
the gap to a higher energy \textquotedblleft shelf.\textquotedblright\ But
high voltages are not normal conditions, so this is not usually the case. A
material that has a large energy gap between it's valance band and an empty
conduction band is called an insulator.

A mental picture for this might be as shown in the next figure.

\FRAME{dhF}{3.2301in}{1.9605in}{0pt}{}{}{Figure}{\special{language
"Scientific Word";type "GRAPHIC";maintain-aspect-ratio TRUE;display
"USEDEF";valid_file "T";width 3.2301in;height 1.9605in;depth
0pt;original-width 4.4763in;original-height 2.7069in;cropleft "0";croptop
"1";cropright "1";cropbottom "0";tempfilename
'LTUWDA7L.wmf';tempfile-properties "XPR";}}The insulator atoms keep their
valence electrons bound to the nuclei of the atoms. But for a conductor, the
valence electrons are free to travel from atom to atom.%
%TCIMACRO{%
%\TeXButton{Question 223.20.6}{\marginpar {
%\hspace{-0.5in}
%\begin{minipage}[t]{1in}
%\small{Question 223.20.6}
%\end{minipage}
%}}}%
%BeginExpansion
\marginpar {
\hspace{-0.5in}
\begin{minipage}[t]{1in}
\small{Question 223.20.6}
\end{minipage}
}%
%EndExpansion

%TCIMACRO{%
%\TeXButton{Question 223.20.7}{\marginpar {
%\hspace{-0.5in}
%\begin{minipage}[t]{1in}
%\small{Question 223.20.7}
%\end{minipage}
%}}}%
%BeginExpansion
\marginpar {
\hspace{-0.5in}
\begin{minipage}[t]{1in}
\small{Question 223.20.7}
\end{minipage}
}%
%EndExpansion
In an isolated conductor, normally the charge is balanced, so the electrons
may move but generally they stay near a nucleus. But if a conductor has
extra electrons, the electrons that can move will move because they repel
each other. So any extra charge will be on the surface of the conductor.

\FRAME{dhF}{1.1467in}{1.1718in}{0pt}{}{}{Figure}{\special{language
"Scientific Word";type "GRAPHIC";display "USEDEF";valid_file "T";width
1.1467in;height 1.1718in;depth 0pt;original-width 1.209in;original-height
1.1381in;cropleft "0";croptop "1";cropright "1";cropbottom "0";tempfilename
'LTUWDA7M.wmf';tempfile-properties "XPR";}}

This happens very quickly, generally we do find the extra charge distributed
on the outside of a conductor.

\subsection{Semiconductors}

The third choice is that there is a band gap, but the band gap is small. In
this case, some electrons will gain enough thermal energy to cross the gap.
Then these electrons will be in the conduction band. Devices that work this
way are called semiconductors. We won't deal with semiconductors much in
this class, but you probably used many of them in ME210. Diodes, and
transistors are made from semiconductors.

\subsection{Charging and discharging conductors}

Conductors can't usually be charged by rubbing. The electrons in the
conductor may move when rubbed, but then they are free to move around in the
conductor, so they don't leave. But if we rub an insulator, the electrons
are not free to travel in the insulator material, so we can break them free.
Once this happens, we can take our charged insulator and place it in contact
with a conductor. The charge can flow from the insulator to the conductor
(and arrange itself on the conductor surface). Once the charge has moved to
the exterior, it will reach what we call \emph{electrostatic equilibrium}.
All of the repelling electrical forces are in balance, so the charges come
to rest with respect to the conductor.

%TCIMACRO{%
%\TeXButton{Question 223.20.8}{\marginpar {
%\hspace{-0.5in}
%\begin{minipage}[t]{1in}
%\small{Question 223.20.8}
%\end{minipage}
%}}}%
%BeginExpansion
\marginpar {
\hspace{-0.5in}
\begin{minipage}[t]{1in}
\small{Question 223.20.8}
\end{minipage}
}%
%EndExpansion
We can remove the extra charge by creating a path for the charge to follow.
Consider charging a balloon by rubbing it on your hair. Then you connect a
wire to the balloon that is also connected to a metal water pipe. The charge
can flow through the metal conducing wire. If there is a large body that can
attract extra charge, the charge will flow. The Earth is such a large body
that can attract the extra charge. The charge will flow through the wire and
pipe and go into the ground.

\FRAME{dhF}{3.3399in}{1.7296in}{0in}{}{}{Figure}{\special{language
"Scientific Word";type "GRAPHIC";maintain-aspect-ratio TRUE;display
"USEDEF";valid_file "T";width 3.3399in;height 1.7296in;depth
0in;original-width 3.2949in;original-height 1.6933in;cropleft "0";croptop
"1";cropright "1";cropbottom "0";tempfilename
'LTUWDA7N.wmf';tempfile-properties "XPR";}}

You may have heard of electrical grounds. This literally means tying your
device to the Earth through a wire. Since you are made mostly of water that
contains positive ions, you are also a conductor. So if we touch a charged
object, we will most likely discharge the object. This is also why we must
be careful with charge. Large amounts of charge flowing through us leads to
death or injury.

If an object is \emph{grounded}, it cannot build up extra charge. This is
good for appliances and houses, and people.

We talked last time about insulator atoms being polarized.\FRAME{dhF}{%
2.5114in}{0.8631in}{0pt}{}{}{Figure}{\special{language "Scientific
Word";type "GRAPHIC";maintain-aspect-ratio TRUE;display "USEDEF";valid_file
"T";width 2.5114in;height 0.8631in;depth 0pt;original-width
5.0194in;original-height 1.7071in;cropleft "0";croptop "1";cropright
"1";cropbottom "0";tempfilename 'LTUWDA7O.wmf';tempfile-properties "XPR";}}

Remember that for each atom the electrons are displaced relative to the
nucleus.\FRAME{dhF}{2.3186in}{1.3491in}{0pt}{}{}{Figure}{\special{language
"Scientific Word";type "GRAPHIC";maintain-aspect-ratio TRUE;display
"USEDEF";valid_file "T";width 2.3186in;height 1.3491in;depth
0pt;original-width 4.171in;original-height 2.4146in;cropleft "0";croptop
"1";cropright "1";cropbottom "0";tempfilename
'LTUWDB7P.wmf';tempfile-properties "XPR";}}We can define a \emph{center of
charge} much like we defined a center of mass. In the case in the figure, we
can define a negative center of charge and a positive center of charge.%
\FRAME{dhF}{2.3341in}{1.8697in}{0pt}{}{}{Figure}{\special{language
"Scientific Word";type "GRAPHIC";maintain-aspect-ratio TRUE;display
"USEDEF";valid_file "T";width 2.3341in;height 1.8697in;depth
0pt;original-width 2.2943in;original-height 1.8317in;cropleft "0";croptop
"1";cropright "1";cropbottom "0";tempfilename
'LTUWDB7Q.wmf';tempfile-properties "XPR";}}%
%TCIMACRO{%
%\TeXButton{Question 223.20.9}{\marginpar {
%\hspace{-0.5in}
%\begin{minipage}[t]{1in}
%\small{Question 223.20.9}
%\end{minipage}
%}}}%
%BeginExpansion
\marginpar {
\hspace{-0.5in}
\begin{minipage}[t]{1in}
\small{Question 223.20.9}
\end{minipage}
}%
%EndExpansion
Notice that the negative and positive center of charge are not in the same
place when the atom is polarized. We have a name for a pair of positive and
negative charges that are separated by a distance, but that are still bound
together. We call it an \emph{electric dipole.} Often we just draw the
centers of charge joined by a line.\FRAME{dhF}{2.1664in}{1.535in}{0pt}{}{}{%
Figure}{\special{language "Scientific Word";type
"GRAPHIC";maintain-aspect-ratio TRUE;display "USEDEF";valid_file "T";width
2.1664in;height 1.535in;depth 0pt;original-width 2.1266in;original-height
1.4987in;cropleft "0";croptop "1";cropright "1";cropbottom "0";tempfilename
'LTUWDB7R.wmf';tempfile-properties "XPR";}}Using this we can explain why
humidity affects our last lecture experiments so much. The water molecule
has two hydrogen atoms and one oxygen atom. The covalent bond between the
oxygen and hydrogen atoms forms when the oxygen \textquotedblleft
shares\textquotedblright\ the hydrogen's electrons. The electrons from the
hydrogen atoms spend their time with the oxygen atom making one side of the
molecule more positive and the other side more negative.\FRAME{dtbpF}{%
1.8559in}{1.6864in}{0in}{}{}{Figure}{\special{language "Scientific
Word";type "GRAPHIC";maintain-aspect-ratio TRUE;display "USEDEF";valid_file
"T";width 1.8559in;height 1.6864in;depth 0in;original-width
1.8187in;original-height 1.6501in;cropleft "0";croptop "1";cropright
"1";cropbottom "0";tempfilename 'M4DV8300.wmf';tempfile-properties "XPR";}}%
Thus if you have a charged balloon on a humid day, one side of the water
molecules in the air will be attracted to the extra charge on the balloon.
The extra charge will attach to the water molecules, and float away with
them. This will discharge the balloon.

\section{Note on drawing charge diagrams}

We will have to draw diagrams in our problem solutions. Normally we won't
draw atoms, so we will be drawing large objects with or without extra
charge. We know that all materials have positive nuclei and negative
electrons. When these are balanced, there is an electron for every proton,
so if we add up the charges we get zero net charge. These charges don't
contribute to net forces because for every attraction there is a repulsion
of equal magnitude.

So we won't draw all of these charges, but we should remember they are
there. We usually draw a cross section, so here is the cross section of a
round, conducting ball.\FRAME{dhF}{1.3318in}{1.26in}{0pt}{}{}{Figure}{%
\special{language "Scientific Word";type "GRAPHIC";maintain-aspect-ratio
TRUE;display "USEDEF";valid_file "T";width 1.3318in;height 1.26in;depth
0pt;original-width 1.849in;original-height 1.7495in;cropleft "0";croptop
"1";cropright "1";cropbottom "0";tempfilename
'LTUWDB7T.wmf';tempfile-properties "XPR";}}But if we have extra charge, we
should draw it. We will just add plus signs or minus signs. We won't draw
little circles to show the electrons (we can't draw them to scale, they are
phenomenally small). Here is an example of two round objects, one positive
and one negative\FRAME{dhF}{2.5157in}{1.2445in}{0pt}{}{}{Figure}{\special%
{language "Scientific Word";type "GRAPHIC";maintain-aspect-ratio
TRUE;display "USEDEF";valid_file "T";width 2.5157in;height 1.2445in;depth
0pt;original-width 4.657in;original-height 2.29in;cropleft "0";croptop
"1";cropright "1";cropbottom "0";tempfilename
'LTUWDB7U.wmf';tempfile-properties "XPR";}}If the objects are not
conductors, the extra charge may be spread out. We draw the charge
throughout the cross section of the object.

\FRAME{dhF}{2.6878in}{1.1787in}{0pt}{}{}{Figure}{\special{language
"Scientific Word";type "GRAPHIC";maintain-aspect-ratio TRUE;display
"USEDEF";valid_file "T";width 2.6878in;height 1.1787in;depth
0pt;original-width 3.1981in;original-height 1.388in;cropleft "0";croptop
"1";cropright "1";cropbottom "0";tempfilename
'LTUWDB7V.wmf';tempfile-properties "XPR";}}Note that if you transfer charge,
from one object to another, you should try to keep the same total number of
\textquotedblleft +\textquotedblright ' or \textquotedblleft
-\textquotedblright\ signs to show the charge is conserved.

%TCIMACRO{%
%\TeXButton{Basic Equations}{\hspace{-1.3in}{\LARGE Basic Equations\vspace{0.25in}}}}%
%BeginExpansion
\hspace{-1.3in}{\LARGE Basic Equations\vspace{0.25in}}%
%EndExpansion

\chapter{Coulomb's Law and Lines of Force}

%TCIMACRO{%
%\TeXButton{Fundamental Concepts}{\hspace{-1.3in}{\LARGE Fundamental Concepts\vspace{0.25in}}}}%
%BeginExpansion
\hspace{-1.3in}{\LARGE Fundamental Concepts\vspace{0.25in}}%
%EndExpansion

\begin{itemize}
\item Our \textquotedblleft charge\textquotedblright\ force is called the
Coulomb force, and is given by $F=k_{e}\frac{\left\vert q_{1}\right\vert
\left\vert q_{2}\right\vert }{r^{2}}$

\item A field is a quantity that has a value (magnitude and direction) at
every point in space

\item The Coulomb force is caused by an electric field

\item We use field lines to give ourselves a mental picture of a field
\end{itemize}

\section{Coulomb's Law}

%TCIMACRO{%
%\TeXButton{Teach Gravitational Force}{\marginpar {
%\hspace{-0.5in}
%\begin{minipage}[t]{1in}
%\small{My experience so far is that Statics and Dynamics did not teach Newton's law of gravitation so teach it here.}
%\end{minipage}
%}}}%
%BeginExpansion
\marginpar {
\hspace{-0.5in}
\begin{minipage}[t]{1in}
\small{My experience so far is that Statics and Dynamics did not teach Newton's law of gravitation so teach it here.}
\end{minipage}
}%
%EndExpansion
Sometime ago in your Dynamics or PH121 class you learned about gravity.
Let's review for a moment.

%TCIMACRO{%
%\TeXButton{Question 223.21.0.1}{\marginpar {
%\hspace{-0.5in}
%\begin{minipage}[t]{1in}
%\small{Question 223.21.0.1}
%\end{minipage}
%}}}%
%BeginExpansion
\marginpar {
\hspace{-0.5in}
\begin{minipage}[t]{1in}
\small{Question 223.21.0.1}
\end{minipage}
}%
%EndExpansion
%TCIMACRO{%
%\TeXButton{Question 223.21.0.2}{\marginpar {
%\hspace{-0.5in}
%\begin{minipage}[t]{1in}
%\small{Question 223.21.0.2}
%\end{minipage}
%}}}%
%BeginExpansion
\marginpar {
\hspace{-0.5in}
\begin{minipage}[t]{1in}
\small{Question 223.21.0.2}
\end{minipage}
}%
%EndExpansion
%TCIMACRO{%
%\TeXButton{Question 223.21.0.3}{\marginpar {
%\hspace{-0.5in}
%\begin{minipage}[t]{1in}
%\small{Question 223.21.0.3}
%\end{minipage}
%}}}%
%BeginExpansion
\marginpar {
\hspace{-0.5in}
\begin{minipage}[t]{1in}
\small{Question 223.21.0.3}
\end{minipage}
}%
%EndExpansion
%TCIMACRO{%
%\TeXButton{Question 223.21.0.4}{\marginpar {
%\hspace{-0.5in}
%\begin{minipage}[t]{1in}
%\small{Question 223.21.0.4}
%\end{minipage}
%}}}%
%BeginExpansion
\marginpar {
\hspace{-0.5in}
\begin{minipage}[t]{1in}
\small{Question 223.21.0.4}
\end{minipage}
}%
%EndExpansion

From our experience we know that more massive things exert a stronger
gravitational pull than less massive things. We also have some idea that the
farther away an object is, the less the gravitational pull. Newton expressed
this as 
\begin{equation*}
F_{g}=G\frac{m_{1}m_{2}}{r_{12}^{2}}
\end{equation*}%
where the two masses involved (say, the Earth and you) are $m_{1}$ and $%
m_{2} $ and the distance between the two masses is $r_{12}$ (e.g. the
distance from the center of the Earth to the center of you). The constant $G$
is a constant that puts the force into nice units that are convenient for us
to use, like newtons $\left( \unit{N}\right) .$ It has a value of 
\begin{equation*}
G=6.67428\times 10^{-11}\frac{\unit{N}\unit{m}^{2}}{\unit{kg}^{2}}
\end{equation*}%
You might ask, how do we know this? The answer is that Newton and others
performed experiments. Newton's law of gravitation is empirical, meaning
that it came from experiment. Lord Cavendish used a clever device to verify
this law. He suspended two masses from a wire. Then he placed two other
masses near the suspended masses. \FRAME{dhF}{2.2502in}{2.6091in}{0pt}{}{}{%
Figure}{\special{language "Scientific Word";type
"GRAPHIC";maintain-aspect-ratio TRUE;display "USEDEF";valid_file "T";width
2.2502in;height 2.6091in;depth 0pt;original-width 2.2113in;original-height
2.5676in;cropleft "0";croptop "1";cropright "1";cropbottom "0";tempfilename
'LZ4ZKW00.wmf';tempfile-properties "XPR";}}He knew the torsion constant of
the wire (how much it resists being twisted). Then by observing how far the
suspended masses moved, he could work out the strength of the gravitational
force. This is called a torsion balance.

Charles Coulomb thought he could use the same device to measure the strength
of the electric force. Here is his experimental design. \FRAME{dhFU}{2.2502in%
}{2.8867in}{0pt}{\Qcb{Coulomb's Torsion Balance Apparatus}}{}{Figure}{%
\special{language "Scientific Word";type "GRAPHIC";maintain-aspect-ratio
TRUE;display "USEDEF";valid_file "T";width 2.2502in;height 2.8867in;depth
0pt;original-width 6.0139in;original-height 7.7357in;cropleft "0";croptop
"1";cropright "1";cropbottom "0";tempfilename
'LTUWDB7W.bmp';tempfile-properties "XPR";}}You can see this is really just a
torsion balance. This time objects with equal mass \emph{and equal charge}
are suspended on either end of a rod. The rod is hung on a wire. Two other 
\emph{charges} are brought an equal distance, $r_{12},$ from the other
charges. Knowing the torsional properties of the wire, the force due to the
charges can be found. \FRAME{dhF}{1.8282in}{1.8749in}{0pt}{}{}{Figure}{%
\special{language "Scientific Word";type "GRAPHIC";maintain-aspect-ratio
TRUE;display "USEDEF";valid_file "T";width 1.8282in;height 1.8749in;depth
0pt;original-width 3.4757in;original-height 3.5665in;cropleft "0";croptop
"1";cropright "1";cropbottom "0";tempfilename
'LTUWDB7X.wmf';tempfile-properties "XPR";}}Coulomb determined that the force
due to a pair of charges has the following properties:

\begin{enumerate}
\item It is directed along a line connecting the two charged particles and
is inversely proportional to the distance between their centers

\item It is proportional to the product of the magnitudes of the charges $%
\left\vert q_{1}\right\vert $ and $\left\vert q_{2}\right\vert .$

\item It is attractive (the charges accelerate towards each other) if the
charges have different signs, and is repulsive (the charges accelerate away
from each other) if the charges have the same signs.
\end{enumerate}

We can write this in an equation%
\begin{equation}
F=k_{e}\frac{\left\vert q_{1}\right\vert \left\vert q_{2}\right\vert }{%
r_{12}^{2}}
\end{equation}%
%TCIMACRO{%
%\TeXButton{Question 223.21.2}{\marginpar {
%\hspace{-0.5in}
%\begin{minipage}[t]{1in}
%\small{Question 223.21.2}
%\end{minipage}
%}}}%
%BeginExpansion
\marginpar {
\hspace{-0.5in}
\begin{minipage}[t]{1in}
\small{Question 223.21.2}
\end{minipage}
}%
%EndExpansion
%TCIMACRO{%
%\TeXButton{Question 223.21.1}{\marginpar {
%\hspace{-0.5in}
%\begin{minipage}[t]{1in}
%\small{Question 223.21.1}
%\end{minipage}
%}}}%
%BeginExpansion
\marginpar {
\hspace{-0.5in}
\begin{minipage}[t]{1in}
\small{Question 223.21.1}
\end{minipage}
}%
%EndExpansion
Note how much this looks like gravitation! In the denominator, we have the
distance, $r_{12}$, between the two charged particles' centers. We have two
things in the numerator. But now we have $\left\vert q_{1}\right\vert $ and $%
\left\vert q_{2}\right\vert $ instead of $m_{1}$ and $m_{2}.$ We have a
constant $k_{e}$ instead of $G,$ but the equation is very much like Newton's
law of gravitation. That should be comforting, because we know how to use
Newton's law of gravitation from PH121 or Dynamics. There is a very big
difference, though. Gravitation can only attract masses, The Force due to
charges can attract \emph{or} \emph{repel}.

Again there is a constant to fix up the units. Our constant is 
\begin{equation}
k_{e}=8.9875\times 10^{9}\frac{\unit{N}\unit{m}^{2}}{\unit{C}^{2}}
\end{equation}%
which allows us to use more meaningful units (to us humans) in the force
equation.

How about strength? Is gravity or is this force due to charge stronger?%
%TCIMACRO{%
%\TeXButton{Comb and paper bits demo}{\marginpar {
%\hspace{-0.5in}
%\begin{minipage}[t]{1in}
%\small{Comb and paper bits demo}
%\end{minipage}
%}}}%
%BeginExpansion
\marginpar {
\hspace{-0.5in}
\begin{minipage}[t]{1in}
\small{Comb and paper bits demo}
\end{minipage}
}%
%EndExpansion

\begin{equation*}
\begin{tabular}{|c|c|c|c|c|c|}
\hline
{\small Force} & {\small Varies with Distance} & {\small Attracts} & {\small %
Repels} & {\small Acts without contact} & {\small Strength} \\ \hline
{\small Gravity} & {\small Yes} & {\small Always} & {\small Never} & {\small %
Yes} & {\small Weaker} \\ \hline
{\small Charge Force} & {\small Yes} & {\small Sometimes} & {\small Sometimes%
} & {\small Yes} & {\small Stronger} \\ \hline
\end{tabular}%
\end{equation*}

\bigskip

Lets try an example problem:

\begin{example}
Calculate the magnitude of the electric force between the proton and
electron in a hydrogen atom. Compare to their gravitational attraction. We
expect the electrical force to be larger. We need some facts about Hydrogen%
\begin{equation*}
\begin{tabular}{|l|l|}
\hline
Item & Value \\ \hline
Proton Mass & $1.67\times 10^{-27}\unit{kg}$ \\ \hline
Electron Mass & $9.11\times 10^{-31}\unit{kg}$ \\ \hline
Proton Charge & $1.6\times 10^{-19}\unit{C}$ \\ \hline
Electron Charge & $-1.6\times 10^{-19}\unit{C}$ \\ \hline
Proton-electron average separation & $5.3\times 10^{-11}\unit{m}$ \\ \hline
\end{tabular}%
\end{equation*}%
then,%
\begin{eqnarray*}
F_{e} &=&k_{e}\frac{\left\vert q_{1}\right\vert \left\vert q_{2}\right\vert 
}{r^{2}} \\
&=&8.9875\times 10^{9}\frac{\unit{N}\unit{m}^{2}}{\unit{C}^{2}}\frac{\left(
-1.6\times 10^{-19}\unit{C}\right) \left( 1.6\times 10^{-19}\unit{C}\right) 
}{\left( 5.3\times 10^{-11}\unit{m}\right) ^{2}} \\
&=&-\allowbreak 8.\,\allowbreak 190\,8\times 10^{-8}\frac{\unit{m}}{\unit{s}%
^{2}}\unit{kg}
\end{eqnarray*}%
and%
\begin{eqnarray*}
F_{g} &=&G\frac{m_{1}m_{2}}{r^{2}} \\
&=&6.67\times 10^{-11}\unit{N}\frac{\unit{m}^{2}}{\unit{kg}^{2}}\frac{\left(
1.67\times 10^{-27}\unit{kg}\right) \left( 9.11\times 10^{-31}\unit{kg}%
\right) }{\left( 5.3\times 10^{-11}\unit{m}\right) ^{2}} \\
&=&\allowbreak 3.\,\allowbreak 612\,5\times 10^{-47}\frac{\unit{m}}{\unit{s}%
^{2}}\unit{kg}
\end{eqnarray*}%
which shows us what we expected, the gravitational force is very small
compared to the electric force.
\end{example}

\subsection{Permittivity of free space}

It is customary to define an additional constant%
\begin{equation}
\epsilon _{o}=\frac{1}{4\pi k_{e}}=8.85\times 10^{-12}\frac{\unit{C}^{2}}{%
\unit{N}\unit{m}^{2}}
\end{equation}%
Using this constant%
\begin{equation}
F=\frac{1}{4\pi \epsilon _{o}}\frac{\left\vert q_{1}\right\vert \left\vert
q_{2}\right\vert }{r^{2}}
\end{equation}%
%TCIMACRO{%
%\TeXButton{Question 223.21.3}{\marginpar {
%\hspace{-0.5in}
%\begin{minipage}[t]{1in}
%\small{Question 223.21.3}
%\end{minipage}
%}}}%
%BeginExpansion
\marginpar {
\hspace{-0.5in}
\begin{minipage}[t]{1in}
\small{Question 223.21.3}
\end{minipage}
}%
%EndExpansion
%TCIMACRO{%
%\TeXButton{Question 223.21.4}{\marginpar {
%\hspace{-0.5in}
%\begin{minipage}[t]{1in}
%\small{Question 223.21.4}
%\end{minipage}
%}}}%
%BeginExpansion
\marginpar {
\hspace{-0.5in}
\begin{minipage}[t]{1in}
\small{Question 223.21.4}
\end{minipage}
}%
%EndExpansion
%TCIMACRO{%
%\TeXButton{Question 223.21.5}{\marginpar {
%\hspace{-0.5in}
%\begin{minipage}[t]{1in}
%\small{Question 223.21.5}
%\end{minipage}
%}}}%
%BeginExpansion
\marginpar {
\hspace{-0.5in}
\begin{minipage}[t]{1in}
\small{Question 223.21.5}
\end{minipage}
}%
%EndExpansion
which really does not seem to be an improvement. But if you go on to take an
advanced class in electrodynamics you will find that this form is more
convenient in other unit systems. So we will adopt it even though it is an
inconvenience now.

\section{Direction of the force}

What about direction? So far we have only calculated the magnitude of the
force. But a force is a vector, so it must have a direction. Notice that our
equation has absolute value signs in it. We will only get positive values
from Coulomb's law.

To find a strategy for getting the direction, let's observe two charged
objects\FRAME{dhF}{1.1459in}{1.0464in}{0pt}{}{}{Figure}{\special{language
"Scientific Word";type "GRAPHIC";maintain-aspect-ratio TRUE;display
"USEDEF";valid_file "T";width 1.1459in;height 1.0464in;depth
0pt;original-width 1.1122in;original-height 1.0127in;cropleft "0";croptop
"1";cropright "1";cropbottom "0";tempfilename
'LZ51XB02.wmf';tempfile-properties "XPR";}}Experiments show that they seem
to be pulled straight toward each other. The force seems to be along the
line that passes through the center of charge for each of the two charged
objects. We have to find this line from the geometry of our situation and
our choice of coordinate systems. To make matters worse, we could have two
of the same kind of charge.

\FRAME{dhF}{1.0905in}{0.9764in}{0pt}{}{}{Figure}{\special{language
"Scientific Word";type "GRAPHIC";maintain-aspect-ratio TRUE;display
"USEDEF";valid_file "T";width 1.0905in;height 0.9764in;depth
0pt;original-width 1.0568in;original-height 0.9435in;cropleft "0";croptop
"1";cropright "1";cropbottom "0";tempfilename
'LZ521Y06.wmf';tempfile-properties "XPR";}}The force will still be on the
line connecting the centers of charge, but it will be in the opposite
direction compared to the last case where the charges were of different
sign. This seems complicated, and it is. We must observe the geometry of our
situation and note whether the charges are the same or different signs to
find the direction. Our equations can't tell us the direction on their own.
You can't put the signs of the charges into the formula and expect a
direction to come out! You have to draw the picture. Here is the process:

\begin{enumerate}
\item Define your coordinate system.

\item Find the line that connects the centers of charge. The force direction
will be on that line.

\item Determine the direction by observing the signs of the charges. If the
charges have the same sign, the force will be repulsive, if the charges have
different signs, it will be attractive.
\end{enumerate}

\section{More than two charges}

It is great that we know the force between two charges, but we have learned
that there are billions of charges in everything we see or touch. It would
be nice to be able to use our simple law of force on more than one or two
charges. We did this with gravity. Let's review.%
%TCIMACRO{%
%\TeXButton{Question 223.21.6 }{\marginpar {
%\hspace{-0.5in}
%\begin{minipage}[t]{1in}
%\small{Question 223.21.6 }
%\end{minipage}
%}}}%
%BeginExpansion
\marginpar {
\hspace{-0.5in}
\begin{minipage}[t]{1in}
\small{Question 223.21.6 }
\end{minipage}
}%
%EndExpansion

Suppose I have a satellite orbiting the Earth. That satellite feels a force
given by 
\begin{eqnarray*}
F_{g} &=&G\frac{M_{E}m_{s}}{r^{2}} \\
&=&G\left( \frac{M_{E}}{r^{2}}\right) m_{s}
\end{eqnarray*}%
but consider that on the Earth below the satellite, there is a rock on the
surface of the Earth. \FRAME{dhF}{3.0606in}{2.2883in}{0pt}{}{}{Figure}{%
\special{language "Scientific Word";type "GRAPHIC";maintain-aspect-ratio
TRUE;display "USEDEF";valid_file "T";width 3.0606in;height 2.2883in;depth
0pt;original-width 3.0173in;original-height 2.2485in;cropleft "0";croptop
"1";cropright "1";cropbottom "0";tempfilename
'LTUWDB7Y.wmf';tempfile-properties "XPR";}}Part of the force due to gravity
on the satellite must be due to this rock. We could write our force due to
gravity as 
\begin{equation*}
F_{g}=G\left( \frac{M_{rest}}{r_{rest}^{2}}\hat{r}_{rest}+\frac{M_{rock}}{%
r_{s}^{2}}\hat{r}_{rock}\right) m_{s}
\end{equation*}%
where $M_{rest}$ is the mass of all the rest of the Earth, minus the rock.
If we take the Earth rock by rock, we would have%
\begin{equation*}
F_{g}=G\left( \Sigma _{i}\frac{M_{i}}{r_{i}^{2}}\mathbf{\hat{r}}_{i}\right)
m_{s}
\end{equation*}%
where $M_{i}$ is the mass of the $i^{th}$ piece of the Earth and $\mathbf{%
\hat{r}}_{i}$ is the direction from $M_{i}$ to $m_{s}$. We would not really
want to do this calculation, because it would take a long time. Instead,
back in PH121 or Dynamics we found we could add up all the mass and treat
the Earth as one big ball of mass and represent it as if the mass was all at
it's center of mass (as long as there is no rotation so no torque). But
let's think about all this mass. Does the force between a rock in China and
our satellite get diminished because our rock in Rexburg is in the way? 
\FRAME{dhF}{4.3059in}{3.1955in}{0pt}{}{}{Figure}{\special{language
"Scientific Word";type "GRAPHIC";maintain-aspect-ratio TRUE;display
"USEDEF";valid_file "T";width 4.3059in;height 3.1955in;depth
0pt;original-width 4.254in;original-height 3.1505in;cropleft "0";croptop
"1";cropright "1";cropbottom "0";tempfilename
'LTUWDB7Z.wmf';tempfile-properties "XPR";}}No, the force due to gravity is
really the sum of all the little forces between all the parts of the Earth
and our satellite. One bit of mass does not interfere with the force from
another bit of mass.

Now let's look at the electric force. Suppose we have many charges in some
configuration (maybe a round ball of charge). We could call the total
charge, $Q_{E}.$ Then our force magnitude on a mover charge $q_{m},$ would
be 
\begin{equation*}
F_{e}=k_{e}\frac{\left\vert Q_{E}\right\vert \left\vert q_{m}\right\vert }{%
r^{2}}
\end{equation*}%
The collection of charge $Q_{E}$ would be the environmental charge. But we
can picture this as the individual parts of $Q_{E}$ all with little forces
pairs acting on $q_{m}$ summing up to get $F_{e}.$%
\begin{equation*}
F_{e}=k_{e}\Sigma _{i}\left( \frac{\left\vert Q_{i}\right\vert }{r_{i}^{2}}%
\mathbf{\hat{r}}_{i}\right) \left\vert q_{m}\right\vert
\end{equation*}%
where $Q_{i}$ is a piece of the total charge $Q_{E}.$

This is an amazingly simple idea. The force on a mover charge, $q_{m},$ due
to any number of charges is just the sum of the forces due to each charge
acting on $q_{m}.$ Sometimes the mover charge is called a \emph{test charge}%
, but we will call it a mover charge and we will call the $Q_{i}$
environmental charges.

Suppose in our ball of charge, we have an element of charge on the opposite
side of the ball and another element of charge close to us. Would the near
charge element \textquotedblleft screen off\textquotedblright\ or some how
reduce the force due to the far charge element?

Like with gravity, it would not. Note that because one charge is farther
away, the force from the far charge is not the same magnitude as that of the
near charge. But we calculate both using our formula, and add them up (a
vector sum) with all the others.

While we are talking about it, it might seem that the rest of the matter in
the ball will screen off the electric force. But matter, itself, does not
interfere with our electric force. Only other charges will change the force,
and then only following the idea of that their forces add as vectors
(remember that for electricity they can cancel, because we have both
positive and negative charges).

Recall that in our study of waves, when we had two waves in a medium we
found we could just added up displacements point for point. We called this 
\emph{superposition}. We will use the same word here, but it has a slightly
different meaning. We are not adding up wave displacements. We are adding up
forces. But we still do it point for point.

Now where there are forces, there will be Newton's second law! Let's
consider a problem. Suppose we have three charges, equally spaced apart as
shown where each has the charge of one electron ($q_{e}$) but the middle
charge is positive and the other two are negative%
%TCIMACRO{%
%\TeXButton{Draw picture on board}{\marginpar {
%\hspace{-0.5in}
%\begin{minipage}[t]{1in}
%\small{Draw picture on board }
%\end{minipage}
%}} }%
%BeginExpansion
\marginpar {
\hspace{-0.5in}
\begin{minipage}[t]{1in}
\small{Draw picture on board }
\end{minipage}
}
%EndExpansion
\FRAME{dhF}{2.9231in}{1.0784in}{0pt}{}{}{Figure}{\special{language
"Scientific Word";type "GRAPHIC";maintain-aspect-ratio TRUE;display
"USEDEF";valid_file "T";width 2.9231in;height 1.0784in;depth
0pt;original-width 10.2039in;original-height 3.7472in;cropleft "0";croptop
"1";cropright "1";cropbottom "0";tempfilename
'NBC7SJ09.wmf';tempfile-properties "XPR";}}We identify the middle charge as
the mover (since we are asked for the force on this charge) and the left and
right charges as the environmental charges. We can draw a free body diagram
for the mover charge.\FRAME{dhF}{3.3719in}{0.9279in}{0pt}{}{}{Figure}{%
\special{language "Scientific Word";type "GRAPHIC";maintain-aspect-ratio
TRUE;display "USEDEF";valid_file "T";width 3.3719in;height 0.9279in;depth
0pt;original-width 10.2039in;original-height 2.789in;cropleft "0";croptop
"1";cropright "1";cropbottom "0";tempfilename
'NBC7SJ0A.wmf';tempfile-properties "XPR";}}and find the net force on the
mover charge, then%
\begin{equation*}
\overrightarrow{\mathbf{F}}_{net}=m\overrightarrow{\mathbf{a}}=%
\overrightarrow{\mathbf{F}_{R}}+\overrightarrow{\mathbf{F}}_{L}
\end{equation*}
We only have $x$-components so we can write this as 
\begin{equation*}
F_{net_{x}}=ma=F_{Rx}-F_{Lx}
\end{equation*}%
where the minus sign is used for $F_{Lx}$ because it is pointing to the left
and that is usually the minus $x$ direction.

We may ask, is this mover charge accelerating? We may suspect that the
answer is no, but here we have something new. We don't know the magnitude of 
$F_{R}$ or $F_{L}.$ We now have to find the magnitudes to know. Back in
PH121 you would have been given the magnitude of the forces, but in a charge
problem we know how to calculate the magnitudes, so let's do that. We can
use the formula for the Coulomb force%
\begin{equation*}
F=k_{e}\frac{\left\vert q_{1}\right\vert \left\vert q_{2}\right\vert }{r^{2}}
\end{equation*}%
we can use $r$ as the distance from the middle charge to each of the other
charges since in this special case they are both the same distance from the
middle charge. Then

\begin{equation*}
F_{R}=k_{e}\frac{q_{e}^{2}}{r^{2}}
\end{equation*}%
\begin{equation*}
F_{L}=k_{e}\frac{q_{e}^{2}}{r^{2}}
\end{equation*}%
these are the magnitudes. We should notice that $F_{L}$ points to the left.
So we need to include a minus sign in front of it's magnitude.%
\begin{equation*}
F_{net_{x}}=ma=F_{Rx}-F_{Lx}
\end{equation*}
\begin{eqnarray*}
F_{net_{x}} &=&ma=k_{e}\frac{q_{e}^{2}}{r^{2}}-k_{e}\frac{q_{e}^{2}}{r^{2}}
\\
&=&0
\end{eqnarray*}%
now we can say that the middle charge is definitely not accelerating.

Of course this is a pretty easy Newton's 2nd law problem. It was all in the $%
x$-direction. But suppose that is not true. Then we need to take components
of the forces vectors. Let's try one of those.%
%TCIMACRO{%
%\TeXButton{Draw picture on board}{\marginpar {
%\hspace{-0.5in}
%\begin{minipage}[t]{1in}
%\small{Draw picture on board }
%\end{minipage}
%}} }%
%BeginExpansion
\marginpar {
\hspace{-0.5in}
\begin{minipage}[t]{1in}
\small{Draw picture on board }
\end{minipage}
}
%EndExpansion
\FRAME{dhF}{0.7922in}{0.7939in}{0pt}{}{}{Figure}{\special{language
"Scientific Word";type "GRAPHIC";maintain-aspect-ratio TRUE;display
"USEDEF";valid_file "T";width 0.7922in;height 0.7939in;depth
0pt;original-width 3.9202in;original-height 3.928in;cropleft "0";croptop
"1";cropright "1";cropbottom "0";tempfilename
'NBC7SJ0B.wmf';tempfile-properties "XPR";}}Here is a new configuration of
our charges. There will be a Coulomb force between each negative charge the
positive charge. What is the net force on the positive charge?

Again we need Newton's second law and the Coulomb force equation. We
identify the positive charge as our mover, and the negative charges as the
environmental charges. Our basic equations are%
\begin{equation*}
F=k_{e}\frac{\left\vert q_{1}\right\vert \left\vert q_{2}\right\vert }{r^{2}}
\end{equation*}%
\begin{equation*}
\overrightarrow{\mathbf{F}}=m\overrightarrow{\mathbf{a}}
\end{equation*}%
but this time we need an $x$ and a $y$ Newton's second law equation. Let's
draw the free body diagram.%
%TCIMACRO{%
%\TeXButton{Draw picture on board}{\marginpar {
%\hspace{-0.5in}
%\begin{minipage}[t]{1in}
%\small{Draw picture on board }
%\end{minipage}
%}} }%
%BeginExpansion
\marginpar {
\hspace{-0.5in}
\begin{minipage}[t]{1in}
\small{Draw picture on board }
\end{minipage}
}
%EndExpansion
I have chosen the positive $y$-direction to be upward and the positive $x$%
-direction to be to the right.\FRAME{dhF}{0.6529in}{0.5241in}{0pt}{}{}{Figure%
}{\special{language "Scientific Word";type "GRAPHIC";maintain-aspect-ratio
TRUE;display "USEDEF";valid_file "T";width 0.6529in;height 0.5241in;depth
0pt;original-width 2.6835in;original-height 2.1508in;cropleft "0";croptop
"1";cropright "1";cropbottom "0";tempfilename
'NBC7SJ0C.wmf';tempfile-properties "XPR";}}The negative charge that is above
our positive charge will cause an upward force. The negative charge to the
right will cause a force that pulls to the right. This is a two-dimensional
problem, so we need to split our Newton's second law into two
one-dimensional problems. 
\begin{eqnarray*}
F_{net_{x}} &=&ma_{x}=F_{L} \\
F_{net_{y}} &=&ma_{y}=F_{up}
\end{eqnarray*}%
so%
\begin{eqnarray*}
F_{net_{x}} &=&k_{e}\frac{q_{e}^{2}}{r^{2}} \\
F_{net_{y}} &=&k_{e}\frac{q_{e}^{2}}{r^{2}}
\end{eqnarray*}%
We can see that there will be a force in both the $x$ and the $y$ direction.
How do we combine these to get the net force? We use our basic equations for
combining vectors:

\begin{eqnarray*}
F_{net} &=&\sqrt{F_{net_{x}}^{2}+F_{net_{y}}^{2}} \\
&=&\sqrt{\left( k_{e}\frac{q_{e}^{2}}{r^{2}}\right) ^{2}+\left( k_{e}\frac{%
q_{e}^{2}}{r^{2}}\right) ^{2}} \\
&=&\sqrt{2}\frac{1}{r^{2}}k_{e}q_{e}^{2}
\end{eqnarray*}%
but we are not done. We need a direction. Generally we use the angle with
respect to the positive $x$-axis.%
\begin{eqnarray*}
\theta &=&\tan ^{-1}\left( \frac{F_{net_{y}}}{F_{net_{x}}}\right) \\
&=&\tan ^{-1}\left( \frac{k_{e}\frac{q_{e}^{2}}{r^{2}}}{k_{e}\frac{q_{e}^{2}%
}{r^{2}}}\right) \\
&=&\frac{\pi }{4}\unit{rad}
\end{eqnarray*}%
so we have a net force of $F=\sqrt{2}\frac{1}{r^{2}}k_{e}q_{e}^{2}$ at a $45%
\unit{%
%TCIMACRO{\U{b0}}%
%BeginExpansion
{{}^\circ}%
%EndExpansion
}$ angle with respect to the $x$-axis.

Of course, this is still fairly simple, we should also review taking
components of vectors that are not directed along the $x$ and the $y$ axis.
Suppose we move the top charge as shown below%
%TCIMACRO{%
%\TeXButton{Draw picture on board}{\marginpar {
%\hspace{-0.5in}
%\begin{minipage}[t]{1in}
%\small{Draw picture on board }
%\end{minipage}
%}}}%
%BeginExpansion
\marginpar {
\hspace{-0.5in}
\begin{minipage}[t]{1in}
\small{Draw picture on board }
\end{minipage}
}%
%EndExpansion
\FRAME{dhF}{0.8968in}{0.7126in}{0pt}{}{}{Figure}{\special{language
"Scientific Word";type "GRAPHIC";maintain-aspect-ratio TRUE;display
"USEDEF";valid_file "T";width 0.8968in;height 0.7126in;depth
0pt;original-width 3.9202in;original-height 3.1081in;cropleft "0";croptop
"1";cropright "1";cropbottom "0";tempfilename
'NBC9FB0J.wmf';tempfile-properties "XPR";}}Once again the positive charge is
the mover and the negative charges are the environment. Now our free body
diagram looks like this:%
%TCIMACRO{%
%\TeXButton{Draw picture on board}{\marginpar {
%\hspace{-0.5in}
%\begin{minipage}[t]{1in}
%\small{Draw picture on board }
%\end{minipage}
%}}}%
%BeginExpansion
\marginpar {
\hspace{-0.5in}
\begin{minipage}[t]{1in}
\small{Draw picture on board }
\end{minipage}
}%
%EndExpansion
\FRAME{dhF}{0.5267in}{0.5172in}{0pt}{}{}{Figure}{\special{language
"Scientific Word";type "GRAPHIC";maintain-aspect-ratio TRUE;display
"USEDEF";valid_file "T";width 0.5267in;height 0.5172in;depth
0pt;original-width 2.5166in;original-height 2.4708in;cropleft "0";croptop
"1";cropright "1";cropbottom "0";tempfilename
'NBC9FB0K.wmf';tempfile-properties "XPR";}}Once again we have a
two-dimensional problem. We need to convert it into two one-dimensional
problems.%
\begin{eqnarray*}
F_{net_{x}} &=&ma_{x}=F_{L_{x}}+F_{2x} \\
F_{net_{y}} &=&ma_{y}=F_{L_{y}}+F_{2y}
\end{eqnarray*}%
but we don't know $F_{L_{x}},F_{2x},F_{L_{y}},$ and $F_{2y}.$ But our basic
equations should include how to make vector components%
\begin{eqnarray*}
v_{x} &=&v\cos \theta \\
v_{y} &=&v\sin \theta
\end{eqnarray*}%
where $\theta $ is measured from the positive $x$-axis. So%
\begin{eqnarray*}
F_{net_{x}} &=&ma_{x}=F_{L}\cos \theta _{L}+F_{2}\cos \theta _{2} \\
F_{net_{y}} &=&ma_{y}=F_{L}\sin \theta _{L}+F_{2}\sin \theta _{2}
\end{eqnarray*}%
and we realize that 
\begin{equation*}
\theta _{L}=0
\end{equation*}%
and that 
\begin{eqnarray*}
\cos \left( 0\right) &=&1 \\
\sin \left( 0\right) &=&0
\end{eqnarray*}%
so 
\begin{eqnarray*}
ma_{x} &=&F_{L}+F_{2}\cos \theta _{2} \\
ma_{y} &=&0+F_{2}\sin \theta _{2}
\end{eqnarray*}%
This gives the $x$ and $y$ components of the net force on the positive
charge. Using our Coulomb force for the magnitudes, we have%
\begin{eqnarray*}
F_{net_{x}} &=&k_{e}\frac{q_{e}^{2}}{r^{2}}+k_{e}\frac{q_{e}^{2}}{r^{2}}\cos
\theta _{2} \\
F_{net_{y}} &=&k_{e}\frac{q_{e}^{2}}{r^{2}}\sin \theta _{2}
\end{eqnarray*}%
I will tell you $\theta =\frac{\pi }{4}\unit{rad}$ (or $45\unit{%
%TCIMACRO{\U{b0}}%
%BeginExpansion
{{}^\circ}%
%EndExpansion
}$). So we can find%
\begin{eqnarray*}
F_{net_{x}} &=&k_{e}\frac{q_{e}^{2}}{r^{2}}+k_{e}\frac{q_{e}^{2}}{r^{2}}%
\left( \frac{\sqrt{2}}{2}\right) =k_{e}\frac{q_{e}^{2}}{r^{2}}\left( 1+\frac{%
\sqrt{2}}{2}\right) \\
F_{net_{y}} &=&k_{e}\frac{q_{e}^{2}}{r^{2}}\left( \frac{\sqrt{2}}{2}\right)
\end{eqnarray*}%
and%
\begin{eqnarray*}
F_{net} &=&\sqrt{F_{net_{x}}^{2}+F_{net_{y}}^{2}} \\
&=&\sqrt{\left( k_{e}\frac{q_{e}^{2}}{r^{2}}\left( 1+\frac{\sqrt{2}}{2}%
\right) \right) ^{2}+\left( k_{e}\frac{q_{e}^{2}}{r^{2}}\left( \frac{\sqrt{2}%
}{2}\right) \right) ^{2}} \\
&=&\allowbreak \frac{k_{e}q_{e}^{2}}{r^{2}}\sqrt{2+\sqrt{2}}
\end{eqnarray*}

This is not so nice and easy. The angle is 
\begin{eqnarray*}
\theta &=&\tan ^{-1}\left( \frac{k_{e}\frac{q_{e}^{2}}{r^{2}}\left( \frac{%
\sqrt{2}}{2}\right) }{k_{e}\frac{q_{e}^{2}}{r^{2}}\left( 1+\frac{\sqrt{2}}{2}%
\right) }\right) \\
&=&\tan ^{-1}\left( \frac{1}{2}\frac{\sqrt{2}}{\frac{1}{2}\sqrt{2}+1}\right)
\\
&=&\allowbreak 0.392\,70\unit{rad}
\end{eqnarray*}%
Note that I am using symbols as long as I can. This will become ever more
important in this course. The problems will become very complicated. It is
easier to make mistakes if you input numbers early.

Also notice that I\ carefully placed the charges the same distance, $r,$
from each other. Of course that will not always be true. If the distances
are different, we will use subscripts (e.g. $r_{1}$, $r_{2}$) to distinguish
the distances.\bigskip

\section{Fields}

If you are taking PH223 you should have already taken PH121 or an equivalent
class. In PH121, you learned about how things move. You learned about forces
and how force relates to acceleration%
\begin{equation*}
\overrightarrow{\mathbf{F}}=m\overrightarrow{\mathbf{a}}
\end{equation*}%
The force, $\overrightarrow{\mathbf{F}},$ is how hard you push or pull. This
push or pull changes the motion of the object, represented by it's mass, $m.$
The change in motion is represented by its acceleration, $\overrightarrow{%
\mathbf{a}}.$ Notice that both $\overrightarrow{\mathbf{F}}$ and $%
\overrightarrow{\mathbf{a}}$ are vectors. We will need all that you learned
about vectors in PH121.

Since physics is the study of how things move, we are going to study the
motion of objects again in this class. But in this section of our class we
will learn about new sources of force, that is, new ways to push or pull
something.

Really these new sources of force are not entirely new. You have heard of
them and probably experienced them. They are electrical charge and
magnetism. You have probably had a sock stick from you after pulling it out
of a dryer, and you have probably had a magnet that sticks to your
refrigerator. So although these new sources of force are new to our study of
physics, they are somewhat familiar in every day lives.

Let's review a particular force, the force due to gravity. This makes sense
to do because our equation for electrical force is so very much like
Newton's equation for gravity. Think of most of our experience with gravity.
We have an object moving near the Earth. There is a force acting on the
object, and that force is because of Earth's gravity.

We can think of the Earth as creating an environment in which the object
moves, feeling the gravitational force. This is a property of all
non-contact forces.

Think of a ball falling, We considered this as an environment of constant
acceleration. In this environment, the ball feels a force proportional to
its mass%
\begin{equation*}
\overrightarrow{\mathbf{F}}=m\overrightarrow{\mathbf{g}}
\end{equation*}%
where $g=9.81\frac{\unit{m}}{\unit{s}}$ is the acceleration due to gravity.
This is true anywhere near the Earth's surface. We could draw this situation
as follows:

\FRAME{dtbpF}{3.869in}{2.2449in}{0pt}{}{}{Figure}{\special{language
"Scientific Word";type "GRAPHIC";maintain-aspect-ratio TRUE;display
"USEDEF";valid_file "T";width 3.869in;height 2.2449in;depth
0pt;original-width 3.9142in;original-height 2.26in;cropleft "0";croptop
"1";cropright "1";cropbottom "0";tempfilename
'NBC7SJ00.wmf';tempfile-properties "XPR";}}where the environment for
constant acceleration is drawn as a series of arrows in the acceleration
direction (downward toward the center of the Earth). Anywhere the ball goes
the environment is the same. So we draw arrows all around the ball to show
that the whole environment around the ball is the same.

Notice that the environment is described by an acceleration, $g$ given by 
\begin{equation*}
\overrightarrow{\mathbf{g}}=\frac{\overrightarrow{\mathbf{F}}}{m}
\end{equation*}%
that is, the environment is described by the force per unit mass.

This environment is caused by the Earth being there. If the Earth suddenly
disappeared, then the acceleration would just as suddenly go to zero. So we
can say that the Earth creates this constant acceleration environment.

Notice that there are two objects involved, the ball and the Earth. Also
notice that one object creates an environment in which the other object
moves. In our case, the Earth created the environment and the ball moved
through the environment. This situation will recur many times, so let's give
the objects these names, the Earth as the \textquotedblleft Environmental
object\textquotedblright , and the ball as the \textquotedblleft
mover.\textquotedblright

We should ask ourselves, does something like this happen with our electrical
force. The electric forces is also a non-contact force. Could we view one
charge as creating an environment in which the other charge moves? And if
there is an environment, what would that environment be. Would it be an
acceleration, or something else?

Michael Faraday came up with answers to this questions. To gain insight into
his answers, let's consider our force again.%
\begin{equation*}
F_{e}=k_{e}\frac{\left\vert Q_{E}\right\vert \left\vert q_{m}\right\vert }{%
r^{2}}
\end{equation*}%
but let's take $q_{m}$ as a very small test charge that we can place near a
larger distribution of charge $Q_{E}.$ This is like the Earth and our small
ball. The large $Q_{E}$ is the environmental charge and the small $q_{m}$ is
the mover charge. We want $q_{m}$ to be so small that it can't make any of
the parts of $Q_{E}$ rearrange themselves or any of the atoms forming the
body that is charged with $Q_{E}$ to polarize. Then we define a new quantity%
\begin{equation*}
\overrightarrow{\mathbf{E}}=\frac{\overrightarrow{\mathbf{F}}}{q_{m}}
\end{equation*}%
This is the force per unit charge. This is very like our gravitational
acceleration which is a force per unit mass. Then 
\begin{equation}
\overrightarrow{\mathbf{F}}=q_{m}\overrightarrow{\mathbf{E}}
\end{equation}%
This is really like 
\begin{equation*}
\overrightarrow{\mathbf{F}}=m\overrightarrow{\mathbf{g}}
\end{equation*}%
but with the mass replaced by $q_{m}$ and the acceleration replaced by this
new force-per-unit-charge thing. For gravity it is the mass that made the
gravitational pull. With the electric force it is the charge that creates
the pull. So replacing $m$ with $q_{m}$ makes some sense. But what does it
mean that the acceleration has been replaced by $\overrightarrow{\mathbf{E}}%
. $ Well, since $\overrightarrow{\mathbf{g}}$ was the representation of the
environment, can see that this new quantity is taking the place of the
environment, but it can't be an acceleration. It does not have the right
units. Let's investigate what it is.

Let's write the magnitude of $E$ 
\begin{eqnarray*}
E &\mathbf{=}&\frac{F}{q_{m}} \\
&=&\frac{k_{e}\frac{\left\vert Q_{E}\right\vert \left\vert q_{m}\right\vert 
}{r^{2}}}{q_{m}} \\
&=&k_{e}\frac{Q_{E}}{r^{2}}
\end{eqnarray*}%
But this is really not a quantity that we have seen before%
%TCIMACRO{%
%\TeXButton{van de Graff and test charge}{\marginpar {
%\hspace{-0.5in}
%\begin{minipage}[t]{1in}
%\small{van de Graff and test charge}
%\end{minipage}
%}} }%
%BeginExpansion
\marginpar {
\hspace{-0.5in}
\begin{minipage}[t]{1in}
\small{van de Graff and test charge}
\end{minipage}
}
%EndExpansion
It depends on how far away we are from the environmental charge $Q_{E}.$ It
has a value at every point in space--the whole universe! (think of our
acceleration environment being all around the moving ball) though it's
values for large $r$ are very small. The quantity is only large in the near
vicinity of the charge, $Q_{E}$.\FRAME{dtbpF}{4.3418in}{2.4605in}{0pt}{}{}{%
Figure}{\special{language "Scientific Word";type
"GRAPHIC";maintain-aspect-ratio TRUE;display "USEDEF";valid_file "T";width
4.3418in;height 2.4605in;depth 0pt;original-width 4.3968in;original-height
2.4782in;cropleft "0";croptop "1";cropright "1";cropbottom "0";tempfilename
'NBC9FB0L.wmf';tempfile-properties "XPR";}}

We can picture this quantity as being like a foot ball field with something
(an environmental charge) hidden out there on the grass. If we know where
the object is, we can tell a searcher how \textquotedblleft
warm\textquotedblright\ or \textquotedblleft cold\textquotedblright\ they
are as they wander around looking for the object. For every location, there
is a value of \textquotedblleft warmness.\textquotedblright\ If we extend
this idea to three dimensions, we are close to a picture of $\overrightarrow{%
\mathbf{E}}\mathbf{.}$ The environment quantity $\overrightarrow{\mathbf{E}}$
has a value at every point in three dimensional space. Since this is a new
quantity, we need to give it a name. We will call it an \emph{electric} 
\emph{field.} 
%TCIMACRO{%
%\TeXButton{Definition: Field}{\marginpar {
%\hspace{-0.5in}
%\begin{minipage}[t]{1in}
%\small{A field is a quantity that has a value (magnitude and/or direction) at every point in space.}
%\end{minipage}
%}} }%
%BeginExpansion
\marginpar {
\hspace{-0.5in}
\begin{minipage}[t]{1in}
\small{A field is a quantity that has a value (magnitude and/or direction) at every point in space.}
\end{minipage}
}
%EndExpansion
But we have to add one more complication. It is a vector, so it also has a
direction at each point in space as well. This direction is the direction
the force would be on $q_{o},$ the mover, if we placed it at that location.

But where does this field come from? We say that an environmental charge $%
Q_{E}$ creates a field 
\begin{equation}
\overrightarrow{\mathbf{E}}=k_{e}\frac{Q_{E}}{r^{2}}\mathbf{\hat{r}}
\end{equation}%
centered at the charge location. The field is our environment for our mover.

Now we can understand our classical model about how gravity works! Have you
wondered how a satellite knows that the Earth is there and that it should be
pulled toward the Earth? We can say the Earth sets up a \emph{gravitational
field} because it has mass. The gravitational field shows up as an
acceleration field. The satellite (the mover) feels the gravitational field
because the field exists at the location of the satellite (it exists at all
locations, so it exists at the satellite's location).%
%TCIMACRO{%
%\TeXButton{Question 223.21.7}{\marginpar {
%\hspace{-0.5in}
%\begin{minipage}[t]{1in}
%\small{Question 223.21.7}
%\end{minipage}
%}} }%
%BeginExpansion
\marginpar {
\hspace{-0.5in}
\begin{minipage}[t]{1in}
\small{Question 223.21.7}
\end{minipage}
}
%EndExpansion
The satellite does not have to know that the Earth is there, because it
feels the field right where it is. The satellite reacts to the field, not
directly with the Earth that created the field.\footnote{%
Here I\ am taking a quantum mechanical view of gravity. In General
Relativity, the \textquotedblleft field\textquotedblright\ is space that is
warped by the mass of the Earth.}

Likewise, our charge $Q_{E}$ has the property of creating an \emph{electric
field} as the environment around it. Other charges (movers) will feel the
field at their locations, and therefore will feel a force due to the field
created by $Q_{E}.$

\section{Field Lines}

We need a way to draw the environment created by the environmental charge $%
Q. $ We could draw lots of arrows like in the previous pictures. and we will
do this sometimes. But there is another way to draw the environment that has
become traditional. Have you ever taken iron filings and placed a magnet
near them? If you do, you will notice that the filings seem to line up.%
%TCIMACRO{%
%\TeXButton{Question 223.21.10}{\marginpar {
%\hspace{-0.5in}
%\begin{minipage}[t]{1in}
%\small{Question 223.21.10}
%\end{minipage}
%}}}%
%BeginExpansion
\marginpar {
\hspace{-0.5in}
\begin{minipage}[t]{1in}
\small{Question 223.21.10}
\end{minipage}
}%
%EndExpansion

%TCIMACRO{%
%\TeXButton{Magnet and Iron Filings}{\marginpar {
%\hspace{-0.5in}
%\begin{minipage}[t]{1in}
%\small{Magnet and Iron Filings}
%\end{minipage}
%}}}%
%BeginExpansion
\marginpar {
\hspace{-0.5in}
\begin{minipage}[t]{1in}
\small{Magnet and Iron Filings}
\end{minipage}
}%
%EndExpansion
If you took PH121 you probably heard that there is a magnetic force. It is a
non-contact force, so we expect it has a \emph{magnetic field}. The iron
filings are aligning because they are acted upon by the field. It is natural
to represent this field as a series of lines like the ones formed by the
iron filings. We will do this in a few lectures!

But there is a similar experiment we can do with the electric force. This is
harder, but we can use small seeds or pieces of thread suspended in oil.
These small things become polarized in an electric field. They line up like
the iron filings. \FRAME{dhFU}{2.6057in}{2.0989in}{0pt}{\Qcb{%
http://stargazers.gsfc.nasa.gov/images/geospace\_images/electricity/elec%
\_field\_lines.jpg}}{}{Figure}{\special{language "Scientific Word";type
"GRAPHIC";maintain-aspect-ratio TRUE;display "USEDEF";valid_file "T";width
2.6057in;height 2.0989in;depth 0pt;original-width 4.1113in;original-height
3.3053in;cropleft "0";croptop "1";cropright "1";cropbottom "0";tempfilename
'NBC9FB0M.bmp';tempfile-properties "XPR";}}We can represent the electric
field by tracing out these lines. The last figure would look like this\FRAME{%
dhF}{3.6115in}{1.5912in}{0pt}{}{}{Figure}{\special{language "Scientific
Word";type "GRAPHIC";maintain-aspect-ratio TRUE;display "USEDEF";valid_file
"T";width 3.6115in;height 1.5912in;depth 0pt;original-width
6.4921in;original-height 2.8444in;cropleft "0";croptop "1";cropright
"1";cropbottom "0";tempfilename 'NBC9FB0N.wmf';tempfile-properties "XPR";}}%
We can't tell if the charge was negative or positive from oil suspension
picture, but if it was positive, by convention we draw the field lines as
coming out of the charge. If it were negative the field lines would be drawn
as going in to the charge. Here is a combination of a negative and a
positive charge or dipole. \FRAME{dhFU}{2.789in}{1.721in}{0pt}{\Qcb{%
http://stargazers.gsfc.nasa.gov/images/geospace\_images/electricity/elec%
\_field\_lines2.jpg}}{}{Figure}{\special{language "Scientific Word";type
"GRAPHIC";maintain-aspect-ratio TRUE;display "USEDEF";valid_file "T";width
2.789in;height 1.721in;depth 0pt;original-width 5.0332in;original-height
3.0943in;cropleft "0";croptop "1";cropright "1";cropbottom "0";tempfilename
'NBC9FB0O.wmf';tempfile-properties "XPR";}}In this case both the positive
and negative charges are working together to make the environment or field
that a third charge could move through. The field line drawing would look
like this.\FRAME{dhF}{3.0191in}{1.9389in}{0pt}{}{}{Figure}{\special{language
"Scientific Word";type "GRAPHIC";maintain-aspect-ratio TRUE;display
"USEDEF";valid_file "T";width 3.0191in;height 1.9389in;depth
0pt;original-width 2.975in;original-height 1.9009in;cropleft "0";croptop
"1";cropright "1";cropbottom "0";tempfilename
'NBC9FB0P.wmf';tempfile-properties "XPR";}}This combination of positive and
negative charges had equal charges, the only difference was the sign change.
Here is one where the positive charge has more charge than the negative
charge.\FRAME{dhF}{2.9213in}{2.7069in}{0pt}{}{}{Figure}{\special{language
"Scientific Word";type "GRAPHIC";maintain-aspect-ratio TRUE;display
"USEDEF";valid_file "T";width 2.9213in;height 2.7069in;depth
0pt;original-width 2.8772in;original-height 2.6654in;cropleft "0";croptop
"1";cropright "1";cropbottom "0";tempfilename
'NBC9FB0Q.wmf';tempfile-properties "XPR";}}Notice that the number of field
lines is proportional to the field, but there is no set proportionality. If
the field from one charge is twice that of the other, we pick a number of
field lines for, say, the negative charge, and double the lines on the
larger positive charge.

This gives us a way to picture the electric field in our minds!

Some things to notice:

\begin{enumerate}
\item The lines begin on positive charges

\item The lines end on negative charges

\item If you don't have matching charges, the lines end infinitely far away
(like the single charges in the first picture).

\item Larger charges have more lines coming from them

\item Field lines cannot cross each other%
%TCIMACRO{%
%\TeXButton{Question 223.21.11}{\marginpar {
%\hspace{-0.5in}
%\begin{minipage}[t]{1in}
%\small{Question 223.21.11}
%\end{minipage}
%}}}%
%BeginExpansion
\marginpar {
\hspace{-0.5in}
\begin{minipage}[t]{1in}
\small{Question 223.21.11}
\end{minipage}
}%
%EndExpansion

\item The lines are only imaginary, they are a way to form a mental picture
of the field.
\end{enumerate}

We only draw the field lines for the environmental charges. Of course the
mover charge also makes a field, but this self-field can't cause the mover
charge to move. If it could we could have perpetual motion and that violates
the second law of thermodynamics. Since the mover's self-field is not
participating in making the motion, we won't take the time to draw it!%
\footnote{%
This picture will be a little more complicated when we allow for
relativistic motion of charges and other more difficult effects, but that
can wait for more advanced physics courses. For most engineering
applications, this is a great approximation.}

Remember, field lines are not real, but are a nice way to draw the field
made by the environmental charge. We will use field lines often in drawing
pictures as part of our problem solving process.

\section{On-Line demonstrations}

An applet that demonstrates the electric field of point charges can be found
here:

http://phet.colorado.edu/sims/charges-and-fields/charges-and-fields\_en.html

If you prefer a video game, try Electric Field Hockey:

http://phet.colorado.edu/en/simulation/electric-hockey

As a wacky example of Coulomb forces, see this video of charged water
droplets orbiting charged knitting needles on the Space Shuttle:

http://www.nasa.gov/multimedia/videogallery/index.html?media\_id=131554451

%TCIMACRO{%
%\TeXButton{Basic Equations}{\hspace{-1.3in}{\LARGE Basic Equations\vspace{0.25in}}}}%
%BeginExpansion
\hspace{-1.3in}{\LARGE Basic Equations\vspace{0.25in}}%
%EndExpansion

\chapter{Electric Fields of Standard Charge Configurations Part I}

%TCIMACRO{%
%\TeXButton{Fundamental Concepts}{\hspace{-1.3in}{\LARGE Fundamental Concepts\vspace{0.25in}}}}%
%BeginExpansion
\hspace{-1.3in}{\LARGE Fundamental Concepts\vspace{0.25in}}%
%EndExpansion

\begin{itemize}
\item Adding of vector fields for point charges

\item Standard configurations of charge
\end{itemize}

\section{Standard Charge Configurations}

Actual engineering projects or experimental designs require detailed
calculations of fields using computers. These field simulations use powerful
numerical techniques that are beyond this sophomore class. But we can gain
some great insight by using some basic models of simple charged objects. We
will often look at the following models:

\begin{equation*}
\begin{tabular}{|c|}
\hline
\textbf{Standard Configurations of Charge} \\ \hline
Point charge \\ \hline
Several point charges \\ \hline
Line of Charge \\ \hline
Semi-infinite sheet of charge \\ \hline
Charged sphere \\ \hline
Charged spherical shell \\ \hline
Ring of Charge \\ \hline
\end{tabular}%
\end{equation*}

\section{Point Charges}

We have already met one of these standard configurations, the point charge%
\begin{equation*}
\overrightarrow{\mathbf{E}}=\frac{1}{4\pi \epsilon _{o}}\frac{q}{r^{2}}%
\mathbf{\hat{r}}
\end{equation*}%
The field of the point charge is represented below\FRAME{dhF}{2.4171in}{%
2.4137in}{0pt}{}{}{Figure}{\special{language "Scientific Word";type
"GRAPHIC";maintain-aspect-ratio TRUE;display "USEDEF";valid_file "T";width
2.4171in;height 2.4137in;depth 0pt;original-width 2.3765in;original-height
2.373in;cropleft "0";croptop "1";cropright "1";cropbottom "0";tempfilename
'LZ6Q7I00.wmf';tempfile-properties "XPR";}}This picture requires a little
explanation. The arrows are larger nearer the charge to show that the field
is stronger. But note that each arrow is the magnitude and direction of the
charge at one point. We really need a three dimensional picture to describe
this, and even then the fact that the arrows have length can be misleading.
The long arrows cover up other points, that should also have arrows. We can
only draw the field at a few points, and at those points the field has both
magnitude and direction. But we must remember that there is really a field
magnitude and direction at every point.

To go beyond single charges we need a group of point charges of some sort.
The fields add like forces%
\begin{equation*}
\overrightarrow{\mathbf{E}}=\sum_{i}\overrightarrow{\mathbf{E}}_{i}
\end{equation*}%
were we recognize that we are summing vectors. Let's take a look at a few
combinations of charges and find their fields

\subsection{Two charges}

Let's go back to our idea of an environmental charge, $Q_{E},$ and a mover
charge, $q_{m}$. The mover charge is considered to be small enough that its
effect on $Q_{E}$ is negligible. So the field due to the large charge is
unaffected by this small charge.

Of course, the total field is a superposition of both fields. We call the
field produced by the little mover charge it's \emph{self-field}. But the
mover charge can't move itself.\footnote{%
This would allow perpetual motion, breaking the second law of thermodynamics.%
} The mover's self-field can't move the mover. So we don't draw the field
due to $q_{m}$. We we can envision an environmental field that is just due
to the environmental charge, $Q_{E},$ as if there are no other charges any
where in the whole universe. Of course this is not the case, but this is how
we think of the field \emph{due to} charge $Q_{E}.$%
%TCIMACRO{%
%\TeXButton{Question 223.22.1}{\marginpar {
%\hspace{-0.5in}
%\begin{minipage}[t]{1in}
%\small{Question 223.22.1}
%\end{minipage}
%}}}%
%BeginExpansion
\marginpar {
\hspace{-0.5in}
\begin{minipage}[t]{1in}
\small{Question 223.22.1}
\end{minipage}
}%
%EndExpansion
%TCIMACRO{%
%\TeXButton{Question 223.22.2}{\marginpar {
%\hspace{-0.5in}
%\begin{minipage}[t]{1in}
%\small{Question 223.22.2}
%\end{minipage}
%}}}%
%BeginExpansion
\marginpar {
\hspace{-0.5in}
\begin{minipage}[t]{1in}
\small{Question 223.22.2}
\end{minipage}
}%
%EndExpansion

We can identify that a charge $q_{m}$ placed in this field due to $Q_{E}$
will feel a force 
\begin{eqnarray*}
\overrightarrow{\mathbf{F}}_{e} &\mathbf{=}&q_{m}\overrightarrow{\mathbf{E}}
\\
&=&\frac{1}{4\pi \epsilon _{o}}\frac{Q_{E}q_{m}}{r^{2}}\mathbf{\hat{r}}
\end{eqnarray*}%
due to the field 
\begin{equation*}
\overrightarrow{\mathbf{E}}=\frac{1}{4\pi \epsilon _{o}}\frac{Q_{E}}{r^{2}}%
\mathbf{\hat{r}}
\end{equation*}%
where this field is just due to $Q_{E}$ and does not contain the
contribution from $q_{m}.$ So the charge $q_{m}$ only feels a force due to
the field created by charge $Q_{E}.$ A third charge, $q_{new}$ brought close
to the other two would feel both $\overrightarrow{\mathbf{E}}_{Q}$ and $%
\overrightarrow{\mathbf{E}}_{q}.$ Then both $Q_{E}$ and our original $q_{m}$
would be environmental charges and the new charge $q_{new}$ would be the
mover. At this point, we would probably relabel $Q_{E}$ and $q_{m}$ as $%
Q_{1} $ and $Q_{2}$ and relabel $q_{new}$ as $q_{m}$ so we could tell that
the original two charges are now the environment and the new charge is the
mover.

\subsection{Vector nature of the field}

%TCIMACRO{%
%\TeXButton{Question 223.22.3}{\marginpar {
%\hspace{-0.5in}
%\begin{minipage}[t]{1in}
%\small{Question 223.22.3}
%\end{minipage}
%}}}%
%BeginExpansion
\marginpar {
\hspace{-0.5in}
\begin{minipage}[t]{1in}
\small{Question 223.22.3}
\end{minipage}
}%
%EndExpansion
%TCIMACRO{%
%\TeXButton{Question 223.22.4}{\marginpar {
%\hspace{-0.5in}
%\begin{minipage}[t]{1in}
%\small{Question 223.22.4}
%\end{minipage}
%}}}%
%BeginExpansion
\marginpar {
\hspace{-0.5in}
\begin{minipage}[t]{1in}
\small{Question 223.22.4}
\end{minipage}
}%
%EndExpansion
Remember that the field is a force per unit charge. Forces add as vectors,
so we should expect fields to add as vectors too. Let's do a problem.\FRAME{%
dtbpF}{3.3288in}{3.3705in}{0in}{}{}{Figure}{\special{language "Scientific
Word";type "GRAPHIC";maintain-aspect-ratio TRUE;display "USEDEF";valid_file
"T";width 3.3288in;height 3.3705in;depth 0in;original-width
3.3634in;original-height 3.4069in;cropleft "0";croptop "1";cropright
"1";cropbottom "0";tempfilename 'NBC9FB0R.wmf';tempfile-properties "XPR";}}

Two charges are separated by a distance $d.$ What is the field a distance $L$
from the center of the two charges?

We should recognize this as our old friend, the dipole.

Note that both of these charges are environmental charges. We are asked in
this problem to find the environment, the field. We don't really have a
mover charge. But we could pretend we do have a mover, $q_{o}$ at point $P$
where we want to know the environment if it helps us picture the situation.
But really we are calculating what the environment around the two charges
will be.

We start by drawing the situation. I\ chose not to draw field lines. Instead
I drew the field vectors at the point, $P,$ where we want the field. The
field lines would tell me about the whole environment everywhere, and that
might be useful. But this problem only wants to know the field at one point, 
$P.$ So it was less work to draw the field using vectors at our one point.%
\FRAME{dtbpF}{3.7856in}{3.579in}{0pt}{}{}{Figure}{\special{language
"Scientific Word";type "GRAPHIC";maintain-aspect-ratio TRUE;display
"USEDEF";valid_file "T";width 3.7856in;height 3.579in;depth
0pt;original-width 3.8291in;original-height 3.6189in;cropleft "0";croptop
"1";cropright "1";cropbottom "0";tempfilename
'NBC9FB0S.wmf';tempfile-properties "XPR";}}Note that I\ need a vector for
each of the environmental charges. Each contributes to the environment. The
contribution to the field due to environmental charge $q_{1}$ is labeled $%
E_{1}$ and likewise the contribution to the field from environmental charge $%
q_{2}$ is labeled $E_{2}.$\FRAME{dtbpF}{3.5036in}{3.3138in}{0pt}{}{}{Figure}{%
\special{language "Scientific Word";type "GRAPHIC";maintain-aspect-ratio
TRUE;display "USEDEF";valid_file "T";width 3.5036in;height 3.3138in;depth
0pt;original-width 3.4566in;original-height 3.2681in;cropleft "0";croptop
"1";cropright "1";cropbottom "0";tempfilename
'NBC9FB0T.wmf';tempfile-properties "XPR";}}The net environment is the
superposition of the fields due to each of the environmental charges. 
\begin{equation*}
\overrightarrow{\mathbf{E}}_{net}=\overrightarrow{\mathbf{E}}_{1}+%
\overrightarrow{\mathbf{E}}_{3}
\end{equation*}

From the figure, we see that if we had a small mover charge, $q_{o}$ on the
axis a distance at point $p$ then we would get two forces, one from each of
the environmental charges $q_{1}$ and $q_{2}$. We can use Newton's second
law to find the net force on our imaginary $q_{o}.$%
\begin{eqnarray*}
F_{net_{z}} &=&ma_{z}=-F_{1}\cos \theta +F_{2}\cos \theta \\
F_{net_{y}} &=&ma_{y}=F_{1}\sin \theta +F_{2}\sin \theta
\end{eqnarray*}%
we can see that the distance from each charge to point $P$ is 
\begin{equation*}
r=\sqrt{\frac{d^{2}}{4}+L^{2}}
\end{equation*}%
so%
\begin{equation*}
\sin \theta =\frac{d}{2\sqrt{\frac{d^{2}}{4}+L^{2}}}
\end{equation*}%
we also know from Coulomb's law that 
\begin{equation*}
F_{1}=F_{2}=\frac{1}{4\pi \epsilon _{o}}\frac{qq_{o}}{r^{2}}
\end{equation*}%
but we want the field, so we need to divide all of this by $q_{o}$%
\begin{equation*}
E_{1}=E_{2}=\frac{1}{4\pi \epsilon _{o}}\frac{q}{r^{2}}
\end{equation*}%
Our Newton's second law becomes an equation for the components of the
combined electric field.%
\begin{eqnarray*}
\frac{F_{net_{z}}}{q_{o}} &=&-\frac{F_{1}}{q_{o}}\cos \theta +\frac{F_{2}}{%
q_{o}}\cos \theta \\
\frac{F_{net_{y}}}{q_{o}} &=&\frac{F_{1}}{q_{o}}\sin \theta +\frac{F_{2}}{%
q_{o}}\sin \theta
\end{eqnarray*}%
or just%
\begin{eqnarray*}
E_{net_{z}} &=&-E_{1}\cos \theta +E_{2}\cos \theta \\
E_{net_{y}} &=&E_{1}\sin \theta +E_{2}\sin \theta
\end{eqnarray*}%
We can see from the figure that in the $x$-direction we will have no net
field, 
\begin{equation*}
E_{z}=-E_{1}\cos \left( \theta \right) +E_{1}\cos \theta =0
\end{equation*}%
But in the $y$-direction we have%
\begin{eqnarray*}
E_{y} &=&E_{1}\sin \theta +E_{2}\sin \theta \\
&=&2E_{1}\sin \theta \\
&=&\frac{2}{4\pi \epsilon _{o}}\frac{q}{r^{2}}\sin \theta
\end{eqnarray*}%
and since we found that 
\begin{equation*}
\sin \theta =\frac{d}{2\sqrt{\frac{d^{2}}{4}+L^{2}}}
\end{equation*}%
we can write our field as%
\begin{eqnarray*}
E_{y} &=&\frac{2}{4\pi \epsilon _{o}}\frac{q}{\frac{d^{2}}{4}+L^{2}}\frac{d}{%
2\sqrt{\frac{d^{2}}{4}+L^{2}}} \\
&=&\frac{1}{4\pi \epsilon _{o}}\frac{qd}{\left( \frac{d^{2}}{4}+L^{2}\right)
^{\frac{3}{2}}}
\end{eqnarray*}%
This is our total field at the distance $L$ away on the axis. This is the
environment that a mover charge could move through.

Note that we pretended that we had a mover, $q_{o}$, but in finding the
field the $q_{o}$ canceled out, so indeed we are left with just the
environment in our calculation, we just have the field.

Now suppose our mover charge is very far away. That is, suppose we make $L$
very large. So large that $L\gg d$ then 
\begin{equation*}
\underset{L\gg d}{\lim }\frac{1}{\left( \frac{d^{2}}{4}+L^{2}\right) ^{\frac{%
3}{2}}}=\frac{1}{L^{3}}
\end{equation*}%
Then our field becomes%
\begin{eqnarray*}
E &=&E_{y}=\frac{2}{4\pi \epsilon _{o}}\frac{qd}{\left( \frac{d^{2}}{4}%
+L^{2}\right) ^{\frac{3}{2}}} \\
&=&\frac{1}{4\pi \epsilon _{o}}\frac{qd}{L^{3}}
\end{eqnarray*}%
Since many charged particles are small, like atoms or molecules, this limit
is often useful.

Now suppose we repeat the calculation, but this time we chose a point that
is $L$ away, but that is on the $y$-axis above the charges, we would find%
\begin{equation*}
E=E_{y}=\frac{2}{4\pi \epsilon _{o}}\frac{qd}{L^{3}}
\end{equation*}%
\FRAME{dhF}{2.5434in}{3.0554in}{0pt}{}{}{Figure}{\special{language
"Scientific Word";type "GRAPHIC";maintain-aspect-ratio TRUE;display
"USEDEF";valid_file "T";width 2.5434in;height 3.0554in;depth
0pt;original-width 2.5028in;original-height 3.0113in;cropleft "0";croptop
"1";cropright "1";cropbottom "0";tempfilename
'NBC9FC0U.wmf';tempfile-properties "XPR";}}The result is similar, but the
field is a little stronger in this direction.

Let's look at one of these cases by graphing it. \FRAME{dtbpFX}{4.4997in}{%
1.389in}{0pt}{}{}{Plot}{\special{language "Scientific Word";type
"MAPLEPLOT";width 4.4997in;height 1.389in;depth 0pt;display
"USEDEF";plot_snapshots TRUE;mustRecompute FALSE;lastEngine "MuPAD";xmin
"-1.010005E-7";xmax "0.00001000101";xviewmin "-1.010005E-7";xviewmax
"0.00001000101";yviewmin "0";yviewmax "100";viewset"XY";rangeset"X";plottype
4;labeloverrides 3;x-label "r (m)";y-label "E (N/C)";axesFont "Times New
Roman,12,0000000000,useDefault,normal";numpoints 100;plotstyle
"patch";axesstyle "normal";axestips FALSE;xis \TEXUX{x};var1name
\TEXUX{$x$};function \TEXUX{$\frac{1.\,\allowbreak 438\,7\times
10^{-9}}{x^{2}}$};linecolor "blue";linestyle 2;pointstyle
"point";linethickness 3;lineAttributes "Dash";var1range
"-1.010005E-7,0.00001000101";num-x-gridlines 100;curveColor
"[flat::RGB:0x000000ff]";curveStyle "Line";function
\TEXUX{$\frac{1.\,\allowbreak 525\times 10^{-19}}{x^{3}}$};linecolor
"green";linestyle 1;pointstyle "point";linethickness 3;lineAttributes
"Solid";var1range "-1.010005E-7,0.00001000101";num-x-gridlines
100;curveColor "[flat::RGB:0x00008000]";curveStyle "Line";VCamFile
'NDLO6700.xvz';valid_file "T";tempfilename
'NBC9FC0V.wmf';tempfile-properties "XPR";}}We can see that the dipole field
(solid green line) falls off much faster than a point charge field (dashed
red line). This makes sense because the farther away we get, the more it
looks like the two charges are right next to each other, and since they are
opposite in sign, they are essentially neutral when viewed together from far
away. We can see why atoms don't exhibit a significant charge forces at
normal distances.

This arrangement of charges we already know as a dipole. We are treating the
two charges as a unit making the environment that other charges might move
in. Since we are treating the two charges as one unit, it is customary to
define a quantity 
\begin{equation*}
p=qd
\end{equation*}%
and to make this a vector by defining the direction of $p$ to be from the
negative to the positive charge along the axis.\FRAME{dhF}{0.8536in}{0.9478in%
}{0pt}{}{}{Figure}{\special{language "Scientific Word";type
"GRAPHIC";maintain-aspect-ratio TRUE;display "USEDEF";valid_file "T";width
0.8536in;height 0.9478in;depth 0pt;original-width 2.0721in;original-height
2.3039in;cropleft "0";croptop "1";cropright "1";cropbottom "0";tempfilename
'NBC9FC0W.wmf';tempfile-properties "XPR";}}Then we can write the dipole
field as 
\begin{equation*}
\overrightarrow{\mathbf{E}}_{y}=\frac{2}{4\pi \epsilon _{o}}\frac{%
\overrightarrow{\mathbf{p}}}{L^{3}}
\end{equation*}%
We could also treat this dipole as a complicated mover charge in some other
environmental field!. Then this quantity $\overrightarrow{\mathbf{p}}$ will
help us understand how a dipole will move when placed in an environmental
electric field. For example, we know that water molecules are dipoles. A
microwave oven creates a strong environmental electric field that makes the
water molecules rotate. When we studied rotational motion we found a
mass-like term that helped us to know how difficult something was to make
rotate. That was the moment of inertia. This dipole term, $\overrightarrow{%
\mathbf{p}},$ will tell us how likely a dipole is to spin, so we will call $%
\overrightarrow{\mathbf{p}}$ the \emph{dipole moment}.

\subsection{Three charges}

\FRAME{dtbpF}{4.1041in}{3.4007in}{0in}{}{}{Figure}{\special{language
"Scientific Word";type "GRAPHIC";maintain-aspect-ratio TRUE;display
"USEDEF";valid_file "T";width 4.1041in;height 3.4007in;depth
0in;original-width 4.1537in;original-height 3.4371in;cropleft "0";croptop
"1";cropright "1";cropbottom "0";tempfilename
'NBC9FC0X.wmf';tempfile-properties "XPR";}}

%TCIMACRO{%
%\TeXButton{Question 223.22.5}{\marginpar {
%\hspace{-0.5in}
%\begin{minipage}[t]{1in}
%\small{Question 223.22.5}
%\end{minipage}
%}}}%
%BeginExpansion
\marginpar {
\hspace{-0.5in}
\begin{minipage}[t]{1in}
\small{Question 223.22.5}
\end{minipage}
}%
%EndExpansion
We are working our way toward many charges that will require using
integration to sum up the contributions to the field. But let's make this
transition slowly. Next let's add just one more environmental charge, for a
total of three.

Let's just start with the fields this time. From our picture, we expect in
this case to have only $z$-components. Since all the charges are the same
sign, \FRAME{dtbpF}{4.2735in}{3.3324in}{0pt}{}{}{Figure}{\special{language
"Scientific Word";type "GRAPHIC";maintain-aspect-ratio TRUE;display
"USEDEF";valid_file "T";width 4.2735in;height 3.3324in;depth
0pt;original-width 4.3267in;original-height 3.3679in;cropleft "0";croptop
"1";cropright "1";cropbottom "0";tempfilename
'NBC9FC0Y.wmf';tempfile-properties "XPR";}}then%
\begin{equation*}
E_{net_{z}}=E_{1}\cos \left( -\theta \right) +E_{2}+E_{3}\cos \left( \theta
\right)
\end{equation*}%
We can guess from symmetry that 
\begin{equation*}
E_{1}=E_{3}=\frac{1}{4\pi \epsilon _{o}}\frac{q}{r^{2}}
\end{equation*}%
But this time, since we have redefined $d,$ the distance from $q_{1}$ and $%
q_{3}$ to the point $P$ where we want to know the field is 
\begin{equation*}
r=\sqrt{d^{2}+L^{2}}
\end{equation*}%
so%
\begin{equation*}
E_{1}=E_{3}=\frac{1}{4\pi \epsilon _{o}}\frac{q}{d^{2}+L^{2}}
\end{equation*}%
and 
\begin{equation*}
E_{2}=\frac{1}{4\pi \epsilon _{o}}\frac{q}{L^{2}}
\end{equation*}%
and observing the triangles formed and remembering our trigonometry, we have 
\begin{equation*}
\cos \theta =\frac{L}{\sqrt{d^{2}+L^{2}}}
\end{equation*}%
so%
\begin{eqnarray*}
E_{z} &=&\frac{1}{4\pi \epsilon _{o}}\frac{q}{d^{2}+L^{2}}\frac{L}{\sqrt{%
d^{2}+L^{2}}} \\
&&+\frac{1}{4\pi \epsilon _{o}}\frac{q}{L^{2}} \\
&&+\frac{1}{4\pi \epsilon _{o}}\frac{q}{d^{2}+L^{2}}\frac{L}{\sqrt{%
d^{2}+L^{2}}}
\end{eqnarray*}%
or%
\begin{equation*}
E_{z}=\frac{q}{4\pi \epsilon _{o}}\left( \frac{2L}{\left( d^{2}+L^{2}\right)
^{\frac{3}{2}}}+\frac{1}{L^{2}}\right)
\end{equation*}%
This is our answer.

Once again let's consider the limit $L\gg d.$ If our answer is right, when
we get very far from the group of charges they should look like a single
charge with the amount of charge being the sum of all three environmental
charges. In this limit 
\begin{equation*}
\underset{L\gg d}{\lim }\frac{1}{\left( d^{2}+L^{2}\right) ^{\frac{3}{2}}}=%
\frac{1}{L^{3}}
\end{equation*}%
so%
\begin{eqnarray*}
E_{z} &\approx &\frac{q}{4\pi \epsilon _{o}}\left( \frac{2L}{L^{3}}+\frac{1}{%
L^{2}}\right) \\
&=&\frac{1}{4\pi \epsilon _{o}}\left( \frac{3q}{L^{2}}\right)
\end{eqnarray*}%
so on the central axis%
\begin{equation*}
\overrightarrow{\mathbf{E}}\approx \frac{1}{4\pi \epsilon _{o}}\left( \frac{%
3q}{L^{2}}\right) \mathbf{\hat{k}}
\end{equation*}%
And indeed, this is very like one charge that is three times as large as our
actual charges if we get far enough away.

This shows us a pattern we will often see. Far away, our field looks like
what we would expect if the net charge were all congregated in a point. Near
the charges, we must calculate the superposition of the fields. But far away
we can treat the distribution as a point charge. This is very like what we
did with mass in PH121 or Dynamics. We could often treat masses as point
masses at the center of mass, if the distances involved were larger than the
mass sizes.%
%TCIMACRO{%
%\TeXButton{Question 223.22.6}{\marginpar {
%\hspace{-0.5in}
%\begin{minipage}[t]{1in}
%\small{Question 223.22.6}
%\end{minipage}
%}}}%
%BeginExpansion
\marginpar {
\hspace{-0.5in}
\begin{minipage}[t]{1in}
\small{Question 223.22.6}
\end{minipage}
}%
%EndExpansion

\section{Combinations of many charges}

We have found the field from a point charge.%
\begin{equation}
\overrightarrow{\mathbf{E}}=\frac{1}{4\pi \epsilon _{o}}\frac{q_{E}}{r^{2}}%
\mathbf{\hat{r}}
\end{equation}%
where the field is in the same direction as $\hat{r}$ if the charge is
positive, and in the opposite direction if the charge is negative (think of
our field lines, they go toward the negative charge). This will become one
of a group of standard charge configurations that we will use to gain a
mental picture of complex configurations of charge. We have done this
already for combinations of point charges. We can combine the point charge
fields to get the total field.

The other standard models are combinations of many, many charges.

\subsection{Line of Charge}

Another is an infinitely long line of charge, or a infinite charged wire.
Since this long line of charge is infinite, it must have an infinite amount
of charge. But we can describe \textquotedblleft how much\textquotedblright\
charge it has with a linear charge density%
\begin{equation*}
\lambda =\frac{Q}{L}
\end{equation*}%
\FRAME{dhF}{4.8222in}{0.5025in}{0pt}{}{}{Figure}{\special{language
"Scientific Word";type "GRAPHIC";maintain-aspect-ratio TRUE;display
"USEDEF";valid_file "T";width 4.8222in;height 0.5025in;depth
0pt;original-width 4.7686in;original-height 0.4722in;cropleft "0";croptop
"1";cropright "1";cropbottom "0";tempfilename
'LTUWDD8G.wmf';tempfile-properties "XPR";}}

\subsection{Semi-infinite sheet of charge}

A sheet or plane of charge,usually a semi-infinite sheet of charge is also
useful\FRAME{dhF}{3.851in}{1.6942in}{0pt}{}{}{Figure}{\special{language
"Scientific Word";type "GRAPHIC";maintain-aspect-ratio TRUE;display
"USEDEF";valid_file "T";width 3.851in;height 1.6942in;depth
0pt;original-width 4.894in;original-height 2.1369in;cropleft "0";croptop
"1";cropright "1";cropbottom "0";tempfilename
'LTUWDD8H.wmf';tempfile-properties "XPR";}}We have the same problem of
having infinite charge, but if we define an amount of charge per unit area%
\begin{equation*}
\eta =\frac{Q}{A}
\end{equation*}%
we can compare sheets that are more charge rich than others.

\subsection{Sphere of charge}

Finally, we have drawn a sphere of charge already\FRAME{dhF}{1.8031in}{%
1.5489in}{0pt}{}{}{Figure}{\special{language "Scientific Word";type
"GRAPHIC";maintain-aspect-ratio TRUE;display "USEDEF";valid_file "T";width
1.8031in;height 1.5489in;depth 0pt;original-width 1.7659in;original-height
1.5134in;cropleft "0";croptop "1";cropright "1";cropbottom "0";tempfilename
'LTUWDD8I.wmf';tempfile-properties "XPR";}}

We can define an amount of charge per unit volume to help describe this
distribution%
\begin{equation*}
\rho =\frac{Q}{V}
\end{equation*}%
The spherical shell of charge is related to a sheet of charge, so we will
include it here\FRAME{dhF}{1.6215in}{1.4927in}{0pt}{}{}{Figure}{\special%
{language "Scientific Word";type "GRAPHIC";maintain-aspect-ratio
TRUE;display "USEDEF";valid_file "T";width 1.6215in;height 1.4927in;depth
0pt;original-width 1.5852in;original-height 1.4572in;cropleft "0";croptop
"1";cropright "1";cropbottom "0";tempfilename
'LTUWDD8J.wmf';tempfile-properties "XPR";}}This configuration of charge is
drawn in cross section like the others. From your calculus experience you
can guess that a spherical shell of charge with a certain volume charge
density might be useful in integration, but we also can easily produce such
a configuration of charge by charging a round balloon or a spherical
conductor.

\FRAME{dhF}{2.5996in}{1.6042in}{0pt}{}{}{Figure}{\special{language
"Scientific Word";type "GRAPHIC";maintain-aspect-ratio TRUE;display
"USEDEF";valid_file "T";width 2.5996in;height 1.6042in;depth
0pt;original-width 2.5581in;original-height 1.5679in;cropleft "0";croptop
"1";cropright "1";cropbottom "0";tempfilename
'LTUWDD8K.wmf';tempfile-properties "XPR";}}

The ring of charge is similar to the spherical shell, but is also much like
the line of charge.

In our next lecture, we will take on the job of finding the fields that
result from these last few charge configurations except the spherical shell,
which will have to wait a few lectures.

\section{On-Line Visualizations}

For a 2D visualization of the field try:

http://www.falstad.com/emstatic/index.html

And here is a 3D visualization:

http://www.falstad.com/vector3de/

%TCIMACRO{%
%\TeXButton{Basic Equations}{\hspace{-1.3in}{\LARGE Basic Equations\vspace{0.25in}}}}%
%BeginExpansion
\hspace{-1.3in}{\LARGE Basic Equations\vspace{0.25in}}%
%EndExpansion

\chapter{Electric Fields of Standard Charge Configurations Part II}

%TCIMACRO{%
%\TeXButton{Fundamental Concepts}{\hspace{-1.3in}{\LARGE Fundamental Concepts\vspace{0.25in}}}}%
%BeginExpansion
\hspace{-1.3in}{\LARGE Fundamental Concepts\vspace{0.25in}}%
%EndExpansion

\begin{itemize}
\item Integrating vector fields for continuous distributions of charge

\begin{itemize}
\item Start with $\overrightarrow{\mathbf{E}}=\frac{1}{4\pi \epsilon _{o}}%
\int \frac{dq}{r^{2}}\mathbf{\hat{r}}$

\item Find an expression for $dq$

\item Use geometry to find expressions for $r$ and to eliminate $\mathbf{%
\hat{r}}$

\item Solve the integral
\end{itemize}
\end{itemize}

\section{Fields from Continuous Charge Distributions}

%TCIMACRO{%
%\TeXButton{Question 223.23.1}{\marginpar {
%\hspace{-0.5in}
%\begin{minipage}[t]{1in}
%\small{Question 223.23.1}
%\end{minipage}
%}}}%
%BeginExpansion
\marginpar {
\hspace{-0.5in}
\begin{minipage}[t]{1in}
\small{Question 223.23.1}
\end{minipage}
}%
%EndExpansion
Suppose we have a continuous distribution of charge with some mover charge $%
q_{m}$ fairly far away. You might ask, how do we get a continuous
distribution of charge? After all, charge seems to be quantized. Well, if we
have a collection of charges where the distances between the individual
charge carriers are much smaller that the distance from the whole collection
of charges to some point where we want to measure the field (where the mover
charge might be), then in our field calculations at this distant point we
can model the charge distribution causing the field as continuous. As an
analogy, think of your computer screen. It is really a collection of dots of
light. But if we are a few feet away, we see a continuous picture. We can
treat the dots as though there were no space in between them. For our
continuous charge model, it is the same. We are supposing we are observing
from far enough away that we won't notice the effects of the charges being
separated by small distances.

We should remember, though, that this is a macroscopic view. At some point
it must break down, since charge is carried in discrete amounts. If we want
the field very close to a distribution of charges, we must treat our charge
distribution as a collection of individual charges like we did in the last
lecture. Notice in our last lecture that we found that the field infinitely
far from the charges was always zero. That is too far away for our
continuous charge model to be useful. But if we went far enough away--but
not too far, the three charge configuration looked like a point charge with
a total charge that was the sum of the individual charges. At such
distances, the separation between the charges become unimportant. This is
the sort of large distance we are talking about in our continuous charge
distribution model.

To find the field due to a continuous charge distribution, we break up the
charged object into small pieces in a calculus sense. Each small piece is
still a continuous distribution of charge. It will have an amount of charge $%
\Delta q_{E},$ where here the $\Delta $ means \textquotedblleft a small
amount of.\textquotedblright\ Then we calculate the field due to this
element of charge. We repeat the process for each element using the
superposition principle to sum up all the individual field contributions.
This is very like our method of finding the field from individual charges,
only instead of a sum we want to let $\Delta q_{E}$ become very small and
use an integral. The field due to this bit of charge is%
\begin{equation*}
\Delta \overrightarrow{\mathbf{E}}=\frac{1}{4\pi \epsilon _{o}}\frac{\Delta
q_{E}}{r^{2}}\mathbf{\hat{r}}
\end{equation*}%
Recall that here $\Delta $ means \textquotedblleft a small bit
of\textquotedblright\ and is not a difference between two charge values or
two field values. We learned that we can sum up the fields from each piece%
\begin{eqnarray*}
\overrightarrow{\mathbf{E}} &\approx &\sum_{i}\Delta \overrightarrow{\mathbf{%
E}}_{i} \\
&\mathbf{\approx }&\frac{1}{4\pi \epsilon _{o}}\dsum\limits_{i}\frac{\Delta
q_{i}}{r_{i}^{2}}\mathbf{\hat{r}}_{i}
\end{eqnarray*}%
and now we use our M215 (or M113) tricks to convert this into an integral.
We let our small element of charge become very small (but not so small that
we violate our assumption that the charge distribution of $\Delta q_{E}$ is
continuous).%
\begin{eqnarray*}
\overrightarrow{\mathbf{E}} &\mathbf{=}&\lim_{\Delta q_{i}\rightarrow 0}%
\frac{1}{4\pi \epsilon _{o}}\dsum\limits_{i}\frac{\Delta q_{i}}{r_{i}^{2}}%
\mathbf{\hat{r}}_{i} \\
&=&\frac{1}{4\pi \epsilon _{o}}\int \frac{dq_{E}}{r^{2}}\mathbf{\hat{r}}
\end{eqnarray*}%
The limits of the integration must include the entire distribution of charge
if we want the total field. This will be our basic equation for finding the
field for continuous distributions of charge.

Let's do some examples.

\subsection{Line of charge}

\FRAME{dtbpF}{3.3287in}{2.8694in}{0pt}{}{}{Figure}{\special{language
"Scientific Word";type "GRAPHIC";maintain-aspect-ratio TRUE;display
"USEDEF";valid_file "T";width 3.3287in;height 2.8694in;depth
0pt;original-width 3.2837in;original-height 2.8262in;cropleft "0";croptop
"1";cropright "1";cropbottom "0";tempfilename
'MIA2QF03.wmf';tempfile-properties "XPR";}}

%TCIMACRO{%
%\TeXButton{Question 223.23.2}{\marginpar {
%\hspace{-0.5in}
%\begin{minipage}[t]{1in}
%\small{Question 223.23.2}
%\end{minipage}
%}}}%
%BeginExpansion
\marginpar {
\hspace{-0.5in}
\begin{minipage}[t]{1in}
\small{Question 223.23.2}
\end{minipage}
}%
%EndExpansion
Let's try this for a line of charge. This may seem like a simple charge
configuration, but this problem is really quite challenging. Let's say that
the charge is evenly distributed along the line. Then we can use the linear
charge density 
\begin{equation*}
\lambda =Q/L
\end{equation*}%
to find $dq.$ The quantity $Q$ is the total amount of charge on the wire and 
$L$ is the length of the wire. Then%
\begin{equation*}
dq=\lambda dy
\end{equation*}

Of course, we may not always have a constant density, then we need to have
and element of charge that varies with position. For a line charge, we would
have 
\begin{equation*}
dq=\lambda \left( y\right) dy
\end{equation*}%
but for now, let's assume the linear charge density is constant. Our basic
formula tells us that we should add up all the $dq$ elements. But we have an
obstacle. We need a different $\mathbf{\hat{r}}_{i}$ for every $dq_{i}.$ How
do we deal with this?

Just like with last lecture, we only need the component of the part of the
field that does not cancel. Here we need to have drawn a good picture. From
our drawing we can tell that, in this case, only the $z$ component will
survive (the $y$-components cancel). So we only need to find%
\begin{equation*}
E_{z}=\overrightarrow{\mathbf{E}}\cdot \mathbf{\hat{k}}
\end{equation*}%
This is a good thing, because our basic equation has an $\mathbf{\hat{r}}$
in it 
\begin{equation*}
\overrightarrow{\mathbf{E}}=\frac{1}{4\pi \epsilon _{o}}\int \frac{dq}{r^{2}}%
\mathbf{\hat{r}}
\end{equation*}%
and we don't know how to do this integral including the $\mathbf{\hat{r}.}$
So we need to eliminate the direction part before we can proceed. Since we
just need the $z$-component, 
\begin{eqnarray*}
E_{z} &=&\overrightarrow{\mathbf{E}}\cdot \mathbf{\hat{k}} \\
&=&\frac{1}{4\pi \epsilon _{o}}\int_{-L/2}^{L/2}\frac{dq}{r^{2}}\mathbf{\hat{%
r}}\cdot \mathbf{\hat{k}}
\end{eqnarray*}%
and we recognize 
\begin{equation*}
\mathbf{\hat{r}}\cdot \mathbf{\hat{k}}=\cos \theta
\end{equation*}%
So we are left with just%
\begin{equation*}
E_{z}=\frac{1}{4\pi \epsilon _{o}}\int_{-L/2}^{L/2}\frac{dq}{r^{2}}\cos
\theta
\end{equation*}%
which is much more likely be be integrable with what we know from M113 or
M215.

Like in our last lecture, we will want to express%
\begin{equation*}
r=\sqrt{y^{2}+z^{2}}
\end{equation*}%
and it makes it easier if we write 
\begin{equation*}
\cos \theta =\frac{z}{\sqrt{y^{2}+z^{2}}}
\end{equation*}%
Then our integral can be written as 
\begin{eqnarray*}
E_{z} &=&\frac{1}{4\pi \epsilon _{o}}\int_{-L/2}^{L/2}\frac{\lambda dy}{%
y^{2}+z^{2}}\frac{z}{\sqrt{y^{2}+z^{2}}} \\
&=&\frac{\lambda z}{4\pi \epsilon _{o}}\int_{-L/2}^{L/2}\frac{dy}{\left(
y^{2}+z^{2}\right) ^{\frac{3}{2}}}
\end{eqnarray*}%
This now looks like a M215 or M113 problem. We can find this integral in an
integral table or you can use your calculator, or a symbolic math package,
or you can remember your M215 or M113 and prove that 
\begin{equation*}
\int_{-L/2}^{L/2}\frac{dx}{\left( x^{2}\pm a^{2}\right) ^{\frac{3}{2}}}=%
\frac{\pm x}{a^{2}\sqrt{x^{2}\pm a^{2}}}
\end{equation*}%
so%
\begin{eqnarray*}
E_{z} &=&\frac{\lambda z}{4\pi \epsilon _{o}}\int_{-L/2}^{L/2}\frac{dy}{%
\left( y^{2}+z^{2}\right) ^{\frac{3}{2}}} \\
&=&\frac{\lambda z}{4\pi \epsilon _{o}}\left[ \frac{y}{z^{2}\sqrt{y^{2}+z^{2}%
}}\right\vert _{-L/2}^{L/2} \\
&=&\frac{\lambda z}{4\pi \epsilon _{o}}\left[ \frac{L/2}{z^{2}\sqrt{\left(
L/2\right) ^{2}+z^{2}}}-\frac{-L/2}{z^{2}\sqrt{\left( -L/2\right) ^{2}+z^{2}}%
}\right] \\
&=&\frac{\lambda }{4\pi z\epsilon _{o}}\frac{L}{\sqrt{\left( L/2\right)
^{2}+z^{2}}} \\
&=&\frac{1}{4\pi \epsilon _{o}}\frac{Q}{z\sqrt{\left( \frac{L}{2}\right)
^{2}+z^{2}}}
\end{eqnarray*}%
This is the field due to a charged rod of length $L.$

Note that there are only a few integrals that we can solve in closed form to
find electric fields. It might be a good idea to build your own integral
table for our exams, including the integrals from the problems and examples
we work.

An infinitely long line of charge is one of our basic charge models. So far
our line of charge is not infinitely long. We can find the field due to an
infinite line of charge by letting $L\ $become large 
\begin{eqnarray*}
E_{z} &=&\underset{L\rightarrow \infty }{\lim }\frac{1}{4\pi \epsilon _{o}}%
\frac{Q}{z\sqrt{\left( \frac{L}{2}\right) ^{2}+z^{2}}} \\
&=&\frac{1}{4\pi \epsilon _{o}}\frac{Q}{z\left( \frac{L}{2}\right) } \\
&=&\frac{1}{4\pi \epsilon _{o}}\frac{2\lambda }{z} \\
&=&\frac{1}{4\pi \epsilon _{o}}\frac{2\lambda }{z}
\end{eqnarray*}%
or if we use $r$ now in place of $z$ to define the distance from the center
of the line of charge (so it is easier to compare to our point charge
formula), we have%
\begin{equation*}
\overrightarrow{\mathbf{E}}_{z}=\frac{1}{4\pi \epsilon _{o}}\frac{2\lambda }{%
r}\mathbf{\hat{k}}
\end{equation*}%
%TCIMACRO{%
%\TeXButton{Question 223.23.3}{\marginpar {
%\hspace{-0.5in}
%\begin{minipage}[t]{1in}
%\small{Question 223.23.3}
%\end{minipage}
%}}}%
%BeginExpansion
\marginpar {
\hspace{-0.5in}
\begin{minipage}[t]{1in}
\small{Question 223.23.3}
\end{minipage}
}%
%EndExpansion
We should get a mental picture of what this means. \FRAME{dhF}{1.1364in}{%
1.8758in}{0pt}{}{}{Figure}{\special{language "Scientific Word";type
"GRAPHIC";maintain-aspect-ratio TRUE;display "USEDEF";valid_file "T";width
1.1364in;height 1.8758in;depth 0pt;original-width 2.5996in;original-height
4.3163in;cropleft "0";croptop "1";cropright "1";cropbottom "0";tempfilename
'LTUWDD8M.wmf';tempfile-properties "XPR";}}The field around a long line of
charge only depends on the distance away from the line, and on the linear
charge density. As we would expect, the field gets weaker as we get farther
away. But it does not get weaker as fast as the point charge case. That
makes some sense, because our infinite line of charge is, well, really big.
You are never really too far away from something that is infinitely big. So
we should not expect such a charge configuration to look very like a point
charge no matter how far away we go. Of course an infinite line of charge is
not something we can really build. So this is a useful approximation near,
say, a charged wire. But farther from the wire the approximation would not
be so good and we would have to go back to our finite line solution.

\subsection{Ring of charge}

\FRAME{dtbpF}{5.1084in}{2.8089in}{0in}{}{}{Figure}{\special{language
"Scientific Word";type "GRAPHIC";maintain-aspect-ratio TRUE;display
"USEDEF";valid_file "T";width 5.1084in;height 2.8089in;depth
0in;original-width 5.0531in;original-height 2.7665in;cropleft "0";croptop
"1";cropright "1";cropbottom "0";tempfilename
'MIA2YP05.wmf';tempfile-properties "XPR";}}

%TCIMACRO{%
%\TeXButton{Question 223.23.4}{\marginpar {
%\hspace{-0.5in}
%\begin{minipage}[t]{1in}
%\small{Question 223.23.4}
%\end{minipage}
%}}}%
%BeginExpansion
\marginpar {
\hspace{-0.5in}
\begin{minipage}[t]{1in}
\small{Question 223.23.4}
\end{minipage}
}%
%EndExpansion
Using what we have learned from the line of charge, we can find the axial
field of a ring of charge. Again, our picture is critically important. We
will need to solve the problem of eliminating $\mathbf{\hat{r}.}$ From the
picture, we can see that we will only have a $z$-component again. So we can
eliminate $\mathbf{\hat{r}}$ the same way as in the last problem. We model
the ring as a line of charge of length $2\pi R$ that has been bent into a
circle. Again we have the basic equation%
\begin{equation*}
\overrightarrow{\mathbf{E}}=\frac{1}{4\pi \epsilon _{o}}\int \frac{dq}{r^{2}}%
\mathbf{\hat{r}}
\end{equation*}%
Since the ring of charge is like a line of charge bent into a hoop. So we
can plan to work this problem very like the the line charge. Start again with%
\begin{equation*}
dq=\lambda dy
\end{equation*}%
but now we know that for the hoop 
\begin{equation*}
dq=\lambda ds
\end{equation*}%
where $s$ is the arc length. Recall that 
\begin{eqnarray*}
s &=&R\phi \\
ds &=&Rd\phi
\end{eqnarray*}%
where $R$ is the radius of the ring and $\phi $ is an angle measured from
the $x$-axis. So our $dq$ expression becomes 
\begin{equation*}
dq=\lambda Rd\phi
\end{equation*}%
For the whole ring 
\begin{eqnarray*}
Q &=&\lambda R2\pi \\
&=&2\pi R\lambda
\end{eqnarray*}%
We also need to use geometry to find $r,$ the distance to our point were we
want to know the field.%
\begin{equation*}
r=\sqrt{y_{i}^{2}+z^{2}}
\end{equation*}%
but since this is a ring, our $y_{i}=R$ for all $i.$ So%
\begin{equation*}
r=\sqrt{R^{2}+z^{2}}
\end{equation*}%
and using the same reasoning as in our last problem, 
\begin{equation*}
\cos \theta =\frac{z}{\sqrt{R^{2}+z^{2}}}
\end{equation*}%
Then we can set up our integral.%
\begin{eqnarray*}
E_{z} &=&\overrightarrow{\mathbf{E}}\cdot \mathbf{\hat{k}} \\
&=&\frac{1}{4\pi \epsilon _{o}}\int_{-L/2}^{L/2}\frac{dq}{r^{2}}\mathbf{\hat{%
r}}\cdot \mathbf{\hat{k}}
\end{eqnarray*}%
Putting in all the parts we have found yields%
\begin{eqnarray*}
E_{z} &=&\frac{1}{4\pi \epsilon _{o}}\int \frac{dq}{R^{2}+z^{2}}\frac{z}{%
\sqrt{R^{2}+z^{2}}} \\
&=&\frac{1}{4\pi \epsilon _{o}}\int \frac{z\lambda Rd\phi }{\left(
R^{2}+z^{2}\right) ^{\frac{3}{2}}} \\
&=&\frac{z\lambda R}{4\pi \epsilon _{o}\left( R^{2}+z^{2}\right) ^{\frac{3}{2%
}}}\int_{0}^{2\pi }d\phi
\end{eqnarray*}%
This is an easy integral to do! and we see that the axial field is%
\begin{equation*}
E_{z}=\frac{z2\pi R\lambda }{4\pi \epsilon _{o}\left( R^{2}+z^{2}\right) ^{%
\frac{3}{2}}}
\end{equation*}%
or, using our form for $Q$ 
\begin{equation*}
\overrightarrow{\mathbf{E}}=\frac{1}{4\pi \epsilon _{o}}\frac{zQ}{\left(
R^{2}+z^{2}\right) ^{\frac{3}{2}}}\mathbf{\hat{k}}
\end{equation*}

Once again we should check to see if this is a reasonable result. If we take
the limit as $z$ goes to infinity, we get zero. That is comforting. But if
we just let $z$ be much larger than $R,$ but not too big%
\begin{eqnarray*}
\underset{z\gg R}{\lim }\overrightarrow{\mathbf{E}} &=&\underset{z\gg R}{%
\lim }\frac{1}{4\pi \epsilon _{o}}\frac{zQ}{\left( R^{2}+z^{2}\right) ^{%
\frac{3}{2}}}\mathbf{\hat{k}} \\
&=&\frac{1}{4\pi \epsilon _{o}}\frac{zQ}{\left( z^{2}\right) ^{\frac{3}{2}}}%
\mathbf{\hat{k}} \\
&=&\frac{1}{4\pi \epsilon _{o}}\frac{zQ}{z^{3}}\mathbf{\hat{k}} \\
&=&\frac{1}{4\pi \epsilon _{o}}\frac{Q}{z^{2}}\mathbf{\hat{k}}
\end{eqnarray*}%
we again have a point charge field with total charge $Q.$ Since a ring of
charge should look like a point charge if we get far enough away, this is
reasonable.

We have worked two problems for continuous charge distributions. The pattern
for solving both problems was the same. And we will follow the same pattern
for solving for the field from continuous charge distributions in all our
problems:

\begin{itemize}
\item Start with $\overrightarrow{\mathbf{E}}=\frac{1}{4\pi \epsilon _{o}}%
\int \frac{dq_{E}}{r^{2}}\mathbf{\hat{r}}$

\item Find an expression for $dq_{E}$

\item Use geometry to find an expressions for $r,$ the distance from $dq_{E}$
to the point, $P,$ where we want to know the field.

\item Eliminate $\mathbf{\hat{r}}$

\item Solve the integral
\end{itemize}

If you have a harder problem, one where you need the field from a continuous
charge distribution at a point that is not on an axis, or your problem has
little symmetry, you can go back to 
\begin{eqnarray*}
\overrightarrow{\mathbf{E}} &\approx &\sum_{i}\Delta \overrightarrow{\mathbf{%
E}}_{i} \\
&\mathbf{\approx }&\frac{1}{4\pi \epsilon _{o}}\dsum\limits_{i}\frac{\Delta
q_{i}}{r_{i}^{2}}\mathbf{\hat{r}}_{i}
\end{eqnarray*}%
and perform the sum numerically. We won't do this in our class, but you
might in practice or in a higher level electrodynamics course.

%TCIMACRO{%
%\TeXButton{Basic Equations}{\hspace{-1.3in}{\LARGE Basic Equations\vspace{0.25in}}}}%
%BeginExpansion
\hspace{-1.3in}{\LARGE Basic Equations\vspace{0.25in}}%
%EndExpansion

The basic equation from this chapter is the equation for finding the field
from a distribution of charge%
\begin{equation*}
\overrightarrow{\mathbf{E}}=\frac{1}{4\pi \epsilon _{o}}\int \frac{dq_{E}}{%
r^{2}}\mathbf{\hat{r}}
\end{equation*}%
The process for using this equation is

\begin{itemize}
\item Start with $\overrightarrow{\mathbf{E}}=\frac{1}{4\pi \epsilon _{o}}%
\int \frac{dq_{E}}{r^{2}}\mathbf{\hat{r}}$

\item Find an expression for $dq$

\item Use geometry to find an expressions for $r$

\item Eliminate $\mathbf{\hat{r}}$ in the usual way by turning a two or
three-dimensional problem into two or three one-dimensional problems (using
vector components, etc.)

\item Solve the integral(s) (Don't forget to report the direction)
\end{itemize}

\chapter{Motion of Charged Particles in Electric Fields}

%TCIMACRO{%
%\TeXButton{Fundamental Concepts}{\hspace{-1.3in}{\LARGE Fundamental Concepts\vspace{0.25in}}}}%
%BeginExpansion
\hspace{-1.3in}{\LARGE Fundamental Concepts\vspace{0.25in}}%
%EndExpansion

\begin{itemize}
\item The capacitor

\item Field of an ideal Capacitor

\item Motion of particles in a constant electric field
\end{itemize}

\section{Sheet of Charge}

%TCIMACRO{%
%\TeXButton{Question 223.23.5}{\marginpar {
%\hspace{-0.5in}
%\begin{minipage}[t]{1in}
%\small{Question 223.23.5}
%\end{minipage}
%}}}%
%BeginExpansion
\marginpar {
\hspace{-0.5in}
\begin{minipage}[t]{1in}
\small{Question 223.23.5}
\end{minipage}
}%
%EndExpansion
Let's try a two dimensional distribution of charge, a uniform flat sheet of
charge. We will assume that the sheet is infinitely large (so we don't have
to deal with what happens at the edges). Let's call the surface charge
density $\eta =Q/A$ where $Q$ is the total charge and $A$ is the total area.
Of course, we can't calculate this surface charge density directly from the
totals, because they are infinite. But we could take a square meter of area
and find the amount of charge in that small area. The ratio should be the
same for any area so long as $\eta $ is uniform. We will find the electric
field to the right of the sheet at point $P$.

\FRAME{dtbpF}{4.042in}{2.2866in}{0in}{}{}{Figure}{\special{language
"Scientific Word";type "GRAPHIC";maintain-aspect-ratio TRUE;display
"USEDEF";valid_file "T";width 4.042in;height 2.2866in;depth
0in;original-width 4.0899in;original-height 2.3008in;cropleft "0";croptop
"1";cropright "1";cropbottom "0";tempfilename
'MV75PX03.wmf';tempfile-properties "XPR";}}

Once again we start with%
\begin{equation*}
\overrightarrow{\mathbf{E}}=\frac{1}{4\pi \epsilon _{o}}\int \frac{dq}{r^{2}}%
\mathbf{\hat{r}}
\end{equation*}%
We need to find $dq,$ an expression for $r,$ and get rid of $\hat{r}$

Since the disk is uniformly charged, then, knowing the surface charge
density 
\begin{equation*}
\eta =\frac{Q}{A}
\end{equation*}%
we can find the total amount of charge for an area 
\begin{equation*}
Q=\eta A
\end{equation*}%
so 
\begin{equation*}
dq=\eta dA
\end{equation*}%
but what area, $dA$, should we use? Notice the green patch in the figure
that is marked $dA$. Think for a moment about arc length%
\begin{equation*}
s=R\phi
\end{equation*}%
This little area is about $ds=Rd\phi $ long, and about $dR$ wide. If we let $%
dA$ be small enough, this is exact. So%
\begin{equation*}
dA=Rd\phi dR
\end{equation*}
Then our $dq$ is just $\eta $ times this 
\begin{equation*}
dq=\eta Rd\phi dR
\end{equation*}

From geometry we identify%
\begin{equation*}
r=\sqrt{z^{2}+R^{2}}
\end{equation*}%
and, due to symmetry we expect only the $z$-component of the field to
survive. So to get rid of $\hat{r}$ we multiply (dot product) by $\hat{k}.$
There will be an angle, $\theta ,$ between $\hat{r}$ and $\hat{k}.$ So we
expect the result of the dot product to be that we multiply by the cosine of 
$\theta $%
\begin{equation*}
\cos \theta =\frac{z}{\sqrt{z^{2}+R^{2}}}
\end{equation*}

We want to put all this into our basic equation. This time the radius $R$
changes, so let's call it $R^{\prime }$ so we recognize that it is a
variable over which we must integrate, then%
\begin{equation*}
\overrightarrow{\mathbf{E}}=\frac{1}{4\pi \epsilon _{o}}\int \frac{dq}{r^{2}}%
\mathbf{\hat{r}}
\end{equation*}%
becomes 
\begin{eqnarray*}
E_{z} &=&\frac{1}{4\pi \epsilon _{o}}\int \frac{\eta R^{\prime }d\phi
dR^{\prime }}{\left( z^{2}+R^{\prime 2}\right) }\mathbf{\hat{r}}\cdot 
\mathbf{\hat{k}} \\
&=&\frac{1}{4\pi \epsilon _{o}}\int \frac{\eta R^{\prime }d\phi dR^{\prime }%
}{\left( z^{2}+R^{\prime 2}\right) }\cos \theta
\end{eqnarray*}%
and we will integrate from $R^{\prime }=0$ to $R^{\prime }=R.$ 
\begin{equation*}
E_{z}=\frac{1}{4\pi \epsilon _{o}}\int_{0}^{2\pi }\int_{0}^{R}\frac{z\eta
R^{\prime }d\phi dR^{\prime }}{\left( z^{2}+R^{\prime 2}\right) ^{\frac{3}{2}%
}}
\end{equation*}%
Performing the integration over $d\phi $ just gives us a factor of $2\pi $%
\begin{equation*}
E_{z}=\frac{z\eta \pi }{4\pi \epsilon _{o}}\int_{0}^{R}\frac{2R^{\prime
}dR^{\prime }}{\left( z^{2}+R^{\prime 2}\right) ^{\frac{3}{2}}}
\end{equation*}%
where, for convenience, we have left the $2$ inside the integral (it will be
useful later).

We need to solve the integral over $dR^{\prime }$. A $u$-substitution is one
way. Suppose we let 
\begin{equation*}
u=z^{2}+R^{\prime 2}
\end{equation*}%
so%
\begin{equation*}
du=2R^{\prime }dR^{\prime }
\end{equation*}

We will need to adjust the limits of integration, for $R^{\prime }=0$ we
have 
\begin{equation*}
u=z^{2}
\end{equation*}%
and for $R^{\prime }=R$%
\begin{equation*}
u=z^{2}+R^{2}
\end{equation*}%
then our integral becomes%
\begin{equation*}
E_{z}=\frac{z\pi \eta }{4\pi \epsilon _{o}}\int_{z^{2}}^{z^{2}+R^{2}}\frac{du%
}{\left( u\right) ^{\frac{3}{2}}}
\end{equation*}%
We get%
\begin{eqnarray*}
E_{z} &=&\frac{z\pi \eta }{4\pi \epsilon _{o}}\left[ \frac{-2}{\left(
u\right) ^{\frac{1}{2}}}\right\vert _{z^{2}}^{z^{2}+R^{2}} \\
&=&\frac{z\pi \eta }{4\pi \epsilon _{o}}\left( \frac{-2}{\left(
z^{2}+R^{2}\right) ^{\frac{1}{2}}}-\frac{-2}{\left( z^{2}\right) ^{\frac{1}{2%
}}}\right) \\
&=&\frac{-2z\pi \eta }{4\pi \epsilon _{o}}\left( \frac{1}{\left(
z^{2}+R^{2}\right) ^{\frac{1}{2}}}-\frac{1}{z}\right) \\
&=&\frac{-2\pi \eta }{4\pi \epsilon _{o}}\left( \frac{z}{\left(
z^{2}+R^{2}\right) ^{\frac{1}{2}}}-1\right) \\
&=&\frac{-2\pi \eta }{4\pi \epsilon _{o}}\left( \frac{1}{\frac{1}{z}\left(
z^{2}+R^{2}\right) ^{\frac{1}{2}}}-1\right)
\end{eqnarray*}%
The result is 
\begin{equation*}
E_{z}=\frac{2\pi \eta }{4\pi \epsilon _{o}}\left( 1-\frac{1}{\left( 1+\frac{%
R^{2}}{z^{2}}\right) ^{\frac{1}{2}}}\right)
\end{equation*}%
or%
\begin{equation*}
E_{z}=\frac{2\pi \eta }{4\pi \epsilon _{o}}\left( 1-\left( 1+\frac{R^{2}}{%
z^{2}}\right) ^{-\frac{1}{2}}\right)
\end{equation*}%
This looks messy, but this is the answer.

But wait, this is really a disk of charge with radius $R.$ We wanted an
infinite sheet of charge. So. suppose we let $R$ get very big. Then

\begin{eqnarray*}
E_{R\rightarrow \infty } &=&\underset{R\rightarrow \infty }{\lim }\frac{2\pi
\eta }{4\pi \epsilon _{o}}\left( 1-\left( 1+\frac{R^{2}}{z^{2}}\right) ^{-%
\frac{1}{2}}\right) \\
&=&\frac{2\pi \eta }{4\pi \epsilon _{o}} \\
&=&\frac{\eta }{2\epsilon _{o}}
\end{eqnarray*}%
%TCIMACRO{%
%\TeXButton{Question 223.23.6}{\marginpar {
%\hspace{-0.5in}
%\begin{minipage}[t]{1in}
%\small{Question 223.23.6}
%\end{minipage}
%}}}%
%BeginExpansion
\marginpar {
\hspace{-0.5in}
\begin{minipage}[t]{1in}
\small{Question 223.23.6}
\end{minipage}
}%
%EndExpansion
This is the field for our semi-infinite sheet of charge.

We should take some time to figure out if this makes sense.

This sheet cuts the entire universe into to two parts. It is so big, that it
is hard to say anything is very far away from it. So we can understand this
answer, The field from such a sheet of charge is constant every where in all
of space. No matter how far away we get, it will never look like a point
charge, in fact, it never really looks any farther away!

Note we did just one side of the sheet, there is a matching field on the
other side. So this sheet of charge fills all of space with a constant field.%
\FRAME{dhF}{3.8856in}{2.3851in}{0pt}{}{}{Figure}{\special{language
"Scientific Word";type "GRAPHIC";maintain-aspect-ratio TRUE;display
"USEDEF";valid_file "T";width 3.8856in;height 2.3851in;depth
0pt;original-width 3.8372in;original-height 2.3454in;cropleft "0";croptop
"1";cropright "1";cropbottom "0";tempfilename
'M4UEK801.wmf';tempfile-properties "XPR";}}

Of course this is not physically possible to build, but we will see that if
we look at a large sheet of charge, like the plate of a capacitor, that near
the center, the field approaches this limit, because the sides of the sheet
are far away.%
%TCIMACRO{%
%\TeXButton{Visualization falstad 3D}{\marginpar {
%\hspace{-0.5in}
%\begin{minipage}[t]{1in}
%\small{Visualization falstad 3D}
%\end{minipage}
%}}}%
%BeginExpansion
\marginpar {
\hspace{-0.5in}
\begin{minipage}[t]{1in}
\small{Visualization falstad 3D}
\end{minipage}
}%
%EndExpansion

Let's go back and consider the disk of charge. 
\begin{equation*}
E_{z}=\frac{2\pi \eta }{4\pi \epsilon _{o}}\left( 1-\left( 1+\frac{R^{2}}{%
z^{2}}\right) ^{-\frac{1}{2}}\right)
\end{equation*}%
Suppose we look at this distribution from vary far away for a finite disk.
We expect that it should look like a point charge with total charge $Q.$
Let's show that this is true. When $z$ gets very large $R/z$ is very small. 
\begin{equation*}
E_{z\gg R}=\frac{2\pi \eta }{4\pi \epsilon _{o}}\left( 1-\left( 1+\frac{R^{2}%
}{z^{2}}\right) ^{-\frac{1}{2}}\right)
\end{equation*}%
Let's look at just the part 
\begin{equation*}
\left( 1+\frac{R^{2}}{z^{2}}\right) ^{-\frac{1}{2}}
\end{equation*}%
This is of the form $\left( 1+x\right) ^{n}$ where $x$ is a small number. We
can use the binomial expansion 
\begin{equation*}
\left( 1+x\right) ^{n}\approx 1+nx\qquad x\ll 1
\end{equation*}%
to write this as%
\begin{equation*}
\left( 1+\frac{R^{2}}{z^{2}}\right) ^{-\frac{1}{2}}\approx 1-\frac{1}{2}%
\frac{R^{2}}{z^{2}}
\end{equation*}%
so in the limit that $z$ is large we have%
\begin{eqnarray*}
E_{z\gg R} &=&\frac{2\pi \eta }{4\pi \epsilon _{o}}\left( 1-1+\frac{1}{2}%
\frac{R^{2}}{z^{2}}\right) \\
&=&\frac{1}{4\pi \epsilon _{o}}\frac{\pi \frac{Q}{\pi R^{2}}R^{2}}{z^{2}} \\
&=&\frac{1}{4\pi \epsilon _{o}}\frac{Q}{z^{2}}
\end{eqnarray*}%
Which looks like a point charge as we expected. We have just a small, disk
of charge very far away. That is looks like a point charge with total charge 
$Q.$

\subsection{Spheres, shells, and other geometries.}

I won't do the problem for the field of a charged sphere or spherical shell.
We could, but we will save them for a new technique for finding fields from
configurations of charge that we will learn soon. This new technique will
attempt to make the integration much easier.

\section{Constant electric fields}

Let's try to use what we know about electric fields to predict the motion of
charged particles that are placed in electric fields. We will start with the
simplest case, a charged particle moving in a constant electric field.
Before we take on such a case, we should think about how we could produce a
constant electric field.

We know that a semi-infinite sheet of charge produces a constant electric
field. But we realize that a semi-infinite object is hard to build and hard
to manage. But if the size of the sheet of charge is very large compared to
the charge size, using our solution for a semi-infinite case might not be
too bad if we are away from the edges of the real sheet.

We want to study just such a device. In fact we will use two finite sheets
of charge.

\subsection{Capacitors}

From what we know about charge and conductors, we can charge a large metal
plate by touching it to something that is charged, like a rubber rod, or a
glass rod that has been rubbed with the right material.\FRAME{dtbpF}{2.2442in%
}{1.7106in}{0pt}{}{}{Figure}{\special{language "Scientific Word";type
"GRAPHIC";maintain-aspect-ratio TRUE;display "USEDEF";valid_file "T";width
2.2442in;height 1.7106in;depth 0pt;original-width 4.0043in;original-height
3.045in;cropleft "0";croptop "1";cropright "1";cropbottom "0";tempfilename
'LTUWDE8Q.wmf';tempfile-properties "XPR";}}If we have two large metal plates
and touch one with a rubber rod and one with a glass rod, we get two
oppositely charged sheets of charge.

What would the field look like for this oppositely charged set of plates?
Here is one of our thread-in-oil pictures of just such a situation. We are
looking at the plates edge-on. \FRAME{fhFU}{1.7755in}{1.5558in}{0pt}{\Qcb{%
{\protect\small http://stargazers.gsfc.nasa.gov/images/geospace%
\_images/electricity/charged\_plates.jpg}}}{\Qlb{capacitor-oil-thread}}{%
Figure}{\special{language "Scientific Word";type
"GRAPHIC";maintain-aspect-ratio TRUE;display "USEDEF";valid_file "T";width
1.7755in;height 1.5558in;depth 0pt;original-width 4.5835in;original-height
4.0136in;cropleft "0";croptop "1";cropright "1";cropbottom "0";tempfilename
'LTUWDE8R.bmp';tempfile-properties "XPR";}} Near the center, the field is
close to constant. Near the sides it is not so much so. We are probably
justified in saying the field in the middle is nearly constant. A look at
the field lines shows us why

\FRAME{dhF}{2.8669in}{1.9519in}{0pt}{}{}{Figure}{\special{language
"Scientific Word";type "GRAPHIC";maintain-aspect-ratio TRUE;display
"USEDEF";valid_file "T";width 2.8669in;height 1.9519in;depth
0pt;original-width 4.1295in;original-height 2.8037in;cropleft "0";croptop
"1";cropright "1";cropbottom "0";tempfilename
'LTUWDE8S.wmf';tempfile-properties "XPR";}}

Note that in between the plates, the electric field from the positive plate
is downward. But so is the electric field from the negative plate. The two
fields will add together. Outside the plates, the field from one plate is in
the opposite direction from that of the other plate. The two fields will
nearly cancel. If our device is made of semi-infinite sheets of charge, they
will precisely cancel, because the field of a semi-infinite sheet of charge
is uniform everywhere.

We call this configuration of two charged plates a \emph{capacitor} and, as
you might guess, this type of device proves to be more useful than just
making nearly constant fields. It is a major component in electronic
devices. Before we can build and iPad or a laptop, we will need to
understand several different types of basic devices. This set of charged
plates is our first.

Of course, for real capacitors, the fields outside cancel completely only
near the center of the plates. Near the edges, the direction of the fields
will change, and we get the sort of behavior that we see in figure \ref%
{capacitor-oil-thread} near the edges.

It is probably worth noting that outside the capacitor the field has a \emph{%
magnitude of zero} (or nearly zero). It is not really correct to say that
there is no field. In fact, there are two superimposed fields, or
alternately, a field from each of the charges on each plates, all
superimposed. The fields are there, but their magnitude is zero.

In the middle, then, we will have 
\begin{eqnarray*}
E &=&E_{+}+E_{-} \\
&\approx &\frac{\eta }{2\epsilon _{o}}+\frac{\eta }{2\epsilon _{o}} \\
&=&\frac{\eta }{\epsilon _{o}} \\
&=&\frac{Q}{A\epsilon _{o}}
\end{eqnarray*}

\section{Particle motion in a uniform field}

%TCIMACRO{%
%\TeXButton{Question 223.24.1}{\marginpar {
%\hspace{-0.5in}
%\begin{minipage}[t]{1in}
%\small{Question 223.24.1}
%\end{minipage}
%}}}%
%BeginExpansion
\marginpar {
\hspace{-0.5in}
\begin{minipage}[t]{1in}
\small{Question 223.24.1}
\end{minipage}
}%
%EndExpansion
Now that we have a way to form a uniform electric field, we can study
charged particles moving in this field. Motion of particles in uniform
fields is really something we are familiar with. It is very much like a ball
in a uniform gravitational field. But we have the complication of having two
different types of charge. The force on such a particle is given by 
\begin{equation*}
\overrightarrow{\mathbf{F}}=q_{m}\overrightarrow{\mathbf{E}}
\end{equation*}%
but we can combine this with Newton's second law%
\begin{equation*}
\overrightarrow{\mathbf{F}}=m\overrightarrow{\mathbf{a}}
\end{equation*}%
to find the particle's acceleration%
\begin{equation*}
\overrightarrow{\mathbf{a}}=\frac{q\overrightarrow{_{m}\mathbf{E}}}{m}
\end{equation*}%
Note, this is NOT true in general. It is only true for constant electric
fields.

%TCIMACRO{%
%\TeXButton{CRT Demo}{\marginpar {
%\hspace{-0.5in}
%\begin{minipage}[t]{1in}
%\small{CRT Demo}
%\end{minipage}
%}}}%
%BeginExpansion
\marginpar {
\hspace{-0.5in}
\begin{minipage}[t]{1in}
\small{CRT Demo}
\end{minipage}
}%
%EndExpansion

\subsection{Millikan}

Let's try a problem. Perhaps you have wondered, \textquotedblleft how do we
know that charge comes in packets of the size of the electron
charge?\textquotedblright\ This is a good story, and uses many of the laws
we have learned.

Robert Millikan devised a clever device in the early 1900's. A picture of
his device is given below.\FRAME{dtbpFU}{5.0301in}{3.061in}{0pt}{\Qcb{%
Millikan's oil-drop apparatus: Diagram taken from orginal Millikan's paper,
1913, Image taken in 1906 (Both Images in the Public Domain)}}{}{Figure}{%
\special{language "Scientific Word";type "GRAPHIC";maintain-aspect-ratio
TRUE;display "USEDEF";valid_file "T";width 5.0301in;height 3.061in;depth
0pt;original-width 5.0975in;original-height 3.0902in;cropleft "0";croptop
"1";cropright "1";cropbottom "0";tempfilename
'MNTMNN04.wmf';tempfile-properties "XPR";}}Schematically we can draw the
experiment like this.\FRAME{dhF}{3.6971in}{2.0046in}{0pt}{}{}{Figure}{%
\special{language "Scientific Word";type "GRAPHIC";maintain-aspect-ratio
TRUE;display "USEDEF";valid_file "T";width 3.6971in;height 2.0046in;depth
0pt;original-width 4.8101in;original-height 2.5953in;cropleft "0";croptop
"1";cropright "1";cropbottom "0";tempfilename
'LTUWDE8T.wmf';tempfile-properties "XPR";}}

Millikan made negatively charged oil drops with an atomizer (fine spray
squirt bottle). The drops are introduced between two charged plates into
what we know is essentially a constant electric field. A light shines off
the oil drops, so you can see them through a telescope (not shown). We can
determine the motion of the oil drops just like we did in PH121 or Dynamics.
If the upper plate has the positive charge, then the electric field $%
\overrightarrow{\mathbf{E}}$ is downward. A free body diagram for a drop is
shown in the figure to the left of the apparatus. We can write out Newton's
second law for the drop (our mover charge).%
\begin{equation*}
\Sigma F_{y}=m_{d}a_{y}=-F_{g}\pm F_{D}+F_{e}
\end{equation*}%
where $F_{D}$ is a drag force because we have air resistance.

If the upper plate has the positive charge, then the electric field $\mathbf{%
\vec{E}}$ is downward. So 
\begin{equation*}
\overrightarrow{\mathbf{F}}_{e}=-q_{m}\overrightarrow{\mathbf{E}}
\end{equation*}%
The field points down, the charge is negative, so the force is upward
(positive in our favorite coordinate system). We can write newtons's second
law as%
\begin{equation*}
m_{d}a_{y}=-F_{g}\pm F_{D}+q_{m}E
\end{equation*}%
If $F_{e}$ is large enough, we can make the oil drop float up! Then the drag
force is downward%
\begin{equation*}
m_{d}a_{y}=-mg-F_{D}+q_{m}E
\end{equation*}%
and if we are very careful, we can balance these forces so we have the drop
float upward at a small constant velocity.%
\begin{equation*}
0=-mg-F_{D}+q_{m}E
\end{equation*}%
The constant speed is really slow, hundredths of a centimeter per second. so
we can watch the drop move with no problem (except for patience). Once he
achieved a constant speed, by knowing the drop size and density Millikan
could calculate the mass, and therefore the charge. 
\begin{equation*}
mg+F_{D}=q_{m}E
\end{equation*}%
we see that%
\begin{equation*}
q_{m}=\frac{mg+F_{D}}{E}
\end{equation*}%
Which is where our problem ends. But Millikan went farther. He had actual
data, so he could compare charges on different droplets. He found that no
matter what the value for $q_{m},$ it was a multiple of a value, $%
q_{e}=1.602\times 10^{-19}\unit{C}.$ So%
\begin{equation*}
q_{m}=nq_{e}\qquad n=0,\pm 1,\pm 2,\ldots
\end{equation*}%
to within about $1\%.$\footnote{%
There is actually some controversy about this. Apparently Millikan and his
students threw out much of their data, keeping only data on drops that
behaved like they thought they should. They were lucky that this poor
analysis technique did not lead to invalid results! (William Broad and
Nicholas Wade, $\emph{Betrayers}$ \emph{of the truth, }Simon and Schuster,
1983)}. So the smallest charge the drops could have added to them was $%
1\times q_{e}$ and any other larger charge would be a larger multiple of $%
q_{e}.$ The conclusion is that charge comes in units of $q_{e}$. We
recognize $q_{e}$ as the electron charge. You can't add half of an electron
charge. This experiment showed that charge seems to only comes in whole
units!

\subsection{Free moving particles}

\FRAME{dhF}{3.0208in}{1.8135in}{0pt}{}{}{Figure}{\special{language
"Scientific Word";type "GRAPHIC";maintain-aspect-ratio TRUE;display
"USEDEF";valid_file "T";width 3.0208in;height 1.8135in;depth
0pt;original-width 4.3656in;original-height 2.6091in;cropleft "0";croptop
"1";cropright "1";cropbottom "0";tempfilename
'LTUWDE8U.wmf';tempfile-properties "XPR";}}

We may recall that for an object falling in a gravitational field, say, near
the Earth's surface, the acceleration, $g,$ is nearly constant. If we have a
charge moving in a constant electric field, we have a constant acceleration.
From Newtons' second law, 
\begin{equation*}
ma=q_{m}E
\end{equation*}%
we can see that this acceleration is 
\begin{equation*}
a=\frac{q_{m}E}{m}
\end{equation*}%
From our Dynamics or PH121 experience, we have a set of equations to handle
problems that involve constant acceleration%
\begin{eqnarray*}
x_{f} &=&x_{i}+v_{ix}t+\frac{1}{2}a_{x}t^{2} \\
v_{xf} &=&v_{xi}+a_{x}t \\
v_{xf}^{2} &=&v_{xi}^{2}+2a_{x}\Delta x
\end{eqnarray*}%
and%
\begin{eqnarray*}
y_{f} &=&y_{i}+v_{iy}t+\frac{1}{2}a_{y}t^{2} \\
v_{yf} &=&v_{yi}+a_{y}t \\
v_{yf}^{2} &=&v_{yi}^{2}+2a_{y}\Delta y
\end{eqnarray*}

These are know as the \emph{kinematic equations}. You derived them if you
took Dynamics (or derived them and then memorized them if you took PH121).
Let's try a brief problem. Suppose we have a positive charge in a uniform
electric field as shown. \FRAME{dhF}{2.9594in}{1.5601in}{0pt}{}{}{Figure}{%
\special{language "Scientific Word";type "GRAPHIC";maintain-aspect-ratio
TRUE;display "USEDEF";valid_file "T";width 2.9594in;height 1.5601in;depth
0pt;original-width 9.3555in;original-height 4.913in;cropleft "0";croptop
"1";cropright "1";cropbottom "0";tempfilename
'LTUWDE8V.wmf';tempfile-properties "XPR";}}Let $y=0$ at the positive plate.
How fast will the particle be going as it strikes the negative plate?

We use the acceleration%
\begin{eqnarray*}
a_{y} &=&\frac{q_{m}E}{m} \\
a_{x} &=&0
\end{eqnarray*}

For this problem we don't have any $x$-motion, So we can limit ourselves to.%
\begin{eqnarray*}
y_{f} &=&y_{i}+v_{iy}t+\frac{1}{2}\left( \frac{q_{m}E}{m}\right) t^{2} \\
v_{yf} &=&v_{yi}+\left( \frac{q_{m}E}{m}\right) t \\
v_{yf}^{2} &=&v_{yi}^{2}+2\left( \frac{q_{m}E}{m}\right) \Delta y
\end{eqnarray*}

We don't have the time of flight of the particle, but we can identify 
\begin{equation*}
\Delta y=d
\end{equation*}%
The particle started from rest, so 
\begin{equation*}
v_{yo}=0
\end{equation*}%
Therefore it makes sense to use the last of the three equations, because we
know everything that shows up in this equation but the final speed, and that
is what we want to find.%
\begin{eqnarray*}
v_{yf}^{2} &=&v_{yi}^{2}+2\left( \frac{q_{m}E}{m}\right) \Delta y \\
v_{yf}^{2} &=&0+2\left( \frac{q_{m}E}{m}\right) d \\
v_{yf} &=&\sqrt{\frac{2q_{m}Ed}{m}}
\end{eqnarray*}

There is a complication, however. With gravity, we only have one kind of
mass. But with charge we have two kinds of charge. Suppose we have a
negative particle.

Of course the negative particle would not move if it was started from the
positive side. It would be attracted to the positive plate. But suppose we
start the negative particle from the negative plate. It would travel
\textquotedblleft up\textquotedblright\ to the positive plate. We defined
the downward direction as the positive $y$-direction without really thinking
about it. Now we realize that the upward direction must be opposite, so
upward is the negative $y$-direction. Our negative particle will experience
a displacement $\Delta y=-d.$

Then 
\begin{eqnarray*}
v_{yf}^{2} &=&v_{yi}^{2}+2\left( \frac{-q_{m}E}{m}\right) \Delta y \\
v_{yf}^{2} &=&0+2\left( \frac{-q_{m}E}{m}\right) \left( -d\right) \\
v_{yf} &=&\sqrt{\frac{2q_{m}Ed}{m}}
\end{eqnarray*}%
we get the same speed, but this illustrates that we will have to be careful
to watch our signs.

In this last problem we have had only an electric force, no gravitational
force. This is important to notice. If there were also a gravitational
force, we would need to use Newton's second law to add up the forces like we
did with the Millikan problem.

Let's take another example. This time let's send in a negatively charged
particle horizontally through the capacitor. The particle will move up due
to the electric field force. How far up will it go as it travels across the
center of the capacitor? \FRAME{dhF}{2.5547in}{1.708in}{0pt}{}{}{Figure}{%
\special{language "Scientific Word";type "GRAPHIC";maintain-aspect-ratio
TRUE;display "USEDEF";valid_file "T";width 2.5547in;height 1.708in;depth
0pt;original-width 4.1848in;original-height 2.789in;cropleft "0";croptop
"1";cropright "1";cropbottom "0";tempfilename
'LTUWDF8W.wmf';tempfile-properties "XPR";}}Let's define the starting
position as 
\begin{eqnarray*}
x_{i} &=&0 \\
y_{i} &=&0
\end{eqnarray*}%
We can identify that 
\begin{eqnarray*}
v_{ix} &=&v_{0} \\
v_{iy} &=&0
\end{eqnarray*}

And that 
\begin{eqnarray*}
a_{y} &=&\frac{q_{m}E}{m} \\
a_{x} &=&0
\end{eqnarray*}%
We can fill in these values in our kinematic equations

\begin{eqnarray*}
x_{f} &=&0+v_{o}t+\frac{1}{2}\left( 0\right) t^{2} \\
v_{xf} &=&v_{o}+\left( 0\right) t \\
v_{xf}^{2} &=&v_{o}^{2}+2\left( 0\right) \Delta x
\end{eqnarray*}%
and%
\begin{eqnarray*}
y_{f} &=&0+\left( 0\right) t+\frac{1}{2}\left( \frac{q_{m}E}{m}\right) t^{2}
\\
v_{yf} &=&\left( 0\right) +\left( \frac{q_{m}E}{m}\right) t \\
v_{yf}^{2} &=&\left( 0\right) +2\left( \frac{q_{m}E}{m}\right) \left(
y_{f}-0\right)
\end{eqnarray*}%
From the first set we see that $v_{xf}=v_{o},$ that is, the $x$-direction
velocity does not change. That makes sense because we have no force
component in the $x$-direction.

After $t$ seconds we see that the charged particle has traveled a distance 
\begin{equation*}
x_{f}=v_{o}t
\end{equation*}%
If we measure $x_{f}=L$ then we can see how long it took for the particle to
travel through the capacitor%
\begin{equation*}
t=\frac{L}{v_{o}}
\end{equation*}%
Now let's look at the deflection. We can use the first equation of the $y$%
-set 
\begin{eqnarray*}
y_{f} &=&\frac{1}{2}\left( \frac{q_{m}E}{m}\right) t^{2} \\
&=&\frac{1}{2}\left( \frac{q_{m}E}{m}\right) \left( \frac{L}{v_{o}}\right)
^{2}
\end{eqnarray*}

Let's see if this makes sense. If the electric field gets larger, the
particle will deflect more.\FRAME{dhF}{2.6602in}{1.7201in}{0pt}{}{}{Figure}{%
\special{language "Scientific Word";type "GRAPHIC";maintain-aspect-ratio
TRUE;display "USEDEF";valid_file "T";width 2.6602in;height 1.7201in;depth
0pt;original-width 5.6161in;original-height 3.6227in;cropleft "0";croptop
"1";cropright "1";cropbottom "0";tempfilename
'LTUWDF8X.wmf';tempfile-properties "XPR";}}

This is right. The field causes the force, so more field gives more effect
from the force. If we increase the charge, the deflection grows since the
force depends on the charge of the moving particle. This also seems
reasonable. If the mass increases, it is harder to move the particle, so it
makes sense that a larger mass makes a smaller deflection. If the particle
is in the field longer, the deflection will increase, so the dependence on $%
L $ makes sense. Finally, if the initial speed is larger the particle spends
less time in the field, so the deflection will be less.

\section{Non uniform fields}

Of course all of this depends on the field being uniform. For a non uniform
field the force is still 
\begin{equation*}
\overrightarrow{\mathbf{F}}=q_{m}\overrightarrow{\mathbf{E}}\left(
x,y,z\right)
\end{equation*}%
but now the field is a function of position. This makes for a more difficult
problem. For now we will stick to constant fields. If we had to take on a
non-uniform field, we would likely use a numerical technique.

%TCIMACRO{%
%\TeXButton{Basic Equations}{\hspace{-1.3in}{\LARGE Basic Equations\vspace{0.25in}}}}%
%BeginExpansion
\hspace{-1.3in}{\LARGE Basic Equations\vspace{0.25in}}%
%EndExpansion

%TCIMACRO{%
%\HTMLButton{Field from continuous charge distributions}{&#092;subsection*&#123;Field from continuous charge distributions&#125;}}%

The magnitude of the electric field due to a disk of charge along the disk's
axis\FRAME{dtbpF}{1.9602in}{1.1105in}{0pt}{}{}{Figure}{\special{language
"Scientific Word";type "GRAPHIC";maintain-aspect-ratio TRUE;display
"USEDEF";valid_file "T";width 1.9602in;height 1.1105in;depth
0pt;original-width 3.6754in;original-height 2.0686in;cropleft "0";croptop
"1";cropright "1";cropbottom "0";tempfilename
'MV73G200.wmf';tempfile-properties "XPR";}}%
\begin{equation*}
E_{z}=\frac{2\pi \eta }{4\pi \epsilon _{o}}\left( 1-\left( 1+\frac{R^{2}}{%
z^{2}}\right) ^{-\frac{1}{2}}\right)
\end{equation*}

The magnitude of the electric field due to a semi-infinite sheet of charge%
\FRAME{dhF}{1.9141in}{1.1752in}{0pt}{}{}{Figure}{\special{language
"Scientific Word";type "GRAPHIC";maintain-aspect-ratio TRUE;display
"USEDEF";valid_file "T";width 1.9141in;height 1.1752in;depth
0pt;original-width 3.8372in;original-height 2.3454in;cropleft "0";croptop
"1";cropright "1";cropbottom "0";tempfilename
'MV73G202.wmf';tempfile-properties "XPR";}} 
\begin{equation*}
E=\frac{\eta }{2\epsilon _{o}}
\end{equation*}

The magnitude of the electric field inside an ideal capacitor\FRAME{dhF}{%
1.625in}{1.1061in}{0pt}{}{}{Figure}{\special{language "Scientific Word";type
"GRAPHIC";maintain-aspect-ratio TRUE;display "USEDEF";valid_file "T";width
1.625in;height 1.1061in;depth 0pt;original-width 4.1295in;original-height
2.8037in;cropleft "0";croptop "1";cropright "1";cropbottom "0";tempfilename
'MV73G201.wmf';tempfile-properties "XPR";}}%
\begin{equation*}
E=\frac{Q}{A\epsilon _{o}}
\end{equation*}

%TCIMACRO{%
%\HTMLButton{Motion of a charged particle in a constant electric field}{&#092;subsection*&#123;Motion of a charged particle in a constant electric field&#125;}}%

Motion of a charged particle in a constant electric field%
\begin{equation*}
\overrightarrow{\mathbf{a}}=\frac{q_{m}\overrightarrow{\mathbf{E}}}{m}
\end{equation*}%
\begin{equation*}
\begin{tabular}{ccc}
$x_{f}=x_{i}+v_{ix}t+\frac{1}{2}a_{x}t^{2}$ &  & $y_{f}=y_{i}+v_{iy}t+\frac{1%
}{2}a_{y}t^{2}$ \\ 
$v_{xf}=v_{xi}+a_{x}t$ &  & $v_{yf}=v_{yi}+a_{y}t$ \\ 
$v_{xf}^{2}=v_{xi}^{2}+2a_{x}\Delta x$ &  & $v_{yf}^{2}=v_{yi}^{2}+2a_{y}%
\Delta y$%
\end{tabular}%
\end{equation*}

\chapter{Dipole motion, Symmetry}

%TCIMACRO{%
%\TeXButton{Fundamental Concepts}{\hspace{-1.3in}{\LARGE Fundamental Concepts\vspace{0.25in}}}}%
%BeginExpansion
\hspace{-1.3in}{\LARGE Fundamental Concepts\vspace{0.25in}}%
%EndExpansion

\begin{itemize}
\item Force and torque on a dipole in a uniform field

\item Force on a dipole in a non-uniform field

\item Drawing the shape of a field using symmetry
\end{itemize}

This lecture combines two topics that might be better separated. The first
relates to forces on charges in uniform fields. This is what we discussed
last lecture. The next is the beginning of the ideas that will allow us to
use symmetry and geometry to avoid integration over charges. But because our
lecture times are only an hour, and we can only do so much at once, they are
combined here together. But they form a nice transition between the two
topics this way. We will first study the motion of dipoles in uniform, and
not so uniform fields. We will find symmetry and geometry plays a part in
our solutions. Then we will study the fields of standard symmetric objects.

\section{Dipole motion in an electromagnetic field}

We remember dipoles, a pair of charges of equal magnitude but opposite in
charge, bound together at set separation distance. Let's take our
environment to be a constant electric field, and our mover to be a dipole.%
\FRAME{dhF}{0.832in}{0.2188in}{0pt}{}{}{Figure}{\special{language
"Scientific Word";type "GRAPHIC";maintain-aspect-ratio TRUE;display
"USEDEF";valid_file "T";width 0.832in;height 0.2188in;depth
0pt;original-width 3.0303in;original-height 0.7766in;cropleft "0";croptop
"1";cropright "1";cropbottom "0";tempfilename
'NBC9FB0E.wmf';tempfile-properties "XPR";}}

%TCIMACRO{%
%\TeXButton{Question 223.25.1}{\marginpar {
%\hspace{-0.5in}
%\begin{minipage}[t]{1in}
%\small{Question 223.25.1}
%\end{minipage}
%}}}%
%BeginExpansion
\marginpar {
\hspace{-0.5in}
\begin{minipage}[t]{1in}
\small{Question 223.25.1}
\end{minipage}
}%
%EndExpansion
Here is a diagram of the situation. \FRAME{dhF}{1.6976in}{1.1796in}{0pt}{}{}{%
Figure}{\special{language "Scientific Word";type
"GRAPHIC";maintain-aspect-ratio TRUE;display "USEDEF";valid_file "T";width
1.6976in;height 1.1796in;depth 0pt;original-width 3.7256in;original-height
2.5815in;cropleft "0";croptop "1";cropright "1";cropbottom "0";tempfilename
'NBC9FB0D.wmf';tempfile-properties "XPR";}}Notice that as usual, just the
environmental field is drawn. There is a field from the dipole, too, \FRAME{%
dhF}{1.2548in}{0.806in}{0pt}{}{}{Figure}{\special{language "Scientific
Word";type "GRAPHIC";maintain-aspect-ratio TRUE;display "USEDEF";valid_file
"T";width 1.2548in;height 0.806in;depth 0pt;original-width
2.975in;original-height 1.9009in;cropleft "0";croptop "1";cropright
"1";cropbottom "0";tempfilename 'NBC9FB0F.wmf';tempfile-properties "XPR";}}%
but this is the mover's self-field and it cannot create a force on the
dipole, so we will not draw it. Of course, if we introduce yet another
charge, $q_{new}$, the environmental field this new charge would feel would
be a combination of both the dipole field and the uniform field! We would
have to draw the superposition of the two fields. \FRAME{dtbpF}{2.1119in}{%
1.5646in}{0pt}{}{}{Figure}{\special{language "Scientific Word";type
"GRAPHIC";maintain-aspect-ratio TRUE;display "USEDEF";valid_file "T";width
2.1119in;height 1.5646in;depth 0pt;original-width 2.9315in;original-height
2.1642in;cropleft "0";croptop "1";cropright "1";cropbottom "0";tempfilename
'NBC9FB0G.wmf';tempfile-properties "XPR";}}But that is a different problem!

Here is our case again. We only draw the environmental field that will cause
the motion of the mover object we are studying. \FRAME{dhF}{1.6976in}{%
1.1796in}{0pt}{}{}{Figure}{\special{language "Scientific Word";type
"GRAPHIC";maintain-aspect-ratio TRUE;display "USEDEF";valid_file "T";width
1.6976in;height 1.1796in;depth 0pt;original-width 3.7256in;original-height
2.5815in;cropleft "0";croptop "1";cropright "1";cropbottom "0";tempfilename
'NBC9FB0H.wmf';tempfile-properties "XPR";}}To understand these figures,we
have to remember that the red field arrows are an \emph{external field} that
is, the dipole is not making this field, so something else must be. We did
not draw that something else. Since it is a uniform field, it is probably a
capacitor. Here is what it might look like\FRAME{dhF}{1.753in}{1.9588in}{0pt%
}{}{}{Figure}{\special{language "Scientific Word";type
"GRAPHIC";maintain-aspect-ratio TRUE;display "USEDEF";valid_file "T";width
1.753in;height 1.9588in;depth 0pt;original-width 2.5166in;original-height
2.8176in;cropleft "0";croptop "1";cropright "1";cropbottom "0";tempfilename
'NBC9FB0I.wmf';tempfile-properties "XPR";}}The positive side must be to the
left, because the red external field arrows come from the left. The negative
side must be to the right, because the field arrows are pointed that
direction. We can get away with not drawing the source of the external field
because the force on the dipole charges is just 
\begin{equation*}
\overrightarrow{\mathbf{F}}=q_{m}\overrightarrow{\mathbf{E}}
\end{equation*}%
If we know $\overrightarrow{\mathbf{E}}$, then we don't need any information
about it's source to find the force. Since the field is the environment that
the mover charges feel, the field is enough. Let's find the net force on the
dipole due to the environmental field.

%TCIMACRO{%
%\TeXButton{Question 223.25.2}{\marginpar {
%\hspace{-0.5in}
%\begin{minipage}[t]{1in}
%\small{Question 223.25.2}
%\end{minipage}
%}}}%
%BeginExpansion
\marginpar {
\hspace{-0.5in}
\begin{minipage}[t]{1in}
\small{Question 223.25.2}
\end{minipage}
}%
%EndExpansion
We use Newton's second law to find that 
\begin{equation*}
F_{net_{x}}=-F_{-}+F_{+}=ma_{x}
\end{equation*}%
and our definition of the electric field to find%
\begin{eqnarray*}
F_{-} &=&q_{-}E \\
F_{+} &=&q_{+}E
\end{eqnarray*}%
so, since $\left\vert q_{-}\right\vert =\left\vert q_{+}\right\vert =q$%
\begin{equation*}
-qE+qE=ma_{x}
\end{equation*}%
which tells us that there is no acceleration, no net force. The center of
mass of a dipole does not accelerate in a uniform field. But we remember
from PH121 that we can make things rotate. \FRAME{dhF}{1.7383in}{1.2375in}{%
0pt}{}{}{Figure}{\special{language "Scientific Word";type
"GRAPHIC";maintain-aspect-ratio TRUE;display "USEDEF";valid_file "T";width
1.7383in;height 1.2375in;depth 0pt;original-width 3.5587in;original-height
2.5261in;cropleft "0";croptop "1";cropright "1";cropbottom "0";tempfilename
'NBC9FC0Z.wmf';tempfile-properties "XPR";}}If the dipole is not aligned with
it's axis in the field direction, then the forces will cause a torque.

%TCIMACRO{%
%\TeXButton{Question 223.25.3}{\marginpar {
%\hspace{-0.5in}
%\begin{minipage}[t]{1in}
%\small{Question 223.25.3}
%\end{minipage}
%}}}%
%BeginExpansion
\marginpar {
\hspace{-0.5in}
\begin{minipage}[t]{1in}
\small{Question 223.25.3}
\end{minipage}
}%
%EndExpansion
We remember that torque is given by 
\begin{equation*}
\overrightarrow{\mathbf{\tau }}=\overrightarrow{\mathbf{r}}\times 
\overrightarrow{\mathbf{F}}
\end{equation*}%
substituting in our force and defining the distance between the charges to
be $a$ we can write this out \FRAME{dhF}{1.7547in}{1.2298in}{0pt}{}{}{Figure%
}{\special{language "Scientific Word";type "GRAPHIC";maintain-aspect-ratio
TRUE;display "USEDEF";valid_file "T";width 1.7547in;height 1.2298in;depth
0pt;original-width 3.8934in;original-height 2.7207in;cropleft "0";croptop
"1";cropright "1";cropbottom "0";tempfilename
'NBC9FC10.wmf';tempfile-properties "XPR";}}The magnitude of the torque is
given by 
\begin{equation*}
\left\vert \tau \right\vert =rF\sin \theta
\end{equation*}%
where $\theta $ is the angle between $\mathbf{r}$ and $\mathbf{F}.$ It is
easier to find that angle if we redraw each displacement vector from the
pivot and each force with their tails together\FRAME{dhF}{1.1398in}{1.0144in%
}{0pt}{}{}{Figure}{\special{language "Scientific Word";type
"GRAPHIC";maintain-aspect-ratio TRUE;display "USEDEF";valid_file "T";width
1.1398in;height 1.0144in;depth 0pt;original-width 4.4763in;original-height
3.9825in;cropleft "0";croptop "1";cropright "1";cropbottom "0";tempfilename
'NBC9FC11.wmf';tempfile-properties "XPR";}}Then for one charge, say, $q_{\_}$
\begin{equation*}
\mathbf{\tau }=\frac{a}{2}qE\sin \theta
\end{equation*}%
We use the right-hand-rule that you learned in Dynamics or PH121 to find the
direction. We can see that the direction will be out of the page. But we
have two charges, so we have a torque from each charge. A quick check with
the right-hand-rule for torques will convince us that the direction for the
torque due to $q_{+}$ is also out of the page, and the magnitude is the
same, so our total torque is 
\begin{eqnarray*}
\tau _{net} &=&\tau _{+}+\tau _{-} \\
&=&aqE\sin \theta
\end{eqnarray*}%
which we can write as%
%TCIMACRO{%
%\TeXButton{Question 223.25.4}{\marginpar {
%\hspace{-0.5in}
%\begin{minipage}[t]{1in}
%\small{Question 223.25.4}
%\end{minipage}
%}} }%
%BeginExpansion
\marginpar {
\hspace{-0.5in}
\begin{minipage}[t]{1in}
\small{Question 223.25.4}
\end{minipage}
}
%EndExpansion
\begin{equation*}
\tau _{net}=pE\sin \theta
\end{equation*}%
or the \emph{dipole moment}, $p,$ multiplied by $E\sin \theta .$ Recalling
the form of a cross product%
\begin{equation*}
\overrightarrow{\mathbf{A}}\times \overrightarrow{\mathbf{B}}=AB\sin \theta 
\mathbf{\hat{n}}
\end{equation*}%
where $\mathbf{\hat{n}}$ is perpendicular to both $\overrightarrow{\mathbf{A}%
}$ and $\overrightarrow{\mathbf{B}},$ we have a hint that we could write our
torque as a cross product. We would have to make $p$ a vector, though. So
let's define $\overrightarrow{\mathbf{p}}$ as a vector with magnitude $aq$
and make its direction along the line connecting the charge centers, with
the direction from negative to positive. \FRAME{dtbpF}{1.0741in}{1.2548in}{%
0pt}{}{}{Figure}{\special{language "Scientific Word";type
"GRAPHIC";maintain-aspect-ratio TRUE;display "USEDEF";valid_file "T";width
1.0741in;height 1.2548in;depth 0pt;original-width 1.6483in;original-height
1.9311in;cropleft "0";croptop "1";cropright "1";cropbottom "0";tempfilename
'NBC9FC12.wmf';tempfile-properties "XPR";}}Then we can write the torque as 
\begin{equation}
\overrightarrow{\mathbf{\tau }}=\overrightarrow{\mathbf{p}}\times 
\overrightarrow{\mathbf{E}}
\end{equation}%
which is our form for the torque on a dipole.

Let's try a problem. Let's find the maximum angular acceleration for a
dipole.

Recall that Newton's second law for rotational motion is 
\begin{equation*}
\Sigma \tau =I\alpha
\end{equation*}%
where $I$ is the moment of inertia and $\alpha $ is the angular
acceleration. Then we can find how the dipole will accelerate%
\begin{equation*}
\alpha =\frac{\tau _{net}}{I}
\end{equation*}%
For a dipole, $I$ is simple%
\begin{eqnarray*}
I &=&m_{-}r_{-}^{2}+m_{+}r_{+}^{2} \\
&=&m\left( \frac{a}{2}\right) ^{2}+m\left( \frac{a}{2}\right) ^{2} \\
&=&\frac{1}{2}ma^{2}
\end{eqnarray*}%
so our acceleration is 
\begin{eqnarray*}
\alpha &=&\frac{pE\sin \theta }{\frac{1}{2}ma^{2}} \\
&=&\frac{2pE\sin \theta }{ma^{2}}
\end{eqnarray*}

Suppose we look at this for a water molecule in a microwave oven. What is
the maximum angular acceleration experienced by the water molecule if the
oven has a field strength of $E=200\unit{V}/\unit{m}$?

The dipole moment for a water molecule is something like 
\begin{equation*}
p_{w}=6.2\times 10^{-30}\unit{C}\unit{m}
\end{equation*}%
and the separation between the charge centers is something like 
\begin{equation*}
a=3.9\times 10^{-12}\unit{m}
\end{equation*}%
and the molecular mass of water is 
\begin{equation*}
M=18\frac{\unit{g}}{\unit{mol}}
\end{equation*}%
which is 
\begin{equation*}
M=mN_{A}
\end{equation*}%
so the mass of a water molecule is%
\begin{eqnarray*}
m &=&\frac{M}{N_{A}}=\frac{18\frac{\unit{g}}{\unit{mol}}}{6.022\times 10^{23}%
\frac{1}{\unit{mol}}} \\
&=&2.\,\allowbreak 989\times 10^{-26}\allowbreak \unit{kg}
\end{eqnarray*}%
then when $\sin \theta =1$ we will have a maximum%
\begin{eqnarray*}
\alpha &=&\frac{2\left( 6.2\times 10^{-30}\unit{C}\unit{m}\right) \left( 200%
\unit{V}/\unit{m}\right) }{\left( 2.\,\allowbreak 989\times
10^{-26}\allowbreak \unit{kg}\right) \left( 3.9\times 10^{-12}\unit{m}%
\right) ^{2}} \\
&=&5.\,\allowbreak 455\times 10^{21}\frac{\unit{rad}}{\unit{s}^{2}}
\end{eqnarray*}%
Our numbers were kind of rough estimates, but still the result is amazing.
Imagine if this happened inside of you! which is why we really should be
careful with microwave ovens and microwave equipment.

\subsection{Induced dipoles}

Suppose that we place a large insulator in a uniform electric field.

\FRAME{dhF}{3.1055in}{2.2001in}{0pt}{}{}{Figure}{\special{language
"Scientific Word";type "GRAPHIC";maintain-aspect-ratio TRUE;display
"USEDEF";valid_file "T";width 3.1055in;height 2.2001in;depth
0pt;original-width 4.5186in;original-height 3.192in;cropleft "0";croptop
"1";cropright "1";cropbottom "0";tempfilename
'LTUWDG97.wmf';tempfile-properties "XPR";}}The atoms tend to polarize and
become dipoles. We say we have \emph{induced} dipoles within the material.
Notice that in the middle of the conductor there is still no net charge. But
because we have made the atoms into dipoles, one side of the insulator
becomes negatively charged and the other side becomes positively charged.
This does not create a net force, but we will find that separating the
charges like this can be useful in building capacitors.

\subsection{Non-uniform fields and dipoles}

Suppose we place our dipole in a non-uniform field? Of course the result
will depend on the field, so let's take an example. Let's place a dipole in
the field due to a point charge.\FRAME{dhF}{2.0574in}{2.0539in}{0pt}{}{}{%
Figure}{\special{language "Scientific Word";type
"GRAPHIC";maintain-aspect-ratio TRUE;display "USEDEF";valid_file "T";width
2.0574in;height 2.0539in;depth 0pt;original-width 5.3246in;original-height
5.3151in;cropleft "0";croptop "1";cropright "1";cropbottom "0";tempfilename
'NBC9MT14.wmf';tempfile-properties "XPR";}}We can see that the field is much
weaker at the location of the positive charge than it is at the negative
charge location. If we zoom in on the location near our dipole we can see
that now we will have an acceleration! \FRAME{dhF}{2.8781in}{1.0205in}{0pt}{%
}{}{Figure}{\special{language "Scientific Word";type
"GRAPHIC";maintain-aspect-ratio TRUE;display "USEDEF";valid_file "T";width
2.8781in;height 1.0205in;depth 0pt;original-width 7.1321in;original-height
2.5123in;cropleft "0";croptop "1";cropright "1";cropbottom "0";tempfilename
'NBC9MT15.wmf';tempfile-properties "XPR";}}%
\begin{equation*}
\Sigma F_{x}=-F_{-}+F_{+}=ma_{x}
\end{equation*}%
so%
\begin{equation*}
-qE_{\text{large}}+qE_{\text{small}}=ma_{x}
\end{equation*}

Let's go back to our charged balloon from many lectures ago. We found that
the charge \textquotedblleft leaked off\textquotedblright\ our balloon. We
can see why now. The water molecules in the air are attracted to the
charges, and stick to them. When the water molecules float off, they will
take our charge with them. For this problem, the dipole is the environment
and our balloon electron is the moving charge. We can calculate the net
force easily with our field from a dipole that we found earlier, 
\begin{equation*}
\overrightarrow{\mathbf{E}}_{y}=\frac{2}{4\pi \epsilon _{o}}\frac{%
\overrightarrow{\mathbf{p}}}{L^{3}}
\end{equation*}%
then the force on the electron on the balloon is 
\begin{eqnarray*}
F &=&q_{e}E \\
&=&\frac{2q_{e}}{4\pi \epsilon _{o}}\frac{p}{L^{3}}
\end{eqnarray*}%
So if the dipole is about a $0.01\unit{cm}$ away 
\begin{eqnarray*}
F &=&\frac{2\left( 1.602\times 10^{-19}\right) }{4\pi \left( 8.85\times
10^{-12}\frac{\unit{C}^{2}}{\unit{N}\unit{m}^{2}}\right) }\frac{6.2\times
10^{-30}\unit{C}\unit{m}}{\left( 0.01\unit{cm}\right) ^{3}} \\
&=&1.\,\allowbreak 786\,2\times 10^{-26}\unit{N}
\end{eqnarray*}%
But wait! we used the dipole as the environmental object and the single
charge as the mover. So this is the force on the single charge! But by
Newton's third law, the force on the dipole due to the electron must have
the same magnitude and opposite direction so%
\begin{equation*}
F_{dipole}=-1.\,\allowbreak 786\,2\times 10^{-26}\unit{N}
\end{equation*}%
We could do this problem the other way, thinking of the point charge as the
environment and the dipole as the moving object. We know Coulomb's law for a
point charge. So we use it to find the force on the individual parts of the
dipole. We have to be careful because the minus charge is at a different $r$
value than the positive charge. 
\begin{eqnarray*}
-qE_{-}+qE_{+} &=&ma_{x} \\
-q\left( \frac{1}{4\pi \epsilon _{o}}\frac{Q}{r_{-}^{2}}\right) +q\left( 
\frac{1}{4\pi \epsilon _{o}}\frac{Q}{r_{+}^{2}}\right) &=&ma_{x}
\end{eqnarray*}%
or 
\begin{equation*}
\frac{Qq}{4\pi \epsilon _{o}}\left( \frac{1}{r_{+}^{2}}-\frac{1}{r_{-}^{2}}%
\right) =ma_{x}=F_{net}
\end{equation*}%
this is the net force on a dipole due to the point charge.

The effective charge on one side of the water molecule is 
\begin{eqnarray*}
q &=&\frac{p}{a}=\frac{6.2\times 10^{-30}\unit{C}\unit{m}}{3.9\times 10^{-12}%
\unit{m}} \\
&=&1.\,\allowbreak 589\,7\times 10^{-18}\allowbreak \unit{A}\unit{s}
\end{eqnarray*}%
(how can this be true?) so if the dipole is abut a $0.01\unit{cm}$ away then 
\begin{eqnarray*}
F_{net} &=&\frac{\left( 1.0\times 10^{-19}\right) \left( 1.\,\allowbreak
589\,7\times 10^{-18}\allowbreak \unit{A}\unit{s}\right) }{4\pi \left(
8.85\times 10^{-12}\frac{\unit{C}^{2}}{\unit{N}\unit{m}^{2}}\right) } \\
&&\times \left( \frac{1}{\left( 0.01\unit{cm}+\frac{3.9\times 10^{-12}\unit{m%
}}{2}\right) ^{2}}-\frac{1}{\left( 0.01\unit{cm}-\frac{3.9\times 10^{-12}%
\unit{m}}{2}\right) ^{2}}\right) \\
&=&-1.\,\allowbreak 115\,0\times 10^{-26}\unit{N}
\end{eqnarray*}%
We expect the negative sign, both forces should be to the left. The answers
are different, but within one order of magnitude. This is pretty good since
for our dipole field we assumed that the distance from the dipole is very
large and $0.01\unit{cm}$ is a somewhat shorter version of very large!

\section{Symmetry}

The symmetry of the uniform field figured strongly in the dipole problem.
When the shape of the field changed, so did the resulting motion. This
suggests that we could solve some problems just knowing the symmetry, or at
least that symmetry might help us do simple predictions to help get a
problem started. We need to be able to predict the field lines of a geometry
to draw a picture to start solving a problem.

We have run into two geometries so far that have been helpful\FRAME{dhF}{%
5.0462in}{2.8876in}{0in}{}{}{Figure}{\special{language "Scientific
Word";type "GRAPHIC";maintain-aspect-ratio TRUE;display "USEDEF";valid_file
"T";width 5.0462in;height 2.8876in;depth 0in;original-width
4.9917in;original-height 2.8444in;cropleft "0";croptop "1";cropright
"1";cropbottom "0";tempfilename 'LTUWDG9A.wmf';tempfile-properties "XPR";}}

The infinite line of charge and the semi-infinite sheet of charge. We have
found for the sheet that the field is constant everywhere. This is strongly
symmetric. We could envision translating the sheet within the plane right or
left. The field would look the same. We could envision reflecting the sheet
so the left side is now the right side. That would also not change the
field. We can say that the field of the sheet would be symmetric about
translation within the plane of the sheet and symmetric on reflection.

Suppose we look at the sheet side-on. Suppose that we thought the field came
off the sheet at an angle as shown. \FRAME{dhF}{1.9069in}{1.1857in}{0pt}{}{}{%
Figure}{\special{language "Scientific Word";type
"GRAPHIC";maintain-aspect-ratio TRUE;display "USEDEF";valid_file "T";width
1.9069in;height 1.1857in;depth 0pt;original-width 3.8787in;original-height
2.4007in;cropleft "0";croptop "1";cropright "1";cropbottom "0";tempfilename
'LTUWDH9B.wmf';tempfile-properties "XPR";}}Notice that if we shift the sheet
right or left, the field would still look the same, but if we reflected the
sheet about the $y$-axis. Then we would have \FRAME{dhF}{1.8395in}{1.1407in}{%
0pt}{}{}{Figure}{\special{language "Scientific Word";type
"GRAPHIC";maintain-aspect-ratio TRUE;display "USEDEF";valid_file "T";width
1.8395in;height 1.1407in;depth 0pt;original-width 4.2263in;original-height
2.6091in;cropleft "0";croptop "1";cropright "1";cropbottom "0";tempfilename
'LTUWDH9C.wmf';tempfile-properties "XPR";}}But (and here is the important
part) the shape of the charge distribution did not change on reflection. The
sheet really looks just the same. It dose not make sense that we should
change the shape of the field if the shape of the charge distribution did
not change. So we can tell that this can't be the right field shape.%
%TCIMACRO{%
%\TeXButton{Question 223.25.5}{\marginpar {
%\hspace{-0.5in}
%\begin{minipage}[t]{1in}
%\small{Question 223.25.5}
%\end{minipage}
%}}}%
%BeginExpansion
\marginpar {
\hspace{-0.5in}
\begin{minipage}[t]{1in}
\small{Question 223.25.5}
\end{minipage}
}%
%EndExpansion

We can do this with any symmetric distribution of charge. Think of the
infinite line of charge. If we move it left or right the field definitely
changes. So it is not symmetric about translation along, say, the $x$-axis.
But if we move the wire along it's own axis, (for my coordinate system,
along the $y$-axis) it should be symmetric because the charge distribution
won't look different. We can guess from the last example that the field must
come straight out perpendicular to the line of charge. It must be
perpendicular, but what direction? Look at this end view. The field lines do
come straight out, so this meets our criteria for being perpendicular to the
line. \FRAME{dhF}{1.4079in}{1.9865in}{0pt}{}{}{Figure}{\special{language
"Scientific Word";type "GRAPHIC";maintain-aspect-ratio TRUE;display
"USEDEF";valid_file "T";width 1.4079in;height 1.9865in;depth
0pt;original-width 1.8905in;original-height 2.6783in;cropleft "0";croptop
"1";cropright "1";cropbottom "0";tempfilename
'LTUWDH9D.wmf';tempfile-properties "XPR";}}We could rotate the line about
the axis of the line. Then the charge distribution would look just the same.
The field would also look just the same on rotation. But if we reflect the
charge distribution across the axis shown, the charge distribution looks
just the same, but the field would change.

\FRAME{dhF}{1.4252in}{2.0237in}{0pt}{}{}{Figure}{\special{language
"Scientific Word";type "GRAPHIC";maintain-aspect-ratio TRUE;display
"USEDEF";valid_file "T";width 1.4252in;height 2.0237in;depth
0pt;original-width 1.3898in;original-height 1.9847in;cropleft "0";croptop
"1";cropright "1";cropbottom "0";tempfilename
'LTUWDH9E.wmf';tempfile-properties "XPR";}}

We can tell that this is not the right field. We can tell that the field
should look more like this. \FRAME{dhF}{1.8827in}{2.2563in}{0pt}{}{}{Figure}{%
\special{language "Scientific Word";type "GRAPHIC";maintain-aspect-ratio
TRUE;display "USEDEF";valid_file "T";width 1.8827in;height 2.2563in;depth
0pt;original-width 2.3359in;original-height 2.8037in;cropleft "0";croptop
"1";cropright "1";cropbottom "0";tempfilename
'LTUWDH9F.wmf';tempfile-properties "XPR";}}

\subsection{Combinations of symmetric charge distributions}

%TCIMACRO{%
%\TeXButton{Question 223.25.6}{\marginpar {
%\hspace{-0.5in}
%\begin{minipage}[t]{1in}
%\small{Question 223.25.6}
%\end{minipage}
%}}}%
%BeginExpansion
\marginpar {
\hspace{-0.5in}
\begin{minipage}[t]{1in}
\small{Question 223.25.6}
\end{minipage}
}%
%EndExpansion
We can combine sheets or lines of charge to build more complex systems. We
did this to form a capacitor\FRAME{dhF}{2.0833in}{1.1684in}{0pt}{}{}{Figure}{%
\special{language "Scientific Word";type "GRAPHIC";maintain-aspect-ratio
TRUE;display "USEDEF";valid_file "T";width 2.0833in;height 1.1684in;depth
0pt;original-width 3.2396in;original-height 1.8049in;cropleft "0";croptop
"1";cropright "1";cropbottom "0";tempfilename
'LTUWDH9G.wmf';tempfile-properties "XPR";}}

The field lines follow our symmetry guidelines. Because of the symmetry of
the sheet of the field lines must be perpendicular to the sheets.

Again building from the line of charge, we can build more complex geometries%
\FRAME{dhF}{3.9972in}{2.93in}{0pt}{}{}{Figure}{\special{language "Scientific
Word";type "GRAPHIC";maintain-aspect-ratio TRUE;display "USEDEF";valid_file
"T";width 3.9972in;height 2.93in;depth 0pt;original-width
3.9487in;original-height 2.8859in;cropleft "0";croptop "1";cropright
"1";cropbottom "0";tempfilename 'LTUWDH9H.wmf';tempfile-properties "XPR";}}%
In the figure we have two positively charged concentric cylinders. The field
is very reminiscent of a line charge field, and we can see that it must be
using the same symmetry rules.

Of course the cylinders don't have to have the same charge.\FRAME{dhF}{%
1.3344in}{1.7919in}{0pt}{}{}{Figure}{\special{language "Scientific
Word";type "GRAPHIC";maintain-aspect-ratio TRUE;display "USEDEF";valid_file
"T";width 1.3344in;height 1.7919in;depth 0pt;original-width
5.687in;original-height 7.6605in;cropleft "0";croptop "1";cropright
"1";cropbottom "0";tempfilename 'LTUWDH9I.wmf';tempfile-properties "XPR";}}%
If the interior cylinder is positively charged and the exterior cylinder is
negatively charged, we have a situation much like the capacitor. Each
cylinder has a field outside the system, but those fields cancel out if
there are equal charges on each cylinder. This situation is similar to a
coaxial cable, and we will revisit it later in the course.\footnote{%
Indeed, this coaxial cables have a capacitance!}

For the charge configurations we have drawn so far, we must keep in mind
that they are infinite in at least one dimension. Finite configurations of
charge in lines or sheets will have curved fields at the ends. The fields
will be symmetric on reflection about their centers, but not on translation
of any sort. Still, we will continue to use semi-infinite approximations in
this class, and these constructs are good mental images under many
circumstances.

Of course we can have a sphere. Spheres are very symmetrical, so we can
guess using our symmetry ideas that the field from a charged sphere should
be perpendicular to the surface of the sphere everywhere.

\FRAME{dhF}{3.0701in}{2.0937in}{0pt}{}{}{Figure}{\special{language
"Scientific Word";type "GRAPHIC";maintain-aspect-ratio TRUE;display
"USEDEF";valid_file "T";width 3.0701in;height 2.0937in;depth
0pt;original-width 4.4486in;original-height 3.0251in;cropleft "0";croptop
"1";cropright "1";cropbottom "0";tempfilename
'LTUWDH9J.wmf';tempfile-properties "XPR";}}

We can see that this is true for both the sphere and for concentric spheres
or any configuration of charge that is spherical.

%TCIMACRO{%
%\TeXButton{Basic Equations}{\hspace{-1.3in}{\LARGE Basic Equations\vspace{0.25in}}}}%
%BeginExpansion
\hspace{-1.3in}{\LARGE Basic Equations\vspace{0.25in}}%
%EndExpansion

\begin{equation*}
\overrightarrow{\mathbf{\tau }}=\overrightarrow{\mathbf{p}}\times 
\overrightarrow{\mathbf{E}}
\end{equation*}

\chapter{Electric Flux}

%TCIMACRO{%
%\TeXButton{Fundamental Concepts}{\hspace{-1.3in}{\LARGE Fundamental Concepts\vspace{0.25in}}}}%
%BeginExpansion
\hspace{-1.3in}{\LARGE Fundamental Concepts\vspace{0.25in}}%
%EndExpansion

\begin{itemize}
\item Electric flux is the amount of electric field that penetrates an area.

\item An area vector is a vector normal to the area surface with a magnitude
equal to the area.

\item For closed surfaces, flux going in is negative and flux going out is
positive by convention.
\end{itemize}

\section{The Idea of Flux}

%TCIMACRO{%
%\TeXButton{Van de Graaff Generator Demo}{\marginpar {
%\hspace{-0.5in}
%\begin{minipage}[t]{1in}
%\small{Van de Graaff Generator Demo}
%\end{minipage}
%}}}%
%BeginExpansion
\marginpar {
\hspace{-0.5in}
\begin{minipage}[t]{1in}
\small{Van de Graaff Generator Demo}
\end{minipage}
}%
%EndExpansion
%TCIMACRO{%
%\TeXButton{Question 223.26.1}{\marginpar {
%\hspace{-0.5in}
%\begin{minipage}[t]{1in}
%\small{Question 223.26.1}
%\end{minipage}
%}}}%
%BeginExpansion
\marginpar {
\hspace{-0.5in}
\begin{minipage}[t]{1in}
\small{Question 223.26.1}
\end{minipage}
}%
%EndExpansion

If you took PH123 or have had a class that deals with fluids, I\ can use an
analogy (if not, you will probably be OK, because you have probably used a
garden hose). Let's recall some fluid dynamics for a moment. Remember what
we called a \emph{flow rate}? This was from the equation of continuity%
\begin{equation*}
v_{1}A_{1}=v_{2}A_{2}
\end{equation*}%
\FRAME{dtbpF}{2.917in}{1.6579in}{0pt}{}{}{Figure}{\special{language
"Scientific Word";type "GRAPHIC";maintain-aspect-ratio TRUE;display
"USEDEF";valid_file "T";width 2.917in;height 1.6579in;depth
0pt;original-width 2.8729in;original-height 1.6215in;cropleft "0";croptop
"1";cropright "1";cropbottom "0";tempfilename
'Electric_Flux/pipe0.wmf';tempfile-properties "XNPR";}}We wanted to know how
much liquid was going by a particular part of the pipe in a given unit of
time. We called $vA$ a \emph{flow rate}.

\subsection{The idea of electric flux}

I want to introduce an analogous concept. But this time I\ want to use the
electric field instead of water speed%
\begin{equation*}
\Phi =EA
\end{equation*}%
This is just like our flow rate in some ways. It is something multiplied by
an area. In fact, it is how much of something goes through an area. We could
guess that it is the amount of electric field that passes through the area, $%
A.$ Now the electric fields we have dealt with so far don't flow. They just
stay put (we will let them change later in the course). So it is only \emph{%
like }a flow rate. But it is useful to think of this as \textquotedblleft
how much of something passes by an area,\textquotedblright\ and the
\textquotedblleft something\textquotedblright\ is the electric field in this
case. Let's consider a picture \FRAME{dhF}{2.1326in}{2.0799in}{0pt}{}{}{%
Figure}{\special{language "Scientific Word";type
"GRAPHIC";maintain-aspect-ratio TRUE;display "USEDEF";valid_file "T";width
2.1326in;height 2.0799in;depth 0pt;original-width 4.3768in;original-height
4.2653in;cropleft "0";croptop "1";cropright "1";cropbottom "0";tempfilename
'Electric_Flux/flux_definition_1.wmf';tempfile-properties "XNPR";}}In this
picture, we have a rectangular area, $A,$ and the red arrows represent the
field lines of the electric field. We can picture the quantity, $\Phi ,$ as
the number of field lines that pass through $A.$ Remember that the number of
field lines we draw is greater if the field strength is higher, so this
quantity, $\Phi ,$ tells us something about the strength of the field over
the area.

%TCIMACRO{%
%\TeXButton{Question 223.26.2}{\marginpar {
%\hspace{-0.5in}
%\begin{minipage}[t]{1in}
%\small{Question 223.26.2}
%\end{minipage}
%}}}%
%BeginExpansion
\marginpar {
\hspace{-0.5in}
\begin{minipage}[t]{1in}
\small{Question 223.26.2}
\end{minipage}
}%
%EndExpansion
But, what if the area, $A,$ is not perpendicular to the field?\FRAME{dhF}{%
1.9873in}{1.9951in}{0pt}{}{}{Figure}{\special{language "Scientific
Word";type "GRAPHIC";maintain-aspect-ratio TRUE;display "USEDEF";valid_file
"T";width 1.9873in;height 1.9951in;depth 0pt;original-width
4.3768in;original-height 4.3941in;cropleft "0";croptop "1";cropright
"1";cropbottom "0";tempfilename
'Electric_Flux/flux_definition_2.wmf';tempfile-properties "XNPR";}}We define
an angle, $\theta $ (our favorite greek letter, but we could of course use $%
\beta $ or $\alpha ,$ or $\zeta $ or whatever) that is the angle between the
field direction and the area. A more mathematical way to do this is to
define a vector that is perpendicular to (normal to) the surface $\mathbf{%
\hat{n}}$. Then we can use this vector and one of the field lines to define $%
\theta .$ It will be the angle between $\mathbf{\hat{n}}$ and the field
lines.

\FRAME{dhF}{2.1084in}{2.4362in}{0pt}{}{}{Figure}{\special{language
"Scientific Word";type "GRAPHIC";maintain-aspect-ratio TRUE;display
"USEDEF";valid_file "T";width 2.1084in;height 2.4362in;depth
0pt;original-width 4.3768in;original-height 5.0609in;cropleft "0";croptop
"1";cropright "1";cropbottom "0";tempfilename
'Electric_Flux/flux_definition_3.wmf';tempfile-properties "XNPR";}}Of course
either way gives the same $\theta .$

Now our definition of $\Phi $ can be made to work. We want the number of
field lines passing through $A,$ but of course, now there are fewer lines
passing through the area because it is tilted. We can find $\Phi $ using $%
\theta $ as%
\begin{equation}
\Phi =EA\cos \theta
\end{equation}%
but let's consider what 
\begin{equation*}
A\cos \theta
\end{equation*}%
means. 
%TCIMACRO{%
%\TeXButton{Tip a flat object}{\marginpar {
%\hspace{-0.5in}
%\begin{minipage}[t]{1in}
%\small{Tip a flat object}
%\end{minipage}
%}}}%
%BeginExpansion
\marginpar {
\hspace{-0.5in}
\begin{minipage}[t]{1in}
\small{Tip a flat object}
\end{minipage}
}%
%EndExpansion
We can start with our original area. \FRAME{dhF}{2.5088in}{1.2522in}{0pt}{}{%
}{Figure}{\special{language "Scientific Word";type
"GRAPHIC";maintain-aspect-ratio TRUE;display "USEDEF";valid_file "T";width
2.5088in;height 1.2522in;depth 0pt;original-width 6.6565in;original-height
3.3088in;cropleft "0";croptop "1";cropright "1";cropbottom "0";tempfilename
'Electric_Flux/flux_definition_4.wmf';tempfile-properties "XNPR";}}If we tip
the area, it looks smaller\FRAME{dhF}{3.32in}{1.2894in}{0pt}{}{}{Figure}{%
\special{language "Scientific Word";type "GRAPHIC";maintain-aspect-ratio
TRUE;display "USEDEF";valid_file "T";width 3.32in;height 1.2894in;depth
0pt;original-width 8.5694in;original-height 3.3088in;cropleft "0";croptop
"1";cropright "1";cropbottom "0";tempfilename
'Electric_Flux/flux_definition_5.wmf';tempfile-properties "XNPR";}}The
smaller area is called the \emph{projected area}.

We can see that by tipping our area, we get fewer field lines that penetrate
that area.

Really the number of field lines is just proportional to $E,$ so we won't
ever really count field lines. But this is a good mental picture for what
flux means. Really we will calculate%
\begin{equation*}
\Phi =EA\cos \theta
\end{equation*}

The $\cos \theta $ with two magnitudes (field strength and area) multiplying
it should remind you of something. It looks like the result of a vector dot
product. If $E$ and $A$ were both vectors, then we could write the flux as%
\begin{equation}
\Phi =\overrightarrow{\mathbf{E}}\cdot \overrightarrow{\mathbf{A}}
\end{equation}%
%TCIMACRO{%
%\TeXButton{Domenstrate with a document with writing on one side}{\marginpar {
%\hspace{-0.5in}
%\begin{minipage}[t]{1in}
%\small{Domenstrate with a document with writing on one side}
%\end{minipage}
%}}}%
%BeginExpansion
\marginpar {
\hspace{-0.5in}
\begin{minipage}[t]{1in}
\small{Domenstrate with a document with writing on one side}
\end{minipage}
}%
%EndExpansion
Well, we can define a vector that has $A$ as it's magnitude and is in the
right direction to make 
\begin{equation*}
\overrightarrow{\mathbf{E}}\cdot \overrightarrow{\mathbf{A}}=EA\cos \theta
\end{equation*}%
We define the \emph{area vector}%
\begin{equation}
\overrightarrow{\mathbf{A}}=\mathbf{\hat{n}}A
\end{equation}

Notice that for an open surface (one that does not form a closed surface
with a empty space inside) we have to choose which side $\mathbf{\hat{n}}$
will point from. We can choose either side. But once we have made the
choice, we have to stick with it for the entire problem we are solving.

\subsection{Flux and Curved Areas}

%TCIMACRO{%
%\TeXButton{Trifold paper }{\marginpar {
%\hspace{-0.5in}
%\begin{minipage}[t]{1in}
%\small{Trifold paper }
%\end{minipage}
%}}}%
%BeginExpansion
\marginpar {
\hspace{-0.5in}
\begin{minipage}[t]{1in}
\small{Trifold paper }
\end{minipage}
}%
%EndExpansion
Suppose the area we have is not flat? Then what? Well let's recall that if
we take a sphere the surface will be curved. But if we take a bigger sphere,
and look at the same amount of area on that sphere, it looks less curved.%
\FRAME{dtbpF}{3.6997in}{2.7501in}{0pt}{}{}{Figure}{\special{language
"Scientific Word";type "GRAPHIC";maintain-aspect-ratio TRUE;display
"USEDEF";valid_file "T";width 3.6997in;height 2.7501in;depth
0pt;original-width 8.6403in;original-height 6.4135in;cropleft "0";croptop
"1";cropright "1";cropbottom "0";tempfilename
'Electric_Flux/radius_of_sphere_and_flatness1.wmf';tempfile-properties
"XNPR";}}This becomes more apparent if we remove the rest of the circle or
sphere to take away the visual cures our eyes and minds use to say something
is curved\FRAME{dhF}{3.9453in}{1.107in}{0pt}{}{}{Figure}{\special{language
"Scientific Word";type "GRAPHIC";maintain-aspect-ratio TRUE;display
"USEDEF";valid_file "T";width 3.9453in;height 1.107in;depth
0pt;original-width 10.7462in;original-height 2.9836in;cropleft "0";croptop
"1";cropright "1";cropbottom "0";tempfilename
'LTUWDH9K.wmf';tempfile-properties "XPR";}}

Suppose we take a curved surface but we just look at a very small part of
that surface. This would be very like magnifying our circle. We would see an
increasingly flat surface piece compared to our increased scale of our image.

This gives us the idea that for an element of area, $\Delta A$ we could find
an element of flux $\Delta \Phi $ for this small part of the whole curved
surface. Essentially $\Delta A$ is flat (or we would just take a smaller $%
\Delta A).$%
\begin{equation}
\Delta \Phi =\overrightarrow{\mathbf{E}}\cdot \Delta \overrightarrow{\mathbf{%
A}}
\end{equation}%
This is just a small piece of the total flux through the curved surface, the
total flux through our whole curved surface is 
\begin{equation}
\Phi _{E}\approx \dsum \Delta \Phi
\end{equation}

Of course, to make this exact, we will take the limit as $\Delta
A\rightarrow 0$ resulting in an integral. We find the flux through a curved
surface to be%
\begin{equation}
\Phi _{E}=\lim_{\Delta A\rightarrow 0}\dsum\limits_{i}\overrightarrow{%
\mathbf{E}}\cdot \Delta \overrightarrow{\mathbf{A}}_{i}=\int_{surface}%
\overrightarrow{\mathbf{E}}\cdot d\overrightarrow{\mathbf{A}}
\end{equation}%
Notice that this is a \emph{surface integral}. It may be that you have not
done surface integrals for some time, but we will practice in the upcoming
lectures.

\subsection{Closed surfaces}

Suppose we build a box with our areas. \FRAME{dhF}{3.3503in}{1.7461in}{0pt}{%
}{}{Figure}{\special{language "Scientific Word";type
"GRAPHIC";maintain-aspect-ratio TRUE;display "USEDEF";valid_file "T";width
3.3503in;height 1.7461in;depth 0pt;original-width 5.9923in;original-height
3.1081in;cropleft "0";croptop "1";cropright "1";cropbottom "0";tempfilename
'LTUWDH9L.wmf';tempfile-properties "XPR";}}Then we would have some lines
going in and some going out. By convention we will call the flux formed by
the ones going in negative and the flux formed by the ones going out
positive. 
%TCIMACRO{%
%\TeXButton{Question 223.26.3 Required}{\marginpar {
%\hspace{-0.5in}
%\begin{minipage}[t]{1in}
%\small{Question 223.26.3  Required}
%\end{minipage}
%}} }%
%BeginExpansion
\marginpar {
\hspace{-0.5in}
\begin{minipage}[t]{1in}
\small{Question 223.26.3  Required}
\end{minipage}
}
%EndExpansion
%TCIMACRO{%
%\TeXButton{Question 223.26.4  Required}{\marginpar {
%\hspace{-0.5in}
%\begin{minipage}[t]{1in}
%\small{Question 223.26.4  Required}
%\end{minipage}
%}} }%
%BeginExpansion
\marginpar {
\hspace{-0.5in}
\begin{minipage}[t]{1in}
\small{Question 223.26.4  Required}
\end{minipage}
}
%EndExpansion
%TCIMACRO{%
%\TeXButton{Question 223.26.5}{\marginpar {
%\hspace{-0.5in}
%\begin{minipage}[t]{1in}
%\small{Question 223.26.5}
%\end{minipage}
%}}}%
%BeginExpansion
\marginpar {
\hspace{-0.5in}
\begin{minipage}[t]{1in}
\small{Question 223.26.5}
\end{minipage}
}%
%EndExpansion
From these questions we see that if there is no charge inside of the box,
the net flux must be zero. We could take any size or shape of closed surface
and this would be true! But if we do have charge inside of the box we expect
there to be a net flux. If it is a negative net charge, it will be an
negative flux and if it is a positive net charge it will be a positive net
flux. Next lecture we will formalize this as a new law of physics, but for
now we need to remember from M215 or M113 how to write an integration over a
closed surface. We use a special integral sign with a circle

\begin{equation}
\Phi _{E}=\doint \overrightarrow{\mathbf{E}}\cdot d\overrightarrow{\mathbf{A}%
}
\end{equation}%
You will also see this written as%
\begin{equation}
\Phi _{E}=\doint E_{n}dA
\end{equation}%
where $E_{n}$ is the component of the field normal to the surface at the
point area increment $dA.$

\subsection{Flux example: a sphere}

For each type of surface we choose, we need an area element to perform the
integration. This is a lot like finding $dq$ in our electric field integral.
Let's take an example, a sphere.

We can start by finding the coordinates of a point, $P,$ on the surface of
the sphere.

\FRAME{dtbpF}{2.4215in}{2.4898in}{0pt}{}{}{Figure}{\special{language
"Scientific Word";type "GRAPHIC";maintain-aspect-ratio TRUE;display
"USEDEF";valid_file "T";width 2.4215in;height 2.4898in;depth
0pt;original-width 4.4529in;original-height 4.5783in;cropleft "0";croptop
"1";cropright "1";cropbottom "0";tempfilename
'LTUWDH9M.wmf';tempfile-properties "XPR";}}We define the coordinates in
terms of two angles, $\theta $ and $\phi .$ Let's look at them one at a
time. First $\theta $

\bigskip

\FRAME{dtbpF}{1.9112in}{2.0894in}{0pt}{}{}{Figure}{\special{language
"Scientific Word";type "GRAPHIC";maintain-aspect-ratio TRUE;display
"USEDEF";valid_file "T";width 1.9112in;height 2.0894in;depth
0pt;original-width 4.4529in;original-height 4.8698in;cropleft "0";croptop
"1";cropright "1";cropbottom "0";tempfilename
'LTUWDH9N.wmf';tempfile-properties "XPR";}}

and now $\phi $

\FRAME{dtbpF}{1.9527in}{2.0859in}{0pt}{}{}{Figure}{\special{language
"Scientific Word";type "GRAPHIC";maintain-aspect-ratio TRUE;display
"USEDEF";valid_file "T";width 1.9527in;height 2.0859in;depth
0pt;original-width 4.4529in;original-height 4.7573in;cropleft "0";croptop
"1";cropright "1";cropbottom "0";tempfilename
'LTUWDH9O.wmf';tempfile-properties "XPR";}}

Let's build an area by defining a sort of box shape on the surface by
allowing a change in $\theta $ and $\phi $ ($\Delta \theta $ and $\Delta
\phi $). First $\Delta \theta ,$

\FRAME{dtbpF}{1.932in}{2.0989in}{0pt}{}{}{Figure}{\special{language
"Scientific Word";type "GRAPHIC";maintain-aspect-ratio TRUE;display
"USEDEF";valid_file "T";width 1.932in;height 2.0989in;depth
0pt;original-width 4.4529in;original-height 4.8404in;cropleft "0";croptop
"1";cropright "1";cropbottom "0";tempfilename
'LTUWDH9P.wmf';tempfile-properties "XPR";}}The angle $\theta $ just defines
a circle that passes through the \textquotedblleft north
pole\textquotedblright\ and \textquotedblleft south pole\textquotedblright\
of our sphere. By changing $\theta $ we get a small bit of arc length. We
remember that the length of an arc is 
\begin{equation}
s_{\theta }=r\theta
\end{equation}%
where $\theta $ is in radians. So we expect that 
\begin{equation}
\Delta s_{\theta }=r\Delta \theta
\end{equation}%
We can check this by integrating%
\begin{equation}
\int_{0}^{2\pi }rd\theta =r\int_{0}^{2\pi }d\theta =2\pi r
\end{equation}%
Just as we expect, the integral of arc length around the whole circle is the
circumference of the circle. Then $\Delta s_{\theta }$ is one side of our
small box-like area, the box height.

Now let's look at $\phi $

\FRAME{dtbpF}{2.4846in}{2.5529in}{0pt}{}{}{Figure}{\special{language
"Scientific Word";type "GRAPHIC";maintain-aspect-ratio TRUE;display
"USEDEF";valid_file "T";width 2.4846in;height 2.5529in;depth
0pt;original-width 4.4529in;original-height 4.5783in;cropleft "0";croptop
"1";cropright "1";cropbottom "0";tempfilename
'LTUWDI9Q.wmf';tempfile-properties "XPR";}}$\phi $ also forms a circle on
the sphere, but it's size depends on $\theta .$ Near the north pole, the
radius of the $\phi $-circle is very small. At $\theta =90\unit{%
%TCIMACRO{\U{b0}}%
%BeginExpansion
{{}^\circ}%
%EndExpansion
}$, the $\phi $-circle is in the $xy$ plane and has radius $r.$ We can write
the radius of the $\phi $-circle as a projection over $90\unit{%
%TCIMACRO{\U{b0}}%
%BeginExpansion
{{}^\circ}%
%EndExpansion
}-\theta $ which gives us a radius of $r\sin \theta .$ Then we use the arc
length formula again to find 
\begin{equation}
s_{\phi }=\left( r\sin \theta \right) \phi
\end{equation}%
a change in arch length will be%
\begin{equation}
\Delta s_{\phi }=\left( r\sin \theta \right) \Delta \phi
\end{equation}%
\FRAME{dtbpF}{1.7184in}{1.8507in}{0pt}{}{}{Figure}{\special{language
"Scientific Word";type "GRAPHIC";maintain-aspect-ratio TRUE;display
"USEDEF";valid_file "T";width 1.7184in;height 1.8507in;depth
0pt;original-width 4.5212in;original-height 4.8698in;cropleft "0";croptop
"1";cropright "1";cropbottom "0";tempfilename
'LTUWDI9R.wmf';tempfile-properties "XPR";}}This is the other side of our
box, the box width.

Now let's combine them. We multiply $\Delta s_{\theta }\times \Delta s_{\phi
}$ to obtain a roughly rectangular area.\FRAME{dtbpF}{3.5717in}{2.7337in}{0pt%
}{}{}{Figure}{\special{language "Scientific Word";type
"GRAPHIC";maintain-aspect-ratio TRUE;display "USEDEF";valid_file "T";width
3.5717in;height 2.7337in;depth 0pt;original-width 5.9914in;original-height
4.5783in;cropleft "0";croptop "1";cropright "1";cropbottom "0";tempfilename
'LTUWDI9S.wmf';tempfile-properties "XPR";}}%
\begin{equation}
\Delta A\approx \Delta s_{\theta }\times \Delta s_{\phi }=r\Delta \theta
r\sin \theta \Delta \phi
\end{equation}%
which is the area of our small box. We have found an element of area on the
surface of the sphere! Let's check our element of area by integration. After
changing $\Delta $ to $d$ and rearranging%
\begin{equation}
dA=r^{2}\sin \theta d\theta d\phi
\end{equation}%
then 
\begin{equation}
A=\int \int r^{2}\sin \theta d\theta d\phi
\end{equation}%
we have to be careful not to over count area. Let's view this as first
integrating around the circle of radius $r\sin \theta $ over the variable $%
\phi ,$ then an integration of all these circles as $\theta $ changes from $%
0 $ to $\pi $

\begin{eqnarray}
A &=&\int_{0}^{\pi }\int_{0}^{2\pi }r^{2}\sin \theta d\phi d\theta \\
&=&r^{2}\int_{0}^{\pi }\sin \theta d\theta \int_{0}^{2\pi }d\phi  \notag \\
&=&2\pi r^{2}\int_{0}^{\pi }\sin \theta d\theta  \notag \\
&=&4\pi r^{2}  \notag
\end{eqnarray}%
as we expect.

We are now ready to do a simple problem.\FRAME{dhF}{1.8749in}{1.0118in}{0pt}{%
}{}{Figure}{\special{language "Scientific Word";type
"GRAPHIC";maintain-aspect-ratio TRUE;display "USEDEF";valid_file "T";width
1.8749in;height 1.0118in;depth 0pt;original-width 3.2119in;original-height
1.7201in;cropleft "0";croptop "1";cropright "1";cropbottom "0";tempfilename
'LTUWDI9T.wmf';tempfile-properties "XPR";}}Let's calculate the flux through
a spherical surface if there is a point charge at the center of the sphere.
The field of the point charge is 
\begin{equation*}
\overrightarrow{\mathbf{E}}=\frac{1}{4\pi \epsilon _{o}}\frac{Q_{E}}{r^{2}}%
\mathbf{\hat{r}}
\end{equation*}%
then the flux through the surface is 
\begin{eqnarray*}
\Phi _{E} &=&\doint \mathbf{\vec{E}\cdot d\vec{A}} \\
&=&\doint \frac{1}{4\pi \epsilon _{o}}\frac{Q_{E}}{r^{2}}\mathbf{\hat{r}%
\cdot d\vec{A}}
\end{eqnarray*}%
but $\mathbf{\hat{r}}$ is always in the same direction as $\mathbf{d\vec{A}}$
for this case, so 
\begin{equation*}
\mathbf{\hat{r}\cdot d\vec{A}}=(1)dA\cos \left( 0\right) =dA
\end{equation*}%
which gives us just 
\begin{eqnarray*}
\Phi _{E} &=&\frac{Q_{E}}{4\pi \epsilon _{o}}\doint \frac{1}{r^{2}}dA \\
&=&\frac{Q_{E}}{4\pi \epsilon _{o}}\doint \frac{1}{r^{2}}r^{2}\sin \theta
d\theta d\phi \\
&=&\frac{Q_{E}}{4\pi \epsilon _{o}}\int_{0}^{\pi }\left( \int_{0}^{2\pi
}d\phi \right) \sin \theta d\theta \\
&=&\frac{Q_{E}}{4\pi \epsilon _{o}}4\pi \\
&=&\frac{Q_{E}}{\epsilon _{o}}
\end{eqnarray*}

Some comments are in order. Our surfaces that we are using to calculate flux
might be a real object. You might calculate the electric flux leaving a
microwave oven, or a computer case to make sure you are in keeping emissions
within FCC rules. But more likely the surface is purely imaginary--\emph{%
just something we make up}.

Symmetry is going to be very important in doing problems with flux. So we
will often make up very symmetrical surfaces to help us with our problems.
In today's problem, the fact that $\mathbf{\hat{r}}$ and $d\mathbf{A}$ were
in the same direction made the integral \emph{much} easier.

Until next lecture, it may not seem beneficial to invent some strange
symmetrical surface and then to calculate the flux through that surface. But
it is, and it will have the effect of turning a long, difficult integral
into a simple one, when we can pull it off.

\subsection{Flux example: a long straight wire}

Let's take another example. A long straight wire.\FRAME{dhF}{2.2087in}{%
1.8836in}{0in}{}{}{Figure}{\special{language "Scientific Word";type
"GRAPHIC";maintain-aspect-ratio TRUE;display "USEDEF";valid_file "T";width
2.2087in;height 1.8836in;depth 0in;original-width 2.1689in;original-height
1.8455in;cropleft "0";croptop "1";cropright "1";cropbottom "0";tempfilename
'LZJWNQ00.wmf';tempfile-properties "XPR";}}

We remember that the field from a long straight wire is approximately 
\begin{equation*}
E=\frac{1}{4\pi \epsilon _{o}}\frac{2\left\vert \lambda \right\vert }{r}
\end{equation*}%
The symmetry of the field suggests an imaginary surface for measuring the
flux. A cylinder matches the geometry well. Let's find the flux through an
imaginary cylinder that is $L$ tall and has a radius $r$ and is concentric
with the line of charge. Note that we are totally making up the cylindrical
surface. There is not really any surface there at all.

The flux will be 
\begin{equation*}
\Phi _{E}=\doint \mathbf{\vec{E}\cdot d\vec{A}}
\end{equation*}%
We can view this as three separate integrals 
\begin{equation*}
\Phi _{E}=\doint_{top}\mathbf{\vec{E}\cdot d\vec{A}+}\doint_{side}\mathbf{%
\vec{E}\cdot d\vec{A}+}\doint_{bottom}\mathbf{\vec{E}\cdot d\vec{A}}
\end{equation*}%
since our cylinder has end caps (the top and bottom) and a curved side.

Let's consider the end caps first. For both the top and the bottom ends, $%
\mathbf{\vec{E}\cdot d\vec{A}}=0$ everywhere. No field goes thorough the
ends. So there is no flux through the ends of the cylinder.

There is flux through the side of the cylinder. Note that the field is
perpendicular to the side surface everywhere. So $\mathbf{\vec{E}\cdot d\vec{%
A}}=EdA.$ We can write our flux as 
\begin{eqnarray*}
\Phi _{E} &=&\doint_{side}EdA \\
&=&\doint \frac{1}{4\pi \epsilon _{o}}\frac{2\left\vert \lambda \right\vert 
}{r}dA
\end{eqnarray*}%
Integrated over the side surface. But we will need an element of surface
area $dA$ for a cylinder side. 
%TCIMACRO{%
%\TeXButton{Question 223.26.6}{\marginpar {
%\hspace{-0.5in}
%\begin{minipage}[t]{1in}
%\small{Question 223.26.6}
%\end{minipage}
%}}}%
%BeginExpansion
\marginpar {
\hspace{-0.5in}
\begin{minipage}[t]{1in}
\small{Question 223.26.6}
\end{minipage}
}%
%EndExpansion
Cylindrical coordinates seem logical so let's try%
\begin{equation*}
dA=rd\theta dz
\end{equation*}%
then%
\begin{eqnarray*}
\Phi _{E} &=&\doint \doint \frac{1}{4\pi \epsilon _{o}}\frac{2\left\vert
\lambda \right\vert }{r}rd\theta dz \\
&=&\frac{2\left\vert \lambda \right\vert }{4\pi \epsilon _{o}}%
\int_{0}^{L}\int_{0}^{2\pi }d\theta dz \\
&=&\frac{\left\vert \lambda \right\vert }{2\pi \epsilon _{o}}\left( 2\pi
L\right) \\
&=&\frac{\left\vert \lambda \right\vert }{\epsilon _{o}}L
\end{eqnarray*}%
So far we have, indeed, made integrals that look hard but are really easy to
do. But note that this would be \emph{much} harder if the wire were not at
the center of the cylinder, or if in the previous example the charge had
been off to one side of the sphere.

We would still like to remove such difficulties if we can. And often we can
by choosing our imaginary surface so that the symmetry is there. But
sometimes that is harder. or worse yet, we don't know exactly where the
charges are in a complicated configuration of charge. We will take this on
next lecture when we study a technique for finding the electric field
invented by Gauss.

%TCIMACRO{%
%\TeXButton{Basic Equations}{\hspace{-1.3in}{\LARGE Basic Equations\vspace{0.25in}}}}%
%BeginExpansion
\hspace{-1.3in}{\LARGE Basic Equations\vspace{0.25in}}%
%EndExpansion

The electric flux is defined as%
\begin{equation*}
\Phi _{E}=\overrightarrow{\mathbf{E}}\cdot \overrightarrow{\mathbf{A}}%
=EA\cos \theta
\end{equation*}%
where the area vector is given by%
\begin{equation*}
\overrightarrow{\mathbf{A}}=\mathbf{\hat{n}}A
\end{equation*}%
and for a curved area, we integrate

\begin{equation*}
\Phi _{E}=\doint \mathbf{\vec{E}\cdot d\vec{A}}
\end{equation*}

\chapter{Gauss' Law and its Applications}

%TCIMACRO{%
%\TeXButton{Fundamental Concepts}{\hspace{-1.3in}{\LARGE Fundamental Concepts\vspace{0.25in}}}}%
%BeginExpansion
\hspace{-1.3in}{\LARGE Fundamental Concepts\vspace{0.25in}}%
%EndExpansion

\begin{itemize}
\item Gauss' Law tells us that the flux through a closed surface is equal to
the charge inside the surface divided by $\epsilon _{o}$: 
\begin{equation*}
\Phi =\frac{Q_{in}}{\epsilon _{o}}
\end{equation*}

\item Gauss' Law combined with our basic flux equation%
\begin{equation*}
\Phi _{E}=\doint \overrightarrow{\mathbf{E}}\cdot d\overrightarrow{\mathbf{A}%
}=\frac{Q_{E}}{\epsilon _{o}}
\end{equation*}
\end{itemize}

\section{Gauss' Law}

Last lecture we did two problems. We found the flux from a point charge
through a spherical surface to be\FRAME{dhF}{1.4581in}{0.787in}{0pt}{}{}{%
Figure}{\special{language "Scientific Word";type
"GRAPHIC";maintain-aspect-ratio TRUE;display "USEDEF";valid_file "T";width
1.4581in;height 0.787in;depth 0pt;original-width 3.2119in;original-height
1.7201in;cropleft "0";croptop "1";cropright "1";cropbottom "0";tempfilename
'LTUWDI9V.wmf';tempfile-properties "XPR";}}%
\begin{equation*}
\Phi _{sphere,point}=\frac{Q_{E}}{\epsilon _{o}}
\end{equation*}%
and the flux from a line of charge through a cylinder to be\FRAME{dhF}{%
0.7524in}{0.9997in}{0pt}{}{}{Figure}{\special{language "Scientific
Word";type "GRAPHIC";maintain-aspect-ratio TRUE;display "USEDEF";valid_file
"T";width 0.7524in;height 0.9997in;depth 0pt;original-width
1.5714in;original-height 2.0954in;cropleft "0";croptop "1";cropright
"1";cropbottom "0";tempfilename 'LTUWDI9W.wmf';tempfile-properties "XPR";}}%
\begin{equation*}
\Phi _{cylindar,line}=\frac{\left\vert \lambda \right\vert }{\epsilon _{o}}L
\end{equation*}%
Let's rewrite the last one using 
\begin{equation*}
\lambda =\frac{Q}{L}
\end{equation*}%
then 
\begin{eqnarray*}
\Phi _{cylindar,line} &=&\frac{\left\vert Q_{E}/L\right\vert }{\epsilon _{o}}%
L \\
&=&\frac{\left\vert Q_{E}\right\vert }{\epsilon _{o}}
\end{eqnarray*}%
which is just what we got for the point charge and sphere! That is amazing!
Think about how much work it was to find each flux, and in the end we got
the same result. Wouldn't it be great if the flux through every closed
surface was this simple? Then we would not have to integrate at all!

To see if we can do this, first let's think of our answer.%
\begin{equation*}
\Phi _{sphere,point}=\frac{Q_{E}}{\epsilon _{o}}
\end{equation*}%
%TCIMACRO{%
%\TeXButton{Question 223.27.1}{\marginpar {
%\hspace{-0.5in}
%\begin{minipage}[t]{1in}
%\small{Question 223.27.1}
%\end{minipage}
%}}}%
%BeginExpansion
\marginpar {
\hspace{-0.5in}
\begin{minipage}[t]{1in}
\small{Question 223.27.1}
\end{minipage}
}%
%EndExpansion
It does not depend on the radius of the spherical surface. So any spherical
surface centered on the charge will do! This makes sense. No matter how big
the sphere, all the field lines must leave it. Since flux gives the amount
of field that penetrates an area, for our charge at the center of a sphere
we see that all of the field penetrates the spherical surface no matter the
size of the sphere. So the flux is the same no matter $r.$\footnote{%
If this still seems strange, remember that the area of a sphere is $4\pi
r^{2}$ and that the field of a point charge is $\frac{1}{4\pi \epsilon _{o}}%
\frac{Q}{r^{2}}$. The flux is like the product of these two quantities. The $%
r^{2}$ terms must cancel. So the fact that the flux is the same for any
sphere is due to the $r^{2}$ dependence of the field.}

\FRAME{dhF}{1.9355in}{1.8256in}{0pt}{}{}{Figure}{\special{language
"Scientific Word";type "GRAPHIC";maintain-aspect-ratio TRUE;display
"USEDEF";valid_file "T";width 1.9355in;height 1.8256in;depth
0pt;original-width 4.0041in;original-height 3.7749in;cropleft "0";croptop
"1";cropright "1";cropbottom "0";tempfilename
'LTUWDI9X.wmf';tempfile-properties "XPR";}}The key to making our last
lecture problems easy was that the field was always perpendicular to the
surface so $\overrightarrow{\mathbf{E}}\cdot d\overrightarrow{\mathbf{A}}%
=EdA $ was easy to find.

Using geometry we can arrange to make nearly all of our flux problems like
this. To demonstrate, let's take the case of a point charge that is off
center in a spherical surface.\FRAME{dhF}{2.7121in}{1.6743in}{0pt}{}{}{Figure%
}{\special{language "Scientific Word";type "GRAPHIC";maintain-aspect-ratio
TRUE;display "USEDEF";valid_file "T";width 2.7121in;height 1.6743in;depth
0pt;original-width 2.6697in;original-height 1.638in;cropleft "0";croptop
"1";cropright "1";cropbottom "0";tempfilename
'LTUWDI9Y.wmf';tempfile-properties "XPR";}}Remember, we made up this
surface. So we can place the surface anywhere we like. And this time we
would like the charge to be off center. We will call these made up surfaces 
\emph{Gaussian surfaces} after the mathematician that thought up this method
of avoiding integrals. Having the charge off center would make for a
difficult integration because $\overrightarrow{\mathbf{E}}$ and $d%
\overrightarrow{\mathbf{A}}$ have different directions as we go around the
sphere. But let's consider, would there be less flux through the surface
than there was when the charge was centered in the sphere? Every field line
that is generated will still leave the surface. Flux gives us the amount of
field that penetrates the surface.\footnote{%
Think of water flow rate again. We could place the end of a garden hose in a
wire mesh container. The water would flow out the hose end and through the
wire mesh sides of the container. The flow rate tells us how much water
passes through the container surface. The flow rate does not depend on the
shape of the container. The hose end is like a charge. The hose is the
source of water, the charge is the source of electric field.} Since flux is
the amount of field penetrating our surface, it seems that the flux should
be exactly the same as when the charge was in the center of the sphere. To
prove this, let's take our surface and approximate it using area segments.
But let's have the area segments be either along a radius of a sphere
centered on the charge, or along the surface of a sphere centered on the
charge.\FRAME{dhF}{3.2984in}{2.162in}{0pt}{}{}{Figure}{\special{language
"Scientific Word";type "GRAPHIC";maintain-aspect-ratio TRUE;display
"USEDEF";valid_file "T";width 3.2984in;height 2.162in;depth
0pt;original-width 3.2534in;original-height 2.1231in;cropleft "0";croptop
"1";cropright "1";cropbottom "0";tempfilename
'LTUWDI9Z.wmf';tempfile-properties "XPR";}}No flux goes through the radial
pieces. And the rest of the pieces are all parts of spheres centered on the
charge. But for the spherical segments, the field will be perpendicular to
the segment no matter what sphere the segment is a part of, because we chose
only spheres that were concentric with the charge. The $r$ we have for the
little spherical pieces does not matter, so on all of these surfaces $%
\overrightarrow{\mathbf{E}}\cdot d\overrightarrow{\mathbf{A}}=EdA.$ Then the
integration for these pieces will be easy.

Of course this surface made of little segments from other spheres is a poor
approximation to the shape of the offset sphere. But we can make our small
segments smaller and smaller. In the limit that they are infinitely small,
our shape becomes the offset sphere. That means that once again our flux is%
\begin{equation*}
\Phi =\frac{Q_{E}}{\epsilon _{o}}
\end{equation*}%
This is fantastic! We don't have to do the integration at all. We just count
up the charge inside our surface and divide by $\epsilon _{o}.$

What happens if the charge is on the outside of the surface?\FRAME{dhF}{%
2.5694in}{2.4223in}{0pt}{}{}{Figure}{\special{language "Scientific
Word";type "GRAPHIC";maintain-aspect-ratio TRUE;display "USEDEF";valid_file
"T";width 2.5694in;height 2.4223in;depth 0pt;original-width
3.2396in;original-height 3.0528in;cropleft "0";croptop "1";cropright
"1";cropbottom "0";tempfilename 'LTUWDIA0.wmf';tempfile-properties "XPR";}}%
Every field line that enters goes back out. We encountered this last time.
The flux going in is negative, the flux going out is positive, and they must
be the same because every line leaves that enters. So the net flux must be
zero. The means we should still write our flux as%
\begin{equation*}
\Phi =\frac{Q_{inside}}{\epsilon _{o}}
\end{equation*}%
because outside charges won't contribute to the flux. So in a way, our
expression works for charges outside our closed surface.

We know that fields superimpose, that is, they add up, so we would expect
that if we have two charges inside a surface, \FRAME{dhF}{3.0191in}{2.4976in%
}{0pt}{}{}{Figure}{\special{language "Scientific Word";type
"GRAPHIC";maintain-aspect-ratio TRUE;display "USEDEF";valid_file "T";width
3.0191in;height 2.4976in;depth 0pt;original-width 2.975in;original-height
2.4569in;cropleft "0";croptop "1";cropright "1";cropbottom "0";tempfilename
'LTUWDIA1.wmf';tempfile-properties "XPR";}}we would add up their
contributions to the total flux%
\begin{equation*}
\Phi _{total}=\Phi _{1}+\Phi _{2}
\end{equation*}%
which means that $Q_{inside}$ is the sum of all the charges inside. We
recognize that if some charges are negative, they will cancel equal amounts
of charge that are positive.

This leaves us with a fantastic time savings law

%TCIMACRO{%
%\TeXButton{Note}{\begin{Note}
%The electric flux $\Phi $ through any closed surface is equal to the net
%charge inside the surface multiplied by $4\pi k_{e}.$ The closed surface is
%often called a \emph{Gaussian Surface.}\begin{equation}
%\Phi _{E}=\doint \vec{E}\cdot d\vec{A}=\frac{Q_{inside}}{\varepsilon _{o}}
%\end{equation}
%\end{Note}}}%
%BeginExpansion
\begin{Note}
The electric flux $\Phi $ through any closed surface is equal to the net
charge inside the surface multiplied by $4\pi k_{e}.$ The closed surface is
often called a \emph{Gaussian Surface.}\begin{equation}
\Phi _{E}=\doint \vec{E}\cdot d\vec{A}=\frac{Q_{inside}}{\varepsilon _{o}}
\end{equation}
\end{Note}%
%EndExpansion

This was first expressed by Gauss, and therefore this expression is called
Gauss' law.

\section{Examples of Gauss' Law}

%TCIMACRO{%
%\TeXButton{Question 223.27.2}{\marginpar {
%\hspace{-0.5in}
%\begin{minipage}[t]{1in}
%\small{Question 223.27.2}
%\end{minipage}
%}}}%
%BeginExpansion
\marginpar {
\hspace{-0.5in}
\begin{minipage}[t]{1in}
\small{Question 223.27.2}
\end{minipage}
}%
%EndExpansion
But why do we get so excited about flux? The reason is that we can use the
idea of flux combined with Gauss' law gives us an easy way to calculate the
electric field from a distribution of charge if we can find a suitable
symmetric surface! If we can find the field, we can find forces, and we can
predict motion.

Let's show how to do this by working some examples.

\subsubsection{Charged Spherical Shell}

First let's take a charged spherical shell and find the field inside.\FRAME{%
fhF}{1.9147in}{2.0764in}{0pt}{}{}{Figure}{\special{language "Scientific
Word";type "GRAPHIC";maintain-aspect-ratio TRUE;display "USEDEF";valid_file
"T";width 1.9147in;height 2.0764in;depth 0pt;original-width
3.4307in;original-height 3.7239in;cropleft "0";croptop "1";cropright
"1";cropbottom "0";tempfilename
'Electric_Flux/Ring_of_charge.wmf';tempfile-properties "XNPR";}}We need to
be able to guess the shape of the field. We use symmetry. We can guess that
the field will be radial both inside and outside of the shell. If it were
not so, then our symmetry tests would fail.

The shell has a total charge of $+Q$. If we place a spherical surface inside
the shell, then we can use Gauss's law.%
\begin{equation*}
\Phi =\frac{Q_{inside}}{\epsilon _{o}}
\end{equation*}%
We can tell from the symmetry of the situation that $\overrightarrow{\mathbf{%
E}}$ is everywhere colinear with (but in the opposite direction as) $d%
\overrightarrow{\mathbf{A}}$ so 
\begin{equation*}
\Phi =\doint \overrightarrow{\mathbf{E}}\cdot d\overrightarrow{\mathbf{A}}%
=-\doint EdA
\end{equation*}%
because the field is everywhere perpendicular to the surface. We can even
make a guess that the field must be constant on this surface, because all
along the spherical Gaussian surface there is extreme symmetry. No change in
reflection, or rotation etc. will change the shape of the charge, so around
the spherical surface the field must have the same value. Then 
\begin{equation*}
\Phi =-E\doint dA=-EA
\end{equation*}%
Equating our flux equations gives

\begin{equation*}
-EA=\frac{Q_{inside}}{\epsilon _{o}}
\end{equation*}%
or%
\begin{equation*}
E=-\frac{Q_{inside}}{A\epsilon _{o}}
\end{equation*}%
but what is $Q_{inside}?$ It is zero! so%
\begin{equation*}
E=-\frac{0}{A\epsilon _{o}}=0
\end{equation*}%
There is no net field inside!

This may seem surprising, but think of placing a test charge, $q_{o}$,
inside the sphere. The next figure shows the forces acting on such a test
charge. The force is stronger between the charge and the near surface, but
there is more of the surface tugging the other way. \FRAME{dhF}{1.593in}{%
1.7158in}{0pt}{}{}{Figure}{\special{language "Scientific Word";type
"GRAPHIC";maintain-aspect-ratio TRUE;display "USEDEF";valid_file "T";width
1.593in;height 1.7158in;depth 0pt;original-width 1.5575in;original-height
1.6786in;cropleft "0";croptop "1";cropright "1";cropbottom "0";tempfilename
'LTUWDIA2.wmf';tempfile-properties "XPR";}}The forces just balance. Since 
\begin{equation*}
F=qE
\end{equation*}%
if the net force is zero, then the field must be zero too.

%TCIMACRO{%
%\TeXButton{Question 223.27.3}{\marginpar {
%\hspace{-0.5in}
%\begin{minipage}[t]{1in}
%\small{Question 223.27.3}
%\end{minipage}
%}}}%
%BeginExpansion
\marginpar {
\hspace{-0.5in}
\begin{minipage}[t]{1in}
\small{Question 223.27.3}
\end{minipage}
}%
%EndExpansion
Is there a field outside of the spherical shell? It is still true that%
\begin{equation*}
\Phi =\doint \overrightarrow{\mathbf{E}}\cdot d\overrightarrow{\mathbf{A}}%
=\doint EdA
\end{equation*}%
but this time we have a positive sign on the last integral because $%
\overrightarrow{\mathbf{E}}$ and $d\overrightarrow{\mathbf{A}}$ are in the
same direction. Then 
\begin{equation*}
EA=+\frac{Q_{inside}}{\epsilon _{o}}
\end{equation*}%
We now choose our surface around the entire shell\FRAME{fhF}{1.567in}{%
1.6527in}{0pt}{}{}{Figure}{\special{language "Scientific Word";type
"GRAPHIC";maintain-aspect-ratio TRUE;display "USEDEF";valid_file "T";width
1.567in;height 1.6527in;depth 0pt;original-width 3.6443in;original-height
3.8458in;cropleft "0";croptop "1";cropright "1";cropbottom "0";tempfilename
'Electric_Flux/ring_of_charge_2.wmf';tempfile-properties "XNPR";}}All of our
analysis is the same as in our last problem, except now $Q_{inside}$ is not
zero%
\begin{equation*}
E=\frac{Q_{inside}}{A\epsilon _{o}}
\end{equation*}%
The area is the area of our imaginary sphere%
\begin{equation*}
E=\frac{Q_{inside}}{\left( 4\pi r^{2}\right) \epsilon _{o}}
\end{equation*}%
and since $Q_{inside}=+Q,$ then%
\begin{equation*}
E=\frac{+Q}{4\pi \epsilon _{o}r^{2}}
\end{equation*}%
and we have found the field.

Note that this field looks very like a point charge at the center of the
spherical shell (at the center of charge), but by now that is not much of a
surprise!

\subsubsection{Strategy for Gauss' law problems}

Let's review what we have done before we go on to our last example. For each
Gauss' law problem, we

\begin{enumerate}
\item draw the charge distribution \FRAME{dtbpF}{1.0564in}{1.0786in}{0in}{}{%
}{Figure}{\special{language "Scientific Word";type
"GRAPHIC";maintain-aspect-ratio TRUE;display "USEDEF";valid_file "T";width
1.0564in;height 1.0786in;depth 0in;original-width 1.0475in;original-height
1.0706in;cropleft "0";croptop "1";cropright "1";cropbottom "0";tempfilename
'N1IQ5L00.wmf';tempfile-properties "XPR";}}

\item Draw the field lines using symmetry \FRAME{dtbpF}{1.6525in}{1.6533in}{%
0in}{}{}{Figure}{\special{language "Scientific Word";type
"GRAPHIC";maintain-aspect-ratio TRUE;display "USEDEF";valid_file "T";width
1.6525in;height 1.6533in;depth 0in;original-width 1.6551in;original-height
1.656in;cropleft "0";croptop "1";cropright "1";cropbottom "0";tempfilename
'N1IQ6E01.wmf';tempfile-properties "XPR";}}

\item Choose (make up, invent) a closed surface that makes $\overrightarrow{%
\mathbf{E}}\cdot d\overrightarrow{\mathbf{A}}$ either just $EdA$ or $0.$%
\FRAME{dtbpF}{1.9301in}{1.9318in}{0pt}{}{}{Figure}{\special{language
"Scientific Word";type "GRAPHIC";maintain-aspect-ratio TRUE;display
"USEDEF";valid_file "T";width 1.9301in;height 1.9318in;depth
0pt;original-width 1.938in;original-height 1.9398in;cropleft "0";croptop
"1";cropright "1";cropbottom "0";tempfilename
'N1IQ8X02.wmf';tempfile-properties "XPR";}}

\item Find $Q_{in}.$

\item Solve $\doint EdA=\frac{Q_{inside}}{\epsilon _{o}}$ for the non, zero
parts
\end{enumerate}

The integral should be trivial now due to our use of symmetry.

\subsubsection{An infinite sheet of charge.}

Spherical cases were easy. Let's try a harder one. Let's try our infinite
sheet of charge. It is a little hard to draw. So we will draw it looking at
it from the side from within the sheet of charge (somewhere in it's middle,
if an infinite sheet can have a middle).\FRAME{dtbpF}{3.6224in}{0.2253in}{0in%
}{}{}{Figure}{\special{language "Scientific Word";type
"GRAPHIC";maintain-aspect-ratio TRUE;display "USEDEF";valid_file "T";width
3.6224in;height 0.2253in;depth 0in;original-width 3.6632in;original-height
0.2005in;cropleft "0";croptop "1";cropright "1";cropbottom "0";tempfilename
'N1IQEJ03.wmf';tempfile-properties "XPR";}}This completes step 1).

For step 2), let's think about what the electric field will look like.\FRAME{%
dtbpF}{4.1018in}{1.5195in}{0pt}{}{}{Figure}{\special{language "Scientific
Word";type "GRAPHIC";maintain-aspect-ratio TRUE;display "USEDEF";valid_file
"T";width 4.1018in;height 1.5195in;depth 0pt;original-width
9.2085in;original-height 3.3944in;cropleft "0";croptop "1";cropright
"1";cropbottom "0";tempfilename 'LTUWDIA3.wmf';tempfile-properties "XPR";}}%
In the figure above I\ have blown up the view on three charge carriers and
drawn some field lines. Notice that in the $x-$direction the fields will
cancel.\FRAME{dtbpF}{4.0032in}{0.8198in}{0pt}{}{}{Figure}{\special{language
"Scientific Word";type "GRAPHIC";maintain-aspect-ratio TRUE;display
"USEDEF";valid_file "T";width 4.0032in;height 0.8198in;depth
0pt;original-width 8.6473in;original-height 1.7487in;cropleft "0";croptop
"1";cropright "1";cropbottom "0";tempfilename
'LTUWDIA4.wmf';tempfile-properties "XPR";}}The $y-$components add\FRAME{dtbpF%
}{4.0923in}{1.267in}{0pt}{}{}{Figure}{\special{language "Scientific
Word";type "GRAPHIC";maintain-aspect-ratio TRUE;display "USEDEF";valid_file
"T";width 4.0923in;height 1.267in;depth 0pt;original-width
8.6559in;original-height 2.661in;cropleft "0";croptop "1";cropright
"1";cropbottom "0";tempfilename 'LTUWDIA5.wmf';tempfile-properties "XPR";}}%
So we have only a field in the $y$ direction \FRAME{dtbpF}{4.2047in}{1.5126in%
}{0pt}{}{}{Figure}{\special{language "Scientific Word";type
"GRAPHIC";maintain-aspect-ratio TRUE;display "USEDEF";valid_file "T";width
4.2047in;height 1.5126in;depth 0pt;original-width 8.6507in;original-height
3.0934in;cropleft "0";croptop "1";cropright "1";cropbottom "0";tempfilename
'LTUWDIA6.wmf';tempfile-properties "XPR";}}Now if we had edges of our sheet
of charge, not all the $x-$components would cancel and the problem would be
harder, but we won't do that problem now. Also note that there is a field in
the $-y$-direction, I\ only drew some of the field lines in the figures.%
\FRAME{dtbpF}{3.2029in}{1.9806in}{0pt}{}{}{Figure}{\special{language
"Scientific Word";type "GRAPHIC";maintain-aspect-ratio TRUE;display
"USEDEF";valid_file "T";width 3.2029in;height 1.9806in;depth
0pt;original-width 3.2357in;original-height 1.9895in;cropleft "0";croptop
"1";cropright "1";cropbottom "0";tempfilename
'N1IR4Q05.wmf';tempfile-properties "XPR";}}This is step 2).

Now we need to choose an imaginary surface over which to integrate $\doint 
\vec{E}\cdot d\vec{A}.$ We want $\vec{E}\cdot d\vec{A}=EdA$ or $\vec{E}\cdot
d\vec{A}=0$ over all parts of the surface. I suggest a cylinder. \FRAME{dhF}{%
2.8089in}{1.7582in}{0pt}{}{}{Figure}{\special{language "Scientific
Word";type "GRAPHIC";maintain-aspect-ratio TRUE;display "USEDEF";valid_file
"T";width 2.8089in;height 1.7582in;depth 0pt;original-width
2.7665in;original-height 1.7201in;cropleft "0";croptop "1";cropright
"1";cropbottom "0";tempfilename 'N1IR4Q06.wmf';tempfile-properties "XPR";}}%
Note that along the top of the cylinder, $E\parallel A$ so $\vec{E}\cdot d%
\vec{A}=EdA\cos \theta =EdA.$ Along the side of the cylinder $E\perp A$ so $%
\vec{E}\cdot d\vec{A}=EdA\cos \theta =0.$ We have a surface that works! This
completes step 3).

Now we need to solve the integral. The flux is just 
\begin{eqnarray*}
\Phi &=&\doint \vec{E}\cdot d\vec{A} \\
\Phi &=&\doint_{side}\vec{E}\cdot d\vec{A}+\doint_{ends}\vec{E}\cdot d\vec{A}
\\
&=&0+\doint_{ends}EdA=2EA
\end{eqnarray*}%
where the factor of $2$ comes because we have two caps and field in the $+y$
and $-y$ directions and where $A$ is the area of one end cap. If we know
that the sheet of charge has a surface charge density of $\eta ,$ then we
can write the charge enclosed by the cylinder as%
\begin{equation*}
Q_{inside}=\eta A
\end{equation*}%
so%
\begin{equation*}
\Phi _{E}=\frac{\eta A}{\varepsilon _{o}}
\end{equation*}%
by Gauss' law. Equating the two expressions for the flux gives%
\begin{equation*}
2EA=\frac{\eta A}{\varepsilon _{o}}
\end{equation*}%
or%
\begin{equation}
E=\frac{\eta }{2\varepsilon _{o}}
\end{equation}%
which is what we found before for an infinite sheet of charge, but this way
was \emph{much} easier. If we can find a suitable surface, Gauss' law is
very powerful!

\subsection{Gauss's law strategy}

In each of our problems today, we found the electric field without a nasty
integration. Usually we want the electric field at a specific point. To make
Gauss' law work we need to do the following for each problem:

\begin{enumerate}
\item Draw the charge distribution

\item Draw the field using symmetry

\item Invent a Gaussian surface that takes advantage of the field symmetry
and that includes our point where we want the field. We will want $%
\overrightarrow{\mathbf{E}}\cdot d\overrightarrow{\mathbf{A}}=EdA$ or $%
\overrightarrow{\mathbf{E}}\cdot d\overrightarrow{\mathbf{A}}=0$ for each
part of the surface we invent.

\item Find the flux by finding the enclosed charge, $Q_{in}$

\item use $\doint \overrightarrow{\mathbf{E}}\cdot d\overrightarrow{\mathbf{A%
}}=\frac{Q_{in}}{\epsilon _{o}}$ integrating over our carefully invented
surface to find the field. If our surface that we imagined was good, then $%
\doint \overrightarrow{\mathbf{E}}\cdot d\overrightarrow{\mathbf{A}}$ will
be very easy.%
%TCIMACRO{%
%\TeXButton{Question 223.27.4}{\marginpar {
%\hspace{-0.5in}
%\begin{minipage}[t]{1in}
%\small{Question 223.27.4}
%\end{minipage}
%}}}%
%BeginExpansion
\marginpar {
\hspace{-0.5in}
\begin{minipage}[t]{1in}
\small{Question 223.27.4}
\end{minipage}
}%
%EndExpansion
\end{enumerate}

\section{Derivation of Gauss' Law}

A formal derivation of Gauss' Law is instructive, and it gives us the
opportunity to introduce the idea of solid angle.\FRAME{dtbpF}{2.5573in}{%
2.6437in}{0pt}{}{}{Figure}{\special{language "Scientific Word";type
"GRAPHIC";maintain-aspect-ratio TRUE;display "USEDEF";valid_file "T";width
2.5573in;height 2.6437in;depth 0pt;original-width 4.4529in;original-height
4.606in;cropleft "0";croptop "1";cropright "1";cropbottom "0";tempfilename
'LTUWDIA8.wmf';tempfile-properties "XPR";}}%
\begin{equation}
\Delta \Omega =\frac{\Delta A}{r^{2}}
\end{equation}%
This is like a two dimensional angle. And just like an angle, it really does
not have dimensions. Note that $\Delta A$ is a length squared, but so is $%
r^{2}.$ The (dimensionless) unit for solid angle is the \emph{steradian}. We
can see that for a sphere we would have a total solid angle of 
\begin{equation}
\Omega _{sphere}=\frac{4\pi r^{2}}{r^{2}}=4\pi \unit{sr}
\end{equation}%
Now let's see why this is useful. Consider a point charge in an arbitrary
closed surface. \FRAME{dtbpF}{1.6431in}{1.5843in}{0in}{}{}{Figure}{\special%
{language "Scientific Word";type "GRAPHIC";maintain-aspect-ratio
TRUE;display "USEDEF";valid_file "T";width 1.6431in;height 1.5843in;depth
0in;original-width 1.6068in;original-height 1.5489in;cropleft "0";croptop
"1";cropright "1";cropbottom "0";tempfilename
'M53MJE00.wmf';tempfile-properties "XPR";}}If we look at a particular
element of surface $\Delta A$ we can find the flux through that surface
element. We can use our idea of solid angle to do this

\FRAME{dhF}{3.6487in}{2.8184in}{0in}{}{}{Figure}{\special{language
"Scientific Word";type "GRAPHIC";maintain-aspect-ratio TRUE;display
"USEDEF";valid_file "T";width 3.6487in;height 2.8184in;depth
0in;original-width 3.6011in;original-height 2.7752in;cropleft "0";croptop
"1";cropright "1";cropbottom "0";tempfilename
'LZR7QF01.wmf';tempfile-properties "XPR";}}

\begin{equation*}
\Delta \Phi _{E}=\mathbf{\vec{E}\cdot \Delta \vec{A}}
\end{equation*}%
Since the field lines are symmetric about $q$ and the surface is arbitrary,
the element $\Delta \mathbf{\vec{A}}$ will be at some angle $\theta $ from
the field direction so 
\begin{equation*}
\mathbf{\vec{E}\cdot \Delta \vec{A}=}E\Delta A\cos \theta
\end{equation*}%
this is no surprise. But now notice that the projection of $\Delta A$ puts
it onto a spherical surface of just about the same distance from $q.$ The
projected area is 
\begin{equation*}
\Delta A_{P}=\Delta A\cos \theta
\end{equation*}%
At this point we should remember that we know the field due to a point charge%
\begin{equation*}
E=\frac{1}{4\pi \varepsilon _{o}}\frac{q}{r^{2}}
\end{equation*}%
so our flux through the area element is%
\begin{eqnarray*}
\Delta \Phi _{E} &=&\frac{1}{4\pi \varepsilon _{o}}\frac{q}{r^{2}}\Delta
A\cos \theta \\
&=&\frac{q}{4\pi \varepsilon _{o}}\frac{\Delta A\cos \theta }{r^{2}}
\end{eqnarray*}%
but 
\begin{equation*}
\frac{\Delta A\cos \theta }{r^{2}}=\Delta \Omega
\end{equation*}%
is the solid angle subtended by the projected area. Then 
\begin{equation*}
\Delta \Phi _{E}=\frac{q}{4\pi \varepsilon _{o}}\Delta \Omega
\end{equation*}%
The total flux though the oddly shaped closed surface is then%
\begin{equation*}
\Phi _{E}=\frac{q}{4\pi \varepsilon _{o}}\doint d\Omega
\end{equation*}%
where we integrate over the entire arbitrary surface, $S$. 
\begin{equation*}
\Phi _{E}=\frac{q}{4\pi \varepsilon _{o}}\doint_{S}d\Omega
\end{equation*}%
but by definition%
\begin{equation*}
\doint_{S}d\Omega =4\pi \unit{sr}
\end{equation*}%
so%
\begin{eqnarray*}
\Phi _{E} &=&\frac{q}{4\pi \varepsilon _{o}}\doint_{S}d\Omega \\
&=&\frac{q}{4\pi \varepsilon _{o}}4\pi \unit{sr} \\
&=&\frac{q}{\varepsilon _{o}}
\end{eqnarray*}%
which is just Gauss' law.

So far we have used mostly charged insulators to find fields. But we know we
will be interested in conductors and their fields in building electronics.
We will take up the study of charged conductors and their fields next.

%TCIMACRO{%
%\TeXButton{Basic Equations}{\hspace{-1.3in}{\LARGE Basic Equations\vspace{0.25in}}}}%
%BeginExpansion
\hspace{-1.3in}{\LARGE Basic Equations\vspace{0.25in}}%
%EndExpansion

Gauss' law%
\begin{equation*}
\Phi =\frac{Q_{inside}}{\epsilon _{o}}
\end{equation*}%
Gauss' law combined with our equation for flu%
\begin{equation*}
\Phi =\doint \overrightarrow{\mathbf{E}}\cdot d\overrightarrow{\mathbf{A}}=%
\frac{Q_{inside}}{\epsilon _{o}}
\end{equation*}

\chapter{Conductors in Equilibrium, Electric Potentials}

%TCIMACRO{%
%\TeXButton{Fundamental Concepts}{\hspace{-1.3in}{\LARGE Fundamental Concepts\vspace{0.25in}}}}%
%BeginExpansion
\hspace{-1.3in}{\LARGE Fundamental Concepts\vspace{0.25in}}%
%EndExpansion

\begin{itemize}
\item Conductors in Equilibrium

\item Electric Potential Energy
\end{itemize}

\section{Conductors in Equilibrium}

Conductors have some special properties because they have movable charge.
Here they are

\begin{enumerate}
\item Any excess static charge (charge added to an uncharged conductor) will
stay on the surface of the conductor.

\item The electric field is zero everywhere \emph{inside} a conductor.

\item The electric field just outside a charged conductor is perpendicular
to the conductor surface.

\item Charge tends to accumulate at sharp points where the radius of
curvature of the surface is smallest.
\end{enumerate}

It is our job to convince ourselves that these are true. Lets take these one
at a time.

\subsection{In Equilibrium, excess charge is on the Surface}

%TCIMACRO{%
%\TeXButton{Question 223.28.1}{\marginpar {
%\hspace{-0.5in}
%\begin{minipage}[t]{1in}
%\small{Question 223.28.1}
%\end{minipage}
%}}}%
%BeginExpansion
\marginpar {
\hspace{-0.5in}
\begin{minipage}[t]{1in}
\small{Question 223.28.1}
\end{minipage}
}%
%EndExpansion
Let's think about what we know about conductors. Most good conductors are
metals. The reason they are good conductors is that the outer electrons in
metals are in open valence bands where there are many energy states
available to the electrons. These electrons are free to travel around. This
means that if we place a charge near a metal object, the free charges will
experience an acceleration. Of course, the charge does not fly out of the
conductor. It will have to stop when it reaches the end of the metal object.
Suppose we go back to our experiment from the first lecture. We took a
charged rod, and placed it near an uncharged conductor.

\FRAME{dhF}{3.7109in}{0.4765in}{0pt}{}{}{Figure}{\special{language
"Scientific Word";type "GRAPHIC";maintain-aspect-ratio TRUE;display
"USEDEF";valid_file "T";width 3.7109in;height 0.4765in;depth
0pt;original-width 9.436in;original-height 1.1805in;cropleft "0";croptop
"1";cropright "1";cropbottom "0";tempfilename
'Charge/metal_rod_netagive_rod.wmf';tempfile-properties "XNPR";}}The free
electrons moved. We ended up with a bunch of electrons all on the right hand
side. They all repel each other. So at some point the force between a free
electron and the charged rod, and the force between a free electrons and the
rest of the free electrons will balance. At that point, there is zero net
force (think of Newton's second law). The free electrons stop moving. We
have a word from PH121 or Statics for when all the forces balance. We say
the charges are in \emph{equilibrium}.

Now suppose we have a conductor just on it's own and suppose we add charge
to it. Where would the extra charge go? \ We have considered this before. In
the picture below, I have a spherical conductor with two extra negative
charges shown. The pair of charges will repel each other. Now because of the 
$r^{2}$ in our electric force equation, the closer the extra charges are,
the stronger the repulsive force. The result is that they will try to go as
far from each other as possible. So the extra charge on a spherical
conductor will all end up on the surface. \FRAME{dhF}{0.921in}{0.8752in}{0pt%
}{}{}{Figure}{\special{language "Scientific Word";type
"GRAPHIC";maintain-aspect-ratio TRUE;display "USEDEF";valid_file "T";width
0.921in;height 0.8752in;depth 0pt;original-width 3.3468in;original-height
3.1808in;cropleft "0";croptop "1";cropright "1";cropbottom "0";tempfilename
'internal_separating_charges.wmf';tempfile-properties "XNPR";}}

\subsection{The Electric Field is Zero \emph{Inside} a Conductor}

%TCIMACRO{%
%\TeXButton{Question 223.28.2}{\marginpar {
%\hspace{-0.5in}
%\begin{minipage}[t]{1in}
%\small{Question 223.28.2}
%\end{minipage}
%}}}%
%BeginExpansion
\marginpar {
\hspace{-0.5in}
\begin{minipage}[t]{1in}
\small{Question 223.28.2}
\end{minipage}
}%
%EndExpansion
We can use Gauss' law to find the field in a conductor. We know that the
extra charge will all be on the surface if there is no electric current.

\FRAME{dhF}{1.4036in}{1.8965in}{0pt}{}{}{Figure}{\special{language
"Scientific Word";type "GRAPHIC";maintain-aspect-ratio TRUE;display
"USEDEF";valid_file "T";width 1.4036in;height 1.8965in;depth
0pt;original-width 3.3927in;original-height 4.5939in;cropleft "0";croptop
"1";cropright "1";cropbottom "0";tempfilename
'LZUSY705.wmf';tempfile-properties "XPR";}}We can then draw a Gaussian
surface, to match the symmetry of the conductor. What is the charge inside
the Gaussian surface? It is net zero, since the reaming charge is all bound
up in atoms and balances out. Since there is no net charge, there is no net
flux. If there is no flux, there is no net field inside a conductor that is
in static equilibrium.

Note that if we connected this conductor to both ends of a battery, we would
have a field in the conductor generated by the battery and the charge flow
it creates, so we must remember that static equilibrium is a special case.

If we don't connect the conductor to the ground or a battery, we can say: 
\emph{The net electric field is zero everywhere inside the conducting
material.}

Consider if this were not true! If there were an electric field inside the
conductor, the free charge there would accelerate and there would be a flow
of charge. If there were a movement of charge, the conductor would not be in
equilibrium. Suppose we place a brick of conductor in a field. We expect
that the charges will be accelerated. Negative charges will move opposite
the field direction. We end up with the situation shown in the next figure.%
\FRAME{dhF}{2.047in}{1.9977in}{0pt}{}{}{Figure}{\special{language
"Scientific Word";type "GRAPHIC";maintain-aspect-ratio TRUE;display
"USEDEF";valid_file "T";width 2.047in;height 1.9977in;depth
0pt;original-width 3.0441in;original-height 2.9706in;cropleft "0";croptop
"1";cropright "1";cropbottom "0";tempfilename
'LZUTER08.wmf';tempfile-properties "XPR";}}Since the negative charges moved,
the other side has a net positive charge. This separation of the charges
creates a new field in the opposite direction of the original field. In
equilibrium, just enough charge is moved to create a field that cancels the
original field.

\subsection{Return to charge being on the surface}

%TCIMACRO{%
%\TeXButton{Question 223.28.3}{\marginpar {
%\hspace{-0.5in}
%\begin{minipage}[t]{1in}
%\small{Question 223.28.3}
%\end{minipage}
%}}}%
%BeginExpansion
\marginpar {
\hspace{-0.5in}
\begin{minipage}[t]{1in}
\small{Question 223.28.3}
\end{minipage}
}%
%EndExpansion
Suppose we have a conductor in equilibrium. We can now ask, what does it
mean that the charge is \textquotedblleft on the surface?\textquotedblright\
Is there a small distance within the metal where we would find extra charge?
or is it all right at the edge of the metal?

Let's look at this again now that we know Gauss' law. Let's envision a
conducting object with a matching Gaussian surface.

\FRAME{dtbpF}{1.6944in}{1.9491in}{0in}{}{}{Figure}{\special{language
"Scientific Word";type "GRAPHIC";maintain-aspect-ratio TRUE;display
"USEDEF";valid_file "T";width 1.6944in;height 1.9491in;depth
0in;original-width 1.6718in;original-height 1.9273in;cropleft "0";croptop
"1";cropright "1";cropbottom "0";tempfilename
'S35K4C00.wmf';tempfile-properties "XPR";}}We know the field inside the
conductor is zero. So no field lines can leave or enter the Gaussian
surface. So no charge can be inside or we would have a net flux, and,
therefor, a field. We can move the Gaussian surface from the center of the
conductor and grow it until it is just barely smaller than the surface of
the conductor, and there still must be no field, so no charge inside. \FRAME{%
dtbpF}{1.7921in}{2.1654in}{0in}{}{}{Figure}{\special{language "Scientific
Word";type "GRAPHIC";maintain-aspect-ratio TRUE;display "USEDEF";valid_file
"T";width 1.7921in;height 2.1654in;depth 0in;original-width
1.7703in;original-height 2.1444in;cropleft "0";croptop "1";cropright
"1";cropbottom "0";tempfilename 'S35K4V01.wmf';tempfile-properties "XPR";}}%
We can make this Gaussian surface as close to the actual surface as we like,
and still there must be no field inside. Thus all the excess charge must be
on the surface. It is not distributed at any depth in the material.\footnote{%
For our chemists, our quantum picture will modify this reasoning a little,
since we will view electrons as waves that extend out into space a bit.}

\subsection{Field lines leave normal to the surface}

%TCIMACRO{%
%\TeXButton{Question 223.28.4}{\marginpar {
%\hspace{-0.5in}
%\begin{minipage}[t]{1in}
%\small{Question 223.28.4}
%\end{minipage}
%}}}%
%BeginExpansion
\marginpar {
\hspace{-0.5in}
\begin{minipage}[t]{1in}
\small{Question 223.28.4}
\end{minipage}
}%
%EndExpansion
In the following picture, we can see that the field lines seem to leave the
surface of these charged conductors at right angles (remember that sometimes
we call this \emph{normal} to the surface). \FRAME{dhF}{1.9268in}{1.6924in}{%
0pt}{}{}{Figure}{\special{language "Scientific Word";type
"GRAPHIC";maintain-aspect-ratio TRUE;display "USEDEF";valid_file "T";width
1.9268in;height 1.6924in;depth 0pt;original-width 2.7527in;original-height
2.4146in;cropleft "0";croptop "1";cropright "1";cropbottom "0";tempfilename
'LTUWDJAA.wmf';tempfile-properties "XPR";}}We have charges all along the
surface, and neighboring charges cancel all but the normal components of the
field, so the field lines go straight out. Notice that farther from the
conductor the field lines may bend, but they start out leaving the surface
perpendicular to the surface. Let's draw a conducting object. \FRAME{dtbpF}{%
1.4224in}{2.1645in}{0pt}{}{}{Figure}{\special{language "Scientific
Word";type "GRAPHIC";maintain-aspect-ratio TRUE;display "USEDEF";valid_file
"T";width 1.4224in;height 2.1645in;depth 0pt;original-width
1.7616in;original-height 2.6965in;cropleft "0";croptop "1";cropright
"1";cropbottom "0";tempfilename 'S35K8T02.wmf';tempfile-properties "XPR";}}%
Consider what would happen if it were not true that the field lines left
perpendicular to a conductor surface when the conductor was in equilibrium. 
\FRAME{dtbpF}{2.0721in}{1.916in}{0in}{}{}{Figure}{\special{language
"Scientific Word";type "GRAPHIC";maintain-aspect-ratio TRUE;display
"USEDEF";valid_file "T";width 2.0721in;height 1.916in;depth
0in;original-width 2.0511in;original-height 1.8941in;cropleft "0";croptop
"1";cropright "1";cropbottom "0";tempfilename
'S35K9Q03.wmf';tempfile-properties "XPR";}}There would be a horizontal
component of the field in such a case. The component of the field along the
surface would cause the charge to move. In the figure there would be a net
force to the left. This force would rearrange the charge until there was no
force. But since $F_{x}=qE_{x}$, then when $F_{x}$ is zero, so is $E_{x}.$
Suppose we place a conductor in an external field. We would see that the
charges within the conductor will rearrange themselves until the field lines
will leave perpendicular to the surface of the conductors.

\FRAME{dhF}{3.4861in}{2.3462in}{0pt}{}{}{Figure}{\special{language
"Scientific Word";type "GRAPHIC";maintain-aspect-ratio TRUE;display
"USEDEF";valid_file "T";width 3.4861in;height 2.3462in;depth
0pt;original-width 6.3252in;original-height 4.2462in;cropleft "0";croptop
"1";cropright "1";cropbottom "0";tempfilename
'LZUSY702.wmf';tempfile-properties "XPR";}}

Notice the square box in the last figure. There is an opening inside the
conductor, but there is no net field inside. The conductor charges rearrange
themselves so that the external field is canceled out. This is part of what
is known as a \emph{Faraday cage} which allows us to cancel out an external
electric field. This is used to protect electronic devices that must operate
in strong electric fields. To complete the effect, we will also need to show
that magnetic fields are canceled by such a conducting box.

We should also consider what happens when we place a charge in a conductive
container. Does this charge get screened off? That is, would the conductive
container prevent us from telling if there was a charge inside?

\FRAME{dhF}{2.4569in}{1.7322in}{0pt}{}{}{Figure}{\special{language
"Scientific Word";type "GRAPHIC";maintain-aspect-ratio TRUE;display
"USEDEF";valid_file "T";width 2.4569in;height 1.7322in;depth
0pt;original-width 4.0456in;original-height 2.8444in;cropleft "0";croptop
"1";cropright "1";cropbottom "0";tempfilename
'LZUSY704.wmf';tempfile-properties "XPR";}}In this case, the answer is no.
The charges in the conductor will move because of the charge contained
inside the conducting container. The negative charge will move as shown, and
it will move to the outside of the container surface. This leaves positive
charges behind on the inner surface. We know that there will be no field
inside the conductor material, itself. But think of placing a Gaussian
surface around all of the container and charge. There will be a net charge
inside the Gaussian surface, so there will be a field. The inner surface
charge does cancel the charge from the charged sphere. But the negative
charge on the conductor surface creates a new field.

\subsection{Charge tends to accumulate at sharp points}

Let's go back to our charged conductor. Notice that the field lines bunch up
at the corners! Where the field lines are closer together, there must be
more charge and the field strength must be higher.\FRAME{dhF}{2.041in}{%
3.0831in}{0pt}{}{}{Figure}{\special{language "Scientific Word";type
"GRAPHIC";maintain-aspect-ratio TRUE;display "USEDEF";valid_file "T";width
2.041in;height 3.0831in;depth 0pt;original-width 2.002in;original-height
3.039in;cropleft "0";croptop "1";cropright "1";cropbottom "0";tempfilename
'LZUSY707.wmf';tempfile-properties "XPR";}}Now that we have an idea of how
charge and conductors act in equilibrium, we would like to motivate charge
to move. To see how this happens, let's review energy.

\section{Electrical Work and Energy}

We remember studying energy back in PH121 or Statics and Dynamics.%
%TCIMACRO{%
%\TeXButton{Question 223.28.5}{\marginpar {
%\hspace{-0.5in}
%\begin{minipage}[t]{1in}
%\small{Question 223.28.5}
%\end{minipage}
%}} }%
%BeginExpansion
\marginpar {
\hspace{-0.5in}
\begin{minipage}[t]{1in}
\small{Question 223.28.5}
\end{minipage}
}
%EndExpansion
Remember the Work-Energy theorem?%
%TCIMACRO{%
%\TeXButton{Far Board}{\marginpar {
%\hspace{-0.5in}
%\begin{minipage}[t]{1in}
%\small{Put this on the far board}
%\end{minipage}
%}}}%
%BeginExpansion
\marginpar {
\hspace{-0.5in}
\begin{minipage}[t]{1in}
\small{Put this on the far board}
\end{minipage}
}%
%EndExpansion
\begin{equation}
W_{nc}=\Delta K+\Delta U
\end{equation}

We started with gravitational potential energy, and, as we found
conservative forces, we defined new potential energies to describe the work
done by those forces. For example, we added spring potential energy%
\begin{equation}
W_{nc}=\Delta K+\Delta U_{g}+\Delta U_{s}
\end{equation}%
I bet you can guess what we will do with our electrical or Coulomb force!%
\begin{equation}
W_{nc}=\Delta K+\Delta U_{g}+\Delta U_{s}+\Delta U_{C}
\end{equation}%
When we do this, we mean that the work done by the Coulomb force $\left(
W_{C}\right) $ is the negative of the electrical potential energy change%
\begin{equation}
W_{C}=-\Delta U_{C}
\end{equation}%
and we are saying that the Coulomb force is conservative. But is the Coulomb
force conservative? Remember that the equation for the force due to gravity
and the equation for the Coulomb force are very alike. So we might guess
that the Coulomb force is conservative like gravity--and we would be right!

\subsection{Energy of a Charge in a uniform field}

%TCIMACRO{%
%\TeXButton{Question 223.28.6}{\marginpar {
%\hspace{-0.5in}
%\begin{minipage}[t]{1in}
%\small{Question 223.28.6}
%\end{minipage}
%}}}%
%BeginExpansion
\marginpar {
\hspace{-0.5in}
\begin{minipage}[t]{1in}
\small{Question 223.28.6}
\end{minipage}
}%
%EndExpansion
%TCIMACRO{%
%\TeXButton{Question 223.28.7}{\marginpar {
%\hspace{-0.5in}
%\begin{minipage}[t]{1in}
%\small{Question 223.28.7}
%\end{minipage}
%}}}%
%BeginExpansion
\marginpar {
\hspace{-0.5in}
\begin{minipage}[t]{1in}
\small{Question 223.28.7}
\end{minipage}
}%
%EndExpansion
%TCIMACRO{%
%\TeXButton{Question 223.28.8}{\marginpar {
%\hspace{-0.5in}
%\begin{minipage}[t]{1in}
%\small{Question 223.28.8}
%\end{minipage}
%}}}%
%BeginExpansion
\marginpar {
\hspace{-0.5in}
\begin{minipage}[t]{1in}
\small{Question 223.28.8}
\end{minipage}
}%
%EndExpansion
Let's use our Coulomb force to calculate work. I\ would like a simple
example, so let's assume we have a uniform electric field. We know that we
can almost really make a uniform electric field by building a large
capacitor.\FRAME{dhF}{3.6763in}{1.4512in}{0pt}{}{}{Figure}{\special{language
"Scientific Word";type "GRAPHIC";maintain-aspect-ratio TRUE;display
"USEDEF";valid_file "T";width 3.6763in;height 1.4512in;depth
0pt;original-width 3.6288in;original-height 1.4157in;cropleft "0";croptop
"1";cropright "1";cropbottom "0";tempfilename
'LTUWCG0D.wmf';tempfile-properties "XPR";}}

We draw some field lines (from the + charges to the - charges). The field
lines will be mostly straight lines in between the plates. Of course,
outside the plates, they will not be at all straight, but we will ignore
this because we want to calculate work just in the uniform part of the field.

I want to place a charge, $q,$ in this uniform field. The charge will
accelerate. Work will be done. I want to find out how much work is done on
the charge.

From our PH121 or Dynamics experience, we know that 
\begin{eqnarray}
W &=&\int \overrightarrow{\mathbf{F}}\cdot \mathbf{d}\overrightarrow{\mathbf{%
x}} \\
&=&F\Delta x\cos \theta  \notag
\end{eqnarray}%
for constant forces. Because we have a constant field, we will have a
constant force.

I will choose the $x$ direction to be vertical and $x=0$ to be near the
positive plate. Then we can write the force due to the electric field as%
\begin{eqnarray*}
W &=&F\Delta x\cos \theta \\
&=&\left( q_{m}E\right) \Delta x\cos \left( 0\unit{%
%TCIMACRO{\U{b0}}%
%BeginExpansion
{{}^\circ}%
%EndExpansion
}\right) \\
&=&q_{m}E\Delta x
\end{eqnarray*}

If there are no non-conservative forces, and we ignore gravity, then we can
say%
%TCIMACRO{%
%\TeXButton{Far Board}{\marginpar {
%\hspace{-0.5in}
%\begin{minipage}[t]{1in}
%\small{Put this on the far board}
%\end{minipage}
%}}}%
%BeginExpansion
\marginpar {
\hspace{-0.5in}
\begin{minipage}[t]{1in}
\small{Put this on the far board}
\end{minipage}
}%
%EndExpansion
\begin{eqnarray*}
W_{nc} &=&\Delta K+\Delta U_{g}+\Delta U_{s}+\Delta U_{C} \\
0 &=&\Delta K+0+\Delta PE_{C} \\
0 &=&\Delta K+0-q_{m}E\Delta x
\end{eqnarray*}%
so 
\begin{equation}
\Delta K=q_{m}E\Delta x
\end{equation}

This is very interesting! This means that for this simple geometry I\ could
ask you questions like, \textquotedblleft after the charge travels $\Delta
x, $ how fast is it going?\textquotedblright

\subsection{Electric and Gravitational potential energy compared}

We have found that the potential energy for the Coulomb force is given by%
\begin{equation*}
\Delta U_{C}=-q_{m}E\Delta x
\end{equation*}%
for a \emph{uniform} electric field (it will change for non-uniform fields).
Let's compare this to the gravitational potential energy%
\begin{equation*}
\Delta U_{g}=-mgh
\end{equation*}%
Let's set up a situation where the electric field and gravitational field
are almost uniform and we have a positively charged particle with charge $q$
and mass $m.$ The height, $h,$ we will call $d$ to match our gravitational
and electrical cases.\FRAME{dhF}{3.4281in}{2.0487in}{0pt}{}{}{Figure}{%
\special{language "Scientific Word";type "GRAPHIC";maintain-aspect-ratio
TRUE;display "USEDEF";valid_file "T";width 3.4281in;height 2.0487in;depth
0pt;original-width 6.0338in;original-height 3.595in;cropleft "0";croptop
"1";cropright "1";cropbottom "0";tempfilename
'LTUWCG0E.wmf';tempfile-properties "XPR";}}

The gravitational potential difference is 
\begin{equation}
\Delta U_{g}=-mgd
\end{equation}%
and the electrical potential difference is%
\begin{equation}
\Delta U_{C}=-q_{m}Ed
\end{equation}%
These equations look a lot alike. We should expect that if we push the
charge $q_{m}$ \textquotedblleft up,\textquotedblright\ we will increase
both potential energies. We will have to do positive work to do that $\left(
W=-\Delta U\right) .$ This is just like doing work in a gravitational field,
so we are familiar with this behavior.

There is a difference, however. We have assumed that our charge $q_{m}$ was
positive. Suppose it is negative? There is only one kind of mass, but we
have two kinds of charge. We will have to get used to negative charges
\textquotedblleft falling up\textquotedblright\ to make the analogy continue.

This analogy helps us to understand how the electric potential energy will
act, and we will continue to use it. There is a difficulty, however, in that
most engineering classes only study gravitation in nearly uniform
gravitational fields. But if we look at large objects (like whole planets)
that are separated from other objects by some distance, then we have very
non-uniform gravitational fields. Unless you are an aerospace engineer,
these cases are less common. So to help us understand electric potential
energy, we will study gravitational potential energy of large things first,
then study the energy associated with individual charges and their very
non-uniform fields. We will take this on next time.

%TCIMACRO{%
%\TeXButton{Basic Equations}{\hspace{-1.3in}{\LARGE Basic Equations\vspace{0.25in}}}}%
%BeginExpansion
\hspace{-1.3in}{\LARGE Basic Equations\vspace{0.25in}}%
%EndExpansion

\chapter{Electric potential Energy}

%TCIMACRO{%
%\TeXButton{Fundamental Concepts}{\hspace{-1.3in}{\LARGE Fundamental Concepts\vspace{0.25in}}}}%
%BeginExpansion
\hspace{-1.3in}{\LARGE Fundamental Concepts\vspace{0.25in}}%
%EndExpansion

\begin{itemize}
\item Gravitational potential energy of point masses and binding energy

\item Electrical potential energy of point charges

\item Electrical potential energy of dipoles
\end{itemize}

\section{Point charge potential energy}

As we said last lecture, we want to use gravitation as an analogy for the
electric potential energy. Gravitation is more intuitive. But chances are
gravitation of whole planets was not stressed in Dynamics (If you took PH121
you should be fine, and this will be a review). So let's take a few moments
out of a PE101 class (introductory planetary engineering) and study
non-uniform gravitational fields.

\subsection{Gravitational analog}

%TCIMACRO{%
%\TeXButton{Question 223.29.1}{\marginpar {
%\hspace{-0.5in}
%\begin{minipage}[t]{1in}
%\small{Question 223.29.1}
%\end{minipage}
%}}}%
%BeginExpansion
\marginpar {
\hspace{-0.5in}
\begin{minipage}[t]{1in}
\small{Question 223.29.1}
\end{minipage}
}%
%EndExpansion
%TCIMACRO{%
%\TeXButton{Question 223.29.2}{\marginpar {
%\hspace{-0.5in}
%\begin{minipage}[t]{1in}
%\small{Question 223.29.2}
%\end{minipage}
%}}}%
%BeginExpansion
\marginpar {
\hspace{-0.5in}
\begin{minipage}[t]{1in}
\small{Question 223.29.2}
\end{minipage}
}%
%EndExpansion
Long, long ago you studied the potential energy of objects in what we can
now call the Earth's gravitational field.

The presentation of the idea of potential energy likely started with 
\begin{equation*}
U_{g}=mgy
\end{equation*}%
where $m$ is the mass of the object, $g$ is the acceleration due to gravity,
and $y$ is how high the object is compared to a $y=0$ point. If you recall,
we got to pick that $y=0$ point. It could be any height.

This all works fairly well so long as we take fairly small objects near the
much larger Earth. But hopefully you also considered objects farther away
from the Earth's surface, or larger objects like the moon. For these
objects, $mgy$ is not enough to describe the potential energy. The reason is
that if we are far away from the center of the Earth we will notice that the
Earth's gravitational field\footnote{%
Of course, the gravitational field is really the warping of space-time. But
that is a subject for another physics class.} is not uniform. It curves and
diminishes with distance. So, if an object is large, it will feel the change
in the gravitational field over its (the object's) large volume.

We have the tools to find the potential energy of this situation. We know
that a change in potential energy is just an amount of work%
\begin{equation*}
\Delta U_{g}=-W_{g}=-\int \overrightarrow{\mathbf{F}}_{g}\cdot d%
\overrightarrow{\mathbf{r}}
\end{equation*}%
The magnitude of the gravitational force is 
\begin{equation*}
F_{g}=G\frac{M_{E}m_{m}}{r_{Em}^{2}}
\end{equation*}%
where $M_{E}$ is the mass of the Earth, $m_{m}$ is the mass of the mover
object, and $r_{Em}$ is the distance between the two. The constant, $G,$ is
the gravitational constant.\FRAME{dtbpF}{3.8623in}{1.1986in}{0pt}{}{}{Figure%
}{\special{language "Scientific Word";type "GRAPHIC";maintain-aspect-ratio
TRUE;display "USEDEF";valid_file "T";width 3.8623in;height 1.1986in;depth
0pt;original-width 3.813in;original-height 1.1649in;cropleft "0";croptop
"1";cropright "1";cropbottom "0";tempfilename
'MIVWPJ00.wmf';tempfile-properties "XPR";}}

The field is radial, so $\overrightarrow{\mathbf{F}}_{g}\cdot d%
\overrightarrow{\mathbf{r}}=-Fdr$ for the configuration we have shown, and
we can perform the integration. Say we move the object a distance $\Delta r$
away were%
\begin{equation*}
\Delta r=R_{2}-R_{1}
\end{equation*}%
and $\Delta r$ is large, comparable to the size of the Earth or larger. Then 
\begin{eqnarray*}
\Delta U_{g} &=&-\int_{R_{1}}^{R_{2}}\left( -G\frac{M_{E}m_{m}}{r^{2}}%
\right) dr \\
&=&GM_{E}m_{m}\int_{R_{1}}^{R_{2}}\frac{dr}{r^{2}}
\end{eqnarray*}%
where $R$ is the distance from the center of the Earth to the center of our
object.%
\begin{eqnarray*}
\Delta U_{g} &=&GM_{E}m_{m}\int_{R_{1}}^{R_{2}}\frac{dr}{r^{2}} \\
&=&GM_{E}m_{m}\left[ -\frac{1}{r}\right\vert _{R_{1}}^{R_{2}} \\
&=&GM_{E}m_{m}\left[ -\frac{1}{R_{2}}-\left( -\frac{1}{R_{1}}\right) \right]
\\
&=&-GM_{E}m_{m}\left[ \frac{1}{R_{2}}-\frac{1}{R_{1}}\right] \\
&=&-G\frac{M_{E}m_{m}}{R_{2}}+G\frac{M_{E}m_{m}}{R_{1}}
\end{eqnarray*}

We recall that we need to set a zero point for the potential energy. Before,
when we used the approximation $m_{m}gy$ we could choose $y=0$ anywhere we
wanted. But now we see an obvious choice for the zero point of the potential
energy. If we let $R_{2}\rightarrow \infty $ and then the first term in our
expression will be zero. Likewise, of we let $R_{1}\rightarrow \infty $ the
second term will be zero. It looks like as we get infinitely far away from
the Earth, the potential energy naturally goes to zero! Mathematically this
makes sense. But we will have to interpret what this choice of zero point
means.

But first, let's see how much work it would take to move the moon out of
obit and move it farther away. Say, from $R_{1,}$ the present orbit radius,
to $R_{2}=2R_{1},$ or twice the original orbit distance. Then%
\begin{eqnarray*}
\Delta U_{g} &=&U_{2}-U_{1}=-G\frac{M_{E}m_{m}}{2R_{1}}+G\frac{M_{E}m_{m}}{%
R_{1}} \\
&=&G\frac{M_{E}m_{m}}{R_{1}}\left( -\frac{1}{2}+1\right) \\
&=&\left( \frac{1}{2}\right) G\frac{M_{E}m_{m}}{R_{1}}
\end{eqnarray*}%
The change is positive. We gained potential energy as we went farther from
the Earth's surface. That makes sense! That is analogous to increasing $y$
in $mgy.$ The potential energy also gets larger if the mass of our object
(like the moon or a satellite) gets larger. Again that makes sense because
in our more familiar approximation the potential energy increases with mass.
So this new form for our equation for potential energy seems to work.

But what does it mean that the potential energy is zero infinitely far away?
Recall that a change in potential energy is an amount of work%
\begin{equation*}
W=-\Delta U
\end{equation*}%
Usually we will consider the potential energy to be the amount of work it
takes to bring the test mass $m_{m}$ from infinitely far away (our zero
point!) to the location where we want it. It is how much energy is stored by
having the object in that position. Like how much energy is stored by
putting a mass high on a shelf. For example we could bring the moon in from
infinitely far away. Then 
\begin{eqnarray*}
\Delta U_{g} &=&U_{2}-U_{1}=-G\frac{M_{E}m_{m}}{R_{2}}+G\frac{M_{E}m_{m}}{%
\infty } \\
U_{2} &=&-G\frac{M_{E}m_{m}}{R_{2}}
\end{eqnarray*}%
This is how much potential energy the moon has as it orbits the Earth
because it is high, above the Earth. But notice, this is a negative number!
What can it mean to have a negative potential energy?%
%TCIMACRO{%
%\TeXButton{Question 223.29.3}{\marginpar {
%\hspace{-0.5in}
%\begin{minipage}[t]{1in}
%\small{Question 223.29.3}
%\end{minipage}
%}}}%
%BeginExpansion
\marginpar {
\hspace{-0.5in}
\begin{minipage}[t]{1in}
\small{Question 223.29.3}
\end{minipage}
}%
%EndExpansion

We use this convention to indicate that the test mass, $m_{m}$ is bound to
the Earth. It would take an input of energy to get the moon free from the
gravitational pull of the Earth. Here is the Moon potential energy plotted
as a function of distance. \FRAME{dtbpFX}{4.4996in}{1.9865in}{0pt}{}{}{Plot}{%
\special{language "Scientific Word";type "MAPLEPLOT";width 4.4996in;height
1.9865in;depth 0pt;display "USEDEF";plot_snapshots TRUE;mustRecompute
FALSE;lastEngine "MuPAD";xmin "6.369637E6";xmax "1.000136E8";xviewmin
"6.369637E6";xviewmax "1.000136E8";yviewmin "-4.608729E30";yviewmax
"0";viewset"XY";rangeset"X";plottype 4;labeloverrides 3;x-label "r
(m)";y-label "U (J)";axesFont "Times New
Roman,12,0000000000,useDefault,normal";numpoints 100;plotstyle
"patch";axesstyle "normal";axestips FALSE;xis \TEXUX{r};var1name
\TEXUX{$r$};function \TEXUX{$-\left( 6.67\times 10^{-11}\right) \frac{\left(
5.98\times 10^{24}\right) \left( 7.36\times 10^{22}\right) }{r}$};linecolor
"red";linestyle 1;pointstyle "point";linethickness 3;lineAttributes
"Solid";var1range "6.369637E6,1.000136E8";num-x-gridlines 100;curveColor
"[flat::RGB:0x00ff0000]";curveStyle "Line";VCamFile
'LTUWDK1C.xvz';valid_file "T";tempfilename
'LTUZ4U07.wmf';tempfile-properties "XPR";}}We can see that you have to go an
infinite distance to overcome the Earth's gravity completely. That makes
sense from our force equation. The force only goes to zero infinitely far
away. When we finally get infinitely far away, there will be no potential
energy due to the gravitational force because the gravitational force will
be zero.

Of course, there are more than just two objects (Earth and Moon) in the
universe, so as we get farther away from the Earth, the gravitational pull
of, say, a galaxy, might dominate. So we might not notice the weak pull of
the Earth as we encounter other objects.

We should show that this form for the potential energy due to gravity
becomes the more familiar $mgh$ if our distances are small compared to the
Earth's radius.

Let our distance from the center of the Earth be $R_{2}=R_{E}+y$ where $%
R_{E} $ is the radius of the Earth and $y\ll R_{E}$. Then 
\begin{eqnarray*}
U &=&-G\frac{M_{E}m_{m}}{R_{2}} \\
&=&-G\frac{M_{E}m_{m}}{R_{E}+y}
\end{eqnarray*}%
We can rewrite this as 
\begin{eqnarray*}
U &=&-G\frac{M_{E}m_{m}}{R_{E}\left( 1+\frac{y}{R_{E}}\right) } \\
&=&-G\frac{M_{E}m_{m}}{R_{E}}\left( 1+\frac{y}{R_{E}}\right) ^{-1}
\end{eqnarray*}%
Since $y$ is small $y/R_{E}$ is very small and we can approximate the therm
in parenthesis using the binomial expansion%
\begin{equation*}
\left( 1\pm x\right) ^{n}\approx 1\mp nx\qquad \text{if }x\ll 1
\end{equation*}%
then we have%
\begin{equation*}
\left( 1+\frac{y}{R_{E}}\right) ^{-1}\approx 1-\left( -1\right) \frac{y}{%
R_{E}}\qquad \text{if }\frac{y}{R_{E}}\ll 1
\end{equation*}%
and our potential energy is%
\begin{equation*}
U=-G\frac{M_{E}m_{m}}{R_{E}}\left( 1+\frac{y}{R_{E}}\right)
\end{equation*}%
then 
\begin{eqnarray*}
U &=&-G\frac{M_{E}m_{m}}{R_{E}}+G\frac{M_{E}m_{m}y}{R_{E}^{2}} \\
&=&U_{o}+m_{m}\left( G\frac{M_{E}}{R_{E}^{2}}\right) y
\end{eqnarray*}%
If we realize that $U_{o}$ is the potential energy of the object at the
surface of the Earth, then the change in potential energy as we lift the
object from the surface to a height $y$ is 
\begin{eqnarray*}
\Delta U &=&\left( U_{o}+m_{m}\left( G\frac{M_{E}}{R_{E}^{2}}\right)
y-\left( U_{o}+m_{m}\left( G\frac{M_{E}}{R_{E}^{2}}\right) \left( 0\right)
\right) \right) \\
&=&m_{m}\left( G\frac{M_{E}}{R_{E}^{2}}\right) y
\end{eqnarray*}%
All that is left is to realize that 
\begin{equation*}
\left( G\frac{M_{E}}{R_{E}^{2}}\right)
\end{equation*}%
has units if acceleration. This is just $g$%
\begin{equation*}
g=\left( G\frac{M_{E}}{R_{E}^{2}}\right)
\end{equation*}%
so we have 
\begin{equation*}
\Delta U=m_{m}gy
\end{equation*}%
and there is no contradiction. But we should realize that this is an
approximation. The more accurate version of our potential energy is 
\begin{equation*}
U_{2}=-G\frac{M_{E}m_{m}}{R_{2}}
\end{equation*}%
Likewise we should expect that for charges 
\begin{equation*}
\Delta U_{C}=-q_{m}Ed
\end{equation*}%
is an approximation that is only good when the field, $E$, can be
approximated as a constant magnitude and direction and that the distribution
of charge, $q_{m},$ is not spatially too big. With this understanding, we
can understand electrical potential energy of point charges.

\subsection{Point charges potential}

Suppose we now take a positive charge and define it's position as $r=0$ and
place a negative mover charge near the positive charge. \FRAME{dhF}{4.4166in%
}{1.3742in}{0pt}{}{}{Figure}{\special{language "Scientific Word";type
"GRAPHIC";display "USEDEF";valid_file "T";width 4.4166in;height
1.3742in;depth 0pt;original-width 4.3656in;original-height 1.388in;cropleft
"0";croptop "1";cropright "1";cropbottom "0";tempfilename
'LZWOOE0E.wmf';tempfile-properties "XPR";}}The work it would take to move
the charge a distance $\Delta r=R_{2}-R_{1}$ would be%
\begin{equation*}
\Delta U_{e}=-W_{e}=-\int \overrightarrow{\mathbf{F}}_{e}\cdot d%
\overrightarrow{\mathbf{r}}
\end{equation*}%
The magnitude of the electrical force is 
\begin{equation*}
F_{e}=\frac{1}{4\pi \epsilon _{o}}\frac{Q_{E}q_{m}}{r^{2}}
\end{equation*}%
once again $\overrightarrow{\mathbf{F}}_{e}\cdot d\overrightarrow{\mathbf{r}}
$ $=-F_{e}dr$ and 
\begin{eqnarray*}
\Delta U_{e} &=&-\int_{R_{1}}^{R_{2}}\left( -\frac{1}{4\pi \epsilon _{o}}%
\frac{Q_{E}q_{m}}{r^{2}}\right) dr \\
&=&\frac{Q_{E}q_{m}}{4\pi \epsilon _{o}}\int_{R_{1}}^{R_{2}}\frac{dr}{r^{2}}
\end{eqnarray*}%
and we realize that this is exactly the same integral we faced in the
gravitational case. The answer must be 
\begin{equation*}
\Delta U_{e}=-\frac{1}{4\pi \epsilon _{o}}\frac{Q_{E}q_{m}}{R_{2}}+\frac{1}{%
4\pi \epsilon _{o}}\frac{Q_{E}q_{m}}{R_{1}}
\end{equation*}

The similarity is hardly a surprise since the force equation for the Coulomb
force is really just like the force equation for gravity.

It makes sense to choose the zero point of the electric potential energy the
same way we did for the gravitational potential energy since the equations
is the same. We will pick $U=0$ at $r=\infty .$ Then we expect that 
\begin{equation*}
U_{e}=-\frac{1}{4\pi \epsilon _{o}}\frac{Q_{E}q_{m}}{r}
\end{equation*}%
is the electrical potential energy stored by having the charges in this
configuration.

%TCIMACRO{%
%\TeXButton{Question 223.29.4}{\marginpar {
%\hspace{-0.5in}
%\begin{minipage}[t]{1in}
%\small{Question 223.29.4}
%\end{minipage}
%}}}%
%BeginExpansion
\marginpar {
\hspace{-0.5in}
\begin{minipage}[t]{1in}
\small{Question 223.29.4}
\end{minipage}
}%
%EndExpansion
Again the negative sign shows that the two opposite charges will be bound
together by the attractive force. Here is a graph of the electrical
potential energy of an electron and a proton pair, like a Hydrogen atom. 
\FRAME{dtbpFX}{4.4996in}{1.5904in}{0pt}{}{}{Plot}{\special{language
"Scientific Word";type "MAPLEPLOT";width 4.4996in;height 1.5904in;depth
0pt;display "USEDEF";plot_snapshots TRUE;mustRecompute FALSE;lastEngine
"MuPAD";xmin "-6.049533E-12";xmax "6.050715E-8";xviewmin
"-6.049533E-12";xviewmax "6.050715E-8";yviewmin "-1.015097E-25";yviewmax
"0";viewset"XY";rangeset"X";plottype 4;labeloverrides 3;x-label "r
(m)";y-label "U (J)";axesFont "Times New
Roman,12,0000000000,useDefault,normal";numpoints 100;plotstyle
"patch";axesstyle "normal";axestips FALSE;xis \TEXUX{r};var1name
\TEXUX{$r$};function \TEXUX{$-\left( \frac{1}{4\pi \left( 8.85\times
10^{-12}\right) }\right) \frac{\left( 1/602\times 10^{-19}\right)
^{2}}{r}$};linecolor "blue";linestyle 1;pointstyle "point";linethickness
3;lineAttributes "Solid";var1range
"-6.049533E-12,6.050715E-8";num-x-gridlines 100;curveColor
"[flat::RGB:0x000000ff]";curveStyle "Line";VCamFile
'MJCY5Z07.xvz';valid_file "T";tempfilename
'LZWORQ0F.wmf';tempfile-properties "XP";}}

Of course we remember that there is a large difference between electrical
and gravitational forces. If the two charges are the same sign, then they
will repel and the potential must be different for that situation. If we
redraw our diagram for this case, we realize that the sign of the force must
change.\FRAME{dhF}{4.4719in}{1.4088in}{0pt}{}{}{Figure}{\special{language
"Scientific Word";type "GRAPHIC";maintain-aspect-ratio TRUE;display
"USEDEF";valid_file "T";width 4.4719in;height 1.4088in;depth
0pt;original-width 4.4209in;original-height 1.3742in;cropleft "0";croptop
"1";cropright "1";cropbottom "0";tempfilename
'LZWOTQ0G.wmf';tempfile-properties "XPR";}}%
\begin{equation*}
\Delta U_{e}=-W_{e}=-\int_{R_{1}}^{R_{2}}\left( +\frac{1}{4\pi \epsilon _{o}}%
\frac{Q_{E}q_{m}}{r^{2}}\right) dr
\end{equation*}%
this will change all the signs in our solution%
\begin{equation*}
\Delta U_{e}=+\frac{1}{4\pi \epsilon _{o}}\frac{Q_{E}q_{m}}{R_{2}}-\frac{1}{%
4\pi \epsilon _{o}}\frac{Q_{E}q_{m}}{R_{1}}
\end{equation*}%
then%
\begin{equation*}
U_{e}=+\frac{1}{4\pi \epsilon _{o}}\frac{Q_{E}q_{o}}{r}
\end{equation*}%
\FRAME{dtbpFX}{4.4996in}{1.7988in}{0pt}{}{}{Plot}{\special{language
"Scientific Word";type "MAPLEPLOT";width 4.4996in;height 1.7988in;depth
0pt;display "USEDEF";plot_snapshots TRUE;mustRecompute FALSE;lastEngine
"MuPAD";xmin "-6.049533E-12";xmax "6.050715E-8";xviewmin
"-6.049533E-12";xviewmax "6.050715E-8";yviewmin "0";yviewmax
"1.015097E-25";viewset"XY";rangeset"X";plottype 4;labeloverrides 3;x-label
"r (m)";y-label "U (J)";axesFont "Times New
Roman,12,0000000000,useDefault,normal";numpoints 100;plotstyle
"patch";axesstyle "normal";axestips FALSE;xis \TEXUX{r};var1name
\TEXUX{$r$};function \TEXUX{$\left( \frac{1}{4\pi \left( 8.85\times
10^{-12}\right) }\right) \frac{\left( 1/602\times 10^{-19}\right)
^{2}}{r}$};linecolor "blue";linestyle 1;pointstyle "point";linethickness
3;lineAttributes "Solid";var1range
"-6.049533E-12,6.050715E-8";num-x-gridlines 100;curveColor
"[flat::RGB:0x000000ff]";curveStyle "Line";VCamFile
'LTUWDK1A.xvz';valid_file "T";tempfilename
'LTUWCG0K.wmf';tempfile-properties "XP";}}Now we can see that the potential
energy gets larger as the two like charges get nearer. It takes energy to
make them get closer. This is clearly not a bound situation.

\subsection{Three point charges.}

%TCIMACRO{%
%\TeXButton{Question 223.29.5}{\marginpar {
%\hspace{-0.5in}
%\begin{minipage}[t]{1in}
%\small{Question 223.29.5}
%\end{minipage}
%}}}%
%BeginExpansion
\marginpar {
\hspace{-0.5in}
\begin{minipage}[t]{1in}
\small{Question 223.29.5}
\end{minipage}
}%
%EndExpansion
Suppose we have three like charges. What will the potential energy of the
three-charge system be?

Let's consider the charges one at a time. If I move one charge, $q_{1},$
from infinitely far away. there is no environmental electric field, so there
is no force, since we need two charges for there to be a force. Then there
is no potential energy. This is like a rock floating in deep space far away
from anything else in the universe. It just sits there, there is no
potential for movement, so no potential energy. But when we bring in another
charge, $q_{2},$ then $q_{1}$ is an environmental charge making a field and $%
q_{2}$ is our mover charge. Then $q_{2}$ will take an amount of work equal
to 
\begin{equation*}
U_{12}=\frac{1}{4\pi \epsilon _{o}}\frac{q_{1}q_{2}}{r_{12}}
\end{equation*}%
to move in the charge because the two charges repeal each other. There is a
force, so now there is an amount of potential energy associated with the
work done to move the charges together.

Suppose we had chosen to bring in the other charge, $q_{3},$ instead. Charge 
$q_{1}$ forms an environmental field. It takes an amount of energy 
\begin{equation*}
U_{13}=\frac{1}{4\pi \epsilon _{o}}\frac{q_{1}q_{3}}{r_{13}}
\end{equation*}%
to bring in the third charge charge But if the second charge were already
there, the second charge also creates an environmental field, so it also
creates a force on the third charge. So it will take more work to bring in
the third charge.%
\begin{equation*}
U_{3}=\frac{1}{4\pi \epsilon _{o}}\frac{q_{1}q_{3}}{r_{13}}+\frac{1}{4\pi
\epsilon _{o}}\frac{q_{2}q_{3}}{r_{23}}
\end{equation*}%
So the total amount of work involved in bringing all three charges together%
\begin{equation*}
U=\frac{1}{4\pi \epsilon _{o}}\frac{q_{1}q_{2}}{r_{12}}+\frac{1}{4\pi
\epsilon _{o}}\frac{q_{1}q_{3}}{r_{13}}+\frac{1}{4\pi \epsilon _{o}}\frac{%
q_{2}q_{3}}{r_{23}}
\end{equation*}%
then the potential energy difference would be 
\begin{eqnarray*}
\Delta U &=&U_{f}-U_{i}=-W \\
&=&U_{f}-0 \\
&=&\frac{1}{4\pi \epsilon _{o}}\frac{q_{1}q_{2}}{r_{12}}+\frac{1}{4\pi
\epsilon _{o}}\frac{q_{1}q_{3}}{r_{13}}+\frac{1}{4\pi \epsilon _{o}}\frac{%
q_{2}q_{3}}{r_{23}}
\end{eqnarray*}%
which we can generalize as 
\begin{equation*}
U=\frac{1}{4\pi \epsilon _{o}}\sum_{i<j}\frac{q_{i}q_{j}}{r_{ij}}
\end{equation*}%
for any number of charges. We simply add up all the potential energies. This
is one reason to use electric potential energy in solving problems. The
electric potential energies just add, and they are not vectors, so the
addition is simple.

\section{Dipole potential energy}

\FRAME{dhF}{2.655in}{1.8005in}{0in}{}{}{Figure}{\special{language
"Scientific Word";type "GRAPHIC";maintain-aspect-ratio TRUE;display
"USEDEF";valid_file "T";width 2.655in;height 1.8005in;depth
0in;original-width 2.6135in;original-height 1.7634in;cropleft "0";croptop
"1";cropright "1";cropbottom "0";tempfilename
'LTUWCG0L.wmf';tempfile-properties "XP";}}

Let's try out our new idea of potential energy for point charges on a
dipole. We will try to keep this easy, so let's consider the dipole to be in
a constant, uniform electric field. We know there will be no net force. The
work done to move a charge we have stated to be 
\begin{equation*}
W=\int \overrightarrow{\mathbf{F}}_{e}\cdot d\overrightarrow{\mathbf{r}}
\end{equation*}%
but in this case, we know the net force on the dipole is zero.

However, we can also do some work in rotating something%
\begin{equation*}
W_{rot}=\int \tau _{e}d\theta
\end{equation*}%
we know from before that the magnitude of the torque is 
\begin{equation*}
\tau =pE\sin \theta
\end{equation*}%
so%
\begin{eqnarray*}
W_{rot} &=&\int_{\theta _{1}}^{\theta _{2}}pE\sin \theta d\theta \\
&=&pE\left( \cos \theta _{2}-\cos \theta _{1}\right)
\end{eqnarray*}%
this must give 
\begin{eqnarray*}
\Delta U &=&-W_{rot}=U_{f}-U_{i} \\
&=&-pE\left( \cos \theta _{2}-\cos \theta _{1}\right)
\end{eqnarray*}%
then we can write as 
\begin{equation*}
U=-pE\cos \theta
\end{equation*}%
This is the rotational potential energy for the dipole. We can write this as
an inner product 
\begin{equation*}
U=-\overrightarrow{\mathbf{p}}\cdot \overrightarrow{\mathbf{E}}
\end{equation*}%
What does this mean? It tells us that we have to do work to turn the dipole.

Let's go back to our example of a microwave oven. If the field is $E=200%
\unit{V}/\unit{m}$, then how much work does it take to turn the water
molecules?

Remember that the dipole moment for a water molecule is something like 
\begin{equation*}
p_{w}=6.2\times 10^{-30}\unit{C}\unit{m}
\end{equation*}%
so we have 
\begin{eqnarray*}
U &=&-\left( 6.2\times 10^{-30}\unit{C}\unit{m}\right) \left( 200\unit{V}/%
\unit{m}\right) \cos \theta \\
&=&-1.\,\allowbreak 24\times 10^{-27}\unit{J}\cos \theta
\end{eqnarray*}%
This is plotted in the next figure.\FRAME{dtbpFX}{4.4996in}{1.6873in}{0pt}{}{%
}{Plot}{\special{language "Scientific Word";type "MAPLEPLOT";width
4.4996in;height 1.6873in;depth 0pt;display "USEDEF";plot_snapshots
TRUE;mustRecompute FALSE;lastEngine "MuPAD";xmin "-200";xmax "200";xviewmin
"-200.040000040008";xviewmax "200.040000040008";yviewmin
"-1.23947721233825E-27";yviewmax "1.24024021449825E-27";rangeset"X";plottype
4;labeloverrides 3;x-label "Theta (deg)";y-label "U (J)";axesFont "Times New
Roman,12,0000000000,useDefault,normal";numpoints 100;plotstyle
"patch";axesstyle "normal";axestips FALSE;xis \TEXUX{v58130};var1name
\TEXUX{$\theta $};function \TEXUX{$-1.\,\allowbreak 24\times 10^{-27}\cos
\left( \theta \unit{\U{b0}}\right) $};linecolor "blue";linestyle
1;pointstyle "point";linethickness 3;lineAttributes "Solid";var1range
"-200,200";num-x-gridlines 100;curveColor
"[flat::RGB:0x000000ff]";curveStyle "Line";VCamFile
'LTUYJ911.xvz';valid_file "T";tempfilename
'LTUWCG0M.wmf';tempfile-properties "XPR";}}\newline
At zero degrees we can see that it takes energy (work) to make the dipole
spin. It will try to stay at zero degrees and a small displacement from zero
degrees will will cause the dipole to oscillate around $\theta =0$ but it
will return to $\theta =0$ as the added energy is dissipated. then $\theta =0%
\unit{rad}$ is a stable equilibrium. Conversely, at $\theta =\pi \unit{rad}$
we are at a maximum potential energy. We get rotational kinetic energy if we
cause any small displacement $\Delta \theta .$ The dipole will angularly
accelerate. $\theta =\pm \pi \unit{rad}$ is an unstable equilibrium.

\section{Shooting $\protect\alpha $-particles}

Let's use electric potentials to think about a famous experiment. Ernest
Rutherford shot $\alpha $-particles, $q=+2q_{e}$ at gold nuclei, $%
q=+79q_{e}. $ How close will the $\alpha $-particles get if the collision is
head-on and the initial speed of the $\alpha $-particles is $3\times 10^{6}%
\unit{m}/\unit{s}?$

The easiest way to approach this is to use conservation of energy. The
energies before and after must be the same because we have no frictional or
dissipative forces.

The before and after pictures are as shown. The $\alpha $-particle, of
course, is our mover.\FRAME{dhF}{2.6135in}{1.9112in}{0pt}{}{}{Figure}{%
\special{language "Scientific Word";type "GRAPHIC";maintain-aspect-ratio
TRUE;display "USEDEF";valid_file "T";width 2.6135in;height 1.9112in;depth
0pt;original-width 2.5719in;original-height 1.8732in;cropleft "0";croptop
"1";cropright "1";cropbottom "0";tempfilename
'LTUZ5608.wmf';tempfile-properties "XPR";}}

We can write 
\begin{equation*}
K_{i}+U_{i}=K_{f}+U_{f}
\end{equation*}%
when the $\alpha $-particles are at their closest distance to the gold
nuclei, then $K_{f}=0$. We can envision starting the $\alpha $-particles
from effectively an infinite distance away. Then $U_{i}\approx 0.$ so 
\begin{equation*}
\frac{1}{2}m_{\alpha }v^{2}=\frac{1}{4\pi \epsilon _{o}}\frac{%
Q_{Au}q_{\alpha }}{r}
\end{equation*}%
Solving for $r$ gives%
\begin{eqnarray*}
r &=&\frac{1}{4\pi \epsilon _{o}}\frac{Q_{Au}q_{\alpha }}{\frac{1}{2}%
m_{\alpha }v^{2}} \\
&=&\frac{1}{2\pi \epsilon _{o}}\frac{\left( 79q_{e}\right) \left(
4q_{e}\right) }{m_{\alpha }v^{2}}
\end{eqnarray*}%
then%
\begin{eqnarray*}
r &=&\frac{1}{2\pi \left( 8.85\times 10^{-12}\frac{\unit{C}^{2}}{\unit{N}%
\unit{m}^{2}}\right) }\frac{158\left( 1.602\times 10^{-19}\unit{C}\right)
^{2}}{\left( 6.\,\allowbreak 642\,2\times 10^{-27}\unit{kg}\right) \left(
3\times 10^{6}\unit{m}/\unit{s}\right) ^{2}} \\
&=&1.\,\allowbreak 219\,8\times 10^{-12}\unit{m}
\end{eqnarray*}%
This is a very small number! and it sets a bound on how large the nucleus of
the gold atom can be.

Next lecture, we will try to make our use of electrical potential energy
more practical by defining the electrical potential energy per unit charge,
and applying this to problems involving moving charges (like those in
electric circuits).

%TCIMACRO{%
%\TeXButton{Basic Equations}{\hspace{-1.3in}{\LARGE Basic Equations\vspace{0.25in}}}}%
%BeginExpansion
\hspace{-1.3in}{\LARGE Basic Equations\vspace{0.25in}}%
%EndExpansion

\chapter{Electric Potentials}

We defined electrical potential energy last time. We used an analogy with
gravitational fields and gravitational potential energy. But there is a
missing piece. The gravitational environment property 
\begin{equation*}
g=\left( G\frac{M_{E}}{R_{E}^{2}}\right)
\end{equation*}%
(where here the subscript $E$ is for the environmental object) showed up in
our equation for the gravitational potential 
\begin{equation*}
U=-\left( G\frac{M_{E}}{R_{E}}\right) m_{o}
\end{equation*}%
We found the same form for the electrical potential energy.%
\begin{equation*}
U_{12}=\frac{1}{4\pi \epsilon _{o}}\frac{q_{1}q_{2}}{r_{12}}
\end{equation*}%
or we could write this as 
\begin{equation*}
U_{12}=\left( \frac{1}{4\pi \epsilon _{o}}\frac{q_{1}}{r_{12}}\right) q_{2}
\end{equation*}%
where charge $q_{2}$ would be our mover charge. By analogy, then 
\begin{equation*}
\frac{1}{4\pi \epsilon _{o}}\frac{q_{1}}{r_{12}}
\end{equation*}%
must represent the environment set up by $q_{1.}$ And sure enough, it has a $%
q_{1}$ in it. But this does not have the units of electric field. So it must
be a new quantity. We will need a name for this new representation of the
environment created by $q_{1}.$

%TCIMACRO{%
%\TeXButton{Fundamental Concepts}{\hspace{-1.3in}{\LARGE Fundamental Concepts\vspace{0.25in}}}}%
%BeginExpansion
\hspace{-1.3in}{\LARGE Fundamental Concepts\vspace{0.25in}}%
%EndExpansion

\begin{itemize}
\item Electric potential is a representation of the electric field
environment.

\item Electric potential is defined as the potential energy per unit charge.

\item Equipotential lines are drawn to show constant electric potential
surfaces

\item The volt as a measure of electric potential

\item The electron-volt as a measure of energy (and speed).
\end{itemize}

\subsection{Electric Potential Difference}

%TCIMACRO{%
%\TeXButton{Question 223.30.1}{\marginpar {
%\hspace{-0.5in}
%\begin{minipage}[t]{1in}
%\small{Question 223.30.1}
%\end{minipage}
%}}}%
%BeginExpansion
\marginpar {
\hspace{-0.5in}
\begin{minipage}[t]{1in}
\small{Question 223.30.1}
\end{minipage}
}%
%EndExpansion
Let's give a symbol and a name to our new environment quantity. 
\begin{equation*}
V_{12}=\frac{1}{4\pi \epsilon _{o}}\frac{q_{1}}{r_{12}}
\end{equation*}%
where we understand that $q_{1}$ is making the environment and we are
measuring that environment a distance $r_{12}$ from $q_{1}.$ Thus $q_{1}$ is
the environmental charge.

Then 
\begin{eqnarray*}
U_{12} &=&\left( \frac{1}{4\pi \epsilon _{o}}\frac{q_{1}}{r_{12}}\right)
q_{2} \\
&=&\left( V_{12}\right) q_{2}
\end{eqnarray*}%
It's traditional to drop the subscripts on the $V$ 
\begin{equation*}
V=\frac{1}{4\pi \epsilon _{o}}\frac{q}{r}
\end{equation*}%
where we understand that an environmental charge labeled just $q$ is making
the environment and $q_{2}$ is a distance $r$ from $q.$ In that case we can
write 
\begin{equation*}
U_{12}=\left( V\right) q_{2}
\end{equation*}%
or 
\begin{equation*}
V=\frac{U_{12}}{q_{2}}
\end{equation*}%
This new environment representation appears to be an amount of potential
energy per unit charge. In general any electrical potential energy $\left(
U\right) $ per unit charge $\left( q\right) $ is is called an \emph{electric
potential}. 
\begin{equation*}
V=\frac{U}{q}
\end{equation*}

This is a somewhat unfortunate name, because it sounds like electric
potential energy. But it is not, it is a representation of the environment
set up by the electric field. We don't get electric potential energy without
multiplying by a charge. $U=Vq_{o}.$

We will give electric potential the symbol $V$ but usually the important
quantity is a change in potential energy, then 
\begin{equation}
\Delta V=\frac{\Delta U}{q}
\end{equation}%
If I\ know $\Delta V$ for a configuration of charge (like our capacitor
plates) then I\ can find the $\Delta U$ of different charges by multiplying
by the amount of charge in each case%
\begin{eqnarray*}
\Delta U_{1} &=&q_{1}\Delta V \\
\Delta U_{2} &=&q_{2}\Delta V \\
&&\vdots
\end{eqnarray*}%
which is convenient if I\ am accelerating many different charges. We do this
in linear accelerators or \textquotedblleft atom smashers\textquotedblright\
so this is important to physicists! We can see that the units of $\Delta V$
must be 
\begin{equation}
\frac{\unit{J}}{\unit{C}}=\unit{V}
\end{equation}%
which has been named the \emph{Volt} and is given the symbol, $\unit{V}.$

Now this may seem familiar. Can you think of anything that carries units of
volts? Let's consider a battery. In our cell phones we have something like a 
$3.8\unit{V}$ lithium-ion battery. Inside the battery we would expect that a
charge would experience a potential energy difference. We use the battery so
we can convert that potential energy into some other form of energy (e.g.
radio wave energy for our phone's wifi). The potential energy achieved
depends on the charge carrier. We would have electrons in metals but we
would have ions in a solution. This is so convenient to express the
potential energy per unit charge, that it is the common form or expressing
the energy given by most electrical sources.%
%TCIMACRO{%
%\TeXButton{Question 223.30.2}{\marginpar {
%\hspace{-0.5in}
%\begin{minipage}[t]{1in}
%\small{Question 223.30.2}
%\end{minipage}
%}}}%
%BeginExpansion
\marginpar {
\hspace{-0.5in}
\begin{minipage}[t]{1in}
\small{Question 223.30.2}
\end{minipage}
}%
%EndExpansion
%TCIMACRO{%
%\TeXButton{Question 223.30.3}{\marginpar {
%\hspace{-0.5in}
%\begin{minipage}[t]{1in}
%\small{Question 223.30.3}
%\end{minipage}
%}}}%
%BeginExpansion
\marginpar {
\hspace{-0.5in}
\begin{minipage}[t]{1in}
\small{Question 223.30.3}
\end{minipage}
}%
%EndExpansion

\subsection{Electric Potential}

Let's write out the electric potential difference between points $A$ and $B.$
It is the change in potential energy per unit charge as the charge travels
from point $A$ to point $B$ 
\begin{equation}
\Delta V=V_{B}-V_{A}=\frac{\Delta U}{q}
\end{equation}%
This is clearly a measure of how the environment changes along our path from 
$A$ to $B.$

Let's reconsider gravitational potential energy. We remember that if the
field is uniform (that is, if we are near the Earth's surface so the field
seems uniform) we can set the zero point of the potential energy anywhere we
find convenient for our problem, with the provision that once it is set for
the problem, we have to stick with our choice.

One logical choice for many electrical appliances is to set the Earth's
potential equal to zero. Note! this is not true for point mass problems
where we have already set the potential energy $U=0$ at $r=\infty .$

In our gravitational analogy, this is a little bit like mean sea level.
Think of river flow. The lowest point on the planet is not mean sea level.
But any water above mean sea level will tend to flow downward to this point.
Of course, if we have land below mean sea level, the water would tend to
continue downward (like water flows to the Dead Sea). The direction of water
flow is given by the potential energy difference, not that actual value of
the potential energy. It is the same way with electric potential. If we have
charge at a potential that is higher than the Earth's potential, then charge
will flow toward the Earth.

%TCIMACRO{%
%\TeXButton{Question 223.30.4}{\marginpar {
%\hspace{-0.5in}
%\begin{minipage}[t]{1in}
%\small{Question 223.30.4}
%\end{minipage}
%}}}%
%BeginExpansion
\marginpar {
\hspace{-0.5in}
\begin{minipage}[t]{1in}
\small{Question 223.30.4}
\end{minipage}
}%
%EndExpansion
Consider a $9\unit{V}$ battery. If the negative terminal is connected to a
grounding rod or metal water pipe, it will be at the electric potential of
the Earth while it's positive terminal will be at $\Delta V=9\unit{V}$ above
the Earth's potential. Likewise, in your home, you probably have a $110\unit{%
V}$ outlet. One wire is likely set to the potential of the Earth by
connecting it to a ground rod. The others are at $\Delta V=110\unit{V}$
above it\footnote{%
House voltages are alternating voltages. We will deal with them later in
this course.}.

In our phones, we don't have a ground wire, so we cannot guarantee that the
negative terminal of the battery is at the same potential as the Earth. If
our appliances in our house are not all grounded to the same potential,
there is a danger that there will be a large enough difference in their
potentials (think potential energy per unit charge) to cause the charges to
accelerate from one appliance to another. It is the difference in potential
that counts! This is a spark or shock that could hurt someone or damage
equipment. That is why we now use grounded outlets. These outlets have a
third wire that is tied to all the other outlet's third wire and also tied
physically to the ground near your house or apartment. This way, all
appliances are ensured to have the same low electric potential point.

\section{Example, potential of a capacitor}

Let's calculate the potential of our favorite device, the capacitor.

\FRAME{dhF}{2.693in}{3.3797in}{0pt}{}{}{Figure}{\special{language
"Scientific Word";type "GRAPHIC";maintain-aspect-ratio TRUE;display
"USEDEF";valid_file "T";width 2.693in;height 3.3797in;depth
0pt;original-width 4.1156in;original-height 5.1759in;cropleft "0";croptop
"1";cropright "1";cropbottom "0";tempfilename
'NBCSHZ16.wmf';tempfile-properties "XPR";}}

The nice uniform field makes this a useful device for thinking about
electric potentials. We have found that field to be%
\begin{equation*}
E=\frac{\eta }{\epsilon _{o}}
\end{equation*}%
with a direction from positive to negative. The work to push a mover charge
from one side to the other is given by 
\begin{equation*}
W=\int F_{e}\cdot dx
\end{equation*}%
The force is uniform since the field is uniform (near the middle at least) 
\begin{equation*}
F_{e}=q_{o}E
\end{equation*}%
then our work becomes 
\begin{eqnarray*}
W &=&\int q_{o}E\cdot dx \\
&=&q_{o}E\Delta x
\end{eqnarray*}%
and the amount of potential energy is 
\begin{equation*}
\left\vert \Delta U\right\vert =\left\vert -q_{o}E\Delta x\right\vert
\end{equation*}

We can set the zero potential energy point any where we want, but it is
tradition to set $U=0$ at the negative plate. If we do this we end up with
the potential energy difference going from the negative plate to the
positive plate being 
\begin{equation*}
\Delta U=q_{o}Ed
\end{equation*}%
Then if we go from the negative plate to the positive plate we have a
positive $\Delta U.$

We have seen all this before when we compared the electric potential energy
of a uniform gravitation field and a uniform electrical field. Now let's
calculate the electric potential difference 
\begin{equation*}
\Delta V=\frac{\Delta U}{q_{o}}=\frac{q_{o}Ed}{q_{o}}=Ed
\end{equation*}%
Remember that the field is created by the charges on the capacitor plates,
so it exists whether we put any $q_{o}$ inside of the capacitor or not. Then
the potential difference must exist whether or not there is a charge $q_{o}$
inside the capacitor.

You probably already know that a voltmeter can measure the electric
potential difference between two points, say, the plates of a capacitor. If
we use such a meter we could find the field inside the capacitor (well,
almost, remember our approximation is good for the center of the plates).%
\begin{equation*}
E=\frac{\Delta V}{d}
\end{equation*}

\subsection{Equipotential Lines}

%TCIMACRO{%
%\TeXButton{Question 223.30.5}{\marginpar {
%\hspace{-0.5in}
%\begin{minipage}[t]{1in}
%\small{Question 223.30.5}
%\end{minipage}
%}}}%
%BeginExpansion
\marginpar {
\hspace{-0.5in}
\begin{minipage}[t]{1in}
\small{Question 223.30.5}
\end{minipage}
}%
%EndExpansion
We need a way to envision this new environmental quantity that, like a
field, has a value throughout all space. Our analogy with gravity gives us
an idea. Suppose we envision the height potential energy as the top of a
hill. Then the low potential energy would be the bottom of the hill. We know
from our Young Men and Young Women's Camp experiences how to show a change
in gravitational potential energy. We plot on a map lines of constant
potential energy. We call it constant elevation, but since near the Earth's
surface $U_{g}=mgh$ the potential energy is proportional to the height, so
we can say these lines are lines of constant potential energy. Here is an
example for Mt. Shasta.\FRAME{dhFU}{4.3152in}{1.7669in}{0pt}{\Qcb{Map
courtesy USGS, Picture is in the Public Domain.}}{}{Figure}{\special%
{language "Scientific Word";type "GRAPHIC";maintain-aspect-ratio
TRUE;display "USEDEF";valid_file "T";width 4.3152in;height 1.7669in;depth
0pt;original-width 12.7491in;original-height 5.1906in;cropleft "0";croptop
"1";cropright "1";cropbottom "0";tempfilename
'LTXT7707.wmf';tempfile-properties "XPR";}}We can think of these lines of
constant potential energy as paths over which the gravitational field does
no work. If we walked along one of these lines we would get neither higher
nor lower and though we might do work to move us to overcome some friction,
the gravitational field would do no work. And we would do no work in
changing elevation.

%TCIMACRO{%
%\TeXButton{Question 223.30.6}{\marginpar {
%\hspace{-0.5in}
%\begin{minipage}[t]{1in}
%\small{Question 223.30.6}
%\end{minipage}
%}}}%
%BeginExpansion
\marginpar {
\hspace{-0.5in}
\begin{minipage}[t]{1in}
\small{Question 223.30.6}
\end{minipage}
}%
%EndExpansion
Likewise we can draw lines of equal potential for our capacitor. When moving
along these lines the electric field would do no work. \FRAME{dhF}{2.4232in}{%
2.9136in}{0pt}{}{}{Figure}{\special{language "Scientific Word";type
"GRAPHIC";maintain-aspect-ratio TRUE;display "USEDEF";valid_file "T";width
2.4232in;height 2.9136in;depth 0pt;original-width 3.3088in;original-height
3.9825in;cropleft "0";croptop "1";cropright "1";cropbottom "0";tempfilename
'LTXTQ30C.wmf';tempfile-properties "XPR";}}Of course we could draw these
lines for a crazier device. Say, for our charged conductor\FRAME{dhF}{%
1.4667in}{2.2053in}{0pt}{}{}{Figure}{\special{language "Scientific
Word";type "GRAPHIC";maintain-aspect-ratio TRUE;display "USEDEF";valid_file
"T";width 1.4667in;height 2.2053in;depth 0pt;original-width
1.4321in;original-height 2.1646in;cropleft "0";croptop "1";cropright
"1";cropbottom "0";tempfilename 'LTXTLQ0A.wmf';tempfile-properties "XPR";}}%
Notice that our equal potential lines are always perpendicular to the field.
From 
\begin{equation*}
W=\int q_{o}\overrightarrow{\mathbf{E}}\cdot d\overrightarrow{\mathbf{x}}
\end{equation*}%
we can see that if the path we travel is perpendicular to the field, no work
is done. This is like us marching along around the mountain neither going up
nor down.

\section{Electron Volt}

Suppose I set up our uniform electric field device again\FRAME{dhF}{2.693in}{%
3.3797in}{0pt}{}{}{Figure}{\special{language "Scientific Word";type
"GRAPHIC";maintain-aspect-ratio TRUE;display "USEDEF";valid_file "T";width
2.693in;height 3.3797in;depth 0pt;original-width 4.1156in;original-height
5.1759in;cropleft "0";croptop "1";cropright "1";cropbottom "0";tempfilename
'LTUWDJAJ.wmf';tempfile-properties "XPR";}}We are not including any
gravitational field, so the directions involved are all relative to the
placement of the capacitor plate orientation.

This time, suppose I make the potential difference $\Delta V=1\unit{V}.$ I
release a proton near the high potential side. What is the kinetic energy of
the proton as it hits the low potential side? From the work energy theorem 
\begin{equation*}
W_{nc}=\Delta K+\Delta U
\end{equation*}%
and if we do this in a vacuum so there is no non-conservative work,%
\begin{eqnarray*}
\Delta K &=&-\Delta U \\
K_{f}-K_{i} &=&-\Delta U \\
K_{f} &=&-\Delta U
\end{eqnarray*}

We can find the potential energy loss from what we just studied%
\begin{equation*}
\Delta V=\frac{\Delta U}{q}
\end{equation*}%
so we can find the potential energy as%
\begin{equation*}
\Delta U=q\Delta V
\end{equation*}%
but remember we are going from a high to a low potential%
\begin{equation*}
\Delta V=V_{f}-V_{i}
\end{equation*}%
this will be negative, so the potential energy change will be negative too.%
\begin{eqnarray*}
K_{f} &=&-\Delta U \\
&=&-q\Delta V
\end{eqnarray*}%
which will be a positive value (which is good, because I don't know what
negative kinetic energy would mean).%
\begin{equation*}
K_{f}=-q\Delta V
\end{equation*}%
We can find the amount of energy in Jules%
\begin{eqnarray*}
K_{f} &=&-\left( 1.6\times 10^{-19}\unit{C}\right) \left( -1\unit{V}\right)
\\
&=&\allowbreak 1.\,\allowbreak 6\times 10^{-19}\unit{J}
\end{eqnarray*}%
since we defined a volt as $\unit{V}=\frac{\unit{J}}{\unit{C}}$ .

You might think this is not very useful, but remember that $K=\frac{1}{2}%
mv^{2}.$ The kinetic energy is related to how fast the proton is going. In a
way, the kinetic energy tells us how fast the particle is going (we know
it's mass). If you read about the Large Hadron Collider at CERN, in
Switzerland the \textquotedblleft speeds\textquotedblright of the particles
will be given in energy units that are multiples of $\allowbreak
1.\,\allowbreak 6\times 10^{-19}\unit{J}.$ We call this unit an
electron-volt ($\unit{eV}$).\FRAME{dtbpFU}{2.3496in}{2.9368in}{0pt}{\Qcb{%
Beam magnet and Section of the Beam Pipe of the LHC. This section is
actually no longer used and is in a service area $100\unit{m}$ above the
operating LHC. The people you see are part of a BYU-I Physics Department
Tour of the facility.}}{}{Figure}{\special{language "Scientific Word";type
"GRAPHIC";maintain-aspect-ratio TRUE;display "USEDEF";valid_file "T";width
2.3496in;height 2.9368in;depth 0pt;original-width 2.3656in;original-height
2.9634in;cropleft "0";croptop "1";cropright "1";cropbottom "0";tempfilename
'MOAAV101.wmf';tempfile-properties "XPR";}}

We can finish this problem by finding the speed of the particle%
\begin{equation*}
K=\frac{1}{2}mv^{2}
\end{equation*}%
so%
\begin{equation*}
\frac{2K}{m}=v^{2}
\end{equation*}%
or%
\begin{eqnarray*}
v &=&\sqrt{\frac{2K}{m}} \\
&=&\sqrt{\allowbreak \frac{2\left( 1.\,\allowbreak 6\times 10^{-19}\unit{J}%
\right) }{1.00728\unit{u}\frac{\allowbreak 1.\,\allowbreak 660\,5\times
10^{-27}\unit{kg}}{1\unit{u}}}} \\
&=&\allowbreak 13832.\frac{\unit{m}}{\unit{s}}
\end{eqnarray*}%
Which is pretty fast, but the Large Hadron Collider at CERN can provide
energies up to $7\times 10^{14}\unit{eV}$ which would give our proton a
speed of $99.9999991\%$ of the speed of light. \FRAME{dtbpFU}{2.7151in}{%
2.0427in}{0pt}{\Qcb{CERN CMS\ detector during a maintenance event. The
bright metal pipe seen in the middle of the detector is the beam pipe
through which the accelerated protons travel. Note the workers near the
scaffolding for scale.}}{}{Figure}{\special{language "Scientific Word";type
"GRAPHIC";maintain-aspect-ratio TRUE;display "USEDEF";valid_file "T";width
2.7151in;height 2.0427in;depth 0pt;original-width 2.739in;original-height
2.0534in;cropleft "0";croptop "1";cropright "1";cropbottom "0";tempfilename
'MOAB5Z02.wmf';tempfile-properties "XPR";}}Note that this energy would seem
to provide a faster speed--faster than light! But with energies this high we
have to use Einstein's theory of Special Relativity to calculate the
particle speed. And, sadly, that is not part of this class. If you are
planning to work on the GPS\ system, or future space craft, you might need
to take yet another physics class so you can do this sort of calculation.

You might guess that we will want to know the electric potential of more
complex configurations of charge. We will take on this job in the next
lecture.

%TCIMACRO{%
%\TeXButton{Basic Equations}{\hspace{-1.3in}{\LARGE Basic Equations\vspace{0.25in}}}}%
%BeginExpansion
\hspace{-1.3in}{\LARGE Basic Equations\vspace{0.25in}}%
%EndExpansion

The electric potential is the electrical potential per unit charge%
\begin{equation*}
\Delta V=V_{B}-V_{A}=\frac{\Delta U}{q}
\end{equation*}%
For the special case of a constant electric field in a capacitor the
electrical potential is just%
\begin{equation*}
\Delta V=E\Delta s
\end{equation*}%
where $\Delta s$ is the distance traveled from one side of the capacitor to
the other.

The unit%
\begin{equation*}
1\unit{eV}=1.\,\allowbreak 6\times 10^{-19}\unit{J}
\end{equation*}

\chapter{Electric potential of charges and groups of charges}

Now that we have a new representation of the environment created by
environmental charges, we will need to be able to calculate values for that
representation for different configurations of charge like we did for
electrical fields. But there is a huge benefit in using the electric
potential representation, electric potentials are not vectors! So we don't
have to deal with the vector nature of the field environment. The vector
nature is still there, but we will ignore it. This means we will give up
being able to give up vector directions for movement of our mover charges in
many cases. But we can know much about the movement and the equations will
be much simpler. We will take on the usual cases of environments from a
point charge, a collection of point charges, and a continuous distribution
of charges.

%TCIMACRO{%
%\TeXButton{Fundamental Concepts}{\hspace{-1.3in}{\LARGE Fundamental Concepts\vspace{0.25in}}}}%
%BeginExpansion
\hspace{-1.3in}{\LARGE Fundamental Concepts\vspace{0.25in}}%
%EndExpansion

\begin{itemize}
\item Finding the electric potential of a point charge

\item Finding the electric potential of two point charges

\item Finding the electric potential of many point charges

\item Finding the electric potential of continuous distributions of point
charges.
\end{itemize}

\section{Point charge potential}

The capacitor was an easy electric potential to describe. Let's go back to a
slightly harder one, the potential due to just one point charge. The
potential energy depends on two charges%
\begin{equation*}
U_{e}=-\frac{1}{4\pi \epsilon _{o}}\frac{Qq}{r}
\end{equation*}%
but the potential just depends on one.%
\begin{equation*}
V=\frac{U}{q}
\end{equation*}%
where $U$ is a function of $q,$ so a charge will cancel. But which charge do
we divide by?

We need two charges to make a force,%
\begin{equation*}
F=\frac{1}{4\pi \epsilon _{o}}\frac{q_{1}q_{2}}{r^{2}}
\end{equation*}
but when we defined the electric field we said the field from charge $1$
would be there whether or not charge $2$ was present. The situation is the
same for electric potential.

We say we have an electric potential due to the first charge even if the
second charge is not there. This is like saying there is a potential energy
per unit rock, even if there is no rock to fall down the hill. The hill is
there whether or not we are throwing rocks down it.

For electric potential, the potential is due to the field, and the field is
there whether another charge is there or not.

Let's find this potential due to just one charge, but let's find it in a way
that demonstrates how to find potentials in any situation. After all, from
what we know about point charges, we can predict that%
\begin{equation*}
V=\frac{U}{q}=\frac{\frac{1}{4\pi \epsilon _{o}}\frac{Qq}{r}}{q}=\frac{1}{%
4\pi \epsilon _{o}}\frac{Q}{r}
\end{equation*}%
But not not all situations come so easily. We only know forms for $U$ for
capacitors and point charges so far. So let's see how to do this in general,
and compare our answer for the point charge with what we have guessed from
knowing $U$.

Symmetry tells us the field will be radial, so the equipotential surfaces
must be concentric spheres. Here is our situation:\FRAME{dtbpF}{2.0055in}{%
1.9614in}{0pt}{}{}{Figure}{\special{language "Scientific Word";type
"GRAPHIC";maintain-aspect-ratio TRUE;display "USEDEF";valid_file "T";width
2.0055in;height 1.9614in;depth 0pt;original-width 2.6922in;original-height
2.6333in;cropleft "0";croptop "1";cropright "1";cropbottom "0";tempfilename
'M5GSYB02.wmf';tempfile-properties "XPR";}}We wish to follow the marked path
from $A$ to $B$ finding the potential difference $\Delta V=V_{B}-V_{A}.$

Remember that the field due to a charge $q$ is radially outward from the
charge. To find the potential we start with what we found last lecture, for
a constant field 
\begin{equation*}
\Delta V=\frac{\Delta U}{q_{o}}=\frac{q_{o}E\Delta s}{q_{o}}=E\Delta s
\end{equation*}%
where $s$ is the path length along our chosen path from $A$ to $B$. For our
capacitor, this was just the distance from one side to the other, but here
we need to be more general. We should really write this as 
\begin{equation*}
\Delta V=\overrightarrow{\mathbf{E}}\cdot \overrightarrow{\Delta s}
\end{equation*}%
Further, our field, $E,$ changes, so technically this value for $\Delta V$
is not correct. But if we take vary small paths, $\Delta \overrightarrow{%
\mathbf{s}},$ then the field will be nearly constant over the small
distances. Then we can add up the contribution of each small distance, $%
\Delta \overrightarrow{\mathbf{s}}_{i}$ to deal with the entire path from $A$
to $B$ for our point charge geometry. \FRAME{dtbpF}{2.1664in}{2.3186in}{0pt}{%
}{}{Figure}{\special{language "Scientific Word";type
"GRAPHIC";maintain-aspect-ratio TRUE;display "USEDEF";valid_file "T";width
2.1664in;height 2.3186in;depth 0pt;original-width 2.7415in;original-height
2.9369in;cropleft "0";croptop "1";cropright "1";cropbottom "0";tempfilename
'M5GT1604.wmf';tempfile-properties "XPR";}}That is, we take a small amount
of path difference $\Delta \overrightarrow{\mathbf{s}}_{i}$ and add up the
contribution, $\overrightarrow{\mathbf{E}}\cdot \overrightarrow{\Delta s}%
_{i} $ from this small path. Then we can repeat this for the next $\Delta 
\overrightarrow{\mathbf{s}}_{i+1}$ and the next, until we have the
contribution of each pice of the path. We can call the contribution from one
piece. 
\begin{equation*}
\Delta V_{i}=\overrightarrow{\mathbf{E}}\cdot \overrightarrow{\Delta s}_{i}
\end{equation*}%
The total potential difference would be%
\begin{equation*}
\Delta V=\sum_{i}\overrightarrow{\mathbf{E}}\cdot \overrightarrow{\Delta s}%
_{i}
\end{equation*}%
In the limit that the $\Delta s_{i}$ become very small this becomes an
integral 
\begin{equation}
\Delta V=-\int_{A}^{B}\overrightarrow{\mathbf{E}}\cdot d\overrightarrow{%
\mathbf{s}}
\end{equation}%
where $A$ and $B$ are any two points. \FRAME{dtbpF}{1.9086in}{1.9847in}{0pt}{%
}{}{Figure}{\special{language "Scientific Word";type
"GRAPHIC";maintain-aspect-ratio TRUE;display "USEDEF";valid_file "T";width
1.9086in;height 1.9847in;depth 0pt;original-width 2.6939in;original-height
2.8003in;cropleft "0";croptop "1";cropright "1";cropbottom "0";tempfilename
'M5GT2C05.wmf';tempfile-properties "XPR";}}Here is an expansion of the
region about $A$ and $B$. \FRAME{dtbpF}{3.557in}{2.4829in}{0pt}{}{}{Figure}{%
\special{language "Scientific Word";type "GRAPHIC";maintain-aspect-ratio
TRUE;display "USEDEF";valid_file "T";width 3.557in;height 2.4829in;depth
0pt;original-width 3.5103in;original-height 2.4422in;cropleft "0";croptop
"1";cropright "1";cropbottom "0";tempfilename
'M5GTBU09.wmf';tempfile-properties "XPR";}}Let's divide up our $d%
\overrightarrow{\mathbf{s}}$ into components in the radial and azimuthal
directions%
\begin{equation*}
d\overrightarrow{\mathbf{s}}=\left( dr\mathbf{\hat{r}}+rd\theta \mathbf{\hat{%
\theta}}\right)
\end{equation*}%
from trigonometry we can see that 
\begin{equation*}
\cos \phi =\frac{dr}{ds}
\end{equation*}%
and 
\begin{equation*}
\sin \phi =\frac{d\theta }{ds}
\end{equation*}%
so%
\begin{eqnarray*}
dr &=&ds\cos \phi \\
d\theta &=&ds\sin \phi
\end{eqnarray*}%
and we can write 
\begin{equation*}
d\overrightarrow{\mathbf{s}}=\left( ds\cos \phi \mathbf{\hat{r}}+rds\sin
\phi \mathbf{\hat{\theta}}\right)
\end{equation*}

The field due to the point charge is 
\begin{equation}
\overrightarrow{\mathbf{E}}=\frac{1}{4\pi \epsilon _{o}}\frac{q}{r^{2}}%
\mathbf{\hat{r}}
\end{equation}%
if we take 
\begin{eqnarray*}
\overrightarrow{\mathbf{E}}\cdot d\overrightarrow{\mathbf{s}} &=&\frac{1}{%
4\pi \epsilon _{o}}\frac{q}{r^{2}}\mathbf{\hat{r}}\cdot d\overrightarrow{%
\mathbf{s}} \\
&=&\frac{1}{4\pi \epsilon _{o}}\frac{q}{r^{2}}\mathbf{\hat{r}}\cdot \left(
ds\cos \phi \mathbf{\hat{r}}+ds\sin \phi \mathbf{\hat{\theta}}\right)
\end{eqnarray*}%
we get only a radial contribution since $\mathbf{\hat{r}}\cdot \mathbf{\hat{%
\theta}=0}$. Then 
\begin{eqnarray*}
\overrightarrow{\mathbf{E}}\cdot d\overrightarrow{\mathbf{s}} &=&\frac{1}{%
4\pi \epsilon _{o}}\frac{q}{r^{2}}\mathbf{\hat{r}}\cdot ds\cos \phi \mathbf{%
\hat{r}}+0 \\
&=&\frac{1}{4\pi \epsilon _{o}}\frac{q}{r^{2}}ds\cos \phi
\end{eqnarray*}%
where $\phi $ is the angle between $d\overrightarrow{\mathbf{s}}$ and $%
\mathbf{\hat{r}}$ and where we recall that $\mathbf{\hat{r}}\cdot \mathbf{%
\hat{r}}=1\mathbf{.}$ Recalling that 
\begin{equation*}
dr=ds\cos \phi
\end{equation*}%
we can eliminate $\phi $ from our equation%
\begin{equation*}
\overrightarrow{\mathbf{E}}\cdot d\overrightarrow{\mathbf{s}}=\frac{1}{4\pi
\epsilon _{o}}\frac{q}{r^{2}}dr
\end{equation*}%
and we can integrate this!%
\begin{eqnarray*}
\Delta V &=&-\int_{r_{A}}^{r_{B}}\frac{1}{4\pi \epsilon _{o}}\frac{q}{r^{2}}%
dr \\
&=&-\frac{q}{4\pi \epsilon _{o}}\int_{r_{A}}^{r_{B}}\frac{1}{r^{2}}dr \\
&=&\left. \frac{q}{4\pi \epsilon _{o}}\frac{1}{r}\right\vert _{r_{A}}^{r_{B}}
\end{eqnarray*}%
so%
\begin{eqnarray*}
\Delta V &=&\frac{q}{4\pi \epsilon _{o}}\left( \frac{1}{r_{B}}-\frac{1}{r_{A}%
}\right) \\
&=&\frac{1}{4\pi \epsilon _{o}}\frac{q}{r_{B}}-\frac{1}{4\pi \epsilon _{o}}%
\frac{q}{r_{A}} \\
&=&V_{B}-V_{A}
\end{eqnarray*}

%TCIMACRO{%
%\TeXButton{Question 223.31.1}{\marginpar {
%\hspace{-0.5in}
%\begin{minipage}[t]{1in}
%\small{Question 223.31.1}
%\end{minipage}
%}}}%
%BeginExpansion
\marginpar {
\hspace{-0.5in}
\begin{minipage}[t]{1in}
\small{Question 223.31.1}
\end{minipage}
}%
%EndExpansion
Note that the potential depends only on the radial distances from the point
charge--not the path. We would expect this for conservative fields (where
energy is conserved).

We know that, like potential energy, we may choose our zero point for the
electric potential. For a point charge, we said we would take the $%
r_{A}=\infty $ point as $V=0$.\footnote{%
Remember this is because $U\rightarrow 0$ when $r\rightarrow \infty .$} So
you will often see the potential for the point charge written as just%
\begin{equation*}
\Delta V=\frac{1}{4\pi \epsilon _{o}}\frac{q}{r_{B}}
\end{equation*}%
or simply as 
\begin{equation}
V=\frac{1}{4\pi \epsilon _{o}}\frac{q}{r}
\end{equation}%
Here is a plot of this with $q=2\times 10^{-9}\unit{C}$ and the charge
placed right at $x=10\unit{m}.$\FRAME{dtbpFX}{4.1597in}{2.8573in}{0pt}{}{}{%
Plot}{\special{language "Scientific Word";type "MAPLEPLOT";width
4.1597in;height 2.8573in;depth 0pt;display "USEDEF";plot_snapshots
TRUE;mustRecompute FALSE;lastEngine "MuPAD";xmin "-50";xmax "50";ymin
"-50";ymax "50";xviewmin "-50";xviewmax "50";yviewmin "-50";yviewmax
"50";zviewmin "2.55818E-9";zviewmax "3.6963E-7";rangeset"XYZ";phi 65;theta
113;cameraDistance "1.21794";cameraOrientation
"[0,0,386729]";cameraOrientationFixed TRUE;plottype 5;labeloverrides
4;z-label "V";axesFont "Times New
Roman,12,0000000000,useDefault,normal";num-x-gridlines 25;num-y-gridlines
25;plotstyle "patch";axesstyle "normal";axestips FALSE;plotshading
"NONE";lighting 0;xis \TEXUX{x};yis \TEXUX{y};var1name \TEXUX{$x$};var2name
\TEXUX{$y$};function \TEXUX{$\left( \frac{9.99\times 10^{-9}\left( 2\right)
}{\sqrt{\left( \frac{x}{10}+1\right) ^{2}+\left( \frac{y}{10}\right)
^{2}}+1\times 10^{-9}}\right) $};linestyle 1;pointstyle
"point";linethickness 1;lineAttributes "Solid";var1range "-50,50";var2range
"-50,50";surfaceColor "[flat::RGB:0x990000ff:0x990000ff]";surfaceStyle
"Color Patch";num-x-gridlines 75;num-y-gridlines 75;surfaceMesh
"Mesh";rangeset"XY";function \TEXUX{$\left( \frac{9.99\times 10^{-9}\left(
2\right) }{\sqrt{\left( \frac{x}{10}+1\right) ^{2}+\left(
\frac{y}{10}\right) ^{2}}+1\times 10^{-9}}\right) $};linestyle 1;pointstyle
"point";linethickness 1;lineAttributes "Solid";var1range "-50,50";var2range
"-50,50";surfaceColor "[flat::RGB:0000000000:0000000000]";surfaceStyle "Wire
Frame";num-x-gridlines 25;num-y-gridlines 25;surfaceMesh "Contour";VCamFile
'MJ56Z90C.xvz';valid_file "T";tempfilename
'MJ56OM00.wmf';tempfile-properties "XPR";}}

It is probably a good idea to state that in common engineering practice we
kind of do all this backwards. We usually say we will charge up something
until it has a particular voltage. This is because we have batteries or
power supplies that are charge delivery services. They can provide enough
charge to make some object have the desired voltage. By \textquotedblleft
desired voltage\textquotedblright\ we always mean the voltage at a conductor
surface in our apparatus.

Early \emph{electrodes} were spherical, so let's consider making a spherical
conductor have a particular potential at it's surface. A sphere of charge
with radius $R$ would have 
\begin{equation*}
V=\frac{1}{4\pi \epsilon _{o}}\frac{Q}{R}
\end{equation*}%
at it's surface. We can guess this because Gauss' law tells us that the
field of a charged sphere is the same as that of a point charge with the
same $Q.$ Then it takes 
\begin{equation*}
Q=4\pi \epsilon _{o}RV
\end{equation*}%
to get the voltage we want. The battery or power supply must provide this.
If the power supply or battery has a large amperage (ability to supply
charge) this happens quickly. But away from the electrode the potential
falls off. We can find how it falls off by again using 
\begin{equation*}
V=\frac{1}{4\pi \epsilon _{o}}\frac{Q}{r}
\end{equation*}%
but with charge%
\begin{equation*}
Q=4\pi \epsilon _{o}RV_{o}
\end{equation*}%
so that%
\begin{equation*}
V=\frac{1}{4\pi \epsilon _{o}}\frac{4\pi \epsilon _{o}RV_{o}}{r}
\end{equation*}
or%
\begin{equation*}
V=\frac{R}{r}V_{o}
\end{equation*}%
where $V_{o}$ is the voltage at the surface. We can see that as $r$
increases, $V$ decreases.

\subsection{Two point charges}

%TCIMACRO{%
%\TeXButton{Question 223.31.2}{\marginpar {
%\hspace{-0.5in}
%\begin{minipage}[t]{1in}
%\small{Question 223.31.2}
%\end{minipage}
%}}}%
%BeginExpansion
\marginpar {
\hspace{-0.5in}
\begin{minipage}[t]{1in}
\small{Question 223.31.2}
\end{minipage}
}%
%EndExpansion
%TCIMACRO{%
%\TeXButton{Question 223.31.3}{\marginpar {
%\hspace{-0.5in}
%\begin{minipage}[t]{1in}
%\small{Question 223.31.3}
%\end{minipage}
%}}}%
%BeginExpansion
\marginpar {
\hspace{-0.5in}
\begin{minipage}[t]{1in}
\small{Question 223.31.3}
\end{minipage}
}%
%EndExpansion
We can guess from our treatment of the potential energy of two point charges
that the electric potential of two point charges is just the sum of the
individual point charge potentials.%
\begin{eqnarray*}
V &=&V_{1}+V_{2} \\
&=&\frac{1}{4\pi \epsilon _{o}}\frac{q_{1}}{r_{1}}+\frac{1}{4\pi \epsilon
_{o}}\frac{q_{2}}{r_{2}} \\
&=&\frac{1}{4\pi \epsilon _{o}}\left( \frac{q_{1}}{r_{1}}+\frac{q_{2}}{r_{2}}%
\right)
\end{eqnarray*}%
It is instructive to look at the special case of two opposite charges (our
dipole). We can plot the electric potential in a plane through the two
charges. \FRAME{dhF}{1.6516in}{2.2911in}{0pt}{}{}{Figure}{\special{language
"Scientific Word";type "GRAPHIC";maintain-aspect-ratio TRUE;display
"USEDEF";valid_file "T";width 1.6516in;height 2.2911in;depth
0pt;original-width 1.6542in;original-height 2.3062in;cropleft "0";croptop
"1";cropright "1";cropbottom "0";tempfilename
'LU1FPS00.wmf';tempfile-properties "XPR";}}It would look like this \FRAME{%
dtbpFX}{4.4996in}{1.3681in}{0pt}{}{}{Plot}{\special{language "Scientific
Word";type "MAPLEPLOT";width 4.4996in;height 1.3681in;depth 0pt;display
"USEDEF";plot_snapshots TRUE;mustRecompute FALSE;lastEngine "MuPAD";xmin
"-40";xmax "40";xviewmin "-40";xviewmax "40";yviewmin
"-2.664533E-7";yviewmax "2.664533E-7";viewset"XY";rangeset"X";plottype
4;labeloverrides 3;x-label "x (m)";y-label "V (V)";axesFont "Times New
Roman,12,0000000000,useDefault,normal";numpoints 100;plotstyle
"patch";axesstyle "normal";axestips FALSE;xis \TEXUX{x};var1name
\TEXUX{$x$};function \TEXUX{$\left( \frac{9.99\times 10^{-9}\left( 2\right)
}{\sqrt{\left( \frac{x}{10}+1\right) ^{2}+\left( \frac{0}{10}\right)
^{2}}+1\times 10^{-9}}+\frac{9.99\times 10^{-9}\left( -2\right)
}{\sqrt{\left( \frac{x}{10}-1\right) ^{2}+\left( \frac{0}{10}\right)
^{2}}+1\times 10^{-9}}\right) $};linecolor "blue";linestyle 1;pointstyle
"point";linethickness 1;lineAttributes "Solid";var1range
"-40,40";num-x-gridlines 100;curveColor "[flat::RGB:0x000000ff]";curveStyle
"Line";VCamFile 'M5GT4I05.xvz';valid_file "T";tempfilename
'M5GT4I06.wmf';tempfile-properties "XPR";}}The charges $\left( q=2\times
10^{-9}\unit{C}\right) $ were placed right at $x=\pm 10\unit{m}.$ The
potential 
\begin{equation*}
V=\frac{1}{4\pi \epsilon _{o}}\left( \frac{q_{1}}{r_{1}}+\frac{q_{2}}{r_{2}}%
\right)
\end{equation*}%
becomes large near $r_{1}=R_{o}$ or $r_{2}=R_{o}$ where $R_{o}$ is the
charge radius (which is very small, since these are point charges). Plotting
the potential in two dimensions is also interesting. We see that near the
positive charge we have a tall mountain-like potential and near the negative
charge we have a deep well-like potential.

\FRAME{dtbpFX}{3.7222in}{2.5573in}{0pt}{}{}{Plot}{\special{language
"Scientific Word";type "MAPLEPLOT";width 3.7222in;height 2.5573in;depth
0pt;display "USEDEF";plot_snapshots TRUE;mustRecompute FALSE;lastEngine
"MuPAD";xmin "-50";xmax "50";ymin "-50";ymax "50";xviewmin "-50";xviewmax
"50";yviewmin "-50";yviewmax "50";zviewmin "-2.664000E-8";zviewmax
"2.664000E-8";viewset"XYZ";rangeset"XYZ";phi 66;theta -37;cameraDistance
"1.01878";cameraOrientation "[0,0,3.0586e+006]";cameraOrientationFixed
TRUE;plottype 5;labeloverrides 7;x-label "x (m)";y-label "y (m)";z-label "V
(V)";axesFont "Times New
Roman,12,0000000000,useDefault,normal";num-x-gridlines 25;num-y-gridlines
25;plotstyle "patch";axesstyle "normal";axestips FALSE;plotshading
"XYZ";lighting 0;xis \TEXUX{x};yis \TEXUX{y};var1name \TEXUX{$x$};var2name
\TEXUX{$y$};function \TEXUX{$\left( \frac{9.99\times 10^{-9}\left( 2\right)
}{\sqrt{\left( \frac{x}{10}+1\right) ^{2}+\left( \frac{y}{10}\right)
^{2}}+1\times 10^{-9}}+\frac{9.99\times 10^{-9}\left( -2\right)
}{\sqrt{\left( \frac{x}{10}-1\right) ^{2}+\left( \frac{y}{10}\right)
^{2}}+1\times 10^{-9}}\right) $};linestyle 1;pointstyle
"point";linethickness 1;lineAttributes "Solid";var1range "-50,50";var2range
"-50,50";surfaceColor "[linear:XYZ:RGB:0x990000ff:0xab800000]";surfaceStyle
"Color Patch";num-x-gridlines 25;num-y-gridlines 25;surfaceMesh
"Mesh";function \TEXUX{$\left( \frac{9.99\times 10^{-9}\left( 2\right)
}{\sqrt{\left( \frac{x}{10}+1\right) ^{2}+\left( \frac{y}{10}\right)
^{2}}+1\times 10^{-9}}+\frac{9.99\times 10^{-9}\left( -2\right)
}{\sqrt{\left( \frac{x}{10}-1\right) ^{2}+\left( \frac{y}{10}\right)
^{2}}+1\times 10^{-9}}\right) $};linestyle 1;pointstyle
"point";linethickness 1;lineAttributes "Solid";var1range "-50,50";var2range
"-50,50";surfaceColor "[flat::RGB:0000000000:0000000000]";surfaceStyle "Wire
Frame";num-x-gridlines 25;num-y-gridlines 25;surfaceMesh "Contour";VCamFile
'MJ56Y80A.xvz';valid_file "T";tempfilename
'MJ56XE01.wmf';tempfile-properties "XPR";}}Notice the equipotential lines.
The more red peak is the positive charge (hill), the more blue the negative
charge (valley).A view from farther away looks like this\FRAME{dtbpFX}{%
3.7222in}{2.5573in}{0pt}{}{}{Plot}{\special{language "Scientific Word";type
"MAPLEPLOT";width 3.7222in;height 2.5573in;depth 0pt;display
"USEDEF";plot_snapshots TRUE;mustRecompute FALSE;lastEngine "MuPAD";xmin
"-70";xmax "70";ymin "-70";ymax "70";xviewmin "-70";xviewmax "70";yviewmin
"-70";yviewmax "70";zviewmin "-9.664000E-8";zviewmax
"9.664000E-8";viewset"XYZ";rangeset"XYZ";phi 65;theta -34;cameraDistance
"1.14313";cameraOrientation "[0,0,3.0586e+006]";cameraOrientationFixed
TRUE;plottype 5;labeloverrides 7;x-label "x (m)";y-label "y (m)";z-label "V
(V)";axesFont "Times New
Roman,12,0000000000,useDefault,normal";num-x-gridlines 25;num-y-gridlines
25;plotstyle "patch";axesstyle "normal";axestips FALSE;plotshading
"XYZ";lighting 0;xis \TEXUX{x};yis \TEXUX{y};var1name \TEXUX{$x$};var2name
\TEXUX{$y$};function \TEXUX{$\left( \frac{9.99\times 10^{-9}\left( 2\right)
}{\sqrt{\left( \frac{x}{10}+1\right) ^{2}+\left( \frac{y}{10}\right)
^{2}}+1\times 10^{-9}}+\frac{9.99\times 10^{-9}\left( -2\right)
}{\sqrt{\left( \frac{x}{10}-1\right) ^{2}+\left( \frac{y}{10}\right)
^{2}}+1\times 10^{-9}}\right) $};linestyle 1;pointstyle
"point";linethickness 1;lineAttributes "Solid";var1range "-70,70";var2range
"-70,70";surfaceColor "[linear:XYZ:RGB:0x990000ff:0xab800000]";surfaceStyle
"Color Patch";num-x-gridlines 25;num-y-gridlines 25;surfaceMesh
"Mesh";function \TEXUX{$\left( \frac{9.99\times 10^{-9}\left( 2\right)
}{\sqrt{\left( \frac{x}{10}+1\right) ^{2}+\left( \frac{y}{10}\right)
^{2}}+1\times 10^{-9}}+\frac{9.99\times 10^{-9}\left( -2\right)
}{\sqrt{\left( \frac{x}{10}-1\right) ^{2}+\left( \frac{y}{10}\right)
^{2}}+1\times 10^{-9}}\right) $};linestyle 1;pointstyle
"point";linethickness 1;lineAttributes "Solid";var1range "-70,70";var2range
"-70,70";surfaceColor "[flat::RGB:0000000000:0000000000]";surfaceStyle "Wire
Frame";num-x-gridlines 25;num-y-gridlines 25;surfaceMesh "Contour";VCamFile
'MJ57050D.xvz';valid_file "T";tempfilename
'MJ570502.wmf';tempfile-properties "XPR";}}Of course the hill and the valley
both approach an infinity at the point charge because of the $1/r$
dependence.

\subsection{Lots of point charges}

%TCIMACRO{%
%\TeXButton{Question 223.31.4}{\marginpar {
%\hspace{-0.5in}
%\begin{minipage}[t]{1in}
%\small{Question 223.31.4}
%\end{minipage}
%}}}%
%BeginExpansion
\marginpar {
\hspace{-0.5in}
\begin{minipage}[t]{1in}
\small{Question 223.31.4}
\end{minipage}
}%
%EndExpansion
Suppose we have many point charges. What is the potential of the group? We
just use superposition and add up the contribution of each point charge%
\begin{equation}
V=\frac{1}{4\pi \epsilon _{o}}\dsum\limits_{i}\frac{q_{i}}{r_{i}}
\end{equation}%
where $r_{i}$ is the distance from the point charge $q_{i}$ to the point of
interest (where we wish to know the potential). Note that this is easier
than adding up the electric field contributions. Electric potentials are not
vectors! They just add as scalars.

\section{Potential of groups of charges}

Suppose we have a continuous distribution of charge. Of course, this would
be made of many, many point charges, but if we have so many point charges
that the distance between the individual charges is negligible, we can treat
them as one continuous thing. If we know the charge distribution we can just
interpret the distribution as a set of small amounts of charge $dq$ acting
like point charges all arranged into some shape. \FRAME{dhF}{1.9285in}{%
1.2842in}{0pt}{}{}{Figure}{\special{language "Scientific Word";type
"GRAPHIC";maintain-aspect-ratio TRUE;display "USEDEF";valid_file "T";width
1.9285in;height 1.2842in;depth 0pt;original-width 1.8905in;original-height
1.2488in;cropleft "0";croptop "1";cropright "1";cropbottom "0";tempfilename
'LU0KA301.wmf';tempfile-properties "XPR";}}Then for each charge $dq$ we will
have a small amount of potential%
\begin{equation}
dV=\frac{1}{4\pi \epsilon _{o}}\frac{dq}{r}
\end{equation}%
and the total potential at some point will be the summation of all these
small amounts of charge 
\begin{equation}
V=\frac{1}{4\pi \epsilon _{o}}\int \frac{dq}{r}
\end{equation}%
This looks a little like our integral for finding the electric field from a
configuration of charge, but there is one large difference. There is no
vector nature to this integral. So our procedure will have one less step

\begin{itemize}
\item Start with $V=\frac{1}{4\pi \epsilon _{o}}\int \frac{dq}{r}$

\item find an expression for $dq$

\item Use geometry to find an expression for $r,$ the distance from the
group of charges, $dq,$ and the point $P$

\item Solve the integral
\end{itemize}

Let's try one together

\subsection{Electric potential due to a uniformly charged disk}

We have found the field due to a charged disk. We can use our summation of
the potential due to small packets of charge to find the electric potential
of an entire charged disk.\FRAME{dtbpF}{4.0571in}{2.3682in}{0in}{}{}{Figure}{%
\special{language "Scientific Word";type "GRAPHIC";maintain-aspect-ratio
TRUE;display "USEDEF";valid_file "T";width 4.0571in;height 2.3682in;depth
0in;original-width 4.1058in;original-height 2.3842in;cropleft "0";croptop
"1";cropright "1";cropbottom "0";tempfilename
'NPZYDH09.wmf';tempfile-properties "XPR";}}

Suppose we have a uniform charge density $\eta $ on the disk, and a total
charge $Q$, with a disk radius $a.$ We wish to find the potential at some
point $P$ along the central axis.

To do this problem let's divide up the disk into small areas, $dA$ each with
a small amount of charge, $dq.$ 
%TCIMACRO{%
%\TeXButton{Question 223.31.5}{\marginpar {
%\hspace{-0.5in}
%\begin{minipage}[t]{1in}
%\small{Question 223.31.5}
%\end{minipage}
%}} }%
%BeginExpansion
\marginpar {
\hspace{-0.5in}
\begin{minipage}[t]{1in}
\small{Question 223.31.5}
\end{minipage}
}
%EndExpansion
The area element is 
\begin{equation*}
dA=Rd\phi dR
\end{equation*}%
so the charge element, $dq,$ is 
\begin{equation*}
dq=\eta Rd\phi dR
\end{equation*}

For each $dq$ we have a small part of the total potential. The variable $r$
is the distance from our small group of charges that we called $dq$ to the
point $P.$ Then $r=\sqrt{R^{2}+z^{2}}$ and our integral becomes%
\begin{eqnarray*}
V &=&\frac{1}{4\pi \epsilon _{o}}\int \frac{dq}{r} \\
&=&\frac{1}{4\pi \epsilon _{o}}\int \int \frac{\eta Rd\phi dR}{\sqrt{%
R^{2}+z^{2}}}
\end{eqnarray*}

We will integrate this. We will integrate over $r$ from $0$ to $a$ and $\phi 
$ from $0$ to $2\pi $ which will account for all the charge on the disk, and
therefore all the potential.

\begin{eqnarray}
V &=&\frac{1}{4\pi \epsilon _{o}}\int_{0}^{2\pi }\int_{0}^{a}\frac{\eta
Rd\phi dR}{\sqrt{R^{2}+z^{2}}} \\
&=&\frac{\eta 2\pi }{4\pi \epsilon _{o}}\int_{0}^{a}\frac{RdR}{\sqrt{%
R^{2}+z^{2}}}  \notag \\
&=&\frac{\eta 2\pi }{4\pi \epsilon _{o}}\allowbreak \left. \sqrt{R^{2}+z^{2}}%
\right\vert _{0}^{a}  \notag \\
&=&\frac{\eta 2\pi }{4\pi \epsilon _{o}}\allowbreak \sqrt{a^{2}+z^{2}}-\frac{%
\eta 2\pi }{4\pi \epsilon _{o}}\allowbreak z  \notag
\end{eqnarray}%
so%
\begin{equation}
V=\frac{\eta }{2\epsilon _{o}}\left( \allowbreak \sqrt{a^{2}+z^{2}}%
-\allowbreak z\right)
\end{equation}%
This is the potential at point $P.$

We compared our electric field solutions with the solution for a point
charge. We can do the same for electric potentials. We can compare our
solution to a point charge potential for an equal amount of charge. Far away
from the disk, we expect the two potentials to look the same. The point
charge equation is%
\begin{equation*}
V=\frac{Q}{4\pi \epsilon _{o}}\frac{1}{z}
\end{equation*}%
Our disk gives%
\begin{equation}
V=\frac{Q}{4\epsilon _{o}\pi }\frac{2}{a^{2}}\left( \allowbreak \sqrt{%
a^{2}+z^{2}}-\allowbreak z\right)
\end{equation}%
They don't look much alike! But plotting both yields

\FRAME{dtbpFX}{4.4997in}{1.9984in}{0pt}{}{}{Plot}{\special{language
"Scientific Word";type "MAPLEPLOT";width 4.4997in;height 1.9984in;depth
0pt;display "USEDEF";plot_snapshots TRUE;mustRecompute FALSE;lastEngine
"MuPAD";xmin "0";xmax "0.0500010";xviewmin "0";xviewmax "0.0500010";yviewmin
"-3.174501E-13";yviewmax "8.775500E-12";viewset"XY";rangeset"X";plottype
4;labeloverrides 3;x-label "z(m)";y-label "V(V)";axesFont "Times New
Roman,12,0000000000,useDefault,normal";numpoints 100;plotstyle
"patch";axesstyle "normal";axestips FALSE;xis \TEXUX{z};var1name
\TEXUX{$z$};function \TEXUX{$\frac{2}{4\pi \left( 8.85\times 10^{12}\right)
}\frac{1}{z}$};linecolor "blue";linestyle 2;pointstyle "point";linethickness
3;lineAttributes "Dash";var1range "0,0.0500010";num-x-gridlines
100;curveColor "[flat::RGB:0x000000ff]";curveStyle "Line";function
\TEXUX{$\frac{2}{4\pi \left( 8.85\times 10^{12}\right)
}\frac{2}{0.005^{2}}\left( \allowbreak \sqrt{\left( 0.005\right)
^{2}+z^{2}}-\allowbreak z\right) $};linecolor "red";linestyle 1;pointstyle
"point";linethickness 3;lineAttributes "Solid";var1range
"0,0.0500010";num-x-gridlines 100;curveColor
"[flat::RGB:0x00ff0000]";curveStyle "Line";VCamFile
'LU0NS90T.xvz';valid_file "T";tempfilename
'LU0NEX08.wmf';tempfile-properties "XPR";}}The dashed line is the point
charge, the solid line is our disk with a radius of $0.05\unit{m}$ and a
total charge of $2\unit{C}.$ This shows that far from the disk the potential
is like a point charge, but close the two are quite different as we would
expect. This is a reasonable result.

We will calculate the potential due to several continuous charge
configurations.

But, you may ask, since we knew the field for the disk of charge, couldn't
we have found the electric potential from our equation of the field? We will
take up this question in the next two lectures.

%TCIMACRO{%
%\TeXButton{Basic Equations}{\hspace{-1.3in}{\LARGE Basic Equations\vspace{0.25in}}}}%
%BeginExpansion
\hspace{-1.3in}{\LARGE Basic Equations\vspace{0.25in}}%
%EndExpansion

The electric potential of a point charge is given by%
\begin{equation*}
V=\frac{1}{4\pi \epsilon _{o}}\frac{Q}{r}
\end{equation*}%
where the zero potential point is set at $r=\infty .$

Electric potentials simply add, so the potential for a collection of point
charges is just%
\begin{equation*}
V=\frac{1}{4\pi \epsilon _{o}}\dsum\limits_{i}\frac{q_{i}}{r_{i}}
\end{equation*}

To find the potential due to a continuous distribution of charge we use the
following procedure:

\begin{itemize}
\item Start with $V=\frac{1}{4\pi \epsilon _{o}}\int \frac{dq}{r}$

\item find an expression for $dq$

\item Use geometry to find an expression for $r$

\item Solve the integral
\end{itemize}

Since electric fields and electric potentials are both representations of
the environment created by the environmental charge, there must be a way to
calculate the potential from the field and \emph{vice versa}. It will take
us two lectures to do both.

\chapter{Connecting potential and field}

%TCIMACRO{%
%\TeXButton{Fundamental Concepts}{\hspace{-1.3in}{\LARGE Fundamental Concepts\vspace{0.25in}}}}%
%BeginExpansion
\hspace{-1.3in}{\LARGE Fundamental Concepts\vspace{0.25in}}%
%EndExpansion

\begin{itemize}
\item The potential and the field are manifestations of the same physical
thing

\item We find the potential from the field using $\Delta V=-\int 
\overrightarrow{\mathbf{E}}\cdot d\overrightarrow{\mathbf{s}}$

\item Fields and potentials come from separated charge
\end{itemize}

\section{Finding the potential knowing the field}

It is time to pause and think about the meaning of this electric potential.
Let's trace our steps backwards. We defined the electric potential as the
potential energy per unit charge:%
%TCIMACRO{%
%\TeXButton{Question 223.32.1}{\marginpar {
%\hspace{-0.5in}
%\begin{minipage}[t]{1in}
%\small{Question 223.32.1}
%\end{minipage}
%}}}%
%BeginExpansion
\marginpar {
\hspace{-0.5in}
\begin{minipage}[t]{1in}
\small{Question 223.32.1}
\end{minipage}
}%
%EndExpansion
\begin{equation*}
\Delta V=\frac{\Delta U}{q}
\end{equation*}%
where $q$ is our mover and $\Delta V$ is a measure of the change in the
environment between two points $r_{1}$ and $r_{2}$ measured from the
environmental charge. $\Delta U$ is the change in potential energy as $q$
moves. 
%TCIMACRO{%
%\TeXButton{Question 223.32.2}{\marginpar {
%\hspace{-0.5in}
%\begin{minipage}[t]{1in}
%\small{Question 223.32.2}
%\end{minipage}
%}} }%
%BeginExpansion
\marginpar {
\hspace{-0.5in}
\begin{minipage}[t]{1in}
\small{Question 223.32.2}
\end{minipage}
}
%EndExpansion
But the potential energy change is equal to the negative of the amount of
work we have done in moving $q$%
\begin{equation*}
\Delta V=\frac{-W}{q}
\end{equation*}%
which is equal to 
\begin{equation*}
\Delta V=\frac{-1}{q}\int \overrightarrow{\mathbf{F}}\cdot d\overrightarrow{%
\mathbf{s}}
\end{equation*}%
where again $d\overrightarrow{\mathbf{s}}$ is a general path length. But
this force was a Coulomb force. which we know is related to the electric
field%
\begin{equation*}
\overrightarrow{\mathbf{E}}=\frac{\overrightarrow{\mathbf{F}}}{q}
\end{equation*}%
so we may rewrite the potential as 
\begin{eqnarray*}
\Delta V &=&-\int \frac{\overrightarrow{\mathbf{F}}}{q}\cdot d%
\overrightarrow{\mathbf{s}} \\
&=&-\int \overrightarrow{\mathbf{E}}\cdot d\overrightarrow{\mathbf{s}}
\end{eqnarray*}%
which we found last lecture by analogy with our capacitor potential. Our
line of reasoning in this lecture has been more formal, but we arrive at the
same conclusion--\textbf{and it is an important one!}%
%TCIMACRO{%
%\TeXButton{Question 223.32.3}{\marginpar {
%\hspace{-0.5in}
%\begin{minipage}[t]{1in}
%\small{Question 223.32.3}
%\end{minipage}
%}} }%
%BeginExpansion
\marginpar {
\hspace{-0.5in}
\begin{minipage}[t]{1in}
\small{Question 223.32.3}
\end{minipage}
}
%EndExpansion
If we add up the component of field magnitude times the displacement along
the path take from $r_{1}$ to $r_{2}$ we get the electric potential (well,
minus the electric potential).

The electric field and the electric potential are not two distinct things.
They are really different ways to look at the same thing--and that thing is
the environment set up by the environmental charge. It is tradition to say
the electric field is the principal quantity. This is because we have good
evidence that the electric field \emph{is} something. That evidence we will
study at the end of these lectures, but in a nutshell it is that we can make
waves in the electric field. If we can make waves in it, it must be
something!\footnote{%
By the end of these lectures, we will try to make this a more convincing
(and more mathmatical) statement!}

in our gravitational analogy, the gravitational field is the real thing.
Gravitational potential energy is a result of the gravitational field being
there. The change in potential energy is an amount of work, and the
gravitational force is what does the work. No force, no potential energy.
The gravitational field makes that force happen.

It is the same for our electrical force. The electrical potential is due to
the Coulomb force, and the Coulomb force exists because the electric field
is there.

If the field and the potential are really different manifestations of the
same thing, we should be able to find one from the other. We have one way to
do this. We can find the potential from the field, but we should be able to
find the field from the potential. We will practice the first%
\begin{equation*}
\Delta V=-\int \overrightarrow{\mathbf{E}}\cdot d\overrightarrow{\mathbf{s}}
\end{equation*}%
today, and then introduce how to find the field from the potential next
lecture.

\subsection{Finding the potential from the field.}

Actually we did an example last lecture. We found the field of a point
charge. But let's take on some harder examples in this lecture.\FRAME{dhF}{%
1.5194in}{1.7367in}{0pt}{}{}{Figure}{\special{language "Scientific
Word";type "GRAPHIC";maintain-aspect-ratio TRUE;display "USEDEF";valid_file
"T";width 1.5194in;height 1.7367in;depth 0pt;original-width
3.0228in;original-height 3.4592in;cropleft "0";croptop "1";cropright
"1";cropbottom "0";tempfilename 'LU3Q5V03.wmf';tempfile-properties "XPR";}}

Let's calculate the electric potential do to an infinite line of charge.
This is like the potential due to a charged wire. We already found the field
due to an infinite line of charge 
\begin{equation*}
E=\frac{1}{4\pi \epsilon _{o}}\frac{2\lambda }{r}\mathbf{\hat{r}}
\end{equation*}%
so we can use this to find the potential difference.%
\begin{equation*}
\Delta V=-\int_{A}^{B}\overrightarrow{\mathbf{E}}\cdot d\overrightarrow{%
\mathbf{s}}
\end{equation*}%
We need $d\overrightarrow{\mathbf{s}}.$ Of course $d\overrightarrow{\mathbf{s%
}}$ could be in any direction. We can take components in cylindrical
coordinates%
\begin{equation*}
d\overrightarrow{\mathbf{s}}=dr\mathbf{\hat{r}}+rd\theta \mathbf{\hat{\theta}%
}+dz\mathbf{\hat{z}}
\end{equation*}%
Putting in our field gives%
\begin{eqnarray*}
\Delta V &=&-\int_{A}^{B}\frac{1}{4\pi \epsilon _{o}}\frac{2\lambda }{r}%
\mathbf{\hat{r}}\cdot \left( dr\mathbf{\hat{r}}+rd\theta \mathbf{\hat{\theta}%
}+dz\mathbf{\hat{z}}\right) \\
&=&-\frac{2\lambda }{4\pi \epsilon _{o}}\int_{A}^{B}\frac{dr}{r}
\end{eqnarray*}%
which we can integrate%
\begin{eqnarray*}
\Delta V &=&\left( -\frac{1}{2\pi }\frac{\lambda }{\epsilon _{o}}\ln
r_{B}-\left( -\frac{1}{2\pi }\frac{\lambda }{\epsilon _{o}}\ln r_{A}\right)
\right) \\
&=&-\frac{1}{2\pi }\frac{\lambda }{\epsilon _{o}}\left( \ln r_{B}-\ln
r_{A}\right)
\end{eqnarray*}%
This example gives us a chance to think about our simple geometries and to
consider when they are reasonable approximations to real charged objects. So
long as neither $r_{A}$ nor $r_{B}$ are infinite, this result is reasonable.
But remember what it looks like to move away from an infinite line of
charge. No matter how far away we go, the line is still infinite. So we
never get very far away. The terms 
\begin{equation*}
V_{A}=\frac{1}{2\pi }\frac{\lambda }{\epsilon _{o}}\left( \ln r_{A}\right)
\end{equation*}%
or%
\begin{equation*}
V_{B}=\frac{1}{2\pi }\frac{\lambda }{\epsilon _{o}}\left( \ln r_{B}\right)
\end{equation*}%
would look something like this\FRAME{dtbpFX}{4.4996in}{2.207in}{0pt}{}{}{Plot%
}{\special{language "Scientific Word";type "MAPLEPLOT";width 4.4996in;height
2.207in;depth 0pt;display "USEDEF";plot_snapshots TRUE;mustRecompute
FALSE;lastEngine "MuPAD";xmin "-5";xmax "5";xviewmin
"-0.000400359116796";xviewmax "5.00049999053696";yviewmin
"-1.60969046264188";yviewmax "0.91631415606388";plottype 4;labeloverrides
3;x-label "r(m)";y-label "V(lambda/(2*pi*eps))";axesFont "Times New
Roman,12,0000000000,useDefault,normal";numpoints 100;plotstyle
"patch";axesstyle "normal";axestips FALSE;xis \TEXUX{r};var1name
\TEXUX{$r$};function \TEXUX{$-\ln r$};linecolor "blue";linestyle
1;pointstyle "point";linethickness 3;lineAttributes "Solid";var1range
"-5,5";num-x-gridlines 100;curveColor "[flat::RGB:0x000000ff]";curveStyle
"Line";VCamFile 'LYTOTN01.xvz';valid_file "T";tempfilename
'LYTOTN00.wmf';tempfile-properties "XPR";}}The curve is definitely not
approaching zero as $r$ gets large. No matter how far we get from an
infinite line of charge, we really never get very far compared with it's
infinite length. So the potential is not going to zero!

Our solution is good only when $r_{A}$ and $r_{B}$ are much smaller than the
length of the line. that is, when our simple geometry is a good
representation for something that is real, in this case, a finite length
wire. But for $r_{A,}r_{B}\ll L$ this works.

We should also pause to think of the implications of this result for
electronic equipment design. Our result means that adjacent wires in a cable
or on a circuit board will feel a potential due to their
neighbors--something we have to take into consideration in the design to
ensure your equipment will work! This is one reason why we use shielded
cables for delicate instruments, and for data lines, etc.

As a second example, let's tackle our friendly capacitor problem again. What
is the potential difference as we cross the capacitor from point $A$ to
point $B$? We already know the answer 
\begin{equation*}
\Delta V=Ed
\end{equation*}%
But when we found this before, we assumed we knew the potential energy. This
time let's practice using 
\begin{equation*}
\Delta V=-\int_{A}^{B}\overrightarrow{\mathbf{E}}\cdot d\overrightarrow{%
\mathbf{s}}
\end{equation*}%
\FRAME{dhF}{2.4232in}{2.9136in}{0pt}{}{}{Figure}{\special{language
"Scientific Word";type "GRAPHIC";maintain-aspect-ratio TRUE;display
"USEDEF";valid_file "T";width 2.4232in;height 2.9136in;depth
0pt;original-width 3.3088in;original-height 3.9825in;cropleft "0";croptop
"1";cropright "1";cropbottom "0";tempfilename
'LU3TJN04.wmf';tempfile-properties "XPR";}} We know the field is%
\begin{equation*}
E=\frac{\eta }{\epsilon _{o}}
\end{equation*}%
so 
\begin{eqnarray*}
\Delta V &=&-\int_{A}^{B}\overrightarrow{\mathbf{E}}\cdot d\overrightarrow{%
\mathbf{s}} \\
&=&-\int_{A}^{B}\frac{\eta }{\epsilon _{o}}ds\cos \theta
\end{eqnarray*}%
where $\theta $ is the angle between the field direction and our $d%
\overrightarrow{\mathbf{s}}$ direction. We could write 
\begin{equation*}
dx=ds\cos \theta
\end{equation*}%
Then 
\begin{eqnarray*}
\Delta V &=&-\frac{\eta }{\epsilon _{o}}\int_{A}^{B}dx \\
&=&-\frac{\eta }{\epsilon _{o}}\left( x_{B}-x_{A}\right) \\
&=&-\frac{\eta }{\epsilon _{o}}\Delta x
\end{eqnarray*}%
This is just 
\begin{equation*}
\Delta V=-E\Delta x
\end{equation*}%
if we consider the negative side to be the zero potential, and we cross the
entire capacitor, then 
\begin{eqnarray*}
\Delta V &=&-E\left( x_{B}-x_{A}\right) \\
&=&-E\left( 0-d\right) \\
&=&Ed
\end{eqnarray*}%
as we expect. Note that we can now see how the positive result comes from
our choice of the zero voltage point.

\section{Sources of electric potential}

%TCIMACRO{%
%\TeXButton{Question 223.32.4}{\marginpar {
%\hspace{-0.5in}
%\begin{minipage}[t]{1in}
%\small{Question 223.32.4}
%\end{minipage}
%}}}%
%BeginExpansion
\marginpar {
\hspace{-0.5in}
\begin{minipage}[t]{1in}
\small{Question 223.32.4}
\end{minipage}
}%
%EndExpansion
We know that the electric potential comes from the electric field. And if we
think about it, we know where the electric field comes from, charge. But we
have found that equal amounts of positive and negative charge produce no net
field. So normal matter does not seem to have any net electric field because
the protons and electrons create oppositely directed fields, with no net
result.

But if we separate the positive and negative charges, we do get a field.
This is the source of all electric fields that we see, and therefore all
electric potentials are due to separated charge.

We have used charge separation devices already in our lectures. Rubbing a
rubber rod with rabbit fur transfers the electrons from the fur to the rod.
Some of the charges that were balanced in the fur are now separated. So
there is an electric field that creates an electric force. Then there must
be an electric potential, since the potential is just a manifestation of the
field.

We have also used a van de Graaff generator. It is time to see how this
works.\FRAME{dhF}{2.3372in}{2.5563in}{0in}{}{}{Figure}{\special{language
"Scientific Word";type "GRAPHIC";maintain-aspect-ratio TRUE;display
"USEDEF";valid_file "T";width 2.3372in;height 2.5563in;depth
0in;original-width 2.3532in;original-height 2.5767in;cropleft "0";croptop
"1";cropright "1";cropbottom "0";tempfilename
'LU4BI60D.wmf';tempfile-properties "XPR";}}

In the base of the van de Graaff, there is a small electrode. It is charged
to a large voltage, and charge leaks off through the air to a rubber belt
that is very close. The rubber belt is connected to a motor. The motor turns
the belt. The extra charge is stuck on the belt, since the belt is not a
conductor. The charge is carried up to the top where there is a large round
electrode. A conducting brush touches the rubber belt, and the charge is
able to escape the belt through the conductor. The charge spreads over the
whole spherical electrode surface.

The belt keeps providing charge. Of course the new charge is repelled by the
charge all ready accumulated on the spherical electrode, so we must do work
to keep the belt turning and the charge ascending to the ball at the top.
This is a mechanical charge separation device. It can easily build potential
differences between the spherical top and the surrounding environment
(including you) of $30000\unit{V}$.

Much larger versions of this device are used to accelerate sub atomic
particles to very high speeds.

\section{Electrochemical separation of charge}

%TCIMACRO{%
%\TeXButton{Question 223.32.5}{\marginpar {
%\hspace{-0.5in}
%\begin{minipage}[t]{1in}
%\small{Question 223.32.5}
%\end{minipage}
%}}}%
%BeginExpansion
\marginpar {
\hspace{-0.5in}
\begin{minipage}[t]{1in}
\small{Question 223.32.5}
\end{minipage}
}%
%EndExpansion
When you eat table salt, the NaCl ionic bond splits when exposed to polar
water molecules, leaving a positively charged Na ion and a negatively
charged Cl ion. This is very like the \textquotedblleft
bleeding\textquotedblright\ of charge from our charged balloons that we
talked about earlier. We already know that the water molecules are polar,
and the mostly positive hydrogens are attracted to the negatively charged Cl
ions. This causes a sort of tug-o-war for the Cl ions. The positively
charged Na ions pull with their coulomb force, and so do the positively
charged hydrogens of the water molecules. If we have lots of water
molecules, they win and the NaCl is broken apart. Water molecules are polar,
but overall neutral. But now, with the Na and Cl ions, we have separated
charge. We can make this charge flow, so we can get electric currents in our
bodies. Our nervous system uses the positively charged Na ions to form tiny
currents into and out of neurons as part of how nerve signaling works. Of
course, NaCl is a pretty simple molecule. We could use more complex chemical
reactions to separate charge.

\section{batteries and emf}

Most of us don't have a van de Graaff generator in our pockets. But most of
us do have a charge separation device that we carry around with us. We call
it a battery. But what does this battery do?

Somehow the battery supples positive charge on one side and negative charge
on the other side. This is accomplished by doing work on the charges. A lead
acid battery is often used in automobiles. The battery is made by suspending
two lead plates in a solution of sulfuric acid and water. \FRAME{dtbpF}{%
3.0415in}{1.8092in}{0in}{}{}{Figure}{\special{language "Scientific
Word";type "GRAPHIC";maintain-aspect-ratio TRUE;display "USEDEF";valid_file
"T";width 3.0415in;height 1.8092in;depth 0in;original-width
2.9983in;original-height 1.772in;cropleft "0";croptop "1";cropright
"1";cropbottom "0";tempfilename 'MJ77O20C.wmf';tempfile-properties "XPR";}}%
One plate is coated with lead dioxide. There is a chemical reaction at each
plate. The sulfuric acid (H$_{2}$SO$_{4})$ splits into two H$^{+}$ ions and
an SO$_{4}^{-2}$ ion. \FRAME{dtbpF}{3.141in}{1.9095in}{0pt}{}{}{Figure}{%
\special{language "Scientific Word";type "GRAPHIC";maintain-aspect-ratio
TRUE;display "USEDEF";valid_file "T";width 3.141in;height 1.9095in;depth
0pt;original-width 3.8285in;original-height 2.3168in;cropleft "0";croptop
"1";cropright "1";cropbottom "0";tempfilename
'MJ77PI0D.wmf';tempfile-properties "XPR";}}The plain lead plate reacts with
the SO$_{4}^{-2}$ ions. \FRAME{dtbpF}{2.1698in}{2.2105in}{0pt}{}{}{Figure}{%
\special{language "Scientific Word";type "GRAPHIC";maintain-aspect-ratio
TRUE;display "USEDEF";valid_file "T";width 2.1698in;height 2.2105in;depth
0pt;original-width 2.6775in;original-height 2.7268in;cropleft "0";croptop
"1";cropright "1";cropbottom "0";tempfilename
'MJ78020J.wmf';tempfile-properties "XPR";}}The overall reaction is%
\begin{equation*}
Pb\left( \text{solid}\right) +H_{2}SO_{4}^{-}\left( \text{aqueous}\right)
\rightarrow PbSO_{4}\left( \text{solid}\right) +2H^{+}\left( \text{aqueous}%
\right) +2e^{-}\left( \text{in conductor}\right)
\end{equation*}%
producing lead sulfate on the electrode, some hydrogen ions in solution and
some extra electrons that are left in the metal plate.\FRAME{dtbpF}{2.5157in%
}{2.5296in}{0in}{}{}{Figure}{\special{language "Scientific Word";type
"GRAPHIC";maintain-aspect-ratio TRUE;display "USEDEF";valid_file "T";width
2.5157in;height 2.5296in;depth 0in;original-width 2.4751in;original-height
2.4881in;cropleft "0";croptop "1";cropright "1";cropbottom "0";tempfilename
'MJ77TH0F.wmf';tempfile-properties "XPR";}}

The coated plate's lead dioxide also reacts with the SO$_{4}^{-2}$ ions and
uses the hydrogen ions and the oxygen from the PbO$_{2}$ coating. \FRAME{%
dtbpF}{3.0173in}{2.7328in}{0in}{}{}{Figure}{\special{language "Scientific
Word";type "GRAPHIC";maintain-aspect-ratio TRUE;display "USEDEF";valid_file
"T";width 3.0173in;height 2.7328in;depth 0in;original-width
2.9741in;original-height 2.6904in;cropleft "0";croptop "1";cropright
"1";cropbottom "0";tempfilename 'MJ77W60G.wmf';tempfile-properties "XPR";}}%
It also uses some some electrons from the lead plate. The PbO$_{2}$ splits
apart and the Pb$^{+4}$ combines with the SO$_{4}^{-2}$ and the two
electrons. The left over O$_{2}$ combines with the hydrogens to form water.
The reaction equation is 
\begin{equation*}
PbO\left( \text{solid}\right) +H_{2}SO_{4}^{-}\left( \text{aqueous}\right)
+2H^{+}\left( \text{solid}\right) +2e^{-}\left( \text{in conductor}\right)
\rightarrow PbSO_{4}\left( \text{solid}\right) +2H_{2}O\left( \text{liquid}%
\right)
\end{equation*}%
\FRAME{dtbpF}{3.0303in}{2.7328in}{0in}{}{}{Figure}{\special{language
"Scientific Word";type "GRAPHIC";maintain-aspect-ratio TRUE;display
"USEDEF";valid_file "T";width 3.0303in;height 2.7328in;depth
0in;original-width 2.9871in;original-height 2.6904in;cropleft "0";croptop
"1";cropright "1";cropbottom "0";tempfilename
'MJ77Y90H.wmf';tempfile-properties "XPR";}}So one lead plate has two extra
electrons, and one lacks two electrons. We have separated charge! \FRAME{%
dtbpF}{5.1197in}{3.3434in}{0in}{}{}{Figure}{\special{language "Scientific
Word";type "GRAPHIC";maintain-aspect-ratio TRUE;display "USEDEF";valid_file
"T";width 5.1197in;height 3.3434in;depth 0in;original-width
5.0652in;original-height 3.2984in;cropleft "0";croptop "1";cropright
"1";cropbottom "0";tempfilename 'MJ77ZW0I.wmf';tempfile-properties "XPR";}}%
If we connect a wire between the plates, the extra electrons from one plate
will move to the other plate, and we have formed a current (something we
will discuss in detail later). Lead acid batteries are rechargeable. The
recharging process places an electric potential across the two lead plates,
and this drives the two chemical reactions backwards.

Now that we see that we can use chemistry to separate charge, let's think
about what this means for an electric circuit. 
\begin{equation*}
W_{chem}=\Delta U
\end{equation*}%
That work is equivalent to an amount of potential energy, so we have a
voltage. That voltage due to the separated charge is 
\begin{equation*}
\Delta V=\frac{W_{chem}}{q}
\end{equation*}%
This is not a chemistry class, so we won't memorize the chemical process
that does this. Instead, I would like to give a mechanical analogy. \FRAME{%
dhF}{4.8411in}{1.8024in}{0pt}{}{}{Figure}{\special{language "Scientific
Word";type "GRAPHIC";maintain-aspect-ratio TRUE;display "USEDEF";valid_file
"T";width 4.8411in;height 1.8024in;depth 0pt;original-width
4.9059in;original-height 1.8086in;cropleft "0";croptop "1";cropright
"1";cropbottom "0";tempfilename 'LU460T02.wmf';tempfile-properties "XPR";}}%
If we have water in a tank and we attach a pump to the tank, we can pump the
water to a higher tank. The water would gain potential energy. This is
essentially what a battery does for charge. A battery is sort of a
\textquotedblleft charge pump\textquotedblright\ that takes charge from a
low potential to a high potential.

The water in the upper tank can now be put to work. It could, say, run a
turbine.

\FRAME{dhF}{4.7853in}{2.1376in}{0in}{}{}{Figure}{\special{language
"Scientific Word";type "GRAPHIC";maintain-aspect-ratio TRUE;display
"USEDEF";valid_file "T";width 4.7853in;height 2.1376in;depth
0in;original-width 4.8474in;original-height 2.1492in;cropleft "0";croptop
"1";cropright "1";cropbottom "0";tempfilename
'LU5CFR09.wmf';tempfile-properties "XPR";}}

A battery can do the same. The battery \textquotedblleft
pumps\textquotedblright\ charge to the higher potential. That charge can be
put to work, say, lighting a light bulb.\FRAME{dhF}{2.127in}{1.6906in}{0in}{%
}{}{Figure}{\special{language "Scientific Word";type
"GRAPHIC";maintain-aspect-ratio TRUE;display "USEDEF";valid_file "T";width
2.127in;height 1.6906in;depth 0in;original-width 2.1385in;original-height
1.6941in;cropleft "0";croptop "1";cropright "1";cropbottom "0";tempfilename
'LU487K05.wmf';tempfile-properties "XPR";}}

Of course, we could string plumps together to gain even more potential
energy difference.\FRAME{dhF}{4.379in}{2.9758in}{0pt}{}{}{Figure}{\special%
{language "Scientific Word";type "GRAPHIC";maintain-aspect-ratio
TRUE;display "USEDEF";valid_file "T";width 4.379in;height 2.9758in;depth
0pt;original-width 4.4349in;original-height 3.0033in;cropleft "0";croptop
"1";cropright "1";cropbottom "0";tempfilename
'LU5CHF0A.wmf';tempfile-properties "XPR";}}likewise we can string two
batteries to get a larger electrical potential difference.\FRAME{dhF}{%
2.0569in}{2.3053in}{0pt}{}{}{Figure}{\special{language "Scientific
Word";type "GRAPHIC";maintain-aspect-ratio TRUE;display "USEDEF";valid_file
"T";width 2.0569in;height 2.3053in;depth 0pt;original-width
2.0676in;original-height 2.3203in;cropleft "0";croptop "1";cropright
"1";cropbottom "0";tempfilename 'LU48C606.wmf';tempfile-properties "XPR";}}

If we had more batteries, we would have more potential difference. Each
battery \textquotedblleft pumping\textquotedblright\ the charge up to a
higher potential. Our analogy is not perfect, but it gives some insight into
why stringing batteries together increases the voltage. A television remote
likely uses two $1.5\unit{V}$ batteries for a total potential difference
from the bottom of the first to the top of the last of 
\begin{equation*}
\Delta V=2\times 1.5\unit{V}=3\unit{V}
\end{equation*}

If you have been introduced to Kirchhoff's loop law, you may see this as
familiar. Kirchhoff said that 
\begin{equation*}
\Delta V_{loop}=\sum_{i}\Delta V_{i}=0
\end{equation*}%
That is, if we go around a loop, we should end up at the same potential
where we started. This would be true for our plumbing example. If we start
at the lower tank, then travel through the pump to the upper tank, then
through the turbine to the lower tank we have 
\begin{equation*}
\Delta U_{total}=\Delta U_{pump}+\Delta U_{turbine}=0
\end{equation*}%
we are at the same elevation, we lost all the potential energy we gained by
being pumped up when we fell back down through the turbine.

Similarly, the battery pumps the charge up an amount $\Delta V_{bat}$ and it
\textquotedblleft falls\textquotedblright\ down an amount $\Delta V_{light}$
returning to where it started%
\begin{equation*}
\Delta V_{total}=\Delta V_{bat}+\Delta V_{light}
\end{equation*}%
This is just conservation of energy. As we go around the loop we must
neither create nor destroy energy. We can convert work into potential
through the pump or battery, then we can create movement of water or charge
and even useful work by letting the charge or water \textquotedblleft
fall\textquotedblright\ back down to the initial state. The change in energy
must be zero if there is no loss mechanism. Eventually we must allow some
loss to occur, but for now we have ideal batteries and wires and lights, so
energy is conserved.

We have a historic name for a charge pump like a battery. We call it an 
\emph{emf}. This is pronounced \textquotedblleft ee em
eff,\textquotedblright\ that is, we say the letters. Emf used to stand for
something, but that something has turned out to be a poor model for electric
current, but the letters describing a charge pump persist. This is a little
like Kentucky Fried Chicken changing it's name to KFC because now they bake
chicken (and no one wants to think about eating fried foods now days). The
letters are the name.

Next lecture we will complete our task. In this lecture we discussed finding
the potential if we know the field. Next lecture we will find out how to
calculate the field if we know the potential.

%TCIMACRO{%
%\TeXButton{Basic Equations}{\hspace{-1.3in}{\LARGE Basic Equations\vspace{0.25in}}}}%
%BeginExpansion
\hspace{-1.3in}{\LARGE Basic Equations\vspace{0.25in}}%
%EndExpansion

\chapter{Calculating fields from potentials}

%TCIMACRO{%
%\TeXButton{Fundamental Concepts}{\hspace{-1.3in}{\LARGE Fundamental Concepts\vspace{0.25in}}}}%
%BeginExpansion
\hspace{-1.3in}{\LARGE Fundamental Concepts\vspace{0.25in}}%
%EndExpansion

\begin{itemize}
\item To find the field knowing the potential, we use $\overrightarrow{%
\mathbf{E}}=-\left( \frac{d}{dx}\mathbf{\hat{\imath}}+\frac{d}{dy}\mathbf{%
\hat{\jmath}+}\frac{d}{dz}\mathbf{\hat{k}}\right) V$

\item The gradient shows the direction of steepest change

\item The potential of conductors in equilibrium
\end{itemize}

\section{Finding electric field from the potential}

We did part-one of relating fields to potentials in the last lecture. Now it
is time for part two, obtaining the electric field from a known potential.
Starting with 
\begin{equation*}
\Delta V=-\int_{A}^{B}\overrightarrow{\mathbf{E}}\cdot d\overrightarrow{%
\mathbf{s}}
\end{equation*}%
we realize that we should be able to write the integrand as a small bit of
potential%
\begin{eqnarray*}
dV &=&-\overrightarrow{\mathbf{E}}\cdot d\overrightarrow{\mathbf{s}} \\
&=&-E_{s}ds
\end{eqnarray*}%
where $E_{s}$ is the component of the electric field in the $\mathbf{\hat{s}}
$ direction. We can rearrange this 
\begin{equation*}
E_{s}=-\frac{dV}{ds}
\end{equation*}%
This tells us that the magnitude of our field is the change in electric
potential. Of course, $\overrightarrow{\mathbf{E}}$ is a vector and $V$ is
not. So the best we can do is to get the magnitude of the component in the $%
\overrightarrow{\mathbf{s}}$ direction.

We can try this out on a geometry we know, say, a point charge along the $x$%
-axis\FRAME{dtbpF}{1.9112in}{1.6111in}{0pt}{}{}{Figure}{\special{language
"Scientific Word";type "GRAPHIC";maintain-aspect-ratio TRUE;display
"USEDEF";valid_file "T";width 1.9112in;height 1.6111in;depth
0pt;original-width 1.8732in;original-height 1.5757in;cropleft "0";croptop
"1";cropright "1";cropbottom "0";tempfilename
'M5MBWX03.wmf';tempfile-properties "XPR";}}We know the potential will be%
\begin{equation*}
V=\frac{1}{4\pi \epsilon _{o}}\frac{q}{x}
\end{equation*}%
then we can try 
\begin{eqnarray*}
E_{s} &=&-\frac{dV}{dx}=-\frac{d}{dx}\frac{1}{4\pi \epsilon _{o}}\frac{q}{x}
\\
&=&\frac{1}{4\pi \epsilon _{o}}\frac{q}{x^{2}}
\end{eqnarray*}%
which gives us just what we expected!\FRAME{dhF}{3.4947in}{2.0375in}{0pt}{}{%
}{Figure}{\special{language "Scientific Word";type
"GRAPHIC";maintain-aspect-ratio TRUE;display "USEDEF";valid_file "T";width
3.4947in;height 2.0375in;depth 0pt;original-width 3.448in;original-height
1.9986in;cropleft "0";croptop "1";cropright "1";cropbottom "0";tempfilename
'LU495N08.wmf';tempfile-properties "XPR";}}

Let's try another. Let's find the electric field due to a disk of charge
along the axis. We have done this problem before. We know the field should
be 
\begin{equation}
E_{z}=\frac{2\pi \eta }{4\pi \epsilon _{o}}\left( 1-\allowbreak \frac{z}{%
\sqrt{a^{2}+z^{2}}}\right)
\end{equation}%
and in the previous lectures we found the potential to be%
\begin{equation}
V=\frac{\eta }{2\epsilon _{o}}\left( \allowbreak \sqrt{a^{2}+z^{2}}%
-\allowbreak z\right)
\end{equation}

Now can we find the electric field at $P$ from $V?$ Let's start by finding
the $z-$component of the field, $E_{z}$%
\begin{eqnarray}
E_{z} &=&-\frac{dV}{dz} \\
&=&-\frac{d}{dz}\left( \frac{\eta }{2\epsilon _{o}}\left( \allowbreak \sqrt{%
a^{2}+z^{2}}-\allowbreak z\right) \right) \\
&=&-\frac{d}{dz}\frac{\eta }{2\epsilon _{o}}\allowbreak \sqrt{a^{2}+z^{2}}+%
\frac{d}{dz}\frac{\eta }{2\epsilon _{o}}z \\
&=&-\frac{\eta }{2\epsilon _{o}}\allowbreak \frac{d}{dz}\sqrt{a^{2}+z^{2}}+%
\frac{\eta }{2\epsilon _{o}} \\
&=&-\frac{\eta }{2\epsilon _{o}}\allowbreak \frac{z}{\sqrt{a^{2}+z^{2}}}+%
\frac{\eta }{2\epsilon _{o}} \\
E_{z} &=&\frac{\eta }{2\epsilon _{o}}\left( 1-\allowbreak \frac{z}{\sqrt{%
a^{2}+z^{2}}}\right)
\end{eqnarray}%
or%
\begin{equation}
E_{z}=\frac{2\pi \eta }{4\pi \epsilon _{o}}\left( 1-\allowbreak \frac{z}{%
\sqrt{a^{2}+z^{2}}}\right)
\end{equation}%
But remember that this situation is highly symmetric. We can see by
inspection that all the $x$ and $y$ components will all cancel out. So this
is our field! And it is just what we found before.

We can graph these functions to compare them (what would you expect?). To do
this we really need values, but instead, let's play a clever trick that some
of you will see in advanced or older books. I am going to substitute in
place of $z$ the variable $u=\frac{z}{a}.$ Then%
\begin{eqnarray*}
V &=&\frac{\eta }{2\epsilon _{o}}_{e}\left( \allowbreak \sqrt{a^{2}+z^{2}}%
-\allowbreak z\right) \\
&=&\frac{\eta a}{2\epsilon _{o}}\left( \allowbreak \sqrt{1+\frac{z^{2}}{a^{2}%
}}-\allowbreak \frac{z}{a}\right) \\
&=&\frac{\eta a}{2\epsilon _{o}}\left( \allowbreak \sqrt{1+u^{2}}%
-\allowbreak u\right)
\end{eqnarray*}%
and 
\begin{eqnarray}
E_{z} &=&\frac{2\pi \eta }{4\pi \epsilon _{o}}\left( 1-\allowbreak \frac{z}{%
\sqrt{a^{2}+z^{2}}}\right) \\
&=&\frac{2\pi \eta }{4\pi \epsilon _{o}}\left( 1-\allowbreak \frac{z}{a\sqrt{%
1+\frac{z^{2}}{a^{2}}}}\right)  \notag \\
&=&\frac{2\pi \eta }{4\pi \epsilon _{o}}\left( 1-\allowbreak \frac{z}{a\sqrt{%
1+\frac{z^{2}}{a^{2}}}}\right)  \notag \\
&=&\frac{2\pi \eta }{4\pi \epsilon _{o}}\left( 1-\allowbreak \frac{\frac{z}{a%
}}{\sqrt{1+\frac{z^{2}}{a^{2}}}}\right)  \notag \\
&=&\frac{2\pi \eta }{4\pi \epsilon _{o}}\left( 1-\allowbreak \frac{u}{\sqrt{%
1+u^{2}}}\right)  \notag
\end{eqnarray}%
Both my equation for $V$ and for $E_{z}$ now are in the form of a set of
constants times a function of $u.$ 
\begin{eqnarray*}
V &=&\frac{\eta a}{2\epsilon _{o}}\left( \allowbreak \sqrt{1+u^{2}}%
-\allowbreak u\right) \\
&=&\frac{\eta a}{2\epsilon _{o}}f\left( u\right)
\end{eqnarray*}%
\begin{eqnarray}
E_{z} &=&\frac{2\pi \eta }{4\pi \epsilon _{o}}\left( 1-\allowbreak \frac{u}{%
\sqrt{1+u^{2}}}\right)  \notag \\
&=&\frac{2\pi \eta }{4\pi \epsilon _{o}}g\left( u\right)
\end{eqnarray}%
If I plot $V$ in units of $\frac{\eta a}{2\epsilon _{o}}$ (the constants out
in front) I can see the shape of the curve. It is the function of $f(u).$ I
can compare this to $E_{z}$ in units of $\frac{2\pi \eta }{4\pi \epsilon _{o}%
}.$ The shape of $E_{z}$ will be $g\left( u\right) .$ Of course we are
plotting terms of $u.$ \FRAME{dtbpFX}{2.5288in}{1.6862in}{0pt}{}{}{Plot}{%
\special{language "Scientific Word";type "MAPLEPLOT";width 2.5288in;height
1.6862in;depth 0pt;display "USEDEF";plot_snapshots TRUE;mustRecompute
FALSE;lastEngine "MuPAD";xmin "0";xmax "6";xviewmin "0";xviewmax
"6";yviewmin "0";yviewmax "1.2";viewset"XY";rangeset"X";plottype
4;labeloverrides 3;x-label "u";y-label "Strange Units";axesFont "Times New
Roman,12,0000000000,useDefault,normal";numpoints 100;plotstyle
"patch";axesstyle "normal";axestips FALSE;xis \TEXUX{u};var1name
\TEXUX{$u$};function \TEXUX{$\left( \allowbreak \sqrt{1+u^{2}}-\allowbreak
u\right) $};linecolor "green";linestyle 1;pointstyle "point";linethickness
1;lineAttributes "Solid";var1range "0,6";num-x-gridlines 100;curveColor
"[flat::RGB:0x00008000]";curveStyle "Line";function \TEXUX{$\left(
1-\allowbreak \frac{u}{\sqrt{1+u^{2}}}\right) $};linecolor "blue";linestyle
2;pointstyle "point";linethickness 1;lineAttributes "Dash";var1range
"0,6";num-x-gridlines 100;curveColor "[flat::RGB:0x000000ff]";curveStyle
"Line";VCamFile 'LU0NUK0V.xvz';valid_file "T";tempfilename
'S3MHHQ01.wmf';tempfile-properties "XPR";}}Now we can ask, is this
reasonable? Does it look like the $E-$field (red dashed line) is the right
shape for the derivative of the potential (solid green line)? It is also
comforting to see that as $u$ (a function of $z$) gets larger the field
falls off to zero and so does the potential as we would expect. When $V$
(green solid curve) has a large slope, $E_{z}$ is a large number (positive
because of the negative sign in the equation 
\begin{equation*}
E_{s}=-\frac{dV}{ds}
\end{equation*}%
and when $V$ is fairly flat, $E_{z}$ is nearly zero. Our strategy for
finding $E$ from $V$ seems to work.

\section{Geometry of field and potential}

You should probably worry that so far our equation 
\begin{equation*}
E_{s}=-\frac{dV}{ds}
\end{equation*}%
is only one dimensional. We know the electric field is a three dimensional
vector field. We may find situations where we need two or three dimensions.
But this is easy to fix. Our equation 
\begin{equation*}
E_{s}=-\frac{dV}{ds}
\end{equation*}%
gives us the field magnitude along the $\mathbf{\hat{s}}$ direction. Let's
choose this to be the $\mathbf{\hat{x}}$ direction. Then 
\begin{equation*}
E_{x}=-\frac{dV}{dx}
\end{equation*}%
is the $x$-component of the electric field. Likewise%
\begin{eqnarray*}
E_{y} &=&-\frac{dV}{dy} \\
E_{z} &=&-\frac{dV}{dz}
\end{eqnarray*}%
The total field will be the vector sum of it's components%
\begin{eqnarray*}
\overrightarrow{\mathbf{E}} &=&E_{x}\mathbf{\hat{\imath}}+E_{y}\mathbf{\hat{%
\jmath}+}E_{z}\mathbf{\hat{k}} \\
&=&-\frac{dV}{dx}\mathbf{\hat{\imath}-}\frac{dV}{dy}\mathbf{\hat{\jmath}-}%
\frac{dV}{dz}\mathbf{\hat{k}}
\end{eqnarray*}%
%TCIMACRO{%
%\TeXButton{Question 223.33.1}{\marginpar {
%\hspace{-0.5in}
%\begin{minipage}[t]{1in}
%\small{Question 223.33.1}
%\end{minipage}
%}}}%
%BeginExpansion
\marginpar {
\hspace{-0.5in}
\begin{minipage}[t]{1in}
\small{Question 223.33.1}
\end{minipage}
}%
%EndExpansion
%TCIMACRO{%
%\TeXButton{Question 223.33.2}{\marginpar {
%\hspace{-0.5in}
%\begin{minipage}[t]{1in}
%\small{Question 223.33.2}
%\end{minipage}
%}}}%
%BeginExpansion
\marginpar {
\hspace{-0.5in}
\begin{minipage}[t]{1in}
\small{Question 223.33.2}
\end{minipage}
}%
%EndExpansion
which we can cryptically write as 
\begin{equation*}
\overrightarrow{\mathbf{E}}=-\left( \frac{d}{dx}\mathbf{\hat{\imath}}+\frac{d%
}{dy}\mathbf{\hat{\jmath}+}\frac{d}{dz}\mathbf{\hat{k}}\right) V
\end{equation*}%
The odd group of operations in the parenthesis is call a \emph{gradient} and
is written as 
\begin{equation*}
\overrightarrow{\mathbf{\nabla }}=\left( \frac{d}{dx}\mathbf{\hat{\imath}}+%
\frac{d}{dy}\mathbf{\hat{\jmath}+}\frac{d}{dz}\mathbf{\hat{k}}\right)
\end{equation*}%
using this we have%
\begin{equation*}
\overrightarrow{\mathbf{E}}=-\overrightarrow{\mathbf{\nabla }}V
\end{equation*}%
which is how the relationship is stated in higher level electrodynamics
books. But what does it mean?

The gradient is really kind of what it sounds like. If you go down a steep
grade, you will notice you are going down hill and will notice if you are
going down the steepest part of the hill. The gradient finds the direction
of steepest decent. That is, the direction where the potential changes
fastest. This is like looking from the top of the hill and taking the
steepest way down! Our relationship tells us that the electric field points
in this steepest direction, and the minus sign tells us that the electric
field points down hill away from a positive charge, never up hill (think of
the acceleration due to gravity being negative). Let's see if this makes
sense for our geometries that we know.%
%TCIMACRO{%
%\TeXButton{Stamp in a circle}{\marginpar {
%\hspace{-0.5in}
%\begin{minipage}[t]{1in}
%\small{Stamp in a circle: mimic a blindfolded person swiveling on one foot and testing the slope with the other}
%\end{minipage}
%}}}%
%BeginExpansion
\marginpar {
\hspace{-0.5in}
\begin{minipage}[t]{1in}
\small{Stamp in a circle: mimic a blindfolded person swiveling on one foot and testing the slope with the other}
\end{minipage}
}%
%EndExpansion

Here is our capacitor. We see that indeed the field points from the high
potential to the low potential. The steepest way \textquotedblleft down the
hill\textquotedblright\ is perpendicular to the equipotential lines.\FRAME{%
dhF}{2.8543in}{3.5896in}{0pt}{}{}{Figure}{\special{language "Scientific
Word";type "GRAPHIC";maintain-aspect-ratio TRUE;display "USEDEF";valid_file
"T";width 2.8543in;height 3.5896in;depth 0pt;original-width
2.88in;original-height 3.6295in;cropleft "0";croptop "1";cropright
"1";cropbottom "0";tempfilename 'LU4BRI0E.wmf';tempfile-properties "XPR";}}%
We also know the shape of the field for a dipole. The equipotential lines we
have seen before.\FRAME{dhF}{5.9605in}{1.9141in}{0pt}{}{}{Figure}{\special%
{language "Scientific Word";type "GRAPHIC";maintain-aspect-ratio
TRUE;display "USEDEF";valid_file "T";width 5.9605in;height 1.9141in;depth
0pt;original-width 6.0457in;original-height 1.9212in;cropleft "0";croptop
"1";cropright "1";cropbottom "0";tempfilename
'LU4AXV0B.wmf';tempfile-properties "XPR";}}But now we can see that the field
lines and equipotential lines are always perpendicular and the field points
\textquotedblleft down hill.\textquotedblright\ The meeting of the field and
equipotential lines at right angles is not a surprise. Think again about our
mountain\FRAME{dhFU}{4.3152in}{1.7669in}{0pt}{\Qcb{Map courtesy USGS,
Picture is in the Public Domain.}}{}{Figure}{\special{language "Scientific
Word";type "GRAPHIC";maintain-aspect-ratio TRUE;display "USEDEF";valid_file
"T";width 4.3152in;height 1.7669in;depth 0pt;original-width
12.7491in;original-height 5.1906in;cropleft "0";croptop "1";cropright
"1";cropbottom "0";tempfilename 'LU4B270C.wmf';tempfile-properties "XPR";}}%
The steepest path is always perpendicular to lines of equal potential energy.

We should try another example of finding the field from the gradient.
Suppose we have a potential that varies as 
\begin{equation*}
V=3x^{2}+2xy
\end{equation*}%
I\ don't know what is making this potential, but let's suppose we have such
a potential. It would look like this.\FRAME{dtbpFX}{2.4517in}{1.6345in}{0pt}{%
}{}{Plot}{\special{language "Scientific Word";type "MAPLEPLOT";width
2.4517in;height 1.6345in;depth 0pt;display "USEDEF";plot_snapshots
TRUE;mustRecompute FALSE;lastEngine "MuPAD";xmin "-5";xmax "5";ymin
"-5";ymax "5";xviewmin "-5";xviewmax "5";yviewmin "-5";yviewmax "5";zviewmin
"-8.33333";zviewmax "125";phi 71;theta 139;cameraDistance
"8.56541";cameraOrientation "[0,0,0.005]";cameraOrientationFixed
TRUE;plottype 5;labeloverrides 4;z-label "V";axesFont "Times New
Roman,12,0000000000,useDefault,normal";num-x-gridlines 25;num-y-gridlines
25;plotstyle "patch";axesstyle "normal";axestips FALSE;plotshading
"XYZ";lighting 0;xis \TEXUX{x};yis \TEXUX{y};var1name \TEXUX{$x$};var2name
\TEXUX{$y$};function \TEXUX{$3x^{2}+2xy$};linestyle 1;pointstyle
"point";lineAttributes "Solid";var1range "-5,5";var2range
"-5,5";surfaceColor "[linear:XYZ:RGB:0x00ff0000:0x000000ff]";surfaceStyle
"Color Patch";num-x-gridlines 25;num-y-gridlines 25;surfaceMesh
"Mesh";VCamFile 'M5MC2W04.xvz';valid_file "T";tempfilename
'M5MC2K04.wmf';tempfile-properties "XPR";}}what is the electric field?%
\begin{equation*}
\overrightarrow{\mathbf{E}}=-\overrightarrow{\mathbf{\nabla }}V
\end{equation*}%
or%
\begin{equation*}
\overrightarrow{\mathbf{E}}=-\left( \frac{d}{dx}\mathbf{\hat{\imath}}+\frac{d%
}{dy}\mathbf{\hat{\jmath}+}\frac{d}{dz}\mathbf{\hat{k}}\right) V
\end{equation*}%
so 
\begin{equation*}
\overrightarrow{\mathbf{E}}=-\left( \frac{d}{dx}\mathbf{\hat{\imath}}+\frac{d%
}{dy}\mathbf{\hat{\jmath}+}\frac{d}{dz}\mathbf{\hat{k}}\right) \left(
3x^{2}+2xy\right)
\end{equation*}

\begin{eqnarray*}
\overrightarrow{\mathbf{E}} &=&-\left( \mathbf{\hat{\imath}}\frac{d}{dx}%
\left( 3x^{2}+2xy\right) +\mathbf{\hat{\jmath}}\frac{d}{dy}\left(
3x^{2}+2xy\right) \mathbf{+\hat{k}}\frac{d}{dz}\left( 3x^{2}+2xy\right)
\right) \\
&=&-\left( \mathbf{\hat{\imath}}\left( 6x+2y\right) +\mathbf{\hat{\jmath}}%
\frac{d}{dy}\left( 2xy\right) \mathbf{+}0\right)
\end{eqnarray*}%
This example shows how to perform the operation, but it does not give much
insight. We have learned to work with our standard charge configurations,
and this is really not one of them. So we don't have much intuitive feel for
this electric field that we found.

To gain more insight, let's return to finding the point charge field from
the point charge potential. The potential for a point charge is 
\begin{equation*}
V=\frac{1}{4\pi \epsilon _{o}}\frac{Q}{r}
\end{equation*}%
And of course we know that the field is 
\begin{equation*}
E=\frac{1}{4\pi \epsilon _{o}}\frac{Q}{r^{2}}\mathbf{\hat{r}}
\end{equation*}%
but we want to show this using 
\begin{equation*}
\overrightarrow{\mathbf{E}}=-\overrightarrow{\mathbf{\nabla }}V
\end{equation*}

So%
\begin{eqnarray*}
\overrightarrow{\mathbf{E}} &=&-\left( \frac{d}{dx}\mathbf{\hat{\imath}}+%
\frac{d}{dy}\mathbf{\hat{\jmath}+}\frac{d}{dz}\mathbf{\hat{k}}\right) V \\
&=&-\left( \frac{d}{dx}\mathbf{\hat{\imath}}+\frac{d}{dy}\mathbf{\hat{\jmath}%
+}\frac{d}{dz}\mathbf{\hat{k}}\right) \frac{1}{4\pi \epsilon _{o}}\frac{Q}{r}
\\
&=&-\left( \frac{d}{dx}\mathbf{\hat{\imath}}+\frac{d}{dy}\mathbf{\hat{\jmath}%
+}\frac{d}{dz}\mathbf{\hat{k}}\right) \frac{1}{4\pi \epsilon _{o}}\frac{Q}{%
\sqrt{x^{2}+y^{2}+z^{2}}} \\
&=&-\frac{Q}{4\pi \epsilon _{o}}\left( \frac{d}{dx}\mathbf{\hat{\imath}}+%
\frac{d}{dy}\mathbf{\hat{\jmath}+}\frac{d}{dz}\mathbf{\hat{k}}\right) \frac{1%
}{\sqrt{x^{2}+y^{2}+z^{2}}} \\
&=&-\frac{Q}{4\pi \epsilon _{o}}\left( -\frac{x}{\left(
x^{2}+y^{2}+z^{2}\right) ^{\frac{3}{2}}}\mathbf{\hat{\imath}}-\frac{y}{%
\left( x^{2}+y^{2}+z^{2}\right) ^{\frac{3}{2}}}\mathbf{\hat{\jmath}}-\frac{k%
}{\left( x^{2}+y^{2}+z^{2}\right) ^{\frac{3}{2}}}\mathbf{\hat{k}}\right) \\
&=&\frac{Q}{4\pi \epsilon _{o}}\frac{\left( x\mathbf{\hat{\imath}}+y\mathbf{%
\hat{\jmath}}+k\mathbf{\hat{k}}\right) }{\left( x^{2}+y^{2}+z^{2}\right) ^{%
\frac{3}{2}}} \\
&=&\frac{Q}{4\pi \epsilon _{o}}\frac{\left( x\mathbf{\hat{\imath}}+y\mathbf{%
\hat{\jmath}}+k\mathbf{\hat{k}}\right) }{\left( x^{2}+y^{2}+z^{2}\right) 
\sqrt{\left( x^{2}+y^{2}+z^{2}\right) }} \\
&=&\frac{1}{4\pi \epsilon _{o}}\frac{Q}{r^{2}}\frac{\left( x\mathbf{\hat{%
\imath}}+y\mathbf{\hat{\jmath}}+k\mathbf{\hat{k}}\right) }{\sqrt{\left(
x^{2}+y^{2}+z^{2}\right) }} \\
&=&\frac{1}{4\pi \epsilon _{o}}\frac{Q}{r^{2}}\mathbf{\hat{r}}
\end{eqnarray*}%
but really, this is a bit of a mess, we don't want to do such a problem in
rectangular coordinates. We could write $\mathbf{\nabla }$ in spherical
coordinates (something we won't derive here, but you should have seen in
M215 or M316). 
\begin{equation*}
\overrightarrow{\mathbf{\nabla }}=\mathbf{\hat{r}}\frac{\partial }{\partial r%
}+\mathbf{\hat{\theta}}\frac{1}{r}\frac{\partial }{\partial \theta }+\mathbf{%
\hat{\phi}}\frac{1}{r\sin \theta }\frac{\partial }{\partial \phi }
\end{equation*}

Let's try this out on our point charge potential. We have 
\begin{eqnarray*}
\overrightarrow{\mathbf{E}} &=&-\left( \mathbf{\hat{r}}\frac{\partial }{%
\partial r}+\mathbf{\hat{\theta}}\frac{1}{r}\frac{\partial }{\partial \theta 
}+\mathbf{\hat{\phi}}\frac{1}{r\sin \theta }\frac{\partial }{\partial \phi }%
\right) V \\
&=&-\left( \mathbf{\hat{r}}\frac{\partial }{\partial r}+\mathbf{\hat{\theta}}%
\frac{1}{r}\frac{\partial }{\partial \theta }+\mathbf{\hat{\phi}}\frac{1}{%
r\sin \theta }\frac{\partial }{\partial \phi }\right) \frac{1}{4\pi \epsilon
_{o}}\frac{Q}{r} \\
&=&-\frac{Q}{4\pi \epsilon _{o}}\left( -\frac{1}{r^{2}}\mathbf{\hat{r}}%
+0+0\right) \\
&=&\frac{1}{4\pi \epsilon _{o}}\frac{Q}{r^{2}}\mathbf{\hat{r}}
\end{eqnarray*}%
just as we expected. But this time the math was much easier. If we can, it
is a good idea to match our expression for $\overrightarrow{\mathbf{\nabla }}
$ to the geometry of the system. A good vector calculus book or a compendium
of math functions will have various versions of $\overrightarrow{\mathbf{%
\nabla }}$ listed.

\section{Conductors in equilibrium again}

\FRAME{dhF}{1.4662in}{2.205in}{0pt}{}{}{Figure}{\special{language
"Scientific Word";type "GRAPHIC";maintain-aspect-ratio TRUE;display
"USEDEF";valid_file "T";width 1.4662in;height 2.205in;depth
0pt;original-width 1.4321in;original-height 2.1646in;cropleft "0";croptop
"1";cropright "1";cropbottom "0";tempfilename
'LUATIC03.wmf';tempfile-properties "XPR";}}

%TCIMACRO{%
%\TeXButton{Question 223.33.3}{\marginpar {
%\hspace{-0.5in}
%\begin{minipage}[t]{1in}
%\small{Question 223.33.3}
%\end{minipage}
%}}}%
%BeginExpansion
\marginpar {
\hspace{-0.5in}
\begin{minipage}[t]{1in}
\small{Question 223.33.3}
\end{minipage}
}%
%EndExpansion
We know that there is no field inside a conductor in electrostatic
equilibrium, but we should ask what that means for the electric potential.
To build circuits or electronic actuators, we will need to know this. Let's
start again with%
\begin{equation}
\Delta V=-\int_{A}^{B}\mathbf{E}\cdot d\mathbf{s}
\end{equation}%
and since the field $E=0$ inside the conductor, then inside 
\begin{equation}
\Delta V_{inside}=0
\end{equation}%
On the surface we see that there is a potential, since there is a field. If
we take our spherical case, \FRAME{dhF}{1.2436in}{1.1441in}{0pt}{}{}{Figure}{%
\special{language "Scientific Word";type "GRAPHIC";maintain-aspect-ratio
TRUE;display "USEDEF";valid_file "T";width 1.2436in;height 1.1441in;depth
0pt;original-width 1.209in;original-height 1.1104in;cropleft "0";croptop
"1";cropright "1";cropbottom "0";tempfilename
'LYZ9DI01.wmf';tempfile-properties "XPR";}}and observe the potential as we
go away from the center, we expect the potential to be constant up to the
surface. Then as we reach the surface, we know from Gauss' law that the
field will be 
\begin{equation*}
E=\frac{1}{4\pi \epsilon _{o}}\frac{Q}{r^{2}}
\end{equation*}%
like a point charge, so the potential at the surface must be%
\begin{equation*}
V=\frac{1}{4\pi \epsilon _{o}}\frac{Q}{R}
\end{equation*}%
where $r=R,$ the radius of our sphere. As we move into the sphere from the
surface, the potential must not change. The interior will have the potential%
\begin{equation}
V_{inside}=\frac{1}{4\pi \epsilon _{o}}\frac{Q}{R}
\end{equation}%
Outside, of course, the potential will drop like the potential due to a
point charge. We expect 
\begin{equation}
V=\frac{1}{4\pi \epsilon _{o}}\frac{Q}{r}
\end{equation}%
For a sphere of radius $R=0.5\unit{m}$ carrying a charge of $0.000002\unit{C}
$ (about what our van de Graaff holds) we would have the situation graphed
in the following figure: \FRAME{dhF}{4.7143in}{3.3457in}{0pt}{}{}{Figure}{%
\special{language "Scientific Word";type "GRAPHIC";maintain-aspect-ratio
TRUE;display "USEDEF";valid_file "T";width 4.7143in;height 3.3457in;depth
0pt;original-width 6.1885in;original-height 4.3835in;cropleft "0";croptop
"1";cropright "1";cropbottom "0";tempfilename
'LUAUWF08.wmf';tempfile-properties "XPR";}}This is an important point. For a
conductor, the electric potential everywhere inside the conductive material
is exactly the same once we reach equilibrium. This is just what we want for
capacitors or electrodes or electrical contacts in circuits.

\subsection{Non spherical conductors}

The field is stronger where the field lines are closer together. One way to
describe this is to use a radii of curvature. That is, suppose we try to fit
a small circle into a bump on the surface of a conductor. \FRAME{dtbpF}{%
4.5165in}{1.2152in}{0pt}{}{}{Figure}{\special{language "Scientific
Word";type "GRAPHIC";maintain-aspect-ratio TRUE;display "USEDEF";valid_file
"T";width 4.5165in;height 1.2152in;depth 0pt;original-width
6.5674in;original-height 1.7469in;cropleft "0";croptop "1";cropright
"1";cropbottom "0";tempfilename 'LUAT9Y01.wmf';tempfile-properties "XPR";}}%
In the figure there are two bumps shown with circles fit into them. The bump
on the right has a much smaller radius circle than the one on the left. The
radius of the circle that fits into the bump is the radius of curvature of
the bump. From what we have said, the bump on the right will have a much
stronger field strength near it than the bump on the left.

Where there is a lot of charge on a conductor, and the field is very high,
electrons from random ionizations of air molecules near the conductor are
accelerated away from the conductor. These electrons hit other atoms,
ionizing them as well. We get a small avalanche of electrons. Eventually the
electrons recombine with ionized atoms, producing an eerie glow. This is
called \emph{corona discharge}. It can be used to find faults in high
tension wires and other high voltage situations.%
%TCIMACRO{%
%\TeXButton{Coronal Discharge Clips}{\marginpar {
%\hspace{-0.5in}
%\begin{minipage}[t]{1in}
%\small{Coronal Discharge Clips}
%\end{minipage}
%}}}%
%BeginExpansion
\marginpar {
\hspace{-0.5in}
\begin{minipage}[t]{1in}
\small{Coronal Discharge Clips}
\end{minipage}
}%
%EndExpansion

\subsection{Cavities in conductors}

Suppose we have a hollow conductor with no charges in the cavity. What is
the field? We know from using Gauss' law what the answer should be, but
let's do this using potentials.

\FRAME{dhF}{1.58in}{1.5074in}{0pt}{}{}{Figure}{\special{language "Scientific
Word";type "GRAPHIC";maintain-aspect-ratio TRUE;display "USEDEF";valid_file
"T";width 1.58in;height 1.5074in;depth 0pt;original-width
1.5437in;original-height 1.471in;cropleft "0";croptop "1";cropright
"1";cropbottom "0";tempfilename 'LYZAHZ02.wmf';tempfile-properties "XPR";}}%
All the parts of the conductor will be at the same potential. So let's take
two points, $A$ and $B,$ and compute%
\begin{equation*}
V_{A}-V_{B}=-\int_{A}^{B}\mathbf{E}\cdot d\mathbf{s}
\end{equation*}%
We know that $V_{A}-V_{B}=0$ because $V_{A}$ must be the same as $V_{B}.$ So
for every path, $s,$ we must have%
\begin{equation*}
-\int_{A}^{B}\mathbf{E}\cdot d\mathbf{s=0}
\end{equation*}%
We can easily conclude that $E$ must equal zero.

So as long as there are no charges inside the cavity, the cavity is a net
field free zone.

It is often much easier to find the potential, and from the potential, find
the field. Much of the study of electrodynamics uses this approach. This is
because it is more straight-forward to differentiate than it is to
integrate. Some of you may use massive computational programs to predict
electric fields. They often use differential equations in the potential to
find the field rather than integral equations to find the field directly.

%TCIMACRO{%
%\TeXButton{Basic Equations}{\hspace{-1.3in}{\LARGE Basic Equations\vspace{0.25in}}}}%
%BeginExpansion
\hspace{-1.3in}{\LARGE Basic Equations\vspace{0.25in}}%
%EndExpansion

\chapter{Capacitance}

%TCIMACRO{%
%\TeXButton{Fundamental Concepts}{\hspace{-1.3in}{\LARGE Fundamental Concepts\vspace{0.25in}}}}%
%BeginExpansion
\hspace{-1.3in}{\LARGE Fundamental Concepts\vspace{0.25in}}%
%EndExpansion

\begin{itemize}
\item The charge on a capacitor is proportional to the potential difference $%
Q=C\Delta V$

\item The constant of proportionality is called the capacitance and for a
parallel plate capacitor, it is given by $C=\frac{A}{d}\epsilon _{o}$

\item In parallel capacitors capacitances add $C_{eq}=C_{1}+C_{2}$

\item In series capacitors capacitances combine as $\frac{1}{C_{tot}}=\frac{1%
}{C_{1}}+\frac{1}{C_{2}}$
\end{itemize}

\section{Capacitance and capacitors}

Consider the following design for a pump-tank system.\FRAME{dhF}{2.7851in}{%
3.3244in}{0pt}{}{}{Figure}{\special{language "Scientific Word";type
"GRAPHIC";maintain-aspect-ratio TRUE;display "USEDEF";valid_file "T";width
2.7851in;height 3.3244in;depth 0pt;original-width 2.8091in;original-height
3.359in;cropleft "0";croptop "1";cropright "1";cropbottom "0";tempfilename
'LUB99Q09.wmf';tempfile-properties "XPR";}}This is may not be an optimal
design. At first there is no problem, water flows into the upper tank just
fine. But once the upper tank begins to fill, the water already in the upper
tank will make it harder to pump in more water. As the tank fills, the
pressure at the bottom increases, and it takes more work for the pump to
overcome the increasing pressure.

Something analogous happens when a capacitor is connected to a battery. 
\FRAME{dhF}{4.7711in}{1.8857in}{0pt}{}{}{Figure}{\special{language
"Scientific Word";type "GRAPHIC";maintain-aspect-ratio TRUE;display
"USEDEF";valid_file "T";width 4.7711in;height 1.8857in;depth
0pt;original-width 4.8332in;original-height 1.8928in;cropleft "0";croptop
"1";cropright "1";cropbottom "0";tempfilename
'LUB0P600.wmf';tempfile-properties "XPR";}}At first the charge is free to
flow to the plates, but as the charge builds, it takes more work to bring on
successive charges.\FRAME{dhF}{4.9112in}{2.3611in}{0pt}{}{}{Figure}{\special%
{language "Scientific Word";type "GRAPHIC";maintain-aspect-ratio
TRUE;display "USEDEF";valid_file "T";width 4.9112in;height 2.3611in;depth
0pt;original-width 4.976in;original-height 2.3771in;cropleft "0";croptop
"1";cropright "1";cropbottom "0";tempfilename
'LUB15702.wmf';tempfile-properties "XPR";}}The charges repel each other, so
the charge already on a capacitor plate repels the new charge arriving from
the battery. The repelling force gets larger until finally the force
repelling the charge balances the force driving the charge from the battery
and the charge stops flowing onto the capacitor.

A capacitor is made from two plates. For us, let's assume they are
semi-infinte sheets of charge. Of course this is not exactly true, but it is
not too wrong near the center of the plates. And we know quite a lot about
semi-infinate sheets of charge because they are one of our standard change
configurations. We know the field for each sheet is 
\begin{equation*}
E=\frac{\eta }{2\epsilon _{o}}
\end{equation*}%
and that for two sheets, one with $+\eta $ and one with $-\eta $ the field
in between will be 
\begin{equation*}
E=\frac{\eta }{\epsilon _{o}}
\end{equation*}%
We also know the potential difference between the two plates is just%
\begin{equation*}
\Delta V=Ed
\end{equation*}%
where $E$ is our electric field and $d$ is the capacitor spacing.

We can guess that we will build up charge until the potential energy
difference of the capacitor is equal to the potential energy difference of
the battery%
\begin{equation*}
\Delta V_{capacitor}=\Delta V_{battery}
\end{equation*}%
because at that point the forces causing the potential energy will be equal.

We can write our electric field between the two plates as 
\begin{equation*}
E=\frac{\eta }{\epsilon _{o}}=\frac{Q}{A\epsilon _{o}}
\end{equation*}%
so 
\begin{equation*}
\Delta V=\frac{d}{A\epsilon _{o}}Q
\end{equation*}%
Then the potential difference is directly proportional to the charge. I\
want to switch this around, and solve for the amount of charge. 
\begin{equation*}
Q=\left( \frac{A\epsilon _{o}}{d}\right) \Delta V
\end{equation*}%
Since all the terms in the parenthesis are constants, we could replace them
with a constant, $C.$ 
\begin{equation}
Q=C\Delta V
\end{equation}%
where 
\begin{equation}
C=\frac{A}{d}\epsilon _{o}
\end{equation}%
is a constant that depends on the geometry and construction of the plates.
This equation tells us that if we build two different sets of plates, say,
one circular and one triangular, and we give them the same potential
difference (say, connect them both to $12\unit{V}$ batteries) then, if both
have the same construction constant $C,$ they will carry the same charge
even though their size and shape are different. We can reduce the burden of
calculation of how much charge a capacitor can hold but asking the person
who manufactured it to calculate the construction constant and mark the
value on the outside of the capacitor. Different capacitors may be
constructed differently (different $A$ or $d$ values) but so long as the
construction constant, $C,$ is the same, the charge amount for a given
voltage will be the same.

%TCIMACRO{%
%\TeXButton{Question 223.34.1}{\marginpar {
%\hspace{-0.5in}
%\begin{minipage}[t]{1in}
%\small{Question 223.34.1}
%\end{minipage}
%}}}%
%BeginExpansion
\marginpar {
\hspace{-0.5in}
\begin{minipage}[t]{1in}
\small{Question 223.34.1}
\end{minipage}
}%
%EndExpansion
The electronics field gives this construction constant a name, \emph{%
capacitance.}%
\begin{equation}
C=\frac{Q}{\Delta V}
\end{equation}%
The capacitance will have units of $\unit{C}/\unit{V}$ but we give this a
name all it's own, the \emph{Farad} $\left( \unit{F}\right) .$ A Farad is a
very large capacitance. Many capacitors in electronic devices are measured
in microfarads. 
%TCIMACRO{%
%\TeXButton{Question 223.34.2}{\marginpar {
%\hspace{-0.5in}
%\begin{minipage}[t]{1in}
%\small{Question 223.34.2}
%\end{minipage}
%}} }%
%BeginExpansion
\marginpar {
\hspace{-0.5in}
\begin{minipage}[t]{1in}
\small{Question 223.34.2}
\end{minipage}
}
%EndExpansion
%TCIMACRO{%
%\TeXButton{Question 223.34.3}{\marginpar {
%\hspace{-0.5in}
%\begin{minipage}[t]{1in}
%\small{Question 223.34.3}
%\end{minipage}
%}} }%
%BeginExpansion
\marginpar {
\hspace{-0.5in}
\begin{minipage}[t]{1in}
\small{Question 223.34.3}
\end{minipage}
}
%EndExpansion
%TCIMACRO{%
%\TeXButton{Question 223.34.4}{\marginpar {
%\hspace{-0.5in}
%\begin{minipage}[t]{1in}
%\small{Question 223.34.4}
%\end{minipage}
%}}}%
%BeginExpansion
\marginpar {
\hspace{-0.5in}
\begin{minipage}[t]{1in}
\small{Question 223.34.4}
\end{minipage}
}%
%EndExpansion

\subsection{Capacitors and sources of potential}

Consider what happens when we connect our two parallel plates to the
terminals of a battery. Assuming the plates are initially uncharged, charge
flows from the battery through the conducting wires and onto the plates.
Recall that for a metal, the entire surface will be at the same potential
under electrostatic conditions. The charge carriers supplied by the battery
will try to achieve electrostatic equilibrium, so we expect the plate that
is connected to the positive terminal of the battery to eventually be at the
same potential as the positive battery terminal. Likewise for the negative
terminal and the plate connected to it.

We can even use our capacitor as a source of electrical power. A camera
flash uses capacitors to make the burst of light that illuminates the
subject of your photo. \FRAME{dhFU}{1.3941in}{1.7132in}{0pt}{\Qcb{Camera
flash unit (Public Domain image by Julo)}}{}{Figure}{\special{language
"Scientific Word";type "GRAPHIC";maintain-aspect-ratio TRUE;display
"USEDEF";valid_file "T";width 1.3941in;height 1.7132in;depth
0pt;original-width 1.9597in;original-height 2.4146in;cropleft "0";croptop
"1";cropright "1";cropbottom "0";tempfilename
'LZ196V00.wmf';tempfile-properties "XPR";}}

\subsection{Single conductor capacitance}

Physicists can't leave a good thing alone. We often calculate the
capacitance of a single conductor! If the geometry is simple we can easily
do this. It is not immediately obvious that a single conductor should even
have a capacitance, so it might be a problem if you forget this in a design
problem for an unusual device.

As an example, let's take a sphere. We will assume there is a spherical
conduction shell that is infinitely far away. This configuration gives
exactly the same field lines that the charged sphere gives on it's own, but
the mental picture is helpful. The imaginary shell will give $V=0$ (we set
our zero potential at $r=\infty $). The potential of the little sphere we
know must be just like the potential of a point charge if we are outside of
the sphere%
\begin{equation*}
V=k_{e}\frac{Q}{r}
\end{equation*}%
for $r=R$, the radius of our little sphere. Then 
\begin{equation*}
\Delta V=k_{e}\frac{Q}{R}-0=k_{e}\frac{Q}{R}
\end{equation*}%
so%
\begin{equation}
C=\frac{Q}{\Delta V}=\frac{Q}{k_{e}\frac{Q}{R}}=\frac{R}{k_{e}}=4\pi
\varepsilon _{o}R
\end{equation}%
This is the capacitance of a single sphere. Note that $C$ only depends on
geometry! not on $Q,$ just as we would expect.

But why would we care? This says that even if we just connect a ball to,
say, the positive terminal of a battery, that there will be some
capacitance. This capacitance will limit the flow of charge to the ball. So
it will take time to charge even a single conductor. This is always true
when a device is initially connected to a power source. Often we can ignore
such \textquotedblleft transient\textquotedblright\ effects because the
charging times are still small. But in special cases, this may not be
possible because the changing voltage or charge could damage sensitive
equipment. So although this is rarely a problem, it is good to keep in the
back of our minds.

\subsection{Capacitance of two parallel plates}

The capacitance of single conductors is profound, but more useful to us in
understanding common electronic components is the parallel plate capacitor.
We found that for parallel plates we also had only geometry factors in the
capacitance. Of course, there are other shapes possible. Let's see if we can
reason out how the capacitance depends on the geometry.

\subsubsection{Plate area}

Since the charge will tend to separate to the surface of a conductor, we
might expect that if the surface area increases, the amount of charge that
the capacitor can hold might increase as well. We see this in our equation
for the parallel plate capacitor.%
\begin{equation*}
C=\frac{A}{d}\epsilon _{o}
\end{equation*}

\subsubsection{Plate separation}

We also see that it matters how far apart the plates are placed. The greater
the distance, the less the capacitance. This makes some sense. If the plates
are farther apart, the Coulomb force is weaker, and less charge can be held
in the capacitor, because the force attracting the charges (the force
between the charges on the opposite plates) is weaker.

\subsection{Capacitance of a cylindrical capacitor}

We should try some harder geometries. A cylindrical capacitor is a good case
to start with

\FRAME{dhF}{1.1682in}{2.0232in}{0pt}{}{}{Figure}{\special{language
"Scientific Word";type "GRAPHIC";maintain-aspect-ratio TRUE;display
"USEDEF";valid_file "T";width 1.1682in;height 2.0232in;depth
0pt;original-width 3.0086in;original-height 5.2376in;cropleft "0";croptop
"1";cropright "1";cropbottom "0";tempfilename
'LUB2YQ03.wmf';tempfile-properties "XPR";}}

(you will do a sphere in the homework problems). We want to find the
capacitance of the cylindrical capacitor. Our strategy will be to find the
voltage difference for the capacitor and the amount of charge on the
capacitor, and then divide to find $C.$%
\begin{equation*}
C=\frac{Q}{\Delta V}
\end{equation*}

Let's begin with our equation relating potential change to field.%
\begin{equation}
V_{b}-V_{a}=-\int_{a}^{b}\mathbf{\vec{E}}\cdot d\mathbf{\vec{s}}
\end{equation}%
Let's assume that there is a linear charge density, $\lambda ,$ along the
cylinder with the center positive and the outside negative. Then 
\begin{equation}
\Phi _{E}=\doint \mathbf{\vec{E}}\cdot d\mathbf{\vec{A}}
\end{equation}%
where I will choose a Gaussian surface that is cylindrical around the
central conductor. \FRAME{dhF}{1.7349in}{2.3053in}{0pt}{}{}{Figure}{\special%
{language "Scientific Word";type "GRAPHIC";maintain-aspect-ratio
TRUE;display "USEDEF";valid_file "T";width 1.7349in;height 2.3053in;depth
0pt;original-width 1.7394in;original-height 2.3203in;cropleft "0";croptop
"1";cropright "1";cropbottom "0";tempfilename
'LUB36504.wmf';tempfile-properties "XPR";}}This is nice, since the field
will be radially out from the conductor (ignoring the end effects) and so no
field will pass through the end caps of the Gaussian surface ($\mathbf{\vec{E%
}}\cdot d\mathbf{\vec{A}=0}$ on the end caps). Moreover, the field strikes
the surface at right angles ($\mathbf{\vec{E}}\cdot d\mathbf{\vec{A}}=EdA$
on the side of the cylinder), and will have the same magnitude all the way
around so 
\begin{eqnarray*}
\Phi _{E} &=&E\doint dA \\
&=&EA
\end{eqnarray*}%
Now we know from Gauss' law that%
\begin{equation*}
\Phi _{E}=\frac{Q_{in}}{\varepsilon _{o}}
\end{equation*}%
where 
\begin{equation*}
Q_{in}=\lambda h
\end{equation*}%
and where $h$ is the height of our Gaussian surface, so%
\begin{equation*}
\Phi _{E}=\frac{\lambda h}{\varepsilon _{o}}=E2\pi rh
\end{equation*}%
\begin{equation*}
\frac{\lambda }{2\pi r\varepsilon _{o}}=E
\end{equation*}%
Now, knowing our field, and taking a radial path from $a$ to $b$, we can
take 
\begin{eqnarray*}
V_{b}-V_{a} &=&-\int_{a}^{b}\frac{\lambda }{2\pi r\varepsilon _{o}}d\mathbf{r%
} \\
&=&-\frac{\lambda }{2\pi \varepsilon _{o}}\int_{a}^{b}\frac{1}{r}d\mathbf{r}
\\
&=&-\frac{\lambda }{2\pi \varepsilon _{o}}\ln \left( \frac{b}{a}\right)
\end{eqnarray*}%
Using this, we can find the capacitance, We have a negative value for $%
\Delta V,$ but this is just due to our choice of making the center of the
concentric cylinders positive and the outside negative. We chose the zero
point on the positive center. The amount of potential change going from $a$
to $b$ is just $\left\vert \Delta V\right\vert .$ Then in finding the
capacitance using 
\begin{equation*}
Q=C\Delta V
\end{equation*}%
We want just the value of $\Delta V$ so we will plug in the absolute value
of our result.%
\begin{equation*}
\left\vert \Delta V\right\vert =\frac{\lambda }{2\pi \varepsilon _{o}}\ln
\left( \frac{b}{a}\right)
\end{equation*}%
Then, solving for $C$ gives%
\begin{eqnarray*}
C &=&\frac{Q}{\Delta V} \\
&=&\frac{Q}{\frac{\lambda }{2\pi \varepsilon _{o}}\ln \left( \frac{b}{a}%
\right) } \\
&=&\frac{Q}{\frac{Q}{2\pi h\varepsilon _{o}}\ln \left( \frac{b}{a}\right) }
\\
&=&\frac{2\pi h\varepsilon _{o}}{\ln \left( \frac{b}{a}\right) }
\end{eqnarray*}%
Wow! That was fun! But more importantly, this is a coaxial cable geometry,
and we can see that coaxial cable will have some capacitance and that that
capacitance will depend on the geometry of the cable including its length
and width. This capacitance can affect signals sent through the cable. Later
in our course we will see why. But for now just know that if I combine a
resistor and a capacitor together it takes more time for the charge to move.
So in our signal cable the signal will get distorted.\FRAME{dhFU}{1.5238in}{%
1.5212in}{0pt}{\Qcb{Increasing amounts of distortion in a signal due to
increasing cable capacitance.}}{}{Figure}{\special{language "Scientific
Word";type "GRAPHIC";maintain-aspect-ratio TRUE;display "USEDEF";valid_file
"T";width 1.5238in;height 1.5212in;depth 0pt;original-width
1.4866in;original-height 1.4849in;cropleft "0";croptop "1";cropright
"1";cropbottom "0";tempfilename 'M0FDL201.wmf';tempfile-properties "XPR";}}%
The nice square pulses that represent digital data will be distorted, and in
extreme cases, undetectable. When designing data lines, this capacitance of
the cable must be taken into account.

\section{Combinations of Capacitors}

%TCIMACRO{%
%\TeXButton{Question 223.34.5}{\marginpar {
%\hspace{-0.5in}
%\begin{minipage}[t]{1in}
%\small{Question 223.34.5}
%\end{minipage}
%}}}%
%BeginExpansion
\marginpar {
\hspace{-0.5in}
\begin{minipage}[t]{1in}
\small{Question 223.34.5}
\end{minipage}
}%
%EndExpansion
We don't want to have to do long calculations to combine capacitors that we
buy from an electronics store. It would be convenient to come up with a way
to combine capacitors using a simple rule.

\FRAME{ftbpF}{1.5584in}{1.8174in}{0pt}{}{}{Figure}{\special{language
"Scientific Word";type "GRAPHIC";maintain-aspect-ratio TRUE;display
"USEDEF";valid_file "T";width 1.5584in;height 1.8174in;depth
0pt;original-width 3.0661in;original-height 3.5808in;cropleft "0";croptop
"1";cropright "1";cropbottom "0";tempfilename
'LTUWDJAM.wmf';tempfile-properties "XPR";}}We need a simple way to write
capacitors in our homework problem drawings, here are the usual symbols for
capacitor and battery. Using these symbols, let's consider two capacitors as
shown below.\FRAME{dhF}{1.8618in}{2.8215in}{0pt}{}{}{Figure}{\special%
{language "Scientific Word";type "GRAPHIC";maintain-aspect-ratio
TRUE;display "USEDEF";valid_file "T";width 1.8618in;height 2.8215in;depth
0pt;original-width 1.868in;original-height 2.8463in;cropleft "0";croptop
"1";cropright "1";cropbottom "0";tempfilename
'LUB4TM06.wmf';tempfile-properties "XPR";}}Remember that a conductor will be
at the same potential over all of its surface. If we connect the capacitors
as shown then all of the left half of this diagram will be at the positive
potential of the battery terminal. Likewise, the right side will all be at
the same potential. It is like we increased the area of the capacitor $C_{1}$
buy adding in the area of capacitor $C_{2}.$ 
\begin{equation*}
C=\frac{A_{1}+A_{2}}{d}\epsilon _{o}=\frac{A_{1}}{d}\epsilon _{o}+\frac{A_{2}%
}{d}\epsilon _{o}
\end{equation*}%
So we may write a combined capacitance for this set up of%
\begin{equation}
C_{eq}=C_{1}+C_{2}
\end{equation}%
We call this set up a \emph{parallel} circuit. This means that each of the
capacitors are hooked directly to the terminals of the battery.

But suppose we hook up the capacitors as in the next drawing\FRAME{ftbpF}{%
2.1414in}{1.7735in}{0pt}{}{}{Figure}{\special{language "Scientific
Word";type "GRAPHIC";maintain-aspect-ratio TRUE;display "USEDEF";valid_file
"T";width 2.1414in;height 1.7735in;depth 0pt;original-width
4.6531in;original-height 3.8482in;cropleft "0";croptop "1";cropright
"1";cropbottom "0";tempfilename 'LTUWDKAO.wmf';tempfile-properties "XPR";}}%
Now we expect the left hand side of $C_{1}$ to be at the positive potential
of the positive terminal of the battery. We expect the right side of $C_{2}$
to be at the same potential as the negative side of the battery. What
happens in the middle?

We can see that we will have negative charge on the right hand plate of $%
C_{2}$ and positive charge on the left plate of $C_{1}.$ This must cause
there to be a positive charge on the right plate of $C_{1}$ and a negative
charge on the left plate of $C_{2.}$ Moreover, all the charges will have the
same magnitude. That means each of the plates will have a potential
difference%
\begin{equation*}
\Delta V_{1}=\frac{Q}{C_{1}}
\end{equation*}%
and%
\begin{equation*}
\Delta V_{2}=\frac{Q}{C_{1}}
\end{equation*}

But the total potential difference is $\Delta V$ of the battery, then 
\begin{equation*}
\Delta V=\Delta V_{1}+\Delta V_{2}
\end{equation*}%
We can again define an equivalent capacitance.%
\begin{equation*}
\Delta V=\frac{Q}{C_{tot}}
\end{equation*}%
then%
\begin{eqnarray*}
\Delta V &=&\Delta V_{1}+\Delta V_{2} \\
\frac{Q}{C_{tot}} &=&\frac{Q}{C_{1}}+\frac{Q}{C_{2}}
\end{eqnarray*}%
The $Q$s are all the same. So%
\begin{equation}
\frac{1}{C_{tot}}=\frac{1}{C_{1}}+\frac{1}{C_{2}}
\end{equation}

We call this type of set up a \emph{series circuit} because the capacitors
came one after the other as you go from one side of the battery to the other.

Now after all this you might ask yourself how to know the capacitance of the
parts you buy to build things. They are designed by engineers and tested at
the factory, and the capacitance is usually printed on the side of the
device. You can, of course, devise a test circuit based on what we have
learned that could test the capacitance.

%TCIMACRO{%
%\TeXButton{Basic Equations}{\hspace{-1.3in}{\LARGE Basic Equations\vspace{0.25in}}}}%
%BeginExpansion
\hspace{-1.3in}{\LARGE Basic Equations\vspace{0.25in}}%
%EndExpansion

\chapter{Dielectrics and Current}

%TCIMACRO{%
%\TeXButton{Fundamental Concepts}{\hspace{-1.3in}{\LARGE Fundamental Concepts\vspace{0.25in}}}}%
%BeginExpansion
\hspace{-1.3in}{\LARGE Fundamental Concepts\vspace{0.25in}}%
%EndExpansion

\begin{itemize}
\item Dielectrics and capacitors

\item Microscopic nature of electric current

\item Current direction is defined as the direction positive charges would
go, regardless of the actual sign of the charge.

\item In a capacitor, the stored energy is $W=\frac{1}{2}C\Delta V^{2}$

\item The energy density in the electric field is $u=\frac{1}{2}\epsilon
_{o}E^{2}$
\end{itemize}

\section{Energy stored in a capacitor}

We have convinced ourselves that $\Delta V$ is the change in potential
energy per unit charge, so when a capacitor is charged, and the wires
connecting it to the battery are removed, is there potential energy
\textquotedblleft stored\textquotedblright\ in the capacitor? The answer is
yes, and we can see it by considering what would happen if we connected a
wire (no battery) between the two plates. Charge would rush from one plate
to the other. This is like storing a tank of water on a hill. If we connect
a pipe from the tank at the top of the hill to a tank at the bottom of the
hill, the water will rush through the pipe to the lower tank.

BE CAREFUL, you are enough of a conductor that by touching different ends of
a capacitor you could create a serious current through your body. The
capacitors in old computer monitors or old TV sets can store enough charge
to kill you!

But how much energy would there be stored in the capacitor? Clearly it must
be related to the amount of energy it takes to move the charge onto the
plates. By analogy, the energy stored in the water was the minimum amount of
energy it took to pump the water to the upper tank $(mgh).$ It is the
minimum, because our pipes might have some resistance, and then we would
have to include more work to overcome the resistance.

But for a capacitor it is a little bit more tricky. When the capacitor is
not charged, it takes no work (or very little) to move charge from one plate
to the other. But once there is a charge there is an electric field between
the plates (think of my poorly designed water storage system from the
beginning of last lecture). This creates a potential difference. And we must
fight against this potential difference to add more charge. This is sort of
like transferring rocks up a hill. The more rocks that we carry, the higher
the hill gets, and the more work it takes to bring up more rocks.

From our formula%
\begin{equation*}
W=q\left( V_{B}-V_{A}\right)
\end{equation*}%
we can see that if we have just a small amount of charge, $\Delta Q,$ we
will have a small amount of work 
\begin{equation*}
\Delta W=\Delta Q\Delta V
\end{equation*}%
to move it onto the capacitor.\footnote{%
Agh! here $\Delta Q$ is a small amount of charge, and $\Delta V$ is $%
V_{f}-V_{i}$. We have used $\Delta $ in two different ways in the same
equation.} If we start with no charge, then go in small $\Delta Q$ steps, we
would see a potential rise as shown in the graph below.\FRAME{dhF}{2.7984in}{%
2.5146in}{0pt}{}{}{Figure}{\special{language "Scientific Word";type
"GRAPHIC";maintain-aspect-ratio TRUE;display "USEDEF";valid_file "T";width
2.7984in;height 2.5146in;depth 0pt;original-width 2.8233in;original-height
2.5341in;cropleft "0";croptop "1";cropright "1";cropbottom "0";tempfilename
'MJIKJQ0C.wmf';tempfile-properties "XPR";}}The quantity $\Delta Q\Delta V$
is the area of the shaded (green) rectangle. So $\Delta W$ is given by the
area of a rectangle under a stair-step on our graph. The shaded rectangle is
just one of many rectangles in the graph. we can write%
\begin{equation}
C=\frac{\Delta Q}{\Delta V}
\end{equation}%
or 
\begin{equation*}
\Delta V=\frac{\Delta Q}{C}
\end{equation*}%
As $\Delta Q$ gets small we can go to a continuous charge model 
\begin{equation*}
\Delta W=\Delta Q\Delta V
\end{equation*}%
We can replaced the small unit of charge $\Delta Q$ with a continuous
variable $q.$to obtain%
\begin{equation*}
dW=dq\left( \Delta V\right)
\end{equation*}%
Recall that%
\begin{equation*}
\Delta V=\frac{q}{C}
\end{equation*}%
so we can write $dW$ as 
\begin{eqnarray*}
dW &=&dq\left( \frac{q}{C}\right) \\
dW &=&\frac{1}{C}qdq
\end{eqnarray*}

Of course, we will integrate this%
\begin{eqnarray}
W &=&\int_{0}^{Q}\frac{1}{C}qdq  \notag \\
W &=&\frac{1}{C}\int_{0}^{Q}qdq  \notag \\
&=&\frac{Q^{2}}{2C}
\end{eqnarray}%
or sometimes using 
\begin{equation*}
Q=C\Delta V
\end{equation*}%
this is written as%
\begin{equation}
W=\frac{1}{2}C\Delta V^{2}
\end{equation}

There is a limit to how much energy we can store. That is because even air
can conduct charge if the potential difference is high enough. We call this
air conduction a spark or coronal discharge. At some point charge jumps from
one plate to another through the air in between. If the potential difference
is very high, the Coulomb force between the charges on opposite plates will
force charge to leave one plate and jump to the other even if there is no
air! 
%TCIMACRO{%
%\TeXButton{Question 223.34.6}{\marginpar {
%\hspace{-0.5in}
%\begin{minipage}[t]{1in}
%\small{Question 223.34.6}
%\end{minipage}
%}} }%
%BeginExpansion
\marginpar {
\hspace{-0.5in}
\begin{minipage}[t]{1in}
\small{Question 223.34.6}
\end{minipage}
}
%EndExpansion
%TCIMACRO{%
%\TeXButton{Question 223.34.7}{\marginpar {
%\hspace{-0.5in}
%\begin{minipage}[t]{1in}
%\small{Question 223.34.7}
%\end{minipage}
%}}}%
%BeginExpansion
\marginpar {
\hspace{-0.5in}
\begin{minipage}[t]{1in}
\small{Question 223.34.7}
\end{minipage}
}%
%EndExpansion

\subsection{Field storage}

We usually consider the energy stored in the capacitor to be stored in the
electric field. The field is proportional to the amount of charge and
related to the potential energy, so this seems reasonable. Let's find the
potential energy stored in the field in the capacitor. Recall for an ideal
parallel plate capacitor 
\begin{equation*}
\Delta V=Ed
\end{equation*}%
and%
\begin{equation*}
C=\epsilon _{o}\frac{A}{d}
\end{equation*}%
We assume that energy provided by the work to move the charges on the
capacitor is all stored as potential energy, so 
\begin{equation}
U_{stored}=\frac{1}{2}C\Delta V^{2}
\end{equation}%
then 
\begin{eqnarray*}
U_{stored} &=&\frac{1}{2}\left( \epsilon _{o}\frac{A}{d}\right) \left(
Ed\right) ^{2} \\
&=&\frac{1}{2}\epsilon _{o}AdE^{2}
\end{eqnarray*}%
We often define and energy density%
\begin{equation*}
u=\frac{U_{stored}}{\mathbb{V}}
\end{equation*}%
In this case the volume $\mathbb{V}$ is just $Ad$ so 
\begin{equation}
u=\frac{1}{2}\epsilon _{o}E^{2}
\end{equation}%
This is the density of energy in the electric field. It turns out that this
is a general formula (not just true for ideal parallel plate capacitors).

This is a step toward our goal of showing that electric fields are a
physically real thing. They can store energy, so they must be a real thing.

\section{Dielectrics and capacitors}

%TCIMACRO{%
%\TeXButton{Question 223.35.1}{\marginpar {
%\hspace{-0.5in}
%\begin{minipage}[t]{1in}
%\small{Question 223.35.1}
%\end{minipage}
%}}}%
%BeginExpansion
\marginpar {
\hspace{-0.5in}
\begin{minipage}[t]{1in}
\small{Question 223.35.1}
\end{minipage}
}%
%EndExpansion
We should ask ourselves a question about our capacitors, does it matter that
there is air in between the plates? For making capacitors, it might be
convenient to coat two sides of a plastic block with metal and solder wires
to the coated sides. Does the plastic have an effect?

Plastic is an insulator, and another name for \textquotedblleft
insulator\textquotedblright\ is \emph{dielectric}. If we perform the
experiment, we will find that when a dielectric is placed in the plates, the
potential difference decreases! \FRAME{dhF}{2.7657in}{2.6256in}{0pt}{}{}{%
Figure}{\special{language "Scientific Word";type
"GRAPHIC";maintain-aspect-ratio TRUE;display "USEDEF";valid_file "T";width
2.7657in;height 2.6256in;depth 0pt;original-width 3.1798in;original-height
3.0175in;cropleft "0";croptop "1";cropright "1";cropbottom "0";tempfilename
'LUD3LY00.wmf';tempfile-properties "XPR";}}We are lucky, though, from
experimentation we have found that it seems to decrease in a nice, linear
way. We can write this as 
\begin{equation}
\Delta V=\frac{\Delta V_{wo}}{\kappa }  \label{DielectricVoltageChange}
\end{equation}%
where $\kappa $ is a constant that depends on what material we choose as our
dielectric\footnote{%
This symbol $\kappa ,$ is the greek letter \textquotedblleft
kappa.\textquotedblright} and $\Delta V_{wo}$ is the potential difference
without the dielectric (the subscripts $wo$ will stand for \textquotedblleft
without the dielectric.\textquotedblright\ But what is happening?

The plates of the capacitor are becoming charged. These charges will
polarize the material in the middle.

\FRAME{dtbpF}{2.6403in}{2.4465in}{0in}{}{}{Figure}{\special{language
"Scientific Word";type "GRAPHIC";maintain-aspect-ratio TRUE;display
"USEDEF";valid_file "T";width 2.6403in;height 2.4465in;depth
0in;original-width 2.5979in;original-height 2.405in;cropleft "0";croptop
"1";cropright "1";cropbottom "0";tempfilename
'S3S21900.wmf';tempfile-properties "XPR";}}%
%TCIMACRO{%
%\TeXButton{Question 223.35.2}{\marginpar {
%\hspace{-0.5in}
%\begin{minipage}[t]{1in}
%\small{Question 223.35.2}
%\end{minipage}
%}}}%
%BeginExpansion
\marginpar {
\hspace{-0.5in}
\begin{minipage}[t]{1in}
\small{Question 223.35.2}
\end{minipage}
}%
%EndExpansion
Notice how the polarized molecules or atoms sill have a net zero charge in
the middle, 
%TCIMACRO{%
%\TeXButton{Ubalanced Handedness Demo}{\marginpar {
%\hspace{-0.5in}
%\begin{minipage}[t]{1in}
%\small{Ubalanced Handedness Demo, Stick out your hands, one side of room has extra left hands, one side extra right hands}
%\end{minipage}
%}}}%
%BeginExpansion
\marginpar {
\hspace{-0.5in}
\begin{minipage}[t]{1in}
\small{Ubalanced Handedness Demo, Stick out your hands, one side of room has extra left hands, one side extra right hands}
\end{minipage}
}%
%EndExpansion
but on the ends, there is a net charge. It is like we have oppositely
charged plates next to our capacitor plates. That reduces the net charge
seen by the capacitor, and so the potential difference is less. There is
effectively less separated charge.

%TCIMACRO{%
%\TeXButton{Question 223.35.3}{\marginpar {
%\hspace{-0.5in}
%\begin{minipage}[t]{1in}
%\small{Question 223.35.3}
%\end{minipage}
%}}}%
%BeginExpansion
\marginpar {
\hspace{-0.5in}
\begin{minipage}[t]{1in}
\small{Question 223.35.3}
\end{minipage}
}%
%EndExpansion
But since our capacitor is not connected to a battery or any other
electrical device, the amount of actual charge on the capacitor plates can't
have changed, so if $\Delta V$ changed, but $Q$ did not , then since 
\begin{equation*}
Q=C\Delta V
\end{equation*}%
we suspect the material properties part--the capacitance--must have changed.%
\begin{equation*}
C=\frac{Q_{wo}}{\Delta V}=\frac{Q_{wo}}{\frac{\Delta V_{wo}}{\kappa }}=\frac{%
\kappa Q_{wo}}{\Delta V_{o}}
\end{equation*}%
but this is just%
\begin{equation}
C=\kappa C_{wo}  \label{DielectricCapacitanceChange}
\end{equation}%
For a parallel plate capacitor, we have 
\begin{equation}
C=\kappa \varepsilon _{o}\frac{A}{d}
\label{DielectricParallelPlateCapacitance}
\end{equation}

So where do you find values for $\kappa ?$ For this class, we will look them
up in the tables in books or on the internet. Here are a few values for our
use.%
\begin{equation*}
\begin{tabular}{|c|c|c|c|c|}
\hline
{\small Material} & ${\small \kappa }$ &  & {\small Material} & ${\small %
\kappa }$ \\ \hline
{\small Vacuum} & ${\small 1.00000}$ &  & {\small Paper} & ${\small 3.7}$ \\ 
\hline
{\small Dry Air} & ${\small 1.0006}$ &  & {\small Waxed Paper} & ${\small 3.5%
}$ \\ \hline
{\small Fused quartz} & ${\small 3.78}$ &  & {\small Polystyrene} & ${\small %
2.56}$ \\ \hline
{\small Pyrex glass} & ${\small 4.7-5.6}$ &  & {\small PVC} & ${\small 3.4}$
\\ \hline
{\small Mylar} & ${\small 3.15}$ &  & {\small Teflon} & ${\small 2.1}$ \\ 
\hline
{\small Nylon} & ${\small 3.4}$ &  & {\small Water} & ${\small 80}$ \\ \hline
\end{tabular}%
\end{equation*}

\section{Induced Charge}

In the last discussion we discovered that if we put a dielectric inside a
capacitor, we end up with polarized charges with the net result that there
will be excess negative charge near the positive plate of the capacitor, and
excess negative charge near the positive plates of the capacitor. In the
middle of the dielectric, the charges are polarized in each atom, but still
for any volume inside, the net charge is zero. The excess charge near each
plate we will call the \emph{induced charge}.

Since we have an induced positive charge on one side and an induced negative
charge on the other side, we expect there will be an electric field directed
from the positive to negative charge inside the dielectric.\FRAME{dtbpF}{%
1.5843in}{2.3168in}{0in}{}{}{Figure}{\special{language "Scientific
Word";type "GRAPHIC";maintain-aspect-ratio TRUE;display "USEDEF";valid_file
"T";width 1.5843in;height 2.3168in;depth 0in;original-width
1.5489in;original-height 2.2771in;cropleft "0";croptop "1";cropright
"1";cropbottom "0";tempfilename 'S3S28N01.wmf';tempfile-properties "XPR";}}

The total field inside the dielectric is 
\begin{equation}
E=E_{wo}-E_{ind}  \label{DielectricInducedField}
\end{equation}%
where $E_{wo}$ is the field due to the capacitor plates without the
dielectric. From our previous discussion, we recall that 
\begin{equation*}
\Delta V=\frac{\Delta V_{wo}}{\kappa }
\end{equation*}%
and we recall that the magnitude of the potential difference is given by 
\begin{equation*}
\Delta V=Ed
\end{equation*}%
Then our new net field can be found%
\begin{equation*}
Ed=\frac{E_{wo}d}{\kappa }
\end{equation*}%
or 
\begin{equation*}
E=\frac{E_{wo}}{\kappa }
\end{equation*}%
and, recalling for a parallel plate capacitor (near the center) the field is
approximately 
\begin{equation*}
E=\frac{\eta }{\varepsilon _{o}}
\end{equation*}%
then%
\begin{equation*}
E=E_{o}-E_{ind}
\end{equation*}%
gives%
\begin{equation*}
\frac{\eta }{\kappa \epsilon _{o}}=\frac{\eta }{\epsilon _{o}}-\frac{\eta
_{ind}}{\epsilon _{o}}
\end{equation*}%
and we can find the induced surface charge density as 
\begin{equation*}
\eta _{ind}=\eta \left( 1-\frac{1}{\kappa }\right)
\end{equation*}

You might guess that the induced charge is attracted to the charge on the
plates, so a force is required (and work is required) to remove the
dielectric once it is in place. If we draw out the dielectric, we can see
that the weaker field outside the capacitor causes little induced charge,
but the stronger field inside the capacitor causes a large induced charge. A
net inward force will result.\FRAME{dtbpF}{1.4356in}{2.2416in}{0in}{}{}{%
Figure}{\special{language "Scientific Word";type
"GRAPHIC";maintain-aspect-ratio TRUE;display "USEDEF";valid_file "T";width
1.4356in;height 2.2416in;depth 0in;original-width 1.4001in;original-height
2.2027in;cropleft "0";croptop "1";cropright "1";cropbottom "0";tempfilename
'S3S29A02.wmf';tempfile-properties "XPR";}}

\section{Electric current}

%TCIMACRO{%
%\TeXButton{Question 223.35.4}{\marginpar {
%\hspace{-0.5in}
%\begin{minipage}[t]{1in}
%\small{Question 223.35.4}
%\end{minipage}
%}}}%
%BeginExpansion
\marginpar {
\hspace{-0.5in}
\begin{minipage}[t]{1in}
\small{Question 223.35.4}
\end{minipage}
}%
%EndExpansion
%TCIMACRO{%
%\TeXButton{Question 223.35.5}{\marginpar {
%\hspace{-0.5in}
%\begin{minipage}[t]{1in}
%\small{Question 223.35.5}
%\end{minipage}
%}}}%
%BeginExpansion
\marginpar {
\hspace{-0.5in}
\begin{minipage}[t]{1in}
\small{Question 223.35.5}
\end{minipage}
}%
%EndExpansion
For some time now, we have been talking about charge moving. We have had
charge move from a battery to the plates of a conductor. We have had charge
flow from one side to another of a conductor, etc. It is time to become more
exact in describing the flow of charge. We should take some time to figure
out why charge will move.

Let's consider a conductor again.\FRAME{dhF}{3.2301in}{1.9605in}{0pt}{}{}{%
Figure}{\special{language "Scientific Word";type
"GRAPHIC";maintain-aspect-ratio TRUE;display "USEDEF";valid_file "T";width
3.2301in;height 1.9605in;depth 0pt;original-width 4.4763in;original-height
2.7069in;cropleft "0";croptop "1";cropright "1";cropbottom "0";tempfilename
'LUDA5J07.wmf';tempfile-properties "XPR";}}We remember that in the
conductor, the valance electrons are free to move. In fact, they do move all
the time. The electrons will have some thermal energy just because the
conductor is not at absolute zero temperature. This thermal energy causes
them to move in random directions. (think of air molecules in a room).

Let's take a piece of a wire $\Delta x$ long. The speed of the electrons
along the wire (in the $x$-direction in this case) is called the \emph{drift
speed}, $v_{d},$ because the electrons just drift from place to place with a
fairly small speed. This drift speed could be due mostly to thermal energy,
so it can be very small or even zero (if no electric potential is applied).
Of course, $v_{d},$ must be an average, each charge carrier will be moving
random directions with slightly different speeds, so the $x$-component of
the velocity will be different for each charge carrier, but on average they
will move at a speed $v_{d}.$

\FRAME{dtbpF}{3.1142in}{1.222in}{0in}{}{}{Figure}{\special{language
"Scientific Word";type "GRAPHIC";maintain-aspect-ratio TRUE;display
"USEDEF";valid_file "T";width 3.1142in;height 1.222in;depth
0in;original-width 3.0701in;original-height 1.1882in;cropleft "0";croptop
"1";cropright "1";cropbottom "0";tempfilename
'S3S2B503.wmf';tempfile-properties "XPR";}}So we will suppose that there are
charge carriers of charge $q_{c}$ that are moving through the wire with
velocity $v_{d}.$ Then we can write some length of wire, $\Delta x,$ as 
\begin{equation*}
\Delta x=v_{d}\Delta t
\end{equation*}%
The volume of the shaded piece of wire is 
\begin{equation*}
\mathbb{V}=A\Delta x
\end{equation*}%
%TCIMACRO{%
%\TeXButton{Question 223.35.6}{\marginpar {
%\hspace{-0.5in}
%\begin{minipage}[t]{1in}
%\small{Question 223.35.6}
%\end{minipage}
%}}}%
%BeginExpansion
\marginpar {
\hspace{-0.5in}
\begin{minipage}[t]{1in}
\small{Question 223.35.6}
\end{minipage}
}%
%EndExpansion
if there are 
\begin{equation*}
n=\frac{\#}{\mathbb{V}}
\end{equation*}%
charge carriers per unit volume, a \emph{volume charge carrier number density%
}, then the total charge in our volume is 
\begin{equation*}
\Delta Q=nA\Delta xq_{c}
\end{equation*}%
If we have electrons as our charge carrier, then $q_{c}$ is just $q_{e}.$

We can substitute for $\Delta x$%
\begin{equation*}
\Delta Q=nAv_{d}\Delta tq_{e}
\end{equation*}%
This gives the charge within our small volume. But it would be nice to know
how much charge is going by, because we want moving charge. We can divide by 
$\Delta t$

\begin{equation}
\frac{\Delta Q}{\Delta t}=nAv_{d}q_{e}  \label{CurrentDefenition}
\end{equation}%
to get a charge flow rate. This is very like our volume or mass flow rate in
fluid flow. We have an amount of charge going by in a time $\Delta t.$

I gave the flow velocity a special name, $v_{d}.$ But I did not give all the
reasons for using an average $x$-component of the velocity. If we think
about it, we will realize that the electrons don't really flow in a straight
line. They continually bump into atoms\footnote{%
We will refine this picture in the next lecture.}. So the actual path the
electrons take looks more like this.\FRAME{dtbpF}{2.6636in}{1.6821in}{0pt}{}{%
}{Figure}{\special{language "Scientific Word";type
"GRAPHIC";maintain-aspect-ratio TRUE;display "USEDEF";valid_file "T";width
2.6636in;height 1.6821in;depth 0pt;original-width 3.2534in;original-height
2.0453in;cropleft "0";croptop "1";cropright "1";cropbottom "0";tempfilename
'LUDA5I06.wmf';tempfile-properties "XPR";}}We only care about the forward
part of this motion. It is that forward component that we call the \emph{%
drift speed} of the electrons. It is much slower than the actual speed the
electrons travel, and it depends on the type of conductor we are using.

We already know the name for the flow rate of charge, it is the electric
current. 
\begin{equation}
\frac{\Delta Q}{\Delta t}=I  \label{CurrentAsRate}
\end{equation}

We should take a minute to think about what to expect when we allow charge
to flow. Think of a garden hose. If the hose is full of water, then when we
open the faucet, water immediately comes out. The water that leaves the
faucet is far from the open end of the hose, though. We have to wait for it
to travel the entire length of the hose. But we get water out of the hose
immediately! Why? Well, from Pascal's principle we know that a change in
pressure will be transmitted uniformly throughout the fluid. This is like
your hydraulic breaks. The new water coming in causes a pressure change that
is transmitted through the hose. The water at the open end is pushed out.%
\FRAME{dhF}{2.6884in}{1.9061in}{0pt}{}{}{Figure}{\special{language
"Scientific Word";type "GRAPHIC";maintain-aspect-ratio TRUE;display
"USEDEF";valid_file "T";width 2.6884in;height 1.9061in;depth
0pt;original-width 4.5484in;original-height 3.2171in;cropleft "0";croptop
"1";cropright "1";cropbottom "0";tempfilename
'LUD3GL00.wmf';tempfile-properties "XPR";}}

Current is a little bit like this. When we flip a light switch, the
electrons near the near the switch start to flow at $v_{d}.$ But there are
already free electrons in all the wire. These experience a
Pascal's-principle-like-push that makes the light turn on almost instantly.

%TCIMACRO{%
%\TeXButton{Question 223.35.7}{\marginpar {
%\hspace{-0.5in}
%\begin{minipage}[t]{1in}
%\small{Question 223.35.7}
%\end{minipage}
%}}}%
%BeginExpansion
\marginpar {
\hspace{-0.5in}
\begin{minipage}[t]{1in}
\small{Question 223.35.7}
\end{minipage}
}%
%EndExpansion
There is a historical oddity with current flow. It is that the current
direction is the direction positive charges would flow. This may seem
strange, since in good conductors, we have said that electrons are doing the
flowing! The truth is that it is very hard to tell the difference between
positive charge flow and negative charge flow. In fact, only one experiment
that I know of shows that the charge carriers in metals are electrons. 
\FRAME{dhF}{2.841in}{1.2435in}{0pt}{}{}{Figure}{\special{language
"Scientific Word";type "GRAPHIC";maintain-aspect-ratio TRUE;display
"USEDEF";valid_file "T";width 2.841in;height 1.2435in;depth
0pt;original-width 2.8658in;original-height 1.2382in;cropleft "0";croptop
"1";cropright "1";cropbottom "0";tempfilename
'LUD49A03.wmf';tempfile-properties "XPR";}}That experiment accelerates a
conductor. The experiment is easier to perform using a centrifuge, but it is
easier to visualize with linear motion. If we accelerate a bar of metal as
shown in the preceding figure, the electrons are free to move about in the
metal but the nuclei are all bound together. If the nuclei are accelerated
they must go as a group. But the electrons will tend to stay with their
initial motion (Newton's first law) until the end of the bar reaches them.
At this point they must move because the electrical force of the mass of
nuclei will keep them bound to the whole mass of metal. But the electrons
will pile up at the tail end of the bar--that is--if it is the electrons
that are free. When this experiment is performed, it is indeed the electrons
that pile up at the tail end, and the forward end is left positive. This can
be measured with a voltmeter.

Ben Franklin chose the direction we now use. He had a 50\% chance if getting
the charge carrier right. All this shows just how hard it is to deal with
all these things we can't see or touch. And even more importantly, in
semiconductors and in biological systems, it \emph{is} positive charge that
flows. In many electrochemical reactions \emph{both} positive and negative
charges flow. We will stick with the convention that the current direction
is the direction that positive charges would flow regardless of the actual
charge carrier sign.\FRAME{dtbpFU}{2.3212in}{2.4339in}{0pt}{\Qcb{Flow of
positive charge through a gate into a neural cell. }}{}{Figure}{\special%
{language "Scientific Word";type "GRAPHIC";maintain-aspect-ratio
TRUE;display "USEDEF";valid_file "T";width 2.3212in;height 2.4339in;depth
0pt;original-width 2.3372in;original-height 2.4507in;cropleft "0";croptop
"1";cropright "1";cropbottom "0";tempfilename
'MJI3E100.wmf';tempfile-properties "XPR";}}

%TCIMACRO{%
%\TeXButton{Basic Equations}{\hspace{-1.3in}{\LARGE Basic Equations\vspace{0.25in}}}}%
%BeginExpansion
\hspace{-1.3in}{\LARGE Basic Equations\vspace{0.25in}}%
%EndExpansion

Voltage if a dielectric is placed between the plates of the
capacitor(equation \ref{DielectricVoltageChange})%
\begin{equation*}
\Delta V=\frac{\Delta V_{o}}{\kappa }
\end{equation*}

Capacitance increases (equation \ref{DielectricVoltageChange})%
\begin{equation*}
C=\kappa C_{o}
\end{equation*}
For parallel plate capacitors we get%
\begin{equation*}
C=\kappa \varepsilon _{o}\frac{A}{d}
\end{equation*}

The induced field in a dielectric is (equation \ref{DielectricInducedField}) 
\begin{equation*}
E=E_{o}-E_{ind}
\end{equation*}

Current is the rate of charge flow (equation \ref{CurrentAsRate})%
\begin{equation*}
\frac{\Delta Q}{\Delta t}=I
\end{equation*}%
Definition of current (equation \ref{CurrentDefenition})%
\begin{equation*}
I=nAv_{d}q_{c}
\end{equation*}

\chapter{Current, Resistance, and Electric Fields}

%TCIMACRO{%
%\TeXButton{Fundamental Concepts}{\hspace{-1.3in}{\LARGE Fundamental Concepts\vspace{0.25in}}}}%
%BeginExpansion
\hspace{-1.3in}{\LARGE Fundamental Concepts\vspace{0.25in}}%
%EndExpansion

\begin{itemize}
\item There is a nonconservative (friction-like) force involved in current
flow called \emph{resistance}.

\item A nonuniform charge distribution creates an electric field, which
provides the force that makes current flow

\item Current flow direction is defined to be the direction positive charge
carriers would go

\item The current density is defined as $J=nq_{e}v_{d}$

\item Charge is conserved, so in a circuit, current is conserved.
\end{itemize}

\section{Current and resistance}

%TCIMACRO{%
%\TeXButton{Question 223.36.1}{\marginpar {
%\hspace{-0.5in}
%\begin{minipage}[t]{1in}
%\small{Question 223.36.1}
%\end{minipage}
%}}}%
%BeginExpansion
\marginpar {
\hspace{-0.5in}
\begin{minipage}[t]{1in}
\small{Question 223.36.1}
\end{minipage}
}%
%EndExpansion
We now have flowing charges, but our PH121 or Dynamics experience tells us
that there is more. If we push or pull an object, we expect that most of the
time there will be dissipative forces. There will be friction. \FRAME{dhF}{%
3.0467in}{3.1401in}{0pt}{}{}{Figure}{\special{language "Scientific
Word";type "GRAPHIC";maintain-aspect-ratio TRUE;display "USEDEF";valid_file
"T";width 3.0467in;height 3.1401in;depth 0pt;original-width
3.0035in;original-height 3.0943in;cropleft "0";croptop "1";cropright
"1";cropbottom "0";tempfilename 'LUGKQ400.wmf';tempfile-properties "XPR";}}

We should ask, is there a friction involved in charge movement? 
%TCIMACRO{%
%\TeXButton{Question 223.36.2}{\marginpar {
%\hspace{-0.5in}
%\begin{minipage}[t]{1in}
%\small{Question 223.36.2}
%\end{minipage}
%}}}%
%BeginExpansion
\marginpar {
\hspace{-0.5in}
\begin{minipage}[t]{1in}
\small{Question 223.36.2}
\end{minipage}
}%
%EndExpansion
We already know how to push a charge, we use an electric field\FRAME{dhF}{%
2.0081in}{2.3653in}{0pt}{}{}{Figure}{\special{language "Scientific
Word";type "GRAPHIC";maintain-aspect-ratio TRUE;display "USEDEF";valid_file
"T";width 2.0081in;height 2.3653in;depth 0pt;original-width
8.3558in;original-height 9.8537in;cropleft "0";croptop "1";cropright
"1";cropbottom "0";tempfilename 'LUGKV101.wmf';tempfile-properties "XPR";}}%
The force is 
\begin{equation*}
F=qE
\end{equation*}%
If we push or pull a box, it will eventually come to rest. In our capacitor
there are no resistive forces for our charge to encounter. But suppose we
place a conductor inside our capacitor, hooked to both plates\FRAME{dhF}{%
2.1387in}{2.0807in}{0pt}{}{}{Figure}{\special{language "Scientific
Word";type "GRAPHIC";maintain-aspect-ratio TRUE;display "USEDEF";valid_file
"T";width 2.1387in;height 2.0807in;depth 0pt;original-width
4.3517in;original-height 4.2324in;cropleft "0";croptop "1";cropright
"1";cropbottom "0";tempfilename 'LUGL2U02.wmf';tempfile-properties "XPR";}}%
Of course, in conductors we now know the charge carrier is an electron and
it is negative, so let's try to redraw this picture to show the actual
charge motion.

%TCIMACRO{%
%\TeXButton{Question 223.36.3}{\marginpar {
%\hspace{-0.5in}
%\begin{minipage}[t]{1in}
%\small{Question 223.36.3}
%\end{minipage}
%}}}%
%BeginExpansion
\marginpar {
\hspace{-0.5in}
\begin{minipage}[t]{1in}
\small{Question 223.36.3}
\end{minipage}
}%
%EndExpansion
Now the charge is free to move inside of the conductor, but it is not
totally unencumbered. The free charges will run into the nuclei of the
atoms. The charges will bounce off. So as they travel through the material
we will expect to see some randomness to their motion. This is compounded by
the fact that the electrons already have random thermal motion. So the path
the charge takes looks somewhat like this

\FRAME{dhF}{3.0606in}{2.6368in}{0in}{}{}{Figure}{\special{language
"Scientific Word";type "GRAPHIC";maintain-aspect-ratio TRUE;display
"USEDEF";valid_file "T";width 3.0606in;height 2.6368in;depth
0in;original-width 3.0173in;original-height 2.5953in;cropleft "0";croptop
"1";cropright "1";cropbottom "0";tempfilename
'LUGWD106.wmf';tempfile-properties "XPR";}}

We can recognize that each path segment after a collision must be parabolic
because the acceleration will be constant%
\begin{equation*}
F=ma=qE
\end{equation*}%
so%
\begin{equation*}
a=\frac{qE}{m}
\end{equation*}%
we can describe the electron motion using the two of the kinematic equations 
\begin{eqnarray*}
x_{f} &=&x_{i}+v_{ix}\Delta t+\frac{1}{2}a_{x}\Delta t^{2} \\
v_{fx} &=&v_{ix}+a_{x}\Delta t
\end{eqnarray*}%
and the path will be%
\begin{equation*}
x_{f}=x_{i}+v_{ox}\Delta t+\frac{1}{2}\left( \frac{qE}{m}\right) \Delta t^{2}
\end{equation*}%
which is parabolic.

Of course, this is just for one electron, and only for a segment between
collisions. We will have millions of electrons, and therefore, many millions
of bounces. But for each electron, between bounces we expect a parabolic
path. For considering current flow, we don't care about motion perpendicular
to the current direction. So we can look only at the component of the motion
in the flow direction. The net flow in the current direction is toward the
positive plate. Let's see how this works.

%TCIMACRO{%
%\TeXButton{Question 223.36.4}{\marginpar {
%\hspace{-0.5in}
%\begin{minipage}[t]{1in}
%\small{Question 223.36.4}
%\end{minipage}
%}}}%
%BeginExpansion
\marginpar {
\hspace{-0.5in}
\begin{minipage}[t]{1in}
\small{Question 223.36.4}
\end{minipage}
}%
%EndExpansion
If we average the velocities of all the electrons we find 
\begin{eqnarray*}
v_{d} &=&\bar{v}_{x} \\
&=&\bar{v}_{ix}+a_{x}\Delta \bar{t}
\end{eqnarray*}%
the first term $\bar{v}_{ix}=0$ because the initial velocities are random
from the thermal and scattering processes. That is, on average, the
electrons have no preferred direction after a bounce. This leaves%
\begin{equation*}
v_{d}=\left( \frac{qE}{m}\right) \Delta \bar{t}
\end{equation*}%
The average time between collisions, $\Delta \bar{t}$, is sometimes given
the symbol $\tau .$ Let's use this. Recall that current is 
\begin{equation*}
I=nAv_{d}q_{c}
\end{equation*}%
Then 
\begin{equation*}
v_{d}=\left( \frac{q\tau }{m}\right) E
\end{equation*}%
and we can write our current equation as 
\begin{equation*}
I=nqA\left( \frac{q\tau }{m}\right) E
\end{equation*}%
We have shown that the current is directly proportional to the field inside
the conductor. It is this field that causes the charges to flow.

%TCIMACRO{%
%\TeXButton{Question 223.36.5}{\marginpar {
%\hspace{-0.5in}
%\begin{minipage}[t]{1in}
%\small{Question 223.36.5}
%\end{minipage}
%}}}%
%BeginExpansion
\marginpar {
\hspace{-0.5in}
\begin{minipage}[t]{1in}
\small{Question 223.36.5}
\end{minipage}
}%
%EndExpansion
But let's look even closer. Suppose we connect our two plates with a wire
instead of filling their gap with a conductor. If current flows through the
wire, there must be a field in the wire. But how does it get started?\FRAME{%
dtbpF}{3.7965in}{2.5469in}{0pt}{}{}{Figure}{\special{language "Scientific
Word";type "GRAPHIC";maintain-aspect-ratio TRUE;display "USEDEF";valid_file
"T";width 3.7965in;height 2.5469in;depth 0pt;original-width
3.7481in;original-height 2.5054in;cropleft "0";croptop "1";cropright
"1";cropbottom "0";tempfilename 'M5X9PI00.wmf';tempfile-properties "XPR";}}%
This figure is supposed to show our wire connected to the capacitor. The
capacitor is in the background, and the wire loops close to us. The end of
the wire that is connected to the positive side of the capacitor will become
positively charged, and the end connected to the negative side of the
capacitor will become negatively charged. But if we look at the wire an
infinitesimal time after the connection has happened, the wire will not be
uniformly charged. It will take some time for the charges to reach
equilibrium. In the mean time, the charge is stronger near the plates, and
diminishes toward the middle.

We can't find the exact field in the conductor without resorting to a
computational solution, but we can mentally model the situation by viewing
the wire as consisting of rings of charge that vary in linear charge
density. We know the field along the axis due to a ring of charge because we
have done this problem in the past. \FRAME{dhF}{3.6487in}{1.9112in}{0pt}{}{}{%
Figure}{\special{language "Scientific Word";type
"GRAPHIC";maintain-aspect-ratio TRUE;display "USEDEF";valid_file "T";width
3.6487in;height 1.9112in;depth 0pt;original-width 3.6011in;original-height
1.8732in;cropleft "0";croptop "1";cropright "1";cropbottom "0";tempfilename
'M0KTH40A.wmf';tempfile-properties "XPR";}}%
\begin{equation*}
\overrightarrow{\mathbf{E}}=\frac{1}{4\pi \epsilon _{o}}\frac{zQ}{\left(
R^{2}+z^{2}\right) ^{\frac{3}{2}}}\mathbf{\hat{k}}
\end{equation*}%
We know the field is along the axis and that it diminishes with distance
from the ring. Now consider the field due to ring $1.$ As we move to the
right, away from ring $1$ that field will diminish with distance. Also
consider the field due to ring $2.$ As we move to the right toward ring $2$
the field due to ring two will grow. The field due to ring two grows at the
same rate that the field from ring 1 diminishes. The fields $1,$ and $2$ add
up to a constant value along the axis for every point in between the two
rings. Now consider the field on the right side of ring $2$ and the field on
the left side of ring $3$. A little thought shows that the situation is the
same as that for rings $1$ and $2$. We will have a constant net field
between the two rings. \FRAME{dtbpF}{4.0387in}{2.7095in}{0pt}{}{}{Figure}{%
\special{language "Scientific Word";type "GRAPHIC";maintain-aspect-ratio
TRUE;display "USEDEF";valid_file "T";width 4.0387in;height 2.7095in;depth
0pt;original-width 3.9894in;original-height 2.6671in;cropleft "0";croptop
"1";cropright "1";cropbottom "0";tempfilename
'M5XAFG01.wmf';tempfile-properties "XPR";}}Likewise for the region between
rings $3$ and $4.$ There is a constant net electric field at all points
along the wire. This field points from positive to negative. It will exert a
force 
\begin{equation*}
F=qE_{net}
\end{equation*}%
on the free charges \emph{inside} the wire. These free charges are not extra
charge. They are the free electrons that are loosely attached to the metal
atoms that make up the wire. So these free charges are distributed
throughout the volume of the wire. These free charges will accelerate,
forming a current inside the wire.

Note that these free charges are not just on the surface, they are inside
the wire, even on the axis of the wire in the center. We no longer have a
static equilibrium, so we no longer have excess charge only on the surface.

All this usually happens very fast, so when we switch on a light, we don't
notice the time it takes for the current to start. But this uneven
distribution of charge is the reason we get a current.

\section{Current density}

%TCIMACRO{%
%\TeXButton{Question 223.36.6}{\marginpar {
%\hspace{-0.5in}
%\begin{minipage}[t]{1in}
%\small{Question 223.36.6}
%\end{minipage}
%}}}%
%BeginExpansion
\marginpar {
\hspace{-0.5in}
\begin{minipage}[t]{1in}
\small{Question 223.36.6}
\end{minipage}
}%
%EndExpansion
We now realize that when there is an electric field inside a wire, there
will be current flow inside the wire. The flow goes through the volume of
the wire.

The rate of flow is given by%
\begin{equation*}
I=\frac{\Delta Q}{\Delta t}=nq_{e}A\left( \frac{q_{e}\tau }{m_{e}}\right) E
\end{equation*}%
for steady current flow. Here we are writing $q=q_{e}$ for the electron
charge and $m=m_{e}$ for the electron mass, since our charge carrier is an
electron..

The unit for current flow is 
\begin{equation*}
\frac{\unit{C}}{\unit{s}}=\unit{A}
\end{equation*}%
where $\unit{A}$ is the symbol for an \emph{Ampere} or, for short, an \emph{%
amp}.

%TCIMACRO{%
%\TeXButton{Question 223.36.7}{\marginpar {
%\hspace{-0.5in}
%\begin{minipage}[t]{1in}
%\small{Question 223.36.7}
%\end{minipage}
%}}}%
%BeginExpansion
\marginpar {
\hspace{-0.5in}
\begin{minipage}[t]{1in}
\small{Question 223.36.7}
\end{minipage}
}%
%EndExpansion
Historically there was no way to tell whether negative charges were flowing
or whether positive charges were flowing. It really did not matter so much
in the early days, since a flow of positive charges one way is equivalent to
a flow of negative charges the other way.\FRAME{dhF}{3.6685in}{3.1894in}{0pt%
}{}{}{Figure}{\special{language "Scientific Word";type
"GRAPHIC";maintain-aspect-ratio TRUE;display "USEDEF";valid_file "T";width
3.6685in;height 3.1894in;depth 0pt;original-width 4.907in;original-height
4.2601in;cropleft "0";croptop "1";cropright "1";cropbottom "0";tempfilename
'LUGVPA02.wmf';tempfile-properties "XPR";}}

Worse, we know that for some systems there are positive charge carriers and
for others negative charge carriers.

\textbf{By convention, we assign the direction of current flow as though the
charge carrier were positive. }

This is great for biologists, where the charge carriers are positive ions.
But for electronics this gives us the uncomfortable situation that the
actual charge carriers, electrons, move in the direction opposite to that of
the current.\FRAME{dhF}{3.4385in}{1.5627in}{0pt}{}{}{Figure}{\special%
{language "Scientific Word";type "GRAPHIC";maintain-aspect-ratio
TRUE;display "USEDEF";valid_file "T";width 3.4385in;height 1.5627in;depth
0pt;original-width 3.3927in;original-height 1.5273in;cropleft "0";croptop
"1";cropright "1";cropbottom "0";tempfilename
'LUGW1S03.wmf';tempfile-properties "XPR";}}Let's look again at our
definition of current%
\begin{equation*}
I=\frac{\Delta Q}{\Delta t}=nq_{e}A\left( \frac{q_{e}\tau }{m_{e}}\right) E
\end{equation*}%
If we, once again, write this in terms of $v_{d}$ 
\begin{equation*}
v_{d}=\left( \frac{q\tau }{m}\right) E
\end{equation*}%
then after rearranging, we have 
\begin{equation*}
I=\left( nq_{e}v_{d}\right) A
\end{equation*}%
%TCIMACRO{%
%\TeXButton{Question 223.36.8}{\marginpar {
%\hspace{-0.5in}
%\begin{minipage}[t]{1in}
%\small{Question 223.36.8}
%\end{minipage}
%}}}%
%BeginExpansion
\marginpar {
\hspace{-0.5in}
\begin{minipage}[t]{1in}
\small{Question 223.36.8}
\end{minipage}
}%
%EndExpansion
the part in parentheses contains only bulk properties of the conductor
material, the number of free charges, the charge of the charge carrier, and
the drift speed which depends on the material structure of the conductor.
The final factor is just the cross sectional area of the wire. It gives the
geometry of the wire we have made out of the bulk material (say, copper). It
is convenient to group all the factors that are due to bulk material
properties%
\begin{equation*}
J=nq_{e}v_{d}
\end{equation*}%
then the current would be%
\begin{equation*}
I=JA
\end{equation*}%
Note how similar this is to a surface charge density%
\begin{equation*}
Q=\eta A
\end{equation*}%
For a static charged surface, $Q$ is the surface charge density multiplied
by the particular area. For our case we have a total current, $I$ that is
the material properties multiplied by an area. By analogy we could call this
new quantity, $J,$ a kind of density, but now our charges are moving. So
let's call it the \emph{current density.}

Notice that it is the cross sectional area of the wire that shows up in our
current equation. This is another indication that the charge is not flowing
along the surface, but that it is deep within the wire as it flows.

\section{Conservation of current}

%TCIMACRO{%
%\TeXButton{Question 223.36.9}{\marginpar {
%\hspace{-0.5in}
%\begin{minipage}[t]{1in}
%\small{Question 223.36.9}
%\end{minipage}
%}}}%
%BeginExpansion
\marginpar {
\hspace{-0.5in}
\begin{minipage}[t]{1in}
\small{Question 223.36.9}
\end{minipage}
}%
%EndExpansion
Let's go back to our pumps and turbines.\FRAME{dhF}{4.7853in}{2.1376in}{0pt}{%
}{}{Figure}{\special{language "Scientific Word";type
"GRAPHIC";maintain-aspect-ratio TRUE;display "USEDEF";valid_file "T";width
4.7853in;height 2.1376in;depth 0pt;original-width 4.8474in;original-height
2.1492in;cropleft "0";croptop "1";cropright "1";cropbottom "0";tempfilename
'LUGXP508.wmf';tempfile-properties "XPR";}}

How much of the water is \textquotedblleft used up\textquotedblright\ in
turning the turbine? Another way to say this is to ask if there are $20\unit{%
l}$ of water entering the turbine, how much water leaves the turbine through
the lower pipe?

If the turbine leaks, then we might lose some water, but if all is going
well, then you can guess that $20\unit{l}$ of water must also leave the
turbine We can't lose or gain water as the turbine is turned. But we must be
losing something! 
%TCIMACRO{%
%\TeXButton{Question 223.36.10}{\marginpar {
%\hspace{-0.5in}
%\begin{minipage}[t]{1in}
%\small{Question 223.36.10}
%\end{minipage}
%}}}%
%BeginExpansion
\marginpar {
\hspace{-0.5in}
\begin{minipage}[t]{1in}
\small{Question 223.36.10}
\end{minipage}
}%
%EndExpansion
We must be giving up something to get useful work out of the system. That
something that we lose is potential energy.

Now consider a battery. How much of the current is \textquotedblleft used
up\textquotedblright\ in making the light bulb light up?\FRAME{dhF}{2.127in}{%
1.6906in}{0pt}{}{}{Figure}{\special{language "Scientific Word";type
"GRAPHIC";maintain-aspect-ratio TRUE;display "USEDEF";valid_file "T";width
2.127in;height 1.6906in;depth 0pt;original-width 2.1385in;original-height
1.6941in;cropleft "0";croptop "1";cropright "1";cropbottom "0";tempfilename
'LUGXP509.wmf';tempfile-properties "XPR";}}This case is really the same as
the water case. The electric current is a flow of electrons. The flow loses
potential energy, but we don't create or destroy electrons as we convert the
potential energy of the battery to useful work (like making light) just like
we did not create or destroy water in making the turbine turn.

But surely the water slowed down as it traveled through the turbine--didn't
it? Well, no, if the water slows down as it goes through the turbine, then
the pipe below the turbine would run dry. This does not happen. The flow
rate through a pipe does not change under normal conditions, and under
abnormal conditions, we would destroy the pump or the turbine! If we throw
rocks off a hill, they actually gain speed when the water loses potential
energy. Now the flow rate is slower with a turbine in the pipe than it would
be with no turbine in the pipe! But with the turbine in the pipe, the flow
rate is the same throughout the whole pipe system.

Like the water case, the flow rate of charge does not change from point to
point in the wire. The same amount of charge per unit time leaves the wire
as went in.

This explains the reasoning behind one of the great laws of electronics

\begin{Note}
The current is the same at all points in a current-carrying wire.
\end{Note}

Like in the water case, the electrons would flow faster if there were no
light bulb and just a continuous wire. We can have different flow rates in
our wire depending on how much resistance there is to the flow. But the flow
rate will be the same in all parts of the wire system.

This leads to the second of the pair of rules called Kirchhoff's laws:

\begin{equation*}
\sum I_{in}=\sum I_{out}
\end{equation*}%
If the wire branches into two or more pieces, the current will divide. This
is not too surprising. The same is true for water in a pipe%
%TCIMACRO{%
%\TeXButton{Question 223.36.11}{\marginpar {
%\hspace{-0.5in}
%\begin{minipage}[t]{1in}
%\small{Question 223.36.11}
%\end{minipage}
%}}}%
%BeginExpansion
\marginpar {
\hspace{-0.5in}
\begin{minipage}[t]{1in}
\small{Question 223.36.11}
\end{minipage}
}%
%EndExpansion
\FRAME{dtbpF}{3.7498in}{2.3921in}{0pt}{}{}{Figure}{\special{language
"Scientific Word";type "GRAPHIC";maintain-aspect-ratio TRUE;display
"USEDEF";valid_file "T";width 3.7498in;height 2.3921in;depth
0pt;original-width 3.7023in;original-height 2.3514in;cropleft "0";croptop
"1";cropright "1";cropbottom "0";tempfilename
'M5XB2U02.wmf';tempfile-properties "XPR";}}

In the figure the flow through pipe segment $A$ is split into two smaller
currents that flow through pipe segments $B$ and $C.$ We would expect that
the flow through $B$ and $C$ combined

must be equal to the flow through $A.$

%TCIMACRO{%
%\TeXButton{Question 223.36.12}{\marginpar {
%\hspace{-0.5in}
%\begin{minipage}[t]{1in}
%\small{Question 223.36.12}
%\end{minipage}
%}}}%
%BeginExpansion
\marginpar {
\hspace{-0.5in}
\begin{minipage}[t]{1in}
\small{Question 223.36.12}
\end{minipage}
}%
%EndExpansion
The same must be true for electrical current. The situation is shown in the
next figure.

\FRAME{dhF}{2.3341in}{1.6604in}{0in}{}{}{Figure}{\special{language
"Scientific Word";type "GRAPHIC";maintain-aspect-ratio TRUE;display
"USEDEF";valid_file "T";width 2.3341in;height 1.6604in;depth
0in;original-width 2.2943in;original-height 1.6233in;cropleft "0";croptop
"1";cropright "1";cropbottom "0";tempfilename
'LUI8GK00.wmf';tempfile-properties "XPR";}}

The current that flows through wires $B$ and $C$ combined must be equal to
the current that came through wire $A.$%
%TCIMACRO{%
%\TeXButton{Question 223.36.13}{\marginpar {
%\hspace{-0.5in}
%\begin{minipage}[t]{1in}
%\small{Question 223.36.13}
%\end{minipage}
%}}}%
%BeginExpansion
\marginpar {
\hspace{-0.5in}
\begin{minipage}[t]{1in}
\small{Question 223.36.13}
\end{minipage}
}%
%EndExpansion

%TCIMACRO{%
%\TeXButton{Basic Equations}{\hspace{-1.3in}{\LARGE Basic Equations\vspace{0.25in}}}}%
%BeginExpansion
\hspace{-1.3in}{\LARGE Basic Equations\vspace{0.25in}}%
%EndExpansion

\chapter{Ohm's law}

%TCIMACRO{%
%\TeXButton{Fundamental Concepts}{\hspace{-1.3in}{\LARGE Fundamental Concepts\vspace{0.25in}}}}%
%BeginExpansion
\hspace{-1.3in}{\LARGE Fundamental Concepts\vspace{0.25in}}%
%EndExpansion

\begin{itemize}
\item The material property of a conductor that tells us how well the
conductor material will allow current to flow through it is called the
conductivity

\item The inverse of conductivity is the resistivity

\item Resistivity may be temperature dependent

\item Resistance depends on the resistivity of the material and the geometry
of the conductor piece. For a wire it is given by $R=\rho A/L$

\item For many conductors, the change in voltage across the conductor is
proportional to the current and the resistance. This is called Ohm's law

\item The ideal voltage delivered by a battery is called the
\textquotedblleft emf\textquotedblright\ and is given the symbol $\mathcal{E}
$

\item Some materials do not follow Ohm's law. They are called nonohmic

\item The Earth has a magnetic field

\item Magnets have \textquotedblleft magnetic charge
centers\textquotedblright\ called poles and there is a magnetic field.

\item Magnetic poles don't seem to exist independently
\end{itemize}

\section{Conductivity and resistivity}

%TCIMACRO{%
%\TeXButton{Question 223.37.1}{\marginpar {
%\hspace{-0.5in}
%\begin{minipage}[t]{1in}
%\small{Question 223.37.1}
%\end{minipage}
%}}}%
%BeginExpansion
\marginpar {
\hspace{-0.5in}
\begin{minipage}[t]{1in}
\small{Question 223.37.1}
\end{minipage}
}%
%EndExpansion
We defined the current density last lecture 
\begin{equation*}
J=nq_{e}v_{d}
\end{equation*}%
but we know that the drift speed is 
\begin{equation*}
v_{d}=\left( \frac{q_{e}\tau }{m_{e}}\right) E
\end{equation*}%
so we can write the current density as%
\begin{eqnarray*}
J &=&nq_{e}\left( \frac{q_{e}\tau }{m_{e}}\right) E \\
&=&\left( \frac{nq_{e}^{2}\tau }{m_{e}}\right) E
\end{eqnarray*}%
The factor in parentheses depends only on the properties of the conducting
material. For example, if the material is copper, then we would have the $%
n_{copper}=8.5\times 10^{28}\frac{1}{\unit{m}^{3}}$ as the number of valence
electrons per unit meter cubed for copper. The mean time between collisions
is something like $\tau _{copper}=2.5\times 10^{-14}\unit{s}.$ So our
quantity in parentheses is 
\begin{eqnarray*}
\left( \frac{nq_{e}^{2}\tau }{m_{e}}\right) &=&\frac{\left( 8.5\times 10^{28}%
\frac{1}{\unit{m}^{3}}\right) \left( 1.6\times 10^{-19}\unit{C}\right)
^{2}\left( 2.5\times 10^{-14}\unit{s}\right) }{9.11\times 10^{-31}\unit{kg}}
\\
&=&5.\,\allowbreak 971\,5\times 10^{7}\frac{\unit{A}^{2}}{\unit{m}^{3}}\frac{%
\unit{s}^{3}}{\unit{kg}} \\
&=&5.\,\allowbreak 971\,5\times 10^{7}\frac{1}{\unit{%
%TCIMACRO{\U{3a9}}%
%BeginExpansion
\Omega%
%EndExpansion
}\unit{m}}
\end{eqnarray*}%
$\allowbreak $

The field is due to something outside of the conducting material (e.g. the
battery). Notice that again we have grouped all the properties of the
material together. Lets give a name to the quantity in parentheses that
contains all the material properties. Since this quantity tells us how
easily the charges will go through the conductive material, we can call this
the \emph{conductivity} of the material.%
\begin{equation*}
\sigma =\frac{nq_{e}^{2}\tau }{m_{e}}
\end{equation*}%
Then 
\begin{equation*}
J=\sigma E
\end{equation*}%
The current density depends on two things, how well the material can allow
the current to flow (bulk material properties related to conduction), $%
\sigma ,$ and and the field that motivates the current to flow, $E.$

The current, then, depends on these two items, as well as the cross
sectional area of the wire%
\begin{eqnarray*}
I &=&JA \\
&=&\sigma EA
\end{eqnarray*}

Really, the conductivity is more complicated than it appears. The mean time
between collisions, $\tau $, depends on the structure of the conductor.
Different crystalline structures for the same element will give different
values. Think of trying to walk quickly through the Manwering Center crowds
during a class break. This takes some maneuvering. But if all the people
were placed at equally spaced, regular intervals, it might be easier to make
it through quickly. It would also be easier if the crowd stood still.
Likewise, the position of the atoms in the conductor make a big difference
in the conductivity, and thermal motion of those atoms also makes a large
difference. We would expect the conductivity to depend on the temperature of
the material.

\subsection{Resistivity}

%TCIMACRO{%
%\TeXButton{Question 223.37.2}{\marginpar {
%\hspace{-0.5in}
%\begin{minipage}[t]{1in}
%\small{Question 223.37.2}
%\end{minipage}
%}}}%
%BeginExpansion
\marginpar {
\hspace{-0.5in}
\begin{minipage}[t]{1in}
\small{Question 223.37.2}
\end{minipage}
}%
%EndExpansion
It is common to speak of the opposite of the concept of conductance. In
other words, how hard it is to get the electrons to travel through the
conductive material. For example, we might want to build a heating device,
like a toaster or space heater. In this case, we want friction in the wires,
because that friction will produce thermal energy. So specifying a
conductive material by how much friction it has is useful. How much the
material impedes the flow of current is the opposite of how much the
material allows the current flow, so we expect this new quantity to be the
inverse of our conductivity%
\begin{equation*}
\rho =\frac{1}{\sigma }=\frac{m_{e}}{nq_{e}^{2}\tau }
\end{equation*}%
Special conductors are often made that use \textquotedblleft
impurities,\textquotedblright\ that is, trace amounts of other atoms, to
increase or decrease the resistivity of those conducive materials. The
thermal dependence can be modeled using the equation

\begin{equation*}
\rho =\rho _{o}\left( 1+\alpha \left( T-T_{o}\right) \right)
\end{equation*}%
where $\rho _{o}$ is the resistivity at some reference temperature (usually $%
20\unit{%
%TCIMACRO{\U{2103}}%
%BeginExpansion
{}^{\circ}{\rm C}%
%EndExpansion
}$) and $\alpha $ is a constant that tells us how our particular material
changes resistance with temperature. It is kind of like the specific heat in
thermodynamics $Q=C\Delta T$. This is an approximation. It is a curve fit
that works over normal temperatures. But we would not expect the same
resistive properties, say, if we melt the material. The position of the
atoms would change if the material goes from solid to liquid. So we will
need to be careful in how we use this formula.

Here are some values of the conductivity, resistivity, and temperature
coefficients for a few common conductive materials.%
%TCIMACRO{%
%\TeXButton{Question 223.37.3}{\marginpar {
%\hspace{-0.5in}
%\begin{minipage}[t]{1in}
%\small{Question 223.37.3}
%\end{minipage}
%}}}%
%BeginExpansion
\marginpar {
\hspace{-0.5in}
\begin{minipage}[t]{1in}
\small{Question 223.37.3}
\end{minipage}
}%
%EndExpansion

\begin{equation*}
\begin{tabular}{|c|c|c|c|}
\hline
{\small Material} & 
\begin{tabular}{c}
{\small Conductivity} \\ 
$\left( \unit{%
%TCIMACRO{\U{3a9}}%
%BeginExpansion
\Omega%
%EndExpansion
}^{-1}\unit{m}^{-1}\right) $%
\end{tabular}
& 
\begin{tabular}{c}
{\small Resistivity} \\ 
$\left( \unit{%
%TCIMACRO{\U{3a9}}%
%BeginExpansion
\Omega%
%EndExpansion
}\unit{m}\right) $%
\end{tabular}
& 
\begin{tabular}{c}
{\small Temp. Coeff.} \\ 
$\left( \unit{K}^{-1}\right) $%
\end{tabular}
\\ \hline
{\small Aluminum} & ${\small 3.5\times 10}^{7}$ & ${\small 2.8\times 10}%
^{-8} $ & $3.9\times 10^{-3}$ \\ \hline
{\small Copper} & ${\small 6.0\times 10}^{7}$ & ${\small 1.7\times 10}^{-8}$
& $3.9\times 10^{-3}$ \\ \hline
{\small Gold} & ${\small 4.1\times 10}^{7}$ & ${\small 2.4\times 10}^{-8}$ & 
$3.4\times 10^{-3}$ \\ \hline
{\small Iron} & ${\small 1.0\times 10}^{7}$ & ${\small 9.7\times 10}^{-8}$ & 
$5.0\times 10^{-3}$ \\ \hline
{\small Silver} & ${\small 6.2\times 10}^{7}$ & ${\small 1.6\times 10}^{-8}$
& $3.8\times 10^{-3}$ \\ \hline
{\small Tungsten} & ${\small 1.8\times 10}^{7}$ & ${\small 5.6\times 10}%
^{-8} $ & $4.5\times 10^{-3}$ \\ \hline
{\small Nichrome} & ${\small 6.7\times 10}^{5}$ & ${\small 1.5\times 10}%
^{-6} $ & $0.4\times 10^{-3}$ \\ \hline
{\small Carbon} & ${\small 2.9\times 10}^{4}$ & ${\small 3.5\times 10}^{-5}$
& $-0.5\times 10^{-3}$ \\ \hline
\end{tabular}%
\end{equation*}

\subsection{Superconductivity}

The relationship%
\begin{equation*}
\rho =\rho _{o}\left( 1+\alpha \left( T-T_{o}\right) \right)
\end{equation*}%
also breaks down at low temperatures. The low end is very important these
days. For some special materials, the resistivity goes to zero when the
material is cold enough. We call these materials superconductors. A
superconductor can carry huge currents, because there is no loss of energy,
and no heat generated without any friction. Unfortunately most
superconducting materials only operate at temperatures near absolute zero.
But a few \textquotedblleft high temperature\textquotedblright\
superconductors operate at temperatures at high as $125\unit{K}.$ This is
still very cold ($-150\unit{%
%TCIMACRO{\U{2103}}%
%BeginExpansion
{}^{\circ}{\rm C}%
%EndExpansion
}$), but these temperatures are achievable, so some superconducting products
are possible. As you can guess, there is very active research in making
superconductors that operate at even higher temperatures.\FRAME{dtbpFU}{%
3.0743in}{2.2485in}{0pt}{\Qcb{{\protect\small Superconducting fiber material
and superconducting magnet at CERN. These superconductors operate at 1.9K.}}%
}{\Qlb{CERNSuperconductingMagnets}}{Figure}{\special{language "Scientific
Word";type "GRAPHIC";maintain-aspect-ratio TRUE;display "USEDEF";valid_file
"T";width 3.0743in;height 2.2485in;depth 0pt;original-width
3.1035in;original-height 2.2636in;cropleft "0";croptop "1";cropright
"1";cropbottom "0";tempfilename 'N2CCNO00.wmf';tempfile-properties "XPR";}}

\subsection{Ohm's law}

Let's pause to review, Current density id given by%
\begin{equation*}
J=\sigma E
\end{equation*}%
or now by 
\begin{equation*}
J=\frac{1}{\rho }E
\end{equation*}

Then the current is given by%
\begin{eqnarray*}
I &=&JA \\
&=&\frac{A}{\rho }E
\end{eqnarray*}%
If the field is similar to our capacitor field, nearly uniform in our
conducting wire, then the potential would be just%
\begin{eqnarray*}
\Delta V &=&Ed \\
&=&E\Delta s
\end{eqnarray*}%
and then the electric field is approximately given by 
\begin{equation*}
E=\frac{\Delta V}{\Delta s}
\end{equation*}%
For our wire of length $L$ this is 
\begin{equation*}
E=\frac{\Delta V}{L}
\end{equation*}%
Then we can use this field to write our current%
\begin{equation*}
I=\frac{A}{\rho }\frac{\Delta V}{L}
\end{equation*}%
Once again, let's group together all the structural and material properties
of the wire. We have 
\begin{equation*}
I=\left( \frac{A}{L\rho }\right) \Delta V
\end{equation*}%
or with a little algebra, 
\begin{equation*}
\Delta V=I\left( \rho \frac{L}{A}\right)
\end{equation*}%
The part in parenthesis contains all the friction terms. It says that the
longer the wire, the more friction we will experience. This makes sense. If
you are familiar with fluid flow. The longer the hose, the more resistance.
It also says that the larger the area, the lower the friction. That is also
reasonable, since the electrons will have more places to go unrestricted if
the area is bigger. In water flow, the larger the pipe, the less the water
interacts with the sides of the pipe and therefore the lower the friction.
This situation is analogous.

%TCIMACRO{%
%\TeXButton{Question 223.37.4}{\marginpar {
%\hspace{-0.5in}
%\begin{minipage}[t]{1in}
%\small{Question 223.37.4}
%\end{minipage}
%}}}%
%BeginExpansion
\marginpar {
\hspace{-0.5in}
\begin{minipage}[t]{1in}
\small{Question 223.37.4}
\end{minipage}
}%
%EndExpansion
We should give a name to this quantity that describes the frictional
properties of the wire. We will call it the \emph{resistance} of the wire.%
\begin{equation*}
R=\rho \frac{L}{A}
\end{equation*}%
so that we can write%
\begin{equation*}
I=\frac{\Delta V}{R}
\end{equation*}

The resistance has units of 
\begin{equation*}
\frac{\unit{V}}{\unit{A}}=\unit{%
%TCIMACRO{\U{3a9}}%
%BeginExpansion
\Omega%
%EndExpansion
}
\end{equation*}%
where $\unit{%
%TCIMACRO{\U{3a9}}%
%BeginExpansion
\Omega%
%EndExpansion
}$ is given the name of \emph{ohm} after the scientist that did pioneering
work on resistance.

The relationship%
\begin{equation*}
I=\frac{\Delta V}{R}
\end{equation*}%
is called \emph{Ohm's law}.

\subsection{Life History of an electric current}

Let's go back and think about our pump model for a battery.

\FRAME{dhF}{5.2382in}{1.3612in}{0pt}{}{}{Figure}{\special{language
"Scientific Word";type "GRAPHIC";maintain-aspect-ratio TRUE;display
"USEDEF";valid_file "T";width 5.2382in;height 1.3612in;depth
0pt;original-width 6.7014in;original-height 1.7201in;cropleft "0";croptop
"1";cropright "1";cropbottom "0";tempfilename
'LUK3A109.wmf';tempfile-properties "XPR";}}

The pump is a source of potential energy \emph{difference}. This is what a
battery does as well. The battery is a charge pump. It moves the charges
from a low to a high potential. So it is a source of electric potential. The
battery's job is to provide the charge separation that creates the electric
field that drives the free charges, making the current.

A positive charge in the wire on the negative side of the battery is pumped
up to the positive side through a chemical process. We can mentally envision
a small charge pump inside of the battery \FRAME{dhF}{3.1151in}{2.3808in}{0pt%
}{}{}{Figure}{\special{language "Scientific Word";type
"GRAPHIC";maintain-aspect-ratio TRUE;display "USEDEF";valid_file "T";width
3.1151in;height 2.3808in;depth 0pt;original-width 5.0194in;original-height
3.8311in;cropleft "0";croptop "1";cropright "1";cropbottom "0";tempfilename
'LUKCGM04.wmf';tempfile-properties "XPR";}}

The battery is the source of the potential. A positive charge near the
negative side of the battery would be pumped up to the positive side of the
battery, It would gain potential energy 
\begin{equation*}
\Delta U_{battery}=q\Delta V_{battery}
\end{equation*}%
Then it would \textquotedblleft fall\textquotedblright\ down the wire. It
must lose all of the potential energy it gained. So it will loose 
\begin{equation*}
\left\vert \Delta U_{wire}\right\vert =\left\vert \Delta
U_{battery}\right\vert
\end{equation*}%
But if the battery potential energy change is positive, the wire change must
be negative. We can see that 
\begin{equation*}
\Delta V_{wire}=-\Delta V_{battery}
\end{equation*}%
so the potential change in the wire is negative. We sometimes call this a
potential \textquotedblleft drop.\textquotedblright

The field forces our charge to move through this wire much like the
gravitational field forces rocks to fall. The positive charge ends up at the
negative end of the battery again, ready to be pumped up to make another
round.

Of course, really this process goes backwards in electrical circuits, since
our charge carriers are negative, but we recall that mathematically negative
charges going the opposite way is the same.\ So we will make this picture
our mental model of a current.

\subsection{Emf}

We have ignored something in our pump model of a battery. In real water
flow, there would be resistance to the flow even inside the pump. This
resistance would be small, but not zero. So the actual potential energy gain
would be 
\begin{equation*}
\Delta U=\Delta U_{\text{ideal}}-U_{\text{loss due to friction in the pump}}
\end{equation*}%
The same is true for an actual battery. There is some resistance in the
battery, itself.%
\begin{equation*}
\Delta V=\Delta V_{\text{ideal}}-\Delta V_{\text{loss due to resistance in
the battery}}
\end{equation*}%
Now that we have Ohm's law, we can see what $\Delta V_{\text{loss due to
resistance in the battery}}$ would be in terms of the internal resistance of
the battery and the current that flows. Referring to the last figure, there
is only one way for the current to go. So for this circuit, the current must
be the same throughout the entire circuit, even in the battery! If we call
the small resistance in the battery $r,$ then 
\begin{equation*}
\Delta V_{\text{loss due to resistance in the battery}}=Ir
\end{equation*}%
then the actual potential energy provided by the battery is 
\begin{equation*}
\Delta V=\Delta V_{\text{ideal}}-Ir
\end{equation*}%
It is traditional to give the ideal voltage a name and a symbol. And we have
already encountered this name. It is \textquotedblleft
emf.\textquotedblright\ Recall that at one time, the letters `e', `m', and
`f' stood for something. But not any more. It is just a name. It is
pronounced \textquotedblleft \={e}-em-ef,\textquotedblright\ and the symbol
is a script capital $\mathcal{E}$. So we can write 
\begin{equation*}
\Delta V=\mathcal{E}-Ir
\end{equation*}%
Sometimes you will hear $\mathcal{E}$ referred to as the voltage you would
get if the battery is not connected (the \textquotedblleft open
circuit\textquotedblright\ voltage). This is the voltage marked on the
battery. Notice that the actual voltage provided at the battery terminals
depends on how much current is being drawn from the battery. So if you are
draining your battery quickly (say, using your electric starter motor to
start your car engine) the voltage supplied by your battery might drop (your
lights might dim while the starer motor runs). You are not getting $12\unit{V%
}$ because the current $I$ is large while the starter motor runs. We will
change to this new symbol for ideal voltage. But we should keep in mind that
actual voltages delivered may be significantly less than this ideal emf
unless we plan our designs carefully.

\subsection{Ohmic or nonohmic}

%TCIMACRO{%
%\TeXButton{Question 223.37.5}{\marginpar {
%\hspace{-0.5in}
%\begin{minipage}[t]{1in}
%\small{Question 223.37.5}
%\end{minipage}
%}}}%
%BeginExpansion
\marginpar {
\hspace{-0.5in}
\begin{minipage}[t]{1in}
\small{Question 223.37.5}
\end{minipage}
}%
%EndExpansion
This simple model of resistance is great for understanding simple things.
Wires, and resistors do work like this. If we were to take a set of
measurements of $\Delta V$ and $I,$ we expect a straight line%
\begin{eqnarray*}
y &=&mx+b \\
\mathcal{E} &\mathcal{=}&\Delta V=RI+0
\end{eqnarray*}%
where $R$ is the slope. \FRAME{dtbpFX}{1.6457in}{1.4243in}{0pt}{}{}{Plot}{%
\special{language "Scientific Word";type "MAPLEPLOT";width 1.6457in;height
1.4243in;depth 0pt;display "USEDEF";plot_snapshots TRUE;mustRecompute
FALSE;lastEngine "MuPAD";xmin "-5";xmax "5";xviewmin "0.009996";xviewmax
"0.050004";yviewmin "0.189924";yviewmax "0.950076";plottype 4;labeloverrides
3;x-label "I";y-label "V";axesFont "Times New
Roman,12,0000000000,useDefault,normal";numpoints 100;plotstyle
"patch";axesstyle "normal";axestips FALSE;xis \TEXUX{x};var1name
\TEXUX{$x$};function
\TEXUX{$\MATRIX{2,5}{c}\VR{,,c,,,}{,,c,,,}{,,,,,}\HR{,,,,,}\CELL{0.01}%
\CELL{0.19}\CELL{0.02}\CELL{0.38}\CELL{0.03}\CELL{0.57}\CELL{0.04}%
\CELL{0.76}\CELL{0.05}\CELL{0.95}$};linecolor "blue";linestyle 1;pointstyle
"point";linethickness 3;lineAttributes "Solid";curveColor
"[flat::RGB:0x000000ff]";curveStyle "Line";VCamFile
'M0MKIG0B.xvz';valid_file "T";tempfilename
'MJJYIX00.wmf';tempfile-properties "XPR";}}

But there are times when the model fails terribly. An incandescent light
bulb is an example that we can quickly understand. The resistance at any one
moment fulfils Ohm's law%
\begin{equation*}
I=\frac{\mathcal{E}}{R}
\end{equation*}%
but light bulbs get hot. The resistance will change in time. So our
relationship is now time dependent. Starting with the resistivity, 
\begin{equation*}
\rho =\rho _{o}\left( 1+\alpha \left( T-T_{o}\right) \right)
\end{equation*}%
let's multiply both sides by $A/L.$ 
\begin{equation*}
\frac{A}{L}\rho =\frac{A}{L}\rho _{o}\left( 1+\alpha \left( T-T_{o}\right)
\right)
\end{equation*}%
this gives 
\begin{equation*}
R=R_{o}\left( 1+\alpha \left( T-T_{o}\right) \right)
\end{equation*}%
So if the resistance is temperature dependent, the slope of the line will
change as we go along making measurements. We might get something like this

\FRAME{dtbpFX}{1.6457in}{1.4243in}{0pt}{}{}{Plot}{\special{language
"Scientific Word";type "MAPLEPLOT";width 1.6457in;height 1.4243in;depth
0pt;display "USEDEF";plot_snapshots TRUE;mustRecompute FALSE;lastEngine
"MuPAD";xmin "-5";xmax "5";xviewmin "0.009996";xviewmax "0.050004";yviewmin
"0.18991117";yviewmax "1.07838883";plottype 4;labeloverrides 3;x-label
"I";y-label "V";axesFont "Times New
Roman,12,0000000000,useDefault,normal";numpoints 100;plotstyle
"patch";axesstyle "normal";axestips FALSE;xis \TEXUX{x};var1name
\TEXUX{$x$};function
\TEXUX{$\MATRIX{2,5}{c}\VR{,,c,,,}{,,c,,,}{,,,,,}\HR{,,,,,}\CELL{0.01}%
\CELL{0.19}\CELL{0.02}\CELL{0.397\,1}\CELL{0.03}\CELL{0.621\,3}\CELL{0.04}%
\CELL{0.828\,4}\CELL{0.05}\CELL{1.\,\allowbreak 078\,3}$};linecolor
"blue";linestyle 1;pointstyle "point";linethickness 3;lineAttributes
"Solid";curveColor "[flat::RGB:0x000000ff]";curveStyle "Line";function
\TEXUX{$\MATRIX{2,5}{c}\VR{,,c,,,}{,,c,,,}{,,,,,}\HR{,,,,,}\CELL{0.01}%
\CELL{0.19}\CELL{0.02}\CELL{0.38}\CELL{0.03}\CELL{0.57}\CELL{0.04}%
\CELL{0.76}\CELL{0.05}\CELL{0.95}$};linecolor "black";linestyle 2;pointstyle
"point";linethickness 1;lineAttributes "Dash";curveColor
"[flat::RGB:0000000000]";curveStyle "Line";VCamFile
'LUKLKF04.xvz';valid_file "T";tempfilename
'LUKLKF0U.wmf';tempfile-properties "XPR";}}The dashed line is what we expect
from Ohm's law. The solid line is what data from a light bulb would actually
look like. We could use our temperature dependent resistance, and realize
that the temperature is a function of time, to obtain 
\begin{equation*}
I=\frac{\mathcal{E}}{R_{o}\left( 1+\alpha \left( T\left( t\right)
-T_{o}\right) \right) }
\end{equation*}%
Since this set of measurements is not strictly following Ohm's law, we will
say that the light bulb is \emph{nonohmic.}

Many common circuit elements are vary nonohmic. A diode, for example, has a $%
\Delta V$ vs. $I$ relationship that looks like this.\FRAME{dtbpFX}{2.4344in}{%
1.6224in}{0pt}{}{}{Plot}{\special{language "Scientific Word";type
"MAPLEPLOT";width 2.4344in;height 1.6224in;depth 0pt;display
"USEDEF";plot_snapshots TRUE;mustRecompute FALSE;lastEngine "MuPAD";xmin
"-0.020018";xmax "5";xviewmin "-0.2";xviewmax "5.00050200049634";yviewmin
"0.052490013610601";yviewmax "0.1429801420124";rangeset"X";plottype
4;labeloverrides 3;x-label "I (mA)";y-label "Delta V (V)";axesFont "Times
New Roman,12,0000000000,useDefault,normal";numpoints 100;plotstyle
"patch";axesstyle "normal";axestips FALSE;gridLines TRUE;xis
\TEXUX{x};var1name \TEXUX{$x$};function \TEXUX{$\allowbreak \frac{\left(
1.38\times 10^{-23}\right) \left( 300\right) }{\left( 1.6\times
10^{-19}\right) }\ln \left( \frac{x}{0.02}+1\right) $};linecolor
"blue";linestyle 1;pointstyle "point";linethickness 3;lineAttributes
"Solid";var1range "-0.020018,5";num-x-gridlines 100;curveColor
"[flat::RGB:0x000000ff]";curveStyle "Line";VCamFile
'M0MKAS08.xvz';valid_file "T";tempfilename
'MWBGGW00.wmf';tempfile-properties "XPR";}}

We can now understand how an electric current is formed. Hopefully you have
taken, are taking, or will take ME210 so you will know how to build simple
circuits with resistances and capacitances. But for this class, we will now
investigate a new force, the magnetic force.

\section{Power in resisters}

We learned that the resistance in a resister depends on the temperature of
the resister, and even have an approximate relationship that shows how this
works%
\begin{equation*}
R=R_{o}(1+\alpha \left( T-T_{o})\right)
\end{equation*}%
so we know that temperature and resistance are related. But most of us have
used a toaster, or an electric stove, or an electric space heater, etc. How
does an electric circuit produce heat? or even light from a light bulb?

To answer this let's think of the energy expended as an electron travels a
circuit. The potential energy expended is 
\begin{equation*}
\Delta U=q\Delta V
\end{equation*}%
where the $\Delta V$ comes from the battery, so we could write this as 
\begin{equation*}
\Delta U=q\mathcal{E}
\end{equation*}%
This is the energy lost as the electron travels from one side of the battery
to the other. We could describe how fast the energy is lost by dividing by
the time it takes the electron to make the trip%
\begin{equation*}
\frac{\Delta U}{\Delta T}=\frac{q}{\Delta t}\mathcal{E}
\end{equation*}%
but of course we want to do this for more than one electron. Let's take a
small amount of charge, $\Delta Q,$ then 
\begin{equation*}
\frac{\Delta U}{\Delta T}=\frac{\Delta Q}{\Delta t}\mathcal{E}
\end{equation*}%
\begin{equation*}
\frac{\Delta U}{\Delta T}=\frac{\Delta Q}{\Delta t}\mathcal{E}
\end{equation*}%
and if we make the small group of charge very small we have 
\begin{equation*}
\frac{dU}{dT}=\frac{dQ}{dt}\mathcal{E}
\end{equation*}%
and we recognize $dU/dt$ as the power and $dQ/dt$ as current, then 
\begin{equation*}
\mathcal{P}=I\mathcal{E}
\end{equation*}

This is the power supplied by the battery in moving the group of electrons
through the circuit. But from conservation of energy, the charge packet must
lose all the energy that the battery provides, so 
\begin{equation*}
\mathcal{P}_{battery}=\mathcal{P}_{R}=I\Delta V_{R}
\end{equation*}%
is the energy that leaves the circuit as the packet of charge moves.

This works for any resistance 
\begin{equation*}
\mathcal{P}_{R}=I\Delta V_{R}
\end{equation*}%
Then we can use Ohm's law%
\begin{equation*}
\Delta V_{R}=IR
\end{equation*}%
to find 
\begin{eqnarray*}
\mathcal{P}_{R} &=&I\left( IR\right) \\
&=&I^{2}R
\end{eqnarray*}

But where does this energy go? This is the energy that makes the heat in the
space heater, or the light in the light bulb.

\section{Magnetism}

Most people have used a magnet. at some time. They come as ads that stick to
a refrigerator. They are the working part of a compass. They hold the pieces
of travel games to their boards, etc. So I think we all know that magnets
stick to metal things. But do they stick to all metal things?

The answer is no, only a few metals work. Iron and Nickel and Cobalt are
some that do. Aluminum and Copper do not. By the time we are done studying
magnetism, we should be able to explain this.

Magnets are very like charged objects in some ways. They can attract or
repel each other They attract \textquotedblleft
unmagnetized\textquotedblright\ materials. But there are some important
differences. 
%TCIMACRO{%
%\TeXButton{Bar Magnet Demo}{\marginpar {
%\hspace{-0.5in}
%\begin{minipage}[t]{1in}
%\small{Bar Magnet Demo -- Make this like the first charge demo}
%\end{minipage}
%}}}%
%BeginExpansion
\marginpar {
\hspace{-0.5in}
\begin{minipage}[t]{1in}
\small{Bar Magnet Demo -- Make this like the first charge demo}
\end{minipage}
}%
%EndExpansion
%TCIMACRO{%
%\TeXButton{Bar Magnet Demo Alternate}{\marginpar {
%\hspace{-0.5in}
%\begin{minipage}[t]{1in}
%\small{Bar Magnet Demo -- Alternate, use the array of iron arrows and an overhead projector with the bar magnet}
%\end{minipage}
%}}}%
%BeginExpansion
\marginpar {
\hspace{-0.5in}
\begin{minipage}[t]{1in}
\small{Bar Magnet Demo -- Alternate, use the array of iron arrows and an overhead projector with the bar magnet}
\end{minipage}
}%
%EndExpansion

Notice that a \textquotedblleft magnetic charge\textquotedblright\ seems to
be induced in some metal objects, but not in other common objects. This is
very different than electric charge and electric polarization! And we should
state explicitly that for magnets, there seem to be both \textquotedblleft
charges\textquotedblright\ in the same object! We call the \textquotedblleft
charge centers\textquotedblright\ the \emph{poles} of the magnet. We find
that one pole attracts one of the poles of a second magnet and repels the
other. If we turn around the first magnet, we find that our pattern of
attraction and repulsion reverses.%
%TCIMACRO{%
%\TeXButton{More Bar Magnet Demo -- Like Poles}{\marginpar {
%\hspace{-0.5in}
%\begin{minipage}[t]{1in}
%\small{More Bar Magnet Demo -- Like Poles}
%\end{minipage}
%}} }%
%BeginExpansion
\marginpar {
\hspace{-0.5in}
\begin{minipage}[t]{1in}
\small{More Bar Magnet Demo -- Like Poles}
\end{minipage}
}
%EndExpansion
Because magnets were used for centuries in navigational compasses, we call
one pole the \emph{north pole} of the compass and the other the \emph{south
pole} of the magnet. The north pole is the pole that would orient toward the
north. Why does this happen?

I hope your high school science class taught you that the Earth has a
magnetic field. \FRAME{dhF}{4.0119in}{1.8213in}{0pt}{}{}{Figure}{\special%
{language "Scientific Word";type "GRAPHIC";maintain-aspect-ratio
TRUE;display "USEDEF";valid_file "T";width 4.0119in;height 1.8213in;depth
0pt;original-width 6.2699in;original-height 2.8305in;cropleft "0";croptop
"1";cropright "1";cropbottom "0";tempfilename
'LUNPGX02.wmf';tempfile-properties "XPR";}}So we constantly live under the
influence of a large magnet! Now lets hang both of our magnets from a
string, and see which way they like to hang. The north facing end we will
label $N$ and the south facing end we will label $S.$ Now we can see that
the two $N$ ends repel each other and the two $S$ ends repel each other. But
a $N$ end and a $S$ end will attract.

\FRAME{dhF}{3.1107in}{1.5532in}{0pt}{}{}{Figure}{\special{language
"Scientific Word";type "GRAPHIC";maintain-aspect-ratio TRUE;display
"USEDEF";valid_file "T";width 3.1107in;height 1.5532in;depth
0pt;original-width 6.5337in;original-height 3.2474in;cropleft "0";croptop
"1";cropright "1";cropbottom "0";tempfilename
'LUKBPT02.wmf';tempfile-properties "XPR";}}

\FRAME{dhF}{3.1107in}{1.5532in}{0pt}{}{}{Figure}{\special{language
"Scientific Word";type "GRAPHIC";maintain-aspect-ratio TRUE;display
"USEDEF";valid_file "T";width 3.1107in;height 1.5532in;depth
0pt;original-width 6.5337in;original-height 3.2474in;cropleft "0";croptop
"1";cropright "1";cropbottom "0";tempfilename
'NBCUUO18.wmf';tempfile-properties "XPR";}}Once again we have a situation
where we can define a mover object and an environmental object. We can
picture one of the magnets making a magnetic field and the other magnet
moving through this field. Of course both magnets make magnetic fields, but
since a magnet can't make a magnetic field that moves itself, we won't draw
this self-field for the mover magnet. We just draw the field for the
environmental magnet. We did this in our Earth-compass picture. The Earth
was the environmental magnet and the compass was the mover magnet.

One quirk of history is that since a $N$ end of a magnet is attracted to the
North part of the Earth. But north end of magnets are attracted to south
poles of magnets, the Earth's geographic north pole must be a magnetic south
pole!

One common misconception is that there is one specific place that is the
magnetic north pole. Really it is a region near Newfoundland where the field
strength actually varies quite a bit. You may have heard people discuss how
the poles switch every so often. This is true, and we don't fully understand
the mechanism for this.

There is a large difference between the magnetic force and the electric
force. Electric charges are easy to separate. But magnetic poles are not at
all easy to separate. If we break a magnet\FRAME{dhF}{1.7495in}{1.1217in}{0pt%
}{}{}{Figure}{\special{language "Scientific Word";type
"GRAPHIC";maintain-aspect-ratio TRUE;display "USEDEF";valid_file "T";width
1.7495in;height 1.1217in;depth 0pt;original-width 4.171in;original-height
2.6654in;cropleft "0";croptop "1";cropright "1";cropbottom "0";tempfilename
'LUKCCP03.wmf';tempfile-properties "XPR";}}we end up with each piece being a
magnet complete with both north and south ends. This is very mysterious!
something about the source of the magnetic field must be very different than
for the source of the electric field. We will investigate the source of a
magnetic field as we go.

The Earth's magnetic fields affects many biological systems. One of these is
a bacteria that contain small permanent magnets inside of them to help them
find the mud they live in.

In the 1990's there was a health fad involving magnets. Many people bought
magnets to strap on their bodies. They were supposed to reduce aging and
give energy. Mostly they stimulated the economy. But we will find that
magnetic fields can alter the flow of blood (but these magnets did not do
so, the FDA would not allow strong enough magnets to be sold as apparel to
have this effect). Another common place to find magnetic fields is the MRI
devices used in hospitals to make images of the interior of bodies.

%TCIMACRO{%
%\TeXButton{Question 223.37.6}{\marginpar {
%\hspace{-0.5in}
%\begin{minipage}[t]{1in}
%\small{Question 223.37.6}
%\end{minipage}
%}}}%
%BeginExpansion
\marginpar {
\hspace{-0.5in}
\begin{minipage}[t]{1in}
\small{Question 223.37.6}
\end{minipage}
}%
%EndExpansion

%TCIMACRO{%
%\TeXButton{Basic Equations}{\hspace{-1.3in}{\LARGE Basic Equations\vspace{0.25in}}}}%
%BeginExpansion
\hspace{-1.3in}{\LARGE Basic Equations\vspace{0.25in}}%
%EndExpansion

\chapter{Magnetic Field}

%TCIMACRO{%
%\TeXButton{Fundamental Concepts}{\hspace{-1.3in}{\LARGE Fundamental Concepts\vspace{0.25in}}}}%
%BeginExpansion
\hspace{-1.3in}{\LARGE Fundamental Concepts\vspace{0.25in}}%
%EndExpansion

%TCIMACRO{%
%\TeXButton{Pass out magnets on sticks}{\marginpar {
%\hspace{-0.5in}
%\begin{minipage}[t]{1in}
%\small{Pass out magnets on sticks}
%\end{minipage}
%}}}%
%BeginExpansion
\marginpar {
\hspace{-0.5in}
\begin{minipage}[t]{1in}
\small{Pass out magnets on sticks}
\end{minipage}
}%
%EndExpansion
%TCIMACRO{%
%\TeXButton{Pass out magnets}{\marginpar {
%\hspace{-0.5in}
%\begin{minipage}[t]{1in}
%\small{Pass out magnets}
%\end{minipage}
%}}}%
%BeginExpansion
\marginpar {
\hspace{-0.5in}
\begin{minipage}[t]{1in}
\small{Pass out magnets}
\end{minipage}
}%
%EndExpansion
We have now experience with two non-contact forces, the gravitational force
and the electric or Coulomb force. In both cases, we have found that there
is a field involved with the production of this force. We can guess that
this is true for the magnetic force as well.

The discovery of this field involved an accidental experiment, and
understanding this experiment gives us great insight into the nature of this
field and where it comes from. So we will spend a little time describing it.

\section{Fundamental Concepts in the Lecture}

\begin{itemize}
\item A long wire that carries a current produces a magnetic field

\item The magnetic field due to a long wither with current becomes weaker
with distance and forms concentric cylinders of constant magnetic field
strength

\item The direction of the long-wire-with-current field is given by a
right-hand-rule.

\item The field due to a moving charge is given by the \emph{Biot-Savart law}%
\begin{equation*}
B=\frac{\mu _{o}}{4\pi }\frac{qv\sin \theta }{r^{2}}
\end{equation*}
\end{itemize}

\section{Discovery of Magnetic Field}

%TCIMACRO{%
%\TeXButton{Question 223.38.1}{\marginpar {
%\hspace{-0.5in}
%\begin{minipage}[t]{1in}
%\small{Question 223.38.1}
%\end{minipage}
%}}}%
%BeginExpansion
\marginpar {
\hspace{-0.5in}
\begin{minipage}[t]{1in}
\small{Question 223.38.1}
\end{minipage}
}%
%EndExpansion
In 1819 a Dutch scientist named Oersted was lecturing on electricity. He was
actually making the point that there was no connection between electricity
and magnetism. He had a large battery connected to a wire. A large current
flowed through the wire. By chance, Oersted placed a compass near the wire.
He had done this before, but this time the wire was in a different
orientation than in previous demonstrations. To his great surprise, the
compass needle changed direction when it was placed near the wire!

A similar experiment, but this time with several compasses, is shown in the
next figure.

\FRAME{dhF}{2.949in}{1.9666in}{0pt}{}{}{Figure}{\special{language
"Scientific Word";type "GRAPHIC";maintain-aspect-ratio TRUE;display
"USEDEF";valid_file "T";width 2.949in;height 1.9666in;depth
0pt;original-width 2.9058in;original-height 1.9285in;cropleft "0";croptop
"1";cropright "1";cropbottom "0";tempfilename
'LUKGOS05.wmf';tempfile-properties "XPR";}}When the current is turned on,
the compasses change direction.\FRAME{dhF}{3.0026in}{2.4933in}{0pt}{}{}{%
Figure}{\special{language "Scientific Word";type
"GRAPHIC";maintain-aspect-ratio TRUE;display "USEDEF";valid_file "T";width
3.0026in;height 2.4933in;depth 0pt;original-width 4.9216in;original-height
4.0811in;cropleft "0";croptop "1";cropright "1";cropbottom "0";tempfilename
'LUKGPJ06.wmf';tempfile-properties "XPR";}}This is a very good clue that
there really \emph{is} a connection between electricity and magnetism.

%TCIMACRO{%
%\TeXButton{Oersted's Experiment Demo}{\marginpar {
%\hspace{-0.5in}
%\begin{minipage}[t]{1in}
%\small{Oersted's Experiment Demo: Use the 106 boards and compasses}
%\end{minipage}
%}}}%
%BeginExpansion
\marginpar {
\hspace{-0.5in}
\begin{minipage}[t]{1in}
\small{Oersted's Experiment Demo: Use the 106 boards and compasses}
\end{minipage}
}%
%EndExpansion
We know that a compass orients itself in the Earth's magnetic field. We can
infer that the compass needle will orient in any magnetic field. In the next
figure you can see that there is a force on each end of the needle due to
the magnetic field.

\FRAME{dhF}{2.1344in}{2.0444in}{0pt}{}{}{Figure}{\special{language
"Scientific Word";type "GRAPHIC";maintain-aspect-ratio TRUE;display
"USEDEF";valid_file "T";width 2.1344in;height 2.0444in;depth
0pt;original-width 3.7118in;original-height 3.5526in;cropleft "0";croptop
"1";cropright "1";cropbottom "0";tempfilename
'LUKIBR09.wmf';tempfile-properties "XPR";}}Notice that we have marked the
environmental magnetic field with the letter $B.$ This is traditional.
Magnetic fields are often called $B$-fields for this reason. But more
importantly, this looks very like an electric dipole in a constant electric
field. We know enough about the dipole situation to predict that there will
be a torque, and that there will be a stable equilibrium when the compass
needle is aligned with the magnetic field.

Since our compasses oriented themselves near the current carrying wire,
there must be a magnetic field caused by the current in the wire. The field
shown in the last figure is uniform, but the field of our wire cannot be
uniform. The compasses pointed different directions. A common way to
describe this field is with a right-hand-rule. We imagine grabbing the wire
with our right hand with our thumb pointing in the current direction. The
field direction is given by our fingers.\FRAME{dhF}{1.8083in}{1.8862in}{0pt}{%
}{}{Figure}{\special{language "Scientific Word";type
"GRAPHIC";maintain-aspect-ratio TRUE;display "USEDEF";valid_file "T";width
1.8083in;height 1.8862in;depth 0pt;original-width 3.6841in;original-height
3.8441in;cropleft "0";croptop "1";cropright "1";cropbottom "0";tempfilename
'LUKGSH08.wmf';tempfile-properties "XPR";}}

%TCIMACRO{%
%\TeXButton{Question 223.38.2}{\marginpar {
%\hspace{-0.5in}
%\begin{minipage}[t]{1in}
%\small{Question 223.38.2}
%\end{minipage}
%}}}%
%BeginExpansion
\marginpar {
\hspace{-0.5in}
\begin{minipage}[t]{1in}
\small{Question 223.38.2}
\end{minipage}
}%
%EndExpansion
Although this is true, it takes some interpretation Let's take some time to
see what it means. Let's redraw the figure. \FRAME{dhF}{1.9986in}{2.4561in}{%
0pt}{}{}{Figure}{\special{language "Scientific Word";type
"GRAPHIC";maintain-aspect-ratio TRUE;display "USEDEF";valid_file "T";width
1.9986in;height 2.4561in;depth 0pt;original-width 1.9597in;original-height
2.4146in;cropleft "0";croptop "1";cropright "1";cropbottom "0";tempfilename
'M0KRTJ02.wmf';tempfile-properties "XPR";}}Now that we have a new figure,
let's reconsider what our right hand rule means. What we mean is that the
magnetic field is constant in magnitude around a circle, and that the
direction of the field is tangent to the circle, with the arrow pointing in
the direction your fingers go with the right-hand-rule. \FRAME{dhF}{2.5019in%
}{2.4699in}{0pt}{}{}{Figure}{\special{language "Scientific Word";type
"GRAPHIC";maintain-aspect-ratio TRUE;display "USEDEF";valid_file "T";width
2.5019in;height 2.4699in;depth 0pt;original-width 2.4613in;original-height
2.4284in;cropleft "0";croptop "1";cropright "1";cropbottom "0";tempfilename
'M60YVR00.wmf';tempfile-properties "XPR";}}This is easier to see in a
top-down view.\FRAME{dhF}{2.0029in}{1.772in}{0pt}{}{}{Figure}{\special%
{language "Scientific Word";type "GRAPHIC";maintain-aspect-ratio
TRUE;display "PICT";valid_file "T";width 2.0029in;height 1.772in;depth
0pt;original-width 5.2278in;original-height 4.6224in;cropleft "0";croptop
"1";cropright "1";cropbottom "0";tempfilename
'LUKISR0F.wmf';tempfile-properties "XPR";}}But in the first figure we only
drew the field around one circle. By using symmetry, we can guess that the
field magnitude must be constant around any circle. It must depend only on $%
r,$if the current is constant. So we could draw constant field lines at any
distance, $r,$ away from the wire. \FRAME{dhF}{2.1802in}{2.4699in}{0pt}{}{}{%
Figure}{\special{language "Scientific Word";type
"GRAPHIC";maintain-aspect-ratio TRUE;display "USEDEF";valid_file "T";width
2.1802in;height 2.4699in;depth 0pt;original-width 2.1404in;original-height
2.4284in;cropleft "0";croptop "1";cropright "1";cropbottom "0";tempfilename
'M0KRVL04.wmf';tempfile-properties "XPR";}}%
%TCIMACRO{%
%\TeXButton{Question 223.38.3}{\marginpar {
%\hspace{-0.5in}
%\begin{minipage}[t]{1in}
%\small{Question 223.38.3}
%\end{minipage}
%}}}%
%BeginExpansion
\marginpar {
\hspace{-0.5in}
\begin{minipage}[t]{1in}
\small{Question 223.38.3}
\end{minipage}
}%
%EndExpansion
But again, this figure is not so good, because the entire wire makes a field
that has a constant value for $B$ at a distance $r$ away. So we could also
draw the field above our hand.\FRAME{dhF}{1.855in}{2.4768in}{0pt}{}{}{Figure%
}{\special{language "Scientific Word";type "GRAPHIC";maintain-aspect-ratio
TRUE;display "USEDEF";valid_file "T";width 1.855in;height 2.4768in;depth
0pt;original-width 2.3635in;original-height 3.1644in;cropleft "0";croptop
"1";cropright "1";cropbottom "0";tempfilename
'M0KRW805.wmf';tempfile-properties "XPR";}}Maybe a better way to draw this
field would be a set of concentric cylinders. Along the surface of the
cylinder (but not the end caps) the field will be constant.\FRAME{dhF}{%
1.9709in}{1.6466in}{0pt}{}{}{Figure}{\special{language "Scientific
Word";type "GRAPHIC";maintain-aspect-ratio TRUE;display "USEDEF";valid_file
"T";width 1.9709in;height 1.6466in;depth 0pt;original-width
1.932in;original-height 1.6103in;cropleft "0";croptop "1";cropright
"1";cropbottom "0";tempfilename 'M0KRWR06.wmf';tempfile-properties "XPR";}}%
Of course, if our wire is infinitely long, the cylinders will be infinitely
long too...\FRAME{dhF}{1.8723in}{2.4837in}{0pt}{}{}{Figure}{\special%
{language "Scientific Word";type "GRAPHIC";maintain-aspect-ratio
TRUE;display "USEDEF";valid_file "T";width 1.8723in;height 2.4837in;depth
0pt;original-width 1.8351in;original-height 2.4431in;cropleft "0";croptop
"1";cropright "1";cropbottom "0";tempfilename
'M0KRXB07.wmf';tempfile-properties "XPR";}}%
%TCIMACRO{%
%\TeXButton{Question 223.38.4}{\marginpar {
%\hspace{-0.5in}
%\begin{minipage}[t]{1in}
%\small{Question 223.38.4}
%\end{minipage}
%}}}%
%BeginExpansion
\marginpar {
\hspace{-0.5in}
\begin{minipage}[t]{1in}
\small{Question 223.38.4}
\end{minipage}
}%
%EndExpansion
And the field does not stop after a few cylinders, it reaches $B=0$ only
when $r=\infty .$ So the field fills all of space.\FRAME{dhF}{2.5019in}{%
1.8836in}{0pt}{}{}{Figure}{\special{language "Scientific Word";type
"GRAPHIC";maintain-aspect-ratio TRUE;display "USEDEF";valid_file "T";width
2.5019in;height 1.8836in;depth 0pt;original-width 2.4613in;original-height
1.8455in;cropleft "0";croptop "1";cropright "1";cropbottom "0";tempfilename
'M0KRXY08.wmf';tempfile-properties "XPR";}}This is a more accurate way to
draw the magnetic field due to a long straight wire, but it takes a long
time to draw such a diagram, so usually we will just draw one circle, and
you will have to mentally fill in the other circles and the concentric
cylinders that they represent.

To use the right hand rule, remember to place your thumb in the current
direction. Then the field direction is given tangent to the circle and
pointing in our finger direction.

\subsection{Making the field--moving charges}

But how does a current in a wire make a magnetic field?

The secret is to look at the individual charges that are moving. When early
scientists caused individual charges to move, they found they created
magnetic fields. The experimental results gave a relationship for the
strength of this field%
\begin{equation*}
B=\frac{\mu _{o}}{4\pi }\frac{qv\sin \theta }{r^{2}}
\end{equation*}%
and the direction is given by the right hand rule by pointing the thumb in
the direction the charges are going and using the figures to indicate the
field direction as we have described above. In a sense, this is a very small
current (one moving charge!). So the field should look very similar.

This relationship was found by two scientists, Biot and Savart, and it
carries their name, the \emph{Biot-Savart law. }The factor $\mu _{o}$ is a
constant very like $\epsilon _{o}.$ It has a value 
\begin{equation*}
\mu _{o}=4\pi \times 10^{-7}\frac{\unit{T}\unit{m}}{\unit{A}}
\end{equation*}%
and is called the \emph{permeability of free space, }The unit $\unit{T}$ is
called a \emph{tesla} and is 
\begin{equation*}
\unit{T}=\frac{\unit{N}}{\unit{A}\unit{m}}
\end{equation*}

The charges already had an electric field before they were accelerated, but
now they have two fields, an electric and a magnetic field. We used unit
vectors to write our $E$-field.%
\begin{equation*}
\overrightarrow{\mathbf{E}}=\frac{1}{4\pi \epsilon _{o}}\frac{q}{r^{2}}%
\mathbf{\hat{r}}
\end{equation*}%
It is convenient to do the same for the magnetic case. We can remember that
a vector cross product is given by 
\begin{equation*}
\overrightarrow{\mathbf{a}}\times \overrightarrow{\mathbf{b}}=ab\sin \theta
\qquad \perp \overrightarrow{\mathbf{a}},\perp \overrightarrow{\mathbf{b}}
\end{equation*}%
where the resulting vector is perpendicular to both $\overrightarrow{\mathbf{%
a}}$ and $\overrightarrow{\mathbf{b}}.$ Thinking about this for a while
allows us to realize this is just what we want for the magnetic field. If
the velocity of the charges is up (say, in the $\hat{z}$ direction) then we
can use our right hand rule to realize we need a vector perpendicular to
both $\hat{z}$ and $\hat{r}.$ This is given by 
\begin{equation*}
\mathbf{\hat{z}}\times \mathbf{\hat{r}}
\end{equation*}%
which is always tangent to the circle indicated by our fingers. Since $v$ is
in the $z$ direction we can use%
\begin{equation*}
\overrightarrow{\mathbf{v}}\times \mathbf{\hat{r}}=v\sin \theta \qquad \perp 
\overrightarrow{\mathbf{v}},\perp \mathbf{\hat{r}}
\end{equation*}%
to write the Biot-Savart law as 
\begin{equation*}
\overrightarrow{\mathbf{B}}=\frac{\mu _{o}}{4\pi }\frac{q\overrightarrow{%
\mathbf{v}}\times \mathbf{\hat{r}}}{r^{2}}
\end{equation*}

We should do a problem to see how this works.

Suppose we accelerate a proton and send it in the $z$-direction to a speed
of $1.0\times 10^{7}\unit{m}/\unit{s}.$ Let's further suppose we have a
magnetic field detector placed $1\unit{mm}$ from the path of the proton.
What field would it measure?

\FRAME{dhF}{4.6709in}{0.972in}{0pt}{}{}{Figure}{\special{language
"Scientific Word";type "GRAPHIC";maintain-aspect-ratio TRUE;display
"USEDEF";valid_file "T";width 4.6709in;height 0.972in;depth
0pt;original-width 7.1044in;original-height 1.4572in;cropleft "0";croptop
"1";cropright "1";cropbottom "0";tempfilename
'LUKKGQ0P.wmf';tempfile-properties "XPR";}}

We know 
\begin{equation*}
\overrightarrow{\mathbf{B}}=\frac{\mu _{o}}{4\pi }\frac{q\overrightarrow{%
\mathbf{v}}\times \mathbf{\hat{r}}}{r^{2}}
\end{equation*}%
and by symmetry we know that $v$ is perpendicular to $\hat{r}$ just as the
proton passes the detector. So, using the right hand rule for cross
products, we put our hand in the $v$-direction and bend our fingers into the 
$r$-direction. Then our thumb shows the resulting direction. In this case it
is in the positive $y$-direction, or out of the page. The magnitude would be 
\begin{eqnarray*}
\overrightarrow{\mathbf{B}} &=&\frac{4\pi \times 10^{-7}\frac{\unit{T}\unit{m%
}}{\unit{A}}}{4\pi }\frac{\left( 1.6\times 10^{-19}\unit{C}\right) \left(
1.0\times 10^{7}\unit{m}/\unit{s}\right) }{\left( 0.001\unit{m}\right) ^{2}}%
\mathbf{\hat{y}} \\
&=&1.\,\allowbreak 6\times 10^{-13}\unit{T}\mathbf{\hat{y}}
\end{eqnarray*}

\chapter{Current loops}

%TCIMACRO{%
%\TeXButton{Fundamental Concepts}{\hspace{-1.3in}{\LARGE Fundamental Concepts\vspace{0.25in}}}}%
%BeginExpansion
\hspace{-1.3in}{\LARGE Fundamental Concepts\vspace{0.25in}}%
%EndExpansion

\begin{itemize}
\item The magnetic field due to a current in a wire is given by the integral
form of the Biot-Savart law $\overrightarrow{\mathbf{B}}=\frac{\mu _{o}I}{%
4\pi }\int \frac{d\overrightarrow{\mathbf{s}}\times \mathbf{\hat{r}}}{r^{2}}$

\item The magnetic field magnitude of a long straight wire with a current is
given by $B=\frac{\mu _{o}I}{2\pi a}$ with the direction given by the right
hand rule we learned last time.

\item The field due to a magnetic dipole is $\overrightarrow{\mathbf{B}}%
\approx \frac{\mu _{o}}{4\pi }\frac{2\overrightarrow{\mu }}{r^{3}}\mathbf{%
\hat{\imath}}$ where $\overrightarrow{\mu }$ is the magnetic dipole moment $%
\mu =IA$ with the direction from south to north pole.
\end{itemize}

\section{Magnetic field of a current}

Last lecture, we learned the Biot-Savart law%
\begin{equation*}
\overrightarrow{\mathbf{B}}=\frac{\mu _{o}}{4\pi }\frac{q\overrightarrow{%
\mathbf{v}}\times \mathbf{\hat{r}}}{r^{2}}
\end{equation*}%
now let's consider our $q$ to be part of a current in a wire. A small amount
of current moves along the wire. Let's call this small amount of charge $%
\Delta Q.$ \FRAME{dhF}{3.2292in}{2.1344in}{0pt}{}{}{Figure}{\special%
{language "Scientific Word";type "GRAPHIC";maintain-aspect-ratio
TRUE;display "USEDEF";valid_file "T";width 3.2292in;height 2.1344in;depth
0pt;original-width 3.1842in;original-height 2.0954in;cropleft "0";croptop
"1";cropright "1";cropbottom "0";tempfilename
'LUORTT06.wmf';tempfile-properties "XPR";}}This small amount of charge will
make a magnetic field, but it will be only a small part of the total field,
because $\Delta Q$ is only a small part of the total amount of charge
flowing in the wire. That part of the field made by $\Delta Q$ is 
\begin{equation*}
\Delta \overrightarrow{\mathbf{B}}=\frac{\mu _{o}}{4\pi }\frac{\Delta Q%
\overrightarrow{\mathbf{v}}\times \mathbf{\hat{r}}}{r^{2}}
\end{equation*}%
Let's look at $\Delta Q\overrightarrow{\mathbf{v}}.$ We can rewrite this as 
\begin{eqnarray*}
\Delta Q\overrightarrow{\mathbf{v}} &=&\Delta Q\frac{\Delta \overrightarrow{%
\mathbf{s}}}{\Delta t} \\
&=&\frac{\Delta Q}{\Delta t}\Delta \overrightarrow{\mathbf{s}} \\
&=&I\Delta \overrightarrow{\mathbf{s}}
\end{eqnarray*}%
then our small amount of field is given by 
\begin{equation*}
\Delta \overrightarrow{\mathbf{B}}=\frac{\mu _{o}}{4\pi }\frac{I\Delta 
\overrightarrow{\mathbf{s}}\times \mathbf{\hat{r}}}{r^{2}}
\end{equation*}%
as usual, where there is a $\Delta ,$ we can predict that we can take a
limit and end up with a $d$%
\begin{equation*}
d\overrightarrow{\mathbf{B}}=\frac{\mu _{o}}{4\pi }\frac{Id\overrightarrow{%
\mathbf{s}}\times \mathbf{\hat{r}}}{r^{2}}
\end{equation*}%
Some things to note about this result

\begin{enumerate}
\item 
%TCIMACRO{%
%\TeXButton{Question 223.39.1}{\marginpar {
%\hspace{-0.5in}
%\begin{minipage}[t]{1in}
%\small{Question 223.39.1}
%\end{minipage}
%}}}%
%BeginExpansion
\marginpar {
\hspace{-0.5in}
\begin{minipage}[t]{1in}
\small{Question 223.39.1}
\end{minipage}
}%
%EndExpansion
The vector $d\overrightarrow{\mathbf{B}}$ is perpendicular to $d%
\overrightarrow{\mathbf{s}}$ and to the unit vector $\mathbf{\hat{r}}$
directed from $d\overrightarrow{\mathbf{s}}$ to some point $P.$

\item 
%TCIMACRO{%
%\TeXButton{Question 223.39.2}{\marginpar {
%\hspace{-0.5in}
%\begin{minipage}[t]{1in}
%\small{Question 223.39.2}
%\end{minipage}
%}}}%
%BeginExpansion
\marginpar {
\hspace{-0.5in}
\begin{minipage}[t]{1in}
\small{Question 223.39.2}
\end{minipage}
}%
%EndExpansion
The magnitude of $d\overrightarrow{\mathbf{B}}$ is inversely proportional to 
$r^{2}$

\item 
%TCIMACRO{%
%\TeXButton{Question 223.39.3}{\marginpar {
%\hspace{-0.5in}
%\begin{minipage}[t]{1in}
%\small{Question 223.39.3}
%\end{minipage}
%}}}%
%BeginExpansion
\marginpar {
\hspace{-0.5in}
\begin{minipage}[t]{1in}
\small{Question 223.39.3}
\end{minipage}
}%
%EndExpansion
The magnitude of $d\overrightarrow{\mathbf{B}}$ is proportional to the
current

\item 
%TCIMACRO{%
%\TeXButton{Question 223.39.4}{\marginpar {
%\hspace{-0.5in}
%\begin{minipage}[t]{1in}
%\small{Question 223.39.4}
%\end{minipage}
%}}}%
%BeginExpansion
\marginpar {
\hspace{-0.5in}
\begin{minipage}[t]{1in}
\small{Question 223.39.4}
\end{minipage}
}%
%EndExpansion
The magnitude of $d\overrightarrow{\mathbf{B}}$ is proportional to the
length of $d\overrightarrow{\mathbf{s}}$

\item 
%TCIMACRO{%
%\TeXButton{Question 223.39.5}{\marginpar {
%\hspace{-0.5in}
%\begin{minipage}[t]{1in}
%\small{Question 223.39.5}
%\end{minipage}
%}}}%
%BeginExpansion
\marginpar {
\hspace{-0.5in}
\begin{minipage}[t]{1in}
\small{Question 223.39.5}
\end{minipage}
}%
%EndExpansion
The magnitude of $d\overrightarrow{\mathbf{B}}$ is proportional to $\sin
\theta $ where $\theta $ is the angle between $d\overrightarrow{\mathbf{s}}$
and $\mathbf{\hat{r}}$
\end{enumerate}

Where there is $d\overrightarrow{\mathbf{B}}$ we will surely integrate. The
field $d\overrightarrow{\mathbf{B}}$ is due to just a small part of the wire 
$d\overrightarrow{\mathbf{s}}.$ We would like the field due to all of the
wire. So we take 
\begin{equation*}
\overrightarrow{\mathbf{B}}=\frac{\mu _{o}I}{4\pi }\int \frac{d%
\overrightarrow{\mathbf{s}}\times \mathbf{\hat{r}}}{r^{2}}
\end{equation*}%
This is a case where the equation actually is as hard to deal with as it
looks. The integration over a cross product is tricky. Let's do an example.

\subsection{The field due to a square current loop}

Suppose we have a square current loop. Of course there would have to be a
battery or some potential source in the loop to make the current, but we
will just draw the loop with a current as shown. The current must be the
same in all parts of the loop.\FRAME{dhF}{1.8447in}{1.9389in}{0pt}{}{}{Figure%
}{\special{language "Scientific Word";type "GRAPHIC";maintain-aspect-ratio
TRUE;display "USEDEF";valid_file "T";width 1.8447in;height 1.9389in;depth
0pt;original-width 1.8075in;original-height 1.9009in;cropleft "0";croptop
"1";cropright "1";cropbottom "0";tempfilename
'M0VQYC01.wmf';tempfile-properties "XPR";}}

Let's find the field in the center of the loop at point $P$.

I will break up the integration into four parts, one for each side of the
loop. For each part, we will need to find $d\overrightarrow{\mathbf{s}}%
\times \mathbf{\hat{r}}$ and $r$ to find the field using%
\begin{equation*}
\overrightarrow{\mathbf{B}}=\frac{\mu _{o}I}{4\pi }\int \frac{d%
\overrightarrow{\mathbf{s}}\times \mathbf{\hat{r}}}{r^{2}}
\end{equation*}

This is very like what we did to find electric fields. For electric fields
we had to find $dq,$ $\mathbf{\hat{r}},$ and $r$ and we integrated using 
\begin{equation*}
\overrightarrow{\mathbf{E}}=\frac{1}{4\pi \epsilon _{o}}\int \frac{dq}{r^{2}}%
\mathbf{\hat{r}}
\end{equation*}

Now we need $d\overrightarrow{\mathbf{s}}\times \mathbf{\hat{r}}$ and $r.$
For electric fields, we needed to deal with the vector $\mathbf{\hat{r}.}$
Now we need to deal with a cross product, $d\overrightarrow{\mathbf{s}}%
\times \mathbf{\hat{r}}$, involving $\mathbf{\hat{r}.}$ For the bottom part
of our loop $d\overrightarrow{\mathbf{s}}\times \mathbf{\hat{r}}$ is just%
\begin{eqnarray*}
d\overrightarrow{\mathbf{s}}\times \mathbf{\hat{r}} &=&-ds\sin \theta 
\mathbf{\hat{k}} \\
&=&-dx\sin \theta \mathbf{\hat{k}}
\end{eqnarray*}%
where $+\mathbf{\hat{k}}$ in out of the page. We can see this in the figure%
\FRAME{dtbpF}{1.3716in}{1.6466in}{0pt}{}{}{Figure}{\special{language
"Scientific Word";type "GRAPHIC";maintain-aspect-ratio TRUE;display
"USEDEF";valid_file "T";width 1.3716in;height 1.6466in;depth
0pt;original-width 1.337in;original-height 1.6103in;cropleft "0";croptop
"1";cropright "1";cropbottom "0";tempfilename
'M68NTL0G.wmf';tempfile-properties "XPR";}}So our field from the bottom wire
is 
\begin{eqnarray*}
\overrightarrow{\mathbf{B}}_{b} &=&\frac{\mu _{o}I}{4\pi }\int \frac{d%
\overrightarrow{\mathbf{s}}\times \mathbf{\hat{r}}}{r^{2}} \\
&=&\frac{\mu _{o}I}{4\pi }\int \frac{-dx\sin \theta \mathbf{\hat{k}}}{r^{2}}
\end{eqnarray*}%
Next we need to find $r.$ We would like to not have more than one variable.
So it would be good to try to pick $x$ or $\theta $ and to put everything in
terms of that one variable. Let's try $\theta .$ From trigonometry we
realize 
\begin{equation*}
\sin \theta =\frac{a}{r}
\end{equation*}%
on the right side of the wire, and 
\begin{equation*}
\sin \left( \pi -\theta \right) =\sin \theta =\frac{a}{r}
\end{equation*}%
on the left side, So all along the bottom wire $r$ is given by 
\begin{equation*}
r=\frac{a}{\sin \theta }
\end{equation*}%
Then our field equation for the bottom wire becomes%
\begin{equation*}
\overrightarrow{\mathbf{B}}_{b}=\frac{\mu _{o}I}{4\pi }\int \frac{-\left(
dx\right) \sin \theta \mathbf{\hat{k}}}{\left( \frac{a}{\sin \theta }\right)
^{2}}
\end{equation*}%
but now we have an integration over $dx$ and our function is in terms of $%
\theta $ which depends on $x.$ We should try to fix this. Let's find $dx$ in
terms of $d\theta .$ We can pick $x=0$ to be the middle of the wire. Then 
\begin{equation*}
\tan \theta =\frac{a}{x}
\end{equation*}%
on the right and\FRAME{dtbpF}{1.7573in}{1.8957in}{0pt}{}{}{Figure}{\special%
{language "Scientific Word";type "GRAPHIC";maintain-aspect-ratio
TRUE;display "USEDEF";valid_file "T";width 1.7573in;height 1.8957in;depth
0pt;original-width 1.7201in;original-height 1.8585in;cropleft "0";croptop
"1";cropright "1";cropbottom "0";tempfilename
'M68OVH0H.wmf';tempfile-properties "XPR";}} 
\begin{equation*}
\tan \left( \pi -\theta \right) =-\tan \theta =\frac{a}{x}
\end{equation*}%
on the left. Since on the left $x$ is negative, this makes sense. So we have
either 
\begin{equation*}
x=\frac{a}{\tan \theta }
\end{equation*}%
or 
\begin{equation*}
x=-\frac{a}{\tan \theta }
\end{equation*}%
depending on which size of the dotted line we are on. We could write these
as 
\begin{equation*}
x=\pm \frac{a}{\tan \theta }=\pm \frac{a\cos \theta }{\sin \theta }
\end{equation*}%
for both cases. We really want $dx$ and moreover we want it as a magnitude
(we deal with the direction in the cross product). So we can take a
derivative and then take the magnitude (absolute value).%
\begin{equation*}
\frac{dx}{d\theta }=\frac{\sin \theta \left( -a\sin \theta \right) -a\cos
\theta \cos \theta }{\sin ^{2}\theta }=\frac{-a}{\sin ^{2}\theta }
\end{equation*}%
This derivative was not obvious! We had to use the quotient rule. But once
we have found it we can rewrite $dx$ as 
\begin{equation*}
dx=\left\vert \frac{-a}{\sin ^{2}\theta }d\theta \right\vert
\end{equation*}%
(now with the absolute value inserted) and since neither $a$ nor $\sin
^{2}\theta $ can be negative we can just write this as%
\begin{equation*}
dx=\frac{a}{\sin ^{2}\theta }d\theta
\end{equation*}%
When we put this in our integral equation for the bottom wire we have 
\begin{equation*}
\overrightarrow{\mathbf{B}}_{b}=\frac{\mu _{o}I}{4\pi }\int \frac{-\left( 
\frac{a}{\sin ^{2}\theta }\right) d\theta \sin \theta \mathbf{\hat{k}}}{%
\left( \frac{a}{\sin \theta }\right) ^{2}}
\end{equation*}%
which we should simplify before we try to integrate. 
\begin{eqnarray*}
\overrightarrow{\mathbf{B}}_{b} &=&\frac{\mu _{o}I}{4\pi }\int \frac{-\sin
\theta d\theta \mathbf{\hat{k}}}{a} \\
&=&-\frac{\mu _{o}I}{4\pi a}\mathbf{\hat{k}}\int \sin \theta d\theta
\end{eqnarray*}%
which is really not too bad considering the integral we had at the start of
this problem. When we get to the corner of the left hand side $\theta =\frac{%
3\pi }{4}$ and when we start on the right hand side $\theta =\frac{\pi }{4}$
and along the bottom wire $\theta $ will be somewhere in between $\frac{\pi 
}{4}$ and $\frac{3\pi }{4}.$ Then $\frac{\pi }{4}$ and $\frac{3\pi }{4}$ are
our limits of integration. We can perform this integral 
\begin{eqnarray*}
\overrightarrow{\mathbf{B}}_{b} &=&-\frac{\mu _{o}I}{4\pi a}\mathbf{\hat{k}}%
\int_{\frac{\pi }{4}}^{\frac{3\pi }{4}}\sin \theta d\theta \\
&=&-\frac{\mu _{o}I}{4\pi a}\mathbf{\hat{k}}\left[ -\cos \theta \mathstrut
\right\vert _{\frac{\pi }{4}}^{\frac{3\pi }{4}} \\
&=&-\frac{\mu _{o}I}{4\pi a}\mathbf{\hat{k}}\left( \frac{\sqrt{2}}{2}-\left(
-\frac{\sqrt{2}}{2}\right) \right) \\
&=&-\frac{\mu _{o}I\sqrt{2}}{4\pi a}\mathbf{\hat{k}}
\end{eqnarray*}

This was just for the bottom of the loop. Now let's look at the top of the
loop. There is finally some good news. The math will all be the same except
for the directions. We had better work out $d\overrightarrow{\mathbf{s}}%
\times \mathbf{\hat{r}}$ to see how different it is.

Now the $d\overrightarrow{\mathbf{s}}$ is to the right and $\mathbf{\hat{r}}$
is downward so%
\begin{equation*}
d\overrightarrow{\mathbf{s}}\times \mathbf{\hat{r}}=-dx\sin \theta \mathbf{%
\hat{k}}
\end{equation*}%
But this is just as before. So even this is the same! The integral across
the top wire will have exactly the same result as the integral across the
bottom wire. We can just multiply our previous result by two.

How about the sides? Again we get the same $d\overrightarrow{\mathbf{s}}%
\times \mathbf{\hat{r}}$ direction and all the rest is the same, so our
total field is 
\begin{equation*}
\overrightarrow{\mathbf{B}}=4\overrightarrow{\mathbf{B}}_{b}=-\frac{\mu _{o}I%
\sqrt{2}}{\pi a}\mathbf{\hat{k}}
\end{equation*}

This was a long hard, messy problem. But current loops are important! Every
electric circuit is a current loop. Does this mean that every circuit is
making a magnetic field? 
%TCIMACRO{%
%\TeXButton{Question 223.39.6}{\marginpar {
%\hspace{-0.5in}
%\begin{minipage}[t]{1in}
%\small{Question 223.39.6}
%\end{minipage}
%}}}%
%BeginExpansion
\marginpar {
\hspace{-0.5in}
\begin{minipage}[t]{1in}
\small{Question 223.39.6}
\end{minipage}
}%
%EndExpansion
The answer is yes! As you might guess, this can have a profound effect on
circuit design. If your circuit is very sensitive, adding extra fields (and
therefore extra forces on the charges) can be disastrous causing the design
to fail. There is some concern about \textquotedblleft electronic
noise\textquotedblright\ and possible effects on the body (cataracts are one
side effect that is well known). And of course, as the circuit changes its
current, the field it creates changes. this can create the opportunity for
espionage. The field exists far away from the circuit. A savvy spy can
determine what your circuit is doing by watching the field change!

\subsection{Long Straight wires}

In our last example, we found that the magnitude of the field due to a wire
is 
\begin{equation*}
B=\left\vert -\frac{\mu _{o}I}{4\pi a}\int \sin \theta d\theta \right\vert
\end{equation*}%
Of course, we would like to relate this to our standard charge
configuration, in this case an infinite line of (now moving) charge. If the
wire is infinitely long, then the limits of integration are just from $%
\theta =0$ to $\theta =\pi $ 
\begin{eqnarray*}
B &=&\left\vert -\frac{\mu _{o}I}{4\pi a}\int_{0}^{\pi }\sin \theta d\theta
\right\vert \\
&=&\left\vert -\left. \frac{\mu _{o}I}{4\pi a}\left( -\cos \theta \right)
\right\vert _{0}^{\pi }\right\vert \\
&=&\frac{\mu _{o}I}{2\pi a}
\end{eqnarray*}%
This is an important result. We can add a new geometry to our list of
special cases, a long straight wire that is carrying a current $I.$ The
direction of the magnetic field, we already know, is given by our
right-hand-rule. Of course, if our wire is not infinitely long, we now know
how to find the actual field. It is all a matter of finding the right limits
of integration. 
%TCIMACRO{%
%\TeXButton{Question 223.39.7}{\marginpar {
%\hspace{-0.5in}
%\begin{minipage}[t]{1in}
%\small{Question 223.39.7}
%\end{minipage}
%}}}%
%BeginExpansion
\marginpar {
\hspace{-0.5in}
\begin{minipage}[t]{1in}
\small{Question 223.39.7}
\end{minipage}
}%
%EndExpansion

\section{Magnetic dipoles}

\FRAME{dhF}{1.6492in}{2.5253in}{0in}{}{}{Figure}{\special{language
"Scientific Word";type "GRAPHIC";maintain-aspect-ratio TRUE;display
"USEDEF";valid_file "T";width 1.6492in;height 2.5253in;depth
0in;original-width 1.6129in;original-height 2.4846in;cropleft "0";croptop
"1";cropright "1";cropbottom "0";tempfilename
'M0VT8905.wmf';tempfile-properties "XPR";}}

As a second example, let's find the magnetic field due to a round loop at
the center of the loop. We start again with%
\begin{equation*}
\overrightarrow{\mathbf{B}}=\frac{\mu _{o}I}{4\pi }\int \frac{d%
\overrightarrow{\mathbf{s}}\times \mathbf{\hat{r}}}{r^{2}}
\end{equation*}%
We need to find $d\overrightarrow{\mathbf{s}}\times \mathbf{\hat{r}}$ and $r$%
, to do the integration. Our steps are:

\begin{enumerate}
\item Find an expression for $d\overrightarrow{\mathbf{s}}\times \mathbf{%
\hat{r}}$

\item Find an expression for $r$

\item Assemble the integral, including limits of integration

\item Solve the integral.
\end{enumerate}

Let's start with the first step. As we go around the loop $d\overrightarrow{%
\mathbf{s}}$ and $\mathbf{\hat{r}}$ will be perpendicular to each other, so 
\begin{equation*}
d\mathbf{s}\times \mathbf{\hat{r}}=ds\mathbf{\hat{k}}
\end{equation*}%
For the second step, we realize that $r$ is just the radius of the loop, $R$%
. Then the integration is quite easy (much easier to set up than the last
case!) 
\begin{equation*}
B=\frac{\mu _{o}I}{4\pi }\int \frac{ds}{R^{2}}\mathbf{\hat{k}}
\end{equation*}%
The limits of integration will be $0$ to $2\pi R.$ We can perform this
integral 
\begin{eqnarray*}
B &=&\frac{\mu _{o}I}{4\pi }\int_{0}^{2\pi R}\frac{ds}{R^{2}}\mathbf{\hat{k}}
\\
&=&\frac{\mu _{o}I}{4\pi }\frac{2\pi R}{R^{2}}\mathbf{\hat{k}}
\end{eqnarray*}%
so%
\begin{equation*}
B=\frac{\mu _{o}I}{2R}\mathbf{\hat{k}}\text{ \qquad loop}
\end{equation*}%
The field is perpendicular to the plane of the loop, which agrees with our
square loop problem.

Let's extend this problem to a point along the axis a distance $z$ away from
the loop. \FRAME{dtbpF}{4.2903in}{2.7071in}{0pt}{}{}{Figure}{\special%
{language "Scientific Word";type "GRAPHIC";maintain-aspect-ratio
TRUE;display "USEDEF";valid_file "T";width 4.2903in;height 2.7071in;depth
0pt;original-width 4.3436in;original-height 2.7292in;cropleft "0";croptop
"1";cropright "1";cropbottom "0";tempfilename
'S44L9301.wmf';tempfile-properties "XPR";}}We need to go back to our basic
equation again. 
\begin{equation*}
\overrightarrow{\mathbf{B}}=\frac{\mu _{o}I}{4\pi }\int \frac{d%
\overrightarrow{\mathbf{s}}\times \mathbf{\hat{r}}}{r^{2}}
\end{equation*}%
Starting with step $1,$ we realize that, in general, our value of $d%
\overrightarrow{\mathbf{s}}\times \mathbf{\hat{r}}$ is 
\begin{equation*}
d\overrightarrow{\mathbf{s}}\times \mathbf{\hat{r}}=ds\sin \phi
\end{equation*}%
where $\phi $ is the angle between $d\overrightarrow{\mathbf{s}}$ and $%
\mathbf{\hat{r}.}$ We can see that for this case $\phi $ will still be $90%
\unit{%
%TCIMACRO{\U{b0}}%
%BeginExpansion
{{}^\circ}%
%EndExpansion
}.$ \FRAME{dtbpF}{4.2052in}{3.2676in}{0in}{}{}{Figure}{\special{language
"Scientific Word";type "GRAPHIC";maintain-aspect-ratio TRUE;display
"USEDEF";valid_file "T";width 4.2052in;height 3.2676in;depth
0in;original-width 4.2566in;original-height 3.3013in;cropleft "0";croptop
"1";cropright "1";cropbottom "0";tempfilename
'S44KMF00.wmf';tempfile-properties "XPR";}}We have tipped $\mathbf{\hat{r}}$
toward our point $P,$ but tipping $\mathbf{\hat{r}}$ from pointing to the
center of the hoop to pointing to a point on the axis just rotated $\mathbf{%
\hat{r}}$ about part of the hoop. We still have $\phi =90\unit{%
%TCIMACRO{\U{b0}}%
%BeginExpansion
{{}^\circ}%
%EndExpansion
}.$ So%
\begin{equation*}
\left\vert d\overrightarrow{\mathbf{s}}\times \mathbf{\hat{r}}\right\vert =ds
\end{equation*}%
with a direction shown in the figure. We have used symmetry to argue that we
can take just $x$ or $y$-components in the past because all the others
clearly canceled out. We can also do that again here. 
%TCIMACRO{%
%\TeXButton{MIT Visualization}{\marginpar {
%\hspace{-0.5in}
%\begin{minipage}[t]{1in}
%\small{MIT Visualization}
%\end{minipage}
%}}}%
%BeginExpansion
\marginpar {
\hspace{-0.5in}
\begin{minipage}[t]{1in}
\small{MIT Visualization}
\end{minipage}
}%
%EndExpansion
Using symmetry we see that only the $z$-component of the magnetic field will
survive. So we can take the projection onto the $z$-axis. 
\begin{equation*}
\overrightarrow{\mathbf{B}}=\frac{\mu _{o}I}{4\pi }\int \frac{ds}{r^{2}}\cos
\theta \mathbf{\hat{k}}
\end{equation*}%
We know how to deal with such a situation, since we have done this before.
From the diagram we can see that 
\begin{equation*}
\cos \theta =\frac{R}{\sqrt{R^{2}+z^{2}}}
\end{equation*}%
And our value of $r$ is now more complicated, but in a way we recognize%
\begin{equation*}
r=\sqrt{R^{2}+z^{2}}
\end{equation*}%
so our field becomes%
\begin{equation*}
\overrightarrow{\mathbf{B}}=\frac{\mu _{o}I}{4\pi }\mathbf{\hat{k}}\int 
\frac{Rds}{\left( R^{2}+z^{2}\right) ^{\frac{3}{2}}}
\end{equation*}%
Fortunately this integral is also not too hard to do. Let's take out all the
terms that don't change with $ds$%
\begin{equation*}
\overrightarrow{\mathbf{B}}=\frac{\mu _{o}IR}{4\pi \left( R^{2}+z^{2}\right)
^{\frac{3}{2}}}\mathbf{\hat{k}}\int_{0}^{2\pi R}ds
\end{equation*}%
The limits of integration are $0$ to $2\pi R$, the circumference of the
circle%
\begin{eqnarray*}
\overrightarrow{\mathbf{B}} &=&\frac{\mu _{o}IR2\pi R}{4\pi \left(
R^{2}+z^{2}\right) ^{\frac{3}{2}}}\mathbf{\hat{k}} \\
&=&\frac{\mu _{o}IR^{2}}{2\left( R^{2}+z^{2}\right) ^{\frac{3}{2}}}\mathbf{%
\hat{k}}
\end{eqnarray*}

Let's take some limiting cases to see if this makes sense. Suppose $z=0,$
then%
\begin{eqnarray*}
\overrightarrow{\mathbf{B}} &=&\frac{\mu _{o}IR^{2}}{2\left( R^{2}+0\right)
^{\frac{3}{2}}}\mathbf{\hat{k}} \\
&=&\frac{\mu _{o}IR^{2}}{2R^{3}}\mathbf{\hat{k}} \\
&=&\frac{\mu _{o}I}{2R}\mathbf{\hat{k}}
\end{eqnarray*}%
which is what we got before for the field at the center of the loop. That is
comforting.

Now suppose that $z\gg R.$ In that case, we can ignore the $R^{2}$ in the
denominator.%
\begin{eqnarray*}
\overrightarrow{\mathbf{B}} &\approx &\frac{\mu _{o}IR^{2}}{2\left(
z^{2}\right) ^{\frac{3}{2}}}\mathbf{\hat{k}} \\
&=&\frac{\mu _{o}IR^{2}}{2z^{3}}\mathbf{\hat{k}}
\end{eqnarray*}%
We have just done this for on-axis positions because the math is easy there.
But we could find the field at other locations. The result looks something
like this.\FRAME{dhF}{3.614in}{1.7348in}{0pt}{}{}{Figure}{\special{language
"Scientific Word";type "GRAPHIC";maintain-aspect-ratio TRUE;display
"USEDEF";valid_file "T";width 3.614in;height 1.7348in;depth
0pt;original-width 4.7686in;original-height 2.2762in;cropleft "0";croptop
"1";cropright "1";cropbottom "0";tempfilename
'LURIFC00.wmf';tempfile-properties "XPR";}}The figure on the left was taken
from the pattern in iron filings that was created by an actual current loop
field. The figure to the right is a top down look. We will use the symbol 
\FRAME{itbpF}{0.1623in}{0.1623in}{0in}{}{}{Figure}{\special{language
"Scientific Word";type "GRAPHIC";maintain-aspect-ratio TRUE;display
"USEDEF";valid_file "T";width 0.1623in;height 0.1623in;depth
0in;original-width 0.3317in;original-height 0.3308in;cropleft "0";croptop
"1";cropright "1";cropbottom "0";tempfilename
'MWKHZM00.wmf';tempfile-properties "XPR";}}to mean \textquotedblleft coming
out of the page at you\textquotedblright\ and the symbol \FRAME{itbpF}{%
0.1747in}{0.1539in}{0in}{}{}{Figure}{\special{language "Scientific
Word";type "GRAPHIC";maintain-aspect-ratio TRUE;display "USEDEF";valid_file
"T";width 0.1747in;height 0.1539in;depth 0in;original-width
0.2191in;original-height 0.1898in;cropleft "0";croptop "1";cropright
"1";cropbottom "0";tempfilename 'MWKI4P01.wmf';tempfile-properties "XPR";}}%
\textquotedblleft going into the page away from you.\textquotedblright\
Imagine these as parts of an arrow. The dot in the circle is the arrow tip
coming at you, and the cross is the fletching going away from you. Notice
that the field is up through the loop, and down on the outside.

As we generalize our solution for the magnetic field far from the loop we
have%
\begin{equation*}
\overrightarrow{\mathbf{B}}\approx \frac{\mu _{o}IR^{2}}{2r^{3}}\mathbf{\hat{%
k}}
\end{equation*}

This looks a lot like the electric field from a dipole%
\begin{equation*}
\overrightarrow{\mathbf{E}}=\frac{2}{4\pi \epsilon _{o}}\frac{%
\overrightarrow{\mathbf{p}}}{r^{3}}
\end{equation*}%
which gives us an idea. We have a dipole moment for the electric dipole.
This magnetic field has the same basic form as the electric dipole. We can
rewrite our field as%
\begin{eqnarray*}
\overrightarrow{\mathbf{B}} &\approx &\frac{\mu _{o}I\left( \pi R^{2}\right) 
}{2\left( \pi \right) r^{3}}\mathbf{\hat{\imath}} \\
&=&\frac{\mu _{o}\left( 2\right) I\left( A\right) }{\left( 2\right) 2\left(
\pi \right) r^{3}}\mathbf{\hat{\imath}} \\
&=&\frac{\mu _{o}}{4\pi }\frac{2IA}{r^{3}}\mathbf{\hat{\imath}}
\end{eqnarray*}%
where $A=\pi R^{2}$ is the area of the loop.

The electric dipole moment is the charge multiplied by the charge separation 
\begin{equation*}
p=qa
\end{equation*}%
we have something like that in our magnetic field, The terms $IA$ describe
the amount of charge and the geometry of the charges. We will call these
terms together the \emph{magnetic dipole moment}%
\begin{equation*}
\mu =IA
\end{equation*}%
and give them a direction so that $\mu $ is a vector. The direction will be
from south to north pole\FRAME{dhF}{3.4696in}{1.6224in}{0pt}{}{}{Figure}{%
\special{language "Scientific Word";type "GRAPHIC";maintain-aspect-ratio
TRUE;display "USEDEF";valid_file "T";width 3.4696in;height 1.6224in;depth
0pt;original-width 5.0194in;original-height 2.3315in;cropleft "0";croptop
"1";cropright "1";cropbottom "0";tempfilename
'LUQ3T304.wmf';tempfile-properties "XPR";}}where we can find the south and
north poles by comparison to the field of a bar magnet.

\begin{equation*}
\overrightarrow{\mu }=IA\qquad \text{from South to North}
\end{equation*}

\FRAME{dhF}{3.1618in}{2.1153in}{0pt}{}{}{Figure}{\special{language
"Scientific Word";type "GRAPHIC";maintain-aspect-ratio TRUE;display
"USEDEF";valid_file "T";width 3.1618in;height 2.1153in;depth
0pt;original-width 3.5172in;original-height 2.3454in;cropleft "0";croptop
"1";cropright "1";cropbottom "0";tempfilename
'LUQ6LI06.wmf';tempfile-properties "XPR";}}This is a way to characterize an
entire current loop.

As we get farther from a loop, the exact shape of the loop becomes less
important. So as long as $r$ is much larger than $R,$ we can write 
\begin{equation*}
\overrightarrow{\mathbf{B}}\approx \frac{\mu _{o}}{4\pi }\frac{2%
\overrightarrow{\mu }}{r^{3}}\mathbf{\hat{k}}
\end{equation*}%
for any shaped current loop.

The integral for of the Biot-Savart law is very powerful. We can use
computers to calculate the field do to any type of current configuration.
But by hand there are only a few cases we can do because the integration
becomes difficult. With electrostatics, we found ways to use geometry to
eliminate or at least make the integration simpler. We will do the same
thing for magnetostatics starting with the next lecture. Our goal will be to
use geometry to avoid using Biot-Savart when we can.

%TCIMACRO{%
%\TeXButton{Basic Equations}{\hspace{-1.3in}{\LARGE Basic Equations\vspace{0.25in}}}}%
%BeginExpansion
\hspace{-1.3in}{\LARGE Basic Equations\vspace{0.25in}}%
%EndExpansion

\chapter{Ampere's law, and Forces on Charges}

%TCIMACRO{%
%\TeXButton{Fundamental Concepts}{\hspace{-1.3in}{\LARGE Fundamental Concepts\vspace{0.25in}}}}%
%BeginExpansion
\hspace{-1.3in}{\LARGE Fundamental Concepts\vspace{0.25in}}%
%EndExpansion

\begin{itemize}
\item The magnetic field can be found more simply for symmetric currents
using Ampere's law $\oint \overrightarrow{B}\cdot d\overrightarrow{s}=\mu
_{o}I_{through}$

\item The force due to the magnetic field on a charge, $q,$is given by $%
\overrightarrow{\mathbf{F}}=q\overrightarrow{\mathbf{v}}\times 
\overrightarrow{\mathbf{B}}$
\end{itemize}

\section{Ampere's Law}

The Biot-Savart law is a powerful technique for finding a magnetic field,
but it is more powerful numerically than in closed-form problems. We can
only find exact solutions to a few problems with special symmetry. Since
problems we can do by hand require special symmetry anyway, we would like to
use symmetry as much as possible to remove the need for difficult
integration.

We saw this situation before with electrostatics. We did some integration to
find fields from charge distributions, but then we learned Gauss' law, and
that was easier because it turned hard integration problems into relatively
easy ones. This still required special symmetry, but when it worked, it was
a fantastic time saver. For non-symmetric problems, there is always the
integration method, and a computer.

Likewise, for magnetostatics there is an easier method. To see how it works,
let's review some math.

In the figure there is a line, divided up into many little segments. \FRAME{%
dhF}{2.2295in}{1.4555in}{0pt}{}{}{Figure}{\special{language "Scientific
Word";type "GRAPHIC";maintain-aspect-ratio TRUE;display "USEDEF";valid_file
"T";width 2.2295in;height 1.4555in;depth 0pt;original-width
3.3088in;original-height 2.1508in;cropleft "0";croptop "1";cropright
"1";cropbottom "0";tempfilename 'LUQ6YU08.wmf';tempfile-properties "XPR";}}%
We can find the length of the line by adding up all the little segment
lengths 
\begin{equation*}
L=\sum_{i}\Delta s_{i}
\end{equation*}%
Integration would make this task less tedious%
\begin{equation*}
L=\int ds
\end{equation*}%
This is called a line integral. Our new method of finding magnetic fields
will involve line integrals. The calculation of the length is too simple,
however. We will have to integrate some quantity along the line. For
example, we could envision integrating the amount of energy lost when
pushing a box along a path. The integral would give the total energy loss.
The amount of energy lost would depend on the specific path. Thus a line
integral 
\begin{equation*}
W=\int \overrightarrow{\mathbf{F}}\cdot d\overrightarrow{\mathbf{s}}
\end{equation*}%
would be useful to find the total amount of work. Each small line segment
would give a differential amount of work%
\begin{equation*}
dW=\overrightarrow{\mathbf{F}}\cdot d\overrightarrow{\mathbf{s}}
\end{equation*}%
and we use the integral to add up the contribution to the work for each
segment of size $ds$ along the path. Notice the dot product. We need the dot
product because only the component of the force in the direction the box is
going adds to the total work done.\FRAME{dhF}{3.5362in}{2.1767in}{0pt}{}{}{%
Figure}{\special{language "Scientific Word";type
"GRAPHIC";maintain-aspect-ratio TRUE;display "USEDEF";valid_file "T";width
3.5362in;height 2.1767in;depth 0pt;original-width 3.4895in;original-height
2.1369in;cropleft "0";croptop "1";cropright "1";cropbottom "0";tempfilename
'LUQ89609.wmf';tempfile-properties "XPR";}}We wish to do a similar thing for
our magnetic field. We wish to integrate the magnetic field along a path.
The integral would look like this%
\begin{equation*}
\int_{a}^{b}\overrightarrow{B}\cdot d\overrightarrow{s}
\end{equation*}%
This may not look like an improvement over integrating using the Biot-Savart
law, but our goal will be to use symmetry to make this integral very easy.
The key is in the dot product. We want only the component of the magnetic
field that is in the $d\overrightarrow{s_{i}}$ direction. There are two
special cases.

If the field is perpendicular to the $d\overrightarrow{s_{i}}$ direction,
then 
\begin{equation*}
\int_{a}^{b}\overrightarrow{B}\cdot d\overrightarrow{s}=0
\end{equation*}%
because $\overrightarrow{B}\cdot d\overrightarrow{s_{i}}=0$ for this case

If the field is in the same direction as $d\overrightarrow{s_{i}},$ then $%
\overrightarrow{B}\cdot d\overrightarrow{s_{i}}=Bds$ and 
\begin{equation*}
\int_{a}^{b}\overrightarrow{B}\cdot d\overrightarrow{s}=\int_{a}^{b}Bds
\end{equation*}%
Further if we can make is so that $B$ is constant and everywhere tangent to
the path, then 
\begin{eqnarray*}
\int_{a}^{b}\overrightarrow{B}\cdot d\overrightarrow{s} &=&\int_{a}^{b}Bds \\
&=&B\int_{a}^{b}ds \\
&=&BL
\end{eqnarray*}%
This process should look familiar. We used similar arguments to make the
integral $\int \overrightarrow{E}\cdot d\overrightarrow{A}$ easy for Gauss'
law.

With Gaussian surfaces, we found we could imagine any surface we wanted. In
a similar way, for our line integral we can pick any path we want. if we can
make $B$ constant and everywhere tangent to the path, then, the integral
will be easy. It is important to realize that we get to make up our path.
There may be some physical thing along the path, but there is no need for
there to be. The paths we will use are imaginary.

Usually we will want our path to be around a closed loop. Let's take the
case of a long straight current-carrying wire. We know the field shape for
this. We can see that if we take a crazy path around the wire, that $%
\overrightarrow{B}\cdot \Delta \overrightarrow{s_{i}}$ will give us the
projection of $\overrightarrow{B}$ onto the $\Delta \overrightarrow{s_{i}}$
direction for each part of the path. \FRAME{dhF}{2.7259in}{2.5953in}{0pt}{}{%
}{Figure}{\special{language "Scientific Word";type
"GRAPHIC";maintain-aspect-ratio TRUE;display "USEDEF";valid_file "T";width
2.7259in;height 2.5953in;depth 0pt;original-width 2.6835in;original-height
2.5538in;cropleft "0";croptop "1";cropright "1";cropbottom "0";tempfilename
'LUQ8UO0A.wmf';tempfile-properties "XPR";}}We get%
\begin{equation*}
\sum_{i}B_{\Vert }\Delta s
\end{equation*}%
where $B_{\Vert }$ is the component of $B$ that is parallel to the $\Delta s$
direction. In integral form this is%
\begin{equation*}
\int B_{\Vert }ds
\end{equation*}%
The strange shape I\ drew is not very convenient. This is neither the case
where $\overrightarrow{B}\cdot d\overrightarrow{s_{i}}=0$ nor where $%
\overrightarrow{B}\cdot d\overrightarrow{s_{i}}=Bds.$ But if we think for a
moment, I\ do know a shape where $\overrightarrow{B}\cdot d\overrightarrow{%
s_{i}}=Bds.$ If we choose a circle, then from symmetry $B$ will be constant,
and it will be in the same direction as $ds$ so$\overrightarrow{B}\cdot d%
\overrightarrow{s_{i}}=Bds$. From our last lecture we even know what the
field should be for a long straight wire. 
\begin{equation*}
B=\frac{\mu _{o}I}{2\pi r}
\end{equation*}%
Let's see if we can use this to form a new general approach. Since $B$ is
constant around the loop (because $r$ is constant around the loop), we can
write our line integral as%
\begin{eqnarray*}
\int \overrightarrow{B}\cdot d\overrightarrow{s} &=&B2\pi r \\
&=&\frac{\mu _{o}I}{2\pi r}2\pi r \\
&=&\mu _{o}I
\end{eqnarray*}%
This is an amazingly simple result. We integrated the magnetic field around
an imaginary loop path, and got that the result is proportional to the
current in the wire. This reminds us of Gauss' law where we integrated the
electric field around a surface and got that the result is proportional to
the amount of charge inside the surface.

\begin{equation*}
\int \overrightarrow{E}\cdot d\overrightarrow{A}=\frac{Q_{in}}{\epsilon _{o}}
\end{equation*}

Let's review. Why did I\ pick a circle as my imaginary path? Because it made
my math easy! I\ don't want to do hard math to compute the field, so I\
tried to find a path over which the math was as easy as possible. Since the
path is imaginary, I\ can choose any path I\ want, so I\ chose a simple one.
I\ want a path where $\overrightarrow{B}\cdot d\overrightarrow{s_{i}}=0$ or
where $\overrightarrow{B}\cdot d\overrightarrow{s_{i}}=Bds.$ This is very
like picking Gaussian surfaces for Gauss' law. If I\ chose a harder path I\
would get the same answer, but it would take more effort. I found the result
of my integral $\int \overrightarrow{B}\cdot d\overrightarrow{s}$ to be just 
$\mu _{o}I.$

We had to integrate around a closed path, so I will change the integral sign
to indicate that we integrated over a closed path.

\begin{equation}
\oint \overrightarrow{B}\cdot d\overrightarrow{s}=\mu _{o}I_{through}
\end{equation}%
and only the current that went through the imaginary surface contributed to
the field, so we can mark the current as being the current that goes through
our imaginary closed path.

This process was first discovered by Ampere, so it is known as Ampere's law.

Let's use Ampere's law to do another problem. Suppose I have a coil of wire.
This coil is effectively a stack of current rings. We know the field from a
single ring.

\FRAME{dhF}{1.8118in}{1.6267in}{0pt}{}{}{Figure}{\special{language
"Scientific Word";type "GRAPHIC";maintain-aspect-ratio TRUE;display
"USEDEF";valid_file "T";width 1.8118in;height 1.6267in;depth
0pt;original-width 3.1566in;original-height 2.8305in;cropleft "0";croptop
"1";cropright "1";cropbottom "0";tempfilename
'LV0ZOJ05.wmf';tempfile-properties "XPR";}}%
\begin{equation*}
B=\frac{\mu _{o}I}{2R}
\end{equation*}%
But what would the field be that is generated by having a current flow
through the coil?

Well, looking at the single loop picture, we see that the direction of the
field due to a loop is right through the middle of the loop. I think it is
reasonable to believe that if I place another loop on top of the one
pictured, that the fields would add, making a stronger field down the
middle. This is just what happens. So I could write our loop field equation
as%
\begin{equation*}
B=N\frac{\mu _{o}I}{2r}
\end{equation*}%
where $N$ is the number of loops I make. It is customary in electronics to
define $n$ as the number of loops per unit length (sort of like the linear
mass density we defined in waves on strings, only now it is linear loop
density). Suppose I take a lot of loops! In the picture I\ have drawn the
loops like a cross section of a spring. But now the loops are not all at the
same location. So we would guess that our field will be different than just $%
N$ times the field due to one loop. We can use Ampere's law to find this
field? \FRAME{dtbpF}{1.5731in}{1.7538in}{0pt}{}{}{Figure}{\special{language
"Scientific Word";type "GRAPHIC";maintain-aspect-ratio TRUE;display
"USEDEF";valid_file "T";width 1.5731in;height 1.7538in;depth
0pt;original-width 3.774in;original-height 4.2116in;cropleft "0";croptop
"1";cropright "1";cropbottom "0";tempfilename
'LUQ9HU0C.wmf';tempfile-properties "XPR";}}Consider current is coming out at
us on the LHS and is going back into the wires on the RHS. Remember our goal
is to use Ampere's law 
\begin{equation*}
\oint \overrightarrow{B}\cdot d\overrightarrow{s}=\mu _{o}I_{through}
\end{equation*}%
to find the field. Let's imagine a rectangular shaped \emph{Amperian} loop
shown as a dotted black line. Note that like Gaussian surfaces, this is an
imaginary loop. Nothing is really there along the loop. Let's look at the
integral by breaking it into four pieces, 
\begin{equation*}
\int_{1}\overrightarrow{B}\cdot d\overrightarrow{s}+\int_{2}\overrightarrow{B%
}\cdot d\overrightarrow{s}+\int_{3}\overrightarrow{B}\cdot d\overrightarrow{s%
}+\int_{4}\overrightarrow{B}\cdot d\overrightarrow{s}=\mu _{o}I_{through}
\end{equation*}%
one for each side of the loop. If I\ have chosen my loop carefully, then $%
\overrightarrow{B}\cdot d\overrightarrow{s_{i}}$ will either be $%
\overrightarrow{B}\cdot d\overrightarrow{s_{i}}=0$ or $\overrightarrow{B}%
\cdot d\overrightarrow{s_{i}}=Bds.$ Let's start with side $2$. We want to
consider 
\begin{equation*}
\overrightarrow{B}\cdot d\overrightarrow{s_{2}}
\end{equation*}%
We see that for our side $2$ the field is perpendicular to $d\overrightarrow{%
s_{2}}$ So 
\begin{equation*}
\mathbf{B}\cdot d\ell _{2}=0
\end{equation*}%
This is great! I can integrate $0$ 
\begin{equation*}
\int 0=0
\end{equation*}%
The same reasoning applies to 
\begin{equation*}
\overrightarrow{B}\cdot d\overrightarrow{s_{4}}=0
\end{equation*}%
From our picture we can see that there is very little field outside of our
coil of loops. So $B_{3}$ is very small, so $\overrightarrow{B}\cdot d%
\overrightarrow{s_{3}}\approx 0.$ It is not exactly zero, but it is small
enough that I\ will call it negligible for this problem. For an infinite
coil, this would be exactly true (but infinite coils are hard to build).

That leaves path $1.$ There the $B$-field is in the same direction as $d%
\overrightarrow{s_{1}}$ so 
\begin{equation*}
\overrightarrow{B}\cdot d\overrightarrow{s_{1}}=Bds_{1}
\end{equation*}%
Again this is great! $B$ is fairly uniform along the coil. Let's say it is
close enough to be considered constant. Then the integral is easy over side $%
1$%
\begin{equation*}
\int Bds_{1}=B\ell
\end{equation*}%
We have performed the integral!%
\begin{eqnarray*}
\oint \overrightarrow{B}\cdot d\overrightarrow{s} &=&\int_{1}\overrightarrow{%
B}\cdot d\overrightarrow{s}+\int_{2}\overrightarrow{B}\cdot d\overrightarrow{%
s}+\int_{3}\overrightarrow{B}\cdot d\overrightarrow{s}+\int_{4}%
\overrightarrow{B}\cdot d\overrightarrow{s} \\
&=&B\ell +0+0+0 \\
&=&B\ell
\end{eqnarray*}

Now we need to find the current in the loop. This is more tricky than it
might appear. It is not just $I$ because we have several loops that go
through our loop, each on it's own carrying current $I$ and each
contributing to the field. We can use a linear loop density\footnote{%
Physicists like densities!} $n$ to find the number of loops.%
\begin{equation*}
N=n\ell
\end{equation*}%
and the current inside the loop will be 
\begin{equation*}
I_{inside}=NI
\end{equation*}%
Then, putting the integration all together, we have 
\begin{equation*}
\oint \overrightarrow{B}\cdot d\overrightarrow{s}=B\ell +0+0+0=\mu _{o}NI
\end{equation*}%
or%
\begin{equation*}
B\ell =\mu _{o}NI
\end{equation*}%
which gives a field of 
\begin{equation*}
B=\mu _{o}\frac{N}{\ell }I
\end{equation*}%
or%
\begin{equation*}
B=\mu _{o}nI
\end{equation*}

This device is so useful it has a name. It is called a \emph{solenoid}. You
may have made a coil as a kid and turned it into an electromagnet by hooking
it to a battery (a source of potential difference) so that a current ran
through it. In engineering solenoids are used as current controlled magnetic
switches.\FRAME{dtbpFU}{2.1181in}{2.4011in}{0pt}{\Qcb{{\protect\small %
Solenoid operated valve system.}}}{}{Figure}{\special{language "Scientific
Word";type "GRAPHIC";maintain-aspect-ratio TRUE;display "USEDEF";valid_file
"T";width 2.1181in;height 2.4011in;depth 0pt;original-width
2.7088in;original-height 3.0752in;cropleft "0";croptop "1";cropright
"1";cropbottom "0";tempfilename 'MP9VNU08.wmf';tempfile-properties "XPR";}}

There is another great thing about a solenoid. In the middle of the
solenoid, the field is really nearly constant. Near the ends, there are edge
effects, but in the middle we have a very uniform field. This is analogous
to the nearly uniform electric field inside a capacitor. We can therefore
see how to generate uniform magnetic fields and consider uniform $B$-fields
in problems. Such a large nearly uniform magnetic field is part of the
Compact Muon Solenoid (CMS) experiment at CERN.

\FRAME{dtbpFU}{4.2495in}{2.833in}{0pt}{\Qcb{{\protect\small CMS Detector at
CERN. The detector is constructed of a very large solenoid to bend the path
of the charge particles.}}}{}{Figure}{\special{language "Scientific
Word";type "GRAPHIC";maintain-aspect-ratio TRUE;display "USEDEF";valid_file
"T";width 4.2495in;height 2.833in;depth 0pt;original-width
4.3027in;original-height 2.8578in;cropleft "0";croptop "1";cropright
"1";cropbottom "0";tempfilename 'MP9OFV02.wmf';tempfile-properties "XPR";}}

\section{Magnetic Force on a moving charge}

Now that we know how to generate a magnetic field, we can return to thinking
about magnetic forces on mover charges. Our magnetic field is slightly more
complicated than the electric field. We can still use a charge and the
force, but now the charge is moving so we expect to have to include the
velocity of the charge. We want an expression that relates $B$ and $F_{mag}$
in both magnitude and direction.

Our expression for the relationship between charge, velocity, field and the
force comes from experiment (although now we can derive it). The experiments
show that when a charged particle moves parallel to the magnetic field,
there is no force! This is radically different from our $E$-field! Worse
yet, the force seems to be perpendicular to both $v$ and $B$ when the angle
between them is not zero! Here is our expression. 
\begin{equation}
\mathbf{F}_{B}=q\mathbf{v}\times \mathbf{B}
\end{equation}%
where $q$ is the mover charge and $B$ is the magnetic field environment.

We have a device that can shoot out electrons. The electrons show up because
they hit a phosphorescent screen. When we bring a magnet close to our beam
of electrons, we find it moves!%
%TCIMACRO{%
%\TeXButton{Magnetic Deflection Demo}{\marginpar {
%\hspace{-0.5in}
%\begin{minipage}[t]{1in}
%\small{Magnetic Deflection Demo}
%\end{minipage}
%}}}%
%BeginExpansion
\marginpar {
\hspace{-0.5in}
\begin{minipage}[t]{1in}
\small{Magnetic Deflection Demo}
\end{minipage}
}%
%EndExpansion

But we did this with moving electrons, what happens if they are not moving?
We might expect the electrons to accelerate just the same--and we would be
wrong! Static charges seem to not notice the presence of the magnet at all!

We expect that, like gravity and electric charge, the force on the moving
electrons must be due to a field, but this \emph{magnetic field} does not
accelerate stationary electrons. We learned before that the reason we know
that there is some force on the electrons came when Oersted, a Dutch
scientist experimenting with electric current, found that his compass acted
strangely when it was near a wire carrying electric current. This discovery
is backwards of our experiment. It implies that moving charges must effect
magnets, but given Newton's third law, If moving electrons make a field that
makes a force on a magnet, then we would expect a magnet will make a field
that makes a force on moving charges as well!

The derivation of the magnitude of the force from the experimental data is
tedious. We will just learn the results, but they are exciting enough! The
magnitude of the force on a moving charge due to a constant magnetic field
is 
\begin{equation}
\overrightarrow{\mathbf{F}}_{B}=q\overrightarrow{\mathbf{v}}\times 
\overrightarrow{\mathbf{B}}  \label{MagneticFieldForceOnCharge}
\end{equation}%
The magnitude is given by 
\begin{equation*}
F=qvB\sin \theta
\end{equation*}%
where $q$ is charge, $v$ is speed, and $B$ is the magnitude of the magnetic
field We need to carefully define $\theta .$ Since we have a cross product, $%
\theta $ is the angle between the field direction and the velocity direction.

We can solve the equation for the magnetic field force (equation \ref%
{MagneticFieldForceOnCharge}) to find the magnitude of the field 
\begin{equation*}
\frac{F}{qv\sin \theta }=B
\end{equation*}

But the strangeness has not ended. we need a direction of the force. And it
turns out that it is perpendicular to both $\overrightarrow{\mathbf{v}}$ and 
$\overrightarrow{\mathbf{B}}$ as the cross product implies! We use our
favorite right hand rule to help us remember.\FRAME{dhF}{1.286in}{1.3673in}{%
0in}{}{}{Figure}{\special{language "Scientific Word";type
"GRAPHIC";maintain-aspect-ratio TRUE;display "USEDEF";valid_file "T";width
1.286in;height 1.3673in;depth 0in;original-width 1.2505in;original-height
1.3327in;cropleft "0";croptop "1";cropright "1";cropbottom "0";tempfilename
'M0XTOA00.wmf';tempfile-properties "XPR";}}We start with our hand pointing
in the direction of $\mathbf{\vec{v}}.$ Curl your fingers in the direction
of $\mathbf{\vec{B}.}$ And your fingers will point in the direction of the
force. We saw this type of right hand rule before with torque, but there is
one big difference. This really is the direction the charge will accelerate!
Note that this works for a positive charge. If the charge is negative, then
the $\ q$ in 
\begin{equation*}
\overrightarrow{\mathbf{F}}_{B}=q\overrightarrow{\mathbf{v}}\times 
\overrightarrow{\mathbf{B}}
\end{equation*}%
will be negative, and so the force will go in the other way. To keep this
straight in my own mind, I still use our right hand rule, and just remember
that if $F$ is negative, it goes the opposite way of my thumb.

\begin{Note}
Right hand rule \#2: We start with our hand pointing in the direction of $%
\mathbf{\vec{v}}.$ Curl your fingers in the direction of $\mathbf{\vec{B}.}$
And your fingers will point in the direction of the force. The magnitude of
the force is given by 
\begin{equation}
F=qvB\sin \theta
\end{equation}
\end{Note}

\section{Motion of a charged particle in a $B$-Field}

%TCIMACRO{%
%\TeXButton{Question 223.40.1}{\marginpar {
%\hspace{-0.5in}
%\begin{minipage}[t]{1in}
%\small{Question 223.40.1}
%\end{minipage}
%}}}%
%BeginExpansion
\marginpar {
\hspace{-0.5in}
\begin{minipage}[t]{1in}
\small{Question 223.40.1}
\end{minipage}
}%
%EndExpansion
%TCIMACRO{%
%\TeXButton{Question 223.40.2}{\marginpar {
%\hspace{-0.5in}
%\begin{minipage}[t]{1in}
%\small{Question 223.40.2}
%\end{minipage}
%}}}%
%BeginExpansion
\marginpar {
\hspace{-0.5in}
\begin{minipage}[t]{1in}
\small{Question 223.40.2}
\end{minipage}
}%
%EndExpansion
%TCIMACRO{%
%\TeXButton{Question 223.40.3}{\marginpar {
%\hspace{-0.5in}
%\begin{minipage}[t]{1in}
%\small{Question 223.40.3}
%\end{minipage}
%}}}%
%BeginExpansion
\marginpar {
\hspace{-0.5in}
\begin{minipage}[t]{1in}
\small{Question 223.40.3}
\end{minipage}
}%
%EndExpansion
%TCIMACRO{%
%\TeXButton{Question 223.40.5}{\marginpar {
%\hspace{-0.5in}
%\begin{minipage}[t]{1in}
%\small{Question 223.40.5}
%\end{minipage}
%}}}%
%BeginExpansion
\marginpar {
\hspace{-0.5in}
\begin{minipage}[t]{1in}
\small{Question 223.40.5}
\end{minipage}
}%
%EndExpansion
We refer to the magnetic field as a $B$-field for short.

Let's set up a constant $B$-field as shown in the figure. We draw a $B$%
-field as a set of vectors just like we did for electric fields. In the
figure, the vectors are all pointing \textquotedblleft into the
paper\textquotedblright\ so all we can see is their tails. \FRAME{dtbpF}{%
2.514in}{1.9995in}{0pt}{}{}{Figure}{\special{language "Scientific Word";type
"GRAPHIC";maintain-aspect-ratio TRUE;display "USEDEF";valid_file "T";width
2.514in;height 1.9995in;depth 0pt;original-width 5.1249in;original-height
4.0698in;cropleft "0";croptop "1";cropright "1";cropbottom "0";tempfilename
'LUQPAU0I.wmf';tempfile-properties "XPR";}}%
%TCIMACRO{%
%\TeXButton{Question 223.40.6}{\marginpar {
%\hspace{-0.5in}
%\begin{minipage}[t]{1in}
%\small{Question 223.40.6}
%\end{minipage}
%}}}%
%BeginExpansion
\marginpar {
\hspace{-0.5in}
\begin{minipage}[t]{1in}
\small{Question 223.40.6}
\end{minipage}
}%
%EndExpansion
If I\ have a charged particle, with velocity $\mathbf{\vec{v},}$ what will
be the motion of the particle in the field? First off, we should recall that 
$\mathbf{\vec{F}}$ is in a direction perpendicular to $\mathbf{\vec{v}}$ and 
$\overrightarrow{\mathbf{B}}$.. Using our right hand rule we see that it
will go to the left. \FRAME{dtbpF}{2.4829in}{1.9752in}{0pt}{}{}{Figure}{%
\special{language "Scientific Word";type "GRAPHIC";maintain-aspect-ratio
TRUE;display "USEDEF";valid_file "T";width 2.4829in;height 1.9752in;depth
0pt;original-width 5.1249in;original-height 4.0698in;cropleft "0";croptop
"1";cropright "1";cropbottom "0";tempfilename
'LUQPAU0J.wmf';tempfile-properties "XPR";}}Remember that $F=ma,$ so the
charge will accelerate in the $-x$ direction.\FRAME{dhF}{2.0686in}{1.6881in}{%
0pt}{}{}{Figure}{\special{language "Scientific Word";type
"GRAPHIC";maintain-aspect-ratio TRUE;display "USEDEF";valid_file "T";width
2.0686in;height 1.6881in;depth 0pt;original-width 2.0297in;original-height
1.6518in;cropleft "0";croptop "1";cropright "1";cropbottom "0";tempfilename
'M0XU7601.wmf';tempfile-properties "XPR";}}Now, if we allow the charged
particle to move, we see that the $v$ direction changes. This makes the
direction of $F$ change. Since $v$ and $a$ are always at $90\unit{%
%TCIMACRO{\U{b0}}%
%BeginExpansion
{{}^\circ}%
%EndExpansion
},$ the motion reminds us of circular motion! Let's see if we can find the
radius of the circular path of the charge. 
\begin{equation*}
F=qvB\sin \theta
\end{equation*}%
will be just 
\begin{equation*}
F=qvB
\end{equation*}%
because $\theta $ is always $90\unit{%
%TCIMACRO{\U{b0}}%
%BeginExpansion
{{}^\circ}%
%EndExpansion
}.$ Then, using Newton's second law%
\begin{equation*}
F=ma=qvB
\end{equation*}%
and noting that the acceleration is center-seeking, and our velocity is
always tangential, we can write it as a centripetal acceleration%
\begin{equation*}
a_{C}=\frac{v_{t}^{2}}{r}
\end{equation*}

Then 
\begin{equation*}
m\frac{v_{t}^{2}}{r}=qv_{t}B
\end{equation*}%
\begin{equation*}
m\frac{v_{t}}{r}=qB
\end{equation*}%
We can find the radius of the circle%
\begin{equation*}
\frac{mv_{t}}{qB}=r
\end{equation*}%
Could we find the angular speed?%
\begin{equation*}
\omega =\frac{v_{t}}{r}=\frac{qB}{m}
\end{equation*}

How about the period? We can take the total distance divided by the total
time for a revolution%
\begin{equation*}
v_{t}=\frac{2\pi r}{T}
\end{equation*}%
to find%
\begin{equation*}
T=\frac{2\pi r}{v_{t}}
\end{equation*}%
and we recognize 
\begin{equation*}
\frac{1}{\omega }=\frac{r}{v_{t}}
\end{equation*}%
so%
\begin{equation*}
T=\frac{2\pi }{\omega }
\end{equation*}%
so, using our angular speed we can say 
\begin{equation*}
T=\frac{2\pi m}{qB}
\end{equation*}

The angular frequency $\omega $ that we found is the frequency of a type of
particle accelerator called a cyclotron. This type of accelerator is used by
places like CERN to start the acceleration of charged particles. The same
concept is used to make the charged particles go in a circular path in the
large accelerators like the LHC at CERN.\FRAME{dtbpFU}{3.0237in}{1.7979in}{%
0pt}{\Qcb{{\protect\small Turning magnets at CERN. This is an actual magnet,
but this magnet is at ground level in the testing facility. The tunnel is a
mock-up of what the actual beam tunnel looks like.}}}{}{Figure}{\special%
{language "Scientific Word";type "GRAPHIC";maintain-aspect-ratio
TRUE;display "USEDEF";valid_file "T";width 3.0237in;height 1.7979in;depth
0pt;original-width 3.053in;original-height 1.8041in;cropleft "0";croptop
"1";cropright "1";cropbottom "0";tempfilename
'MP9O7H01.wmf';tempfile-properties "XPR";}}Within the detector systems, like
the CMS, charged product particles can be tracked along curved paths for
identification.

%TCIMACRO{%
%\TeXButton{Question 223.40.7}{\marginpar {
%\hspace{-0.5in}
%\begin{minipage}[t]{1in}
%\small{Question 223.40.7}
%\end{minipage}
%}}}%
%BeginExpansion
\marginpar {
\hspace{-0.5in}
\begin{minipage}[t]{1in}
\small{Question 223.40.7}
\end{minipage}
}%
%EndExpansion
%TCIMACRO{%
%\TeXButton{Question 223.40.8}{\marginpar {
%\hspace{-0.5in}
%\begin{minipage}[t]{1in}
%\small{Question 223.40.8}
%\end{minipage}
%}}}%
%BeginExpansion
\marginpar {
\hspace{-0.5in}
\begin{minipage}[t]{1in}
\small{Question 223.40.8}
\end{minipage}
}%
%EndExpansion
%TCIMACRO{%
%\TeXButton{Question 223.40.9}{\marginpar {
%\hspace{-0.5in}
%\begin{minipage}[t]{1in}
%\small{Question 223.40.9}
%\end{minipage}
%}}}%
%BeginExpansion
\marginpar {
\hspace{-0.5in}
\begin{minipage}[t]{1in}
\small{Question 223.40.9}
\end{minipage}
}%
%EndExpansion
But it is also interesting to know that charged particles that enter a
magnetic field with some initial speed will gain a circular motion as well. 
\FRAME{dhF}{2.4881in}{1.6881in}{0pt}{}{}{Figure}{\special{language
"Scientific Word";type "GRAPHIC";maintain-aspect-ratio TRUE;display
"USEDEF";valid_file "T";width 2.4881in;height 1.6881in;depth
0pt;original-width 2.4474in;original-height 1.6518in;cropleft "0";croptop
"1";cropright "1";cropbottom "0";tempfilename
'M0XV2203.wmf';tempfile-properties "XPR";}}An example is the charged
particles from the Sun entering the Earth's magnetic field. the particles
will spiral around the magnetic field lines. \FRAME{dhF}{3.2984in}{1.5212in}{%
0pt}{}{}{Figure}{\special{language "Scientific Word";type
"GRAPHIC";maintain-aspect-ratio TRUE;display "USEDEF";valid_file "T";width
3.2984in;height 1.5212in;depth 0pt;original-width 3.2534in;original-height
1.4849in;cropleft "0";croptop "1";cropright "1";cropbottom "0";tempfilename
'MP9VWA0A.wmf';tempfile-properties "XPR";}}As the helical motion tightens
near the poles, the particles will sometimes give off patterns of light as
they hit atmospheric atoms. \FRAME{dtbpFU}{3.6748in}{2.449in}{0pt}{\Qcb{%
Aurora Borealis: Sand Creek Ponds Idaho 2013}}{}{Figure}{\special{language
"Scientific Word";type "GRAPHIC";maintain-aspect-ratio TRUE;display
"USEDEF";valid_file "T";width 3.6748in;height 2.449in;depth
0pt;original-width 3.7164in;original-height 2.4667in;cropleft "0";croptop
"1";cropright "1";cropbottom "0";tempfilename
'MP9VW909.wmf';tempfile-properties "XPR";}}The light is what we call the
aurora borealis. A more high-tech use for this helical motion is the
confinement of charged particles in a magnetic field for fusion reaction.

\subsection{The velocity selector}

%TCIMACRO{%
%\TeXButton{Question 223.40.10}{\marginpar {
%\hspace{-0.5in}
%\begin{minipage}[t]{1in}
%\small{Question 223.40.10}
%\end{minipage}
%}}}%
%BeginExpansion
\marginpar {
\hspace{-0.5in}
\begin{minipage}[t]{1in}
\small{Question 223.40.10}
\end{minipage}
}%
%EndExpansion

\FRAME{dhF}{2.7812in}{1.7158in}{0in}{}{}{Figure}{\special{language
"Scientific Word";type "GRAPHIC";maintain-aspect-ratio TRUE;display
"USEDEF";valid_file "T";width 2.7812in;height 1.7158in;depth
0in;original-width 2.7389in;original-height 1.6786in;cropleft "0";croptop
"1";cropright "1";cropbottom "0";tempfilename
'M0Y2O407.wmf';tempfile-properties "XPR";}}

This device shows up on tests, especially finals, because it has both an
electric field and a magnetic field--you test two sets of knowledge at once!
So let's see how it works. Our question should be, what is the velocity of a
charged particle that travels through the field without being deflected?

$E$\textbf{-field}

We remember that the force on a positively charged particle will be 
\begin{equation*}
F_{E}=qE
\end{equation*}%
directed in the field direction so it is downward.

$B$\textbf{-Field}

Now we know that 
\begin{equation*}
\mathbf{F}_{B}=q\mathbf{v}\times \mathbf{B}
\end{equation*}%
and we use our right hand rule to find that the direction will be upward
with a magnitude of 
\begin{eqnarray*}
F_{B} &=&qvB\sin \theta \\
&=&qvB
\end{eqnarray*}

So there will be no deflection (no acceleration) when the forces in the $y$%
-direction balance.%
\begin{equation*}
\Sigma F_{y}=0=-F_{E}+F_{B}
\end{equation*}%
or%
\begin{equation*}
qE=qvB
\end{equation*}%
which gives 
\begin{equation*}
v=\frac{E}{B}
\end{equation*}%
as the speed that will be \textquotedblleft selected.\textquotedblright\ 

\subsection{Bainbridge Mass Spectrometer}

You may use a mass-spec some time in your careers. I have had samples
identified by mass-spectrometers several times in my industrial career. They
are very useful devices--especially when chemical identification is hard or
impossible.\FRAME{dhF}{4.011in}{2.5676in}{0in}{}{}{Figure}{\special{language
"Scientific Word";type "GRAPHIC";maintain-aspect-ratio TRUE;display
"USEDEF";valid_file "T";width 4.011in;height 2.5676in;depth
0in;original-width 3.9617in;original-height 2.5261in;cropleft "0";croptop
"1";cropright "1";cropbottom "0";tempfilename
'M0Y35608.wmf';tempfile-properties "XPR";}}The Bainbridge device is one type
that we can easily understand. It starts with a velocity selector which
sends charged particles at a particular speed into a region of uniform
magnetic field. The charged particles then follow curved paths on their way
to an array of detectors. When they hit the array, their spatial location is
recorded. Where they hit depends on their ration of charge to mass. From our
study of the rotational motion we found 
\begin{equation*}
r=\frac{mv}{qB_{o}}
\end{equation*}%
so the charge to mass ratio is 
\begin{equation*}
\frac{q}{m}=\frac{v}{rB_{o}}
\end{equation*}%
Since we know the initial velocity will be 
\begin{equation*}
v=\frac{E}{B}
\end{equation*}%
from the velocity selector, then 
\begin{equation*}
\frac{q}{m}=\frac{E}{rBB_{o}}
\end{equation*}

One way this is often used is to separate a sample of substance, say, carbon
to find the relative amount of each isotope. The carbon atoms will all
ionize to the same charge. Then the position at which they are detected
depends on the mass.

I used a mass-spec in my last industry project to identify large carbon
compounds and their relative concentration in complex oil leaks. This data
helped us look for possible leak detection targets so pipeline leaks could
be detected before the oil was visible to the naked eye.

\subsection{Classical Cyclotron}

We already found the period of rotation of a charged particle in a uniform
magnetic field.%
\begin{equation*}
T=\frac{2\pi m}{qB}
\end{equation*}%
Note that this does not depend on the speed of the particle! So it will have
the same travel time regardless of how fast it goes. We can use this to
accelerate particles. But we add in an electric field to do the
acceleration. The device is shown in the figure below \FRAME{dhFU}{1.8447in}{%
1.8005in}{0pt}{\Qcb{Basic Geometry of the Cyclotron. (Public Domain image
courtesy KlausFoehl)}}{}{Figure}{\special{language "Scientific Word";type
"GRAPHIC";maintain-aspect-ratio TRUE;display "USEDEF";valid_file "T";width
1.8447in;height 1.8005in;depth 0pt;original-width 1.8075in;original-height
1.7634in;cropleft "0";croptop "1";cropright "1";cropbottom "0";tempfilename
'M0Y11405.wmf';tempfile-properties "XPR";}}The particle starts in the center
circling around in the magnetic field. but the device is divided into halves
(called \textquotedblleft Ds\textquotedblright ). There is a gap between the
Ds, and the electric field is created in the gap. One side at high potential
and the other at low potential. When the particle is in the gap, it
accelerates. It will gain a kinetic energy equal to the potential energy
difference across the gap%
\begin{equation*}
\Delta K=q\Delta V
\end{equation*}%
As the particle travels around the D to the other side of the, the
cyclotron, the cyclotron switches the polarity of the potential difference.
So as the particle passes the gap on the other side of the cyclotron, it is
again accelerated with an additional $\Delta K=q\Delta V.$ Since $r$ does
depend on the speed,%
\begin{equation*}
r=\frac{mv}{qB}
\end{equation*}%
the radius increases with each \textquotedblleft kick.\textquotedblright\
Finally the particle leaves the cyclotron with a velocity of 
\begin{equation*}
\frac{qBr_{\max }}{m}=v
\end{equation*}%
Since we often describe the velocity of particles in energy terms, the
kinetic energy of the particle%
\begin{eqnarray*}
K &=&\frac{1}{2}mv^{2} \\
&=&\frac{1}{2}m\left( \frac{qBr_{\max }}{m}\right) ^{2} \\
&=&\frac{q^{2}B^{2}r_{\max }^{2}}{2m}
\end{eqnarray*}

\section{Hall Effect}

%TCIMACRO{%
%\TeXButton{Hall Effect Demo}{\marginpar {
%\hspace{-0.5in}
%\begin{minipage}[t]{1in}
%\small{Hall Effect Demo}
%\end{minipage}
%}}}%
%BeginExpansion
\marginpar {
\hspace{-0.5in}
\begin{minipage}[t]{1in}
\small{Hall Effect Demo}
\end{minipage}
}%
%EndExpansion
The Hall probe is a cool little device that measures the magnitude of the
magnetic field. It is used in rotation and angle detection in engineering.
We should find out how it works.\FRAME{dtbpF}{2.0721in}{1.9847in}{0pt}{}{}{%
Figure}{\special{language "Scientific Word";type
"GRAPHIC";maintain-aspect-ratio TRUE;display "USEDEF";valid_file "T";width
2.0721in;height 1.9847in;depth 0pt;original-width 2.034in;original-height
1.9458in;cropleft "0";croptop "1";cropright "1";cropbottom "0";tempfilename
'M6EFO000.wmf';tempfile-properties "XPR";}}

Let's take a piece of material that has a current going through it. If we
place it in a magnetic field, then the charge carriers will feel a force.
Suppose it is a metal, and that the charge carriers are electrons. The force
is perpendicular to the current direction. So the electrons are accelerated
toward the top of the piece of metal as shown in the drawing. This creates a
negative charge on the top side of the metal piece. Then the bottom side
will be positively charged relative to the top. With separated charge like
this, we think of a capacitor and the electric field created by such a
separation of charges. There will be a field in the conductor with a
potential difference between the top and bottom of the conductor. We call
this potential difference%
\begin{equation*}
\Delta V_{H}
\end{equation*}%
the Hall potential after the man who first observed it.

Now if the charge carriers were positive, we would still build up a
potential, but it would be in the opposite polarity. We wish to find this
hall potential. The electric field of the charges will try to push them back
down as more charge builds up. So at some point the upward force due to the
magnetic field on the electrons will be balanced by the built up electric
field. At that point%
\begin{equation*}
\Sigma F_{y}=0=F_{B}-F_{E}
\end{equation*}%
so 
\begin{equation*}
qv_{d}B=qE_{H}
\end{equation*}%
where $E_{H}$ is the field due to the separation of charges.

So 
\begin{equation*}
E_{H}=v_{d}B
\end{equation*}%
The potential is nearly equal to 
\begin{equation*}
\Delta V\approx E_{H}d
\end{equation*}%
where $d$ is the top-to-bottom distance of the conductor , so 
\begin{equation*}
\Delta V\approx v_{d}Bd
\end{equation*}%
Since we know 
\begin{equation*}
I=nqAv_{d}
\end{equation*}%
then 
\begin{equation*}
v_{d}=\frac{I}{nqA}
\end{equation*}%
and the area $A$ is 
\begin{equation*}
A=td
\end{equation*}%
where $t$ is the thickness of the conductor, then 
\begin{equation*}
v_{d}=\frac{I}{nqtd}
\end{equation*}%
and%
\begin{equation*}
\Delta V\approx \frac{IB}{nqt}
\end{equation*}%
You may find this expressed in terms of the Hall coefficient%
\begin{equation*}
R_{H}=\frac{1}{nq}
\end{equation*}%
so%
\begin{equation*}
\Delta V\approx R_{H}\frac{IB}{t}
\end{equation*}%
To do a good job of finding $R_{H}$ for metals and semiconductors, you have
to go beyond classical theory. But if we know $B,$ $I,$ $t,$ and $\Delta V,$
which can all be measured, then we can find $R_{H}.$ Once this is done, we
can place the Hall probe in different magnetic fields to find their
strength. One way to do this is to control $I$ and measure $\Delta V,$ so%
\begin{equation*}
B\approx \frac{t}{R_{H}I}\Delta V
\end{equation*}

%TCIMACRO{%
%\TeXButton{Basic Equations}{\hspace{-1.3in}{\LARGE Basic Equations\vspace{0.25in}}}}%
%BeginExpansion
\hspace{-1.3in}{\LARGE Basic Equations\vspace{0.25in}}%
%EndExpansion

\chapter{Magnetic forces on wires}

%TCIMACRO{%
%\TeXButton{Fundamental Concepts}{\hspace{-1.3in}{\LARGE Fundamental Concepts\vspace{0.25in}}}}%
%BeginExpansion
\hspace{-1.3in}{\LARGE Fundamental Concepts\vspace{0.25in}}%
%EndExpansion

\begin{itemize}
\item The magnetic force on moving charges extends to wires with currents

\item The force on a wire with current is given by $\mathbf{F}_{I}=I\mathbf{%
L\times B}$

\item The torque on a current loop is $\mathbf{\tau =\mu }\times \mathbf{B}$
where $\mathbf{\mu }=I\mathbf{A}$
\end{itemize}

\section{Magnetic forces on Current-Carrying wires}

%TCIMACRO{%
%\TeXButton{Question 223.41.1}{\marginpar {
%\hspace{-0.5in}
%\begin{minipage}[t]{1in}
%\small{Question 223.41.1}
%\end{minipage}
%}}}%
%BeginExpansion
\marginpar {
\hspace{-0.5in}
\begin{minipage}[t]{1in}
\small{Question 223.41.1}
\end{minipage}
}%
%EndExpansion
%TCIMACRO{%
%\TeXButton{Question 223.41.2}{\marginpar {
%\hspace{-0.5in}
%\begin{minipage}[t]{1in}
%\small{Question 223.41.2}
%\end{minipage}
%}}}%
%BeginExpansion
\marginpar {
\hspace{-0.5in}
\begin{minipage}[t]{1in}
\small{Question 223.41.2}
\end{minipage}
}%
%EndExpansion
If there is a force on a single moving charge due to a magnetic field, then
there must be a force on lots of moving charges! We call lots of moving
charges an electric current%
\begin{equation*}
I=\frac{\Delta Q}{\Delta t}
\end{equation*}%
For charges in a wire, we know that the charges move along the wire with a
velocity $v_{d}.$ We would expect the total force on all the charges to be
the sum of all the forces on the individual charges.%
\begin{equation*}
F_{I}=\dsum\limits_{i}F_{q_{i}}=\dsum\limits_{i}q_{i}vB\sin \theta
\end{equation*}%
but, since in our wire all the charge carriers are the same, this is just 
\begin{equation*}
F_{I}=Nq_{i}v_{d}B\sin \theta
\end{equation*}%
where here $N$ is the number of charge carriers in the part of the wire that
is experiencing the field. We used a charge density $n$ before. Let's use it
again to make an expression for $N$%
\begin{equation*}
N=nV=nAL
\end{equation*}%
where $A$ is the cross sectional area of the wire and $L$ is the length of
the wire. So%
\begin{equation*}
F_{I}=nALq_{i}v_{d}B\sin \theta
\end{equation*}%
Now let's think back to our definition of current. We know that 
\begin{equation*}
I=nq_{i}v_{d}A
\end{equation*}%
so our force on the current carrying wire is%
\begin{eqnarray*}
F_{I} &=&\left( nq_{i}v_{d}A\right) LB\sin \theta \\
&=&ILB\sin \theta
\end{eqnarray*}%
Remember that $\theta $ is the angle between the field direction and the
velocity. In this case $I$ is in the direction of the velocity (we still
assume positive charge carriers, even though we know they are electrons
going the other way). So $\theta $ is the angle between the field direction
and the direction of the current. We can write this as a cross product%
\begin{equation}
\overrightarrow{\mathbf{F}}_{I}=I\overrightarrow{\mathbf{L}}\mathbf{\times }%
\overrightarrow{\mathbf{B}}
\end{equation}%
where $\overrightarrow{\mathbf{L}}$ is in the current direction.

\subsection{Force between two wires}

%TCIMACRO{%
%\TeXButton{Question 223.41.2.3}{\marginpar {
%\hspace{-0.5in}
%\begin{minipage}[t]{1in}
%\small{Question 223.41.2.3}
%\end{minipage}
%}}}%
%BeginExpansion
\marginpar {
\hspace{-0.5in}
\begin{minipage}[t]{1in}
\small{Question 223.41.2.3}
\end{minipage}
}%
%EndExpansion
We can use what we have learned to find the force between two wires.

If I\ have two wires with current, I will have a field created by each wire.
Let's suppose that $I_{1}$ and $I_{2}$ are in the same direction\FRAME{dhF}{%
2.194in}{2.2883in}{0pt}{}{}{Figure}{\special{language "Scientific Word";type
"GRAPHIC";maintain-aspect-ratio TRUE;display "USEDEF";valid_file "T";width
2.194in;height 2.2883in;depth 0pt;original-width 2.1543in;original-height
2.2485in;cropleft "0";croptop "1";cropright "1";cropbottom "0";tempfilename
'M0ZI3O02.wmf';tempfile-properties "XPR";}}and let's calculate the force on
wire $1$ due to the field of wire $2.$ The field due to wire $2$ at the
location of wire $1$ will be 
\begin{equation*}
B_{12}=\frac{\mu _{o}I_{2}}{2\pi d}
\end{equation*}%
where $d$ is how far away wire $1$ is from wire $2.$ We know%
\begin{equation*}
F_{12}=I_{1}LB_{2}\sin \theta
\end{equation*}%
We can see that $\sin \theta =1$ since $I_{1}$ will be perpendicular to $%
B_{12.}$%
\begin{equation*}
F_{12}=I_{1}LB_{12}
\end{equation*}

and using our expression for $B_{12}$%
\begin{eqnarray}
F_{12} &=&I_{1}L\frac{\mu _{o}I_{2}}{2\pi d}  \notag \\
&=&L\frac{\mu _{o}I_{2}I_{1}}{2\pi d}
\end{eqnarray}%
Would you expect $F_{21}$ to be very different?

\section{Torque on a Current Loop}

%TCIMACRO{%
%\TeXButton{Question 223.41.4}{\marginpar {
%\hspace{-0.5in}
%\begin{minipage}[t]{1in}
%\small{Question 223.41.4}
%\end{minipage}
%}}}%
%BeginExpansion
\marginpar {
\hspace{-0.5in}
\begin{minipage}[t]{1in}
\small{Question 223.41.4}
\end{minipage}
}%
%EndExpansion
%TCIMACRO{%
%\TeXButton{Question 223.41.5}{\marginpar {
%\hspace{-0.5in}
%\begin{minipage}[t]{1in}
%\small{Question 223.41.5}
%\end{minipage}
%}}}%
%BeginExpansion
\marginpar {
\hspace{-0.5in}
\begin{minipage}[t]{1in}
\small{Question 223.41.5}
\end{minipage}
}%
%EndExpansion
Remember that in PH121 or Statics and Dynamics we defined angular
displacement%
\begin{equation}
\Delta \theta =\theta _{f}-\theta _{i}
\end{equation}%
and this told us how far in angle we had traveled from a starting point $%
\theta _{i}.$

We also defined the angular velocity%
\begin{equation}
\omega =\frac{\Delta \theta }{\Delta t}
\end{equation}%
which told us how fast an object was spinning in radians per second. The
direction of this angular velocity we found using a right hand rule.

We also defined an angular acceleration%
\begin{equation}
\alpha =\frac{\Delta \omega }{\Delta t}
\end{equation}%
and we used angular acceleration in combination with a moment of inertia to
express a rotational form of Newton's second law%
\begin{equation}
\sum \tau =I\alpha
\end{equation}%
where $\tau $ is a torque. We found torque with the expression 
\begin{equation}
\overrightarrow{\mathbf{\tau }}=\overrightarrow{\mathbf{r}}\times 
\overrightarrow{\mathbf{F}}
\end{equation}%
We wish to apply these ideas to our new force on wires due to magnetism.

Let's take a specific example. I want to use a current loop. This is just
the simple loop of current we have seen before.\FRAME{dtbpF}{2.009in}{%
2.2606in}{0pt}{}{}{Figure}{\special{language "Scientific Word";type
"GRAPHIC";maintain-aspect-ratio TRUE;display "USEDEF";valid_file "T";width
2.009in;height 2.2606in;depth 0pt;original-width 3.4826in;original-height
3.9219in;cropleft "0";croptop "1";cropright "1";cropbottom "0";tempfilename
'LUQPL60R.wmf';tempfile-properties "XPR";}}I want to place this into a
magnetic field.\FRAME{dtbpF}{2.1698in}{1.516in}{0pt}{}{}{Figure}{\special%
{language "Scientific Word";type "GRAPHIC";maintain-aspect-ratio
TRUE;display "USEDEF";valid_file "T";width 2.1698in;height 1.516in;depth
0pt;original-width 5.8608in;original-height 4.0785in;cropleft "0";croptop
"1";cropright "1";cropbottom "0";tempfilename
'LUQPL60S.wmf';tempfile-properties "XPR";}}

I drew the current loop as a rectangle on purpose, I want to look at the
force on the current for each part of the loop. Each side of our loop is a
straight wire segment. Remember that the magnitude of the force on a wire is
given by 
\begin{equation*}
F_{I}=ILB\sin \theta
\end{equation*}%
where $\theta $ is the angle between $I$ and $B$ so if $\theta =0$ or if $%
\theta =\pi \unit{rad}$, then $\sin \theta $ will be zero. The magnitude of
the force will then be zero. So the top and bottom parts of the loop will
not experience a force. The sides will, though, and since for $\theta =\frac{%
\pi }{2}$ or $\theta =-\frac{\pi }{2}$ ($\theta =-\frac{\pi }{2}$ is the
same as $\theta =\frac{3\pi }{2}$) then $\sin \theta =1$ and the force will
be a maximum.%
\begin{equation*}
F_{I}=IbB
\end{equation*}%
on each side wire segment. But we need to consider direction. The force will
be perpendicular to both $I$ and $B.$ We use our right hand rule. Fingers in
the direction of $I,$ curl to the direction of $B.$ We see the force is out
of the figure for the left hand side and into the figure for the right hand
side. The next figure is a bottom-up view.

\FRAME{dhF}{1.6319in}{2.0911in}{0pt}{}{}{Figure}{\special{language
"Scientific Word";type "GRAPHIC";maintain-aspect-ratio TRUE;display
"USEDEF";valid_file "T";width 1.6319in;height 2.0911in;depth
0pt;original-width 2.1266in;original-height 2.7337in;cropleft "0";croptop
"1";cropright "1";cropbottom "0";tempfilename
'M0ZHCJ00.wmf';tempfile-properties "XPR";}}Clearly the loop will want to
turn! This looks like a nice problem for us to describe with a torque. We
have a force acting at a distance from a pivot. We have a torque%
\begin{equation*}
\tau =rF\sin \psi
\end{equation*}%
We have already used $\theta ,$ and our torque angle is the angle between $r$
and $F$, so we needed a new greek letter. I\ have used $\psi $\footnote{%
which is a \emph{psi}}. Then $\psi $ is the angle between $r$ and $F.$

Let's fill in the details of our total torque. Remember we have two torques,
one for the left hand side, and one for the right and side. Their magnitudes
are the same, and the directions we need to get from yet another right hand
rule. Both are in the same direction so 
\begin{eqnarray*}
\tau &=&\frac{a}{2}F_{I}\sin \left( \psi \right) +\frac{a}{2}\left(
F_{I}\right) \sin \left( \psi \right) \\
&=&aF_{I}\sin \left( \psi \right)
\end{eqnarray*}%
Putting in the force magnitude gives%
\begin{equation*}
\tau =a\left( IbB\right) \sin \psi
\end{equation*}%
and rearranging lets us see%
\begin{eqnarray*}
\tau &=&\left( ab\right) IB\sin \psi \\
&=&\left( A\right) IB\sin \psi
\end{eqnarray*}%
where $A=ab$ is the area of our loop. Of course we can write this as%
\begin{equation}
\overrightarrow{\mathbf{\tau }}\mathbf{=}I\overrightarrow{\mathbf{A}}\times 
\overrightarrow{\mathbf{B}}
\end{equation}%
The torque is the cross product of the area vector and the magnetic field
multiplied by the current.

We did this for a square loop. It turns out that it works for any loop shape.

When things rotate, we expect to use moments. We defined a magnetic dipole
moment for a current loop. Now we can see why it is useful. The magnetic
moment tells us about how much torque we will get for a particular current
loop.%
\begin{equation*}
\overrightarrow{\mathbf{\mu }}_{d}=I\overrightarrow{\mathbf{A}}
\end{equation*}%
using this we have%
\begin{equation*}
\overrightarrow{\mathbf{\tau }}\mathbf{=}\overrightarrow{\mathbf{\mu }}%
_{d}\times \overrightarrow{\mathbf{B}}
\end{equation*}

We could envision our loop as a single circle of wire connected to a
battery. But we could just as easily double up the wire. If we do this, what
is our torque? Well we would have twice the force, because we now have twice
the current (the current goes trough both turns of the wire). So now we have 
\begin{equation*}
\tau =2\left( A\right) IB\sin \psi
\end{equation*}%
But why stop there? We could make three loops all together.%
\begin{equation*}
\tau =3\left( A\right) IB\sin \psi
\end{equation*}%
or many more, say $N$ loops,%
\begin{equation*}
\tau =NAIB\sin \psi
\end{equation*}

Thinking of our magnetic dipole moment, we see that 
\begin{equation*}
\tau =N\mu _{d}B\sin \psi
\end{equation*}%
for a coil. We could combine the effects of all the loops into one magnetic
moment that represents the coil. 
\begin{equation}
\overrightarrow{\mathbf{\mu }}=N\overrightarrow{\mathbf{A}}I
\end{equation}%
then 
\begin{equation*}
\tau =\mu B\sin \psi
\end{equation*}%
or in cross product form%
\begin{equation}
\overrightarrow{\mathbf{\tau }}\mathbf{=}\overrightarrow{\mathbf{\mu }}%
\times \overrightarrow{\mathbf{B}}
\end{equation}%
Using this total magnetic moment, we can more easily do problems with coils
in magnetic fields.

For example, we found that there was a potential energy associated with
spinning dipoles, for a spinning current loop we also expect a potential
energy. We have a simple formula for this potential energy in terms of the
magnetic moment.

\begin{equation}
U=-\overrightarrow{\mathbf{\mu }}\cdot \overrightarrow{\mathbf{B}}
\end{equation}

%TCIMACRO{%
%\TeXButton{Question 223.41.5}{\marginpar {
%\hspace{-0.5in}
%\begin{minipage}[t]{1in}
%\small{Question 223.41.5}
%\end{minipage}
%}}}%
%BeginExpansion
\marginpar {
\hspace{-0.5in}
\begin{minipage}[t]{1in}
\small{Question 223.41.5}
\end{minipage}
}%
%EndExpansion

\subsection{Galvanometer}

We finally know enough to understand how to measure a current. The device is
called a \emph{galvanometer}. \FRAME{dtbpF}{1.7536in}{1.7953in}{0pt}{}{}{%
Figure}{\special{language "Scientific Word";type
"GRAPHIC";maintain-aspect-ratio TRUE;display "USEDEF";valid_file "T";width
1.7536in;height 1.7953in;depth 0pt;original-width 1.7158in;original-height
1.7573in;cropleft "0";croptop "1";cropright "1";cropbottom "0";tempfilename
'MFGN3Z01.wmf';tempfile-properties "XPR";}}In the picture, we see the
typical design of a galvanometer. It has a coil of wire (shown looking at
the side of the coil) and a spring. The coil is placed between the ends of a
magnet. When there is a current in the wire, there will be a torque on the
coil that will compress the spring. The amount of torque depends on the
current. As the current increases, the spring is more compressed. A marker
(large needle) is attached to the apparatus. As the spring is compressed,
the indicator moves across the scale. Since this movement is proportional to
the current, a galvanometer can easily measure current.

\subsection{Electric Motors}

%TCIMACRO{%
%\TeXButton{Question 223.41.6}{\marginpar {
%\hspace{-0.5in}
%\begin{minipage}[t]{1in}
%\small{Question 223.41.6}
%\end{minipage}
%}}}%
%BeginExpansion
\marginpar {
\hspace{-0.5in}
\begin{minipage}[t]{1in}
\small{Question 223.41.6}
\end{minipage}
}%
%EndExpansion
With our new understanding of torque on a current loop, we should be able to
see how an electric motor works. A current loop is placed in between two
magnets to form a magnetic field. The loop will turn because of the torque
due to the $B$-field. But we have to get clever. What happens when the loop
turns half way around so the current is now going the opposite way?\FRAME{%
dtbpF}{3.6893in}{2.6195in}{0pt}{}{}{Figure}{\special{language "Scientific
Word";type "GRAPHIC";maintain-aspect-ratio TRUE;display "USEDEF";valid_file
"T";width 3.6893in;height 2.6195in;depth 0pt;original-width
5.924in;original-height 4.1926in;cropleft "0";croptop "1";cropright
"1";cropbottom "0";tempfilename 'LUQPN20Z.wmf';tempfile-properties "XPR";}}%
Now the torque switches direction and the loop will come to rest. We don't
want that if we are building a motor, so we have to switch the current
direction every time the loop turns half way.\FRAME{dhF}{2.3817in}{1.9346in}{%
0pt}{}{}{Figure}{\special{language "Scientific Word";type
"GRAPHIC";maintain-aspect-ratio TRUE;display "USEDEF";valid_file "T";width
2.3817in;height 1.9346in;depth 0pt;original-width 12.1792in;original-height
9.8813in;cropleft "0";croptop "1";cropright "1";cropbottom "0";tempfilename
'LUQPN210.wmf';tempfile-properties "XPR";}}The way we do this is to have
electrical contacts that are flexible, called brushes. The brushes contact a
metal ring. The metal ring is connected to the loop. But the ring has two
slits cut out of it.\FRAME{dhF}{1.1234in}{1.0585in}{0pt}{}{}{Figure}{\special%
{language "Scientific Word";type "GRAPHIC";maintain-aspect-ratio
TRUE;display "USEDEF";valid_file "T";width 1.1234in;height 1.0585in;depth
0pt;original-width 2.9334in;original-height 2.7622in;cropleft "0";croptop
"1";cropright "1";cropbottom "0";tempfilename
'LUQPN211.wmf';tempfile-properties "XPR";}}The ring with slits is called a
commutator As the loop turns, the commutator turns, and when it has turned a
half turn, the brushes switch sides. This changes the current direction,
which puts us back at maximum torque.\FRAME{dhF}{2.2943in}{1.7236in}{0pt}{}{%
}{Figure}{\special{language "Scientific Word";type
"GRAPHIC";maintain-aspect-ratio TRUE;display "USEDEF";valid_file "T";width
2.2943in;height 1.7236in;depth 0pt;original-width 5.0004in;original-height
3.7498in;cropleft "0";croptop "1";cropright "1";cropbottom "0";tempfilename
'LUQPN212.wmf';tempfile-properties "XPR";}}This keeps the motor going the
same direction.

%TCIMACRO{%
%\TeXButton{Basic Equations}{\hspace{-1.3in}{\LARGE Basic Equations\vspace{0.25in}}}}%
%BeginExpansion
\hspace{-1.3in}{\LARGE Basic Equations\vspace{0.25in}}%
%EndExpansion

\chapter{Permanent Magnets, Induction}

%TCIMACRO{%
%\TeXButton{Fundamental Concepts}{\hspace{-1.3in}{\LARGE Fundamental Concepts\vspace{0.25in}}}}%
%BeginExpansion
\hspace{-1.3in}{\LARGE Fundamental Concepts\vspace{0.25in}}%
%EndExpansion

\begin{itemize}
\item Using classical physics, we can't quite explain a permanent magnet.

\item Using a semiclassical model, the permanent magnet's field is due to
spinning electrons.

\item Alignment of the spinning electrons creates what we call magnetism.

\item Temporary alignment results in paramagnetism and diamagnetism.

\item More permanent alignment yields ferromagnetism.

\item A changing magnetic field can create an emf.
\end{itemize}

\section{Finally, why magnets work}

We started our study of magnetism by looking at bar magnets and considering
that if we break one, we end up with two magnets. We don't end up with a
north end and a south end as separate pieces. This is different than charge
and electric fields. Then we studied how moving charge makes a magnetic
field. But we didn't say how bar magnets work yet.%
%TCIMACRO{%
%\TeXButton{Question 223.42.1}{\marginpar {
%\hspace{-0.5in}
%\begin{minipage}[t]{1in}
%\small{Question 223.42.1}
%\end{minipage}
%}} }%
%BeginExpansion
\marginpar {
\hspace{-0.5in}
\begin{minipage}[t]{1in}
\small{Question 223.42.1}
\end{minipage}
}
%EndExpansion
Can a permanent magnet have something to do with current loops?\FRAME{dhF}{%
3.4696in}{1.6224in}{0pt}{}{}{Figure}{\special{language "Scientific
Word";type "GRAPHIC";maintain-aspect-ratio TRUE;display "USEDEF";valid_file
"T";width 3.4696in;height 1.6224in;depth 0pt;original-width
5.0194in;original-height 2.3315in;cropleft "0";croptop "1";cropright
"1";cropbottom "0";tempfilename 'LV2M2102.wmf';tempfile-properties "XPR";}}%
Well, lets look at the field due to a current loop. It looks a lot like the
field due to a magnet. Could there be current loops inside a bar magnet? The
answer is well, sort of... We have electrons that sort of travel around the
atom. Suppose the electrons orbit like planets. Then there would be a
current as they travel. For one electron the current would be%
\begin{equation*}
I=\frac{q_{e}}{T}
\end{equation*}%
where $T$ is the period of rotation. And we recall from PH121 or Statics and
Dynamics that the period of rotation can be found from 
\begin{equation*}
\omega =\frac{2\pi }{T}
\end{equation*}%
so that 
\begin{equation*}
T=\frac{2\pi }{\omega }
\end{equation*}%
Then the current is 
\begin{equation*}
I=\frac{q_{e}\omega }{2\pi }
\end{equation*}%
It is an amount of charge per unit time. We can write this as 
\begin{equation*}
I=q_{e}\frac{\omega }{2\pi }
\end{equation*}%
and recalling 
\begin{equation*}
v_{t}=\omega r
\end{equation*}%
then%
\begin{equation*}
I=q_{e}\frac{v_{t}}{2\pi r}
\end{equation*}%
We can find a magnetic moment (a good review of what we have learned!) 
\begin{eqnarray*}
\mu &=&NIA=\left( 1\right) IA \\
&=&q_{e}\frac{v_{t}}{2\pi r}\left( \pi r^{2}\right) \\
&=&\frac{q_{e}v_{t}r}{2}
\end{eqnarray*}%
Physicists often write this in terms of angular momentum. Just to review,
angular momentum is given by%
\begin{equation*}
L=\mathbb{I}_{m}\omega
\end{equation*}%
where $\mathbb{I}_{m}$ is the moment of inertia. Then 
\begin{eqnarray*}
L &=&\mathbb{I}_{m}\omega \\
&=&\left( mr^{2}\right) \left( \frac{v_{t}}{r}\right) \\
&=&mrv_{t}
\end{eqnarray*}%
so the magnetic moment of the orbiting electron would be%
\begin{equation}
\mu =\frac{q_{e}L}{2m}
\end{equation}%
which gives us a magnetic moment related to the angular momentum of the
electron. And if we have a magnetic moment this not only means the atoms
would orient in an external field but it also means that the atoms work as
little magnets. We will have a magnetic field

\subsection{Quantum effects}

%TCIMACRO{%
%\TeXButton{Question 223.42.2}{\marginpar {
%\hspace{-0.5in}
%\begin{minipage}[t]{1in}
%\small{Question 223.42.2}
%\end{minipage}
%}}}%
%BeginExpansion
\marginpar {
\hspace{-0.5in}
\begin{minipage}[t]{1in}
\small{Question 223.42.2}
\end{minipage}
}%
%EndExpansion
All of this works well for Hydrogen. We find that individual hydrogen atoms
do act like small magnets. But if the hydrogen is in a compound, it is more
complicated because we then have many electrons and they are
\textquotedblleft orbiting\textquotedblright\ in different directions. It is
even true that most atoms have many electrons, and within the atom these
electrons fly around in all different directions. The magnetic field due to
one electron in the atom cancels out the magnetic field due to another, so
there is no net magnetic field do to the \textquotedblleft
motion\textquotedblright\ of the electrons in their orbitals. So in general
there is no net magnetic field from orbital motion. Even for Hydrogen in a
compound the overall magnetic moment of the compound tends to cancel out.

Further, we know that electrons do not travel like planets in circular
orbits. So our model for magnetism is not really correct yet.\footnote{%
Strictly speaking, the electrons don't move like little planets around the
nucleus. So it is not clear that orbital \textquotedblleft
motion\textquotedblright\ makes much sense. The electrons form standing
waves around the nucleus that oscillate in time. But they do have angular
momentum, and that orbital angular momentum cancels out for most atoms. For
more on this, take PH279, Modern Physics.} To understand the current model
of electron orbitals takes some quantum mechanics (and a few more years of
physics). But we can understand a little, because quantum mechanics does
tell us that the electrons have angular momentum. The big difference is that
the angular momentum is \emph{quantized} meaning it can have only certain
values (think of the quantized modes of an oscillating string). The smallest
magnetic moment for an electron turns out to be 
\begin{equation}
\mu =\sqrt{2}\frac{q_{e}}{2m_{e}}\hslash
\end{equation}%
where 
\begin{equation*}
\hslash =\frac{h}{2\pi }=1.05\times 10^{-34}\unit{J}\unit{s}
\end{equation*}%
is pronounced \textquotedblleft h-bar\textquotedblright\ and is a constant.
We encountered Planck's constant $h$ before ($h=6.63\times 10^{-34}\unit{J}%
\unit{s}$). This is just Planck's constant divided by $2\pi .$ So it would
seem that with only certain values being available the magnetic moments
might be more likely to line up.

But it turns out that even in quantum mechanics, the magnetic moments of the
electrons due to their orbits cancel each other out most of the time.

But there is another contribution to the magnetic moment, this time from the
electron, itself. The electron has an amount of angular momentum. It is as
though it spins on an axis. This spin angular momentum is also quantized. It
can take values of%
\begin{equation}
S=\pm \frac{\sqrt{3}}{2}\hslash
\end{equation}%
My mental picture of this is a charged ball spinning on an axis.\footnote{%
But this is just a mental model. Electrons don't spin. They do have angular
momentum. But how electrons generate their angular momentum is still hard to
tell. Physicists are working on this.} \FRAME{dhF}{1.0343in}{0.921in}{0pt}{}{%
}{Figure}{\special{language "Scientific Word";type
"GRAPHIC";maintain-aspect-ratio TRUE;display "USEDEF";valid_file "T";width
1.0343in;height 0.921in;depth 0pt;original-width 1.0006in;original-height
0.8882in;cropleft "0";croptop "1";cropright "1";cropbottom "0";tempfilename
'LV16G408.wmf';tempfile-properties "XPR";}} The magnetic moment due to spin
is 
\begin{equation}
\mu _{s}=\frac{q_{e}\hslash }{2m_{e}}
\end{equation}

This means that electrons, themselves are little magnets. Where does this
magnetic moment come from? Well it is \emph{as though} the electron is
constantly spinning. It is not really, but this is a semi-classical mental
model that we can use to envision the source of the electron's magnetic
field. The \textquotedblleft spinning\textquotedblright\ electron is
charged, so the electron acts like a miniscule current loop. The electron,
itself is a source of the magnetic field for permanent magnets.

The spin magnetic moment was given the strange name \emph{Bohr magneton} in
honor of Niels Bohr. If there are many electrons in the atom, there will be
many contributions to the total atomic magnetic moment. The nucleus also has
a magnetic moment (a detail we will not discuss at any length in our class)
and there are other details like electron spin states pairing up. But those
are topics for PH279 and our senior quantum mechanics class. It turns out
that this spin magnetic moment is the major cause that produces permanent
magnetism in some metals. We don't want to wade though a senior level
physics class now (well, you probably don't anyway) so we need a more
macroscopic description of magnetism. But fundamentally, if we can get the
electrons spins in a material to line up, we will have a magnet.

\subsection{Ferromagnetism}

%TCIMACRO{%
%\TeXButton{Question 223.42.3}{\marginpar {
%\hspace{-0.5in}
%\begin{minipage}[t]{1in}
%\small{Question 223.42.3}
%\end{minipage}
%}}}%
%BeginExpansion
\marginpar {
\hspace{-0.5in}
\begin{minipage}[t]{1in}
\small{Question 223.42.3}
\end{minipage}
}%
%EndExpansion
Because of the spin magnetic moment, we can see some hope for how a
permanent might work. But these spin magnetic moments are also mostly
randomly arranged. So again, most atoms won't have an overall magnetic
moment. But some atoms do have a slight net field. They have an odd number
of electrons. So the last electron can have an unbalanced magnetic moment.
That atom would act as a magnet

Still, this does not produce much of an effect, because neighboring atoms
all are oriented differently. So neighboring atoms cancel each other out. In
a few materials, though, the atoms within small volumes will align their
magnetic moments. These little domains form small magnets. But still the
overall effect is very small because the domains are all oriented in
different directions.\FRAME{dhF}{1.3837in}{1.3396in}{0pt}{}{}{Figure}{%
\special{language "Scientific Word";type "GRAPHIC";maintain-aspect-ratio
TRUE;display "USEDEF";valid_file "T";width 1.3837in;height 1.3396in;depth
0pt;original-width 1.3482in;original-height 1.305in;cropleft "0";croptop
"1";cropright "1";cropbottom "0";tempfilename
'LV0YTR03.wmf';tempfile-properties "XPR";}}If we place these materials in a
magnetic field, we can make the domains align, and then we have something!%
\FRAME{dhF}{1.2756in}{1.2445in}{0pt}{}{}{Figure}{\special{language
"Scientific Word";type "GRAPHIC";maintain-aspect-ratio TRUE;display
"USEDEF";valid_file "T";width 1.2756in;height 1.2445in;depth
0pt;original-width 1.8075in;original-height 1.7634in;cropleft "0";croptop
"1";cropright "1";cropbottom "0";tempfilename
'LV0YUF04.wmf';tempfile-properties "XPR";}}Few materials can do this. The
ones that can are called ferromagnetic. Iron is one material. We can make
the domains align, but the alignment decays quickly. That is why iron
objects stick to a magnet, but don't stick to each other when they are taken
away from the magnet. But if we can force the domains to stay in one
direction, say, by heating the ferromagnetic metal in a magnetic field and
letting it cool and form crystals, then we can make a magnet that will last
longer. The magnetic moments will get stuck all pointing about the same
direction as the ferromagnetic metal cools. Some materials like Cobalt form
very long lasting permanent magnets.

\subsection{Magnetization vector}

We now know that each atom of a substance may have a magnetic moment. For a
block of the material, it is useful to think of the magnetic moment per unit
volume. We will call this $\mathbf{M.}$ It must be a vector, so that if
there is an overall magnetic moment, we have a magnet! Let's see how to use
this new quantity.

Suppose I have a current carrying wire that produces a field $\mathbf{B}%
_{o}. $ But I also have a material where $\mathbf{M}$ is not zero. Then
there must be a a field due to the magnetic material $\mathbf{B}_{m}.$ So
the total field will be%
\begin{equation}
\mathbf{B}=\mathbf{B}_{o}+\mathbf{B}_{m}
\end{equation}%
and all we have to do is determine the relationship between $\mathbf{B}_{m}$
and $\mathbf{M.}$

\subsection{Solenoid approximation}

Lets look at two atoms, We will model them as little current loops, since
they have magnetic moments.

\FRAME{dtbpF}{2.2053in}{0.7481in}{0pt}{}{}{Figure}{\special{language
"Scientific Word";type "GRAPHIC";maintain-aspect-ratio TRUE;display
"USEDEF";valid_file "T";width 2.2053in;height 0.7481in;depth
0pt;original-width 4.4836in;original-height 1.506in;cropleft "0";croptop
"1";cropright "1";cropbottom "0";tempfilename
'LUQPN217.wmf';tempfile-properties "XPR";}}notice that in between the loops,
the currents go opposite directions. We could think of them as canceling. We
get a net current that is to the outside of the loops\FRAME{dtbpF}{2.1638in}{%
0.9444in}{0pt}{}{}{Figure}{\special{language "Scientific Word";type
"GRAPHIC";maintain-aspect-ratio TRUE;display "USEDEF";valid_file "T";width
2.1638in;height 0.9444in;depth 0pt;original-width 4.5916in;original-height
1.9887in;cropleft "0";croptop "1";cropright "1";cropbottom "0";tempfilename
'LUQPN218.wmf';tempfile-properties "XPR";}}Now let's take many current loops.%
\FRAME{dtbpF}{1.8637in}{1.9847in}{0pt}{}{}{Figure}{\special{language
"Scientific Word";type "GRAPHIC";maintain-aspect-ratio TRUE;display
"USEDEF";valid_file "T";width 1.8637in;height 1.9847in;depth
0pt;original-width 4.6464in;original-height 4.9476in;cropleft "0";croptop
"1";cropright "1";cropbottom "0";tempfilename
'LUQPN219.wmf';tempfile-properties "XPR";}}again, the inside currents
cancel, leaving an overall current along the outside. Now if we view a
material as a stack of such current loops\FRAME{dtbpF}{1.2756in}{1.1in}{0pt}{%
}{}{Figure}{\special{language "Scientific Word";type
"GRAPHIC";maintain-aspect-ratio TRUE;display "USEDEF";valid_file "T";width
1.2756in;height 1.1in;depth 0pt;original-width 2.4206in;original-height
2.0832in;cropleft "0";croptop "1";cropright "1";cropbottom "0";tempfilename
'LUQPN21A.wmf';tempfile-properties "XPR";}}we can model a magnetic material
like a solenoid! That is great, because we know how to find the field of a
solenoid.

\begin{eqnarray*}
B_{m} &=&\mu _{o}nI \\
&=&\mu _{o}\frac{NIA}{\ell A}
\end{eqnarray*}%
I didn't cancel the $A$s because I want to recognize the numerator as the
magnetic moment 
\begin{equation*}
\mu =NIA
\end{equation*}%
so 
\begin{equation*}
B_{m}=\mu _{o}\frac{\mu }{\ell A}
\end{equation*}%
But note that $\ell A$ is just the volume of the piece of magnetic material,
so 
\begin{equation*}
B_{m}=\mu _{o}\frac{\mu }{V}
\end{equation*}%
which is gives us our new quantity, the magnetization vector%
\begin{equation}
M=\frac{\mu }{V}
\end{equation}%
Well, this is the magnitude, anyway, so 
\begin{equation}
B_{m}=\mu _{o}M
\end{equation}%
and of course the directions must be the same, since $\mu _{o}$ is just a
scalar constant%
\begin{equation}
\mathbf{B}_{m}=\mu _{o}\mathbf{M}
\end{equation}

So the total field is given by

\begin{equation}
\mathbf{B}=\mathbf{B}_{o}+\mu _{o}\mathbf{M}
\end{equation}

\subsection{Magnetic Field Strength (another confusing name)}

%TCIMACRO{%
%\TeXButton{Only do this if you have extra time}{\marginpar {
%\hspace{-0.5in}
%\begin{minipage}[t]{1in}
%\small{Only do this if you have extra time}
%\end{minipage}
%}}}%
%BeginExpansion
\marginpar {
\hspace{-0.5in}
\begin{minipage}[t]{1in}
\small{Only do this if you have extra time}
\end{minipage}
}%
%EndExpansion
Sometimes we physicists just can't let things alone. So when we arrived at
the equation%
\begin{equation}
\mathbf{B}=\mathbf{B}_{o}+\mu _{o}\mathbf{M}
\end{equation}%
someone wanted to define a new term%
\begin{equation}
\frac{\mathbf{B}_{o}}{\mu _{o}}
\end{equation}%
so we could write the equation%
\begin{equation}
\mathbf{B}=\mu _{o}\left( \frac{\mathbf{B}_{o}}{\mu _{o}}+\mathbf{M}\right)
\end{equation}%
This new term is given an unfortunate name. The \emph{magnetic field strength%
}. \textbf{It is not the magnitude of the magnetic field}, but is the
magnitude divided by the constant $\mu _{o}.$ It has it's own symbol, $%
\mathbf{H}.$ So you may write our total field equation as%
\begin{equation}
\mathbf{B}=\mu _{o}\left( \mathbf{H}+\mathbf{M}\right)
\end{equation}

You might find this change unnecessary and confusing (I do) but it is
tradition to use this notation, and is not bad once you get used to it.

\subsection{Macroscopic properties of magnetic materials}

We want a way to describe how \textquotedblleft magnetic\textquotedblright\
different substances are without doing quantum mechanics. This will allow us
to classify materials, and choose the proper material for whatever
experiment or device we are designing.

For many substances we find that the magnetization vector is proportional to
the field strength (which is why field strength hangs around in usage)%
\begin{equation}
\mathbf{M}=\chi \mathbf{H}
\end{equation}

For many materials, this nice linear relationship applies, and we can look
up the constant of proportionality in a table. The name of the constant $%
\chi $ is the \emph{magnetic susceptibility}.

If $\chi $ is positive ($M$ is in the same direction as $H)$, we call the
material \emph{paramagnetic}.

If $\chi $ is negative ($M$ is in the opposite direction as $H)$, we call
the material \emph{diamagnetic}.

Using this new notation, our total field becomes%
\begin{equation*}
\mathbf{B}=\mu _{o}\left( \mathbf{H}+\mathbf{M}\right)
\end{equation*}%
\begin{equation*}
\mathbf{B}=\mu _{o}\left( \mathbf{H}+\chi \mathbf{H}\right)
\end{equation*}%
\begin{equation}
\mathbf{B}=\mu _{o}\left( 1+\chi \right) \mathbf{H}
\end{equation}%
The quantity $\mu _{o}\left( 1+\chi \right) $ is also given a name, 
\begin{equation}
\mu _{m}=\mu _{o}\left( 1+\chi \right)
\end{equation}%
it is called the magnetic permeability. Now you see why $\mu _{o}$ is called
the permeability of free space! (the name was not so random after all!). If $%
\chi =0$ then 
\begin{equation}
\mu _{m}=\mu _{o}
\end{equation}%
and this is the case for free space. We can write definitions of
paramagnetism and diamagnetism in terms of the permeability.

\begin{equation*}
\begin{tabular}{ll}
Paramagnetic & $\mu _{m}>\mu _{o}$ \\ 
Diamagnetic & $\mu _{m}<\mu _{o}$ \\ 
Free Space & $\mu _{m}=\mu _{o}$%
\end{tabular}%
\end{equation*}

For paramagnetic and diamagnetic materials, $\mu _{m}$ is usually not too
different from $\mu _{o}$ but for ferromagnetic materials $\mu _{m}$ is much
larger than $\mu _{o}.$ Note that we have not included ferromagnetic
substances in this discussion. That is because 
\begin{equation*}
\mathbf{M}=\chi \mathbf{H}
\end{equation*}%
is not true for ferromagnetic materials.

\subsection{Ferromagnetism revisited}

%TCIMACRO{%
%\TeXButton{Question 223.42.4}{\marginpar {
%\hspace{-0.5in}
%\begin{minipage}[t]{1in}
%\small{Question 223.42.4}
%\end{minipage}
%}}}%
%BeginExpansion
\marginpar {
\hspace{-0.5in}
\begin{minipage}[t]{1in}
\small{Question 223.42.4}
\end{minipage}
}%
%EndExpansion
But why is ferromagnetism different? To try to understand, let's take a iron
toroid (doughnut shape) and wrap it with a coil as shown. \FRAME{dtbpF}{%
2.2978in}{2.4647in}{0pt}{}{}{Figure}{\special{language "Scientific
Word";type "GRAPHIC";maintain-aspect-ratio TRUE;display "USEDEF";valid_file
"T";width 2.2978in;height 2.4647in;depth 0pt;original-width
4.2359in;original-height 4.5455in;cropleft "0";croptop "1";cropright
"1";cropbottom "0";tempfilename 'LUQPN21B.wmf';tempfile-properties "XPR";}}%
We have a magnetic field meter that measures the field inside the windings
of the coil. When we throw the switch, the coil produces a magnetic field.
The field will produce a magnetization vector in the iron toroid and,
therefore, a field strength. We can plot the applied magnetic field vs. the
field strength to see how much effect the applied field has on the magnetic
properties of the iron toroid. We won't do this mathematically, but the
result is shown in the figure.\FRAME{dtbpF}{2.8228in}{2.5875in}{0pt}{}{}{%
Figure}{\special{language "Scientific Word";type
"GRAPHIC";maintain-aspect-ratio TRUE;display "USEDEF";valid_file "T";width
2.8228in;height 2.5875in;depth 0pt;original-width 2.7795in;original-height
2.5469in;cropleft "0";croptop "1";cropright "1";cropbottom "0";tempfilename
'M6DYEK00.wmf';tempfile-properties "XPR";}}As we throw the switch, we go
from no alignment of the domains so zero $M$ and therefore zero induced
field in the iron toroid to a value that represents the almost complete
alignment of the magnetic moments of each atom of the iron. This is point $%
a. $ It may take a bit of current, but in theory we can always do this. All
the domains are aligned and $M$ is maximum.

Now we reduce the current from our battery, and we find that the field due
to the aligned domains drops as expected, but not along the same path that
we started on! We go from $a$ to $b$ as the current decreases. At point $b$
there is no current, but we still have a magnetic field in the toroid!

We can even keep going and reverse the field by changing the polarity of our
power supply contacts. Since we still have some field in the toroid, it
actually takes some reverse current to overcome the residual field. But if
we apply enough reverse current, then we get alignment in the other
direction. Almost complete alignment is at point $d.$ If we again reduce the
current and find that--once again--it does not retrace the same path!

Each time we align the domains with our applied external field from the
coil, the domains in the iron toroid seem to want to stay aligned. Most do
lose alignment, but some stay put. We have created a weak permanent magnet
by placing our ferromagnetic material in a strong external magnetic field.

This strangely shaped curve is the \emph{magnetization curve} for the
material. The fact that the path is a strange loop instead of always
following the same path is called \emph{magnetic hysteresis}. We can see now
that the external field (represented by the current $I,$ since $%
B_{extrnal}\varpropto I$) and magnetization don't behave in a simple
relationship like they did for diamagnetic or permanganic materials.

The thickness of the area traced by the hysteresis curve depends on the
material. It also represents the energy required to take the material
through the hysteresis cycle.

If we add enough thermal energy, it is hard to keep the atomic dipole
moments aligned. The next figure shows this effect.

\FRAME{dhF}{1.4659in}{1.446in}{0pt}{}{}{Figure}{\special{language
"Scientific Word";type "GRAPHIC";maintain-aspect-ratio TRUE;display
"USEDEF";valid_file "T";width 1.4659in;height 1.446in;depth
0pt;original-width 2.2796in;original-height 2.2485in;cropleft "0";croptop
"1";cropright "1";cropbottom "0";tempfilename
'LV2M0B01.wmf';tempfile-properties "XPR";}}At a temperature called the Curie
temperature, the material no longer acts ferromagnetic. It becomes simply
paramagnetic. So if we heat up a permanent magnet, we expect it to lose it's
alignment and therefore to stop being a magnet. This is what happens to
ferromagnetic materials when they are heated due to volcanism. The domains
are destroyed and all the atoms lose alignment. Whey the material cools, the
Earth's magnetic field acts as an external field and some of the domains
will be aligned with this field. This is how we know that the Earth's
magnetic field switches polarity. We can see which way the magnetization
vector points in the cooled lava deposits from places like the Mid-Atlantic
Trench.

This is also how old fashioned magnetic tapes and disks work.

\subsection{Paramagnetism}

We said that if $\mu _{m}>\mu _{o}$ we get paramagnetism. But what is
paramagnetism? It comes from the material having a small natural magnetic
susceptibility.%
\begin{equation}
0<\chi \ll 1
\end{equation}%
So in the presence of an external magnetic field, you can force the little
magnetic moments to line up. You are competing with thermal motion as we saw
in ferromagnetism, so the effect is usually weak. A rule of thumb for
paramagnetism is that 
\begin{equation}
M=C\frac{B_{o}}{T}
\end{equation}%
where $C$ contains the particular material properties of the substance you
are investigating (another thing to look up in tables in data sets), $B_{o}$
is the applied field, and $T$ is the temperature. In other words, if it is
cool enough, a paramagnetic material becomes a magnet in the presence of an
external magnetic field. This is a little like polarization of neutral
insulators in the presence of an electric field. For paramagnetic materials,
the induced magnetic field is in the same direction as the external field.

Some examples of paramagnetic materials and their susceptibilities are given
below%
\begin{equation*}
\begin{tabular}{ll}
Material & Susceptibility \\ 
Tungsten & $6.8\times 10^{-5}$ \\ 
Aluminium & $2.2\times 10^{-5}$ \\ 
Sodium & $0.72\times 10^{-5}$%
\end{tabular}%
\end{equation*}

\subsection{Diamagnetism.}

If $\mu _{m}<\mu _{o}$ we said we would have diamagnetism. This is
fundamentally quite different from paramagnetism. It comes from the material
having paired electrons that orbit the atom (classical model). The magnetic
moments of the electrons will have equal magnitudes, but opposite directions
(a little bit of quantum mechanics to go with our classical model). When the
external field is applied, one electron's orbit is enhanced by the field,
and the other is diminished (think $q\mathbf{v}\times B$). So there will be
a net magnetic moment. If you think about this for a while, you will realize
that the new net magnetic moment is in the opposite direction of the applied
external field! So diamagnetism will always repel.

There is always some diamagnetism in all mater. We can enhance the effect
using a superconductor. The diamagnetism of the superconductor repeals the
external field entirely! Why does this happen only for superconductors?
Well, that will take more theory to discover (a great topic for our junior
level electrodynamics class). But the phenomena is called the Meissner
effect.

%TCIMACRO{%
%\TeXButton{Meissner effect demo}{\marginpar {
%\hspace{-0.5in}
%\begin{minipage}[t]{1in}
%\small{Meissner effect demo}
%\end{minipage}
%}}}%
%BeginExpansion
\marginpar {
\hspace{-0.5in}
\begin{minipage}[t]{1in}
\small{Meissner effect demo}
\end{minipage}
}%
%EndExpansion

\section{Back to the Earth}

So now we can see that the Earth is a magnet and we know how magnets are
formed. But wait, why is the Earth a magnet? The real answer is that we
don't know. But we believe that again it is because of current loops. We
believe there is a current of ionized Nickel and Iron in near the center of
the Earth. So the flow of these charged liquid metals will create a magnetic
field. This is a vary large current loop! The evidence for this is that
magnetic field seems proportional to the spin rate of the planet. But this
is an area of active research.

It is curious that the magnetic pole and the geographic pole are not in the
same place. The magnetic pole also moves around like a precession. Then,
every couple of hundred thousand years, the polarity of the Earth's field
switches altogether!

There is still plenty of good research to do in this area.

The location of the magnetic pole explains the declination adjustment you
have to use when using a compass. What you are really doing is accounting
for the difference in pole location.

\section{Induced currents}

We spend most of the last two lectures building a relationship between
moving charge (current) and magnetic fields.%
%TCIMACRO{%
%\TeXButton{Question 223.42.5}{\marginpar {
%\hspace{-0.5in}
%\begin{minipage}[t]{1in}
%\small{Question 223.42.5}
%\end{minipage}
%}} }%
%BeginExpansion
\marginpar {
\hspace{-0.5in}
\begin{minipage}[t]{1in}
\small{Question 223.42.5}
\end{minipage}
}
%EndExpansion
But suppose we have moving magnetic fields. Could a moving magnetic field
make a current?

If we think of relative motion, it seems like it should. After all, how do
we know that it is the charge that is moving and not a moving $B$-field. In
fact, moving $B$-fields \emph{do} cause a current. We say that a moving or
changing magnetic field \emph{induces} a current.

Faraday discovered this effect. He described it as an \emph{induced emf}. An
emf is something that \textquotedblleft pumps\textquotedblright\ the charges
in the wire. It takes them from a lower to a higher potential so they can
form a current. The changing magnetic field must be \textquotedblleft
pumping\textquotedblright\ the charges as it changes!

What is really going on here? Think for a minute what must be happening.%
\FRAME{dhF}{1.8663in}{2.3419in}{0pt}{}{}{Figure}{\special{language
"Scientific Word";type "GRAPHIC";maintain-aspect-ratio TRUE;display
"USEDEF";valid_file "T";width 1.8663in;height 2.3419in;depth
0pt;original-width 4.1156in;original-height 5.1759in;cropleft "0";croptop
"1";cropright "1";cropbottom "0";tempfilename
'LV1B050A.wmf';tempfile-properties "XPR";}}When we defined the electric
potential, we use a capacitor. We found that there was a field directed from
the $+$ charges to the $-$ charges. And in this field, charges had an amount
of potential energy. When a current flows from the $+$ end of the battery to
the $-$ end. there must be an electric field acting on the charge in the
wire! That is what creates the electric potential. So, then, does a moving
magnetic field create an electric field?

The answer is yes! We say that an electric field is \emph{induced} by a
moving magnetic field. This is really the same as saying that there is an
induced emf for our current loop.

\FRAME{dhF}{4.0949in}{1.6786in}{0pt}{}{}{Figure}{\special{language
"Scientific Word";type "GRAPHIC";maintain-aspect-ratio TRUE;display
"USEDEF";valid_file "T";width 4.0949in;height 1.6786in;depth
0pt;original-width 6.0615in;original-height 2.4708in;cropleft "0";croptop
"1";cropright "1";cropbottom "0";tempfilename
'LV2TIH04.wmf';tempfile-properties "XPR";}}Faraday actually set up his
experiment with two coils of wire. One coil was connected to a battery. We
now know this coil will make a magnetic field. As the current starts flowing
the field will form. While it is forming, it will induce an emf in the
second coil. But this is just using an electromagnet instead of a permanent
magnet.

To be able to calculate how much current flows, we will need to investigate
changing magnetic fields. We will do this next lecture with our concept of
flux.

%TCIMACRO{%
%\TeXButton{Basic Equations}{\hspace{-1.3in}{\LARGE Basic Equations\vspace{0.25in}}}}%
%BeginExpansion
\hspace{-1.3in}{\LARGE Basic Equations\vspace{0.25in}}%
%EndExpansion

\chapter{Induction}

%TCIMACRO{%
%\TeXButton{Fundamental Concepts}{\hspace{-1.3in}{\LARGE Fundamental Concepts\vspace{0.25in}}}}%
%BeginExpansion
\hspace{-1.3in}{\LARGE Fundamental Concepts\vspace{0.25in}}%
%EndExpansion

\begin{itemize}
\item Conductors moving in magnetic fields separate charge. creating a
potential difference that we call \textquotedblleft motional
emf.\textquotedblright

\item Motional emfs generate currents, even in solid pieces of conductor.
These currents in conductors are called \textquotedblleft eddy
currents.\textquotedblright

\item Magnetic flux is found by integrating the dot product of the magnetic
field and a differential element of area over the area. $\Phi _{B}=\int_{A}%
\overrightarrow{B}\cdot d\overrightarrow{A}$
\end{itemize}

\section{Motional emf}

Last lecture, we studied Faraday's experiment. He created a magnetic field,
and then used that magnetic field to make a current. But currents are caused
by electric fields! Did Faraday's magnetic field create an electric field?

To investigate Faraday's result, let's see if we can find a way to use
charge motion and a magnetic field to make an electric field. Let's take a
bar of metal and move it in a magnetic field. The bar has free charges in it
(electrons). We have given them a velocity. So we expect a magnetic force 
\begin{equation*}
\overrightarrow{\mathbf{F}}_{B}=q\overrightarrow{\mathbf{v}}\times 
\overrightarrow{\mathbf{B}}
\end{equation*}%
The free charges will accelerate together, but the positive stationary
charges can't move. We have found another way to separate charge. We know
that separated charge creates a potential difference. We often call this
induced potential difference the \emph{motional emf }because it is created
by moving our apparatus.

Let's take an example to see how it works. \FRAME{dhF}{2.9776in}{2.3999in}{%
0pt}{}{}{Figure}{\special{language "Scientific Word";type
"GRAPHIC";maintain-aspect-ratio TRUE;display "USEDEF";valid_file "T";width
2.9776in;height 2.3999in;depth 0pt;original-width 2.9334in;original-height
2.3592in;cropleft "0";croptop "1";cropright "1";cropbottom "0";tempfilename
'LVA5EM00.wmf';tempfile-properties "XPR";}}For this example, let's look at a
piece of wire moving in a constant field. To make the math easy, let's move
the wire with a velocity perpendicular to the $B$-field.

As the figure shows, the electrons will feel a force. Using our right hand
rule, we get an upward force for positive charge carriers, but we know the
electrons are negative charge carriers, so the force is downward. We find
that the magnitude of the force is 
\begin{equation*}
F_{B}=qvB
\end{equation*}

The electrons will bunch up at the bottom of the piece of wire, until their
electric force of repulsion forces them to stop. That force is 
\begin{equation*}
F_{E}=qE
\end{equation*}%
By separating the charges along the wire so that there is excess positive
charge on one end and excess negative charge on the other end, we now have
and $E$-field in the wire. We can solve for $E$ when we have reached
equilibrium.%
\begin{equation*}
\Sigma F=0=-F_{B}+F_{E}
\end{equation*}%
or%
\begin{equation*}
qE=qvB
\end{equation*}%
which tells us%
\begin{equation}
E=vB
\end{equation}%
Now, we know that electric fields cause potential differences. The $E$-field
in the wire will be nearly uniform. Then it looks much like a capacitor with
separated charges. The potential difference will be 
\begin{eqnarray*}
\Delta V &=&\int \overrightarrow{E}\cdot d\overrightarrow{s} \\
&\approx &EL
\end{eqnarray*}%
where $L$ is the length of our wire. So 
\begin{equation}
\Delta V\approx vBL
\end{equation}

This is like a battery. The magnetic field is \textquotedblleft
pumping\textquotedblright\ charge. If we connected the two ends somehow with
a wire that is not moving, a current will flow (that is tricky to actually
do!).%
%TCIMACRO{%
%\TeXButton{Question 223.43.0.1}{\marginpar {
%\hspace{-0.5in}
%\begin{minipage}[t]{1in}
%\small{Question 223.43.0.1}
%\end{minipage}
%}}}%
%BeginExpansion
\marginpar {
\hspace{-0.5in}
\begin{minipage}[t]{1in}
\small{Question 223.43.0.1}
\end{minipage}
}%
%EndExpansion

\FRAME{dhF}{3.7775in}{1.708in}{0pt}{}{}{Figure}{\special{language
"Scientific Word";type "GRAPHIC";maintain-aspect-ratio TRUE;display
"USEDEF";valid_file "T";width 3.7775in;height 1.708in;depth
0pt;original-width 6.4792in;original-height 2.9153in;cropleft "0";croptop
"1";cropright "1";cropbottom "0";tempfilename
'LVA6FI01.wmf';tempfile-properties "XPR";}}

Let's take another example. We wish to make a bar of metal move in a $B$%
-field. To make the rest of the circuit, we allow the bar to slide along two
wires as shown. We will call the two wires \textquotedblleft
rails\textquotedblright\ since they look a little like railroad rails. Then
we have a connection between our moving piece of metal, and the rest of the
circuit. What we have is very like the circuit on the right hand side of the
last figure.

We will have to apply a force $F_{pull}$ to move the bar. This is because
there is another force, marked as $F_{resistive}$ in the figure. This force
is one we know, but might not recognize unless we think about it. We now
have a current flowing through a wire, and the wire is in a magnetic field.
So there will be a force%
\begin{eqnarray*}
F_{resistive} &=&I\overrightarrow{L}\times \overrightarrow{B} \\
&=&ILB\sin \theta \\
&=&ILB
\end{eqnarray*}%
pushing to the left. This force resists our pull.

From Ohm's law, the current in the wire will be 
\begin{eqnarray*}
I &=&\frac{\Delta V}{R} \\
&=&\frac{vBL}{R}
\end{eqnarray*}%
so the force is 
\begin{eqnarray*}
F_{resistive} &=&\left( \frac{vBL}{R}\right) LB \\
&=&\frac{vB^{2}L^{2}}{R}
\end{eqnarray*}%
Thus we have to push with an equal force 
\begin{equation*}
F_{push}=F_{resistive}
\end{equation*}%
to keep the bar moving along the rails. If $F_{push}<F_{resistive}$ then the
bar will have an acceleration, and it will be in the opposite direction from
the velocity, so the bar will slow down.

\section{Eddy Currents}

%TCIMACRO{%
%\TeXButton{Question 223.43.1}{\marginpar {
%\hspace{-0.5in}
%\begin{minipage}[t]{1in}
%\small{Question 223.43.1}
%\end{minipage}
%}}}%
%BeginExpansion
\marginpar {
\hspace{-0.5in}
\begin{minipage}[t]{1in}
\small{Question 223.43.1}
\end{minipage}
}%
%EndExpansion
%TCIMACRO{%
%\TeXButton{Pendulum-loop}{\marginpar {
%\hspace{-0.5in}
%\begin{minipage}[t]{1in}
%\small{Pendulum-loop}
%\end{minipage}
%}}}%
%BeginExpansion
\marginpar {
\hspace{-0.5in}
\begin{minipage}[t]{1in}
\small{Pendulum-loop}
\end{minipage}
}%
%EndExpansion
So if we have a conductive loop and part of that loop moves in a magnetic
field, we get a current. I chose to make our apparatus a pendulum. \FRAME{%
dtbpF}{1.1185in}{1.3979in}{0pt}{}{}{Figure}{\special{language "Scientific
Word";type "GRAPHIC";maintain-aspect-ratio TRUE;display "USEDEF";valid_file
"T";width 1.1185in;height 1.3979in;depth 0pt;original-width
2.1554in;original-height 2.7026in;cropleft "0";croptop "1";cropright
"1";cropbottom "0";tempfilename 'MPMG9001.wmf';tempfile-properties "XPR";}}%
So as the pendulum swings, through the magnetic field, we get a current.
What if we have a solid sheet of conductor and we move that sheet through
the magnetic field, will there be a current?%
%TCIMACRO{%
%\TeXButton{Pendulum-plate}{\marginpar {
%\hspace{-0.5in}
%\begin{minipage}[t]{1in}
%\small{Pendulum-plate}
%\end{minipage}
%}}}%
%BeginExpansion
\marginpar {
\hspace{-0.5in}
\begin{minipage}[t]{1in}
\small{Pendulum-plate}
\end{minipage}
}%
%EndExpansion
%TCIMACRO{%
%\TeXButton{Question 223.43.2}{\marginpar {
%\hspace{-0.5in}
%\begin{minipage}[t]{1in}
%\small{Question 223.43.2}
%\end{minipage}
%}} }%
%BeginExpansion
\marginpar {
\hspace{-0.5in}
\begin{minipage}[t]{1in}
\small{Question 223.43.2}
\end{minipage}
}
%EndExpansion
\FRAME{dtbpF}{1.5336in}{2.0347in}{0pt}{}{}{Figure}{\special{language
"Scientific Word";type "GRAPHIC";maintain-aspect-ratio TRUE;display
"USEDEF";valid_file "T";width 1.5336in;height 2.0347in;depth
0pt;original-width 1.5345in;original-height 2.0445in;cropleft "0";croptop
"1";cropright "1";cropbottom "0";tempfilename
'MPMGHU04.wmf';tempfile-properties "XPR";}}The answer is yes. We call this
current an \emph{eddy current}. Let's see that this must be true with
another experiment.%
%TCIMACRO{%
%\TeXButton{Question 223.43.3}{\marginpar {
%\hspace{-0.5in}
%\begin{minipage}[t]{1in}
%\small{Question 223.43.3}
%\end{minipage}
%}}}%
%BeginExpansion
\marginpar {
\hspace{-0.5in}
\begin{minipage}[t]{1in}
\small{Question 223.43.3}
\end{minipage}
}%
%EndExpansion
Let's cut groves in the plate. \FRAME{dtbpF}{1.6658in}{2.15in}{0in}{}{}{%
Figure}{\special{language "Scientific Word";type
"GRAPHIC";maintain-aspect-ratio TRUE;display "USEDEF";valid_file "T";width
1.6658in;height 2.15in;depth 0in;original-width 1.6684in;original-height
2.1625in;cropleft "0";croptop "1";cropright "1";cropbottom "0";tempfilename
'MPMGC703.wmf';tempfile-properties "XPR";}}The current is broken by the
grooves, so there is little opposing magnetic field%
%TCIMACRO{%
%\TeXButton{Al plate and strong magnets}{\marginpar {
%\hspace{-0.5in}
%\begin{minipage}[t]{1in}
%\small{Al plate and strong magnets}
%\end{minipage}
%}}}%
%BeginExpansion
\marginpar {
\hspace{-0.5in}
\begin{minipage}[t]{1in}
\small{Al plate and strong magnets}
\end{minipage}
}%
%EndExpansion
This effect due to the eddy currents is often used to slow down machines.
Rotating blades, and even trains use this effect to provide breaking.%
%TCIMACRO{%
%\TeXButton{Floating Plate Demo}{\marginpar {
%\hspace{-0.5in}
%\begin{minipage}[t]{1in}
%\small{Floating Plate Demo}
%\end{minipage}
%}}}%
%BeginExpansion
\marginpar {
\hspace{-0.5in}
\begin{minipage}[t]{1in}
\small{Floating Plate Demo}
\end{minipage}
}%
%EndExpansion

\section{Magnetic flux}

Remember long ago we defined the electric flux. \FRAME{dhF}{1.8862in}{%
1.8282in}{0pt}{}{}{Figure}{\special{language "Scientific Word";type
"GRAPHIC";maintain-aspect-ratio TRUE;display "USEDEF";valid_file "T";width
1.8862in;height 1.8282in;depth 0pt;original-width 2.8496in;original-height
2.7622in;cropleft "0";croptop "1";cropright "1";cropbottom "0";tempfilename
'LVA8Q30D.wmf';tempfile-properties "XPR";}}Recall that the electric flux is
given by 
\begin{eqnarray*}
\Phi _{E} &=&\overrightarrow{E}\cdot \overrightarrow{A} \\
&=&EA\cos \theta
\end{eqnarray*}%
But we now have a magnetic field. \FRAME{dhF}{2.3091in}{2.2355in}{0pt}{}{}{%
Figure}{\special{language "Scientific Word";type
"GRAPHIC";maintain-aspect-ratio TRUE;display "USEDEF";valid_file "T";width
2.3091in;height 2.2355in;depth 0pt;original-width 2.7389in;original-height
2.6507in;cropleft "0";croptop "1";cropright "1";cropbottom "0";tempfilename
'LVA8Q30E.wmf';tempfile-properties "XPR";}}We define a magnetic flux 
\begin{equation}
\Phi _{B}=\overrightarrow{B}\cdot \overrightarrow{A}
\end{equation}

\FRAME{dhF}{2.1767in}{2.5728in}{0pt}{}{}{Figure}{\special{language
"Scientific Word";type "GRAPHIC";maintain-aspect-ratio TRUE;display
"USEDEF";valid_file "T";width 2.1767in;height 2.5728in;depth
0pt;original-width 2.7665in;original-height 3.275in;cropleft "0";croptop
"1";cropright "1";cropbottom "0";tempfilename
'LVA8Q30B.wmf';tempfile-properties "XPR";}}%
\begin{equation}
\Phi _{B}=BA\cos \theta
\end{equation}

where $\theta $ is the angle between $\overrightarrow{B}$ and $%
\overrightarrow{A}.$ \FRAME{dhF}{2.0159in}{2.1897in}{0pt}{}{}{Figure}{%
\special{language "Scientific Word";type "GRAPHIC";maintain-aspect-ratio
TRUE;display "USEDEF";valid_file "T";width 2.0159in;height 2.1897in;depth
0pt;original-width 2.5028in;original-height 2.7207in;cropleft "0";croptop
"1";cropright "1";cropbottom "0";tempfilename
'LVFGPR00.wmf';tempfile-properties "XPR";}}

We found that the electric flux was very useful. We used Gauss' law to find
fields using the idea of electric flux. It turns out that this magnetic flux
is also a very useful idea. There is a difference, though. With electric
fluxes, we had imaginary areas that the field penetrated. Often when we
measure magnetic flux, we actually have something at the location of our
area. We generally want to know the flux through a wire loop.

Just like with electric flux, we expect the flux to be proportional to the
number of field lines that pass through the area.

\subsection{Non uniform magnetic fields}

So far in this lecture we have only drawn uniform magnetic fields and
considered their flux. But we can easily imagine a non-uniform field. We
tackled non-uniform electric field fluxes. We should take on non-uniform
magnetic field fluxes as well. Suppose we have the situation shown in the
following figure.\FRAME{dtbpF}{2.6125in}{2.0979in}{0in}{}{}{Figure}{\special%
{language "Scientific Word";type "GRAPHIC";maintain-aspect-ratio
TRUE;display "USEDEF";valid_file "T";width 2.6125in;height 2.0979in;depth
0in;original-width 2.549in;original-height 2.0412in;cropleft "0";croptop
"1";cropright "1";cropbottom "0";tempfilename
'S4SE3K00.wmf';tempfile-properties "XPR";}}We have a loop of wire, and the
loop is in a flux that changes from left to right.

To find the flux through such a loop of wire, we can envision a small
element of area, $d\overrightarrow{A}$ as shown. The flux through this area
element is 
\begin{equation*}
d\Phi _{B}=\overrightarrow{B}\cdot d\overrightarrow{A}
\end{equation*}%
We can integrate this to find the total flux%
\begin{equation}
\Phi _{B}=\int_{A}\overrightarrow{B}\cdot d\overrightarrow{A}
\end{equation}

But what could make such a varying $B$-field? Consider a long straight wire
again.\FRAME{dhF}{4.1234in}{3.237in}{0pt}{}{}{Figure}{\special{language
"Scientific Word";type "GRAPHIC";maintain-aspect-ratio TRUE;display
"USEDEF";valid_file "T";width 4.1234in;height 3.237in;depth
0pt;original-width 4.0741in;original-height 3.192in;cropleft "0";croptop
"1";cropright "1";cropbottom "0";tempfilename
'LVEDQH00.wmf';tempfile-properties "XPR";}}We know that the field due to the
current-carrying wire will be 
\begin{equation*}
B=\frac{\mu _{o}I}{2\pi r}
\end{equation*}%
where $r$ is the distance from the wire and the direction is given by one of
our right hand rules. \FRAME{dhFX}{2.8141in}{2.0755in}{0pt}{}{}{Figure}{%
\special{language "Scientific Word";type "GRAPHIC";maintain-aspect-ratio
TRUE;display "PICT";valid_file "T";width 2.8141in;height 2.0755in;depth
0pt;original-width 8.9672in;original-height 6.6072in;cropleft "0";croptop
"1";cropright "1";cropbottom "0";tempfilename
'LVADQK08.wmf';tempfile-properties "XPR";}}%
%TCIMACRO{%
%\TeXButton{Question 223.43.4}{\marginpar {
%\hspace{-0.5in}
%\begin{minipage}[t]{1in}
%\small{Question 223.43.4}
%\end{minipage}
%}}}%
%BeginExpansion
\marginpar {
\hspace{-0.5in}
\begin{minipage}[t]{1in}
\small{Question 223.43.4}
\end{minipage}
}%
%EndExpansion
%TCIMACRO{%
%\TeXButton{Question 223.43.5}{\marginpar {
%\hspace{-0.5in}
%\begin{minipage}[t]{1in}
%\small{Question 223.43.5}
%\end{minipage}
%}}}%
%BeginExpansion
\marginpar {
\hspace{-0.5in}
\begin{minipage}[t]{1in}
\small{Question 223.43.5}
\end{minipage}
}%
%EndExpansion
The flux through the green rectangular area is almost constant. The little
area is given by%
\begin{equation*}
dA=dydr
\end{equation*}%
The area is perpendicular to the field, so the angle between $B$ and $A$ is $%
90\unit{%
%TCIMACRO{\U{b0}}%
%BeginExpansion
{{}^\circ}%
%EndExpansion
}.$ Then 
\begin{equation*}
d\Phi _{B}=\frac{\mu _{o}I}{2\pi r}dydr\left( 1\right)
\end{equation*}%
and we can integrate this to find the total flux%
\begin{eqnarray*}
\Phi _{B} &=&\int_{r_{o}}^{r=a}\int_{o}^{b}\frac{\mu _{o}I}{2\pi r}dydr \\
&=&\frac{\mu _{o}I}{2\pi }\int_{r_{o}}^{r=a}\frac{1}{r}dr\int_{o}^{b}dy \\
&=&\frac{\mu _{o}Ib}{2\pi }\int_{r_{o}}^{r_{o}+a}\frac{1}{r}dr \\
&=&\frac{\mu _{o}Ib}{2\pi }\left( \ln \left( r_{o}+a\right) -\ln \left(
r_{o}\right) \right) \\
&=&\frac{\mu _{o}Ib}{2\pi }\ln \left( \frac{r_{o}+a}{r_{o}}\right)
\end{eqnarray*}

We can even put in some numbers for this case. Suppose our loop has a height
of $b=0.05\unit{m}$ and a width of $a=0.01\unit{m}$ and that it is a
distance $r_{o}=a$ away from the current carrying wire and that the current
is $I=0.5\unit{A}$. Then 
\begin{eqnarray*}
\Phi _{B} &=&\frac{\left( 4\pi \times 10^{-7}\frac{\unit{T}\unit{m}}{\unit{A}%
}\right) \left( 0.5\unit{A}\right) \left( 0.05\unit{m}\right) }{2\pi }\ln
\left( \frac{0.01\unit{m}+0.01\unit{m}}{0.01\unit{m}}\right) \\
&=&3.\,\allowbreak 465\,7\times 10^{-9}\unit{Wb}
\end{eqnarray*}%
the unit of magnetic flux is called the weber and it is given by : 
\begin{equation*}
\unit{Wb}=\unit{T}\unit{m}^{2}=\frac{\unit{m}^{2}}{\unit{A}}\frac{\unit{kg}}{%
\unit{s}^{2}}
\end{equation*}

We know now how to calculate magnetic flux, but you should expect that we
can do something with this flux to simplify problems. And your expectation
would be right. We used electric flux in Gauss' law. We will use magnetic
flux to find the induced emf. An induced emf can create a current, and this
is the basic idea behind a generator. The law that governs this relationship
between induced emf and magnetic flux is called \emph{Faraday's law} after
the scientist that discovered it. We will study this law in our next lecture.

%TCIMACRO{%
%\TeXButton{Basic Equations}{\hspace{-1.3in}{\LARGE Basic Equations\vspace{0.25in}}}}%
%BeginExpansion
\hspace{-1.3in}{\LARGE Basic Equations\vspace{0.25in}}%
%EndExpansion

\chapter{Faraday and Lenz}

%TCIMACRO{%
%\TeXButton{Fundamental Concepts}{\hspace{-1.3in}{\LARGE Fundamental Concepts\vspace{0.25in}}}}%
%BeginExpansion
\hspace{-1.3in}{\LARGE Fundamental Concepts\vspace{0.25in}}%
%EndExpansion

We talked about an induced electric field created by a magnetic field last
lecture. We want to formalize that relationship in this lecture.%
%TCIMACRO{%
%\TeXButton{Question 223.44.1}{\marginpar {
%\hspace{-0.5in}
%\begin{minipage}[t]{1in}
%\small{Question 223.44.1}
%\end{minipage}
%}}}%
%BeginExpansion
\marginpar {
\hspace{-0.5in}
\begin{minipage}[t]{1in}
\small{Question 223.44.1}
\end{minipage}
}%
%EndExpansion
%TCIMACRO{%
%\TeXButton{Question 223.44.2}{\marginpar {
%\hspace{-0.5in}
%\begin{minipage}[t]{1in}
%\small{Question 223.44.2}
%\end{minipage}
%}}}%
%BeginExpansion
\marginpar {
\hspace{-0.5in}
\begin{minipage}[t]{1in}
\small{Question 223.44.2}
\end{minipage}
}%
%EndExpansion
%TCIMACRO{%
%\TeXButton{Question 223.44.3}{\marginpar {
%\hspace{-0.5in}
%\begin{minipage}[t]{1in}
%\small{Question 223.44.3}
%\end{minipage}
%}} }%
%BeginExpansion
\marginpar {
\hspace{-0.5in}
\begin{minipage}[t]{1in}
\small{Question 223.44.3}
\end{minipage}
}
%EndExpansion
Let's go back to our motional emf problem. \FRAME{dhF}{1.8092in}{1.7781in}{%
0pt}{}{}{Figure}{\special{language "Scientific Word";type
"GRAPHIC";maintain-aspect-ratio TRUE;display "USEDEF";valid_file "T";width
1.8092in;height 1.7781in;depth 0pt;original-width 2.6974in;original-height
2.6507in;cropleft "0";croptop "1";cropright "1";cropbottom "0";tempfilename
'M1985104.wmf';tempfile-properties "XPR";}}We have a sliding bar, and a
u-shaped conductor and a magnetic field. The moving bar makes the current
flow. But now we know another way to express this. We can see that there is
a magnetic flux through the loop consisting of the u-shaped conductor and
the sliding bar. This flux going through the loop is changing. The area is
getting larger, so the amount of field going through the loop is increasing.
We can say the induced current is due to the changing loop area in the
presence of the magnetic field, or a changing magnetic flux.

An important thing we learned is that the moving bar feels a resistive force
due to the current and magnetic field. It seems like the magnetic field and
current are resisting any change in our set up. We will see in this lecture
that this is true in general.

It turns out that there is more than one way to cause an induced current.
Any change in the magnetic flux is found to make a current flow. Remember in
class we found that putting a magnet into or pulling the magnet out of a
coil makes a current. In this case, the strength of the magnetic field
changes, so the flux changes. Really any change in magnetic flux makes a
current flow.

\section{Fundamental Concepts in the Lecture}

\begin{itemize}
\item Changing magnetic flux makes an electric field--which has an
associated potential difference or emf.

\item The current caused by the induced emf travels in the direction that
creates a magnetic field with flux opposing the change in the original flux
through the circuit.

\item The emf (potential difference) generated by a changing magnetic field
is given by $\mathcal{E}=-N\frac{\Delta \Phi _{B}}{\Delta t}$
\end{itemize}

\section{Lenz}

What we are saying is that if we change the magnetic flux through a loop, we
will get a current. The direction of current flow is not obvious. Lenz
experimentally determined which way it will go. Here is his rule

\begin{Note}
The current caused by the induced emf travels in the direction that creates
a magnetic field with flux opposing the change in the original flux through
the circuit.
\end{Note}

This takes a moment to digest. Let's take an example\FRAME{dhF}{3.1592in}{%
2.93in}{0pt}{}{}{Figure}{\special{language "Scientific Word";type
"GRAPHIC";maintain-aspect-ratio TRUE;display "USEDEF";valid_file "T";width
3.1592in;height 2.93in;depth 0pt;original-width 3.1142in;original-height
2.8859in;cropleft "0";croptop "1";cropright "1";cropbottom "0";tempfilename
'LVEIIK03.wmf';tempfile-properties "XPR";}}Consider the case shown in the
picture. Suppose the $B$-field gets smaller in time. If that is the case,
then the induced current will try to keep the same number of field lines
going through the loop. To do this, it will have to add field lines, because
our field that is getting smaller will have fewer and fewer field lines. So
in this case, the induced field $\mathbf{\vec{B}}_{ind}$ will be in the same
direction as $\mathbf{\vec{B}}$ to try to keep the number of field lines the
same. We find the current using our current-carrying wire right hand rule\
for magnetism. We imagine grabbing the wire such that our fingers curl into
the loop the way $\mathbf{\vec{B}}_{ind}$ goes through the loop. Then our
thumb is in the direction of the current. 
%TCIMACRO{%
%\TeXButton{Question 223.44.4 - 223.44.11}{\marginpar {
%\hspace{-0.5in}
%\begin{minipage}[t]{1in}
%\small{Question 223.44.4 - 223.44.11}
%\end{minipage}
%}}}%
%BeginExpansion
\marginpar {
\hspace{-0.5in}
\begin{minipage}[t]{1in}
\small{Question 223.44.4 - 223.44.11}
\end{minipage}
}%
%EndExpansion

\section{Faraday}

In our motional emf problem, the sliding bar in the magnetic field creates a
potential difference, $\Delta V.$ It becomes an emf. We can use the symbol $%
\mathcal{E}$ for our emf.

But then in considering Lenz's law, it was experimentally found that any
change in flux causes a current. Then any change in flux must create an emf.

\FRAME{dhF}{1.9268in}{1.8939in}{0pt}{}{}{Figure}{\special{language
"Scientific Word";type "GRAPHIC";maintain-aspect-ratio TRUE;display
"USEDEF";valid_file "T";width 1.9268in;height 1.8939in;depth
0pt;original-width 2.6974in;original-height 2.6507in;cropleft "0";croptop
"1";cropright "1";cropbottom "0";tempfilename
'M1984X03.wmf';tempfile-properties "XPR";}}In this case the area is getting
larger, and so the flux is getting larger. The induced current will oppose
the change. So the induced magnetic field should go up through the center of
the loop. Imagine sticking your fingers through the loop out of the page,
then grabbing the loop (fingers still out of the page in the inside of the
loop). Anywhere you grab the wire, your thumb is in the induced current
direction.

\subsection{Faraday's law of Magnetic Induction}

Faraday wrote an equation to describe the emf that was given by changing a $%
B $-field. It combines what we know about magnetic flux and current from
Lenz's law. Faraday did not know the source of the emf, it is a purely
empirical equation. Here it is

\begin{Note}
\begin{equation}
\mathcal{E}=-N\frac{\Delta \Phi _{B}}{\Delta t}
\end{equation}
\end{Note}

The $N$ is the number of turns in the coil (remember he used a coil, not
just one loop). $d\Phi _{B}$ is the change in the magnetic flux. Our
definition of magnetic flux is 
\begin{equation*}
\Phi _{B}=\int \overrightarrow{\mathbf{B}}\cdot d\overrightarrow{\mathbf{A}}
\end{equation*}%
but for simple open surfaces we can gain some insight by writing the flux as 
\begin{equation*}
\Phi _{B}=BA\cos \theta
\end{equation*}%
Then the induced emf would be given by%
\begin{eqnarray}
\mathcal{E} &=&-N\frac{\Delta \Phi }{\Delta t} \\
&=&-N\frac{\left( B_{2}A_{2}\cos \theta _{2}-B_{1}A_{1}\cos \theta
_{1}\right) }{\Delta t}
\end{eqnarray}%
and we see that we get an emf if $B,$ $A,$ or $\theta $ change. We can write
this as a differential if we let $\Delta t$ get very small.%
\begin{equation}
\mathcal{E}=-N\frac{d\Phi _{B}}{dt}
\end{equation}

Suppose we have a simple flux $\Phi _{B}=\overrightarrow{\mathbf{B}}\cdot 
\overrightarrow{\mathbf{A}},$ then for this simple case 
\begin{eqnarray*}
\mathcal{E} &=&-N\frac{d}{dt}\left( \overrightarrow{\mathbf{B}}\cdot 
\overrightarrow{\mathbf{A}}\right) \\
&=&-N\left( \overrightarrow{\mathbf{B}}\cdot \frac{d}{dt}\overrightarrow{%
\mathbf{A}}+\overrightarrow{\mathbf{A}}\cdot \frac{d}{dt}\overrightarrow{%
\mathbf{B}}\right)
\end{eqnarray*}

The first term shows our motional emf case. The area is changing in time.
But the second term shows that if the field changes, we get an emf. This is
the moving magnet in the coil case.

There are some great applications of induced emfs, from another design for
circuit breakers to electric guitar pickups! 
%TCIMACRO{%
%\TeXButton{Question 223.44.12 - Question 223.44.17}{\marginpar {
%\hspace{-0.5in}
%\begin{minipage}[t]{1in}
%\small{Question 223.44.12 - Question 223.44.17}
%\end{minipage}
%}}}%
%BeginExpansion
\marginpar {
\hspace{-0.5in}
\begin{minipage}[t]{1in}
\small{Question 223.44.12 - Question 223.44.17}
\end{minipage}
}%
%EndExpansion

\section{Return to Lenz's law}

Remember that Lenz's law says the current caused by the induced emf travels
in the direction that creates a magnetic field with flux opposing the change
in the original flux through the circuit. What if the current went the other
way?

If that happened, then we could set up our bar on the rails, and give it a
push to the right. With the current going down instead of up (for positive
charge carriers) then we would have a force on our bar-like segment of wire%
\begin{equation*}
F_{I}=BIL\sin \phi
\end{equation*}%
here $\sin \phi =1$ so 
\begin{equation*}
F_{I}=BIL
\end{equation*}%
It will be directed to the right. So the bar would accelerate to the right.
That would increase the size of the loop, increasing the current. That would
increase the force to the right, and our bar would soon zip off at amazing
speed. But that does not happen. It would take ever more energy to make the
bar go faster, with no input energy. So this would violate conservation of
energy. Really Lenz's law just gives us conservation of energy again.

\section{Pulling a loop from a magnetic field.}

Let's try a problem. Suppose we have a wire loop. The loop is rectangular,
with side lengths $\ell $ and $x.$ Further suppose that the loop is in a
region with magnetic field, but that it is on the edge of that field, so
that if we pull it to the right, it will leave the field.\FRAME{dhF}{2.5019in%
}{2.3713in}{0pt}{}{}{Figure}{\special{language "Scientific Word";type
"GRAPHIC";maintain-aspect-ratio TRUE;display "USEDEF";valid_file "T";width
2.5019in;height 2.3713in;depth 0pt;original-width 2.4613in;original-height
2.3315in;cropleft "0";croptop "1";cropright "1";cropbottom "0";tempfilename
'M195TG01.wmf';tempfile-properties "XPR";}}let's see if we can find the
induced emf and current.

The Magnetic flux through the loop is changing. We can find an expression
for the flux%
\begin{equation*}
\Phi _{B}=\overrightarrow{\mathbf{B}}\cdot \overrightarrow{\mathbf{A}}
\end{equation*}%
or in this case 
\begin{equation*}
\Phi _{B}=B\ell x
\end{equation*}%
We know the emf from Faraday's law 
\begin{equation*}
\mathcal{E}=-N\frac{d\Phi _{B}}{dt}
\end{equation*}%
then 
\begin{equation*}
\mathcal{E}=-\left( 1\right) \frac{d}{dt}\left( B\ell x\right)
\end{equation*}%
The field is not changing strength, and the length $\ell $ is not changing.
But along the $x$ side, we are losing field. Remember that $A$ in our flux
equation is the area that actually has field and we have less area that has
field all the time. We can see that 
\begin{equation*}
\mathcal{E}=-\left( 1\right) \frac{d}{dt}\left( B\ell x\right) =-B\ell \frac{%
dx}{dt}=-B\ell v
\end{equation*}%
where $v$ is the speed at which we are pulling the wire loop. That is the
speed at which our flux changes.

We can use Ohm's law to find the current, 
\begin{equation*}
I=\frac{\Delta V}{R}=\frac{\mathcal{E}}{R}
\end{equation*}%
or%
\begin{equation*}
I=\frac{B\ell v}{R}
\end{equation*}

We could ask, how much work does it take to pull the wire out of the field?
This is like our capacitor problem where we pulled a dielectric out of the
middle of the capacitor.

The net force on the loop is not zero, because the field is no longer
uniform. The right hand side of the loop is outside the field, and the left
hand side is not. Of course, the top and bottom of the loop have opposite
forces that balance each other. So the net force is due to the left hand
side of the loop. Recall that%
\begin{equation*}
\overrightarrow{\mathbf{F}}=I\overrightarrow{\mathbf{L}}\times 
\overrightarrow{\mathbf{B}}
\end{equation*}%
We can see that in this case $I\ $is upward, and $B$ is into the page. So
there is a force to the left resisting our change flux. We must pull to
overcome this force. The magnitude of this force is 
\begin{equation*}
F=I\ell B
\end{equation*}%
and we know $I$ so 
\begin{equation*}
F=\frac{B\ell v}{R}\ell B=\frac{B^{2}\ell ^{2}v}{R}
\end{equation*}

Now we need to find the work done. 
\begin{equation*}
W=\int Fdx
\end{equation*}%
or, since our force will be constant until the loop leaves the magnetic
field entirely,%
\begin{equation*}
W=F\int dx
\end{equation*}%
which is not a hard integral to do. But instead of performing the integral,
let's look at the integrand. 
\begin{equation*}
dW=Fdx
\end{equation*}%
if we divide both sides of our equation by $dt$ we have 
\begin{equation*}
\frac{dW}{dt}=F\frac{dx}{dt}
\end{equation*}%
we know that $P=dW/dt$ and $\frac{dx}{dt}=v$ and so we can write our
equation as%
\begin{eqnarray*}
P &=&Fv \\
&=&\frac{B^{2}\ell ^{2}v^{2}}{R}
\end{eqnarray*}%
which is how much power the magnetic field force provides in resisting. We
must provide and equal power to move the loop. It will take time 
\begin{equation*}
\Delta t=\frac{\Delta x}{v}
\end{equation*}%
to pull the loop a distance $\Delta x.$ If we define our coordinates such
that $x_{i}=0$ then to pull out the loop, we will write this time as 
\begin{equation*}
\Delta t=\frac{x}{v}
\end{equation*}%
so the work is 
\begin{eqnarray*}
W &=&P\Delta t \\
&=&\frac{B^{2}\ell ^{2}v^{2}}{R}\frac{x}{v} \\
&=&\frac{B^{2}\ell ^{2}xv}{R}
\end{eqnarray*}

Incidentally, we learned from our demonstrations that induced currents can
take energy out of a system, creating heat energy. From Ohm's law the power
lost due to resistive heating would be 
\begin{eqnarray*}
P &=&I^{2}R \\
&=&\left( \frac{B\ell v}{R}\right) ^{2}R \\
&=&\frac{B2\ell ^{2}v^{2}}{R}
\end{eqnarray*}%
which is just the power we had to provide to make our loop move. So our work
has moved the loop and heated up the wire.

We have created a current in a wire. This is the first step in building a
generator. It cost us work to do this. In the next lecture, we will tackle
more practical design and build generators and transformers. Then we will
pause to think philosophically about what it means that a changing magnetic
flux creates an electric field.

%TCIMACRO{%
%\TeXButton{Basic Equations}{\hspace{-1.3in}{\LARGE Basic Equations\vspace{0.25in}}}}%
%BeginExpansion
\hspace{-1.3in}{\LARGE Basic Equations\vspace{0.25in}}%
%EndExpansion

\chapter{Induced Fields}

%TCIMACRO{%
%\TeXButton{Fundamental Concepts}{\hspace{-1.3in}{\LARGE Fundamental Concepts\vspace{0.25in}}}}%
%BeginExpansion
\hspace{-1.3in}{\LARGE Fundamental Concepts\vspace{0.25in}}%
%EndExpansion

\begin{itemize}
\item Changing the commutator for slip rings makes a motor into a Generators

\item Using alternating current, we can build an inductive device that can
change from one voltage to another. This device is called a transformer.

\item A more general form of Faraday's law is $\int \mathbf{E}\cdot d\mathbf{%
s}=-\frac{d\Phi _{B}}{dt}$
\end{itemize}

\section{Generators}

%TCIMACRO{%
%\TeXButton{Question 223.45.1}{\marginpar {
%\hspace{-0.5in}
%\begin{minipage}[t]{1in}
%\small{Question 223.45.1}
%\end{minipage}
%}}}%
%BeginExpansion
\marginpar {
\hspace{-0.5in}
\begin{minipage}[t]{1in}
\small{Question 223.45.1}
\end{minipage}
}%
%EndExpansion
Whether you are just plugging in an appliance, or preparing for an
emergency, you likely would think of a generator as a source of electrical
energy. Our studies so far have strongly hinted on how we would build an
electric generator. In this lecture, we will fill in the details.\FRAME{dhF}{%
2.8651in}{2.3851in}{0in}{}{}{Figure}{\special{language "Scientific
Word";type "GRAPHIC";maintain-aspect-ratio TRUE;display "USEDEF";valid_file
"T";width 2.8651in;height 2.3851in;depth 0in;original-width
2.8219in;original-height 2.3454in;cropleft "0";croptop "1";cropright
"1";cropbottom "0";tempfilename 'M1CEUT02.wmf';tempfile-properties "XPR";}}

We can learn a lot by studying this device as an example. The figure shows
the important parts of the generator (and a light bulb, which is not an
important part of a generator, but just represents some device that will use
the electrical current we make). The generator has at least one magnet. In
the figure, there is one with a north end on the left and a south end on the
right. A generator also has a wire loop. Usually in real generators, there
are thousands of turns of wire forming the loop. In our picture, there is
just one. The wire loop is connected to two metal rings. The rings will turn
as the loop turns. Metal contacts (brushes) that can slip along the rings,
but maintain an electrical connection, are placed on the rings. So as the
rings turn, current can still flow through the connected wires (to the light
bulb in this case).

%TCIMACRO{%
%\TeXButton{Question 223.45.2}{\marginpar {
%\hspace{-0.5in}
%\begin{minipage}[t]{1in}
%\small{Question 223.45.2}
%\end{minipage}
%}}}%
%BeginExpansion
\marginpar {
\hspace{-0.5in}
\begin{minipage}[t]{1in}
\small{Question 223.45.2}
\end{minipage}
}%
%EndExpansion
This should look familiar. This is the same basic setup as the motor, with a
few exceptions. An important exception is that the commutator has been
replaced by the set of rings. We will call these ring contacts \emph{slip
rings} because the wires can slip along them while still maintaining
electrical contact because of the brushes. We have a current loop in a
(nearly) uniform, constant field. If I look from the slip ring side of the
loop, I\ have the same geometry we had before when we considered motors.
This time I want to consider doing work to turn the loop, and find the
induced emf in the loop. We start with Faraday's law%
\begin{equation}
\mathcal{E}=-N\frac{d\Phi _{B}}{dt}
\end{equation}%
since in our special case we only have one loop, this is just%
\begin{equation}
\mathcal{E}=-\frac{d\Phi _{B}}{dt}
\end{equation}%
Here is a the view looking at the cross section of the loop facing toward
the slip rings.\FRAME{dtbpF}{3.3001in}{3.0502in}{0pt}{}{}{Figure}{\special%
{language "Scientific Word";type "GRAPHIC";maintain-aspect-ratio
TRUE;display "USEDEF";valid_file "T";width 3.3001in;height 3.0502in;depth
0pt;original-width 3.2552in;original-height 3.007in;cropleft "0";croptop
"1";cropright "1";cropbottom "0";tempfilename
'MPQRHP00.wmf';tempfile-properties "XPR";}}Let's consider the flux through
the loop. The definition we have for flux is%
\begin{eqnarray*}
\Phi _{B} &=&\mathbf{B}\cdot \mathbf{A} \\
&=&BA\cos \theta \\
&=&BA_{proj}
\end{eqnarray*}%
where $\theta $ is the angle between the loop area vector and the magnetic
field direction.

I want to write the flux in terms of the lengths of the wire. When the loop
is standing up straight along the $y$-direction the projected area is just
the area%
\begin{equation*}
A=\ell a
\end{equation*}%
Then the projected area is 
\begin{equation*}
A_{proj}=\ell a\cos \theta
\end{equation*}
Let's check to make sure this works. When the loop is standing up straight
along the $y$-direction $\theta =0\unit{%
%TCIMACRO{\U{b0}}%
%BeginExpansion
{{}^\circ}%
%EndExpansion
},$ and $\cos \theta =1$ so 
\begin{equation*}
A_{_{proj}\max }=\ell a\cos \theta =\ell a
\end{equation*}%
so this works.

To find the emf generated, we\ need 
\begin{equation*}
\mathcal{E}=-\frac{d\Phi _{B}}{dt}
\end{equation*}%
and only the area is changing. so 
\begin{equation*}
\mathcal{E}=-\frac{d\Phi _{B}}{dt}=-B\frac{dA_{proj}}{dt}
\end{equation*}%
We realize that $\theta $ must change in time. We remember from Dynamics or
PH121 that we can use $\theta =\omega t$ where $\omega $ is the angular
speed of the rotating loop. Then%
\begin{equation*}
A_{proj}=\ell a\cos \omega t
\end{equation*}%
and%
\begin{equation*}
\mathcal{E}=-\frac{d\Phi _{B}}{dt}=-B\frac{d}{dt}\ell a\cos \omega t
\end{equation*}%
We recognize that $\theta $ changes as the loop turns Since $B$ is not
changing, the change in flux per unit time is just $B$ times the change in
area with time.

\begin{equation*}
\mathcal{E}=B\ell a\omega \sin \left( \omega t\right)
\end{equation*}%
Look at what we got! it is a sinusoidal emf. This will make a sinusoidal
current! 
\begin{eqnarray*}
I &=&\frac{\mathcal{E}}{R} \\
&=&\frac{B\ell a\omega \sin \left( \omega t\right) }{R}
\end{eqnarray*}%
for a circuit. Our emf looks like 
\begin{equation}
\mathcal{E}=\mathcal{E}_{\max }\sin \left( \omega t\right)
\end{equation}%
where%
\begin{equation}
\mathcal{E}_{\max }=B\ell a\omega
\end{equation}%
Here is a plot of the function \FRAME{dtbpFX}{2.6161in}{1.0075in}{0pt}{}{}{%
Plot}{\special{language "Scientific Word";type "MAPLEPLOT";width
2.6161in;height 1.0075in;depth 0pt;display "USEDEF";plot_snapshots
TRUE;mustRecompute FALSE;lastEngine "MuPAD";xmin "0";xmax "0.035";xviewmin
"0";xviewmax "0.035";yviewmin "-1";yviewmax
"1";viewset"XY";rangeset"X";plottype 4;plottickdisable TRUE;labeloverrides
3;x-label "t";y-label "EMF_max";axesFont "Times New
Roman,12,0000000000,useDefault,normal";numpoints 100;plotstyle
"patch";axesstyle "normal";axestips FALSE;xis \TEXUX{t};var1name
\TEXUX{$t$};function \TEXUX{$\sin \left( 2\pi \left( 60\right) t\right)
$};linecolor "blue";linestyle 1;pointstyle "point";linethickness
3;lineAttributes "Solid";var1range "0,0.035";num-x-gridlines 100;curveColor
"[flat::RGB:0x000000ff]";curveStyle "Line";VCamFile
'S4UWRU02.xvz';valid_file "T";tempfilename
'S4UWRM00.wmf';tempfile-properties "XPR";}}

Of course this sinusoidal emf will create what we call an \emph{alternating
current}. This is how the current in the outlets in your house is generated.

Of course, our generator only has one coil. Actual generators have multiple
coils. \FRAME{dhFU}{2.7812in}{1.6604in}{0pt}{\Qcb{Double Armature Generator
(Public Domain Image)}}{}{Figure}{\special{language "Scientific Word";type
"GRAPHIC";maintain-aspect-ratio TRUE;display "USEDEF";valid_file "T";width
2.7812in;height 1.6604in;depth 0pt;original-width 2.7389in;original-height
1.6233in;cropleft "0";croptop "1";cropright "1";cropbottom "0";tempfilename
'M1CGCL05.wmf';tempfile-properties "XPR";}}and we need a source of work to
turn the generator. A water turbine is an example,\FRAME{dtbpFU}{2.3134in}{%
2.4102in}{0pt}{\Qcb{Water Turbine driven Generator (Public Domain Image
courtesy U.S. Army Corps of Engineers)}}{}{Figure}{\special{language
"Scientific Word";type "GRAPHIC";maintain-aspect-ratio TRUE;display
"USEDEF";valid_file "T";width 2.3134in;height 2.4102in;depth
0pt;original-width 3.2266in;original-height 3.3641in;cropleft "0";croptop
"1";cropright "1";cropbottom "0";tempfilename
'MPQSBH01.wmf';tempfile-properties "XPR";}}or for emergencies, you might
have a gasoline powered generator, or in a nuclear reactor you might have a
steam driven generator.

\subsection{DC current from a generator}

We can also make a non-alternating current with a generator, but we have to
get tricky to do it. We use the same idea we used to make a motor. We cut
slots in the slip rings, so the current will switch directions every half
turn. We get a kind of poor quality current from this because the emf still
varies a lot. \FRAME{dtbpFX}{2.6161in}{1.7443in}{0pt}{}{}{Plot}{\special%
{language "Scientific Word";type "MAPLEPLOT";width 2.6161in;height
1.7443in;depth 0pt;display "USEDEF";plot_snapshots TRUE;mustRecompute
FALSE;lastEngine "MuPAD";xmin "0";xmax "0.035";xviewmin "0";xviewmax
"0.035";yviewmin "-1";yviewmax "1";viewset"XY";rangeset"X";plottype
4;plottickdisable TRUE;labeloverrides 3;x-label "t";y-label
"EMF_max";axesFont "Times New
Roman,12,0000000000,useDefault,normal";numpoints 100;plotstyle
"patch";axesstyle "normal";axestips FALSE;xis \TEXUX{t};var1name
\TEXUX{$t$};function \TEXUX{$\left\vert -\sin \left( 2\pi \left( 60\right)
t\right) \right\vert $};linecolor "blue";linestyle 1;pointstyle
"point";linethickness 1;lineAttributes "Solid";var1range
"0,0.035";num-x-gridlines 100;curveColor "[flat::RGB:0x000000ff]";curveStyle
"Line";VCamFile 'M1CE3I0O.xvz';valid_file "T";tempfilename
'M1CGCL07.wmf';tempfile-properties "XPR";}}Clever engineers design
generators for non-alternating or \emph{direct current} generators by
overlapping several current loops at different angles. Each loop has it's
own cut slip rings. The combined currents smooth out the ripples we see in
the previous figure. For semiconductor devices, special circuits are used to
make the current very smooth.

\subsection{Back emf}

We can see that a motor is a DC generator run backwards. I just want to
mention that when we talk about motors, we have to realize that as we send
current into the motor coils, there will be an induced emf that will try to
maintain the existing flux as the motor's loops turn. This emf will be in
the opposite direction of the applied current! So it reduces the amount of
work the motor can do. This is like the resistive force we encountered when
we pulled a loop from a magnetic field last lecture. This resistive force is
called the \emph{back emf} and must be accounted for in motor design.

\subsection{rms voltage}

We can realize that we have a slight problem in talking about alternating
voltages. The voltage constantly changes. How do we describe what the
voltage is?

We could give the max voltage--the amplitude of our $\mathcal{E}\left(
t\right) $ curve. But the voltage is at the max only a small percentage of
the time. We can't take the average. That is zero. And zero really doesn't
describe our voltage well!.

The average doesn't work because our generators make the emf go negative. We
could fix this by squaring the emf before we average it 
\begin{eqnarray*}
\overline{\mathcal{E}\left( t\right) } &=&0 \\
\overline{\mathcal{E}^{2}\left( t\right) } &\neq &0
\end{eqnarray*}%
and this could work. But then we have the average voltage squared, and we
really want just the voltage. No problem, let's take a square root.

\begin{equation*}
\sqrt{\overline{\mathcal{E}^{2}\left( t\right) }}
\end{equation*}%
This has units of volts, is like an average of the emf, but doesn't cancel
out because $\mathcal{E}\left( t\right) $ goes negative. Here is the process
graphically.

first $\mathcal{E}\left( t\right) $\FRAME{dtbpFX}{2.6161in}{1.0075in}{0pt}{}{%
}{Plot}{\special{language "Scientific Word";type "MAPLEPLOT";width
2.6161in;height 1.0075in;depth 0pt;display "USEDEF";plot_snapshots
TRUE;mustRecompute FALSE;lastEngine "MuPAD";xmin "0";xmax "0.035";xviewmin
"0";xviewmax "0.035";yviewmin "-1";yviewmax
"1";viewset"XY";rangeset"X";plottype 4;plottickdisable TRUE;labeloverrides
3;x-label "t";y-label "emf";axesFont "Times New
Roman,12,0000000000,useDefault,normal";numpoints 100;plotstyle
"patch";axesstyle "normal";axestips FALSE;xis \TEXUX{t};var1name
\TEXUX{$t$};function \TEXUX{$\sin \left( 2\pi \left( 60\right) t\right)
$};linecolor "blue";linestyle 1;pointstyle "point";linethickness
3;lineAttributes "Solid";var1range "0,0.035";num-x-gridlines 100;curveColor
"[flat::RGB:0x000000ff]";curveStyle "Line";VCamFile
'S4WG0T0G.xvz';valid_file "T";tempfilename
'S4WFKY00.wmf';tempfile-properties "XPR";}}now $\mathcal{E}^{2}\left(
t\right) $\FRAME{dtbpFX}{2.6161in}{1.0075in}{0pt}{}{}{Plot}{\special%
{language "Scientific Word";type "MAPLEPLOT";width 2.6161in;height
1.0075in;depth 0pt;display "USEDEF";plot_snapshots TRUE;mustRecompute
FALSE;lastEngine "MuPAD";xmin "0";xmax "0.035";xviewmin "0";xviewmax
"0.035";yviewmin "-1";yviewmax "1";viewset"XY";rangeset"X";plottype
4;plottickdisable TRUE;labeloverrides 3;x-label "t";y-label "emf^2";axesFont
"Times New Roman,12,0000000000,useDefault,normal";numpoints 100;plotstyle
"patch";axesstyle "normal";axestips FALSE;xis \TEXUX{t};var1name
\TEXUX{$t$};function \TEXUX{$\left( \sin \left( 2\pi \left( 60\right)
t\right) \right) ^{2}$};linecolor "blue";linestyle 1;pointstyle
"point";linethickness 3;lineAttributes "Solid";var1range
"0,0.035";num-x-gridlines 100;curveColor "[flat::RGB:0x000000ff]";curveStyle
"Line";VCamFile 'S4WG0S0F.xvz';valid_file "T";tempfilename
'S4WGJF03.wmf';tempfile-properties "XPR";}}and finally $\sqrt{\overline{%
\mathcal{E}^{2}\left( t\right) }}$\FRAME{dtbpFX}{2.6161in}{1.0075in}{0pt}{}{%
}{Plot}{\special{language "Scientific Word";type "MAPLEPLOT";width
2.6161in;height 1.0075in;depth 0pt;display "USEDEF";plot_snapshots
TRUE;mustRecompute FALSE;lastEngine "MuPAD";xmin "0";xmax "0.035";xviewmin
"0";xviewmax "0.035";yviewmin "-1";yviewmax
"1";viewset"XY";rangeset"X";plottype 4;plottickdisable TRUE;labeloverrides
3;x-label "t";y-label "emf_rms";axesFont "Times New
Roman,12,0000000000,useDefault,normal";numpoints 100;plotstyle
"patch";axesstyle "normal";axestips FALSE;xis \TEXUX{t};var1name
\TEXUX{$t$};function \TEXUX{$\left( \sin \left( 2\pi \left( 60\right)
t\right) \right) $};linecolor "blue";linestyle 2;pointstyle
"point";linethickness 1;lineAttributes "Dash";var1range
"0,0.035";num-x-gridlines 100;curveColor "[flat::RGB:0x000000ff]";curveStyle
"Line";function \TEXUX{$1/\sqrt{2}$};linecolor "blue";linestyle 1;pointstyle
"point";linethickness 3;lineAttributes "Solid";var1range
"0,0.035";num-x-gridlines 100;curveColor "[flat::RGB:0x000000ff]";curveStyle
"Line";VCamFile 'S4WFYS0B.xvz';valid_file "T";tempfilename
'S4WFM602.wmf';tempfile-properties "XPR";}} What we did is take the square
Root of the Mean of the Square of the emf. We can call this process the
root-mean-square process or rms for short. This is more useful than the mean
voltage--which is zero for alternating voltages. It is a better estimate of
the overall potential than the peak voltage value. So we often use rms
voltages to describe sources of alternating current.

We can come up with a convenient way to find the rms emf. Consider that our
alternating emf is given by 
\begin{equation*}
\mathcal{E}=\mathcal{E}_{\max }\sin \left( \omega t\right)
\end{equation*}%
but we squared this%
\begin{equation*}
\mathcal{E}^{2}=\mathcal{E}_{\max }^{2}\sin ^{2}\left( \omega t\right)
\end{equation*}%
and then took an average. Suppose we average $\sin ^{2}\left( \omega
t\right) $ over a long time so that $\omega t$ gets large. We could say that
And we want the case where $\theta _{\max }$ is large enough, our
alternating voltage make many cycles. We would have would be 
\begin{eqnarray*}
\overline{\mathcal{E}^{2}} &=&\frac{1}{\Delta t}\int_{t_{i}}^{t_{f}}\mathcal{%
E}_{\max }^{2}\sin ^{2}\left( \omega t\right) dt \\
&=&\frac{\mathcal{E}_{\max }^{2}}{\Delta t}\int_{t_{i}}^{t_{f}}\sin
^{2}\left( \omega t\right) dt
\end{eqnarray*}%
but we have run into the integral of sine squared before. In equation \ref%
{average_of_sine_squared} we found that 
\begin{equation*}
\int_{\text{many T}}\sin ^{2}\left( \frac{k\left( r_{2}+r_{1}\right) }{2}%
-\omega t+\phi _{o}\right) dt=\frac{1}{2}
\end{equation*}%
and really it didn't matter much what the argument of $\sin ^{2}$ was so
long as we integrated over many periods. The integral is always $1/2.$ We
can see this is true in our case by using a trig identity 
\begin{equation*}
\sin ^{2}\left( \theta \right) =\frac{1}{2}\left( 1-\cos \left( 2\theta
\right) \right)
\end{equation*}
\begin{eqnarray*}
\overline{\mathcal{E}^{2}} &=&\frac{\mathcal{E}_{\max }^{2}}{\Delta t}%
\int_{t_{i}}^{t_{f}}\sin ^{2}\left( \omega t\right) dt \\
&=&\frac{\mathcal{E}_{\max }^{2}}{\Delta t}\allowbreak
\int_{t_{i}}^{t_{f}}\left( \frac{1}{2}-\frac{1}{2}\cos 2t\omega \right) \,dt
\\
&=&\frac{\mathcal{E}_{\max }^{2}}{\Delta t}\left( \allowbreak
\int_{t_{i}}^{t_{f}}\frac{1}{2}dt-\int_{t_{i}}^{t_{f}}\frac{1}{2}\cos
2t\omega \,dt\right) \\
&=&\frac{\mathcal{E}_{\max }^{2}}{\Delta t}\left( \frac{1}{2}\Delta
t\allowbreak -\int_{t_{i}}^{t_{f}}\frac{1}{2}\cos 2t\omega \,dt\right)
\end{eqnarray*}

The second integral over $\cos \left( 2\omega t\right) $ will be zero. So we
have 
\begin{eqnarray*}
\overline{\mathcal{E}^{2}} &=&\frac{\mathcal{E}_{\max }^{2}}{\Delta t}\frac{1%
}{2}\Delta t\allowbreak \\
&=&\frac{1}{2}\mathcal{E}_{\max }^{2}
\end{eqnarray*}%
Now we need to take a square root to finish the rms process.%
\begin{eqnarray*}
\mathcal{E}_{rms} &=&\sqrt{\overline{\mathcal{E}^{2}}} \\
&=&\sqrt{\frac{1}{2}\mathcal{E}_{\max }^{2}} \\
&=&\frac{1}{\sqrt{2}}\mathcal{E}_{\max }
\end{eqnarray*}%
So the rms emf can be found from the max emf by dividing by the square root
of 2.

This gives a pretty good idea of the nature of the voltage for alternating
current. For example an $rms$ emf value of $120\unit{V}$ would have a peak
emf of 
\begin{eqnarray*}
\mathcal{E}_{\max } &=&\sqrt{2}\mathcal{E}_{rms} \\
&=&\sqrt{2}\left( 120\unit{V}\right) \\
&=&169.\,\allowbreak 71\unit{V}
\end{eqnarray*}%
The $rms$ value isn't the peak, it isn't the average $\left( 0\right) $ but
it gives us a measure of how much voltage (and risk) we have.

We could find an $rms$ current by using Ohm's law%
\begin{equation*}
\Delta V=IR
\end{equation*}%
This would be true for 
\begin{equation*}
\mathcal{E}_{\max }=I_{\max }R
\end{equation*}%
then we could find 
\begin{equation*}
I_{\max }=\frac{\mathcal{E}_{\max }}{R}
\end{equation*}%
then if we divide by $\sqrt{2}$ we have 
\begin{equation*}
\frac{I_{\max }}{\sqrt{2}}=\frac{\mathcal{E}_{\max }}{\sqrt{2}R}
\end{equation*}%
The right hand side is just 
\begin{equation*}
\frac{I_{\max }}{\sqrt{2}}=\frac{\mathcal{E}_{rms}}{R}
\end{equation*}%
so let's take 
\begin{equation*}
I_{rms}=\frac{I_{\max }}{\sqrt{2}}
\end{equation*}%
then Ohm's law for alternating currents becomes 
\begin{equation*}
I_{rms}=\frac{\mathcal{E}_{rms}}{R}
\end{equation*}%
or 
\begin{equation*}
\mathcal{E}_{rms}=I_{rms}R
\end{equation*}

\section{Transformers}

%TCIMACRO{%
%\TeXButton{Question 223.45.3}{\marginpar {
%\hspace{-0.5in}
%\begin{minipage}[t]{1in}
%\small{Question 223.45.3}
%\end{minipage}
%}}}%
%BeginExpansion
\marginpar {
\hspace{-0.5in}
\begin{minipage}[t]{1in}
\small{Question 223.45.3}
\end{minipage}
}%
%EndExpansion
The power comes into our houses at about $120V$. Your iPhone probably
requires $3\unit{V}$ to $5\unit{V}.$ How do we get the voltage we want out
of what the power company delivers? You know the answer is to plug in your
phone using a special adaptor. Lets see how it works.

Let's consider Faraday's law again. We know that 
\begin{equation*}
\mathcal{E=}\Delta V\left( t\right) =-N\frac{\Delta \Phi }{\Delta t}
\end{equation*}%
Suppose we use Faraday's idea and hook two coils up next to each other.%
\FRAME{dhF}{3.9418in}{1.7443in}{0pt}{}{}{Figure}{\special{language
"Scientific Word";type "GRAPHIC";maintain-aspect-ratio TRUE;display
"USEDEF";valid_file "T";width 3.9418in;height 1.7443in;depth
0pt;original-width 3.8934in;original-height 1.7071in;cropleft "0";croptop
"1";cropright "1";cropbottom "0";tempfilename
'M1CGCL08.wmf';tempfile-properties "XPR";}}One side we will hook to an
alternating emf. We will call this side coil $1$. The other side we will
hook a second coil with some resistive load like a light bulb. We will call
this coil $2.$The iron core keeps the magnetic field inside, so the flux
through coil $1$ ends up going through coil $2.$ (think of all the little
domains in the iron lining up along the field lines, and enhancing the field
lines with their own induced fields).

The alternating potential from the source will create a change in flux in
coil $1.$%
\begin{equation*}
\mathcal{E}_{1}\left( t\right) =-N_{1}\frac{\Delta \Phi _{1}}{\Delta t}
\end{equation*}%
If little flux is lost in the iron, then we will retrieve most of the flux
in coil $2$ and an emf will be induced in the resister (light bulb in our
case).%
\begin{equation*}
\mathcal{E}_{2}\left( t\right) =-N_{2}\frac{\Delta \Phi _{2}}{\Delta t}
\end{equation*}%
we just convinced ourselves that 
\begin{equation*}
\frac{\Delta \Phi _{1}}{\Delta t}\approx \frac{\Delta \Phi _{2}}{\Delta t}
\end{equation*}%
so we can solve each equation for the change in flux term, and set them
equal.%
\begin{eqnarray*}
\frac{\mathcal{E}_{1}\left( t\right) }{N_{1}} &=&-\frac{\Delta \Phi _{1}}{%
\Delta t} \\
\frac{\mathcal{E}_{2}\left( t\right) }{N_{2}} &=&-\frac{\Delta \Phi _{2}}{%
\Delta t}
\end{eqnarray*}%
so we have%
\begin{equation}
\frac{\mathcal{E}_{1}\left( t\right) }{N_{1}}=\frac{\mathcal{E}_{2}\left(
t\right) }{N_{2}}
\end{equation}%
If we solve for $\mathcal{E}_{2}\left( t\right) $ we can find the emf in
coil $2.$%
\begin{equation}
\frac{N_{2}}{N_{1}}\mathcal{E}_{1}\left( t\right) =\mathcal{E}_{2}\left(
t\right)
\end{equation}%
%TCIMACRO{%
%\TeXButton{Question 223.45.4}{\marginpar {
%\hspace{-0.5in}
%\begin{minipage}[t]{1in}
%\small{Question 223.45.4}
%\end{minipage}
%}}}%
%BeginExpansion
\marginpar {
\hspace{-0.5in}
\begin{minipage}[t]{1in}
\small{Question 223.45.4}
\end{minipage}
}%
%EndExpansion
You have probably already guessed how we make $\mathcal{E}_{2}$ to be some
emf amount we want. We take, say, our wall current that has a $rms$ value of 
$\mathcal{E}_{1}=120\unit{V}$. We pass it through this device we have built.
We design the device so that $\frac{N_{2}}{N_{1}}\mathcal{E}_{1}$ gives just
the potential that we want for $\mathcal{E}_{2}.$ If we want a lower emf,
say $12\unit{V},$ then we make $\frac{N_{2}}{N_{1}}=0.1$ so 
\begin{equation}
\frac{N_{2}}{N_{1}}\mathcal{E}_{1}=0.1\left( 120\unit{V}\right) =12\unit{V}
\end{equation}%
This is part of what the wall adaptor does. Usually wall adapters also have
some circuitry to make the alternating current into direct current.

Note that there is a cost to doing this. The power must be the same on both
sides (or a little less on side $2$). So 
\begin{equation*}
\mathcal{P}_{av}=I_{1,rms}\mathcal{E}_{1,rms}=I_{2,rms}\mathcal{E}_{2,rms}
\end{equation*}%
We can change the emf, but it will effect our current.

This device is called a transformer. Real transformers do lose power. Some
loss is due to the fact that not all the $B$-field from coil $1$ makes it
inside coil $2.$ But real transformers are not too bad with efficiencies
ranging from $90\%$ to $99\%$.

%TCIMACRO{%
%\TeXButton{Question 223.45.5}{\marginpar {
%\hspace{-0.5in}
%\begin{minipage}[t]{1in}
%\small{Question 223.45.5}
%\end{minipage}
%}}}%
%BeginExpansion
\marginpar {
\hspace{-0.5in}
\begin{minipage}[t]{1in}
\small{Question 223.45.5}
\end{minipage}
}%
%EndExpansion

\section{Induced Electric Fields}

Consider again a magnetic field and a moving charge. If the field changes,
the flux changes. Say, for example, that the field is increasing in strength.%
\FRAME{dhF}{2.1525in}{1.8836in}{0pt}{}{}{Figure}{\special{language
"Scientific Word";type "GRAPHIC";maintain-aspect-ratio TRUE;display
"USEDEF";valid_file "T";width 2.1525in;height 1.8836in;depth
0pt;original-width 2.1136in;original-height 1.8455in;cropleft "0";croptop
"1";cropright "1";cropbottom "0";tempfilename
'LVENES04.wmf';tempfile-properties "XPR";}}The charge will move in a circle
within the wire. We now understand that this is because we have induced an
emf. But think again about a battery.

The battery makes an electric field inside a wire. Recall this figure\FRAME{%
dhF}{3.1151in}{2.3808in}{0pt}{}{}{Figure}{\special{language "Scientific
Word";type "GRAPHIC";maintain-aspect-ratio TRUE;display "USEDEF";valid_file
"T";width 3.1151in;height 2.3808in;depth 0pt;original-width
5.0194in;original-height 3.8311in;cropleft "0";croptop "1";cropright
"1";cropbottom "0";tempfilename 'LVEONW0D.wmf';tempfile-properties "XPR";}}%
We must conclude that if we create an emf, we must have created an electric
field.\FRAME{dhF}{2.4016in}{1.9372in}{0pt}{}{}{Figure}{\special{language
"Scientific Word";type "GRAPHIC";maintain-aspect-ratio TRUE;display
"USEDEF";valid_file "T";width 2.4016in;height 1.9372in;depth
0pt;original-width 3.1566in;original-height 2.5399in;cropleft "0";croptop
"1";cropright "1";cropbottom "0";tempfilename
'LVENG105.wmf';tempfile-properties "XPR";}}This is really interesting. We
now have a hint at how wireless chargers might work (we will return to this
later). But now let's ask ourselves, do we need the wire there for this
electric field to happen? Of course, the force on the charge is the same if
there is no wire, so the $E$-field must be there whether or not there is a
wire.\FRAME{dhF}{3.0191in}{2.4137in}{0pt}{}{}{Figure}{\special{language
"Scientific Word";type "GRAPHIC";maintain-aspect-ratio TRUE;display
"USEDEF";valid_file "T";width 3.0191in;height 2.4137in;depth
0pt;original-width 2.975in;original-height 2.373in;cropleft "0";croptop
"1";cropright "1";cropbottom "0";tempfilename
'LVENIX06.wmf';tempfile-properties "XPR";}}%
%TCIMACRO{%
%\TeXButton{Question 223.45.6}{\marginpar {
%\hspace{-0.5in}
%\begin{minipage}[t]{1in}
%\small{Question 223.45.6}
%\end{minipage}
%}}}%
%BeginExpansion
\marginpar {
\hspace{-0.5in}
\begin{minipage}[t]{1in}
\small{Question 223.45.6}
\end{minipage}
}%
%EndExpansion
%TCIMACRO{%
%\TeXButton{Question 223.45.7}{\marginpar {
%\hspace{-0.5in}
%\begin{minipage}[t]{1in}
%\small{Question 223.45.7}
%\end{minipage}
%}}}%
%BeginExpansion
\marginpar {
\hspace{-0.5in}
\begin{minipage}[t]{1in}
\small{Question 223.45.7}
\end{minipage}
}%
%EndExpansion
In fact, the electric field is there in every place the magnetic field
exists so long as the magnetic field continues to increase.\FRAME{dhF}{%
3.0191in}{2.4137in}{0pt}{}{}{Figure}{\special{language "Scientific
Word";type "GRAPHIC";maintain-aspect-ratio TRUE;display "USEDEF";valid_file
"T";width 3.0191in;height 2.4137in;depth 0pt;original-width
2.975in;original-height 2.373in;cropleft "0";croptop "1";cropright
"1";cropbottom "0";tempfilename 'LVENRN08.wmf';tempfile-properties "XPR";}}%
This is quite a profound statement. We have said that a changing magnetic
field \emph{creates} an electric field. Before, only charges could create
electric fields, but in this case, the magnetic field is creating the
electric field. Of course, we know that moving charges are making the
magnetic field, so it is not totally surprising that the fields would be
related.

This electric field is just like a field produced by charges in that it
exerts a force 
\begin{equation*}
F=q_{o}E
\end{equation*}%
on a charge $q_{o}.$ But the electric field source is now very different.

\section{Relationship between induced fields}

\FRAME{dtbpF}{2.0868in}{1.8299in}{0pt}{}{}{Figure}{\special{language
"Scientific Word";type "GRAPHIC";maintain-aspect-ratio TRUE;display
"USEDEF";valid_file "T";width 2.0868in;height 1.8299in;depth
0pt;original-width 5.1249in;original-height 4.491in;cropleft "0";croptop
"1";cropright "1";cropbottom "0";tempfilename
'LVAHGM0C.wmf';tempfile-properties "XPR";}}It would be nice to have a
relationship between the changing $B$-field and the $E$-field that is
created. 
%TCIMACRO{%
%\TeXButton{Question 223.45.8}{\marginpar {
%\hspace{-0.5in}
%\begin{minipage}[t]{1in}
%\small{Question 223.45.8}
%\end{minipage}
%}} }%
%BeginExpansion
\marginpar {
\hspace{-0.5in}
\begin{minipage}[t]{1in}
\small{Question 223.45.8}
\end{minipage}
}
%EndExpansion
It would be good to obtain the most general relationship we can that relates
the electric field to the magnetic field. By understanding this
relationship, we can\ hope to gain insight into how to build things, and
into how the universe works. Let's start with a thought experiment.

Suppose we have a uniform but time varying magnetic field into the paper. In
this field, we have a conducting ring. If the field strength is increasing,
then the charges in the conducting loop shown will feel an induced emf, and
they will form a current that is tangent to the ring.

Let's find the work required to move a charge once around the loop. The
amount of potential energy difference is equal to the work done, so 
\begin{equation*}
\left\vert \Delta U\right\vert =\left\vert W\right\vert
\end{equation*}%
but in terms of the electric potential this is 
\begin{equation*}
\Delta U=q\Delta V=q\mathcal{E}
\end{equation*}%
so%
\begin{equation*}
\left\vert W\right\vert =\left\vert q\mathcal{E}\right\vert
\end{equation*}%
Now let's do this another way. Let's use 
\begin{equation*}
W=\int \mathbf{F}\cdot d\mathbf{s}
\end{equation*}%
The force making the current move is due to the induced potential
difference. This is just 
\begin{equation*}
F=qE
\end{equation*}%
which will not change as we go around the loop. The path will be along the
loop, so 
\begin{equation*}
W=\int_{loop}Fds
\end{equation*}%
and since the $E$-field is uniform in space at any given time as we travel
around the loop, 
\begin{equation*}
W=F\int_{loop}ds=qE2\pi r
\end{equation*}%
So we have two expressions for the work. Let's set them equal to each other%
\begin{equation*}
q\mathcal{E}=qE2\pi r
\end{equation*}%
The field is then%
\begin{equation}
\frac{\mathcal{E}}{2\pi r}=E
\end{equation}%
but 
\begin{equation*}
\mathcal{E}=-N\frac{d\Phi _{B}}{dt}
\end{equation*}%
so%
\begin{eqnarray*}
E &=&\frac{-N}{2\pi r}\frac{d\Phi _{B}}{dt} \\
&=&\frac{-1}{2\pi r}\frac{d\Phi _{B}}{dt}
\end{eqnarray*}

So if we know how our $B$-field varies in time, we can find the $E$-field.
Let's rewrite this one more time%
\begin{equation*}
2\pi rE=-\frac{d\Phi _{B}}{dt}
\end{equation*}%
Since the $E$-field is constant in as we go around the loop, we can
recognize the LHS as%
\begin{equation*}
2\pi rE=\int \overrightarrow{\mathbf{E}}\cdot d\overrightarrow{\mathbf{s}}
\end{equation*}%
which should be little surprise, since we found 
\begin{equation*}
\Delta V=\int \overrightarrow{\mathbf{E}}\cdot d\overrightarrow{\mathbf{s}}
\end{equation*}%
to be our basic definition of the electric potential. So%
\begin{equation}
\int \overrightarrow{\mathbf{E}}\cdot d\overrightarrow{\mathbf{s}}=-\frac{%
d\Phi _{B}}{dt}
\end{equation}%
This is a more general form of Faraday's law of induction.

This electric field is fundamentally different than the $E$-fields we
studied before. It is not a static field. If it were, then $\int 
\overrightarrow{\mathbf{E}}\cdot d\overrightarrow{\mathbf{s}}$ would be zero
around a ring of current. Think of conservation of energy. Around a closed
loop $\Delta V=0$ normally. Then 
\begin{equation*}
\Delta V=\int \overrightarrow{\mathbf{E}}\cdot d\overrightarrow{\mathbf{s}}%
\mathbf{=0\qquad }\text{no magnetic field}
\end{equation*}%
But since$\int \overrightarrow{\mathbf{E}}\cdot d\overrightarrow{\mathbf{s}}%
\neq 0$ for our induced $E$-field, we must recognize that this field is
different from those made by static charges. We call this field that does
not return the charge to the same energy state on traversing the loop a 
\emph{nonconservative field}. It is still just an electric field, but we are
gaining energy from the magnetic field, so $\Delta V$ around the loop is not
zero.

The equation 
\begin{equation}
\int \overrightarrow{\mathbf{E}}\cdot d\overrightarrow{\mathbf{s}}=-\frac{%
d\Phi _{B}}{dt}
\end{equation}%
is the most general form of Faraday's equation, but it is hard to use in
calculation for normal circuits where there is no magnetic field or where
the fields are weak. So we won't use it as we design normal circuits (we
will use the idea of inductance instead). But it plays a large part in the
electromagnetic theory of optics (PH375). We will just get a taste of this
here.

\section{Electromagnetic waves}

%TCIMACRO{%
%\TeXButton{Question 223.45.9}{\marginpar {
%\hspace{-0.5in}
%\begin{minipage}[t]{1in}
%\small{Question 223.45.9}
%\end{minipage}
%}}}%
%BeginExpansion
\marginpar {
\hspace{-0.5in}
\begin{minipage}[t]{1in}
\small{Question 223.45.9}
\end{minipage}
}%
%EndExpansion
Let's return to the idea that a changing magnetic field makes an electric
field.\FRAME{dhF}{3.0882in}{1.8836in}{0pt}{}{}{Figure}{\special{language
"Scientific Word";type "GRAPHIC";maintain-aspect-ratio TRUE;display
"USEDEF";valid_file "T";width 3.0882in;height 1.8836in;depth
0pt;original-width 3.0441in;original-height 1.8455in;cropleft "0";croptop
"1";cropright "1";cropbottom "0";tempfilename
'LVEOCT0B.wmf';tempfile-properties "XPR";}}But what about a changing
electric field? \FRAME{dhF}{3.4523in}{2.2883in}{0pt}{}{}{Figure}{\special%
{language "Scientific Word";type "GRAPHIC";maintain-aspect-ratio
TRUE;display "USEDEF";valid_file "T";width 3.4523in;height 2.2883in;depth
0pt;original-width 3.4065in;original-height 2.2485in;cropleft "0";croptop
"1";cropright "1";cropbottom "0";tempfilename
'LVEOEK0C.wmf';tempfile-properties "XPR";}}For the electric and magnetic
field equations to be symmetric, the changing electric field must create a
magnetic field.%
%TCIMACRO{%
%\TeXButton{Question 223.45.10}{\marginpar {
%\hspace{-0.5in}
%\begin{minipage}[t]{1in}
%\small{Question 223.45.10}
%\end{minipage}
%}} }%
%BeginExpansion
\marginpar {
\hspace{-0.5in}
\begin{minipage}[t]{1in}
\small{Question 223.45.10}
\end{minipage}
}
%EndExpansion
There is no requirement that the universe display such symmetry, but we have
found that it usually does. Indeed, a changing electric field creates a
magnetic field.

This foreshadows our final study of light. We learned earlier that light is
an \emph{electromagnetic} wave. What this means is that light is a wave in
both the electric \emph{and} magnetic fields.

Maxwell first predicted that such a wave could exist. The electric field of
the wave changes in time like a sinusoid. But this change will produce a
magnetic field that will also change in time. This changing magnetic field
recreates the electric field--which recreates the magnetic field, etc. Thus
the electromagnetic wave is \emph{self-sustaining}. It can break off from
the charges that create it and keep going forever because the electric field
and magnetic field of the wave create each other. You often see the
electromagnetic wave drawn like this:\FRAME{dhF}{1.8343in}{1.9657in}{0pt}{}{%
}{Figure}{\special{language "Scientific Word";type
"GRAPHIC";maintain-aspect-ratio TRUE;display "USEDEF";valid_file "T";width
1.8343in;height 1.9657in;depth 0pt;original-width 3.365in;original-height
3.6089in;cropleft "0";croptop "1";cropright "1";cropbottom "0";tempfilename
'LVEPLW0E.wmf';tempfile-properties "XPR";}}Where you can see the electric
and magnetic fields being created and recreated to make the wave self
sustaining.

This is a direct result of Maxwell's study of electromagnetic field theory.
Our more complete version of Faraday's law is one of the fundamental
equations describing electromagnetic waves known as \emph{Maxwell's
Equations.}%
\begin{equation*}
\int \overrightarrow{\mathbf{E}}\cdot d\overrightarrow{\mathbf{s}}=-\frac{%
d\Phi _{B}}{dt}
\end{equation*}%
You might guess that the symmetry we have observed would give another
similar equation relating the magnetic field and the electric flux.%
\begin{equation*}
\int \overrightarrow{\mathbf{B}}\cdot d\overrightarrow{\mathbf{s}}=+\frac{%
d\Phi _{E}}{dt}
\end{equation*}%
and we will find that this is true! But we have yet to show that is so. Note
that $\int \overrightarrow{\mathbf{B}}\cdot d\overrightarrow{\mathbf{s}}$
shows up in Ampere's law, 
\begin{equation*}
\int \overrightarrow{\mathbf{B}}\cdot d\overrightarrow{\mathbf{s}}\mathbf{%
=\mu }_{o}I
\end{equation*}%
so this last equation is not complete, but we are guessing that there is
also the possibility of an induced magnetic field from a changing electric
field, so we can predict that we need to modify Ampere's law to be%
\begin{equation*}
\int \overrightarrow{\mathbf{B}}\cdot d\overrightarrow{\mathbf{s}}\mathbf{%
=\mu }_{o}I+\frac{d\Phi _{E}}{dt}
\end{equation*}%
but again we will have to show this later.

In the next lecture, we will take a break from this deep theoretical
discussion, and learn how to use induction to make useful circuit devices
that you used in ME210.

%TCIMACRO{%
%\TeXButton{Basic Equations}{\hspace{-1.3in}{\LARGE Basic Equations\vspace{0.25in}}}}%
%BeginExpansion
\hspace{-1.3in}{\LARGE Basic Equations\vspace{0.25in}}%
%EndExpansion

\chapter{Inductors}

%TCIMACRO{%
%\TeXButton{Fundamental Concepts}{\hspace{-1.3in}{\LARGE Fundamental Concepts\vspace{0.25in}}}}%
%BeginExpansion
\hspace{-1.3in}{\LARGE Fundamental Concepts\vspace{0.25in}}%
%EndExpansion

\begin{itemize}
\item The self inductance $L$ has all the geometric and material properties
of a coil or other inductor an it can be found using $L=N\frac{d\Phi _{B}}{dI%
}$

\item The emf induced by an inductor is given by $\mathcal{E}\equiv -L\frac{%
\Delta I}{\Delta t}$

\item For a solenoid, the inductance can be found to be $L=\mu _{o}n^{2}V$

\item The energy stored in the magnetic field is $U_{L}=\frac{1}{2}LI^{2}$
and the energy density in the magnetic field is $u_{B}=\frac{1}{2}\frac{1}{%
\mu _{o}}B^{2}$

\item There is an \emph{apparent }voltage drop across an inductor of $\Delta
V_{L_{apparent}}=-L\frac{dI}{dt}$

\item There is also a mutual inductance between two inductors given by $%
M_{12}=\frac{N_{2}\Phi _{12}}{I_{1}}$
\end{itemize}

\section{Self Inductance}

%TCIMACRO{%
%\TeXButton{Question 223.46.1}{\marginpar {
%\hspace{-0.5in}
%\begin{minipage}[t]{1in}
%\small{Question 223.46.1}
%\end{minipage}
%}}}%
%BeginExpansion
\marginpar {
\hspace{-0.5in}
\begin{minipage}[t]{1in}
\small{Question 223.46.1}
\end{minipage}
}%
%EndExpansion
When we put capacitors and resisters in a circuit, we found that the current
did not jump to it's ultimate current value all at once. There was a time
dependence. But really, even if we just have a resister (and we always have
some resistance) the current does not reach it's full value instantaneously.
Think of our circuits, they are current loops! So as the current starts to
flow, Lenz's law tells us that there will be an induced emf that will oppose
the flow. The potential drop across the resister in a simple
battery-resister circuit is the potential drop due to the battery emf, \emph{%
minus the induced emf}.

We can use this fact to control current in circuits. To see how, we can
study a new case\FRAME{dhF}{1.977in}{1.0378in}{0pt}{}{}{Figure}{\special%
{language "Scientific Word";type "GRAPHIC";maintain-aspect-ratio
TRUE;display "USEDEF";valid_file "T";width 1.977in;height 1.0378in;depth
0pt;original-width 2.9058in;original-height 1.5134in;cropleft "0";croptop
"1";cropright "1";cropbottom "0";tempfilename
'LVJB5204.wmf';tempfile-properties "XPR";}}Let's take a coil of wire wound
around an iron cylindrical core. We start with a current as shown in the
figure above. Using our right hand rule we can find the direction of the $B$%
-field. But we now will allow the current to change. As it gets larger, we
know 
\begin{equation*}
\mathcal{E}=-N\frac{d\Phi _{B}}{dt}
\end{equation*}%
and we know that as the current changes, the magnitude of the $B$-field will
change, so the flux through the coil will change. We will have an induced
emf. We could derive this expression, but I think you can see that the
induced emf is proportional to the \emph{rate of change} of the current.%
\begin{equation*}
\mathcal{E}\equiv -L\frac{\Delta I}{\Delta t}
\end{equation*}

You might ask if the number of loops in the coil matters. The answer
is--yes. Does the size and shape of the coil matter--yes. But we will
include all these geometrical effects in the constant $L$ called the \emph{%
inductance}. It will hold all the material properties of the iron cored coil.%
\begin{equation*}
\mathcal{E}=-N\frac{d\Phi _{B}}{dt}\equiv -L\frac{dI}{dt}
\end{equation*}%
so for this case%
\begin{equation*}
-N\frac{d\Phi _{B}}{dt}\frac{dt}{dI}\equiv -L
\end{equation*}%
or%
\begin{equation*}
L=N\frac{d\Phi _{B}}{dI}
\end{equation*}%
If we start with no current (so no flux), then our change in flux is the
current flux minus zero. We can then say that%
\begin{equation*}
L=N\frac{\Phi _{B}}{I}
\end{equation*}

It might be more useful to write the inductance as 
\begin{equation*}
L=-\frac{\mathcal{E}_{L}}{\frac{dI}{dt}}
\end{equation*}
In designing circuits, we will usually just look up the inductance of the
device we choose, like we looked up the resistance of resisters or the
capacitance of the capacitors we use.

But for our special case of a simple coil, we can calculate the inductance,
because we know the induced emf using Faraday's law

\subsection{Inductance of a solenoid\protect\footnote{%
Think of this like the special case of a capacitor made from two flat large
plates, the parallel plate capacitor. It was somewhat ideal in the way we
treated it. Our treatment of the special case of a coil will likewise be
somewhat ideal.}}

%TCIMACRO{%
%\TeXButton{Question 223.46.2}{\marginpar {
%\hspace{-0.5in}
%\begin{minipage}[t]{1in}
%\small{Question 223.46.2}
%\end{minipage}
%}}}%
%BeginExpansion
\marginpar {
\hspace{-0.5in}
\begin{minipage}[t]{1in}
\small{Question 223.46.2}
\end{minipage}
}%
%EndExpansion
Let's extend our inductance calculation for a coil. Really the only easy
case we can do is that of a solenoid (that's probably a hint for the test).
So let's do it! We will just fill our solenoid with air instead of iron (if
we have iron, we have to take into account the magnetization, so it is not
terribly hard, but this is not what we\ want to concentrate on now). If the
solenoid has $N$ turns with length $L$ and we assume that $L$ is much bigger
than the radius $r$ of the loops then we can use our solution for the $B$%
-field created by a solenoid%
\begin{eqnarray*}
B &=&\mu _{o}nI \\
&=&\mu _{o}\frac{N}{\ell }I
\end{eqnarray*}%
The flux through each turn is then 
\begin{equation*}
\Phi _{B}=BA=\mu _{o}\frac{N}{\ell }IA
\end{equation*}%
where $A$ is the area of one of the solenoid loops. Then we use our equation
for inductance for a coil%
\begin{eqnarray*}
L &=&N\frac{\Phi _{B}}{I} \\
&=&N\frac{\left( \mu _{o}\frac{N}{\ell }IA\right) }{I} \\
&=&\frac{\left( \mu _{o}N^{2}A\right) }{\ell } \\
&=&\frac{\left( \mu _{o}N^{2}A\right) }{\ell }\frac{\ell }{\ell } \\
&=&\frac{\mu _{o}N^{2}A\ell }{\ell ^{2}} \\
&=&\frac{\mu _{o}N^{2}V}{\ell ^{2}} \\
&=&\mu _{o}n^{2}V
\end{eqnarray*}%
where we used the fact that the volume of the solenoid is $V=A\ell .$

Many inductors built for use in electronics are just this, air filled
solenoids. So this really is a somewhat practical solution.

\section{Energy in a Magnetic Field}

%TCIMACRO{%
%\TeXButton{Question 223.46.3}{\marginpar {
%\hspace{-0.5in}
%\begin{minipage}[t]{1in}
%\small{Question 223.46.3}
%\end{minipage}
%}}}%
%BeginExpansion
\marginpar {
\hspace{-0.5in}
\begin{minipage}[t]{1in}
\small{Question 223.46.3}
\end{minipage}
}%
%EndExpansion
An inductor, like a capacitor, stores energy in it's field. We would like to
know how much energy an inductor can store. From basic circuit theory we
know the power in a circuit will be%
\begin{equation*}
\mathcal{P}=I\Delta V
\end{equation*}%
If we just have an inductor, then the power removed from the circuit is 
\begin{eqnarray*}
\mathcal{P}_{cir} &=&I\Delta V=I\mathcal{E} \\
&=&I\left( -L\frac{dI}{dt}\right) \\
&=&-LI\frac{dI}{dt}
\end{eqnarray*}%
As with a resistor, we are taking power \emph{from the circuit} so the
result is negative. But unlike a resistor, this power is not being
dissipated as heat. It is going into the magnetic field of the inductor.
Therefore, we expect the power stored in the inductor field to be 
\begin{equation*}
\mathcal{P}_{L}=-\mathcal{P}_{cir}=LI\frac{dI}{dt}
\end{equation*}%
Power is the time rate of change of energy, so we can write this power
delivered to the inductor as 
\begin{equation*}
\frac{dU_{L}}{dt}=LI\frac{dI}{dt}
\end{equation*}%
Multiplying by $dt$ gives%
\begin{equation*}
dU_{L}=LIdI
\end{equation*}%
To find the total energy stored in the inductor we must integrate over $I.$%
\begin{eqnarray*}
U_{L} &=&\int dU_{L} \\
&=&\int_{0}^{I}LIdI \\
&=&L\int_{0}^{I}IdI \\
&=&\frac{1}{2}LI^{2}
\end{eqnarray*}

Thus, 
\begin{equation*}
U_{L}=\frac{1}{2}LI^{2}
\end{equation*}%
is the energy stored in the magnetic field of the inductor.

Suppose we have an inductor $L=30.0\times 10^{-3}\unit{H}.$ Plotting shows
us the dependence of $U_{L}$ on $I.$

\FRAME{dtbpFX}{2.6039in}{1.7365in}{0pt}{}{}{Plot}{\special{language
"Scientific Word";type "MAPLEPLOT";width 2.6039in;height 1.7365in;depth
0pt;display "USEDEF";plot_snapshots TRUE;mustRecompute FALSE;lastEngine
"MuPAD";xmin "0";xmax "4";xviewmin "0";xviewmax "4";yviewmin "0";yviewmax
"0.375037";viewset"XY";rangeset"X";plottype 4;labeloverrides 3;x-label
"I";y-label "U";axesFont "Times New
Roman,12,0000000000,useDefault,normal";numpoints 100;plotstyle
"patch";axesstyle "normal";axestips FALSE;xis \TEXUX{v73};var1name
\TEXUX{$I$};function \TEXUX{$\frac{1}{2}30.0\times 10^{-3}I^{2}$};linecolor
"blue";linestyle 1;pointstyle "point";linethickness 2;lineAttributes
"Solid";var1range "0,4";num-x-gridlines 100;curveColor
"[flat::RGB:0x000000ff]";curveStyle "Line";VCamFile
'LVJB520A.xvz';valid_file "T";tempfilename
'LVJB5202.wmf';tempfile-properties "XPR";}}

We should take a moment to see how our inductor compares to a capacitor as
an energy storage device. The energy stored in the electric field of a
capacitor%
\begin{equation*}
U_{L}=\frac{1}{2}L\left( I\right) ^{2}
\end{equation*}%
\begin{equation*}
U_{C}=\frac{1}{2}C\left( \Delta V\right) ^{2}
\end{equation*}%
Notice that Remarkable similarity!

\subsection{Energy Density in the magnetic field}

%TCIMACRO{%
%\TeXButton{Question 223.46.4}{\marginpar {
%\hspace{-0.5in}
%\begin{minipage}[t]{1in}
%\small{Question 223.46.4}
%\end{minipage}
%}}}%
%BeginExpansion
\marginpar {
\hspace{-0.5in}
\begin{minipage}[t]{1in}
\small{Question 223.46.4}
\end{minipage}
}%
%EndExpansion
We found that there was energy stored in the electric field of a capacitor.
Is the energy stored in the inductor really stored in the magnetic field of
the inductor? We believe that this is just the case, the energy, $U_{L},$ is
stored in the field. We would like to have an expression for the density of
the energy in the field.

To see this, let's start with the inductance of a solenoid.%
\begin{equation*}
L=\mu _{o}n^{2}A\ell
\end{equation*}%
The magnetic field is given by

\begin{equation*}
B=\mu _{o}nI
\end{equation*}%
then the energy in the field is given by 
\begin{eqnarray*}
U_{B} &=&\frac{1}{2}LI^{2} \\
&=&\frac{1}{2}\mu _{o}n^{2}A\ell I^{2}
\end{eqnarray*}%
If we rearrange this, we can see the solenoid field is found in the
expression twice%
\begin{eqnarray*}
U_{B} &=&\frac{1}{2}\left( \mu _{o}nI\right) A\ell \frac{\mu _{o}}{\mu _{o}}%
nI \\
&=&\frac{1}{2\mu _{o}}B^{2}A\ell
\end{eqnarray*}%
and the energy density is 
\begin{eqnarray*}
u_{B} &=&\frac{U_{B}}{A\ell } \\
&=&\frac{1}{2}\frac{1}{\mu _{o}}B^{2}
\end{eqnarray*}

Just like our energy density for the electric field, we derived this for a
specific case, a solenoid. But this expression is general. We should compare
to the energy density in the electric field.

\begin{equation*}
u_{E}=\frac{1}{2}\epsilon _{o}E^{2}
\end{equation*}%
Again, note the similarity!

\subsection{Oscillations in an LC Circuit}

We introduce a new circuit symbol for inductors\FRAME{dhF}{1.3284in}{0.2093in%
}{0pt}{}{}{Figure}{\special{language "Scientific Word";type
"GRAPHIC";maintain-aspect-ratio TRUE;display "USEDEF";valid_file "T";width
1.3284in;height 0.2093in;depth 0pt;original-width 1.2929in;original-height
0.1807in;cropleft "0";croptop "1";cropright "1";cropbottom "0";tempfilename
'LVJB5205.wmf';tempfile-properties "XPR";}}It looks like a coil, for obvious
reasons. We can place this new circuit element in a circuit. But what will
it do? To investigate this, let's start with a simple case, a circuit with a
charged capacitor and an inductor and nothing else.\FRAME{dtbpF}{2.1958in}{%
1.6042in}{0pt}{}{}{Figure}{\special{language "Scientific Word";type
"GRAPHIC";maintain-aspect-ratio TRUE;display "USEDEF";valid_file "T";width
2.1958in;height 1.6042in;depth 0pt;original-width 2.1577in;original-height
1.5679in;cropleft "0";croptop "1";cropright "1";cropbottom "0";tempfilename
'LVJB5203.wmf';tempfile-properties "XPR";}}Let us make two unrealistic
assumptions (we will relax these assumptions later).

\begin{quotation}
Assumption 1: There is no resistance in our LC circuit.

Assumption 2: There is no radiation emitted from the circuit.
\end{quotation}

Given these two assumptions, there is no mechanism for energy to escape the
circuit. Energy must be conserved. Can we describe the charge on the
capacitor, the current, and the energy as a function of time?

%TCIMACRO{%
%\TeXButton{Question 223.46.5}{\marginpar {
%\hspace{-0.5in}
%\begin{minipage}[t]{1in}
%\small{Question 223.46.5}
%\end{minipage}
%}}}%
%BeginExpansion
\marginpar {
\hspace{-0.5in}
\begin{minipage}[t]{1in}
\small{Question 223.46.5}
\end{minipage}
}%
%EndExpansion
It may pay off to recall some details of oscillators. Energy of the Simple
Harmonic Oscillator

\FRAME{dtbpF}{1.753in}{0.7878in}{0pt}{}{}{Figure}{\special{language
"Scientific Word";type "GRAPHIC";maintain-aspect-ratio TRUE;display
"USEDEF";valid_file "T";width 1.753in;height 0.7878in;depth
0pt;original-width 2.6757in;original-height 1.1882in;cropleft "0";croptop
"1";cropright "1";cropbottom "0";tempfilename
'S4Y8T701.wmf';tempfile-properties "XPR";}}

Remember from Dynamics or PH121 that a mass-spring system will oscillate.
The mass has kinetic energy because the mass is moving 
\begin{equation}
K=\frac{1}{2}mv^{2}
\end{equation}%
for our Simple Harmonic Oscillator we know that the position of the mass as
a function of time is given by 
\begin{equation*}
x\left( t\right) =x_{\max }\cos \left( \omega t+\phi \right)
\end{equation*}%
and the speed as a function of time is 
\begin{equation*}
v\left( t\right) =-\omega x_{\max }\sin \left( \omega t+\phi \right)
\end{equation*}%
then the kinetic energy as a function of time is%
\begin{eqnarray*}
K &=&\frac{1}{2}m\left( -\omega x_{\max }\sin \left( \omega t+\phi \right)
\right) ^{2} \\
&=&\frac{1}{2}m\omega ^{2}x_{\max }^{2}\sin ^{2}\left( \omega t+\phi \right)
\\
&=&\frac{1}{2}m\frac{k}{m}x_{\max }^{2}\sin ^{2}\left( \omega t+\phi \right)
\\
&=&\frac{1}{2}kx_{\max }^{2}\sin ^{2}\left( \omega t+\phi \right)
\end{eqnarray*}%
The spring has potential energy given by 
\begin{equation}
U=\frac{1}{2}kx^{2}
\end{equation}%
For our mechanical oscillator the potential as a function of time is%
\begin{equation*}
U=\frac{1}{2}kx_{\max }^{2}\cos ^{2}\left( \omega t+\phi \right)
\end{equation*}%
The total energy is given by%
\begin{eqnarray*}
E &=&K+U \\
&=&\frac{1}{2}kx_{\max }^{2}\sin ^{2}\left( \omega t+\phi \right) +\frac{1}{2%
}kx_{\max }^{2}\cos ^{2}\left( \omega t+\phi \right) \\
&=&\frac{1}{2}kx_{\max }^{2}\left( \sin ^{2}\left( \omega t+\phi \right)
+\cos ^{2}\left( \omega t+\phi \right) \right) \\
&=&\frac{1}{2}kx_{\max }^{2}
\end{eqnarray*}

We can see that the total energy won't change, and the energy switches back
and forth from kinetic to potential as the mass moves back and forth. If we
plot the kinetic and potential energy at points along the mass' path we get
something like this.

\FRAME{dtbpF}{4.7539in}{3.5699in}{0pt}{}{}{Figure}{\special{language
"Scientific Word";type "GRAPHIC";maintain-aspect-ratio TRUE;display
"USEDEF";valid_file "T";width 4.7539in;height 3.5699in;depth
0pt;original-width 10.0258in;original-height 7.5247in;cropleft "0";croptop
"1";cropright "1";cropbottom "0";tempfilename
'LVK7E002.wmf';tempfile-properties "XPR";}}

%TCIMACRO{%
%\TeXButton{Question 223.46.6}{\marginpar {
%\hspace{-0.5in}
%\begin{minipage}[t]{1in}
%\small{Question 223.46.6}
%\end{minipage}
%}}}%
%BeginExpansion
\marginpar {
\hspace{-0.5in}
\begin{minipage}[t]{1in}
\small{Question 223.46.6}
\end{minipage}
}%
%EndExpansion
One of the important uses of an inductor is to create \emph{electrical
oscillations.} Having recalled what oscillations look like, we can see that
a LC circuit will have an oscillating current.

here is our circuit again.\FRAME{dtbpF}{2.1958in}{1.6042in}{0pt}{}{}{Figure}{%
\special{language "Scientific Word";type "GRAPHIC";maintain-aspect-ratio
TRUE;display "USEDEF";valid_file "T";width 2.1958in;height 1.6042in;depth
0pt;original-width 2.1577in;original-height 1.5679in;cropleft "0";croptop
"1";cropright "1";cropbottom "0";tempfilename
'LVK7E005.wmf';tempfile-properties "XPR";}}

We will start with the switch open the capacitor charged to its maximum
value $Q_{\max }.$ For $t>0$ the switch is closed. Recall that the energy
stored in the capacitor is 
\begin{equation*}
U_{C}=\frac{Q^{2}}{2C}
\end{equation*}%
and the energy stored in the inductor is%
\begin{equation*}
U_{L}=\frac{1}{2}I^{2}L
\end{equation*}%
The total energy (because of our assumptions) is%
\begin{eqnarray*}
U &=&U_{C}+U_{L} \\
&=&\frac{Q^{2}}{2C}+\frac{1}{2}I^{2}L
\end{eqnarray*}%
The change in energy over time must be zero (again because of our
assumptions) so%
\begin{eqnarray*}
\frac{dU}{dt} &=&0 \\
&=&\frac{d}{dt}\left( \frac{Q^{2}}{2C}+\frac{1}{2}I^{2}L\right) \\
&=&\frac{Q}{C}\frac{dQ}{dt}+LI\frac{dI}{dt}
\end{eqnarray*}%
We recall that 
\begin{equation*}
I=\frac{dQ}{dt}
\end{equation*}

\begin{eqnarray*}
0 &=&\frac{Q}{C}\left( \frac{dQ}{dt}\right) +LI\frac{dI}{dt} \\
0 &=&\frac{Q}{C}\left( I\right) +LI\frac{dI}{dt} \\
0 &=&\frac{Q}{C}I+LI\frac{d\left( \frac{dQ}{dt}\right) }{dt} \\
0 &=&\frac{Q}{C}+L\frac{d^{2}Q}{dt^{2}}
\end{eqnarray*}%
or%
\begin{equation*}
\frac{d^{2}Q}{dt^{2}}=-\frac{Q}{LC}
\end{equation*}%
This is a differential equation that we recognize from M316. It looks just
like the differential equation for oscillatory motion! We try a solution of
the form%
\begin{equation*}
Q=A\cos \left( \omega t+\phi \right)
\end{equation*}%
then%
\begin{equation*}
\frac{dQ}{dt}=-A\omega \sin \left( \omega t+\phi \right)
\end{equation*}%
and%
\begin{equation*}
\frac{d^{2}Q}{dt^{2}}=-A\omega ^{2}\cos \left( \omega t+\phi \right)
\end{equation*}%
thus%
\begin{equation*}
A\omega ^{2}\cos \left( \omega t+\phi \right) =-\frac{1}{LC}A\cos \left(
\omega t+\phi \right)
\end{equation*}%
This is indeed a solution if 
\begin{equation*}
\omega =\frac{1}{\sqrt{LC}}
\end{equation*}%
When $\cos \left( \omega t+\phi \right) =1$, $Q=Q_{\max },$ thus 
\begin{equation*}
Q=Q_{\max }\cos \left( \omega t+\phi \right)
\end{equation*}

Now recall, 
\begin{eqnarray*}
I &=&\frac{dQ}{dt} \\
&=&\frac{d}{dt}\left( Q_{\max }\cos \left( \omega t+\phi \right) \right) \\
&=&-\omega Q_{\max }\sin \left( \omega t+\phi \right)
\end{eqnarray*}

We would like to determine $\phi .$ We use the initial conditions $t=0,$ $%
I=0 $ and $Q=Q_{\max }.$ Then

\begin{equation*}
0=-\omega Q_{\max }\sin \left( \phi \right)
\end{equation*}%
This is true for $\phi =0.$ Then%
\begin{eqnarray*}
Q &=&Q_{\max }\cos \left( \omega t\right) \\
I &=&-\omega Q_{\max }\sin \left( \omega t\right) \\
&=&-I_{\max }\sin \left( \omega t\right)
\end{eqnarray*}

We can use the solution for the charge on the capacitor and the current in
the inductor as a function of time to expand our energy equation%
\begin{eqnarray*}
U &=&U_{C}+U_{L} \\
&=&\frac{Q^{2}}{2C}+\frac{1}{2}I^{2}L \\
&=&\frac{1}{2C}Q_{\max }^{2}\cos ^{2}\left( \omega t\right) +\frac{1}{2}%
LI_{\max }^{2}\sin ^{2}\left( \omega t\right)
\end{eqnarray*}%
This looks a lot like our kinetic and potential energy equation for a
mass-spring system. The energy shifts from the capacitor to the inductor and
back like energy shifted from kinetic to potential energy for our
mass-spring, with the components out of phase by $90\unit{%
%TCIMACRO{\U{b0}}%
%BeginExpansion
{{}^\circ}%
%EndExpansion
}.$ By energy conservation, we know that 
\begin{equation*}
\frac{1}{2C}Q_{\max }^{2}=\frac{1}{2}LI_{\max }^{2}
\end{equation*}%
that is, the maximum energy in the capacitor equals the maximum energy in
the inductor. Then the total energy%
\begin{eqnarray*}
U &=&\frac{1}{2C}Q_{\max }^{2}\cos ^{2}\left( \omega t\right) +\frac{1}{2}%
LI_{\max }^{2}\sin ^{2}\left( \omega t\right) \\
&=&\frac{1}{2C}Q_{\max }^{2}\cos ^{2}\left( \omega t\right) +\frac{1}{2C}%
Q_{\max }^{2}\sin ^{2}\left( \omega t\right) \\
&=&\frac{Q_{\max }^{2}}{2C}
\end{eqnarray*}%
which must be the case if energy is conserved. We can plot the capacitor and
inductor energies at points in time as the current switches back and forth.%
\FRAME{dtbpF}{3.039in}{2.2822in}{0pt}{}{}{Figure}{\special{language
"Scientific Word";type "GRAPHIC";maintain-aspect-ratio TRUE;display
"USEDEF";valid_file "T";width 3.039in;height 2.2822in;depth
0pt;original-width 9.8104in;original-height 7.3526in;cropleft "0";croptop
"1";cropright "1";cropbottom "0";tempfilename
'LVJB5207.wmf';tempfile-properties "XPR";}}This is very much like our
harmonic oscillator picture. We can see that we have, indeed made an
electronic oscillator.

This type of circuit is a major component of radios which need a local
oscillatory circuit to operate.

\subsection{The RLC circuit}

As fascinating as the last section was, we know there really is some
resistance in the wire. So the restriction of no resistance needs to be
relaxed in our analysis.

\FRAME{dhF}{2.0755in}{1.6466in}{0pt}{}{}{Figure}{\special{language
"Scientific Word";type "GRAPHIC";maintain-aspect-ratio TRUE;display
"USEDEF";valid_file "T";width 2.0755in;height 1.6466in;depth
0pt;original-width 3.1566in;original-height 2.4984in;cropleft "0";croptop
"1";cropright "1";cropbottom "0";tempfilename
'M1B3ZN08.wmf';tempfile-properties "XPR";}}We can use the circuit in the
picture to imagine an LRC circuit. At first, we will keep $S_{2}$ open and
close $S_{1}$ to charge up the capacitor. Then we will close $S_{1}$ and
open $S_{2}.$ What will happen?

It is easier to find the current and charge on the capacitor as a function
of time by using energy arguments. The resistor will remove energy from the
circuit by dissipation (getting hot). The circuit has energy%
\begin{equation}
U=\frac{Q^{2}}{2C}+\frac{1}{2}LI^{2}
\end{equation}%
so from the work energy theorem, 
\begin{equation*}
W_{nc}=\Delta U
\end{equation*}%
the energy lost will be related to a change in the energy in the capacitor
and the inductor. Let's look at the rate of energy loss again

\begin{eqnarray}
\frac{dU}{dt} &=&\frac{d}{dt}\left( \frac{Q^{2}}{2C}+\frac{1}{2}LI^{2}\right)
\\
&=&\frac{Q}{C}\frac{dQ}{dt}+LI\frac{dI}{dt}  \notag
\end{eqnarray}%
but this must be equal to the loss rate. The power lost will be $P=I^{2}R$%
\begin{equation}
-I^{2}R=\frac{Q}{C}\frac{dQ}{dt}+LI\frac{dI}{dt}
\end{equation}%
This is a differential equation we can solve, let's first rearrange,
remembering that 
\begin{equation*}
I=\frac{dQ}{dt}
\end{equation*}%
then%
\begin{eqnarray*}
-I^{2}R &=&\frac{Q}{C}I+LI\frac{dI}{dt} \\
-IR &=&\frac{Q}{C}+L\frac{dI}{dt}
\end{eqnarray*}%
again using $I=\frac{dQ}{dt}$%
\begin{equation}
+L\frac{d^{2}Q}{dt^{2}}+\frac{dQ}{dt}R+\frac{Q}{C}=0
\end{equation}%
This is a good exercise for those of you who have taken math 316. This is
just like the equation governing a damped harmonic oscillator. The solution
is 
\begin{equation}
Q=Q_{\max }e^{-\frac{Rt}{2L}}\cos \omega _{d}t
\end{equation}%
where the angular frequency, $\omega _{d}$ is given by 
\begin{equation}
\omega _{d}=\left( \frac{1}{LC}-\left( \frac{R}{2L}\right) ^{2}\right) ^{%
\frac{1}{2}}
\end{equation}

Remember that for a damped harmonic oscillator 
\begin{equation*}
x\left( t\right) =Ae^{-\frac{b}{2m}t}\cos \left( \omega t+\phi \right)
\end{equation*}%
and 
\begin{equation*}
\omega =\left( \frac{k}{m}-\left( \frac{b}{2m}\right) ^{2}\right) ^{\frac{1}{%
2}}
\end{equation*}

The resistance acts like a damping coefficient! Suppose

\begin{equation*}
\begin{tabular}{l}
$Q_{\max }=0.05\unit{C}$ \\ 
$R=5\unit{%
%TCIMACRO{\U{3a9}}%
%BeginExpansion
\Omega%
%EndExpansion
}$ \\ 
$L=50\unit{H}$ \\ 
$C=0.02\unit{F}$%
\end{tabular}%
\end{equation*}

we have a graph that looks like this. \FRAME{dtbpFX}{2.3817in}{1.5878in}{0pt%
}{}{}{Plot}{\special{language "Scientific Word";type "MAPLEPLOT";width
2.3817in;height 1.5878in;depth 0pt;display "USEDEF";plot_snapshots
TRUE;mustRecompute FALSE;lastEngine "MuPAD";xmin "0";xmax "500";xviewmin
"0";xviewmax "200";yviewmin "-0.050771";yviewmax
"0.05";viewset"XY";rangeset"X";plottype 4;labeloverrides 3;x-label
"t";y-label "Q";axesFont "Times New
Roman,12,0000000000,useDefault,normal";numpoints 100;plotstyle
"patch";axesstyle "normal";axestips FALSE;xis \TEXUX{t};var1name
\TEXUX{$t$};function \TEXUX{$\left( 0.05\right) e^{-\frac{0.05}{2\left(
0.5\right) }t}$};linecolor "gray";linestyle 1;pointstyle
"point";linethickness 1;lineAttributes "Solid";var1range
"0,500";num-x-gridlines 900;curveColor "[flat::RGB:0x00c0c0c0]";curveStyle
"Line";rangeset"X";function \TEXUX{$-\left( 0.05\right)
e^{-\frac{0.05}{2\left( 0.5\right) }t}$};linecolor "yellow";linestyle
1;pointstyle "point";linethickness 1;lineAttributes "Solid";var1range
"0,200";num-x-gridlines 100;curveColor "[flat::RGB:0x00808000]";curveStyle
"Line";function \TEXUX{$\left( 0.05\right) e^{-\frac{0.05}{2\left(
0.5\right) }t}\cos \left( \left( \left( \frac{0.5}{0.5}-\left(
\frac{0.05}{2\left( 0.5\right) }\right) ^{2}\right) ^{\frac{1}{2}}\right)
t\right) $};linecolor "blue";linestyle 1;pointstyle "point";linethickness
1;lineAttributes "Solid";var1range "0,500";num-x-gridlines 900;curveColor
"[flat::RGB:0x000000ff]";curveStyle "Line";rangeset"X";VCamFile
'M1I1BM0C.xvz';valid_file "T";tempfilename
'M1I1BM00.wmf';tempfile-properties "XPR";}}The gray lines are%
\begin{equation}
\pm Q_{\max }e^{-\frac{Rt}{2L}}
\end{equation}%
They describe how the amplitude changes. We call this the \emph{envelope} of
the curve.

Let's look at 
\begin{equation}
\omega _{d}=\left( \frac{1}{LC}-\left( \frac{R}{2L}\right) ^{2}\right) ^{%
\frac{1}{2}}
\end{equation}%
If $\omega _{d}=0$ then 
\begin{eqnarray*}
0 &=&\frac{1}{LC}-\left( \frac{R}{2L}\right) ^{2} \\
\frac{1}{LC} &=&\left( \frac{R}{2L}\right) ^{2} \\
2L\sqrt{\frac{1}{LC}} &=&R
\end{eqnarray*}%
or%
\begin{equation}
R=\sqrt{\frac{4L}{C}}
\end{equation}%
We know that if $\omega _{d}=0$ there is no oscillation. We will call this
the critical resistance, $R_{c}.$ When the resistance is $R\geq R_{c}$ there
will be no oscillation. These represent the cases of being critically damped 
$\left( R=R_{c}\right) $ and overdamped $\left( R>R_{c}\right) .$ If $%
R<R_{c} $ we are underdamped, and the circuit will oscillate.

We don't know how to make electromagnetic waves yet, but we will in a few
lecture. Those waves carry what we call radio signals. To make the waves, we
often use circuits with resisters, capacitors, and inductors to provide the
oscillation. You can guess that if $Q$ on the capacitor oscillates, so does
the current. This oscillating current is what we use to drive the radio
antenna.

Now that we have some resistance, we could consider a circuit with just an
inductor and a resistor and a battery.\FRAME{dhF}{2.0028in}{1.5123in}{0pt}{}{%
}{Figure}{\special{language "Scientific Word";type
"GRAPHIC";maintain-aspect-ratio TRUE;display "USEDEF";valid_file "T";width
2.0028in;height 1.5123in;depth 0pt;original-width 2.7807in;original-height
2.0924in;cropleft "0";croptop "1";cropright "1";cropbottom "0";tempfilename
'LVK3K601.wmf';tempfile-properties "XPR";}}This is a little harder to deal
with than it might appear. Let's examine the difficulties in thinking about
such a circuit in the next section.

\section{Return to Non-Conservative Fields}

A few decades ago, we could have stopped here in an engineering class in
considering and LRC\ circuit. But as electrical devices become ever more
complicated, it might be good if we examine circuits with inductors and
resistors more carefully. A few lectures ago we found that 
\begin{equation*}
\int \mathbf{E}\cdot d\mathbf{s}=-\frac{d\Phi _{B}}{dt}
\end{equation*}%
implies a non-conservative electric field. We should take a moment to see
what this means. We should also investigate mutual inductance, which has
become a major engineering technique for wireless power. First let's
consider the following circuit.\cite{lewin2002}\FRAME{dtbpF}{2.2485in}{%
1.2168in}{0pt}{}{}{Figure}{\special{language "Scientific Word";type
"GRAPHIC";maintain-aspect-ratio TRUE;display "USEDEF";valid_file "T";width
2.2485in;height 1.2168in;depth 0pt;original-width 2.2087in;original-height
1.1831in;cropleft "0";croptop "1";cropright "1";cropbottom "0";tempfilename
'MEGNB701.wmf';tempfile-properties "XPR";}}notice that there is no battery.
If the field flux changes, will there be a potential difference measured by
the voltmeters? Let's use conservation of energy to analyze the circuit. I\
can draw in guesses for the currents.\FRAME{dtbpF}{2.4379in}{1.3197in}{0pt}{%
}{}{Figure}{\special{language "Scientific Word";type
"GRAPHIC";maintain-aspect-ratio TRUE;display "USEDEF";valid_file "T";width
2.4379in;height 1.3197in;depth 0pt;original-width 2.3964in;original-height
1.2851in;cropleft "0";croptop "1";cropright "1";cropbottom "0";tempfilename
'MEGND802.wmf';tempfile-properties "XPR";}}At the junction, we can use
conservation of charge to see how the currents combine or divide. This will
allow us to find the voltages.

But recall that 
\begin{equation*}
\doint \mathbf{E}\cdot ds=0
\end{equation*}%
is a statement of conservation of energy. In electronics, we sometimes call
this Kirchhoff's loop rule. And we learned that this is not true for induced
emfs. So in the middle loop Kirchhoff's loop rule--conservation of
energy--is not true! Some energy is transferred into or out of the circuit.
We now know that is because of the changing magnetic field, 
\begin{equation*}
\doint \mathbf{E}\cdot ds=\mathcal{E}=-\frac{d\Phi _{B}}{dt}
\end{equation*}%
for the middle loop. In this case, $\mathcal{E}$ comes just from the
changing external flux. It does \emph{not} depend on $R_{1}$ or on $R_{2}.$

We can write a conservation of energy equation (per unit charge) for each
loop.%
\begin{eqnarray*}
I_{1}R_{i}-IR_{1} &=&0 \\
-IR_{1}-IR_{2}+\mathcal{E} &=&0 \\
I_{2}R_{i}-IR_{2} &=&0
\end{eqnarray*}%
where $R_{i}$ is the internal resistance of the voltmeters. If there were no 
$\mathcal{E}$, then the volt meters would not read anything, but now we see
that 
\begin{eqnarray*}
\left\vert V_{1}\right\vert &=&I_{1}R_{i}\approx IR_{1} \\
\left\vert V_{2}\right\vert &=&I_{2}R_{i}\approx IR_{2}
\end{eqnarray*}%
This seems crazy. Each volt meter reads a different voltage.

To understand this, remember that our induced field is not a conservative
field. It is providing some energy. As we go around the loop we no longer
expect to get back to our starting voltage. We have gained (or lost) some
energy from the changing magnetic field. And for non-conservative fields, $%
\doint \mathbf{E}\cdot ds$ \emph{is path dependent}.

So as crazy as it seems, this is actually what we would find, each volt
meter reads a different voltage.

To try to make this idea of inductance make some sense, let's take another
strange circuit.\FRAME{dtbpF}{1.382in}{0.8077in}{0pt}{}{}{Figure}{\special%
{language "Scientific Word";type "GRAPHIC";maintain-aspect-ratio
TRUE;display "USEDEF";valid_file "T";width 1.382in;height 0.8077in;depth
0pt;original-width 3.8246in;original-height 2.2241in;cropleft "0";croptop
"1";cropright "1";cropbottom "0";tempfilename
'JXG90001.wmf';tempfile-properties "XPR";}}There is a battery, and resister,
and a single loop inductor. When the switch is thrown, the current will flow
as shown. The current will create a magnetic field that is out of the page
in the center of the loop. Since the loop, itself, is creating this field,
let's call this field a \emph{self field}.\FRAME{dtbpF}{1.3733in}{0.9409in}{%
0pt}{}{}{Figure}{\special{language "Scientific Word";type
"GRAPHIC";maintain-aspect-ratio TRUE;display "USEDEF";valid_file "T";width
1.3733in;height 0.9409in;depth 0pt;original-width 3.8246in;original-height
2.6122in;cropleft "0";croptop "1";cropright "1";cropbottom "0";tempfilename
'JXG94F02.wmf';tempfile-properties "XPR";}}Consider this self-field for a
moment. When we studied charge, we found that charge created an electric
field. That electric field could make \emph{another} charge accelerate. But
the electric field crated by a charge does not make the charge that created
it accelerate. This is an instance of a self-field, an electric self-field.
Now with this background, let's return to our magnetic self-field.

Let's take Faraday's law and apply it to this circuit. Let me choose an area
vector $\mathbf{A}$ that is the area of the big loop and positive out of the
page. Again, let's use conservation of energy (Kirchhoff's loop law). Let's
find $\doint \mathbf{E}\cdot ds$ for the entire circuit. We can start with
the battery. Since there is an electric field inside the battery we will
have a component of $\doint_{bat}\mathbf{E}\cdot ds$ as we cross it. The
battery field goes from positive to negative. If we go counter-clockwise,
our $ds$ direction traverses this from negative to positive, so the electric
field is up and the $ds$ direction is down, we have 
\begin{equation*}
\doint_{bat}\mathbf{E}\cdot ds=-\mathcal{E}_{bat}
\end{equation*}%
for this section of the circuit. Suppose we have ideal wires. If the wire
has no resistance, then it takes no work to move the charges through the
wire. In this case, an electron launched by the electric field in the
battery just coasts from the battery to the resister. There is no need to
have an acceleration in the ideal wire. The electric potential won't change
from the battery to the resister. So there doesn't need to be a field in
this ideal wire part to keep the charges going. But let's next we consider
the resister. There is a potential change as we go across it. And if there
is a change in potential, there must be an electric field. So the resister
also has an electric field inside of it. We have a component of $\doint_{R}%
\mathbf{E}\cdot ds$ that is equal to $\mathcal{E}_{R}=IR$ from this field. 
\begin{equation*}
\doint_{R}\mathbf{E}\cdot ds=IR
\end{equation*}%
Now we come to the big loop part. Since we have ideal wire, there is no
resistance in this part so there is no voltage drop for this part of the
circuit. All the energy that was given to the electrons by the battery was
lost in the resister. They just coast back to the other terminal of the
battery. Since there is no voltage drop in the big loop, 
\begin{equation*}
\mathcal{E}_{\text{big loop}}=0
\end{equation*}%
there is no electric field in the big loop either. Along the big loop, $ds$
is certainly not zero. so 
\begin{equation*}
\mathcal{E}_{\text{big loop}}=\doint_{\text{big loop}}\mathbf{E}\cdot ds=0
\end{equation*}

For the total loop we would have%
\begin{equation}
\doint \mathbf{E}\cdot ds=-\mathcal{E}_{batt}+IR+0
\end{equation}%
Normally, conservation of energy would tell us that all this must be zero,
since the sum of the energy changes around the loop must be zero if no
energy is lost. But now we know energy \emph{is} lost in making a magnetic
field.

Consider the magnetic flux through the circuit. The magnetic field is made
by the current in the circuit. Note that we\ arranged the circuit so the
battery and resister are in a part that has very little area, so we can
ignore the flux through that part of the circuit. Most of the flux will go
through the big loop part. The magnetic field is out of the paper inside of
the loop. The flux is 
\begin{equation}
\Phi _{B}=\doint \mathbf{B}\cdot d\mathbf{A}
\end{equation}%
and $\mathbf{B}$ and $\mathbf{A}$ are in the same direction. $\Phi _{B}$ is
positive.

Then from Biot-Savart%
\begin{equation}
\mathbf{B}=\frac{\mu _{o}I}{4\pi }\doint \frac{d\mathbf{s}\times \mathbf{%
\hat{r}}}{r^{2}}
\end{equation}%
Let's write this as 
\begin{eqnarray}
\mathbf{B} &=&I\left( \frac{\mu _{o}}{4\pi }\doint \frac{d\mathbf{s}\times 
\mathbf{\hat{r}}}{r^{2}}\right) \\
&=&I\left( \text{geometry factor}\right)  \notag
\end{eqnarray}%
If the geometry of the situation does not change, then $B$ and $I$ are
proportional. Since $B\varpropto I,$ then $\Phi _{B}\varpropto I$ since the
integral in Biot-Savart is the surface integral of $\mathbf{B,}$ and $%
\mathbf{B}$ is everywhere proportional to $I.$ Instead of using Biot-Savart,
let's just define a constant of proportionality that will contain all the
geometric factors. We could give it the symbol, $L$. Then 
\begin{equation}
\Phi _{B}=LI
\end{equation}%
where $L$ is my geometry factor. But we recognize this geometry factor. It
is just our inductance! This is what inductance is. It is all the geometry
factors that make up our loop that will make the magnetic field if we put a
current through it.

Assuming I don't change the geometry, then the inductance won't change and
we have 
\begin{equation}
\frac{d\Phi _{B}}{dt}=L\frac{dI}{dt}
\end{equation}%
and Faraday's law gives us 
\begin{equation}
\mathcal{E}=-\frac{d\Phi _{B}}{dt}=-L\frac{dI}{dt}
\end{equation}

Which says that we should not have expected $\doint \mathbf{E}\cdot ds=0$
for our case as we traverse the entire circuit. Integrating $\doint \mathbf{E%
}\cdot ds$ around the whole circuit including the big loop should not bring
us back to zero voltage. We have lost energy in making the field. Instead it
gives 
\begin{equation*}
\doint \mathbf{E}\cdot ds=-L\frac{dI}{dt}
\end{equation*}%
We are dealing with non-conservative fields. So we have some energy loss
like we would with a frictional force. It took some energy to make the
magnetic field!

With this insight, we can now make a new statement of conservation of energy
for such a situation. Integrating around the whole circuit gives%
\begin{equation*}
\doint \mathbf{E}\cdot ds=-\mathcal{E}_{bat}+\mathcal{E}_{R}
\end{equation*}%
Which we now realize should give $-L\frac{dI}{dt}$ so 
\begin{equation*}
\doint \mathbf{E}\cdot ds=-\mathcal{E}_{bat}+\mathcal{E}_{R}=-L\frac{dI}{dt}
\end{equation*}%
or more succinctly 
\begin{equation*}
-\mathcal{E}_{batt}+IR=-L\frac{dI}{dt}
\end{equation*}%
Now I\ can take the RHS to the left and find%
\begin{equation}
\mathcal{E}_{batt}-IR-L\frac{dI}{dt}=0
\end{equation}%
which accounts for all of the energy in the situation, so now we see that
energy is conserved. For those of you who go on in your study of
electronics. this looks like a Kirchhoff's rule with $-L\frac{dI}{dt}$ being
a voltage drop across the single loop inductor. Under most conditions we can
just treat $-L\frac{dI}{dt}$ as a voltage drop and it works fine. Most of
the time thinking this way does not cause much of a problem. But technically
it is not right!

We should consider where our magnetic flux came from. The magnetic flux was
created by the current. It is a self-field. The current can't make a
magnetic flux that would then modify that current. This self-flux won't make
an electric field in the wire. So there is no electric field in the big
loop, so there is no potential drop in that part of the circuit. It is just
that $\doint \mathbf{E}\cdot ds\neq 0$ because our field is not
conservative. We had to take some energy to create the magnetic field.

Now, if you are doing simple circuit design, you can pretend you don't know
about Faraday's law and this complication and just treat $-L\frac{dI}{dt}$
as though it were a voltage drop. But really it is just that going around
the loop we should expect 
\begin{equation*}
\doint \mathbf{E}\cdot ds=L\frac{dI}{dt}
\end{equation*}%
not%
\begin{equation*}
\doint \mathbf{E}\cdot ds=0
\end{equation*}%
The danger is that if you are designing a complicated device that depends on
there being an electric field in the inductor, your device will not work. We
have \emph{no external magnetic field,} our only magnetic field is the \emph{%
self-field} which will not produce an electric field (or at least will form
a very small electric field compared to the electric fields in the resistor
and the battery, due to the small resistance in the real wire we use to make
the big loop).

This is very subtile, and I struggle to remember this! Fortunately in most
circuit design it does not matter. We just treat the inductor as though it
were a true voltage drop.

I can make it even more exasperating by asking what you will see if you
place a voltmeter across the inductor. What I measure is a \textquotedblleft
voltage drop\textquotedblright\ of $LdI/dt,$ so maybe the there is a voltage
drop after all! But no, that is not right. The problem is that in
introducing the voltmeter, we have created a new loop. For this loop, the
field from our big loop \emph{is} an external field. .

\FRAME{dtbpF}{2.2442in}{1.4313in}{0pt}{}{}{Figure}{\special{language
"Scientific Word";type "GRAPHIC";maintain-aspect-ratio TRUE;display
"USEDEF";valid_file "T";width 2.2442in;height 1.4313in;depth
0pt;original-width 5.8538in;original-height 3.7284in;cropleft "0";croptop
"1";cropright "1";cropbottom "0";tempfilename
'JXGARJ03.wmf';tempfile-properties "XPR";}}So the changing magnetic field
through this voltmeter loop will produce an emf that will just match $LdI/dt$%
. And there will be an electric field--but it will be in the internal
resistor in the voltmeter. And that is what you will measure!

This may all seem very far fetched. But if you are designing radio
communications you \emph{want} to have a loss into the magnetic field,
because that energy transferred to the magnetic field becomes your radio
signal. This could be important!

%TCIMACRO{%
%\TeXButton{Pick it up here}{\marginpar {
%\hspace{-0.5in}
%\begin{minipage}[t]{1in}
%\small{Pick it up here}
%\end{minipage}
%}}}%
%BeginExpansion
\marginpar {
\hspace{-0.5in}
\begin{minipage}[t]{1in}
\small{Pick it up here}
\end{minipage}
}%
%EndExpansion
The bottom line is that for non-conservative fields you need to be careful.
If you are just designing simple circuits, you can just treat $LdI/dt$ \emph{%
as though it were a voltage drop}, but you may be badly burned by this if
your system is more complicated, depending on the existence of a real
electric field. You can see that if you are designing complicated sensing
devices, you may need to deeply understand the underlying physics to get
them to work. When in doubt, consult with a really good electrical engineer!

\subsection{RL Circuits: Solving for the current as a function of time}

\FRAME{dtbpF}{1.4615in}{1.0594in}{0pt}{}{}{Figure}{\special{language
"Scientific Word";type "GRAPHIC";maintain-aspect-ratio TRUE;display
"USEDEF";valid_file "T";width 1.4615in;height 1.0594in;depth
0pt;original-width 2.0072in;original-height 1.446in;cropleft "0";croptop
"1";cropright "1";cropbottom "0";tempfilename
'MEGP0I03.wmf';tempfile-properties "XPR";}}

The equation we found from Faraday's law or incorrectly from Kirchhoff's
rule is%
\begin{equation}
\mathcal{E}-IR-L\frac{dI}{dt}=0  \label{Potential Equation}
\end{equation}%
This is a differential equation. We can solve it for the current. To do so,
let's define a variable 
\begin{equation*}
x=\frac{\mathcal{E}}{R}-I
\end{equation*}%
and then we see that 
\begin{equation*}
dx=-dI
\end{equation*}%
Then we can write our differential equation as%
\begin{eqnarray*}
\frac{\mathcal{E}}{R}-I-\frac{L}{R}\frac{dI}{dt} &=&0 \\
x+\frac{L}{R}\frac{dx}{dt} &=&0
\end{eqnarray*}%
and so%
\begin{equation*}
x=-\frac{L}{R}\frac{dx}{dt}
\end{equation*}%
You might be able to guess the solution at this point from your M316
experience. But let's work it out as a review. We see that our $x$ equation
separates into 
\begin{equation*}
\frac{dx}{x}=-\frac{R}{L}dt
\end{equation*}

Integration yields%
\begin{equation*}
\int_{x_{o}}^{x}\frac{dx}{x}=-\int_{0}^{t}\frac{R}{L}dt
\end{equation*}%
\begin{equation*}
\ln \left( \frac{x}{x_{o}}\right) =-\frac{R}{L}t
\end{equation*}%
exponentiating both sides gives%
\begin{equation*}
\left( \frac{x}{x_{o}}\right) =e^{-\frac{R}{L}t}
\end{equation*}%
Now we replace $x$ with $\frac{\mathcal{E}}{R}-I$%
\begin{equation*}
\left( \frac{\frac{\mathcal{E}}{R}-I}{\frac{\mathcal{E}}{R}-I_{o}}\right)
=e^{-\frac{R}{L}t}
\end{equation*}%
And because at $t=0$, $I=0$

\begin{equation*}
\left( \frac{\frac{\mathcal{E}}{R}-I}{\frac{\mathcal{E}}{R}}\right) =e^{-%
\frac{R}{L}t}
\end{equation*}%
rearranging gives

\begin{equation}
I=\frac{\mathcal{E}}{R}\left( 1-e^{-\frac{Rt}{L}}\right)
\end{equation}%
or, defining another time constant%
\begin{equation}
\tau =\frac{L}{R}
\end{equation}%
we have 
\begin{equation}
I=\frac{\mathcal{E}}{R}\left( 1-e^{-\frac{t}{\tau }}\right)
\end{equation}

We can see that 
\begin{equation}
\frac{\mathcal{E}}{R}=I_{\max }
\end{equation}%
comes from Ohm's law. So just like with our capacitor-resister circuit, we
have a current that grows in time, approaching the maximum value we get
after a time $t$ which is much longer than $\tau .$

\FRAME{dtbpFX}{2.3679in}{1.58in}{0pt}{}{}{Plot}{\special{language
"Scientific Word";type "MAPLEPLOT";width 2.3679in;height 1.58in;depth
0pt;display "USEDEF";plot_snapshots TRUE;mustRecompute FALSE;lastEngine
"MuPAD";xmin "0.001";xmax "50";xviewmin "-0.0039999";xviewmax
"50.0049999";yviewmin "-0.00035";yviewmax
"3.50035";viewset"XY";rangeset"X";plottype 4;labeloverrides 3;x-label
"t";y-label "I_max";axesFont "Times New
Roman,12,0000000000,useDefault,normal";numpoints 100;plotstyle
"patch";axesstyle "normal";axestips FALSE;xis \TEXUX{t};var1name
\TEXUX{$t$};function \TEXUX{$\left( 1\right) \left( 3\right) \left(
1-e^{-\frac{t}{\left( 10\right) \left( 1\right) }}\right) $};linecolor
"blue";linestyle 1;pointstyle "point";linethickness 3;lineAttributes
"Solid";var1range "0.001,50";num-x-gridlines 100;curveColor
"[flat::RGB:0x000000ff]";curveStyle "Line";function \TEXUX{$3$};linecolor
"green";linestyle 2;pointstyle "point";linethickness 1;lineAttributes
"Dash";var1range "0.001,50";num-x-gridlines 100;curveColor
"[flat::RGB:0x00008000]";curveStyle "Line";VCamFile
'LVJ97Z07.xvz';valid_file "T";tempfilename
'LVJB5201.wmf';tempfile-properties "XPR";}}

You might expect that, like for a capacitor, there is an equation for an
inductor who has a maximum current flowing but for which the current source
is shorted (disconnected, and replaced with a resistanceless wire). The
equation is 
\begin{equation}
I=I_{o}e^{-\frac{t}{\tau }}
\end{equation}

\section{Magnetic Field Energy in Circuits}

We found that just like with a $RC$ circuit, we should expect there to be
energy stored in a $RL$ circuit. 
\begin{equation*}
U_{L}=\frac{1}{2}LI^{2}=\frac{1}{2}C\left( \Delta V\right) ^{2}
\end{equation*}%
Consider once again the $RL$ circuit shown below. \FRAME{dtbpF}{2.2719in}{%
1.3119in}{0pt}{}{}{Figure}{\special{language "Scientific Word";type
"GRAPHIC";maintain-aspect-ratio TRUE;display "USEDEF";valid_file "T";width
2.2719in;height 1.3119in;depth 0pt;original-width 2.2321in;original-height
1.2773in;cropleft "0";croptop "1";cropright "1";cropbottom "0";tempfilename
'MEGPAU05.wmf';tempfile-properties "XPR";}} Recall that the current in the
right-hand loop decays exponentially with time according to the expression 
\begin{equation*}
I=I_{o}e^{-\frac{t}{\tau }}
\end{equation*}%
where $I_{o}=\mathcal{E}/R$ is the initial current in the circuit and $\tau
=L/R$ is the time constant. As an example problem, let's show that all the
energy initially stored in the magnetic field of the inductor appears as
internal energy in the resistor as the current decays to zero.

Recall that energy is delivered to the resister%
\begin{equation*}
\frac{dU}{dt}=P=I^{2}R
\end{equation*}%
where $I$ is the instantaneous current.%
\begin{eqnarray*}
\frac{dU}{dt} &=&I^{2}R \\
\frac{dU}{dt} &=&\left( I_{o}e^{-\frac{t}{\tau }}\right) ^{2}R \\
\frac{dU}{dt} &=&I_{o}^{2}e^{-2\frac{t}{\tau }}R
\end{eqnarray*}

To find the total energy delivered to the resister we integrate

\begin{equation*}
dU=I_{o}^{2}e^{-2\frac{t}{\tau }}Rdt
\end{equation*}%
\begin{equation*}
\int dU=\int_{0}^{\infty }I_{o}^{2}e^{-2\frac{t}{\tau }}Rdt
\end{equation*}%
\begin{equation*}
U=\int_{0}^{\infty }I_{o}^{2}e^{-2\frac{t}{\tau }}Rdt
\end{equation*}%
\begin{equation*}
U=I_{o}^{2}R\int_{0}^{\infty }e^{-2\frac{t}{\tau }}dt
\end{equation*}%
Use your calculator, or an integral table, or Maple, or your very good
memory to recall that%
\begin{equation*}
\int e^{-ax}dx=-\frac{1}{a}e^{-ax}
\end{equation*}

If we let 
\begin{equation*}
a=-\frac{2}{\tau }
\end{equation*}%
then we can obtain%
\begin{equation*}
U=-\left. \frac{L}{2R}I_{o}^{2}Re^{-2\frac{t}{\tau }}\right\vert
_{0}^{\infty }
\end{equation*}%
\begin{equation*}
U=\frac{-L}{2}I_{o}^{2}\left( 0-1\right)
\end{equation*}%
\begin{equation}
U=\frac{1}{2}I_{o}^{2}L
\end{equation}

which is the initial energy stored in the magnetic field. All of the energy
that started in the inductor was delivered to the resistor.

\section{Mutual Induction}

Suppose we have two coils near each other. If either of the coils carries a
current, will there be an induced current in the other coil?\FRAME{dtbpF}{%
2.4863in}{2.0038in}{0pt}{}{}{Figure}{\special{language "Scientific
Word";type "GRAPHIC";maintain-aspect-ratio TRUE;display "USEDEF";valid_file
"T";width 2.4863in;height 2.0038in;depth 0pt;original-width
4.4547in;original-height 3.5846in;cropleft "0";croptop "1";cropright
"1";cropbottom "0";tempfilename 'JXFJ7W0U.wmf';tempfile-properties "XPR";}}

We define $\Phi _{12}$ as the flux through coil 2 due to the current in coil
1. Likewise if the battery is placed on coil $2$ we would have $\Phi _{21},$
the flux through coil 1 due to the current in coil 2.

We define the mutual inductance 
\begin{equation}
M_{12}=\frac{N_{2}\Phi _{12}}{I_{1}}
\end{equation}

BE CAREFUL! Not all books write the subscripts in the same order!

We can write the flux as%
\begin{equation*}
\Phi _{12}=\frac{M_{12}I_{1}}{N_{2}}
\end{equation*}
Then, using Faraday's law, we find the induced emf in coil 2%
\begin{eqnarray*}
\mathcal{E}_{2} &=&-N_{2}\frac{d\Phi _{B}}{dt} \\
&=&-N_{2}\frac{d}{dt}\left( \frac{M_{12}I_{1}}{N_{2}}\right) \\
&=&-M_{12}\frac{d}{dt}\left( I_{1}\right)
\end{eqnarray*}%
We state without proof the $M_{12}=M_{21}.$ Then%
\begin{equation*}
\mathcal{E}_{2}=-M\frac{dI_{1}}{dt}
\end{equation*}

\subsubsection{Example : \textquotedblleft Wireless\textquotedblright\
battery charger}

\FRAME{dhFU}{1.4538in}{2.316in}{0pt}{\Qcb{Rechargeable Toothbrush with an
inductive charger (Public Domain Image courtesy Jonas Bergsten)}}{}{Figure}{%
\special{language "Scientific Word";type "GRAPHIC";maintain-aspect-ratio
TRUE;display "USEDEF";valid_file "T";width 1.4538in;height 2.316in;depth
0pt;original-width 1.4183in;original-height 2.2762in;cropleft "0";croptop
"1";cropright "1";cropbottom "0";tempfilename
'M1B37N07.wmf';tempfile-properties "XPR";}}A rechargeable toothbrush needs a
connection that is not affected by water. We can use induction to form this
connection. We need two coils. One coil is the base, the other the handle.
The base carries current $I.$ The base has length $l$ and area $A$ and $%
N_{B} $ turns. The handle has $N_{H}$ turns and completely covers the base
solenoid. What is the mutual inductance?

Solution:

The magnetic field in the base solenoid is%
\begin{equation*}
\doint \mathbf{B\cdot }d\mathbf{s=B}\boldsymbol{\ell =}\mu _{o}N_{B}I
\end{equation*}%
or%
\begin{equation*}
B=\frac{\mu _{o}N_{B}I_{B}}{\ell }
\end{equation*}%
Because the handle surrounds the base, the flux through the handle is the
interior field of the base. The flux is 
\begin{equation*}
\Phi _{BH}=BA
\end{equation*}%
The mutual inductance is 
\begin{eqnarray*}
M &=&\frac{N_{H}\Phi _{BH}}{I_{B}} \\
&=&\frac{N_{H}BA}{I_{B}} \\
&=&\frac{N_{H}\left( \frac{\mu _{o}N_{B}I_{B}}{\ell }\right) A}{I_{B}} \\
&=&\mu _{o}\frac{N_{H}N_{B}A}{\ell }
\end{eqnarray*}

\subsection{Example: Rectangular Loop and a coil}

A rectangular loop of N close-packed turns is positioned near a long
straight wire.\FRAME{dtbpF}{1.0369in}{1.6838in}{0pt}{}{}{Figure}{\special%
{language "Scientific Word";type "GRAPHIC";maintain-aspect-ratio
TRUE;display "USEDEF";valid_file "T";width 1.0369in;height 1.6838in;depth
0pt;original-width 3.3581in;original-height 5.4743in;cropleft "0";croptop
"1";cropright "1";cropbottom "0";tempfilename
'JXFJ7W0W.wmf';tempfile-properties "XPR";}}What is the coefficient of mutual
inductance $M$ for the loop-wire combination?

The basic equations are

\begin{equation*}
M_{12}=\frac{N_{2}\Phi _{12}}{I_{1}}
\end{equation*}

\begin{equation*}
\doint \mathbf{B\cdot }d\mathbf{s}=\mu _{o}I
\end{equation*}%
\begin{equation*}
\doint \mathbf{B\cdot }d\mathbf{A}=\Phi _{B}
\end{equation*}

The field from the wire 
\begin{equation*}
\doint \mathbf{B\cdot }d\mathbf{s}=\mu _{o}I
\end{equation*}%
Take the path to be a circle surrounding the wire then $\mathbf{B}$ is
constant along the path and the direction of $\mathbf{B}$ is tangent to the
path.%
\begin{eqnarray*}
B\doint ds &=&\mu _{o}I \\
B2\pi r &=&\mu _{o}I
\end{eqnarray*}%
or 
\begin{equation*}
B=\frac{\mu _{o}I}{2\pi r}
\end{equation*}%
The flux through the rectangular loop is then perpendicular to the plane of
the loop 
\begin{equation*}
\doint \mathbf{B\cdot }d\mathbf{A=\Phi }_{B}
\end{equation*}%
\begin{eqnarray*}
\mathbf{\Phi }_{B} &=&\int Bydr \\
&=&\int_{a}^{b+a}\frac{\mu _{o}I}{2\pi r}ydr \\
&=&\frac{\mu _{o}Iy}{2\pi }\ln \frac{b+a}{a}
\end{eqnarray*}%
then%
\begin{equation*}
M=N\frac{\mu _{o}y}{2\pi }\ln \frac{b+a}{a}
\end{equation*}%
Suppose the loop has $N=100$ turns, $a=1\unit{cm}$, $b=8\unit{cm}$, $y=30%
\unit{cm}$, $\mu _{o}=4\pi \times 10^{-7}\frac{\unit{T}\unit{m}}{\unit{A}}$
what is the value of the mutual inductance?

\begin{equation*}
M=N\frac{\mu _{o}y}{2\pi }\ln \frac{b+a}{a}=\allowbreak \frac{%
1.\,\allowbreak 318\,3\times 10^{-3}}{\unit{A}}\unit{T}\unit{m}\unit{cm}%
=\allowbreak \frac{1.\,\allowbreak 318\,3\times 10^{-5}}{\unit{A}^{2}}\frac{%
\unit{m}^{2}}{\unit{s}^{2}}\unit{kg}
\end{equation*}

\begin{equation*}
\unit{H}=\allowbreak \allowbreak \frac{1}{\unit{A}^{2}}\frac{\unit{m}^{2}}{%
\unit{s}^{2}}\unit{kg}
\end{equation*}

%TCIMACRO{%
%\TeXButton{Basic Equations}{\hspace{-1.3in}{\LARGE Basic Equations\vspace{0.25in}}}}%
%BeginExpansion
\hspace{-1.3in}{\LARGE Basic Equations\vspace{0.25in}}%
%EndExpansion

\chapter{The Electromagnetic field}

We started off our study of electricity and magnetism saying we would
consider the environment made by a charge and how that environment affected
a mover charge. Then we found that moving charges are affected by the
environment created by other moving charges (currents). It is time to
consider the overall environment created by both electric and magnetic
fields acting together.

%TCIMACRO{%
%\TeXButton{Fundamental Concepts}{\hspace{-1.3in}{\LARGE Fundamental Concepts\vspace{0.25in}}}}%
%BeginExpansion
\hspace{-1.3in}{\LARGE Fundamental Concepts\vspace{0.25in}}%
%EndExpansion

\begin{itemize}
\item The electric and magnetic fields are really different manifestations
of the electromagnetic field. Which is manifest depends on our relative
motion.

\item The Galilean field transformations are%
\begin{eqnarray*}
\overrightarrow{\mathbf{E}}^{\prime } &=&\overrightarrow{\mathbf{E}}_{\text{%
charges}}+\overrightarrow{\mathbf{V}}_{S^{\prime }S}\times \overrightarrow{%
\mathbf{B}}_{\text{environment}} \\
\overrightarrow{\mathbf{B}}^{\prime } &=&\overrightarrow{\mathbf{B}}_{\text{%
magnet}}-\frac{1}{c^{2}}\left( \overrightarrow{\mathbf{V}}_{S^{\prime
}S}\times \overrightarrow{\mathbf{E}}_{\text{environment}}\right) \\
\overrightarrow{\mathbf{E}} &=&\mathbf{E}_{\text{charges}}^{\prime }-%
\overrightarrow{\mathbf{V}}_{S^{\prime }S}\times \overrightarrow{\mathbf{B}}%
_{\text{environment}}^{\prime } \\
\overrightarrow{\mathbf{B}} &=&\overrightarrow{\mathbf{B}}_{\text{magnet}%
}^{\prime }+\frac{1}{c^{2}}\left( \overrightarrow{\mathbf{V}}_{S^{\prime
}S}\times \overrightarrow{\mathbf{E}}_{\text{ environment}}^{\prime }\right)
\end{eqnarray*}

\item Gauss' law for magnetic fields is $\oint \mathbf{B}\times d\mathbf{A}%
=0 $
\end{itemize}

\section{Relative motion and field theory}

Long ago in your study of physics we talked about relative motion when we
discussed moving objects and Doppler shift. We considered two reference
frames with a relative velocity $v\hat{\imath}.$ We called them frame $A$
and frame $B$ \FRAME{dtbpF}{3.367in}{1.2302in}{0pt}{}{}{Figure}{\special%
{language "Scientific Word";type "GRAPHIC";maintain-aspect-ratio
TRUE;display "USEDEF";valid_file "T";width 3.367in;height 1.2302in;depth
0pt;original-width 5.6181in;original-height 2.0374in;cropleft "0";croptop
"1";cropright "1";cropbottom "0";tempfilename
'Sound_Waves/relative_motion0.wmf';tempfile-properties "XNPR";}}We need to
return to relative motion, considering what happens when there are fields
and charged particles involved. We will need to relabel our diagram to avoid
confusion because now $B$ will represent a magnetic field. So let's call the
two reference frames $S$ and $S^{\prime }.$ We will label each axis with a
prime in the $S^{\prime }$ frame.

\FRAME{dhF}{3.8096in}{1.8547in}{0pt}{}{}{Figure}{\special{language
"Scientific Word";type "GRAPHIC";maintain-aspect-ratio TRUE;display
"USEDEF";valid_file "T";width 3.8096in;height 1.8547in;depth
0pt;original-width 5.7326in;original-height 2.7762in;cropleft "0";croptop
"1";cropright "1";cropbottom "0";tempfilename
'NBD0L900.wmf';tempfile-properties "XPR";}}

%TCIMACRO{%
%\TeXButton{Question 223.47.1}{\marginpar {
%\hspace{-0.5in}
%\begin{minipage}[t]{1in}
%\small{Question 223.47.1}
%\end{minipage}
%}}}%
%BeginExpansion
\marginpar {
\hspace{-0.5in}
\begin{minipage}[t]{1in}
\small{Question 223.47.1}
\end{minipage}
}%
%EndExpansion
%TCIMACRO{%
%\TeXButton{Question 223.47.1.5}{\marginpar {
%\hspace{-0.5in}
%\begin{minipage}[t]{1in}
%\small{Question 223.47.1.5}
%\end{minipage}
%}}}%
%BeginExpansion
\marginpar {
\hspace{-0.5in}
\begin{minipage}[t]{1in}
\small{Question 223.47.1.5}
\end{minipage}
}%
%EndExpansion
%TCIMACRO{%
%\TeXButton{Question 223.47.2}{\marginpar {
%\hspace{-0.5in}
%\begin{minipage}[t]{1in}
%\small{Question 223.47.2}
%\end{minipage}
%}}}%
%BeginExpansion
\marginpar {
\hspace{-0.5in}
\begin{minipage}[t]{1in}
\small{Question 223.47.2}
\end{minipage}
}%
%EndExpansion
%TCIMACRO{%
%\TeXButton{Question 223.47.3}{\marginpar {
%\hspace{-0.5in}
%\begin{minipage}[t]{1in}
%\small{Question 223.47.3}
%\end{minipage}
%}}}%
%BeginExpansion
\marginpar {
\hspace{-0.5in}
\begin{minipage}[t]{1in}
\small{Question 223.47.3}
\end{minipage}
}%
%EndExpansion
Now let's assume we have a magnetic field in the region of space where our
two reference frames exist. Let's say that the magnetic field is stationary
in frame $S.$ This will be our environment. Let's also give a charge to the
person in frame $S^{\prime }$. This will be our mover charge.

\FRAME{dhF}{4.8544in}{2.7665in}{0pt}{}{}{Figure}{\special{language
"Scientific Word";type "GRAPHIC";maintain-aspect-ratio TRUE;display
"USEDEF";valid_file "T";width 4.8544in;height 2.7665in;depth
0pt;original-width 4.9201in;original-height 2.7905in;cropleft "0";croptop
"1";cropright "1";cropbottom "0";tempfilename
'NBD0L901.wmf';tempfile-properties "XPR";}}Is there a force on the charge?

If we are with the person in reference frame $S,$ then we must say yes. The
charge is moving along with frame $S^{\prime }$ with a velocity $%
\overrightarrow{\mathbf{v}}=V\hat{\imath}$ so there will be a force 
\begin{eqnarray*}
\overrightarrow{\mathbf{F}} &=&q\overrightarrow{\mathbf{v}}\times 
\overrightarrow{\mathbf{B}} \\
&=&qv\hat{\imath}\times B\left( -\hat{k}\right) \\
&=&qVB\hat{\jmath}
\end{eqnarray*}%
in the $\widehat{j}$ direction.

Now let's ride along with the person in frame $S^{\prime }.$ From this
frame, the charge looks stationary. So $v=0$ and 
\begin{equation*}
F=q\left( \mathbf{0}\right) \times \overrightarrow{\mathbf{B}}=0
\end{equation*}%
Both can't be true! So which is it? Is there a force on the charge or not?
Consider that the existence of a force is something we can test. A force
causes motion to change in ways we can detect. (the person in frame $%
S^{\prime }$ would \emph{feel} the pull on the charge he is holding). So
ultimately we can perform the experiment and see that there really is a
force. But where does the force come from?

Let's consider our fields. We have come to see fields as the source of
electric and magnetic forces. Electric forces come from electric fields
which come from environmental charges. Magnetic forces come from
environmental magnetic fields which come from moving charges.

And here is the difficulty, we are having trouble recognizing when the
charge is moving. We know from our consideration of relative motion that we
could view this situation as frame $S^{\prime }$ moving to the right with
frame $S$ stationary, or frame $S$ moving to the left with frame $S^{\prime
} $ stationary. There is no way to say that only one of these views is
correct. Both are equally valid.

\FRAME{dtbpF}{3.48in}{1.7677in}{0in}{}{}{Figure}{\special{language
"Scientific Word";type "GRAPHIC";maintain-aspect-ratio TRUE;display
"USEDEF";valid_file "T";width 3.48in;height 1.7677in;depth
0in;original-width 3.4333in;original-height 1.7296in;cropleft "0";croptop
"1";cropright "1";cropbottom "0";tempfilename
'NBD0L902.wmf';tempfile-properties "XPR";}}In our case, we are considering
that person $S$ sees a moving charge. We have learned that moving charge
will make \emph{both} an electric field \emph{and} a magnetic field. This is
the situation from frame $S.$ But person $S^{\prime }$ sees a static charge.
This charge will \emph{only} make an electric field. We need a way to
resolve this apparent contradiction.

\subsection{Galilean transformation}

To resolve this difficulty, let's go back to forces. Here is our case of a
constant magnetic field that is stationary in frame $S$ with a charge in
frame $S^{\prime }$ again. \FRAME{dhF}{4.8544in}{2.7665in}{0pt}{}{}{Figure}{%
\special{language "Scientific Word";type "GRAPHIC";maintain-aspect-ratio
TRUE;display "USEDEF";valid_file "T";width 4.8544in;height 2.7665in;depth
0pt;original-width 4.9201in;original-height 2.7905in;cropleft "0";croptop
"1";cropright "1";cropbottom "0";tempfilename
'NBD0L903.wmf';tempfile-properties "XPR";}}We can't see fields, but we can
see acceleration of a particle. Since by Newton's second law 
\begin{equation*}
F=ma
\end{equation*}%
we will know if there is an acceleration and therefore we will know if there
is a force! So are the forces and accelerations of a charged particle the
same in each frame? Let's find out.

Remember from Dynamics or PH121 that the speed of a particle transforms like
this%
\begin{eqnarray}
\overrightarrow{\mathbf{v}}^{\prime } &=&\overrightarrow{\mathbf{v}}-%
\overrightarrow{\mathbf{V}}_{S^{\prime }S} \\
\overrightarrow{\mathbf{v}} &=&\overrightarrow{\mathbf{v}}^{\prime }+%
\overrightarrow{\mathbf{V}}_{S^{\prime }S}  \notag
\end{eqnarray}%
where $V_{S^{\prime }S}$ is the relative speed between the two frames. What
this means is that if we have a particle moving with speed $v^{\prime }$ in
frame $S^{\prime }$ and we observe this particle in frame $S$ the speed of
that particle will seem to be $\overrightarrow{\mathbf{v}}=\overrightarrow{%
\mathbf{v}}^{\prime }+\overrightarrow{\mathbf{V}}_{S^{\prime }S}$. In our
case, $\overrightarrow{\mathbf{V}}_{SS^{\prime }}=V_{x}\hat{\imath}$ so $%
\overrightarrow{\mathbf{v}}=\overrightarrow{\mathbf{v}}^{\prime }+V_{x}\hat{%
\imath}.$

A quick example might help. Suppose we have a person in the gym running on a
treadmill. \FRAME{dtbpF}{1.5973in}{1.094in}{0pt}{}{}{Figure}{\special%
{language "Scientific Word";type "GRAPHIC";maintain-aspect-ratio
TRUE;display "USEDEF";valid_file "T";width 1.5973in;height 1.094in;depth
0pt;original-width 1.5618in;original-height 1.0603in;cropleft "0";croptop
"1";cropright "1";cropbottom "0";tempfilename
'NBD0L904.wmf';tempfile-properties "XPR";}}The treadmill track belt has a
relative speed $\overrightarrow{\mathbf{V}}_{S^{\prime }S}=-2\frac{\unit{m}}{%
\unit{s}}\hat{\imath}$ with respect to the room. We will say that the room
is frame $S$. Then if we envision a reference frame riding along the
treadmill, that would be frame $S^{\prime }$. A person standing on the
treadmill in frame $S^{\prime }$ sees themselves as not moving, and the rest
of the room as moving the opposite direction. \FRAME{dtbpF}{2.5244in}{%
2.0254in}{0pt}{}{}{Figure}{\special{language "Scientific Word";type
"GRAPHIC";maintain-aspect-ratio TRUE;display "USEDEF";valid_file "T";width
2.5244in;height 2.0254in;depth 0pt;original-width 2.4837in;original-height
1.9873in;cropleft "0";croptop "1";cropright "1";cropbottom "0";tempfilename
'NBD0L905.wmf';tempfile-properties "XPR";}}The notation $V_{S^{\prime }S}$
means the speed of the reference frame $S^{\prime }$ with respect to frame $%
S $ or in our case the speed of the treadmill with respect to the room $%
\overrightarrow{\mathbf{V}}_{S^{\prime }S}=-2\frac{\unit{m}}{\unit{s}}\hat{%
\imath}.$

Now suppose the person is running at speed $\overrightarrow{\mathbf{v}}%
^{\prime }=1.9\frac{\unit{m}}{\unit{s}}\hat{\imath}^{\prime }$ on the tread
mill in the tread mill frame $S^{\prime }$. \FRAME{dtbpF}{2.6403in}{1.9614in%
}{0in}{}{}{Figure}{\special{language "Scientific Word";type
"GRAPHIC";maintain-aspect-ratio TRUE;display "USEDEF";valid_file "T";width
2.6403in;height 1.9614in;depth 0in;original-width 2.5979in;original-height
1.9233in;cropleft "0";croptop "1";cropright "1";cropbottom "0";tempfilename
'NBD0L906.wmf';tempfile-properties "XPR";}}What is his/her speed with
respect to the room? It seems obvious that we take the two speeds and add
them. 
\begin{equation*}
\overrightarrow{\mathbf{v}}=1.9\frac{\unit{m}}{\unit{s}}\hat{\imath}^{\prime
}-2\frac{\unit{m}}{\unit{s}}\hat{\imath}=-0.1\frac{\unit{m}}{\unit{s}}\hat{%
\imath}
\end{equation*}%
since the $i$ and $\hat{\imath}^{\prime }$ directions are the same.

The person is going to fall off the end of the treadmill unless they pick up
the pace! This example just used the second equation in our transformation.%
\begin{equation*}
\overrightarrow{\mathbf{v}}=\overrightarrow{\mathbf{v}}^{\prime }+%
\overrightarrow{\mathbf{V}}_{S^{\prime }S}
\end{equation*}%
likewise, if we want to know how fast the person is walking with respect to
the treadmill frame, we take the room speed $\overrightarrow{\mathbf{v}}=-0.1%
\frac{\unit{m}}{\unit{s}}\hat{\imath}$ and subtract from it the
treadmill/room relative speed $\overrightarrow{V}_{S^{\prime }S}=-2\frac{%
\unit{m}}{\unit{s}}\hat{\imath}$ to obtain%
\begin{equation*}
\overrightarrow{\mathbf{v}}^{\prime }=-0.1\frac{\unit{m}}{\unit{s}}\hat{%
\imath}-\left( -2\frac{\unit{m}}{\unit{s}}\hat{\imath}\right) =1.9\frac{%
\unit{m}}{\unit{s}}\hat{\imath}=1.9\frac{\unit{m}}{\unit{s}}\hat{\imath}%
^{\prime }
\end{equation*}

Armed with the Galilean transform, we can find the acceleration by taking a
derivative%
\begin{eqnarray*}
\frac{d\overrightarrow{\mathbf{v}}^{\prime }}{dt} &=&\frac{d\overrightarrow{%
\mathbf{v}}}{dt}-\frac{d\overrightarrow{\mathbf{V}}_{S^{\prime }S}}{dt} \\
\frac{d\overrightarrow{\mathbf{v}}}{dt} &=&\frac{d\overrightarrow{\mathbf{v}}%
^{\prime }}{dt}+\frac{d\overrightarrow{\mathbf{V}}_{S^{\prime }S}}{dt}
\end{eqnarray*}%
then 
\begin{eqnarray*}
\overrightarrow{\mathbf{a}}^{\prime } &=&\overrightarrow{\mathbf{a}}-\frac{d%
\overrightarrow{\mathbf{V}}_{S^{\prime }S}}{dt} \\
\overrightarrow{\mathbf{a}} &=&\overrightarrow{\mathbf{a}}^{\prime }+\frac{d%
\overrightarrow{\mathbf{V}}_{S^{\prime }S}}{dt}
\end{eqnarray*}%
but we will only consider constant relative motion\footnote{%
Accelerating reference frames are treated by General Relatively and are
treated with the notation of contravariant and covariant vectors, which are
beyond this course. They are taken up in a graduate level electricity and
magnetism course.}, so 
\begin{equation*}
\frac{d\overrightarrow{\mathbf{V}}_{S^{\prime }S}}{dt}=0
\end{equation*}%
then both equations tell us 
\begin{equation*}
\overrightarrow{\mathbf{a}}^{\prime }=\overrightarrow{\mathbf{a}}
\end{equation*}%
This was a lot of work, but the end of all this talk about reference frames
shows us that there \emph{must be a force }%
\begin{equation*}
\overrightarrow{\mathbf{F}}=m\overrightarrow{\mathbf{a}}=m\overrightarrow{%
\mathbf{a}}^{\prime }
\end{equation*}%
in both frame $S$ and $S^{\prime }.$ The mass is the same in both frames,
and so is the acceleration.

We can gain some insight into finding the mysterious missing force in frame $%
S^{\prime }$ by considering the net force in the case of both an electric
and a magnetic field%
\begin{equation*}
\overrightarrow{\mathbf{F}}_{net}=q\overrightarrow{\mathbf{E}}+q%
\overrightarrow{\mathbf{v}}\times \overrightarrow{\mathbf{B}}
\end{equation*}%
This was first written by Lorentz, so it is called the \emph{Lorentz force},
and is usually written as 
\begin{equation*}
\overrightarrow{\mathbf{F}}_{net}=q\left( \overrightarrow{\mathbf{E}}+%
\overrightarrow{\mathbf{v}}\times \overrightarrow{\mathbf{B}}\right)
\end{equation*}%
Using this, let's consider the view point of each frame.

Going back to our two guys on different frames, In frame $S,$ the person
sees 
\begin{eqnarray*}
\overrightarrow{\mathbf{F}} &=&q\left( 0+\overrightarrow{\mathbf{V}}%
_{S^{\prime }S}\times \overrightarrow{\mathbf{B}}\right) =qV_{x}\hat{\imath}%
\times B\left( -\hat{k}\right) \\
&=&qVB\hat{\jmath}
\end{eqnarray*}%
and in frame $S^{\prime }$ the person sees 
\begin{equation*}
\overrightarrow{\mathbf{F}}^{\prime }=q\left( \overrightarrow{\mathbf{E}}%
^{\prime }+0\times \overrightarrow{\mathbf{B}}^{\prime }\right) =q%
\overrightarrow{\mathbf{E}}^{\prime }
\end{equation*}%
It seems that the only way that $\overrightarrow{\mathbf{F}}=\overrightarrow{%
\mathbf{F}}^{\prime }$ is that $\overrightarrow{\mathbf{E}}^{\prime }\neq 0$
in the primed frame! So in frame $S^{\prime }$ our person must conclude that
there is an external electric field that produces the force $\overrightarrow{%
\mathbf{F}}^{\prime }.$ In frame $S$ the person is convinced that the
magnetic field, $\overrightarrow{\mathbf{B}}\mathbf{,}$ is making the force.
In frame $S^{\prime }$ the person is convinced that the electric field $%
\overrightarrow{\mathbf{E}}^{\prime }$ is making the force.

%TCIMACRO{%
%\TeXButton{Question 223.47.4}{\marginpar {
%\hspace{-0.5in}
%\begin{minipage}[t]{1in}
%\small{Question 223.47.4}
%\end{minipage}
%}}}%
%BeginExpansion
\marginpar {
\hspace{-0.5in}
\begin{minipage}[t]{1in}
\small{Question 223.47.4}
\end{minipage}
}%
%EndExpansion
We can find the strength of this electric field by setting the forces equal%
\begin{eqnarray*}
\overrightarrow{\mathbf{F}} &=&\overrightarrow{\mathbf{F}}^{\prime } \\
q\overrightarrow{\mathbf{V}}_{S^{\prime }S}\times \overrightarrow{\mathbf{B}}
&\mathbf{=}&q\overrightarrow{\mathbf{E}}^{\prime }
\end{eqnarray*}%
so%
\begin{equation*}
\overrightarrow{\mathbf{E}}^{\prime }=\overrightarrow{\mathbf{V}}_{S^{\prime
}S}\times \overrightarrow{\mathbf{B}}
\end{equation*}%
and the direction must be 
\begin{equation*}
\overrightarrow{\mathbf{E}}^{\prime }=V_{S^{\prime }S}B\hat{\jmath}
\end{equation*}

%TCIMACRO{%
%\TeXButton{Question 223.47.5}{\marginpar {
%\hspace{-0.5in}
%\begin{minipage}[t]{1in}
%\small{Question 223.47.5}
%\end{minipage}
%}}}%
%BeginExpansion
\marginpar {
\hspace{-0.5in}
\begin{minipage}[t]{1in}
\small{Question 223.47.5}
\end{minipage}
}%
%EndExpansion
%TCIMACRO{%
%\TeXButton{Question 223.47.6}{\marginpar {
%\hspace{-0.5in}
%\begin{minipage}[t]{1in}
%\small{Question 223.47.6}
%\end{minipage}
%}}}%
%BeginExpansion
\marginpar {
\hspace{-0.5in}
\begin{minipage}[t]{1in}
\small{Question 223.47.6}
\end{minipage}
}%
%EndExpansion
Our interpretation of this result is mind-blowing. It seems that whether we
see a magnetic field or an electric field causing the force depends on our
reference frame! The implication is that the electric and magnetic fields
are not really two different things. They are one field viewed from
different reference frames!

Anther way to say what we have found might be that moving magnetic fields
show up as electric fields.

So far we have been talking about external fields only. The field $%
\overrightarrow{\mathbf{B}}$ in our case study is created by some outside
agent. So the field $\overrightarrow{\mathbf{E}}^{\prime }$ observed in
frame $S^{\prime }$ is also an environmental field. But the charge, itself,
creates a field. So the total electric field in frame $S^{\prime }$ is the
environmental field $\overrightarrow{\mathbf{E}}^{\prime }$ plus the field
due to the charge, itself $\overrightarrow{\mathbf{E}}_{\text{self}},$ or 
\begin{eqnarray*}
\overrightarrow{\mathbf{E}}_{tot}^{\prime } &=&\overrightarrow{\mathbf{E}}_{%
\text{self}}+\overrightarrow{\mathbf{E}}_{\text{environment}}^{\prime } \\
&=&\overrightarrow{\mathbf{E}}_{\text{self}}+\overrightarrow{\mathbf{V}}%
_{S^{\prime }S}\times \overrightarrow{\mathbf{B}}_{\text{environment}}
\end{eqnarray*}%
which we usually just write as%
\begin{equation*}
\overrightarrow{\mathbf{E}}^{\prime }=\overrightarrow{\mathbf{E}}_{\text{self%
}}+\overrightarrow{\mathbf{V}}_{S^{\prime }S}\times \overrightarrow{\mathbf{B%
}}_{\text{environment}}
\end{equation*}

We would predict that if we had a charge that is stationary in frame $S$ and
we rode along with frame $S^{\prime }$ that we would see a field 
\begin{equation*}
\overrightarrow{\mathbf{E}}=\mathbf{E}_{\text{self}}^{\prime }-%
\overrightarrow{\mathbf{V}}_{S^{\prime }S}\times \overrightarrow{\mathbf{B}}%
_{\text{environment}}^{\prime }
\end{equation*}%
Of course, $\overrightarrow{\mathbf{E}}_{\text{self}}$ can't create a force
on the charge, because it is a self-field. So we only need to be concerned
with $\overrightarrow{\mathbf{E}}_{\text{self}}$ if we have other charges
that could move. We could actually have a group of charges riding along with
frame $S^{\prime }.$ In that case we would have an additional field $E_{%
\text{charges }}^{\prime }.$ We could write this as 
\begin{equation*}
\overrightarrow{\mathbf{E}}=\mathbf{E}_{\text{charges in }S\prime }^{\prime
}-\overrightarrow{\mathbf{V}}_{S^{\prime }S}\times \overrightarrow{\mathbf{B}%
}_{\text{environment}}^{\prime }
\end{equation*}%
or just 
\begin{equation*}
\overrightarrow{\mathbf{E}}=\mathbf{E}_{\text{charges}}^{\prime }-%
\overrightarrow{\mathbf{V}}_{S^{\prime }S}\times \overrightarrow{\mathbf{B}}%
_{\text{environment}}^{\prime }
\end{equation*}

What we have developed is important! We have an equation that let's us
determine the electric field in a frame, given the fields measured in
another frame.

We would expect that a similar thing would happen if we replaced the
magnetic fields with electric fields. Suppose we have an electric field in
the region of our frames and that this electric field is stationary with
respect to frame $S^{\prime }$ this time. 
%TCIMACRO{%
%\TeXButton{Question 223.47.7}{\marginpar {
%\hspace{-0.5in}
%\begin{minipage}[t]{1in}
%\small{Question 223.47.7}
%\end{minipage}
%}}}%
%BeginExpansion
\marginpar {
\hspace{-0.5in}
\begin{minipage}[t]{1in}
\small{Question 223.47.7}
\end{minipage}
}%
%EndExpansion
Will frame $S$ see a magnetic field?\FRAME{dhF}{3.5125in}{1.9806in}{0pt}{}{}{%
Figure}{\special{language "Scientific Word";type
"GRAPHIC";maintain-aspect-ratio TRUE;display "USEDEF";valid_file "T";width
3.5125in;height 1.9806in;depth 0pt;original-width 5.3476in;original-height
3.0033in;cropleft "0";croptop "1";cropright "1";cropbottom "0";tempfilename
'NBD0L907.wmf';tempfile-properties "XPR";}}

To see that this is true, let's examine the case where we have no external
fields, and we just have a charge moving along with frame $S^{\prime }.$
Then in frame $S^{\prime }$ we have the fields 
\begin{eqnarray*}
\overrightarrow{\mathbf{E}}^{\prime } &=&\frac{1}{4\pi \epsilon _{o}}\frac{q%
}{r^{2}}\mathbf{\hat{r}} \\
\overrightarrow{\mathbf{B}}^{\prime } &=&0
\end{eqnarray*}%
in frame $S$ the electric field is 
\begin{eqnarray*}
\overrightarrow{\mathbf{E}} &=&\overrightarrow{\mathbf{E}}_{\text{charges}%
}^{\prime }-\overrightarrow{\mathbf{V}}_{S^{\prime }S}\times \overrightarrow{%
\mathbf{B}}_{\text{environment}}^{\prime } \\
&=&\frac{1}{4\pi \epsilon _{o}}\frac{q}{r^{2}}\mathbf{\hat{r}}+%
\overrightarrow{\mathbf{V}}_{S^{\prime }S}\times \mathbf{0} \\
&=&\frac{1}{4\pi \epsilon _{o}}\frac{q}{r^{2}}\mathbf{\hat{r}}
\end{eqnarray*}

so 
\begin{equation*}
\overrightarrow{\mathbf{E}}=\overrightarrow{\mathbf{E}}_{\text{ }}^{\prime }=%
\frac{1}{4\pi \epsilon _{o}}\frac{q}{r^{2}}\mathbf{\hat{r}}
\end{equation*}%
We see the same electric field due to the point charge being there in both
frames.

But in frame $S$ we are expecting the person to see a magnetic field because
to person $S$ the charge is moving. Using the Biot-Savart law%
\begin{equation*}
\overrightarrow{\mathbf{B}}=\frac{\mu _{o}}{4\pi }\frac{q\overrightarrow{%
\mathbf{v}}\times \mathbf{\hat{r}}}{r^{2}}
\end{equation*}%
since our charge is moving along with the $S^{\prime }$ frame $%
\overrightarrow{\mathbf{v}}=\overrightarrow{\mathbf{V}}_{S^{\prime }S}$ so 
\begin{equation*}
\overrightarrow{\mathbf{B}}=\frac{\mu _{o}}{4\pi }\frac{q}{r^{2}}\left( 
\overrightarrow{\mathbf{V}}_{S^{\prime }S}\times \mathbf{\hat{r}}\right)
\end{equation*}%
but we can rewrite this by rearranging terms%
\begin{eqnarray*}
\overrightarrow{\mathbf{B}} &=&\frac{\mu _{o}}{4\pi }\frac{q}{r^{2}}\left( 
\overrightarrow{\mathbf{V}}_{S^{\prime }S}\times \mathbf{\hat{r}}\right) \\
&=&\left( \overrightarrow{\mathbf{V}}_{S^{\prime }S}\times \frac{\mu _{o}}{%
4\pi }\frac{q}{r^{2}}\mathbf{\hat{r}}\right)
\end{eqnarray*}%
which looks vaguely familiar. Let's multiply top and bottom by $\epsilon
_{o} $%
\begin{eqnarray*}
\overrightarrow{\mathbf{B}} &=&\left( \overrightarrow{\mathbf{V}}_{S^{\prime
}S}\times \frac{\mu _{o}\epsilon _{o}}{4\pi \epsilon _{o}}\frac{q}{r^{2}}%
\mathbf{\hat{r}}\right) \\
&=&\left( \overrightarrow{\mathbf{V}}_{S^{\prime }S}\times \mu _{o}\epsilon
_{o}\left( \frac{1}{4\pi \epsilon _{o}}\frac{q}{r^{2}}\mathbf{\hat{r}}%
\right) \right) \\
&=&\left( \overrightarrow{\mathbf{V}}_{S^{\prime }S}\times \mu _{o}\epsilon
_{o}\left( \overrightarrow{\mathbf{E}}_{\text{ }}^{\prime }\right) \right) \\
&=&\mu _{o}\epsilon _{o}\left( \overrightarrow{\mathbf{V}}_{S^{\prime
}S}\times \overrightarrow{\mathbf{E}}_{\text{ }}^{\prime }\right)
\end{eqnarray*}%
which is really quite astounding! Our $B$-fields have apparently always just
been due to moving electric fields after all! Of course, we could have an
additional magnet riding along with frame $S^{\prime }.$ To allow for that
case, let's include a term $\overrightarrow{\mathbf{B}}_{\text{magnet}%
}^{\prime }$. 
\begin{equation*}
\overrightarrow{\mathbf{B}}_{total}=\overrightarrow{\mathbf{B}}_{\text{%
magnets in }S^{\prime }}^{\prime }+\mu _{o}\epsilon _{o}\left( 
\overrightarrow{\mathbf{V}}_{S^{\prime }S}\times \overrightarrow{\mathbf{E}}%
_{\text{environment }}^{\prime }\right)
\end{equation*}%
or just 
\begin{equation*}
\overrightarrow{\mathbf{B}}=\overrightarrow{\mathbf{B}}_{\text{magnets}%
}^{\prime }+\mu _{o}\epsilon _{o}\left( \overrightarrow{\mathbf{V}}%
_{S^{\prime }S}\times \overrightarrow{\mathbf{E}}_{\text{environment }%
}^{\prime }\right)
\end{equation*}%
and we would expect that if we worked this problem from the other frame's
point of view we would likewise find 
\begin{equation*}
\overrightarrow{\mathbf{B}}^{\prime }=\overrightarrow{\mathbf{B}}_{\text{%
magnet}}-\mu _{o}\epsilon _{o}\left( \overrightarrow{\mathbf{V}}_{S^{\prime
}S}\times \overrightarrow{\mathbf{E}}_{\text{environment}}\right)
\end{equation*}%
where the minus sign comes from the relative velocity being in the other
direction. Again $\overrightarrow{\mathbf{B}}_{\text{magnet}}$ is a
self-field. It won't move the magnet creating it, but it might be important
if we have a second magnet in our experiment. Then $\overrightarrow{\mathbf{B%
}}_{\text{magnet}}$ would cause a force on this second magnet.

Once again we have found a way to find a field, the magnetic field this
time, in one frame if we know the fields on another frame! We call this sort
of equation a \emph{transformation.}

We should take a moment to look at the constants $\mu _{o}\epsilon _{o}.$
Let's put in their values%
\begin{eqnarray*}
\mu _{o}\epsilon _{o} &=&\left( 8.85\times 10^{-12}\frac{\unit{C}^{2}}{\unit{%
N}\unit{m}^{2}}\right) \left( 4\pi \times 10^{-7}\frac{\unit{T}\unit{m}}{%
\unit{A}}\right) \\
&=&1.\,\allowbreak 112\,1\times 10^{-17}\frac{\unit{s}^{2}}{\unit{m}^{2}}
\end{eqnarray*}%
This is a very small number, and it may not appear to be interesting. We can
see that the additional magnetic fields due to the movement of the charges
can be quite small unless the electric field is large or the relative speed
is large (or both). So much of the time this additional field due to the
moving charge is negligible. But let's calculate%
\begin{eqnarray*}
\frac{1}{\sqrt{\mu _{o}\epsilon _{o}}} &=&\frac{1}{\sqrt{\left( 8.85\times
10^{-12}\frac{\unit{C}^{2}}{\unit{N}\unit{m}^{2}}\right) \left( 4\pi \times
10^{-7}\frac{\unit{T}\unit{m}}{\unit{A}}\right) }} \\
&=&2.\,\allowbreak 998\,6\times 10^{8}\allowbreak \frac{\unit{m}}{\unit{s}}
\\
&=&c
\end{eqnarray*}%
This is the speed of light! It even has units of $\unit{m}/\unit{s}.$ This
seems an amazing coincidence--too amazing. And this was one of the clues
that Maxwell used to discover that light is a wave in what we will now call
the \emph{electromagnetic field} (because they are different aspects of one
thing).

We can write the transformation equations for the fields as 
\begin{eqnarray*}
\overrightarrow{\mathbf{E}}^{\prime } &=&\overrightarrow{\mathbf{E}}_{\text{%
charges}}+\overrightarrow{\mathbf{V}}_{S^{\prime }S}\times \overrightarrow{%
\mathbf{B}}_{\text{environment}} \\
\overrightarrow{\mathbf{B}}^{\prime } &=&\overrightarrow{\mathbf{B}}_{\text{%
magnet}}-\frac{1}{c^{2}}\left( \overrightarrow{\mathbf{V}}_{S^{\prime
}S}\times \overrightarrow{\mathbf{E}}_{\text{environment}}\right) \\
\overrightarrow{\mathbf{E}} &=&\mathbf{E}_{\text{charges}}^{\prime }-%
\overrightarrow{\mathbf{V}}_{S^{\prime }S}\times \overrightarrow{\mathbf{B}}%
_{\text{environment}}^{\prime } \\
\overrightarrow{\mathbf{B}} &=&\overrightarrow{\mathbf{B}}_{\text{magnet}%
}^{\prime }+\frac{1}{c^{2}}\left( \overrightarrow{\mathbf{V}}_{S^{\prime
}S}\times \overrightarrow{\mathbf{E}}_{\text{ environment}}^{\prime }\right)
\end{eqnarray*}

Let's do a problem. Suppose we have a metal loop moving into an area where
there is a magnetic field as shown. Let's show that there is a force on
charges in this loop no matter what frame we consider. First, lets consider
the frame where the magnetic field is stationary and the loop moves.\FRAME{%
dtbpF}{2.9297in}{1.664in}{0pt}{}{}{Figure}{\special{language "Scientific
Word";type "GRAPHIC";maintain-aspect-ratio TRUE;display "USEDEF";valid_file
"T";width 2.9297in;height 1.664in;depth 0pt;original-width
2.9572in;original-height 1.6666in;cropleft "0";croptop "1";cropright
"1";cropbottom "0";tempfilename 'NBD0L908.wmf';tempfile-properties "XPR";}}%
There should be an upward force on the positive charge because the charge is
moving in a magnetic field. Let's say that \textquotedblleft
up\textquotedblright\ is the $\hat{\jmath}$ direction and that
\textquotedblleft to the right\textquotedblright\ is the $\hat{\imath}$
direction. Then The Lorentz force is 
\begin{eqnarray*}
\overrightarrow{\mathbf{F}} &=&q\left( \overrightarrow{\mathbf{E}}+%
\overrightarrow{\mathbf{v}}\times \overrightarrow{\mathbf{B}}\right) \\
&=&q\left( \overrightarrow{\mathbf{E}}+\overrightarrow{\mathbf{V}}%
_{S^{\prime }S}\times \overrightarrow{\mathbf{B}}\right)
\end{eqnarray*}%
Now $\overrightarrow{\mathbf{V}}_{S^{\prime }S}$ means the speed of the
reference frame $S^{\prime }$ with respect to frame $S.$ That is $+V\hat{%
\imath}.$ And there is no electric field in frame $S,$ so 
\begin{eqnarray*}
\overrightarrow{\mathbf{F}} &=&q\left( \overrightarrow{\mathbf{E}}+%
\overrightarrow{\mathbf{V}}_{S^{\prime }S}\times \overrightarrow{\mathbf{B}}%
\right) \\
&=&q\left( 0+V\hat{\imath}\times B\left( -\hat{k}\right) \right) \\
&=&q\left( V\hat{\imath}\times \mathbf{B}\left( -\hat{k}\right) \right) \\
&=&qVB\hat{\jmath}
\end{eqnarray*}%
Now suppose we change reference frames so we are riding along with the loop
in frame, $S^{\prime }.$ In this frame, the loop is not moving, and the
magnetic field is moving by us the opposite direction. We'll call this the
\textquotedblleft prime frame.\textquotedblright\ We should get the same
force if we change frames to ride along with the loop. \FRAME{dtbpF}{3.9763in%
}{1.6436in}{0pt}{}{}{Figure}{\special{language "Scientific Word";type
"GRAPHIC";maintain-aspect-ratio TRUE;display "USEDEF";valid_file "T";width
3.9763in;height 1.6436in;depth 0pt;original-width 4.0242in;original-height
1.6462in;cropleft "0";croptop "1";cropright "1";cropbottom "0";tempfilename
'NBD0L909.wmf';tempfile-properties "XPR";}} Let's use our transformations to
find the $E$ and $B$-fields in the new reference frame. Then 
\begin{eqnarray*}
\overrightarrow{\mathbf{E}}^{\prime } &=&\overrightarrow{\mathbf{E}}_{\text{%
self-charge}}+\overrightarrow{\mathbf{V}}_{S^{\prime }S}\times 
\overrightarrow{\mathbf{B}}_{\text{environment}} \\
\overrightarrow{\mathbf{B}}^{\prime } &=&\overrightarrow{\mathbf{B}}_{\text{%
magnet}}-\frac{1}{c^{2}}\left( \overrightarrow{\mathbf{V}}_{S^{\prime
}S}\times \overrightarrow{\mathbf{E}}_{\text{environment}}\right)
\end{eqnarray*}%
so in the prime frame we have an electric field%
\begin{equation*}
\overrightarrow{\mathbf{E}}^{\prime }=\overrightarrow{\mathbf{E}}_{\text{%
charges}}+\overrightarrow{\mathbf{V}}_{S^{\prime }S}\times \overrightarrow{%
\mathbf{B}}_{\text{environment}}
\end{equation*}%
and in particular, we have an external field%
\begin{equation*}
\overrightarrow{\mathbf{E}}_{\text{environment}}^{\prime }=\overrightarrow{%
\mathbf{V}}_{S^{\prime }S}\times \overrightarrow{\mathbf{B}}_{\text{%
environment}}
\end{equation*}%
(we left off the $\overrightarrow{\mathbf{E}}_{\text{charge}}$ because it
can't move the charge that made it, so it is not part of the force).

Note that $\overrightarrow{\mathbf{V}}_{S^{\prime }S}$ is the speed of the
primed frame as viewed from the unprimed frame. So $\overrightarrow{\mathbf{V%
}}_{S^{\prime }S}=+V\hat{\imath}$ 
\begin{eqnarray*}
\overrightarrow{\mathbf{E}}^{\prime } &=&V\left( \hat{\imath}\right) \times
B\left( -\hat{k}\right) \\
&=&VB\hat{\jmath}
\end{eqnarray*}%
That is our electric field in the primed frame.

The magnetic field in the primed frame is given by 
\begin{equation*}
\overrightarrow{\mathbf{B}}^{\prime }=\overrightarrow{\mathbf{B}}_{\text{%
magnet}}-\frac{1}{c^{2}}\left( \overrightarrow{\mathbf{V}}_{S^{\prime
}S}\times \overrightarrow{\mathbf{E}}_{\text{environment}}\right)
\end{equation*}%
but there is no external electric field in the unprimed frame, so 
\begin{eqnarray*}
\overrightarrow{\mathbf{B}}^{\prime } &=&\overrightarrow{\mathbf{B}}_{\text{%
magnet}}-\frac{1}{c^{2}}\left( \overrightarrow{\mathbf{V}}_{S^{\prime
}S}\times 0\right) \\
&=&\overrightarrow{\mathbf{B}}_{\text{magnet}}
\end{eqnarray*}%
where here \textquotedblleft magnet\textquotedblright\ means what ever is
making the magnetic field in the unprimed frame. Something must be there
making the field, and it is not our charge. It could be an electromagnet, or
a permanent magnet, we have not been told. But it is not our charge, so we
know $\overrightarrow{\mathbf{B}}_{\text{magnet}}$ must be there and can act
on our charge. So 
\begin{equation*}
\overrightarrow{\mathbf{B}}^{\prime }=\overrightarrow{\mathbf{B}}
\end{equation*}%
The magnetic field in the primed frame is just the same as the magnetic
field we see in the unprimed frame. Then in the primed frame the Lorentz
force is 
\begin{eqnarray*}
\overrightarrow{\mathbf{F}}^{\prime } &=&q\left( \overrightarrow{\mathbf{E}}%
^{\prime }+\overrightarrow{\mathbf{v}}\times \overrightarrow{\mathbf{B}}%
^{\prime }\right) \\
&=&q\left( VB\hat{\jmath}+\mathbf{0}\times \overrightarrow{\mathbf{B}}\right)
\\
&=&qVB\hat{\jmath}
\end{eqnarray*}%
Which is exactly the same force (magnitude and direction) as we got in the
unprimed frame.

\section{Field Laws}

A \textquotedblleft law\textquotedblright\ in physics is a mathematical
statement of a physical principal or theory. We have been collecting laws
for what we will now call the \emph{electromagnetic field theory}. Let's
review:

\subsection{Gauss' law}

\FRAME{dhF}{2.8245in}{2.4249in}{0pt}{}{}{Figure}{\special{language
"Scientific Word";type "GRAPHIC";maintain-aspect-ratio TRUE;display
"USEDEF";valid_file "T";width 2.8245in;height 2.4249in;depth
0pt;original-width 3.6357in;original-height 3.1169in;cropleft "0";croptop
"1";cropright "1";cropbottom "0";tempfilename
'NBD0L90A.wmf';tempfile-properties "XPR";}}

We found that the electric flux through an imaginary closed surface that
incloses some charge is%
\begin{equation*}
\Phi _{E}=\oint \mathbf{E}\cdot d\mathbf{A}=\frac{Q_{in}}{\epsilon _{o}}
\end{equation*}%
We called this Gauss' law.

But consider the situation with a magnet.%
%TCIMACRO{%
%\TeXButton{Question 223.47.8}{\marginpar {
%\hspace{-0.5in}
%\begin{minipage}[t]{1in}
%\small{Question 223.47.8}
%\end{minipage}
%}} }%
%BeginExpansion
\marginpar {
\hspace{-0.5in}
\begin{minipage}[t]{1in}
\small{Question 223.47.8}
\end{minipage}
}
%EndExpansion
We can define a magnetic flux just like we defined the electric flux. And
now we know they must be related. Is there a Gauss' law for magnetism? Let's
consider the magnetic flux. 
\begin{equation*}
\Phi _{B}=\oint \mathbf{B}\cdot d\mathbf{A}
\end{equation*}%
This should be proportional to the number of \textquotedblleft magnetic
charges\textquotedblright\ inclosed in the surface.\ \FRAME{dhF}{3.3501in}{%
2.4632in}{0pt}{}{}{Figure}{\special{language "Scientific Word";type
"GRAPHIC";maintain-aspect-ratio TRUE;display "USEDEF";valid_file "T";width
3.3501in;height 2.4632in;depth 0pt;original-width 9.6397in;original-height
7.0746in;cropleft "0";croptop "1";cropright "1";cropbottom "0";tempfilename
'NBD0L90B.wmf';tempfile-properties "XPR";}}We can see that every field line
that leaves comes back in. That is how we defined zero net flux, so 
\begin{equation*}
\Phi _{B}=\oint \mathbf{B}\cdot d\mathbf{A}=0
\end{equation*}%
Which would tell us that there are no free \textquotedblleft magnetic
charges\textquotedblright\ or no single magnetic poles. A single magnetic
pole is called a \emph{monopole} and indeed we have never discovered one.
These two forms of Gauss' law form the first two of our electromagnetic
field equations.

The differences between them have to do with the fact that magnetic fields
are due to moving charges.

\FRAME{dhF}{1.9277in}{1.5757in}{0pt}{}{}{Figure}{\special{language
"Scientific Word";type "GRAPHIC";maintain-aspect-ratio TRUE;display
"USEDEF";valid_file "T";width 1.9277in;height 1.5757in;depth
0pt;original-width 2.7086in;original-height 2.2087in;cropleft "0";croptop
"1";cropright "1";cropbottom "0";tempfilename
'Lecture_Notes/MyAmperesLaw0.wmf';tempfile-properties "XNPR";}}

We have a third electromagnetic field law, Ampere's law. We found Ampere's
law by integrating around a closed loop with a current penetrating the loop. 
\begin{equation*}
\oint \mathbf{B}\cdot d\mathbf{s}=\mu _{o}I_{through}
\end{equation*}

We also know Faraday's law%
\begin{equation*}
\mathcal{E}=\oint \mathbf{E}\cdot d\mathbf{s}=-\frac{d\Phi _{B}}{dt}
\end{equation*}%
which told us that changing magnetic fields created an electric field. We
have found that the opposite must be true, that a changing electric field
must create a magnetic field. We express this as 
\begin{equation*}
\oint \mathbf{B}\cdot d\mathbf{s}\propto \frac{d\Phi _{E}}{dt}
\end{equation*}%
Which gives two expressions for $\oint \mathbf{B}\cdot d\mathbf{s}$. But we
have yet to show that this equation is true. That is the subject of our next
lecture. If we can accomplish this, we will have a complete set of field
equations that describe how the electromagnetic field works. In the
following lecture we will complete the set of field equations, and then in
the next lecture we will show that we get electromagnetic waves from these
equations.

%TCIMACRO{%
%\TeXButton{Basic Equations}{\hspace{-1.3in}{\LARGE Basic Equations\vspace{0.25in}}}}%
%BeginExpansion
\hspace{-1.3in}{\LARGE Basic Equations\vspace{0.25in}}%
%EndExpansion

Rules for finding fields in different coordinate systems%
\begin{eqnarray*}
\overrightarrow{\mathbf{E}}^{\prime } &=&\overrightarrow{\mathbf{E}}_{\text{%
charges}}+\overrightarrow{\mathbf{V}}_{S^{\prime }S}\times \overrightarrow{%
\mathbf{B}}_{\text{environment}} \\
\overrightarrow{\mathbf{B}}^{\prime } &=&\overrightarrow{\mathbf{B}}_{\text{%
magnet}}-\frac{1}{c^{2}}\left( \overrightarrow{\mathbf{V}}_{S^{\prime
}S}\times \overrightarrow{\mathbf{E}}_{\text{environment}}\right) \\
\overrightarrow{\mathbf{E}} &=&\mathbf{E}_{\text{charges}}^{\prime }-%
\overrightarrow{\mathbf{V}}_{S^{\prime }S}\times \overrightarrow{\mathbf{B}}%
_{\text{environment}}^{\prime } \\
\overrightarrow{\mathbf{B}} &=&\overrightarrow{\mathbf{B}}_{\text{magnet}%
}^{\prime }+\frac{1}{c^{2}}\left( \overrightarrow{\mathbf{V}}_{S^{\prime
}S}\times \overrightarrow{\mathbf{E}}_{\text{ environment}}^{\prime }\right)
\end{eqnarray*}%
Gauss' law for magnetic fields%
\begin{equation*}
\Phi _{B}=\oint \mathbf{B}\times d\mathbf{A}=0
\end{equation*}

\chapter{Field Equations and Waves in the Field}

We started this class with a study of waves. We learned about optics, and
finally electromagnetic field theory. In this lecture we will take on a case
study that involves all three. We will have come full circle and in the
process, hopefully understand all three topics a little better.

%TCIMACRO{%
%\TeXButton{Fundamental Concepts}{\hspace{-1.3in}{\LARGE Fundamental Concepts\vspace{0.25in}}}}%
%BeginExpansion
\hspace{-1.3in}{\LARGE Fundamental Concepts\vspace{0.25in}}%
%EndExpansion

\begin{itemize}
\item Changing electric fields produce magnetic fields

\item A changing electric flux is described as a displacement current $%
I_{d}=\varepsilon _{o}\frac{d\Phi _{E}}{dt}$

\item The complete version of Ampere's law is $\doint \mathbf{B}\cdot d\ell
=\mu _{o}\left( I+I_{d}\right) $

\item Maxwell's equations give a complete classical picture of
electromagnetic fields

\item Maxwell's equations plus the Lorentz force describe all of
electrodynamics.
\end{itemize}

\section{Displacement Current}

Last time we listed Ampere's law as one of the basic field equations. But we
did not discuss it at all. That is because we were saving it for our
discussion in this lecture We need to look deeply into Ampere's law. Here is
what we have for Ampere's law so far%
\begin{equation*}
\oint \mathbf{B}\cdot d\mathbf{s}=\mu _{o}I_{through}
\end{equation*}%
To see why we need to consider it further, let's do a hard problem with
Ampere's law. Let's set up a circuit with a battery a switch and a circular
plate capacitor in the wire. \FRAME{dtbpF}{1.4918in}{0.9608in}{0pt}{}{}{%
Figure}{\special{language "Scientific Word";type
"GRAPHIC";maintain-aspect-ratio TRUE;display "USEDEF";valid_file "T";width
1.4918in;height 0.9608in;depth 0pt;original-width 1.4563in;original-height
0.9279in;cropleft "0";croptop "1";cropright "1";cropbottom "0";tempfilename
'MEJEQQ00.wmf';tempfile-properties "XPR";}}Using this circuit, let's
calculate the magnetic field using Ampere's law. Here is a detailed diagram
of the capacitor.\FRAME{dtbpF}{1.4728in}{0.7697in}{0pt}{}{}{Figure}{\special%
{language "Scientific Word";type "GRAPHIC";maintain-aspect-ratio
TRUE;display "USEDEF";valid_file "T";width 1.4728in;height 0.7697in;depth
0pt;original-width 3.7402in;original-height 1.9465in;cropleft "0";croptop
"1";cropright "1";cropbottom "0";tempfilename
'LVJWAX07.wmf';tempfile-properties "XPR";}}I could find the magnetic field
using the Biot-Savart equation, but that would be hard. I don't know how to
solve the resulting integral. So let's try Ampere's law. Let's start at $%
P_{1}.$ We add in an imaginary surface at $P_{1}.$ I will choose a simple
circular surface.\FRAME{dtbpF}{1.7815in}{1.1243in}{0pt}{}{}{Figure}{\special%
{language "Scientific Word";type "GRAPHIC";maintain-aspect-ratio
TRUE;display "USEDEF";valid_file "T";width 1.7815in;height 1.1243in;depth
0pt;original-width 3.9157in;original-height 2.4612in;cropleft "0";croptop
"1";cropright "1";cropbottom "0";tempfilename
'LVJWAX08.wmf';tempfile-properties "XPR";}}We have done this before. If we
choose $P_{1}$ so that it is far from the capacitor, then we know what the
magnetic field will look like.\FRAME{dtbpF}{2.2778in}{1.5176in}{0pt}{}{}{%
Figure}{\special{language "Scientific Word";type
"GRAPHIC";maintain-aspect-ratio TRUE;display "USEDEF";valid_file "T";width
2.2778in;height 1.5176in;depth 0pt;original-width 2.2928in;original-height
1.5176in;cropleft "0";croptop "1";cropright "1";cropbottom "0";tempfilename
'MKMU5U00.wmf';tempfile-properties "XPR";}}Right at $P_{1}$ it will be out
of the page. We also know that for a long straight wire, the field magnitude
does not change as we go around the wire, so we can write our integral as 
\begin{equation*}
\doint \mathbf{B}\cdot d\ell =B\doint d\ell =B2\pi r=\mu _{o}I
\end{equation*}%
so 
\begin{equation*}
B2\pi r=\mu _{o}I
\end{equation*}%
so the field is 
\begin{equation*}
B=\frac{\mu _{o}I}{2\pi r}
\end{equation*}%
which is very familiar, just the equation for a field from a long straight
wire.

%TCIMACRO{%
%\TeXButton{Question 223.48.1}{\marginpar {
%\hspace{-0.5in}
%\begin{minipage}[t]{1in}
%\small{Question 223.48.1}
%\end{minipage}
%}}}%
%BeginExpansion
\marginpar {
\hspace{-0.5in}
\begin{minipage}[t]{1in}
\small{Question 223.48.1}
\end{minipage}
}%
%EndExpansion
Now Let's try this at $P_{2}.$ What would we expect? Will the magnetic field
change much as we pass by the capacitor?\FRAME{dtbpF}{2.0686in}{1.3344in}{0pt%
}{}{}{Figure}{\special{language "Scientific Word";type
"GRAPHIC";maintain-aspect-ratio TRUE;display "USEDEF";valid_file "T";width
2.0686in;height 1.3344in;depth 0pt;original-width 3.9157in;original-height
2.5151in;cropleft "0";croptop "1";cropright "1";cropbottom "0";tempfilename
'LVJWAX09.wmf';tempfile-properties "XPR";}}Again we could use Biot-Savart,
but think about what the current does at the plate. It would be very hard to
do the integration!. So again let's try Ampere's law. If we use the same
size surface 
\begin{equation*}
\doint \mathbf{B}\cdot d\ell =B\doint d\ell =B2\pi r
\end{equation*}%
but this is equal to $\mu _{o}I_{through}.$ There is no $I$ going through
the capacitor! so 
\begin{equation}
B2\pi r=0
\end{equation}%
and this would give $B=0.$ But, our wires are not really ideal and
infinitely long. And even if they were, would we really expect the field to
be zero if we just have a small gap in our capacitor? It get's even worse!%
\FRAME{dtbpF}{2.4258in}{1.4641in}{0pt}{}{}{Figure}{\special{language
"Scientific Word";type "GRAPHIC";maintain-aspect-ratio TRUE;display
"USEDEF";valid_file "T";width 2.4258in;height 1.4641in;depth
0pt;original-width 4.6987in;original-height 2.8273in;cropleft "0";croptop
"1";cropright "1";cropbottom "0";tempfilename
'LVJWAX0A.wmf';tempfile-properties "XPR";}}Ampere's law tells us we need a
surface, but it does not say it has to be a circular surface. In fact, we
could use the strange surface labeled $S_{2}$ in the figure above. This is a
perfectly good surface to associate with the loop at $P_{1}.$ So this gives
us 
\begin{equation*}
\doint \mathbf{B}\cdot d\ell =\mu _{o}I=0
\end{equation*}%
at $P_{1}$! So we have two different results with Ampere's law for the same
point. This can't be!

%TCIMACRO{%
%\TeXButton{Question 223.48.2}{\marginpar {
%\hspace{-0.5in}
%\begin{minipage}[t]{1in}
%\small{Question 223.48.2}
%\end{minipage}
%}}}%
%BeginExpansion
\marginpar {
\hspace{-0.5in}
\begin{minipage}[t]{1in}
\small{Question 223.48.2}
\end{minipage}
}%
%EndExpansion
Ampere knew this was a problem, but did not find a solution. Maxwell solved
this. He asked himself, what was different inside the capacitor that might
be making a difference. Of course, there is an electric field inside the
capacitor!\FRAME{dtbpF}{2.1326in}{1.3621in}{0pt}{}{}{Figure}{\special%
{language "Scientific Word";type "GRAPHIC";maintain-aspect-ratio
TRUE;display "USEDEF";valid_file "T";width 2.1326in;height 1.3621in;depth
0pt;original-width 3.7402in;original-height 2.3785in;cropleft "0";croptop
"1";cropright "1";cropbottom "0";tempfilename
'LVJWAX0B.wmf';tempfile-properties "XPR";}}We know that in the limit that
the plates can be considered to be very big the field is approximately 
\begin{equation*}
E=\frac{\eta }{\varepsilon _{o}}=\frac{Q}{\pi R^{2}\varepsilon _{o}}
\end{equation*}%
but we know that the charge is changing in time once the switch is thrown.
We can find the rate of change of the field, then%
\begin{equation*}
\frac{dE}{dt}=\frac{1}{\pi R^{2}\varepsilon _{o}}\frac{dQ}{dt}
\end{equation*}

By definition%
\begin{equation*}
I=\frac{dQ}{dt}
\end{equation*}%
is a current, but what current? It must be the current that is supplying the
charge to the capacitor. That current is what is changing the $Q$ in the
capacitor, and it is the $Q$ separation that is making the field. So the
time derivative of the electric field is 
\begin{equation*}
\frac{dE}{dt}=\frac{I}{\pi R^{2}\varepsilon _{o}}
\end{equation*}%
where $I$ is the current in the wire, and only if the wire current is zero
will there be no change in the electric field.

This gives us an idea. A changing electric field creates a magnetic field.
Suppose this changing electric field created a magnetic field like the
current does? It would as though there were a current with a value%
\begin{equation}
I_{d}=\pi R^{2}\varepsilon _{o}\frac{dE}{dt}
\end{equation}%
It doesn't really cause a current in the capacitor. What really happens is
that the changing electric field is creating a magnetic field. But that
magnetic field is just like the field that a current would create. So we can
(somewhat incorrectly) say that the changing electric field has created
something like a current in the capacitor. But no charge is crossing the
capacitor.

Note that in this we have the area of the plate, $A_{plate}=\pi R^{2}$
multiplied by the time rate of change of the electric field. Also note, that
in our approximation for our capacitor, there is only an electric field
inside the plates. So, remembering electric flux,%
\begin{equation*}
\Phi _{E}=\int \mathbf{E}\cdot d\mathbf{A}
\end{equation*}%
our flux though the surface at $P_{2}$ would be%
\begin{eqnarray*}
\Phi _{E} &=&EA \\
&=&\pi R^{2}E
\end{eqnarray*}%
so we can identify 
\begin{equation*}
\pi R^{2}dE=A_{plate}dE=d\Phi _{E}
\end{equation*}%
as a small amount of \emph{electric} flux. Then our equivalent current will
be%
\begin{equation}
I_{d}=\varepsilon _{o}\frac{d\Phi _{E}}{dt}
\end{equation}%
Maxwell decided that, since this looked like equivalent to a current, he
would call it a current and include it in Ampere's law.%
\begin{eqnarray*}
\doint \mathbf{B}\cdot d\ell &=&\mu _{o}\left( I+I_{d}\right) \\
&=&\mu _{o}\left( I+\varepsilon _{o}\frac{d\Phi _{E}}{dt}\right)
\end{eqnarray*}%
but remember it is not really a current. What we have is a changing electric
field that is making a magnetic field \emph{as though there were a current }$%
I_{d}.$ 
%TCIMACRO{%
%\TeXButton{Question 223.48.3}{\marginpar {
%\hspace{-0.5in}
%\begin{minipage}[t]{1in}
%\small{Question 223.48.3}
%\end{minipage}
%}}}%
%BeginExpansion
\marginpar {
\hspace{-0.5in}
\begin{minipage}[t]{1in}
\small{Question 223.48.3}
\end{minipage}
}%
%EndExpansion
We can try this on or capacitor problem. We have done our capacitor problem
for $S_{1}$ where we expect $\frac{d\Phi _{E}}{dt}\approx 0$ so our original
calculation stands 
\begin{equation*}
B_{S_{1}}=\frac{\mu _{o}I}{2\pi r}
\end{equation*}%
but now we know that if we use $S_{2}$ we have $\frac{d\Phi _{E}}{dt}\neq 0$%
, and we realize that at $P_{2}$ the current $I=0$ so 
\begin{equation*}
\doint \mathbf{B}\cdot d\ell =\mu _{o}\left( 0+\varepsilon _{o}\frac{d\Phi
_{E}}{dt}\right)
\end{equation*}%
and for our geometry we found$\frac{d\Phi _{E}}{dt}$ 
\begin{equation*}
\doint \mathbf{B}\cdot d\ell =\mu _{o}\left( 0+\pi R^{2}\varepsilon _{o}%
\frac{dE}{dt}\right)
\end{equation*}%
and we calculated $\frac{dE}{dt}$ so we can substitute it in%
\begin{equation*}
\doint \mathbf{B}\cdot d\ell =\mu _{o}\left( 0+\pi R^{2}\varepsilon _{o}%
\frac{I}{\pi R^{2}\varepsilon _{o}}\right)
\end{equation*}%
where we remember that the current $I$ is the current making the electric
field--the current in the wire. Then we have 
\begin{equation*}
B2\pi r=\mu _{o}\left( 0+I\right)
\end{equation*}%
and our field is%
\begin{equation*}
B=\frac{\mu _{o}I}{2\pi r}
\end{equation*}%
which is just what we found using $S_{1}.$ Maxwell seems to have saved the
day! There is no dip in the magnetic field magnitude.

There is one more fix we will have to do to Ampere's law eventually. We
found this form of Ampere's law with the capacitor empty--not even
containing air. But we could do the same derivation with a dielectric filled
capacitor. We also could have magnetic materials involved.

But what we have done so far is really a momentous result. We have shown
that, indeed, we should have an equation that provides symmetry with
Faraday's law. We suspected that%
\begin{equation*}
\oint \mathbf{B}\cdot d\mathbf{s}\propto \frac{d\Phi _{E}}{dt}
\end{equation*}%
and we can write the constants of proportionality as 
\begin{equation*}
\oint \mathbf{B}\cdot d\mathbf{s}=\mu _{o}\epsilon _{o}\frac{d\Phi _{E}}{dt}
\end{equation*}%
but because we have $\oint \mathbf{B}\cdot d\mathbf{s}$ also in Ampere's
law, we can combine the two to yield 
\begin{eqnarray*}
\doint \mathbf{B}\cdot d\mathbf{s} &=&\mu _{o}\left( I+I_{d}\right) \\
&=&\mu _{o}\left( I+\varepsilon _{o}\frac{d\Phi _{E}}{dt}\right)
\end{eqnarray*}

This is the last of our field equations. It is called the Maxwell-Ampere law.

Let's use this to solve for the magnetic field inside the capacitor. A
changing electric field will make a magnetic field.

Take a surface inside the plates that is a circle of radius $r<R.$ Then%
\begin{equation*}
\doint \mathbf{B}\cdot d\mathbf{s=}B2\pi r
\end{equation*}%
and from our modified Ampere's equation%
\begin{eqnarray*}
\doint \mathbf{B}\cdot d\mathbf{s} &=&\mu _{o}\left( I+I_{d}\right) \\
&=&\mu _{o}\left( I+\varepsilon _{o}\frac{d\Phi _{E}}{dt}\right)
\end{eqnarray*}%
so 
\begin{eqnarray*}
B2\pi r &=&\mu _{o}\left( 0+\varepsilon _{o}\frac{d\Phi _{E}}{dt}\right) \\
&=&\pi r^{2}\mu _{o}\varepsilon _{o}\frac{dE}{dt} \\
&=&\pi r^{2}\mu _{o}\varepsilon _{o}\frac{I}{\pi R^{2}\varepsilon _{o}} \\
&=&\mu _{o}\frac{r^{2}I}{R^{2}}
\end{eqnarray*}%
so%
\begin{equation}
B=\mu _{o}\frac{rI}{2\pi R^{2}}
\end{equation}

We should pause to realize what we have just done. We have shown that,
indeed, a changing electric field can produce a magnetic field. This
statement is a profound look at the way the universe works!

\section{Maxwell Equations}

We have developed a powerful set of understanding equations for electricity
and magnetism. Maxwell summarized our knowledge in a series of four equations%
\begin{equation}
\begin{tabular}{ll}
$\doint \mathbf{E}\cdot d\mathbf{A=}\frac{Q_{in}}{\varepsilon _{o}}$ & 
Gauss's law for electric fields \\ 
$\doint \mathbf{B}\cdot d\mathbf{A}=0$ & Gauss's law for magnetic fields \\ 
$\doint \mathbf{E}\cdot d\mathbf{s=}-\frac{d\Phi _{B}}{dt}$ & Faraday's law
\\ 
$\doint \mathbf{B}\cdot d\mathbf{s=\mu }_{o}I+\varepsilon _{o}\mu _{o}\frac{%
d\Phi _{E}}{dt}$ & Ampere-Maxwell Law%
\end{tabular}%
\end{equation}%
If we have a dielectric, we might see these written as\cite{lewin2002}

\begin{equation}
\begin{tabular}{ll}
$\doint \mathbf{E}\cdot d\mathbf{A=}\frac{Q_{in}}{\varepsilon _{o}\kappa }$
& Gauss's law for electric fields \\ 
$\doint \mathbf{B}\cdot d\mathbf{A=}0$ & Gauss's law for magnetic fields \\ 
$\doint \mathbf{E}\cdot d\mathbf{s=}-\frac{d\Phi _{B}}{dt}$ & Faraday's law
\\ 
$\doint \mathbf{B}\cdot d\mathbf{s=}\mu _{o}\kappa _{m}\left( I+\varepsilon
_{o}\kappa \frac{d\Phi _{E}}{dt}\right) $ & Ampere-Maxwell Law%
\end{tabular}%
\end{equation}

Since we have all had multivariate calculus, we may also see these written as%
\begin{equation}
\begin{tabular}{ll}
$\mathbf{\nabla }\cdot \mathbf{E}=\frac{\rho }{\varepsilon _{o}}$ & Gauss's
law for electric fields \\ 
$\mathbf{\nabla }\cdot \mathbf{B}=0$ & Gauss's law for magnetic fields \\ 
$\mathbf{\nabla }\times \mathbf{E=}-\frac{d\mathbf{B}}{dt}$ & Faraday's law
\\ 
$c^{2}\mathbf{\nabla }\times \mathbf{B=}\frac{\mathbf{J}}{\varepsilon _{o}}+%
\frac{d\mathbf{E}}{dt}$ & Ampere-Maxwell Law%
\end{tabular}%
\end{equation}%
I'll let you remember the process to do the translation from $\doint \mathbf{%
B}\cdot d\mathbf{A}$ to $\mathbf{\nabla }\cdot \mathbf{B.}$

But we are familiar with all of these equations now. These four equations
are the basis of all of classical electrodynamics. In an electromagnetic
problem, we find the fields using the Maxwell equations to find the fields,
and then apply the fields to find the Lorentz forces%
\begin{equation}
\mathbf{F}=q\mathbf{E}+q\mathbf{v}\times \mathbf{B}
\end{equation}

It turns out that these four equations strongly imply that there can be
waves in the fields. Maxwell took the hint that $\mu _{o}\epsilon _{o}$ was
related to $c,$ the speed of light and he thought that light might be a wave
in the electromagnetic field. We know about waves. We can describe a wave by
looking for a surface of constant amplitude--a wave crest. We already know
from our study of optics that these waves are what we call light. A point
source will cause spherical surfaces of constant amplitude. A half-wave
antenna makes a toroidal shaped wave front. We will not deal with spherical
or worse wave shapes. Unfortunately, many antennas send out complicated wave
patterns that take spherical harmonics to describe well. That is beyond the
math we want to do in this course. We will stick to simple shapes. But we
will see how waves in the electromagnetic field describe light in our next
lecture.

\chapter{Waves in the Field}

We started this class with a study of waves. We learned about optics, and
finally electromagnetic field theory. In this lecture we will take on a case
study that involves all three. We will have come full circle and in the
process, hopefully understand all three topics a little better.

%TCIMACRO{%
%\TeXButton{Fundamental Concepts}{\hspace{-1.3in}{\LARGE Fundamental Concepts\vspace{0.25in}}}}%
%BeginExpansion
\hspace{-1.3in}{\LARGE Fundamental Concepts\vspace{0.25in}}%
%EndExpansion

\begin{itemize}
\item Maxwell's equation lead directly to the liner wave equation for both
the electric and the magnetic field with the speed of light being the speed
of the waves.

\item The magnitude of the $E$ and $B$ fields are related in an
electromagnetic wave by $E_{\max }=cB_{\max }$\FRAME{dhFU}{4.1572in}{2.0851in%
}{0pt}{\Qcb{A representation of a plane wave. Remember that the planes are
really of infinite extent. Image is public domain.}}{\Qlb{Planewave}}{Figure%
}{\special{language "Scientific Word";type "GRAPHIC";maintain-aspect-ratio
TRUE;display "USEDEF";valid_file "T";width 4.1572in;height 2.0851in;depth
0pt;original-width 27.0833in;original-height 13.5421in;cropleft "0";croptop
"1";cropright "1";cropbottom "0";tempfilename
'Lecture_Notes/planewave3D0.bmp';tempfile-properties "XNPR";}}
\end{itemize}

Let's picture our wave front far from the source. No matter what the total
shape, if we look at a small patch of the fields far away, they will look
like the plane wave in the last figure. Since this is a useful and common
situation (except if you use lasers), we will perform some calculations
assuming plane wave geometry.

We will assume we are in empty space, so the charge $q$ and current $I$ will
both be zero. Then our Maxwell Equations become%
\begin{equation}
\begin{tabular}{ll}
$\doint \overrightarrow{\mathbf{E}}\cdot d\overrightarrow{\mathbf{A}}=0$ & 
Gauss's law for electric fields \\ 
$\doint \overrightarrow{\mathbf{B}}\cdot d\overrightarrow{\mathbf{A}}=0$ & 
Gauss's law for magnetic fields \\ 
$\doint \overrightarrow{\mathbf{E}}\cdot d\overrightarrow{\mathbf{s}}=-\frac{%
d\Phi _{B}}{dt}$ & Faraday's law \\ 
$\doint \overrightarrow{\mathbf{B}}\cdot d\overrightarrow{\mathbf{s}}%
=\varepsilon _{o}\mu _{o}\frac{d\Phi _{E}}{dt}$ & Ampere-Maxwell Law%
\end{tabular}%
\end{equation}

Our goal is to show that these equations tell us that we can have waves in
the field. To do this, we will show that Maxwell's equations really contain
the linear wave equation within them. As a reminder, here is the linear wave
equation%
%TCIMACRO{%
%\TeXButton{Far Board}{\marginpar {
%\hspace{-0.5in}
%\begin{minipage}[t]{1in}
%\small{Far Board}
%\end{minipage}
%}}}%
%BeginExpansion
\marginpar {
\hspace{-0.5in}
\begin{minipage}[t]{1in}
\small{Far Board}
\end{minipage}
}%
%EndExpansion
\begin{equation*}
\frac{\partial ^{2}y}{\partial x^{2}}\mathbf{=}\frac{1}{v^{2}}\frac{\partial
^{2}y}{\partial t^{2}}
\end{equation*}%
it is a second order differential equation where the left side derivatives
are take with respect to position, and the right side derivatives are taken
with respect to time. The quantity, $v,$ is the wave speed. In this form of
the equation $y$ is the displacement of a medium. Our medium will be the
electromagnetic field.

\subsection{Rewriting of Faraday's law}

%TCIMACRO{%
%\TeXButton{Skip this}{\marginpar {
%\hspace{-0.5in}
%\begin{minipage}[t]{1in}
%\small{Skip this}
%\end{minipage}
%}}}%
%BeginExpansion
\marginpar {
\hspace{-0.5in}
\begin{minipage}[t]{1in}
\small{Skip this}
\end{minipage}
}%
%EndExpansion
Let's start with Faraday's law%
\begin{equation}
\doint \overrightarrow{\mathbf{E}}\cdot d\overrightarrow{\mathbf{s}}=-\frac{%
d\Phi _{B}}{dt}
\end{equation}%
Given our geometry, we can say the wave is traveling in the $x$ direction
with the $\overrightarrow{\mathbf{E}}$ field positive in the $y$ direction.
From our discussion of displacement currents we have a strong hint that the $%
\overrightarrow{\mathbf{E}}$ and $\overrightarrow{\mathbf{B}}$ fields will
be perpendicular. So let's take the magnetic field as positive in the $z$
direction. So as the light wave moves from the source along a line we could
draw the $\overrightarrow{\mathbf{E}}$ and $\overrightarrow{\mathbf{B}}$
fields something like this. \FRAME{dtbpF}{3.2301in}{2.5365in}{0in}{}{}{Figure%
}{\special{language "Scientific Word";type "GRAPHIC";maintain-aspect-ratio
TRUE;display "USEDEF";valid_file "T";width 3.2301in;height 2.5365in;depth
0in;original-width 3.1851in;original-height 2.4958in;cropleft "0";croptop
"1";cropright "1";cropbottom "0";tempfilename
'S50KLF0W.wmf';tempfile-properties "XPR";}}Let's take a small rectangle of
area to find $\doint \overrightarrow{\mathbf{E}}\cdot d\overrightarrow{%
\mathbf{s}}$ \FRAME{dhF}{1.7893in}{1.9112in}{0pt}{}{}{Figure}{\special%
{language "Scientific Word";type "GRAPHIC";maintain-aspect-ratio
TRUE;display "USEDEF";valid_file "T";width 1.7893in;height 1.9112in;depth
0pt;original-width 1.7521in;original-height 1.8732in;cropleft "0";croptop
"1";cropright "1";cropbottom "0";tempfilename
'M1NH5Q00.wmf';tempfile-properties "XPR";}}The top and bottom of the
rectangle don't contribute because $\overrightarrow{\mathbf{E}}\cdot d%
\overrightarrow{\mathbf{s}}=\mathbf{0}$ along these paths. On the sides, the
field is either in the $d\overrightarrow{\mathbf{s}}$ or it is in the
opposite direction. So 
\begin{equation*}
\doint \overrightarrow{\mathbf{E}}\cdot d\overrightarrow{\mathbf{s}}=\doint
Eds
\end{equation*}%
or 
\begin{equation*}
\doint \overrightarrow{\mathbf{E}}\cdot d\overrightarrow{\mathbf{s}}=-\doint
Eds
\end{equation*}%
along the sides. Let's say we travel counter-clockwise along the loop. Then
the left side will be negative and the right side will be positive.

\begin{equation*}
\doint \overrightarrow{\mathbf{E}}\cdot d\overrightarrow{\mathbf{s}}%
=\int_{right}Eds-\int_{left}Eds
\end{equation*}%
On the left side, we are at a position $x$ away from the axis, and on the
right side we are a position $x+\Delta x$ away from the axis. Then the field
of the left side is $E\left( x,t\right) $ and the field on the right hand
side is approximately 
\begin{equation}
E\left( x+\Delta x,t\right) \approx E\left( x,t\right) +\frac{\partial E}{%
\partial x}\Delta x
\end{equation}%
so if our loop is small, then $\ell $ is small and $E$ won't change much so
we can write approximately 
\begin{eqnarray}
\doint \overrightarrow{\mathbf{E}}\cdot d\overrightarrow{\mathbf{s}}
&=&\int_{right}Eds-\int_{left}Eds \\
&\approx &E\left( x+\Delta x,t\right) \ell -E\left( x,t\right) \ell \\
&=&\left( E\left( x,t\right) +\frac{\partial E}{\partial x}\Delta x\right)
\ell -E\left( x,t\right) \ell  \notag \\
&=&\left( E\left( x,t\right) +\frac{\partial E}{\partial x}\Delta x\right)
\ell -E\left( x,t\right) \ell  \notag \\
&=&\ell \frac{\partial E}{\partial x}\Delta x
\end{eqnarray}%
So far then, Faraday's law \footnote{%
We need $ds$ to be very small, much smaller than the wavelength of the wave.}
\begin{equation*}
\doint \overrightarrow{\mathbf{E}}\cdot d\overrightarrow{\mathbf{s}}=-\frac{%
d\Phi _{B}}{dt}
\end{equation*}%
becomes%
\begin{equation*}
\ell \frac{\partial E}{\partial x}\Delta x=-\frac{d\Phi _{B}}{dt}
\end{equation*}%
Let's move on to the right hand side of Faraday's law. We need to find $\Phi
_{B}$ so that we can find the time rate of change of the flux. We can say
that $B$ is nearly constant over such a small area, so 
\begin{eqnarray*}
\Phi _{B} &=&\mathbf{B}\cdot \mathbf{A} \\
&=&BA\cos \theta \\
&=&BA \\
&=&B\ell \Delta x
\end{eqnarray*}%
where here $\Delta x$ means \textquotedblleft a small
distance\textquotedblright\ as it did above. Then 
\begin{eqnarray*}
\frac{d\Phi _{B}}{dt} &=&\frac{d}{dt}\left( B\ell \Delta x\right) \\
&=&\ell \Delta x\left. \frac{dB}{dt}\right\vert _{x\text{ constant}} \\
&=&\ell \Delta x\frac{\partial B}{\partial t}
\end{eqnarray*}%
where we have held $x$ constant because we are not changing our small area,
so Faraday's law%
\begin{equation*}
\doint \overrightarrow{\mathbf{E}}\cdot d\overrightarrow{\mathbf{s}}=-\frac{%
d\Phi _{B}}{dt}
\end{equation*}%
becomes%
\begin{eqnarray}
\ell \frac{\partial E}{\partial x}\Delta x &\mathbf{=}&-\ell \Delta x\frac{%
\partial B}{\partial t}  \notag \\
\frac{\partial E}{\partial x} &\mathbf{=}&-\frac{\partial B}{\partial t}
\end{eqnarray}%
We have made some progress, we have a differential equation relating the
fields, but it is a mixed equation containing both the electric and magnetic
fields. We are only half way there.

\subsection{Rewriting of the Maxwell-Ampere Law}

We have used one field equation so far and that took us part of the way. We
have the Maxwell-Ampere law as well. We can use this to modify our result
from Faraday's law to find the linear wave equation that we expect. The
Maxwell-Ampere law with no sources (charges or currents) states 
\begin{equation*}
\doint \overrightarrow{\mathbf{B}}\cdot d\overrightarrow{\mathbf{s}}%
=\varepsilon _{o}\mu _{o}\frac{d\Phi _{E}}{dt}
\end{equation*}%
This time we must consider the magnetic field path integral\FRAME{dhF}{%
2.0271in}{2.1906in}{0pt}{}{}{Figure}{\special{language "Scientific
Word";type "GRAPHIC";maintain-aspect-ratio TRUE;display "USEDEF";valid_file
"T";width 2.0271in;height 2.1906in;depth 0pt;original-width
1.9882in;original-height 2.1508in;cropleft "0";croptop "1";cropright
"1";cropbottom "0";tempfilename 'M1NH6W01.wmf';tempfile-properties "XPR";}}%
We can do the same thing we did with Faraday's law with an area, but this
time we will use the area within the magnetic field (shown in the figure
above). Again, let's start with the left hand side of the equation. We see
that the sides of our area that are parallel to the $x$-axis do not matter
because $\overrightarrow{\mathbf{B}}\cdot d\overrightarrow{\mathbf{s}}=0$
along these sides, but the other two are in the direction (or opposite
direction) of the field. They do contribute to the line integral.%
\begin{eqnarray}
\doint \overrightarrow{\mathbf{B}}\cdot d\overrightarrow{\mathbf{s}}
&=&B\left( x,t\right) \ell -B\left( x+\Delta x,t\right) \ell \\
&\approx &-\ell \frac{\partial B}{\partial x}\Delta x  \notag
\end{eqnarray}

Now for the left hand side, we need the electric flux. For such a small
area, the field is nearly constant so%
\begin{eqnarray*}
\Phi _{E} &\approx &EA\cos \theta \\
&=&EA \\
&=&E\ell \Delta x
\end{eqnarray*}%
so 
\begin{equation}
\frac{\partial \Phi _{E}}{\partial t}=\ell \Delta x\frac{\partial E}{%
\partial t}
\end{equation}%
Combining both sides%
\begin{eqnarray}
\doint \overrightarrow{\mathbf{B}}\cdot d\overrightarrow{\mathbf{s}}
&=&\varepsilon _{o}\mu _{o}\frac{d\Phi _{E}}{dt}  \notag \\
-\ell \frac{\partial B}{\partial x}\Delta x &=&\varepsilon _{o}\mu _{o}\ell
\Delta dx\frac{\partial E}{\partial t}  \notag \\
\frac{\partial B}{\partial x} &=&-\varepsilon _{o}\mu _{o}\frac{\partial E}{%
\partial t}
\end{eqnarray}%
We now have a second differential equation relating $B$ and $E.$ But it is
also a mixed differential equation.

\section{Wave equation for plane waves}

This leaves us with two equations to work with

\begin{equation}
\frac{\partial E}{\partial x}\mathbf{=}-\frac{\partial B}{\partial t}
\end{equation}%
\begin{equation}
\frac{\partial B}{\partial x}=-\varepsilon _{o}\mu _{o}\frac{\partial E}{%
\partial t}
\end{equation}%
Remember that these are all partial derivatives. Taking the derivative of
the first equation with respect to $x$ gives%
\begin{eqnarray*}
\frac{\partial }{\partial x}\frac{\partial E}{\partial x} &\mathbf{=}&\frac{%
\partial }{\partial x}\left( -\frac{\partial B}{\partial t}\right) \\
\frac{\partial ^{2}E}{\partial x^{2}} &\mathbf{=}&-\frac{\partial }{\partial
x}\left( \frac{\partial }{\partial t}B\right) \\
\frac{\partial ^{2}E}{\partial x^{2}} &\mathbf{=}&-\frac{\partial }{\partial
t}\left( \frac{\partial B}{\partial x}\right)
\end{eqnarray*}%
In the last equation we swapped the order of differentiation for the right
hand side. In parenthesis, we have $\partial B/\partial x$ on the right hand
side. But we know what $\partial B/\partial x$ is from our second equation.
We substitute from our second equation to obtain%
\begin{equation*}
\frac{\partial ^{2}E}{\partial x^{2}}\mathbf{=}-\frac{\partial }{\partial t}%
\left( -\varepsilon _{o}\mu _{o}\frac{\partial E}{\partial t}\right)
\end{equation*}%
\begin{equation}
\frac{\partial ^{2}E}{\partial x^{2}}\mathbf{=}\varepsilon _{o}\mu _{o}\frac{%
\partial ^{2}E}{\partial t^{2}}
\end{equation}%
We can do the same thing, but taking derivatives with respect to time to give%
\begin{equation}
\frac{\partial ^{2}B}{\partial x^{2}}\mathbf{=}\varepsilon _{o}\mu _{o}\frac{%
\partial ^{2}B}{\partial t^{2}}
\end{equation}

You will recognize both of these last equations as being in the form of the
linear wave equation. 
\begin{equation*}
\frac{\partial ^{2}y}{\partial x^{2}}\mathbf{=}\frac{1}{v^{2}}\frac{\partial
^{2}y}{\partial t^{2}}
\end{equation*}%
This means that both the $E$ field and the $B$ field are governed by the
linear wave equation with the speed of the waves given by 
\begin{equation}
v=\frac{1}{\sqrt{\varepsilon _{o}\mu _{o}}}
\end{equation}

We have studied waves, so we know the solution to this equation is a sine or
cosine function%
\begin{eqnarray}
E &=&E_{\max }\cos \left( kx-\omega t\right) \\
B &=&B_{\max }\cos \left( kx-\omega t\right)
\end{eqnarray}%
with 
\begin{equation*}
k=\frac{2\pi }{\lambda }
\end{equation*}%
and 
\begin{equation*}
\omega =2\pi f
\end{equation*}%
then 
\begin{equation*}
\frac{\omega }{k}=\frac{2\pi f}{\frac{2\pi }{\lambda }}=\lambda f
\end{equation*}%
which is the wave speed.

We can show that the magnitude of $E$ is related to $B$.

Lets take derivatives of $E$ and $B$ with respect to $x$ and $t.$%
\begin{eqnarray*}
\frac{\partial E}{\partial x} &=&-kE_{\max }\sin \left( kx-\omega t\right) \\
\frac{\partial B}{\partial t} &=&\omega B_{\max }\sin \left( kx-\omega
t\right)
\end{eqnarray*}%
then we can use one of our half-way-point equations from above%
\begin{equation*}
\frac{\partial E}{\partial x}\mathbf{=}-\frac{\partial B}{\partial t}
\end{equation*}%
and by substitution obtain%
\begin{eqnarray*}
-kE_{\max }\sin \left( kx-\omega t\right) &=&-\omega B_{\max }\sin \left(
kx-\omega t\right) \\
-kE_{\max } &=&-\omega B_{\max }
\end{eqnarray*}%
or%
\begin{equation*}
\frac{E_{\max }}{B_{\max }}=\frac{\omega }{k}=v
\end{equation*}%
The speed is the speed of light, $c$, so 
\begin{equation}
\frac{E_{\max }}{B_{\max }}=c
\end{equation}%
It is one of the odd things about the universe that speed of electromagnetic
waves is a constant. It does not vary in vacuum, and the in-vacuum value, $c$
is the maximum speed. It was a combination of Maxwell's work in predicting $%
c $ and the observations confirming the predictions that launched Einstein
to form the Special Theory of Relativity!

Note that the last equation shows why we often only deal with the electric
field wave when we do optics. Since the magnetic field is proportional to
the electric field, we can always find it from the electric field.

\section{Properties of EM waves}

%TCIMACRO{%
%\TeXButton{Pick up here}{\marginpar {
%\hspace{-0.5in}
%\begin{minipage}[t]{1in}
%\small{Pick up here}
%\end{minipage}
%}}}%
%BeginExpansion
\marginpar {
\hspace{-0.5in}
\begin{minipage}[t]{1in}
\small{Pick up here}
\end{minipage}
}%
%EndExpansion
Knowing that the electric and magnetic fields form plane waves, we can
investigate these plane wave solutions to see what they imply.

\subsection{Energy in an EM\ wave}

The electromagnetic (EM) wave is a wave. Waves transfer energy. It is
customary find a vector that describes the flow of energy in the
electromagnetic wave. This is like the ray vectors we have been drawing for
some time, but with the magnitude of the vector giving the energy flow rate.

The rate of at which energy travels with the EM wave is given the symbol $%
\mathbf{S}$ and is called the Poynting vector after the person who thought
of it. It is 
\begin{equation}
\mathbf{S}=\frac{1}{\mu _{o}}\mathbf{E}\times \mathbf{B}
\end{equation}%
Let's deal with a dumb name first: The Poynting vector. It is named after a
scientist with the last name Poynting. The name is really meaningless. There
is nothing particularly \textquotedblleft pointy\textquotedblright\ about
this vector more than any other vector.

Instead of a formal derivation, let's just see what we get from Poynting's
equation for a plane wave.

For our plane wave case, $E$ and $B$ are at $90\unit{%
%TCIMACRO{\U{b0}}%
%BeginExpansion
{{}^\circ}%
%EndExpansion
}$ angles\footnote{%
For other fields this might not be true, but it is generally true for light.}%
. so 
\begin{equation}
S=\frac{1}{\mu _{o}}EB
\end{equation}%
and $S$ will be perpendicular to both. Notice from our preceding figures
that this is also the direction that the wave travels! That is comforting.
That should be true for a EM wave. The energy, indeed, goes the way the
Poynting vector points.

Using 
\begin{equation*}
\frac{E}{B}=c
\end{equation*}%
we can write the magnitude of the Poynting vector as%
\begin{equation}
S=\frac{E^{2}}{c\mu _{o}}
\end{equation}%
We could also express this in terms of $B$ only.

You will remember that our eyes don't track the oscillations of the
electromagnetic waves. Few detectors (if any) can. For visible light, the
frequency is very high. We usually see a time average. This time average of
the Poynting vector is called the \emph{intensity} of the wave%
\begin{equation*}
I=S_{ave}
\end{equation*}

\subsection{Intensity of the waves}

When we studied waves, we learned that waves have an intensity. The
intensity of electromagnetic waves must relate to the strength of the
fields. We can write it as%
\begin{equation*}
I=\frac{EB}{2\mu _{o}}
\end{equation*}%
(can you remember where the \textquotedblleft $1/2$\textquotedblright\ came
from?)\footnote{%
This is because the average value of $\sin ^{2}\left( \omega t\right) $ over
a period is given by $\frac{1}{T}\int_{0}^{T}\sin ^{2}\left( \omega t\right)
dt=\allowbreak \frac{\omega }{2\pi }\int_{0}^{\frac{2\pi }{\omega }}\sin
^{2}\left( \omega t\right) dt=\allowbreak \frac{1}{2}$}. Again using%
\begin{equation*}
E=cB
\end{equation*}%
we can write the intensity as%
\begin{equation}
I=\frac{1}{2\mu _{o}c}E^{2}
\end{equation}%
We remember that $I$ is proportional to the square of the maximum electric
field strength from our previous consideration of light intensity. But
before we only said that it was proportional. Now we know the constant of
proportionality. Of course we could also write the intensity as 
\begin{equation}
I=\frac{c}{2\mu _{o}}B^{2}
\end{equation}%
but this is less traditional. We have said already that the intensity, $I$,
is the magnitude of the average Poynting vector $S_{ave}.$

Recall that we know the energy densities in the fields%
\begin{eqnarray*}
u_{E} &=&\frac{1}{2}\varepsilon _{o}E^{2} \\
u_{B} &=&\frac{1}{2}\frac{B^{2}}{\mu _{o}}
\end{eqnarray*}%
again, since 
\begin{equation}
E=cB
\end{equation}%
we can write 
\begin{eqnarray}
u_{B} &=&\frac{1}{2}\frac{B^{2}}{\mu _{o}} \\
&=&\frac{1}{2}\frac{E^{2}}{c^{2}\mu _{o}}  \notag \\
&=&\frac{1}{2}\varepsilon _{o}E^{2}  \notag
\end{eqnarray}%
so for a plane electromagnetic wave 
\begin{equation}
u_{E}=u_{B}
\end{equation}%
The total energy in the field is just the sum%
\begin{equation}
u=u_{E}+u_{B}=\varepsilon _{o}E^{2}
\end{equation}%
But when we do the time average to find the intensity, we pick up a factor
of a half%
\begin{equation}
u_{ave}=\frac{1}{2}\varepsilon _{o}E^{2}
\end{equation}%
Comparing this to our equation for intensity gives%
\begin{equation*}
I=\frac{1}{2\mu _{o}c}E_{\max }^{2}=S_{ave}
\end{equation*}%
and then%
\begin{eqnarray}
S_{ave} &=&\frac{1}{\epsilon _{o}\mu _{o}c}\frac{1}{2}\epsilon _{o}E^{2} \\
&=&\frac{1}{\epsilon _{o}\mu _{o}c}u_{ave}  \notag \\
&=&\frac{1}{\frac{1}{c^{2}}c}u_{ave}  \notag \\
&=&cu_{ave}  \notag
\end{eqnarray}

If you have already taken your course on thermodynamics you, learned that we
could transfer energy by radiation. This is our radiation! And we see that
it does indeed transfer energy. We learned about this by discussing solar
heating and by talking about Army weapons that apply energy to crowds.\FRAME{%
dhFU}{2.2814in}{2.5002in}{0pt}{\Qcb{{\protect\small US Army Active Denial
System (ADS).
https://jnlwp.defense.gov/Portals/50/Documents/Resources/Presentations/Joint%
\_Integration\_Program\_Advanced\_Planning\_Briefs\_to\_Industry/2017\%20DoD%
\_OGA\%20NLW\%20APBI\%20Army.pdf?ver=2020-08-25-125259-783}}}{}{Figure}{%
\special{language "Scientific Word";type "GRAPHIC";maintain-aspect-ratio
TRUE;display "USEDEF";valid_file "T";width 2.2814in;height 2.5002in;depth
0pt;original-width 2.2955in;original-height 2.5199in;cropleft "0";croptop
"1";cropright "1";cropbottom "0";tempfilename
'LVRLHK06.wmf';tempfile-properties "XPR";}}but we really use this every day
when we microwave something. Microwaves are electromagnetic waves!

\subsection{Momentum of light}

One of the strangest things is that there is also momentum in the
electromagnetic waves. If the waves are absorbed, the momentum is 
\begin{equation}
p=\frac{U}{c}
\end{equation}%
or if the waves are reflected it is 
\begin{equation}
p=\frac{2U}{c}
\end{equation}%
(think of balls bouncing off a wall, the change in momentum is always $2mv$
for a bounce)$.$

We can think of the light exerting a pressure on the surface. Force is given
by 
\begin{eqnarray*}
F &=&ma \\
&=&m\frac{dv}{dt} \\
&=&\frac{dp}{dt}
\end{eqnarray*}%
then using this force, the pressure is%
\begin{equation}
P=\frac{F}{A}=\frac{1}{A}\frac{dp}{dt}
\end{equation}%
then 
\begin{equation}
P=\frac{F}{A}=\frac{1}{cA}\frac{dU}{dt}
\end{equation}%
We found $\frac{1}{A}\frac{dU}{dt}$ to be the energy rate per unit area,
which is the magnitude of the Poynting vector, $S$. So our pressure due to
light is%
\begin{equation}
P=\frac{S}{c}
\end{equation}%
for perfect absorption. If there is perfect reflection%
\begin{equation}
P=\frac{2S}{c}
\end{equation}

This may seem a little strange. Water or sound waves would exert a pressure
because the water or air particles can strike a surface, exerting a force.
But remember the electromagnetic fields will create forces on the electrons
in atoms\footnote{%
Protons too, but the protons are more tightly bound due to the nuclear
strong force and the nuclei are bound in the material. their resonant
frequencies are usually not assessable to visible light, so I will ignore
their effect in our treatment. But if you consider x-rays or gamma rays,
they would be important.}, and most of the electrons are bound to the atoms
in materials by the Coulomb force. So there really is a force on the
material due to the electromagnetic wave. Quantum mechanics tells us about
electrons being knocked out of shells into higher energy shells (absorbing
photons of light) and re-emitting the light when the electrons fall back
down to lower shells. This is a little like catching a frisbee, and then
throwing it. Momentum is transferred both at the catch and at the release.

A cool use of this phenomena is called laser levitation

\FRAME{dtbpFU}{3.7913in}{2.8487in}{0pt}{\Qcb{{\protect\small Laser Levation
(Skigh Lewis, Larry Baxter, Justin Peatross (BYU), Laser Levitation:
Determination of Particle Reactivity, ACERC Conference Presentation,
February 17, 2005)}}}{}{Figure}{\special{language "Scientific Word";type
"GRAPHIC";maintain-aspect-ratio TRUE;display "USEDEF";valid_file "T";width
3.7913in;height 2.8487in;depth 0pt;original-width 5.2088in;original-height
3.9064in;cropleft "0";croptop "1";cropright "1";cropbottom "0";tempfilename
'S4WP8304.wmf';tempfile-properties "XPR";}}

In the picture you are seeing a single small particle that is floating on a
laser beam. the laser beam is directed upward. The force due to gravity
would make the particle fall, but the laser light keeps it up!

\subsection{Antennas Revisited}

We talked about antennas before. Let's try to put all we have done together
to make a radio wave. First, we know from our analysis that we need changing
fields. Neither static charges, nor constant currents will do. If we think
about this for a minute, we will realize that the charges will \emph{%
accelerate}. Fundamentally, this is the mechanism for making EM waves.

The half wave antenna is simple to understand, so let's take it as our
example.\FRAME{dhF}{4.7668in}{2.2191in}{0pt}{}{}{Figure}{\special{language
"Scientific Word";type "GRAPHIC";maintain-aspect-ratio TRUE;display
"USEDEF";valid_file "T";width 4.7668in;height 2.2191in;depth
0pt;original-width 4.7124in;original-height 2.1785in;cropleft "0";croptop
"1";cropright "1";cropbottom "0";tempfilename
'M1NOF802.wmf';tempfile-properties "XPR";}}It is made from two long wires
connected to an alternating current source (the radio transmitter).The
charges are separated in the antenna as shown. But the separation switches
as the alternating current changes direction. The charges accelerate back
and forth, like a dipole switching direction. Radio people call this antenna
a \emph{simple dipole}.

Note the direction of the $E$ and $B$ fields. The Poynting vector is to the
right. The antenna field sets up a situation far from the antenna, itself,
where the changing electric field continually induces a magnetic field and
the changing magnetic field continually induces a changing electric field.
The wave becomes self sustaining! And the energy it carries travels outward.

Below you can see a graph of the sort of toroidal angular dependence of the
dipole antenna emission pattern.\FRAME{dtbpFUX}{3.531in}{2.354in}{0pt}{\Qcb{%
Angular dependence of $S$ for a dipole scatterer.}}{}{Plot}{\special%
{language "Scientific Word";type "MAPLEPLOT";width 3.531in;height
2.354in;depth 0pt;display "USEDEF";plot_snapshots TRUE;mustRecompute
FALSE;lastEngine "MuPAD";xmin "-3.1416";xmax "3.1416";ymin "0";ymax
"3.1416";xviewmin "-1.500000";xviewmax "1.500000";yviewmin
"-1.500000";yviewmax "1.500000";zviewmin "-0.5";zviewmax
"0.5";viewset"XYZ";rangeset"XYZ";phi 34;theta 40;cameraDistance
"6.0218";cameraOrientation "[0,0,0.86996]";cameraOrientationFixed
TRUE;plottype 15;labeloverrides 7;x-label "x";y-label "y";z-label
"z";axesFont "Times New Roman,12,0000000000,useDefault,normal";constrained
TRUE;num-x-gridlines 25;num-y-gridlines 25;plotstyle "patch";axesstyle
"normal";axestips FALSE;plotshading "XYZ";lighting 0;xis \TEXUX{v58130};yis
\TEXUX{v58144};var1name \TEXUX{$\theta $};var2name \TEXUX{$\phi $};function
\TEXUX{$\frac{\sin ^{2}\phi }{1}$};linestyle 1;pointstyle
"point";linethickness 1;lineAttributes "Solid";coordinateSystem
"spherical";var1range "-3.1416,3.1416";var2range "0,3.1416";surfaceColor
"[linear:XYZ:RGB:0x00ff0000:0x000000ff]";surfaceStyle "Color
Patch";num-x-gridlines 25;num-y-gridlines 25;surfaceMesh "Mesh";VCamFile
'LVRN4L0T.xvz';valid_file "T";tempfilename
'LVRNRC0A.wmf';tempfile-properties "XPR";}}From this you can see why we
usually stand antennas straight up and down. Then the transmission travels
parallel to the Earth's surface, where receivers are more likely to be.

Speaking of receivers, of course the receiver works like a transmitter, only
backwards. The EM waves that hit the receiving antenna accelerate the
electrons in the wire of the antenna. The induced current passed through an
LRC circuit who's resonance frequency allows amplification of just one small
band of frequencies (the one your favorite radio station is using) and then
the amplified signal is sent to a speaker.

\section{The Electromagnetic Spectrum}

Maxwell predicted how fast his field waves would travel by finding the
linear wave equation from the fields and noticing the speed indicated by the
result. We have seen how he did this. The answer is 
\begin{equation}
v=\frac{1}{\sqrt{\varepsilon _{o}\mu _{o}}}
\end{equation}%
this speed is so special in physics that it get's its own letter 
\begin{equation}
c=2.99792\times 10^{8}\frac{\unit{m}}{\unit{s}}
\end{equation}%
which is of course the speed of light. In fact, that this was the measured
speed of light was strong evidence leading us to conclude that light was
really a type of these waves. There are a few more types of electromagnetic
waves. In the following chart you can see that visible light is just a small
part of what we call the \emph{electromagnetic spectrum.}\FRAME{dtbpFU}{%
4.5109in}{0.3537in}{0pt}{\Qcb{Electromagnetic Spectrum (Public Domain image
courtesy NASA)}}{}{Figure}{\special{language "Scientific Word";type
"GRAPHIC";maintain-aspect-ratio TRUE;display "USEDEF";valid_file "T";width
4.5109in;height 0.3537in;depth 0pt;original-width 4.0309in;original-height
0.2897in;cropleft "0";croptop "1";cropright "1";cropbottom "0";tempfilename
'MFFY4Z01.wmf';tempfile-properties "XPR";}}

The speed of light is always a constant in vacuum. This is strange. It
caused a lot of problems when it was discovered. 
\begin{equation}
v=f\lambda
\end{equation}%
or 
\begin{equation}
c=f\lambda
\end{equation}%
where we can see that for light and electromagnetic waves, knowing the
wavelength is always enough to know the frequency as well (in a vacuum).

As an example of what problems can come, let's consider a Doppler effect for
light. Remember for sound waves, we had a Doppler effect. We will have a
Doppler effect for electromagnetic waves too. But light does not change it's
speed relative to a reference frame. This is \emph{really weird}. The speed
of light in a vacuum is \emph{always }$c$--\emph{no matter what frame we
measure it in}.

Einstein's theory of Special Relativity is required to deal with this
constant speed of light in every reference frame. From Relativity, the
Doppler equation is 
\begin{equation}
f^{\prime }=f\frac{\sqrt{1+\frac{v}{c}}}{\sqrt{1-\frac{v}{c}}}
\end{equation}%
or, if we let $u$ be the relative velocity between the source and the
detector, and insist that $u\ll c$%
\begin{equation}
f^{\prime }=f\left( \frac{c+u}{c}\right)
\end{equation}%
Where of course $f^{\prime }$ is the observed frequency and $f$ is the
frequency emitted by the source. This is usually written as%
\begin{equation}
f^{\prime }=f\left( 1\pm \frac{u}{c}\right)
\end{equation}%
but it is really the same equation\footnote{%
This equation is only really true for relative speeds $u$ that are much less
than the speed of light. Since is is very hard to make something travel even
close to to the speed of light, we will find it is nearly always true.}.
Just like with sound, we use the positive sign when the source and observer
are approaching each other.

This means that if things are moving closer to each other the frequency
increases. Think of 
\begin{equation}
\lambda =\frac{c}{f}
\end{equation}%
this means that as a source and emitter approach each other, then the light
will have a shorter wavelength. Think of our chart on the electromagnet
spectrum. This means the light will get bluer. If they move farther apart,
the light will get redder.

This is what gave us the hint that has lead to our cosmological theories
like the big bang. Although this theory is now much more complicated, the
facts are that as we look at far away objects, we see they are all \emph{red
shifted}. That is, they all show absorption spectra for known elements, but
at longer wavelengths that we expect from laboratory experiments. We
interpret this as meaning they are all going away from us!

\subsection{Summary}

Here is what we have learned so far about the properties of light

\begin{enumerate}
\item Electromagnetic waves travel at the speed of light

\item Electromagnetic waves are transverse electric and magnetic waves that
are oriented perpendicular to each other.

\item $E=cB$

\item Electromagnetic waves carry energy $\emph{and}$ \emph{momentum}
\end{enumerate}

\subsection{Photons}

Our understanding of light is not complete yet. If you went on to take PH279
you would find that light still operates much like a particle at times. This
should not be a surprise, since Newton and others explained much of optics
(the study of light) assuming light was a particle.

Einstein and others noticed that for some metals, light would strike the
surface and electrons would leave the surface. The energy of a wave is
proportional to the amplitude of the wave. It was expected that if the
amplitude of the electromagnetic wave was increased, the number of electrons
leaving the surface would increase. This proved to be true most of the time.
But Hertz and others decided to try different frequencies of light. It turns
out that as you lower the frequency, all of a sudden no electrons leave no
matter how big the amplitude of the wave. Something was wrong with our wave
theory of light. The answer came from Einstein who used the idea of a
\textquotedblleft packet\textquotedblright\ of light to explain this \emph{%
photoelectric effect}. For now, we should know just that the waves of light
exist in \emph{quantized} packets called \emph{photons}. The energy of a
photon is 
\begin{equation}
E=hf
\end{equation}%
where $E$ is the energy, $f$ is the frequency of the light wave, and $h$ is
a constant%
\begin{equation}
h=6.63\times 10^{-34}\unit{J}\unit{s}
\end{equation}

A beam of light is many, many photons all superimposing. We know how waves
combine using superposition, so it is easy to see that we can get a big wave
from many little waves.

Knowing that light is made from electric and magnetic fields, and that these
fields are vector fields, we should expect some directional quality in
light. And there is such a directional quality that we will study next
lecture.

\chapter{Polarization}

%TCIMACRO{%
%\TeXButton{Fundamental Concepts}{\hspace{-1.3in}{\LARGE Fundamental Concepts\vspace{0.25in}}}}%
%BeginExpansion
\hspace{-1.3in}{\LARGE Fundamental Concepts\vspace{0.25in}}%
%EndExpansion

\begin{itemize}
\item The direction of the electric field in a plane wave is called the
polarization direction.

\item Natural light is usually a superposition of many waves with random
polarization directions. This light is called unpolarized light.

\item Some materials allow light with one polarization to pass through,
while stopping other polarizations. The polaroid is one such material
polaroids. will have a final intensity that follows the relationship $%
I=I_{\max }\cos ^{2}\left( \theta \right) $

\item Light reflecting off a surface may be polarized because of the
absorption and re-emission pattern of light interacting with the material
atoms.

\item Scattered light may be polarized because of anisotropies in the
scatterers.

\item Birefringent materials have different wave speeds in different
directions. This affects the polarization of light entering these materials.
\end{itemize}

\section{Polarization of Light Waves}

We said much earlier in our study of light that it was a transverse wave.
Last lecture we saw that we have an electric and magnetic field direction,
and that these directions are perpendicular to each other and the direction
of energy flow. We will now show some implications of this fact. In a course
in electromagnetic theory, we often draw light as in the figure below.\FRAME{%
dhF}{3.681in}{3.673in}{0pt}{}{}{Figure}{\special{language "Scientific
Word";type "GRAPHIC";maintain-aspect-ratio TRUE;display "USEDEF";valid_file
"T";width 3.681in;height 3.673in;depth 0pt;original-width
3.7218in;original-height 3.7156in;cropleft "0";croptop "1";cropright
"1";cropbottom "0";tempfilename 'LVRNRC0B.wmf';tempfile-properties "XPR";}}%
We will continue to ignore the magnetic field (marked in the figure as $B$).
We will look at the $E$ field an notice that it goes up and down in the
figure. But we could have light in any orientation. If we look directly at
an approaching beam of light we would \textquotedblleft
see\textquotedblright\ many different orientations as shown in the next
figure.\FRAME{dtbpF}{1.5307in}{1.5895in}{0pt}{}{}{Figure}{\special{language
"Scientific Word";type "GRAPHIC";maintain-aspect-ratio TRUE;display
"USEDEF";valid_file "T";width 1.5307in;height 1.5895in;depth
0pt;original-width 1.4944in;original-height 1.5541in;cropleft "0";croptop
"1";cropright "1";cropbottom "0";tempfilename
'METLZR00.wmf';tempfile-properties "XPR";}}

When light beams have waves with many orientations, we say they are \emph{%
unpolarized}. But suppose we were able to align all the light so that all
the waves in the beam were transverse waves in the same orientation. Say,
the one in the next figure.\FRAME{dhF}{0.3226in}{1.3508in}{0pt}{}{}{Figure}{%
\special{language "Scientific Word";type "GRAPHIC";maintain-aspect-ratio
TRUE;display "USEDEF";valid_file "T";width 0.3226in;height 1.3508in;depth
0pt;original-width 1.4688in;original-height 6.2923in;cropleft "0";croptop
"1";cropright "1";cropbottom "0";tempfilename
'LVRNLQ09.wmf';tempfile-properties "XPR";}}

Then we would describe the light as \emph{linearly polarized}. The plane
that contains the $E$-field is known as the \emph{polarization plane}.

\subsection{Polarization by removing all but one wave orientation}

One way to make polarized light is to remove all but one orientation of an
unpolarized beam. A material that does this at visible wavelengths is called
a \emph{polaroid}. It is made of long-chain hydrocarbons that have been
treated with iodine to make them conductive. The molecules are all oriented
in one direction by stretching during the manufacturing process. The
molecules have electrons that can move when light hits them. They can move
farther in the long direction of the molecule, so in this direction the
molecules act like little antennas. The molecules' electrons are driven into
harmonic motion along the length of the molecule. This takes energy (and
therefore, light) out of the beam. Little electron motion is possible in the
short direction of the molecule, so light is given a preferential
orientation. The light is passed if it is perpendicular to the long
direction of the molecules. This direction is called the \emph{transmission
axis}.

We can take two pieces of polaroid material to study polarization.\FRAME{dhF%
}{3.7879in}{1.4088in}{0in}{}{}{Figure}{\special{language "Scientific
Word";type "GRAPHIC";maintain-aspect-ratio TRUE;display "USEDEF";valid_file
"T";width 3.7879in;height 1.4088in;depth 0in;original-width
3.7395in;original-height 1.3742in;cropleft "0";croptop "1";cropright
"1";cropbottom "0";tempfilename 'M1B7CL03.wmf';tempfile-properties "XPR";}}

Unpolarized light is initially polarized by the first piece of polaroid
called the \emph{polarizer}. The second piece of polaroid then receives the
light. This piece is called the \emph{analyzer}. If there is an angular
difference in the orientation of the transmission axes of the polarizer and
analyzer, there will be a reduction of light through the system. We expect
that if the transmission axes are separated by $90\unit{%
%TCIMACRO{\U{b0}}%
%BeginExpansion
{{}^\circ}%
%EndExpansion
}$ no light will be seen. If they are separated by $0\unit{%
%TCIMACRO{\U{b0}}%
%BeginExpansion
{{}^\circ}%
%EndExpansion
},$ then there will be a maximum. It is not hard to believe that the
intensity will be given by%
\begin{equation}
I=I_{\max }\cos ^{2}\left( \theta \right)
\end{equation}%
remembering that we must have a squared term because $I\varpropto E^{2}.$

\subsection{Polarization by reflection}

If we look at light reflected off of a desk or table through a piece of
polaroid, we can see that at some angles of orientation, the reflection
diminishes or even disappears! Light is often polarized on reflection. Let's
consider a beam of light made of just two polarizations. We will define a
plane of incidence. This plane is the plane of the paper or computer screen.
This plane is perpendicular to the reflective or refractive surface in the
figure below.\FRAME{dhF}{2.5996in}{2.1906in}{0in}{}{}{Figure}{\special%
{language "Scientific Word";type "GRAPHIC";maintain-aspect-ratio
TRUE;display "USEDEF";valid_file "T";width 2.5996in;height 2.1906in;depth
0in;original-width 2.5581in;original-height 2.1508in;cropleft "0";croptop
"1";cropright "1";cropbottom "0";tempfilename
'M1B66900.wmf';tempfile-properties "XPR";}}

One of our polarizations is defined as parallel to this plane. This
direction is represented by orange (lighter grey in black and white) arrows
in the figure. The other polarization is perpendicular to the plane of
incidence (the plane of the paper). This is represented by the black dots in
the figure. These dots are supposed to look like arrows coming out of the
paper.

When the light reaches the interface between $n_{1}$ and $n_{2}$ it drives
the electrons in the medium into SHM. The perpendicular polarization finds
electrons that are free to move in the perpendicular direction and
re-radiate in that direction. Even for a dielectric, the electron orbitals
change shape and oscillate with the incoming electromagnetic wave.

The parallel ray is also able to excite SHM, but a electromagnetic analysis
tells us that these little \textquotedblleft antennas\textquotedblright\
will not radiate at an angle $90\unit{%
%TCIMACRO{\U{b0}}%
%BeginExpansion
{{}^\circ}%
%EndExpansion
}$ from their excitation direction. Think of little dipole radiators. We can
plot the amplitude of the electric field as a function of direction around
the antenna.\FRAME{dtbpFUX}{1.5939in}{1.0629in}{0pt}{\Qcb{Angular dependence
of $S$ for a dipole scatterer.}}{}{Plot}{\special{language "Scientific
Word";type "MAPLEPLOT";width 1.5939in;height 1.0629in;depth 0pt;display
"USEDEF";plot_snapshots TRUE;mustRecompute FALSE;lastEngine "MuPAD";xmin
"-3.1416";xmax "3.1416";ymin "0";ymax "3.1416";xviewmin "-1.500000";xviewmax
"1.500000";yviewmin "-1.500000";yviewmax "1.500000";zviewmin "-0.5";zviewmax
"0.5";viewset"XYZ";rangeset"XYZ";phi 34;theta 40;cameraDistance
"6.0218";cameraOrientation "[0,0,0.86996]";cameraOrientationFixed
TRUE;plottype 15;labeloverrides 7;x-label "x";y-label "y";z-label
"z";axesFont "Times New Roman,12,0000000000,useDefault,normal";constrained
TRUE;num-x-gridlines 25;num-y-gridlines 25;plotstyle "patch";axesstyle
"normal";axestips FALSE;plotshading "XYZ";lighting 0;xis \TEXUX{v58130};yis
\TEXUX{v58144};var1name \TEXUX{$\theta $};var2name \TEXUX{$\phi $};function
\TEXUX{$\frac{\sin ^{2}\phi }{1}$};linestyle 1;pointstyle
"point";linethickness 1;lineAttributes "Solid";coordinateSystem
"spherical";var1range "-3.1416,3.1416";var2range "0,3.1416";surfaceColor
"[linear:XYZ:RGB:0x00ff0000:0x000000ff]";surfaceStyle "Color
Patch";num-x-gridlines 25;num-y-gridlines 25;surfaceMesh "Mesh";VCamFile
'METMCS02.xvz';valid_file "T";tempfilename
'METMCS01.wmf';tempfile-properties "XPR";}}We see that along the antenna
axis, the field amplitude is zero. This means that the wave really does not
go that direction. So in our case, the amount of polarization in the
parallel direction decreases with the angle between the reflected and
refracted rays until at $90\unit{%
%TCIMACRO{\U{b0}}%
%BeginExpansion
{{}^\circ}%
%EndExpansion
}$ there is no reflected ray in the parallel direction. \FRAME{dhF}{3.7057in%
}{3.0753in}{0pt}{}{}{Figure}{\special{language "Scientific Word";type
"GRAPHIC";maintain-aspect-ratio TRUE;display "USEDEF";valid_file "T";width
3.7057in;height 3.0753in;depth 0pt;original-width 5.0055in;original-height
4.1494in;cropleft "0";croptop "1";cropright "1";cropbottom "0";tempfilename
'M1UZAX02.wmf';tempfile-properties "XPR";}}

The incidence angle that creates an angular difference between the refracted
and reflected rays of $90\unit{%
%TCIMACRO{\U{b0}}%
%BeginExpansion
{{}^\circ}%
%EndExpansion
}$ is called the Brewster's\emph{\ angle} after its discoverer. At this
angle the reflected beam will be completely linearly polarized.

We can predict this angle. Remember Snell's law.%
\begin{equation*}
n_{1}\sin \theta _{1}=n_{2}\sin \theta _{2}
\end{equation*}%
Let's re-lable the incidence angle $\theta _{1}=$ $\theta _{b}.$ We take $%
n_{1}=1$ and $n_{2}=n$ so%
\begin{equation*}
n=\frac{\sin \theta _{b}}{\sin \theta _{2}}
\end{equation*}%
Now notice that for Brewster's angle, we have 
\begin{equation*}
\theta _{b}+90\unit{%
%TCIMACRO{\U{b0}}%
%BeginExpansion
{{}^\circ}%
%EndExpansion
}+\theta _{2}=180\unit{%
%TCIMACRO{\U{b0}}%
%BeginExpansion
{{}^\circ}%
%EndExpansion
}
\end{equation*}%
so%
\begin{equation*}
\theta _{2}=90\unit{%
%TCIMACRO{\U{b0}}%
%BeginExpansion
{{}^\circ}%
%EndExpansion
}-\theta _{b}
\end{equation*}%
so we have%
\begin{equation*}
n=\frac{\sin \theta _{b}}{\sin \left( 90\unit{%
%TCIMACRO{\U{b0}}%
%BeginExpansion
{{}^\circ}%
%EndExpansion
}-\theta _{b}\right) }
\end{equation*}%
ah, but we remember that $\sin \left( 90\unit{%
%TCIMACRO{\U{b0}}%
%BeginExpansion
{{}^\circ}%
%EndExpansion
}-\theta \right) =\cos \left( \theta \right) $ so%
\begin{equation*}
n=\frac{\sin \theta _{b}}{\cos \theta _{b}}
\end{equation*}%
but again we remember that 
\begin{equation*}
\tan \theta =\frac{\sin \theta }{\cos \theta }
\end{equation*}%
so%
\begin{equation}
n=\tan \theta _{b}
\end{equation}%
which we can solve for $\theta _{b}.$%
\begin{equation*}
\theta _{b}=\tan ^{-1}\left( n\right)
\end{equation*}

This phenomena is why we wear polarizing sunglasses to reduce glare.

\subsection{Birefringence}

Glass is an amorphic solid--that is--it has no crystal structure to speak
of. But some minerals do have definite order. Sometimes the difference in
the crystal structure creates a difference in the speed of propagation of
light in the crystal. This is not to hard to believe. We said before that
the reason light slows down in a substance is because it encounters atoms
which absorb and re-emit the light. If there are more atoms in one direction
than another in a crystal, it makes sense that there could be a different
speed in each direction.

Calcite crystals exhibit this phenomena. We can describe what happens by
defining two polarizations. One parallel to the plane of the figure below,
and one perpendicular.

\bigskip

\FRAME{dhF}{3.1337in}{2.0117in}{0in}{}{}{Figure}{\special{language
"Scientific Word";type "GRAPHIC";maintain-aspect-ratio TRUE;display
"USEDEF";valid_file "T";width 3.1337in;height 2.0117in;depth
0in;original-width 3.1647in;original-height 2.0214in;cropleft "0";croptop
"1";cropright "1";cropbottom "0";tempfilename
'LVWSC702.wmf';tempfile-properties "XPR";}}With a careful setup, we can
arrange things so the perpendicular ray is propagated just as we would
expect for glass. We call this the $O$-ray (for \emph{ordinary}). The second
ray is polarized parallel to the incidence plane. It will have a different
speed, and therefore a different index of refraction. We call it the \emph{%
Extraordinary ray} or $E$-ray.

\FRAME{dhF}{2.1137in}{2.1234in}{0in}{}{}{Figure}{\special{language
"Scientific Word";type "GRAPHIC";maintain-aspect-ratio TRUE;display
"USEDEF";valid_file "T";width 2.1137in;height 2.1234in;depth
0in;original-width 2.1252in;original-height 2.135in;cropleft "0";croptop
"1";cropright "1";cropbottom "0";tempfilename
'LVWSQU03.wmf';tempfile-properties "XPR";}}

If we were to put a light source in a calcite crystal, we would see the $O$%
-ray send out a sphere of light as shown in the figure above. But the $E$%
-ray would send out an ellipse. The speed for the $E$-ray depends on
orientation. There is one direction where the speeds are equal. This
direction is called the \emph{optic axis} of the crystal.\FRAME{dhF}{2.3514in%
}{1.0617in}{0pt}{}{}{Figure}{\special{language "Scientific Word";type
"GRAPHIC";maintain-aspect-ratio TRUE;display "USEDEF";valid_file "T";width
2.3514in;height 1.0617in;depth 0pt;original-width 2.3674in;original-height
1.0528in;cropleft "0";croptop "1";cropright "1";cropbottom "0";tempfilename
'LVWRYX01.wmf';tempfile-properties "XPR";}}If our light entering our calcite
crystal is unpolarized, then we will have two images leaving the other side
that are slightly offset because the $O$-rays and $E$-rays both form images.

\subsection{Optical Stress Analysis}

Some materials (notably plastics) become birefringent under stress. A
plastic or other stress birefringent material is molded in the form planned
for a building or other object (usually made to scale). The model is placed
under a stress, and the system is placed between to polaroids. When
unstressed, no light is seen, but under stress, the model changes the
polarization state of the light, and bands of light are seen.\FRAME{dhF}{%
3.1479in}{2.0676in}{0pt}{}{}{Figure}{\special{language "Scientific
Word";type "GRAPHIC";maintain-aspect-ratio TRUE;display "USEDEF";valid_file
"T";width 3.1479in;height 2.0676in;depth 0pt;original-width
3.1798in;original-height 2.0782in;cropleft "0";croptop "1";cropright
"1";cropbottom "0";tempfilename 'LVWRX900.wmf';tempfile-properties "XPR";}}

\subsection{Polarization due to scattering}

It is important to understand that light is also polarized by scattering. It
really takes a bit of electromagnetic theory to describe this. So for a
moment, lets just comment that blue light is scattered more than red light.
In fact, the relative intensity of scattered light goes like $1/\lambda
^{4}. $ This has nothing to do with polarization, but it is nice to know.

Now suppose we have long pieces of wire in the air, say, a few microns long.
The pieces of wire would have electrons that could be driven into SHM when
light hits them. If the wires were all oriented in a common direction, we
would expect light to be absorbed if it was polarized in the long direction
of the particles and not absorbed in a direction perpendicular to the
orientation of the particles. This is exactly what happens when long ice
particles in the atmosphere orient in the wind (think of the moment of
inertia). We often get impressive halo's around the sun due to scattering
from ice particles.

Rain drops also have a preferential scattering direction because they are
shaped like oblate spheroids (not \textquotedblleft rain drop
shape\textquotedblright\ like we were told in grade school).

It is also true that small molecules will act like tiny antennas and will
scatter light preferentially in some directions and not in others. This is
called \emph{Rayleigh scattering} and is very like small dipole antennas.

\subsection{Optical Activity}

Some substances will rotate the polarization of a beam of light. This is
called being \emph{optically active}. The polarization state of the light
exiting the material depends on the length of the path through the material.
Your calculator display works this way. An electric field changes the
optical activity of the liquid crystal. There are polarizers over the liquid
crystal, so sometimes light passes through the display and sometimes it is
black.

\subsection{Laser polarization}

One last comment. Lasers are usually polarized. This is because the laser
light is generated in a \emph{cavity} created by two mirrors. The mirror is
tipped so light approaches it at the Brewster angle. Light with the right
polarization (parallel to the plane of the drawing) is reflected back nearly
completely, but light with the opposite polarization is not reflected at
all. This reduces the usual loss in reflection from a mirror, because in one
polarization the light must be reflected completely.

\FRAME{dtbpF}{4.2653in}{1.03in}{0in}{}{}{Figure}{\special{language
"Scientific Word";type "GRAPHIC";maintain-aspect-ratio TRUE;display
"USEDEF";valid_file "T";width 4.2653in;height 1.03in;depth
0in;original-width 4.2151in;original-height 0.9963in;cropleft "0";croptop
"1";cropright "1";cropbottom "0";tempfilename
'METNDE00.wmf';tempfile-properties "XPR";}}

\section{Retrospective}

We have thought about many things in this class. It has been a class \emph{%
about} science. It has not been a class where we have tried to discover new
science, or practiced the scientific method. This is on purpose, this being
an engineering class designed to teach the principles of physics for use in
designing machines.

But we should pause to think, just for a moment, about the philosophy of
science. Is everything in these lectures true? We did not perform
experiments to show every principle we learned. So does it all work?

The answer is--maybe. Experiments have been done to show that the equations
we have learned work at least sometimes. But science is an inductive
process. We can't prove anything true with science. We can only prove things
false. So what we have studied is what has not been proven false, yet. Of
course, even then, we have taken approximations from time to time, but we
pointed these out along the way. You will know when the approximations will
fail, because we talked about their valid ranges.

It is important to remember that we are not done discovering new things, and
proving old things false. The laws of Newton are approximations that work at
low speeds. Relativity provides mechanical equations for very high speeds
(e.g. the satellite motion involved in the GPS\ system). But is Relativity
correct? We think it works pretty well, but really we don't know. We may
never know for sure. But we know it works within the range of things we have
tried.

There are physicists today that are working on a fundamentally new model of
the universe. It is called \textquotedblleft String
Theory\textquotedblright\ and it would replace most of our thoughts about
how matter is made and how it interacts. The equations would reduce to the
ones we used in class for the conditions we considered. That is because the
new equations have to match the results of the experiments that we have
already done or they can't be correct. But the explanations might be very
different.

Often, it is in using physics to build something that we learn about the
limitations of physical theory. You may be part of that process. It is a
happy process because extending our understanding allows us to build new
things. But don't be surprised if some of the things we learned in this
class are different by the time your children take their engineering physics
course. That is what we should expect of an inductive process.

It is also important to note that revealed truth is not an inductive
process. It is still not static (see article of faith 9), but it \emph{can}
prove something true as well as prove things false. I\ hope your FDSCI 101
experience gave you some insight into doing science as well as learning
about science.

Some members view science and revelation as in opposition. But I\ think they
are complementary. The scientific process allows us to eliminate things that
are not true, allowing us to follow D\&C 9:8 in preparation for seeking
revelation. During a recent convocation speech, Elder Scott described using
this process as a nuclear engineer during his engineering career . We can
use this combination in our personal lives as well. I hope you will consider
this in your careers and lives.

I have tried to give at least equal time to conceptual understanding and
mathematical solving. I hope you review and refresh the conceptual
understanding of the physics of what you build. Most of my industrial
career, we built what we designed very well. We always did our calculations
well. But we did, at times, build the wrong thing because the conceptual
basis of the design was wrong. Such mistakes are difficult to fix.
Conceptual understanding is a guiding principle for a successful design
career. I hope this class has contributed to that conceptual understanding.

\label{AppendixMarker}

\chapter{Summary of Right Hand Rules}

\section{PH121 or Dynamics Right Hand Rules}

We had two right hand rules on PH121 We didn't give them numbers back then,
so we will do that now.

\subsection{Right hand rule \#0:}

We found that angular velocity had a direction that was given by imagining
you grab the axis of rotation with your right hand so that your fingers seem
to curl the same way the object is rotating. Then your thumb gives the
direction of $\overrightarrow{\mathbf{\omega }}$

\FRAME{dhF}{2.5048in}{1.8165in}{0in}{}{}{Figure}{\special{language
"Scientific Word";type "GRAPHIC";maintain-aspect-ratio TRUE;display
"USEDEF";valid_file "T";width 2.5048in;height 1.8165in;depth
0in;original-width 2.5244in;original-height 1.8219in;cropleft "0";croptop
"1";cropright "1";cropbottom "0";tempfilename
'LW7JUE0D.wmf';tempfile-properties "XPR";}}You curl the fingers of your
right hand (sorry left handed people, you have to use your right hand for
this) in the direction of rotation. Then your thumb points in the direction
of the vector.

\subsection{Right hand rule \#0.5:}

To find the direction of torque, we used the following procedure

\begin{enumerate}
\item Put your fingers of your right hand in the direction of $\mathbf{\vec{r%
}}$\FRAME{dhF}{2.5719in}{1.8628in}{0pt}{}{}{Figure}{\special{language
"Scientific Word";type "GRAPHIC";maintain-aspect-ratio TRUE;display
"USEDEF";valid_file "T";width 2.5719in;height 1.8628in;depth
0pt;original-width 8.022in;original-height 5.8012in;cropleft "0";croptop
"1";cropright "1";cropbottom "0";tempfilename
'LW7ILD01.wmf';tempfile-properties "XPR";}}

\item Curl them toward $\mathbf{\vec{F}}$ \FRAME{dhF}{2.4016in}{1.5281in}{0pt%
}{}{}{Figure}{\special{language "Scientific Word";type
"GRAPHIC";maintain-aspect-ratio TRUE;display "USEDEF";valid_file "T";width
2.4016in;height 1.5281in;depth 0pt;original-width 6.9099in;original-height
4.3855in;cropleft "0";croptop "1";cropright "1";cropbottom "0";tempfilename
'LW7ILD02.wmf';tempfile-properties "XPR";}}

\item The direction of your thumb is the torque direction \FRAME{dhF}{%
1.9597in}{1.8472in}{0pt}{}{}{Figure}{\special{language "Scientific
Word";type "GRAPHIC";maintain-aspect-ratio TRUE;display "USEDEF";valid_file
"T";width 1.9597in;height 1.8472in;depth 0pt;original-width
6.9099in;original-height 6.5086in;cropleft "0";croptop "1";cropright
"1";cropbottom "0";tempfilename 'LW7ILD03.wmf';tempfile-properties "XPR";}}

\item The angle $\theta $ is the angle \emph{between }$\mathbf{\vec{r}}$ and 
$\mathbf{\vec{F}}$ \FRAME{dhF}{1.9285in}{1.8178in}{0pt}{}{}{Figure}{\special%
{language "Scientific Word";type "GRAPHIC";maintain-aspect-ratio
TRUE;display "USEDEF";valid_file "T";width 1.9285in;height 1.8178in;depth
0pt;original-width 6.9099in;original-height 6.5086in;cropleft "0";croptop
"1";cropright "1";cropbottom "0";tempfilename
'LW7ILD04.wmf';tempfile-properties "XPR";}}
\end{enumerate}

The magnitude of the torque is 
\begin{equation*}
\tau =rF\sin \theta
\end{equation*}

\section{PH223 Right Hand Rules}

We have four more right hand rules this semester having to do with charges
and fields.

\subsection{Right hand rule \#1:}

From this rule we get the \textbf{direction of the force on a moving charged
particle} as it travels thorough a \textbf{magnetic field}.

This rule is very like torque. We start with our hand pointing in the
direction of $\mathbf{\vec{v}}.$ Curl your fingers in the direction of $%
\mathbf{\vec{B}.}$ And your thumb will point in the direction of the force.
The magnitude of the force is given by 
\begin{equation}
F=qvB\sin \theta
\end{equation}%
\FRAME{dhF}{1.7633in}{1.2435in}{0in}{}{}{Figure}{\special{language
"Scientific Word";type "GRAPHIC";maintain-aspect-ratio TRUE;display
"USEDEF";valid_file "T";width 1.7633in;height 1.2435in;depth
0in;original-width 1.7678in;original-height 1.2382in;cropleft "0";croptop
"1";cropright "1";cropbottom "0";tempfilename
'LW7J3Y09.wmf';tempfile-properties "XPR";}}

\subsection{Right hand rule \#2:}

From this rule we get the direction of the force on current carrying wire
that is in a magnetic field.

This rule is very like right hand rule \#1 above. We start with our hand
pointing in the direction of $\mathbf{I}.$ Curl your fingers in the
direction of $\mathbf{\vec{B}.}$ And your thumb will point in the direction
of the force. The magnitude of the force is given by 
\begin{equation}
F=ILB\sin \theta
\end{equation}

\subsection{Right hand rule \#3:}

From this rule we get the \textbf{direction of the magnetic field that
surrounds a long current carrying wire}.

This rule is quite different. It is reminiscent of the rule for angular
velocity, but there are some major differences as well. The field is a
magnitude and a direction at every point in space. We can envision drawing
surfaces of constant field strength. They will form concentric circles
(really cylinders) centered on the wire. At any one point on the circle the
field direction will be along a tangent to the circle. The direction of the
vector is given by imaging you grab the wire with your right hand (don't
really do it). Grab such that your right thumb is in the direction of the
current. Your fingers will naturally curl in the direction of the field.%
\FRAME{dhF}{1.8618in}{1.844in}{0in}{}{}{Figure}{\special{language
"Scientific Word";type "GRAPHIC";maintain-aspect-ratio TRUE;display
"USEDEF";valid_file "T";width 1.8618in;height 1.844in;depth
0in;original-width 1.868in;original-height 1.8511in;cropleft "0";croptop
"1";cropright "1";cropbottom "0";tempfilename
'LW7J8C0A.wmf';tempfile-properties "XPR";}}

\subsection{Right Hand Rule \#4:}

From this rule we get the \textbf{direction of the induced current when a
loop is in a changing magnetic field}.

This rule is only used when we have a loop with a changing external magnetic
field. The rule gives the direction of the induced current. The induced
magnetic field will oppose the change in the external field, trying to
prevent a change in the flux. The current direction is found by imagining we
stick our right hand into the loop in the direction of the induced field.
Keeping our hand inside the loop we grab a side of the loop. The current
goes in the direction indicated by our thumb.\FRAME{dhF}{2.3931in}{2.2068in}{%
0pt}{}{}{Figure}{\special{language "Scientific Word";type
"GRAPHIC";maintain-aspect-ratio TRUE;display "USEDEF";valid_file "T";width
2.3931in;height 2.2068in;depth 0pt;original-width 2.4099in;original-height
2.2201in;cropleft "0";croptop "1";cropright "1";cropbottom "0";tempfilename
'LW7IPX08.wmf';tempfile-properties "XPR";}}In the figure above, the external
field is upward but decreasing. So the induced field is upward. The current
flows because there is an induced $emf$ given by%
\begin{eqnarray*}
\mathcal{E} &=&-N\frac{\Delta \Phi }{\Delta t} \\
&=&-N\frac{\left( B_{2}A_{2}\cos \theta _{2}-B_{1}A_{1}\cos \theta
_{1}\right) }{\Delta t}
\end{eqnarray*}

\chapter{Some Helpful Integrals}

\begin{equation*}
\int \frac{rdr}{\sqrt{r^{2}+x^{2}}}=\allowbreak \sqrt{r^{2}+x^{2}}
\end{equation*}

\begin{equation*}
\int \frac{dx}{\left( x^{2}\pm a^{2}\right) ^{\frac{3}{2}}}=\frac{\pm x}{%
a^{2}\sqrt{x^{2}\pm a^{2}}}
\end{equation*}%
\begin{equation*}
\int \frac{xdx}{\left( x^{2}\pm a^{2}\right) ^{\frac{3}{2}}}=\frac{-1}{\sqrt{%
x^{2}\pm a^{2}}}
\end{equation*}

\begin{equation*}
\int \frac{dx}{x}=\allowbreak \ln x
\end{equation*}

\begin{equation*}
\int \frac{dx}{x^{2}}=\allowbreak -\frac{1}{x}
\end{equation*}%
\begin{equation*}
\int_{0}^{2\pi }\int_{0}^{\pi }\sin \theta d\theta d\phi =\allowbreak 4\pi
\end{equation*}%
\begin{equation*}
\int_{0}^{2\pi }\int_{0}^{\pi }\int_{0}^{R}r^{2}dr\sin \theta d\theta d\phi
=\allowbreak \frac{4}{3}\pi R^{3}
\end{equation*}%
\begin{equation*}
\int_{0}^{2\pi }\int_{0}^{R}rdrd\phi =\pi R^{2}
\end{equation*}

\chapter{Table of Physical Constants}

Charge and mass of elementary particles

\begin{tabular}{|l|l|}
\hline
Proton Mass & $m_{p}=1.6726231\times 10^{-27}\unit{kg}$ \\ \hline
Neutron Mass & $m_{n}=1.6749286\times 10^{-27}\unit{kg}$ \\ \hline
Electron Mass & $m_{e}=9.1093897\times 10^{-31}\unit{kg}$ \\ \hline
Electron Charge & $q_{e}=-1.60217733\times 10^{-19}\unit{C}$ \\ \hline
Proton Charge & $q_{p}=1.60217733\times 10^{-19}\unit{C}$ \\ \hline
\end{tabular}

\begin{tabular}{|l|l|}
\hline
$\alpha $-particle mass\QQfnmark{%
http://physics.nist.gov/cgi-bin/cuu/Value?mal} & $m_{\alpha
}=6.64465675(29)\times 10^{-27}\unit{kg}$ \\ \hline
$\alpha $-particle charge & $q_{\alpha }=2q_{e}$ \\ \hline
\end{tabular}%
\QQfntext{0}{
http://physics.nist.gov/cgi-bin/cuu/Value?mal}\bigskip

Fundamental constants

\begin{tabular}{|l|l|}
\hline
Permittivity of free space & $\epsilon _{o}=8.854187817\times 10^{-12}\frac{%
\unit{C}^{2}}{\unit{N}\unit{m}^{2}}$ \\ \hline
Permeability of free space & $\mu _{o}=4\pi \times 10^{-7}\frac{\unit{T}%
\unit{m}}{\unit{A}}$ \\ \hline
Coulomb Constant & $K=\frac{1}{4\pi \epsilon _{o}}=8.98755\times 10^{9}\unit{%
N}\unit{m}^{2}\unit{C}^{-2}$ \\ \hline
Gravitational Constant & $G=6.67259\times 10^{-11}\unit{m}^{3}\unit{kg}^{-1}%
\unit{s}^{-2}$ \\ \hline
Speed of light & $c=2.99792458\times 10^{8}\unit{m}\unit{s}^{-1}$ \\ \hline
Avogadro's Number & $6.0221367\times 10^{23}\unit{mol}^{-1}$ \\ \hline
Fundamental unit of charge & $q_{f}=1.60217733\times 10^{-19}\unit{C}$ \\ 
\hline
\end{tabular}%
\bigskip

Astronomical numbers

\begin{tabular}{|l|l|}
\hline
Mass of the Earth\QQfnmark{%
http://nssdc.gsfc.nasa.gov/planetary/factsheet/earthfact.html} & $%
5.9726\times 10^{24}\unit{kg}$ \\ \hline
Mass of the Moon\QQfnmark{%
http://nssdc.gsfc.nasa.gov/planetary/factsheet/moonfact.html} & $%
0.07342\times 10^{24}\unit{kg}$ \\ \hline
Earth-Moon distance (mean)\QQfnmark{%
http://solarsystem.nasa.gov/planets/profile.cfm?Display=Facts\&Object=Moon}
& $384400\unit{km}$ \\ \hline
Mass of the Sun\QQfnmark{%
http://nssdc.gsfc.nasa.gov/planetary/factsheet/sunfact.html} & $%
1,988,500\times 10^{24}\unit{kg}$ \\ \hline
Earth-Sun distance\QQfnmark{%
http://nssdc.gsfc.nasa.gov/planetary/factsheet/index.html} & $149.6\times
10^{6}\unit{kg}$ \\ \hline
\end{tabular}%
\QQfntext{-4}{
http://nssdc.gsfc.nasa.gov/planetary/factsheet/earthfact.html}
\QQfntext{1}{
http://nssdc.gsfc.nasa.gov/planetary/factsheet/moonfact.html}
\QQfntext{1}{
http://solarsystem.nasa.gov/planets/profile.cfm?Display=Facts\&Object=Moon}
\QQfntext{1}{
http://nssdc.gsfc.nasa.gov/planetary/factsheet/sunfact.html}
\QQfntext{1}{
http://nssdc.gsfc.nasa.gov/planetary/factsheet/index.html}\bigskip

Conductivity and resistivity of various metals

$%
\begin{tabular}{|c|c|c|c|}
\hline
{\small Material} & 
\begin{tabular}{c}
{\small Conductivity} \\ 
$\left( \unit{%
%TCIMACRO{\U{3a9}}%
%BeginExpansion
\Omega%
%EndExpansion
}^{-1}\unit{m}^{-1}\right) $%
\end{tabular}
& 
\begin{tabular}{c}
{\small Resistivity} \\ 
$\left( \unit{%
%TCIMACRO{\U{3a9}}%
%BeginExpansion
\Omega%
%EndExpansion
}\unit{m}\right) $%
\end{tabular}
& 
\begin{tabular}{c}
{\small Temp. Coeff.} \\ 
$\left( \unit{K}^{-1}\right) $%
\end{tabular}
\\ \hline
{\small Aluminum} & ${\small 3.5\times 10}^{7}$ & ${\small 2.8\times 10}%
^{-8} $ & $3.9\times 10^{-3}$ \\ \hline
{\small Copper} & ${\small 6.0\times 10}^{7}$ & ${\small 1.7\times 10}^{-8}$
& $3.9\times 10^{-3}$ \\ \hline
{\small Gold} & ${\small 4.1\times 10}^{7}$ & ${\small 2.4\times 10}^{-8}$ & 
$3.4\times 10^{-3}$ \\ \hline
{\small Iron} & ${\small 1.0\times 10}^{7}$ & ${\small 9.7\times 10}^{-8}$ & 
$5.0\times 10^{-3}$ \\ \hline
{\small Silver} & ${\small 6.2\times 10}^{7}$ & ${\small 1.6\times 10}^{-8}$
& $3.8\times 10^{-3}$ \\ \hline
{\small Tungsten} & ${\small 1.8\times 10}^{7}$ & ${\small 5.6\times 10}%
^{-8} $ & $4.5\times 10^{-3}$ \\ \hline
{\small Nichrome} & ${\small 6.7\times 10}^{5}$ & ${\small 1.5\times 10}%
^{-6} $ & $0.4\times 10^{-3}$ \\ \hline
{\small Carbon} & ${\small 2.9\times 10}^{4}$ & ${\small 3.5\times 10}^{-5}$
& $-0.5\times 10^{-3}$ \\ \hline
\end{tabular}%
$

\end{document}
